
% !TeX spellcheck = pt_BR
% !TEX encoding = UTF-8 Unicode

\chapter{Estudos de funções}\label{Cap:Estudos}

\ifdefined\updateans
% Only need to run once in a lifetime, when the file ans.tex needs to be updated.
\Writetofile{ans}{\protect\section*{Capítulo \ref{Cap:Estudos}}}
\fi

Neste capítulo juntaremos as técnicas desenvolvidas anteriormente
para estudar \emph{funções}.
Já estudamos algumas funções em bastante detalhes no último capítulo, 
ao resolver problemas de otimização.\\

Antes de estudar casos particulares, faremos mais dois comentários
sobre o comportamento
de uma função quando $x\to \pm \infty$.


\section{Sobre o crescimento das funções no $\infty$}
\index{crescimento no $\infty$}

É importante se lembrar, ao estudar funções, de quais são os
comportamentos das funções fundamentais  
(polinômios, exponenciais e logaritmos) que tendem ao infinito
quando $x\to \infty$.\\
%Nesta pequena seção usaremos 
%a regra de Bernoulli-l'Hôpital para estabelecer uma
%\emph{hierarquia} entre essas funções no infinito.\\

Para começar, 
já vimos na Seção~\ref{sec_Lim_parenteseAldo}
(ou no item~\eqref{itexBH12ab} do Exercício~\ref{Exo:BHbasic})
que 
$$\lim_{x\to \infty}\frac{x}{e^{x}}=0\,.$$
Pode também ser mostrado que para qualquer $p>0$,
\eq{\label{eq:expcrescemaisrapidoquepolin}
\lim_{x\to \infty}\frac{x^p}{e^x}=0\,.}
Podemos resumir esse fato da seguinte maneira: seja
$P(x)$ um polinômio cujo coeficiente de grau maior é positivo.
Então $P(x)\nearrow \infty$ e $e^x\nearrow \infty$, mas 
$$\boxed{P(x)\ll e^x\,,\quad\text{ quando
}x\to \infty\,.}$$
O símbolo ``$\ll$'' é usado para significar: ``é muito menor que''.
Em palavras: \emph{no infinito, o crescimento exponencial é
muito mais rápido que qualquer crescimento polinomial}.\\

Vimos também que
$$
\lim_{x\to \infty}\frac{\ln x}{x}=0\,,\quad\quad 
\lim_{x\to \infty}\frac{(\ln x)^2}{x}=0\,,
$$
e pode ser mostrado (veja exercício abaixo) que para qualquer
$p>0$ e qualquer $q>0$,
\eq{\label{eq:polincrescemaisrapidoquelog}\lim_{x\to \infty}\frac{(\ln
x)^p}{x^q}=0\,.}
Como $x^q$ pode também ser trocado
por qualquer polinômio $P(x)$ (supondo que o coeficiente do
seu termo de grau maior é positivo), esse fato costuma ser
resumido da seguinte
maneira:
$$\boxed{(\ln x)^p\ll P(x)\,,\quad\text{ quando }x\to \infty\,.}$$
Isto é: $(\ln x)^p\nearrow \infty$, e 
$P(x)\nearrow \infty$ quando $x\to \infty$,
mas \emph{o crescimento polinomial é muito mais rápido que
qualquer crescimento logaritmico.}

\begin{exo}
Mostre que para qualquer $p>0$, e $q>0$, $\lim_{x\to \infty}\frac{(\ln
x)^p}{x^q}=0$.
\begin{sol}
(Já vimos no Exemplo \ref{Ex:logsurx} que a afirmação vale para $p=1$, $q=1$.)
Observe que 
$\frac{(\ln
x)^p}{x^q}=(\frac{(\ln
x)^{p/q}}{x})^q$. Logo, basta provar a afirmação para $q=1$ e $p>0$ qualquer:
$\lim_{x\to \infty}\frac{(\ln
x)^p}{x}=0$.
Mostremos por indução que se a afirmação vale para $p>0$ 
($\lim_{x\to \infty}\frac{(\ln
x)^{p}}{x}=0$), então ela vale para $p+1$. De fato, pela regra de B.H., 
$$
\lim_{x\to \infty}\frac{(\ln x)^{p+1}}{x}=\lim_{x\to \infty}\frac{(p+1)(\ln
x)^{p}\tfrac{1}{x}}{1}=
(p+1)\lim_{x\to \infty}\frac{(\ln
x)^{p}}{x}=0\,.
$$
Então, a afirmação estará provada para qualquer $p>0$ se ela for provada para
$0<p\leq 1$. Mas para tais $p$, $(\ln x)^p\leq \ln x$ para todo $x>1$, logo, 
$$
\lim_{x\to \infty}\frac{(\ln x)^p}{x}\leq \lim_{x\to \infty}\frac{\ln x}{x}=0\,,
$$ 
pelo Exemplo \ref{Ex:logsurx}.
\end{sol}
\end{exo}

Assim, \emph{quando $x\to \infty$}, a hierarquia entre logaritmo, polinômio e
exponencial é
\eq{\boxed{(\ln x)^p\ll P(x)\ll e^x\,.}}
\begin{exo}
Mostre que para qualquer $p>0$, $\lim_{x\to \infty}\frac{x^p}{e^{x}}=0$.
\end{exo}

\begin{exo}
Estude os seguintes limites
\begin{multicols}{2}
\begin{enumerate}
\item\label{itexBHlast1} $\lim_{x\to \infty}\frac{x^{1000}+e^{-x}}{x^{100}+e^x}$
\item\label{itexBHlast2} $\lim_{x\to \infty}\frac{e^{(\ln x)^2}}{2^{x}}$
\item\label{itexBHlast3} $\lim_{x\to \infty}(x^3-(\ln x)^5-\frac{e^x}{x^7})$
\item\label{itexBHlast4} $\lim_{x\to \infty}x^{\ln x}e^{-x/2}$
\item\label{itexBHlast5_a} $\lim_{x\to \infty}\frac{x}{e^{(\ln x)^2}}$
\item\label{itexBHlast5_b} $\lim_{x\to \infty}\frac{\sqrt{x}}{e^{\sqrt{\ln x}}}$
\item\label{itexBHlast6} $\lim_{x\to
\infty}\frac{\ln(\ln(\ln(x)))}{\ln(\ln(x))}$
\item\label{itexBHlast24}
$\lim_{x\to\infty}\{e^{\sqrt{(\ln x)^2+1}}-x\}$
\end{enumerate}
\end{multicols}
\vspace{0.01cm}
\begin{sol}
\eqref{itexBHlast1} $0$
\eqref{itexBHlast2} $0$
\eqref{itexBHlast3} $-\infty$
\eqref{itexBHlast4} $0$
\eqref{itexBHlast5_a} $0$ 
\eqref{itexBHlast5_b} $\infty$ 
\eqref{itexBHlast6} $0$
\eqref{itexBHlast24} $\infty$
\end{sol}
\end{exo}


\section{Assíntotas oblíquas}\label{Sec:Obliquas}
\index{assíntota! oblíqua}
A noção de \emph{assíntota} permitiu obter informações a respeito do comportamento
qualitativo de uma função \emph{longe da origem}, em direções paralelas aos eixos de
coordenadas: ou horizontal, ou vertical.\\

Veremos nesta seção que existem funções cujo gráfico, longe da origem, 
se aproxima de uma reta que não é nem vertical, nem horizontal, mas \emph{oblíqua},
isto é de inclinação finita  e não nula. Comecemos com um exemplo.

\begin{ex}\label{ex:assolbbb}
Considere a função $f(x)=\frac{x^3+1}{2 x^2}$. 
É claro que esta função possui a reta $x=0$ como assíntota vertical, já que 
$$
\lim_{x\to 0^+}\frac{x^3+1}{2x^2}=+\infty\,,\quad \lim_{x\to
0^-}\frac{x^3+1}{2x^2}=+\infty\,.
$$
Por outro lado, $f$ não possui assíntotas horizontais, já que 
$$
\lim_{x\to +\infty}\frac{x^3+1}{2x^2}=+\infty\,,\quad \lim_{x\to
-\infty}\frac{x^3+1}{2x^2}=-\infty\,.
$$
\begin{center}
\begin{bmlimage}\begin{tikzpicture}[scale=0.8]
\newcommand{\funcao}[1]{ (((#1)^3+1)/(2*(#1)^2)) }
\pgfmathsetmacro{\a}{4};
\pgfmathsetmacro{\b}{3};
\draw[ ->] ({-\a+1},0)--({\a-1},0) node[right]{$x$};
\draw[ ->] (0,{-\b+1})--(0,\b);
\draw[thick, domain=-\a:-0.4, samples=50] plot (\x,{\funcao{\x}});
\draw[thick, domain=0.4:\a, samples=50] plot (\x,{\funcao{\x}});
\end{tikzpicture}\end{bmlimage}
\end{center}
Apesar de não possuir assíntota horizontal, vemos que longe da origem, 
o gráfico parece se aproximar de uma reta de inclinação
positiva. Como determinar essa reta?\\

Para começar, demos uma idéia do que está acontecendo. 
Observe primeiro que
$\frac{x^3+1}{2x^2}=
\frac{x}{2}+\frac{1}{2x^2}$. Logo, quando $x$ for grande, a
contribuição 
do termo $\frac{1}{2x^2}$ é desprezível em relação a
$\frac{x}{2}$, e $f(x)$ é aproximada por
$$
f(x)\simeq \frac{x}{2}\,.$$
Ora, a função $x\mapsto \frac{x}{2}$ é uma reta de inclinação $\frac{1}{2}$. De fato,
esboçando o gráfico de $f$ junto com a reta $y=\frac{x}{2}$:

\begin{center}
\begin{bmlimage}\begin{tikzpicture}[scale=0.8]
\newcommand{\funcao}[1]{ (((#1)^3+1)/(2*(#1)^2)) }
\pgfmathsetmacro{\a}{4};
\pgfmathsetmacro{\b}{3};
\draw[ ->] ({-\a+1},0)--({\a-1},0) node[right]{$x$};
\draw[ ->] (0,{-\b+1})--(0,\b);
\draw[thick, domain=-\a:-0.4, samples=50] plot (\x,{\funcao{\x}});
\draw[thick, domain=0.4:\a, samples=50] plot (\x,{\funcao{\x}});
\draw[dashed, domain=-\a:\a] plot (\x,{\x/2}) node[below right]{$y=\frac{x}{2}$};
\end{tikzpicture}\end{bmlimage}
\end{center}
Podemos agora verificar que de fato, 
quando $x\to \infty$, \emph{a distância entre o gráfico
de $f$ e a reta $y=\frac{x}{2}$ tende a zero}:
\begin{equation}\label{eq:biduleolbik}
\lim_{x\to \infty}\bigl|f(x)-\tfrac{x}{2}\bigr|=
\lim_{x\to \infty}\bigl|(\tfrac{x}{2}+\tfrac{1}{2x^2})-\tfrac{x}{2}\bigr|=
\lim_{x\to \infty}\tfrac{1}{2x^2}=0\,.
\end{equation}
Portanto, a reta $y=\frac{x}{2}$ é chamada de \emph{assíntota oblíqua} da função $f$.
\end{ex}

O exemplo anterior leva naturalmente à seguinte definição:

\begin{defin}
A reta de equação $y=mx+h$ é chamada de \grasA{assíntota oblíqua para $f$} 
se pelo menos um
dos limites abaixo existe e é nulo:
$$\lim_{x\to +\infty}\bigl|f(x)-(mx+h)\bigr|\,,\quad\lim_{x\to
-\infty}\bigl|f(x)-(mx+h)\bigr|\,.$$
(Obs: quando $m=0$, essa definição coincide com a de assíntota
horizontal.)
\end{defin}

Como saber se uma função possui uma assíntota oblíqua? E se ela tiver uma, como
identificar os coeficientes $m$ e $h$?\\

Para começar, observe que $h$ pode ser obtido a partir de $m$, já que 
$$
\lim_{x\to \pm\infty}\bigl \{f(x)-(mx+h)\bigr\}=
\lim_{x\to \pm\infty}\bigl \{(f(x)-mx)-h\bigr\}
$$
é zero se e somente se 
\begin{equation}\label{eq:hparaoblik}
h=\lim_{x\to \pm\infty}\{f(x)-mx\}\,.
\end{equation}

Para identificar $m$, podemos escrever
$$\lim_{x\to \pm\infty}\bigl \{f(x)-(mx+h)\bigr\}
=\lim_{x\to \pm\infty}x\cdot \bigl\{\tfrac{f(x)}{x}-(m+\tfrac{h}{x})\bigr\}\,,
$$
e observar que para este último limite 
existir e ser igual a zero quando $x\to \pm \infty$, é necessário 
que 
$\lim_{x\to \pm \infty} \bigl\{\tfrac{f(x)}{x}-(m+\tfrac{h}{x})\bigr\}=0$. 
Como $\frac{h}{x}\to 0$, isso implica que
\begin{equation}\label{eq:mparaoblik}
m=\lim_{x\to \infty}\frac{f(x)}{x}\,.
\end{equation}

Assim, vemos que se $f$ possui uma 
assíntota oblíqua, esta tem uma inclinação
dada por \eqref{eq:mparaoblik}, e uma abcissa na origem dada por
\eqref{eq:hparaoblik}.
Por outro lado, é claro que se os dois limites em
\eqref{eq:mparaoblik} e 
\eqref{eq:hparaoblik} existirem e forem \emph{ambos finitos}, então $f$
possui uma assíntota oblíqua dada por $y=mx+h$. 
É claro que os limites $x\to +\infty$ precisam ser calculados
\emph{separadamente}, pois uma função pode possuir assíntotas
oblíquas diferentes em $+\infty$ e $-\infty$.\\

Voltando para o Exemplo \ref{ex:assolbbb}, temos
$$
m=\lim_{x\to \pm \infty} \frac{f(x)}{x}=\lim_{x\to\pm\infty}
\frac{\frac{x^3+1}{2x^2}}{x}=
\lim_{x\to\pm\infty}\frac{x^3+1}{2x^3}
=\lim_{x\to\pm\infty}\bigl\{
\tfrac12+\tfrac{1}{2x^3}
\bigr\}=\tfrac12\,,
$$
e, como já visto anteriormente,
$$h=\lim_{x\to\pm \infty}
\{f(x)-\tfrac{1}{2}x\}=\lim_{x\to\pm \infty}
\tfrac{1}{2x^3}=0\,.
$$
Logo, $y=\tfrac12 x+0$ é assíntota oblíqua.
Vejamos como usar   o critério acima em outros exemplos.


\begin{ex}
Considere $f(x)=\sqrt{x^2+2x}$.
Primeiro, tentaremos procurar uma inclinação. Pela presença
da raiz quadrada, cuidamos de distinguir os
limites $x\to-\infty$ e $x\to-\infty$:
$$
\lim_{x\to +\infty}\frac{f(x)}{x}=\lim_{x\to+\infty}
\frac{\sqrt{x^2+2x}}{x}=\lim_{x\to +\infty}
\frac{x\sqrt{1+\tfrac{2}{x}}}{x}=+1
$$
Em seguida calculemos 
\begin{align*}
\lim_{x\to\infty}\{f(x)-(+1)x\}=\lim_{x\to+\infty}\{\sqrt{x^2+2x}-x\}
&=\lim_{x\to+\infty}\frac{2x}{\sqrt{x^2+2x}+x}\\
&=\lim_{x\to+\infty}\frac{2}{\sqrt{1+\frac{2}{x}}+1}=1\,.
\end{align*}
Assim, $f$ possui a assíntota oblíqua $y=x+1$ em
$+\infty$. Refazendo contas parecidas para $x\to-\infty$,
obtemos
$$
\lim_{x\to-\infty}\frac{f(x)}{x}=-1\,,\text{ e }\quad
\lim_{x\to-\infty}\{f(x)-(-1)x\}=-1\,,
$$
logo $f$ possui a assíntota oblíqua $y=-x-1$ em
$-\infty$. De fato (observe que $f$ tem domínio
$D=(-\infty,-2]\cup[0,+\infty)$),
\begin{center}
\begin{bmlimage}\begin{tikzpicture}[scale=1]
\newcommand{\funcao}[1]{ (sqrt((#1)^2+2*(#1))) }
\pgfmathsetmacro{\a}{3};
\pgfmathsetmacro{\b}{4};
\draw[ ->] ({-\a-1},0)--({\a+1},0) node[right]{$x$};
\draw[ ->] (0,{-0.2})--(0,\b) node[above]{$\sqrt{x^2+2x}$};
\draw[thick, domain={-\a-2}:-2, samples=30] plot (\x,{\funcao{\x}});
\draw[thick, domain=0:\a,  samples=30] plot (\x,{\funcao{\x}});
\draw[dashed] (-1,0)--({-\a-2},{\a+1})
node[right]{{$y=-x-1$}};
\draw[dashed] (-1,0)--({\a},{\a+1})
node[left]{{$y=x+1$}};
\end{tikzpicture}\end{bmlimage}
\end{center}
\end{ex}

\begin{ex}
Considere $f(x)=x+\sqrt{x}$, definida somente se $x>0$.
Então 
$$\lim_{x\to\infty}\frac{f(x)}{x}=
\lim_{x\to\infty}\bigl\{1+\frac{\sqrt{x}}{x}\bigr\}=1\,.
$$
Mas, como
$$
\lim_{x\to\infty}\{f(x)-x\}=\lim_{x\to\infty}\sqrt{x}=\infty\,,
$$
vemos que $f$ \emph{não} possui assíntota oblíqua (apesar de
$\lim_{x\to\infty}\frac{f(x)}{x}$ existir e ser finita).
\end{ex}

\begin{exo}
Determine quais das funções abaixo possuem assíntotas
 (se tiver, calcule-as).
\begin{multicols}{3}
\begin{enumerate}
\item\label{itasobl1} $4x-5$
\item\label{itasobl11} $x^2$
\item\label{itasobl6} $\frac{x^2-1}{x+2}$
\item\label{itasobl12} $\ln( x^6+1)$
\item\label{itasobl2} $\ln(1+e^x)$
\item\label{itasobl3} $\sqrt{x^2-\ln x}$
\item\label{itasobl4} $\ln(\cosh x)$
\item\label{itasobl5} $e^{\sqrt{(\ln x)^2+1}}$
\end{enumerate}
\end{multicols}
\vspace{0.01cm}
\begin{sol}
\eqref{itasobl1} A função é a sua própria assíntota oblíqua.
\eqref{itasobl11} Não possui ass.
\eqref{itasobl6} $y=-2$ (vertical), $y=x-2$ em $\pm\infty$. 
\eqref{itasobl12} Não possui ass.
\eqref{itasobl2} $y=0$ em $-\infty$, $y=x$ em $+\infty$.
\eqref{itasobl3} $y=x$ em $+\infty$.
\eqref{itasobl4} $y=x-\ln 2$ em $+\infty$, $y=-x-\ln 2$ em
$-\infty$.
\eqref{itasobl5} Não possui assíntotas: apesar de
$m=\lim_{x\to \infty}\frac{e^{\sqrt{\ln^2x+1}}}{x}$
existir e valer $1$,
$\lim_{x\to\infty}\{e^{\sqrt{\ln^2x+1}}-x\}=\infty$. 
\end{sol}
\end{exo}

\begin{exo}
Se uma função possui uma assíntota oblíqua $y=mx+h$ em
$+\infty$, é
verdade que $\lim_{x\to\infty}f'(x)=m$?
\begin{sol}
Em geral, náo. 
Por exemplo, $f(x)=x+\tfrac{1}{x}\sen (x^2)$ possui $y=x$ como assíntota
oblíqua em $+\infty$, 
mas $f'(x)=1-\frac{\sen x^2}{x^2}+2\cos (x^2)$ 
não possui limite quando $x\to\infty$.
Na verdade, uma função pode possuir uma assíntota (oblíqua ou
outra)
sem sequer ser derivável.
\end{sol}
\end{exo}

\section{Estudos de funções}\label{Sec:Estudos}
\index{estudos de funções}
Podemos agora juntar as técnicas conhecidas para estabelecer um roteiro
para o estudo completo de uma função $f$:
\begin{itemize}
 \item Para começar, encontrar o \emph{domínio} de $f$. O domínio precisa ser
especificado para evitar divisões por zero e raizes (ou logaritmos) de números
negativos. A função poderá depois ser estudada na vizinança de alguns dos
pontos que não pertencem ao domínio, caso
sejam associados a assíntotas verticais.
\item Se for possível (e não sempre é), 
estudar os \emph{zeros} e o \emph{sinal} de $f$.
\item Determinar se $f$ possui algumas \emph{simetrias}, via o estudo da
\emph{paridade}: $f$ é par se $f(-x)=f(x)$, ímpar se $f(-x)=-f(x)$. 
\item Estudar o comportamento assíntotico de $f$, isto é, $f(x)$ quando $x\to
\pm \infty$ (se o domínio
o permite). Se um dos limites $\lim_{x\to \pm\infty}f(x)$ existir (esses limites
podem precisar da regra de Bernoulli-l'Hôpital), então a
função possui uma \emph{assíntota horizontal}. 
Lembre que pode ter assíntotas
horizontais diferentes em $+\infty$ e $-\infty$.
Se um dos limites $\lim_{x\to\infty}f(x)$ for infinito, poderá
procurar saber se existem \emph{assíntotas oblíquas}, como
descrito na Seção \ref{Sec:Obliquas}.
\item Procurar pontos na vizinhança dos quais $f(x)$ toma valores
arbitrariamente grandes, isto é:
\emph{assíntotas verticais}. Calculando os
limites laterais $\lim_{x\to a^+}f(x)$ e $\lim_{x\to a^-}f(x)$ nos pontos $a$
perto dos quais $f$ não é limitada. Isto acontece em geral perto de uma divizão
por zero, ou quando a variável de um logaritmo tende a zero.
\item Estudar a primeira derivada de $f$ (se existir). Em particular,
procurar os \emph{pontos críticos de $f$}. Deduzir a
\emph{variação} de $f$ via o estudo do sinal de $f'$. Determinar os pontos de
mínimo e máximo,  locais ou globais.
\item Estudar $f''$ e a convexidade/concavidade de $f$, via o sinal de $f''$.
O sinal de $f''$ nos pontos críticos (se tiver) permite
determinar quais são mínimos/máximos locais.
Os \emph{pontos de inflexão} são aqueles onde $f$ passa de convexa para
côncava, ou o contrário.
\item Juntando essas informações, montar o \emph{gráfico} de $f$.
Por exemplo, se $f$ é par, o gráfico é simétrico com respeito ao eixo $y$.
Para montar um gráfico completo, 
pode ser necessário calcular mais alguns limites, por exemplo para observar o
comportamento da derivada perto de alguns pontos particulares.
\end{itemize}

\begin{ex} Comecemos com $f(x)=\frac{x+1}{1-x}$, cujo domínio é $D=\bR\setminus
\{1\}$. A função se anula no ponto $x=-1$, e o seu sinal é dado por:
\begin{center}
\begin{bmlimage}\begin{tikzpicture}
\tkzTabInit[lgt=3, nocadre, espcl=2, colorC=red, colorV=gray]
{Valores de $x$: /.6,  $x+1$ /.6, $1-x$ /.6, $f(x)$ /.8}%
{,$-1$, $1$,}
\tkzTabLine{,-,z,+, ,+,}
\tkzTabLine{,+, ,+,z,-,}
\tkzTabLine{,-,z,+,d,-,}
\end{tikzpicture}\end{bmlimage}
\end{center}
(A dupla barra em $x=1$ é para indicar que $f$ não é definida em $x=1$.)
A funçao não é nem par, nem ímpar.
Como
$$
\lim_{x\to \pm\infty}\frac{x+1}{1-x}=\lim_{x\to
\pm\infty}\frac{1+\frac{1}{x}}{\frac{1}{x}-1}=
\frac{1}{-1}=-1\,,
$$
$f$ possui a reta $y=-1$ como assíntota horizontal.
Por outro lado, como
$$
\lim_{x\to 1^+}\frac{x+1}{1-x}=-\infty\,,\quad 
\lim_{x\to 1^-}\frac{x+1}{1-x}=+\infty\,,\quad 
$$
$f$ possui a reta $x=1$ como assíntota vertical.
A derivada existe em todo $x\neq 1$, e vale
$$
f'(x)=\frac{(x+1)'(1-x)-(x+1)(1-x)'}{(1-x)^2}=
\frac{1-x+(x+1)}{(1-x)^2}=
\frac{2}{(1-x)^2}\,.
$$
O sinal de $f'$ dá logo a tabela de variação de $f$:
\begin{center}
\begin{bmlimage}\begin{tikzpicture}[scale=0.8]
\tkzTabInit[nocadre, espcl=2,  color, colorV=lightgray!5, colorL=gray!15,
colorC=gray!15]
{$x$ /.6, $f'(x)$ /.6, Variaç. de $f$ /1.2}%
{,$1$, }%
\tkzTabLine{,+,d,+,}
\tkzTabVar{-/,+D-/$\scriptscriptstyle{+\infty}$/$\scriptscriptstyle{-\infty}$,+/
}
%\tkzTabLine{,\searrow,\text{mín.},h,\text{mín.},\nearrow,}
\end{tikzpicture}\end{bmlimage}
\end{center}
(Indicamos o fato de $x=1$ ser uma assíntota vertical.)
Assim, $f$ não possui pontos críticos, e é crescente nos intervalos
 $(-\infty,1)$ e $(1,\infty)$.
A segunda derivada se calcula facilmente (para $x\neq 0$):
$$f''(x)=2((1-x)^{-2})'=2(-2)(1-x)^{-3}(-1)=\frac{4}{(1-x)^3}\,.$$
Esta muda de sinal em $x=1$, e permite descrever a convexidade de $f$:
\begin{center}
\begin{bmlimage}\begin{tikzpicture}[scale=0.8]
\tkzTabInit[nocadre, espcl=2,  color, colorV=lightgray!5, colorL=gray!15,
colorC=gray!15]
{$x$ /.6, $f''(x)$ /.6, Conv. de $f$ /1.2}%
{,$1$, }%
\tkzTabLine{,+,d,-,}
\tkzTabLine{,\smile,d,\frown,}
\end{tikzpicture}\end{bmlimage}
\end{center}
Isto é, $f$ é convexa em $(-\infty,1)$, côncava em $(1,\infty)$. Assim, o
gráfico é da forma
\begin{center}
\begin{bmlimage}\begin{tikzpicture}[yscale=0.6]
\draw[ ->] (-4,0)--(5,0);
\draw[ ->] (0,-4)--(0,+2.5);
\draw[dashed] (-4,-1)node[below right]{$y=-1$}--(5,-1);
\draw[dashed] (1,-4)--(1,3)node[right]{$x=1$};
\draw[thick, domain=-4:0.5, samples=50] plot (\x,{(\x+1)/(1-\x)});
\draw[thick, domain=1.5:5, samples=50] plot (\x,{(\x+1)/(1-\x)});
\draw (-1,0) node{$\shortmid$} node[above]{$-1$};
\draw (1,0) node{$\shortmid$} node[above right]{$1$};
\end{tikzpicture}\end{bmlimage}
\end{center}
\end{ex}

\begin{ex}
Estudemos agora a função $f(x)=\frac{x^2-1}{x^2+1}$.
%\eqref{itexoEstudA2}
O seu domínio é $D=\bR$, e o seu sinal: $f(x)$ é $\geq 0$ se $|x|\geq 1$,
$<0$ caso contrário.
Como $f(-x)=\frac{(-x)^2-1}{(-x)^2+1}=\frac{x^2-1}{x^2+1}=f(x)$, $f$ é par.
Como 
$$
\lim_{x\to \pm \infty}\frac{x^2-1}{x^2+1}=\lim_{x\to \pm
\infty}\frac{1-\tfrac{1}{x^2}}{1+\tfrac{1}{x^2}}=1\,,
$$
a reta $y=1$ é assíntota horizontal. Não tem assíntotas verticais (o
denominador não se anula em nenhum ponto).
A primeira derivada é dada por $f'(x)=\frac{4x}{(x^2+1)^2}$. Logo,
\begin{center}
\begin{bmlimage}\begin{tikzpicture}
\tkzTabInit[nocadre,espcl=2,  color, colorV=lightgray!5, colorL=gray!15,
colorC=gray!15]
%{$x$ /.6,  $f'(x)$ /.9, Variação de $f$ /1.5}%
{$x$ /.5,  $f'(x)$ /.5, Var. de $f$ /1}{,$0$,}
%\tkzTabLine{,+,z,+,,+,}
\tkzTabLine{,-,z,+,}
\tkzTabVar{+/,-/\text{min.},+/,}
%\tkzTabLine{,\searrow,\text{mín.},h,\text{mín.},\nearrow,}
\end{tikzpicture}\end{bmlimage}
\end{center}
O mínimo local (que é global também) tem coordenada $(0,f(0))=(0,-1)$. A
segunda derivada é dada por
$f''(x)=\frac{4(1-3x^2)}{(x^2+1)^3}$, logo:
\begin{center}
\begin{bmlimage}\begin{tikzpicture}
\tkzTabInit[nocadre,espcl=2,  color, colorV=lightgray!5, colorL=gray!15,
colorC=gray!15]
%{$x$ /.5,  $f''(x)$ /.7, Conc. de $f$ /1.3}
{$x$ /.5,  $f''(x)$ /.5, Conc. de $f$ /1}
{,$-1/\sqrt{3}$, $-1/\sqrt{3}$,}
\tkzTabLine{,-,z,+,z,-,}
\tkzTabLine{,{\frown},,\smile,,\frown,}
\end{tikzpicture}\end{bmlimage}
\end{center}
Os pontos de inflexão estão em
$(\tfrac{-1}{\sqrt{3}},f(\tfrac{-1}{\sqrt{3}}))=(\tfrac{-1}{\sqrt{3}},-\tfrac{1}
{2})$,
e
$(\tfrac{+1}{\sqrt{3}},f(\tfrac{+1}{\sqrt{3}}))=(\tfrac{+1}{\sqrt{3}},-\tfrac{1
}{2})$. Finalmente,
\begin{center}
\begin{bmlimage}\begin{tikzpicture}[scale=1.3]
\draw [thick, domain=-4:4, samples=100] plot (\x,{((\x)^2-1)/((\x)^2+1)});
\draw [ ->] (-4,0)--(4,0) node[right] {$x$};
\draw [ ->] (0,-0.1)--(0,1.5) node[left] {$f(x)$};
\draw [dotted] (-4,1)--(4,1) node[above left] {$y=1$};
%\fill (1,0) circle (0.35mm);
%\fill (-1,0) circle (0.35mm);
\fill (0,-1) circle (0.35mm);
\draw (0,-1) node[below] {$(0,-1)$};
\fill (-0.577,-0.5) circle (0.35mm);
\draw (-0.577,-0.6) node[left]{$(\tfrac{-1}{\sqrt{3}},-\half)$};
\fill (+0.577,-0.5) circle (0.35mm);
\draw (+0.577,-0.6) node[right]{$(\tfrac{+1}{\sqrt{3}},-\half)$};
\end{tikzpicture}\end{bmlimage}
\end{center}
\end{ex}

\begin{exo}\label{Exo:DoisEstudosLegais} Faça um estudo completo das seguintes
funções. 
% \begin{multicols}{1}
\begin{enumerate}
\item\label{itexoEstudA1} $\bigl(\frac{x-1}{x}\bigr)^2$ (Segunda prova,
primeiro semestre 2011)
%\item \label{itexoEstudA2} $\frac{x^2-1}{x^2+1}$
\item \label{itexoEstudA3} $x(\ln x)^2$ (Segunda prova, primeiro semestre 2010)
\end{enumerate}
% \end{multicols}
\vspace{0.01cm}
\begin{sol}
%%%%%%%%%%%%%%%%%%%%%%%%%%%%%%%%5
\eqref{itexoEstudA1}:
O domínio de $\bigl(\frac{x-1}{x}\bigr)^2$ é $D=\bR\setminus \{0\}$, o sinal é
sempre não-negativo, tem um zero
em $x=1$. $f$ não é par, nem ímpar.
Os limites relevantes são $\lim_{x\to 0^{\pm}}f(x)=+\infty$, logo $x=0$ é
assíntota vertical, e
$$\lim_{x\to \pm\infty}\bigl(\frac{x-1}{x}\bigr)^2=\Bigl(\lim_{x\to \pm
\infty}\frac{x-1}{x}\Bigr)^2==\Bigl(\lim_{x\to \pm
\infty}\bigl(1-\frac{1}{x}\bigr)\Bigr)^2=1^2=1\,.$$
Logo, $y=1$ é assíntota horizontal. 
$f$ é derivável em $D$, e $f'(x)=\frac{2(x-1)}{x^3}$.
\begin{center}
\begin{bmlimage}\begin{tikzpicture}
\tkzTabInit[nocadre,espcl=2,  color, colorV=lightgray!5, colorL=gray!15,
colorC=gray!15]
{$x$ /.6,  $f'(x)$ /.6, Var. de $f$ /1.3}%
{,$0$, $1$,}%
%\tkzTabLine{,+,z,+,,+,}
\tkzTabLine{,+,d,-,z,+,}
\tkzTabVar{-/,+D+/$+\infty$/$+\infty$,-/mín,+/,}
%\tkzTabLine{,\searrow,\text{mín.},h,\text{mín.},\nearrow,}
\end{tikzpicture}\end{bmlimage}
\end{center}
$f$ possui um mínimo global em $(1,0)$.
A segunda derivada é dada por $f''(x)=\frac{2(3-2x)}{x^4}$. Ela se anula em
$x=\tfrac32$, e muda de sinal neste ponto:
\begin{center}
\begin{bmlimage}\begin{tikzpicture}
\tkzTabInit[nocadre,espcl=2,  color, colorV=lightgray!5, colorL=gray!15,
colorC=gray!15]
{$x$ /.6,  $f''(x)$ /.7, Conv. de $f$ /1.2}%
{,$0$, $\tfrac32$,}%
\tkzTabLine{,+,d,+,z,-,}%
\tkzTabLine{,\smile,d,\smile,z,\frown,}%
\end{tikzpicture}\end{bmlimage}
\end{center}
Logo, $f$ é convexa em $(-\infty,0)$ e $(0,\frac32)$,  côncava em
$(\frac32,\infty)$, e possui um ponto de inflexão em 
$(\tfrac{3}{2},f(\tfrac{3}{2}))=(\tfrac{3}{2},\tfrac19)$.
\begin{center}
\begin{bmlimage}\begin{tikzpicture}
\draw [thick, domain=-4:-1.2, samples=100] plot
(\x,{((\x)-1)^2/((\x)^2)});
\draw [thick, domain=0.4:4, samples=100] plot (\x,{(\x-1)^2/((\x)^2)});
\draw [ ->] (-4,0)--(4,0) node[right] {$x$};
\draw [ ->] (0,-0.1)--(0,3) node[left] {$f(x)$};
\draw [dotted] (-4,1)--(4,1) node[above] {$y=1$};
\draw [dotted] (0,0)--(0,3.5) node[right] {$x=0$};
\fill (1,0) circle (0.35mm);
\draw (1,0) node[below] {$(1,0)$};
\fill (1.5,0.1111) circle (0.35mm);
\draw [ <-] (1.52,0.0911)--(2,-0.3) node[right]
{$(\tfrac{3}{2},\tfrac{1}{9})$};
\end{tikzpicture}\end{bmlimage}
\end{center}
\eqref{itexoEstudA3}:
O domínio de  $f(x)=x(\ln x)^2$ é
$D=(0,+\infty)$, e o seu sinal é: $f(x)\geq 0$ para todo $x\in D$.
A função não é { par} nem { ímpar}.
Como $\lim_{x\to \infty}f(x)=+\infty$, não tem assintota horizontal.
Para ver se tem assíntota vertical em $x=0$, calculemos 
$\lim_{x\to 0^+}f(x)=\lim_{x\to 0^+}\frac{(\ln x)^2}{1/x}$. Como ambas funções
$(\ln x)^2$ e $1/x$ são deriváveis em $(0,1)$ e tendem a $+\infty$ quando $x\to
0^+$, apliquemos a regra de B.H.:
$$
\lim_{x\to 0^+}\frac{(\ln x)^2}{1/x}=
\lim_{x\to 0^+}\frac{2(\ln x)1/x}{-1/x^2}=
-2\lim_{x\to 0^+}x\ln x\,.
$$
Usando a regra de B.H. de novo, pode ser mostrado que esse segundo limite é
zero (ver Exemplo \ref{Ex:xlogxemzero}). Logo, $\lim_{x\to 0^+}f(x)=0$: não
tem assíntota vertical em $x=0$.
A derivada é dada por $f'(x)=\ln x(\ln x+2)$.
\begin{center}
\begin{bmlimage}\begin{tikzpicture}[scale=0.8]
\tkzTabInit[nocadre, espcl=2,  color, colorV=lightgray!5, colorL=gray!15,
colorC=gray!15]
{$x$ /.6, $f'(x)$ /.6, Variaç. de $f$ /1.2}%
{,$e^{-2}$, $1$, }%
\tkzTabLine{,+,z,-,z,+}
\tkzTabVar{-/,+/{máx.},-/{mín.},+/}
%\tkzTabLine{,\searrow,\text{mín.},h,\text{mín.},\nearrow,}
\end{tikzpicture}\end{bmlimage}
\end{center}
O máximo local está em
$(e^{-2},f(e^{-2}))=(e^{-2},4e^{- 2})$, e o
mínimo global em $(1,f(1))=(1,0)$.
A {segunda derivada} de $f$ é dada por
$f''(x)=\frac{2(\ln x+1)}{x}$.
\begin{center}
\begin{bmlimage}\begin{tikzpicture}[scale=0.8]
\tkzTabInit[nocadre, espcl=2,  color, colorV=lightgray!5, colorL=gray!15,
colorC=gray!15]
{$x$ /.6, $f''(x)$ /.6, Conv. de $f$ /1.2}%
{,$e^{-1}$, }%
\tkzTabLine{,-,z,+,}
\tkzTabLine{,\frown,,\smile,}
\end{tikzpicture}\end{bmlimage}
\end{center}
Logo, $f$ é côncava em $(0,e^{-1})$, possui um ponto de inflexão em
$(e^{-1},f(e^{-1}))=(e^{-1},e^{-1})$, e é convexa em $(e^{-1},+\infty)$.
\begin{center}
\begin{bmlimage}\begin{tikzpicture}[scale=1.3]
\draw [thick, domain=0.001:2.5, samples=100] plot (\x,{(\x)*(ln(\x))^2});
 \draw [ ->] (0,0)--(2.5,0) node[right] {$x$};
 \draw [ ->] (0,-0.1)--(0,2);
% \draw [dotted] (-4,1)--(4,1) node[above left] {Assíntota horiz.: $y=1$};
 \fill (1,0) circle (0.35mm);
 \draw (1,0) node[below] {$\scriptscriptstyle{(1,0)}$};
 \fill (0.367,0.367) circle (0.35mm);
 \draw[<-] (0.39,0.39)--(0.9,0.5) node[above]
{$\scriptscriptstyle{(e^{-1},e^{-1})}$};
 \fill (0.1353,0.541) circle (0.35mm);
 \draw[<-] (0.14,0.58)--(0.9,1.5) node[above]
{$\scriptscriptstyle{(e^{-2},4e^{-2})}$};
\end{tikzpicture}\end{bmlimage}
\end{center}
Podemos também notar que $\lim_{x\to 0^+}f'(x)=+\infty$.
\end{sol}
\end{exo}

\begin{exo}(Segunda prova, segundo semestre de 2011)
Para $f(x)\pardef\frac{x^2-4}{x^2-16}$, estude:
o sinal, os zeros, as assíntotas
(se tiver), a variação, e a posição dos pontos de mín./máx. (se tiver). A
partir dessas informações, monte o gráfico de $f$. Em seguida,
complete a sua análise com a determinação dos intervalos em que $f$ é
convexa/côncava.
\begin{sol}
$D=\bR\backslash \{\pm 4\}$. Os zeros de $f(x)\pardef\frac{x^2-4}{x^2-16}$ são
$x=-2$, $x=+2$, e o seu sinal:
\begin{center}
\begin{bmlimage}\begin{tikzpicture}
\tkzTabInit[lgt=3, nocadre, espcl=2]
{ /.6,  $x^2-4$ /.6, $x^2-16$ /.6, $f(x)$ /.8}%
{,$-4$, $-2$, $2$, $4$,}%
%\tkzTabLine{,+,z,+,,+,}
\tkzTabLine{,+,,+,z,-,z,+,,+,}
\tkzTabLine{,+,z,-,,-,,-,z,+,}
\tkzTabLine{,+,d,-,z,+,z,-,d,+,}
%\tkzTabLine{,+,z,-,z,+,}
%\tkzTabVar{-/,+/\text{a.v.},-/$0$,+/,}
%\tkzTabLine{,\searrow,\text{mín.},h,\text{mín.},\nearrow,}
\end{tikzpicture}\end{bmlimage}
\end{center}
Como 
$$
\lim_{x\to \pm\infty}f(x)=\lim_{x\to
\pm \infty}\frac{1-\frac{4}{x^2}}{1-\frac{16}{x^2}}
=1\,,$$
a reta $y=1$ é assíntota horizontal.
Como 
$$
\lim_{x\to -4^\pm}f(x)=\mp \infty\,,\quad \lim_{x\to +4^\pm}f(x)=\pm \infty\,,$$
as retas $x=-4$ e $x=+4$ são assíntotas verticais. 
A primeira derivada se calcula facilmente: $f'(x)=\frac{-24 x }{(x^2-16)^2}$,
logo a variação de $f$ é dada por:
\begin{center}
\begin{bmlimage}\begin{tikzpicture}[scale=0.8]
\tkzTabInit[nocadre, espcl=2,  color, colorV=lightgray!5, colorL=gray!15,
colorC=gray!15]
{$x$ /.6, $f'(x)$ /.6, Variaç. de $f$ /1.2}%
{,$-4$,$0$,$4$, }%
\tkzTabLine{,+,d,+,z,-,d,-}
\tkzTabVar{-/,+D-/{},+/{máx.},-D+/{},-/}
%\tkzTabLine{,\searrow,\text{mín.},h,\text{mín.},\nearrow,}
\end{tikzpicture}\end{bmlimage}
\end{center}
A posição do máximo local é: $(0,f(0))=(0,\tfrac14)$.
O gráfico:
\begin{center}
\begin{bmlimage}\begin{tikzpicture}[scale=0.7]
\pgfmathsetmacro{\a}{10}
\pgfmathsetmacro{\b}{4}
\newcommand{\funcao}[1]{ ( (#1)^2-4 )/( (#1)^2-16) }
\draw[->] (-\a,0)--(\a,0);
\draw[->] (0,-\b)--(0,\b);
\draw[thick, domain=-\a:-4.5, samples=50] plot (\x,{\funcao{\x}});
\draw[thick, domain=-3.5:3.5, samples=50] plot (\x,{\funcao{\x}});
\draw[thick, domain=4.5:\a, samples=50] plot (\x,{\funcao{\x}});
\draw[dashed] (-\a,1)node[below]{$y=1$}--(\a,1);
\draw[dashed] (-4,-\b)node[left]{$x=-4$}--(-4,\b);
\draw[dashed] (4,-\b)node[right]{$x=+4$}--(4,\b);
\draw[<-] (0.1,0.3)--(2,2)node[above]{máx.: $(0,\frac14)$};
\draw (-2,0) node{$\shortmid$} node[above]{$-2$};]
\draw (2,0) node{$\shortmid$} node[above]{$+2$};]
\end{tikzpicture}\end{bmlimage}
\end{center}
A segunda derivada: $f''(x)=24\frac{16+3x^2}{(x^2-16)^3}$, e a convexidade é
dada por
\begin{center}
\begin{bmlimage}\begin{tikzpicture}[scale=0.8]
\tkzTabInit[nocadre, espcl=2,  color, colorV=lightgray!5, colorL=gray!15,
colorC=gray!15]
{$x$ /.6, $f''(x)$ /.6, Conv. $f$ /1.2}%
{,$-4$,$4$, }%
\tkzTabLine{,+,d,-,d,+}
\tkzTabLine{,\smile,d, \frown,d,\smile,}
%\tkzTabLine{,\searrow,\text{mín.},h,\text{mín.},\nearrow,}
\end{tikzpicture}\end{bmlimage}
\end{center}
\end{sol}
\end{exo}


\begin{exo}\label{Exo:EstudosBasicos}
Faça um estudo completo das funções abaixo:
\begin{multicols}{3}
\begin{enumerate}
%\item\label{itEstBas9a} $\frac{x^2}{(x+1)^2}$
\item\label{itEstBas1} $x+\frac{1}{x}$
\item\label{itEstBas6} $x+\frac{1}{x^2}$
\item\label{itEstBas9} $\frac{1}{x^2+1}$
\item\label{itEstBas2} $\frac{x}{x^2-1}$
\item\label{itEstBas3} $xe^{-x^2}$
%\item\label{itEstBas5} $x^4-x^2$
\item\label{itEstBas7} $\senh x$
\item\label{itEstBas8} $\cosh x$
\item\label{itEstBas8t} $\tanh x$
\item\label{itEstBas13} $\frac{x^3-1}{x^3+1}$, 
\item\label{itEstBas14} $\tfrac12\sen (2x)-\sen(x)$, 
\item\label{itEstBas15} $\frac{x}{\sqrt{x^2+1}}$
\item\label{itEstFuncB30} $\frac{\sqrt{x^2-1}}{x-2}$
%\item\label{itEstBas4} $\frac{(x+1)^3}{(x-1)^4}$
\end{enumerate}
\end{multicols}
\vspace{0.01cm}
\begin{sol}
%%%%%%%%%%%%%%%%%%
% \eqref{itEstBas9a} 
%  { Domínio}: $D=\R\backslash \{-1\}$. { Sinal}:
% $f(x)\geq 0$ para todo $x\in D$, e $f(x)=0$ se e somente se $x=0$.
% Assíntotas:
% como $\lim_{x\to -1^-}f(x)=\lim_{x\to -1^+}f(x)=+\infty$, { a reta $x=-1$ é
% assíntota vertical} 
% (é a única). Como 
% $$\lim_{x\to \pm\infty}\frac{x^2}{(x+1)^2}=\lim_{x\to
% \pm\infty}\frac{1}{(1+\frac{1}{x})^2}=1\,,$$ 
%  { a reta $y=1$ é assíntota
% horizontal} (a direita e a esquerda). Como
% $f'(x)=\frac{2x}{(x+1)^3}$, 
% \begin{center}
% \begin{bmlimage}\begin{tikzpicture}[scale=0.8]
% \tkzTabInit[nocadre, espcl=2,  color, colorV=lightgray!5, colorL=gray!15,
% colorC=gray!15]
% {$x$ /.6, $f'(x)$ /.6, Variaç. de $f$ /1.2}%
% {,$-1$, $0$, }%
% \tkzTabLine{,+,d,-,z,+}
% \tkzTabVar{-/,+D+/,-/mín.,+/}
% %\tkzTabLine{,\searrow,\text{mín.},h,\text{mín.},\nearrow,}
% \end{tikzpicture}\end{bmlimage}
% \end{center}
% $f$ possui um mínimo local no ponto $(0,f(0))=(0,0)$.
% Como $f''(x)=\frac{2(1-2x)}{(x+1)^4}$, temos:
% \begin{center}
% \begin{bmlimage}\begin{tikzpicture}[scale=0.8]
% \tkzTabInit[nocadre, espcl=2,  color, colorV=lightgray!5, colorL=gray!15,
% colorC=gray!15]
% {$x$ /.6, $f''(x)$ /.6, Conv. de $f$ /1.2}%
% {,$-1$, $\tfrac12$, }%
% \tkzTabLine{,+,d,+,z,-}
% \tkzTabLine{,\smile,d,\smile,z,\frown}
% %\tkzTabVar{-/,+D+/${+\infty}$/${+\infty}$,-/mín.,+/}
% %\tkzTabLine{,\searrow,\text{mín.},h,\text{mín.},\nearrow,}
% \end{tikzpicture}\end{bmlimage}
% \end{center}
% Logo, $f$ é convexa nos intervalos $]-\infty,-1[$ e $]-1,\tfrac12[$, possui um
% ponto de inflexão em $(\tfrac12,f(\tfrac12))=(\tfrac12,\tfrac19)$, e é côncava
% em $(\tfrac12,+\infty)$. Gráfico:
% \begin{center}
% \begin{bmlimage}\begin{tikzpicture}
% \draw[ ->] (-4,0)--(4,0);
% \draw[dashed] (-5,1)node[below]{$y=1$}--(4,1);
% \draw[ ->] (0,-0.5)--(0,4);
% \draw[dashed] (-1,-0.5)node[below]{$x=-1$}--(-1,4);
% \draw[thick, domain=-5:-1.9, samples=50] plot (\x,{\x^2/(\x+1)^2});
% \draw[thick, domain=-0.65:4, samples=100] plot (\x,{\x^2/(\x+1)^2});
% \end{tikzpicture}\end{bmlimage}
% \end{center}
% Observe que esse gráfico é o gráfico da função $(\frac{x}{x-1})$
OBS: Para as demais funções, colocamos somente um \emph{resumo} das
soluções, na forma de um gráfico no qual o leitor pode verificar os resultados
do seu estudo.

\eqref{itEstBas1} Ass. vert.: $x=0$. Ass. oblíqua: $y=x$.
\begin{center}
\begin{bmlimage}\begin{tikzpicture}[yscale=0.7]
\draw [thick, domain=-3:-0.3, samples=100] plot (\x,{(\x)+1/(\x)});
\draw [thick, domain=0.3:3, samples=100] plot (\x,{(\x)+1/(\x)});
 \draw [ ->] (-3,0)--(3,0) node[right] {$x$};
 \draw [ ->] (0,-3)--(0,3) node[left]{$x+\tfrac{1}{x}$};
 \fill (1,2) circle (0.45mm);
  \draw (1,2) node[below] {$\scriptscriptstyle{(1,2)}$};
  \fill (-1,-2) circle (0.45mm);
  \draw (-1,-2) node[above] {$\scriptscriptstyle{(-1,-2)}$};
\end{tikzpicture}\end{bmlimage}
\end{center}


\eqref{itEstBas6} 
Ass. vert.: $x=0$. Ass. obl.: $y=x$.
\begin{center}
\begin{bmlimage}\begin{tikzpicture}[yscale=0.7]
\draw [thick, domain=-3:-0.5, samples=100] plot
(\x,{\x+1/((\x)^2)});
\draw [thick, domain=0.6:3, samples=100] plot
(\x,{\x+1/((\x)^2)})node[right]{$x+\tfrac{1}{x^2}$};
 \draw [ ->] (-3,0)--(3,0) node[right] {$x$};
 \draw [ ->] (0,-3)--(0,3);
 \fill (1.256,1.88) circle (0.45mm);
  \draw (1.256,1.88) node[below]
{$\scriptscriptstyle{(2^{1/3},2^{1/3}+2^{-2/3})}$};
\draw (-1,0) node{$\shortmid$} node[above left]{$-1$};
\end{tikzpicture}\end{bmlimage}
\end{center}

\eqref{itEstBas9} 
\begin{center}
\begin{bmlimage}\begin{tikzpicture}
\draw [thick, domain=-3:3, samples=100] plot
(\x,{1/((\x)^2+1)});
\draw [ ->] (-3,0)--(3,0);
\draw [ ->] (0,-0.5)--(0,1.5)node[right]{$\tfrac{1}{x^2+1}$};
\fill (0.577,0.75) circle (0.45mm);
\draw[<-] (0.6,0.8)--(1.3,1)
node[right]{inflex: $(\tfrac{1}{\sqrt{3}},\tfrac34)$};
\fill (-0.577,0.75) circle (0.45mm);
\draw[<-] (-0.6,0.8)--(-1.3,1)
node[left]{inflex: $(-\tfrac{1}{\sqrt{3}},\tfrac34)$};
\draw (-5,2) node[left]{$\displaystyle{f'(x)=\frac{-2x}{(x^2+1)^2}}$};
\draw (-5,0.5) node[left]{$\displaystyle{f''(x)=\frac{2(3x^2-1}{(x^2+1)^3}}$};
\end{tikzpicture}\end{bmlimage}
\end{center}



\eqref{itEstBas2} 
\begin{center}
\begin{bmlimage}\begin{tikzpicture}[yscale=0.7]
\draw [ ->] (-3,0)--(3,0) node[right] {$x$};
\draw [ ->] (0,-3)--(0,3) node[left]{$\frac{x}{x^2-1}$};
\draw [thick, domain=-3:-1.2, samples=100] plot (\x,{\x/((\x)^2-1)});
\draw[dashed] (-1,-2.5)--(-1,2.5) node[below left]{$\scriptscriptstyle{x=-1}$};
\draw [thick, domain=-0.8:0.8, samples=100] plot (\x,{\x/((\x)^2-1)});
\draw [thick, domain=1.2:3, samples=100] plot (\x,{\x/((\x)^2-1)});
\draw[dashed] (1,-2.5)node[right]{$\scriptscriptstyle{x=1}$}--(1,2.5) ;
\draw[<-] (0.3,-0.1)--(1.5,-1)
node[right]{pt. inflex.: $\scriptstyle{(0,0)}$};
\draw (4,2) node[right]{$\displaystyle{f'(x)=\frac{-(1+x^2)}{(x^2-1)^2}}$};
\draw (4,0.5)
node[right]{$\displaystyle{f''(x)=\frac{-2x(3x^2+1)}{(x^2-1)^3}}$};
\end{tikzpicture}\end{bmlimage}
\end{center}

\eqref{itEstBas3}
\begin{center}
\begin{bmlimage}\begin{tikzpicture}
\draw [ ->] (-3,0)--(3,0) node[right] {$x$};
\draw [ ->] (0,-1)--(0,1) node[above]{$xe^{-x^2}$};
\draw [thick, domain=-2.5:2.5, samples=100] plot (\x,{\x*exp(-(\x)^2)});
\fill (0.707,0.428) circle (0.45mm);
  \draw[<-] (0.71,0.44)-- (0.9,1) node[right]
{$\scriptstyle{(\tfrac{1}{\sqrt{2}},\tfrac{1}{\sqrt{2}}e^{-\tfrac12})}$};
\fill (-0.707,-0.428) circle (0.45mm);
  \draw[<-] (-0.71,-0.5)-- (-0.9,-1) node[left]
{$\scriptstyle{(-\tfrac{1}{\sqrt{2}},-\tfrac{1}{\sqrt{2}}e^{-\tfrac12})}$};
\draw[<-] (0.1,-0.1)--(0.5,-1.3) node[right]{pt. inflex. $\scriptstyle{(0,0)}$};
\fill (1.225,0.273) circle (0.40mm);
\fill (-1.225,-0.273) circle (0.40mm);
\draw[<-] (1.225,0.24)--(1.5,-0.6)
node[right]{pt. inflex.: $\scriptstyle{(\sqrt{3/2},f(\sqrt{3/2}))}$};
\draw[<-] (-1.225,-0.24)--(-1.5,0.6)
node[left]{pt. inflex.: $\scriptstyle{(-\sqrt{3/2},f(\sqrt{3/2}))}$};
\draw (4,1.2) node[right]{$\displaystyle{f'(x)=(1-2x^2)e^{-x^2}}$};
\draw (4,0.5)
node[right]{$\displaystyle{f''(x)=-2x(3-2x^2)e^{-x^2}}$};
\end{tikzpicture}\end{bmlimage}
\end{center}

\eqref{itEstBas7}, \eqref{itEstBas8},
\eqref{itEstBas8t}:
\begin{center}
\begin{bmlimage}\begin{tikzpicture}[scale=0.5]
\draw [ ->] (0,-0.1)--(0,3);
\pgfmathsetmacro{\a}{2};
\draw [ ->] (-\a,0)--(\a,0);
\draw [thick, domain=-\a:\a, samples=100] plot (\x,{(exp(\x)+exp(-\x))/2})
node[right]{$\cosh x$};

\begin{scope}[xshift=9cm, yshift=1cm]
\draw [ ->] (0,-2)--(0,2);
\pgfmathsetmacro{\a}{1.6};
\draw [ ->] (-\a,0)--(\a,0);
\draw [thick, domain=-\a:\a, samples=100] plot (\x,{(exp(\x)-exp(-\x))/2})
node[right]{$\senh x$};
\end{scope}

\begin{scope}[xshift=18cm, yshift=1cm]
\draw [ ->] (0,-1.5)--(0,1.5);
\pgfmathsetmacro{\a}{3};
\draw [ ->] (-\a,0)--(\a,0);
\draw [thick, domain=-\a:\a, samples=100] plot
(\x,{(exp(\x)-exp(-\x))/(exp(\x)+exp(-\x))})
node[below right]{$\tanh x$};
\draw[dashed] (0,1)--(\a,1) node[above]{$x=+1$};
\draw[dashed] (0,-1)--(-\a,-1) node[below]{$x=-1$};
\end{scope}

\end{tikzpicture}\end{bmlimage}
\end{center}

\eqref{itEstBas13}
\begin{center}
\begin{bmlimage}\begin{tikzpicture}[yscale=0.7]
\draw [ ->] (-3,0)--(3,0);
\draw [ ->] (0,-3)--(0,3) node[right]{$\frac{x^3-1}{x^3+1}$};
\draw [thick, domain=-3:-1.2, samples=100] plot (\x,{((\x)^3-1)/((\x)^3+1)});
\draw [thick, domain=-0.8:3, samples=100] plot (\x,{((\x)^3-1)/((\x)^3+1)});
\draw[dashed] (-1,-3)node[left]{$\scriptscriptstyle{x=-1}$}--(-1,3) ;
\draw[dashed] (-3,1) node[below]{$\scriptscriptstyle{x=1}$}--(3,1) ;
\fill (0,-1) circle (0.45mm);
\fill (0.793, -0.3333) circle (0.45mm);
\draw[<-] (0.1,-1.1)--(1,-3)node[right]{Pt. de inflexão e crítico: $(0,-1)$};
\draw[<-] (0.8, -0.4)--(1.2,-1) node[right]{Pt. de inflexão:
$(2^{-1/3},-1/3)$};
\draw (1,0) node{$\shortmid$} node[above]{$1$};
\draw (4,2) node[right]{$\displaystyle{f'(x)=\frac{6x^2}{(x^3+1)^2}}$};
\draw (4,0.5)
node[right]{$\displaystyle{f''(x)=\frac{12x(1-2x^3)}{(x^3+1)^3}}$};
\end{tikzpicture}\end{bmlimage}
\end{center}

\eqref{itEstBas14}:
\begin{center}
\begin{bmlimage}\begin{tikzpicture}
\draw [ ->] (-0.2,0)--(2*pi+0.5,0);
\draw [ ->] (0,-1.5)--(0,1.5) node[right]{$\scriptstyle{\tfrac12\sen
(2x)-\sen(x)}$};
\draw [color=gray!20, domain=-1:2*pi+1, samples=100] plot (\x,{0.5*sin(2*\x
r)-sin(\x r)});
\draw [thick, domain=0:2*pi, samples=100] plot (\x,{0.5*sin(2*\x r)-sin(\x r)});
\foreach \k in {0, 0.666, 1.333, 2} {
\draw ({\k*pi},0) node{$\shortmid$};
}
\draw (0.666*pi,0) node[above]{$\tfrac{2\pi}{3}$};
\draw (1.333*pi,0) node[below]{$\tfrac{4\pi}{3}$};
\fill (1.318,-0.726) circle (0.40mm); 
\fill (4.965,0.726) circle (0.40mm); 
\fill (0,0) circle (0.40mm); 
\fill (pi,0) circle (0.40mm); 
\fill (2*pi,0) circle (0.40mm); 
% \draw[dashed] (-1,-3)node[left]{$\scriptscriptstyle{x=-1}$}--(-1,3) ;
% \draw[dashed] (-3,1) node[below]{$\scriptscriptstyle{x=1}$}--(3,1) ;
% \fill (0,-1) circle (0.45mm);
% \fill (0.793, -0.3333) circle (0.45mm);
% \draw[<-] (0.1,-1.1)--(1,-3)node[right]{Pt. de inflexão e crítico: $(0,-1)$};
% \draw[<-] (0.8, -0.4)--(1.2,-1) node[right]{Pt. de inflexão:
% $(2^{1/3},f(2^{1/2}))$};
% \draw (1,0) node{$\shortmid$} node[above]{$1$};
\end{tikzpicture}\end{bmlimage}
\end{center}

\eqref{itEstBas15}: 
\begin{center}
\begin{bmlimage}\begin{tikzpicture}
\draw [ ->] (-5,0)--(5,0);
\draw [ ->] (0,-1.3)--(0,1.3) node[left]{$\frac{x}{\sqrt{x^2+1}}$};
\draw [thick, domain=-5:5, samples=50] plot (\x,{\x/(sqrt((\x)^2+1))});
 \draw[dashed] (0,1)--(5,1)node[above]{$\scriptscriptstyle{y=1}$};
 \draw[dashed] (-5,-1) node[below]{$\scriptscriptstyle{y=-1}$}--(0,-1) ;
 \draw[<-] (0.2, -0.2)--(0.5,-0.7) node[right]{Pt. de inflexão: $(0,0)$};
\draw (6,0.5) node[right]{$\displaystyle{f'(x)=\frac{1}{(x^2+1)^{3/2}}}$};
\draw (6,-0.5)
node[right]{$\displaystyle{f''(x)=\frac{-3x}{(x^2+1)^{5/2}}}$};
\end{tikzpicture}\end{bmlimage}
\end{center}


\end{sol}
\end{exo}


\begin{exo}
Faça um estudo completo das seguintes funções.
\begin{multicols}{3}
\begin{enumerate}
\item\label{itEstFuncB1} $\ln |2-5x|$
\item\label{itEstFuncB3} $\ln(\ln x)$
\item\label{itEstFuncB7} $e^{-x}(x^2-2x)$.
\item\label{itEstFuncB70} $\sqrt[x]{x}$.
\item\label{itEstFuncB4} $\frac{\ln x}{\sqrt{x}}$
\item\label{itEstFuncB5} $\frac{\ln x-2}{(\ln x)^2}$
\item\label{itEstFuncB2} $\ln(e^{2x}-e^x+3)$
\item\label{itEstFuncB29} $(e^{|x|}-2)^3$
\item\label{itEstFuncB33} $\frac{e^x}{e^x-x}$
\item\label{itEstFuncB33a} $\arcos(\frac{1-x^2}{1+x^2})$
\item\label{itEstFuncB36} $\sqrt[5]{x^4(x-1)}$
%\item\label{itEstFuncB6} $\arcsen(2x^2-1)$
%\item   $\frac{\sqrt{x^2-1}}{x-2}$
%\item\label{itEstFuncB8} $\frac{\sqrt{x^2-1}}{x-2}$ 
%\item\label{itEstFuncB9} $(\ln x)^2+\ln x$.
\end{enumerate}
\end{multicols}
\vspace{0.01cm}

\begin{sol}
\eqref{itEstFuncB1}
\begin{center}
\begin{bmlimage}\begin{tikzpicture}[yscale=0.8]
\draw [ ->] (-4,0)--(4,0);
\draw [ ->] (0,-1.3)--(0,2.3) node[above]{$\ln |2-5x|$};
\draw [thick, domain=-4:0.3, samples=100] plot (\x,{ln(abs(2-5*(\x)))});
\draw[dashed] (0.4,-1.5)node[right]{$\scriptscriptstyle{x=\frac25}$}--(0.4,1.5);
\draw [thick, domain=0.5:4, samples=100] plot (\x,{ln(abs(2-5*\x))});
%  \draw[dashed] (-5,-1) node[below]{$\scriptscriptstyle{y=-1}$}--(0,-1) ;
% \fill (0,-1) circle (0.45mm);
% \fill (0.793, -0.3333) circle (0.45mm);
% \draw[<-] (0.1,-1.1)--(1,-3)node[right]{Pt. de inflexão e crítico: $(0,-1)$};
% \draw[<-] (0.2, -0.2)--(0.5,-0.7) node[right]{Pt. de inflexão: $(0,0)$};
% \draw (1,0) node{$\shortmid$} node[above]{$1$};
\end{tikzpicture}\end{bmlimage}
\end{center}


\eqref{itEstFuncB3}
\begin{center}
\begin{bmlimage}\begin{tikzpicture}[yscale=0.7]
\draw [ ->] (0,0)--(5,0);
\draw [ ->] (0,-1.3)--(0,2.3) node[left]{$\ln(\ln x)$};
\draw [thick, domain=1.2:5, samples=100] plot (\x,{ln(ln(\x))});
% \draw[dashed]
%(0.4,-1.5)node[right]{$\scriptscriptstyle{x=\frac25}$}--(0.4,1.5);
% \draw [thick, domain=0.5:4, samples=100] plot (\x,{ln(abs(2-5*\x))});
 \draw[dashed] (1,-2) node[left]{$\scriptscriptstyle{x=1}$}--(1,2) ;
% \fill (-0.693,1.012) circle (0.45mm);
% \fill (-1.365,1.033) circle (0.45mm);
% \fill (2.46,4.86) circle (0.45mm);
% \fill (0.793, -0.3333) circle (0.45mm);
% \draw[<-] (0.1,-1.1)--(1,-3)node[right]{Pt. de inflexão e crítico: $(0,-1)$};
% \draw[<-] (0.2, -0.2)--(0.5,-0.7) node[right]{Pt. de inflexão: $(0,0)$};
% \draw (1,0) node{$\shortmid$} node[above]{$1$};
\end{tikzpicture}\end{bmlimage}
\end{center}

\eqref{itEstFuncB7}
\begin{center}
\begin{bmlimage}\begin{tikzpicture}
\newcommand{\funcao}[1]{2.5*exp( -1*(#1) )*( (#1)^2 - 2*(#1))}
\draw [ ->] (-1,0)--(6.5,0);
\draw [ ->] (0,-1)--(0,2.3) node[right]{$e^{-x}(x^2-2x)$};
\draw [thick, domain=-0.35:5.5, samples=100] plot (\x,{\funcao{\x}});
\fill ({2-sqrt(2)},{\funcao{2-sqrt(2)}}) circle (0.40mm);
\draw ({2-sqrt(2)},{\funcao{2-sqrt(2)}})
node[below]{$\scriptstyle{(2-\sqrt{2},f(2-\sqrt{2}))}$};
\draw ({2+sqrt(2)},{\funcao{2+sqrt(2)}})
node[above]{$\scriptstyle{(2+\sqrt{2},f(2+\sqrt{2}))}$};
\fill ({2+sqrt(2)},{\funcao{2+sqrt(2)}}) circle (0.40mm);
\fill ({(6+sqrt(10))/2},{\funcao{(6+sqrt(10))/2}}) circle (0.40mm);
\draw[<-] ({(6+sqrt(10))/2+0.1},{\funcao{(6+sqrt(10))/2}+0.1})--
({(6+sqrt(10))/2+0.5},{\funcao{(6+sqrt(10))/2}+0.3})
node[right]{$\scriptstyle{(3+\sqrt{10}/2,f(3+\sqrt{10}/2))}$};
\fill ({(6-sqrt(10))/2},{\funcao{(6-sqrt(10))/2}}) circle (0.40mm);
\draw[<-] ({(6-sqrt(10))/2-0.1},{\funcao{(6-sqrt(10))/2}+0.1})--
({(6-sqrt(10))/2-0.5},{\funcao{(6-sqrt(10))/2}+2})
node[right]{$\scriptstyle{(3-\sqrt{10}/2,f(3-\sqrt{10}/2))}$};
\draw[<-] (5,-0.2)--(4.5,-0.5)node[below]{ass. horiz.: $y=0$};
\draw (6,2)
node[right]{$\displaystyle{f'(x)=-(x^2-4x+2)e^{-x}}$};
\draw (6,1.5)
node[right]{$\displaystyle{f''(x)=(x^2-6x+6)e^{-x}}$};
\end{tikzpicture}\end{bmlimage}
\end{center}




\eqref{itEstFuncB70}


\begin{center}
\begin{bmlimage}\begin{tikzpicture}[yscale=0.7]
\draw [ ->] (0,0)--(2.5,0);
\draw [ ->] (0,-0.3)--(0,1.5) node[left]{$\sqrt[x]{x}$};
\draw [thick, domain=0.2:6, samples=100, <-] plot 
(\x,{exp(ln(\x)/\x)});
\fill (2.718,1.444) circle (0.45mm) node[above]{máx. glob.:
$(e,\sqrt[e]{e})$};
\coordinate (A) at (0.539,0.318);
\coordinate (B) at (5.04,1.37);
\fill (A) circle (0.45mm);
\fill (B) circle (0.45mm);
\draw[<-] (A)--(1.2,-0.3) node[right]{pt. infl.:
$(x_1,f(x_1))$};
\draw[<-] (B)--(5.2,1.9) node[right]{pt. infl.:
$(x_2,f(x_2))$};
%\draw[<-] (-0.67,0.9)--(0.2,0.4)node[right]{mín. global: $(\ln \tfrac12,f(\ln
%\tfrac12))$};
%\fill (-1.365,1.033) circle (0.45mm);
%\draw[<-] (-1.4,0.9)--(-1.6,0.5)node[left]{pt. infl.};
%\fill (2.46,4.86) circle (0.45mm);
%\draw[<-] (2.55,4.7)--(3,4)node[right]{pt. infl.};
\draw[dashed] (0,1)--(6,1);
\draw (2,1) node[below right]{Ass. Horiz.: $y=1$};
%\draw (5,2.5)
%node[right]{$\displaystyle{f'(x)=\frac{e^x(2e^x-1)}{e^{2x}-e^x+3}}$};
%\draw (5,0.5)
%node[right]{$\displaystyle{f''(x)=\frac{e^x(12e^x-e^{2x}
%-3)}{(e^{2x}-e^x+3)^2}}$};
\end{tikzpicture}\end{bmlimage}
\end{center}
Os pontos de inflexão são soluções da equação $(1-\ln
x)^2-3x+2x\ln x=0$. Pode ser mostrado que esses satisfazem
$x_1\simeq 0.58$, $x_1\simeq 4.37$.  

\eqref{itEstFuncB4}
\begin{center}
\begin{bmlimage}\begin{tikzpicture}[yscale=0.7]
\newcommand{\funcao}[1]{ln( 5*(#1) )/sqrt( 5*(#1) ) }
\draw [ ->] (0,0)--(8,0);
\draw [ ->] (0,-1.3)--(0,2.3) node[left]{$\frac{\ln x}{\sqrt{x}}$};
\draw [thick, domain=0.1:8, samples=100] plot (\x,{\funcao{\x}});
% \draw[dashed] (1,-2) node[left]{$\scriptscriptstyle{x=1}$}--(1,2) ;
\fill ({2.718^2/5},{\funcao{2.718^2/5}}) circle (0.40mm);
\draw ({2.718^2/5},{\funcao{2.718^2/5}}) node[above]{$(e^2,2/e)$};
\fill ({2.718^(8/3)/5},{\funcao{2.718^(8/3)/5}}) circle (0.40mm);
\draw[<-] ({2.718^(8/3)/5+0.1},{\funcao{2.718^(8/3)/5}+0.2})--
({2.718^(8/3)/5+0.3},{\funcao{2.718^(8/3)/5}+1.3})
node[above]{pt. infl.: $(e^{8/3},f(e^{8/3}))$};
\draw (6,2.8)
node[right]{$\displaystyle{f'(x)=\frac{2-\ln x}{2x^{3/2}}}$};
\draw (6,1.5)
node[right]{$\displaystyle{f''(x)=-\frac{\sqrt{x}}{2}\frac{4-\tfrac32 \ln
x}{|x|^3}}$};
\draw[<-] (5,-0.2)--(4,-0.6) node[below]{ass. horiz.: $y=0$};

\end{tikzpicture}\end{bmlimage}
\end{center}

\eqref{itEstFuncB5} 
%$\frac{\ln x-2}{(\ln x)^2}$
\begin{center}
\begin{bmlimage}\begin{tikzpicture}[yscale=0.5]
\newcommand{\funcao}[1]{( ln( (#1) ) -2)/ ( (ln( (#1) ))^2 ) }
\draw [ ->] (0,0)--(12,0);
\draw [ ->] (0,-1.3)--(0,2.3) node[left]{$\frac{\ln x-2}{(\ln x)^2}$};
\draw[dashed] (1,1)node[above]{$x=1$}--(1,-5);
\draw [thick, domain=0.1:0.5, samples=100] plot (\x,{\funcao{\x}});
\draw [thick, domain=1.6:11, samples=100] plot (\x,{\funcao{\x}});
% \draw[dashed] (1,-2) node[left]{$\scriptscriptstyle{x=1}$}--(1,2) ;
% \fill ({2.718^2/5},{\funcao{2.718^2/5}}) circle (0.40mm);
% \draw ({2.718^2/5},{\funcao{2.718^2/5}}) node[above]{$(e^2,2/e)$};
% \fill ({2.718^(8/3)/5},{\funcao{2.718^(8/3)/5}}) circle (0.40mm);
% \draw[<-] ({2.718^(8/3)/5+0.1},{\funcao{2.718^(8/3)/5}+0.2})--
% ({2.718^(8/3)/5+0.3},{\funcao{2.718^(8/3)/5}+1.3})
% node[above]{pt. infl.: $(e^{8/3},f(e^{8/3}))$};
% \draw (6,2.8)
% node[right]{$\displaystyle{f'(x)=\frac{2-\ln x}{2x^{3/2}}}$};
% \draw (6,1.5)
% node[right]{$\displaystyle{f''(x)=-\frac{\sqrt{x}}{2}\frac{4-\tfrac32 \ln
% x}{|x|^3}}$};
\draw[<-] (6,0.2)--(5,0.6) node[above]{ass. horiz.: $y=0$};
\draw (7,-1.5) node[right]{máx. global em $(e^4,f(e^4))$};
\draw (7,-3) node[right]{pt. infl. em
$(e^{1+\sqrt{13}},f(e^{1+\sqrt{13}})$};
\draw (5,-4.5) node[right]{$f'(x)=\frac{4-\ln x}{x(\ln x)^3}$, 
$f''(x)=\frac{(\ln x)^2-2\ln x-12}{x^2(\ln x)^4}$};
\end{tikzpicture}\end{bmlimage}
\end{center}


\eqref{itEstFuncB2}
Ass. horiz.: $y=\ln 3$. Ass. obl.: $y=2x$.
\begin{center}
\begin{bmlimage}\begin{tikzpicture}[yscale=0.7]
\draw [ ->] (-4,0)--(2.5,0);
\draw [ ->] (0,-1.3)--(0,2.3) node[left]{$\ln(e^{2x}-e^x+3)$};
\draw [thick, domain=-4:2.6, samples=100] plot (\x,{ln(exp(2*\x)-exp(\x)+3)});
\fill (-0.693,1.012) circle (0.45mm);
\draw[<-] (-0.67,0.9)--(0.2,0.4)node[right]{mín. global: $(\ln \tfrac12,f(\ln
\tfrac12))$};
\fill (-1.365,1.033) circle (0.45mm);
\draw[<-] (-1.4,0.9)--(-1.6,0.5)node[left]{pt. infl.};
\fill (2.46,4.86) circle (0.45mm);
\draw[<-] (2.55,4.7)--(3,4)node[right]{pt. infl.};
\draw[dashed] (-4,{ln(3)})node[above]{$y=\ln 3$}--(-1,{ln(3)});
\draw (5,2.5)
node[right]{$\displaystyle{f'(x)=\frac{e^x(2e^x-1)}{e^{2x}-e^x+3}}$};
\draw (5,0.5)
node[right]{$\displaystyle{f''(x)=\frac{e^x(12e^x-e^{2x}
-3)}{(e^{2x}-e^x+3)^2}}$};
\end{tikzpicture}\end{bmlimage}
\end{center}

\eqref{itEstFuncB29} Observe que $(e^{|x|}-2)^3$ é par, e não
é derivável em $x=0$. 
\begin{center}
\begin{bmlimage}\begin{tikzpicture}[yscale=0.7]
\draw [ ->] (-2,0)--(2,0);
\draw [ ->] (0,-1.3)--(0,2.3) node[above]{$(e^{|x|}-2)^3$};
\draw [thick, domain=0:1.2, samples=50] plot
(\x,{(exp(\x)-2)^3});
\draw [thick, domain=0:1.2, samples=50] plot
(-\x,{(exp(\x)-2)^3});
%\fill (-0.693,1.012) circle (0.45mm);
\draw[<-] (0.1,-1.05)--(1,-1.3) node[right]{mín. global: $(0,-1)$};
%\fill (-1.365,1.033) circle (0.45mm);
\draw[<-] (0.683,-0.1)--({0.693+0.5},-0.5) 
node[right]{pt. infl.: $(\ln 2,0)$};
\draw[<-] (-0.683,-0.1)--({-0.693-0.5},-0.5) 
node[left]{pt. infl.: $(-\ln 2,0)$};
\fill (0.693,0) circle (0.45mm);
\fill (-0.693,0) circle (0.45mm);
%\draw[<-] (2.55,4.7)--(3,4)node[right]{pt. infl.};
%\draw[dashed] (-4,{ln(3)})node[above]{$y=\ln 3$}--(-1,{ln(3)});
%\draw (5,2.5)
%node[right]{$\displaystyle{f'(x)=\frac{e^x(2e^x-1)}{e^{2x}-e^x+3}}$};
%\draw (5,0.5)
%node[right]{$\displaystyle{f''(x)=\frac{e^x(12e^x-e^{2x}
%-3)}{(e^{2x}-e^x+3)^2}}$};
\end{tikzpicture}\end{bmlimage}
\end{center}

\eqref{itEstFuncB33}
\begin{center}
\begin{bmlimage}\begin{tikzpicture}[yscale=0.7]
\draw [ ->] (-3,0)--(3,0);
\draw [ ->] (0,-0.3)--(0,1.4) node[above]{$\frac{e^x}{e^x-x}$};
\draw [thick, domain=-3:3, samples=50] plot
(\x,{exp(\x)/(exp(\x)-\x)});
\draw[dashed] (0,1)--(3,1);
%\draw [thick, domain=0:1.2, samples=50] plot
%(-\x,{(exp(\x)-2)^3});
%\fill (-0.693,1.012) circle (0.45mm);
%\draw[<-] (0.1,-1.05)--(1,-1.3) node[right]{mín. global: $(0,-1)$};
%\fill (-1.365,1.033) circle (0.45mm);
%\draw[<-] (0.683,-0.1)--({0.693+0.5},-0.5) 
%node[right]{pt. infl.: $(\ln 2,0)$};
%\draw[<-] (-0.683,-0.1)--({-0.693-0.5},-0.5) 
%node[left]{pt. infl.: $(-\ln 2,0)$};
%\fill (0.693,0) circle (0.45mm);
%\fill (-0.693,0) circle (0.45mm);
%\draw[<-] (2.55,4.7)--(3,4)node[right]{pt. infl.};
%\draw[dashed] (-4,{ln(3)})node[above]{$y=\ln 3$}--(-1,{ln(3)});
%\draw (5,2.5)
%node[right]{$\displaystyle{f'(x)=\frac{e^x(2e^x-1)}{e^{2x}-e^x+3}}$};
%\draw (5,0.5)
%node[right]{$\displaystyle{f''(x)=\frac{e^x(12e^x-e^{2x}
%-3)}{(e^{2x}-e^x+3)^2}}$};
\end{tikzpicture}\end{bmlimage}
\end{center}

\eqref{itEstFuncB33a}
\begin{center}
\begin{bmlimage}\begin{tikzpicture}[yscale=1]
\draw [ ->] (-4,0)--(4,0);
\draw [ ->] (0,-0.3)--(0,3.3) node[above right]{$\displaystyle{
\arcos(\frac{1-x^2}{1+x^2})}$};
\draw [thick, domain=-3.7:3.7, samples=51] plot
(\x,{3.1415/180*acos((1-\x*\x)/(1+\x*\x))});
\draw[dashed] (-4,3.14)--(4,3.14) node[right]{$y=\pi$};
\draw (5,1.7) node{Obs: a função não é derivável em $x=0$!};
\end{tikzpicture}\end{bmlimage}
\end{center}

\eqref{itEstFuncB36}

\begin{center}
\begin{bmlimage}\begin{tikzpicture}[yscale=0.9]
%\newcommand{\funcao}[1]{(abs(#1))^(0.8)*(abs((#1)-1))^(0.2)*(-1)};
\draw [ ->] (-3,0)--(3,0);
\draw [ ->] (0,-0.3)--(0,1.4) node[above]{
$\sqrt[5]{x^4(x-1)}$};
\pgfmathsetmacro{\e}{0.002};
\coordinate (A) at (0.8,-0.606);
\fill (A) circle (0.45mm);
\draw[<-] (0.8,-0.73)--(1.3,-1) node[right]{mín. loc.: 
$(\tfrac45,f(\tfrac45))$};
\draw [thick, domain=-2:-\e, samples=50] plot
%%PROBLEMA:
(\x,{-exp(0.8*ln(abs(\x))+0.2*ln(abs(\x-1)))});
\draw [thick, domain=\e:{1-\e}, samples=50] plot
%(\x,{(abs(\x))^(0.8)*(abs(\x-1))^(0.2)*(-1)});
(\x,{-exp(0.8*ln(abs(\x))+0.2*ln(abs(\x-1)))});
\draw [thick, domain={1+\e}:3, samples=50] plot
(\x,{exp(0.8*ln(abs(\x))+0.2*ln(abs(\x-1)))});
\draw[<-] (-0.1,0.1)--(-1.5,1) node[left]{máx. loc.: $(0,0)$};
\draw[thick]
({1-\e},{-exp(0.8*ln(abs(1-\e))+0.2*ln(abs(1-\e-1)))})--
({1+\e},{exp(0.8*ln(abs(1+\e))+0.2*ln(abs(1+\e-1)))});
\draw[dashed] (-2,-2.2)--(3,2.8) node[right]{Ass. obl.:
$y=x-\tfrac15$.};
\end{tikzpicture}\end{bmlimage}
\end{center}
Obs: $f'(x)=f(x)\varphi(x)$, onde
$\varphi(x)=\tfrac15(\tfrac{4}{x}+\tfrac{1}{x-1})$. 
A função não é derivável nem em $x=0$, nem em $x=1$
(apesar de ser contínua nesses pontos).
$f''(x)=(\varphi(x)^2+\varphi'(x))f(x)=-\tfrac{4}{25}
\frac{f(x)}{x^2(x-1)^2}$, logo, $f$ é convexa em
$(-\infty,0)$ e $(0,1)$, côncava em $(1,\infty)$.
Essa função possui uma assíntota \emph{oblíqua}:
$y=x-\tfrac15$.
\end{sol}
\end{exo}

%\begin{itemize}
% \item Antes de tudo, encontrar o \emph{domínio} de $f$. O domínio precisa ser
%especificado para evitar indeterminações, divisões por zero, e raizes
%ou logaritmos de números negativos.
%Em particular, alguns pontos excluidos do domínio serão estudados depois, caso
%sejam associados a assíntotas verticais.
%\item Se for possível, estudar os \emph{zeros} e o \emph{sinal} de $f$.
%\item Determinar se $f$ possui algumas \emph{simetrias}, via o estudo da
%\emph{paridade} de $f$. Exemplos de funções pares são $x^p$ ($p$ par), $\cos
%x$, $\cosh x$, etc.
%\item Estudar o comportamento de $f$ para valores de $x$ grandes (se o domínio
%o permite). Isto é, procurar \emph{assíntotas horizontais}. Os limites
%envolvidos podem precisar da regra de Bernoulli-l'Hopital.
% \item Estudar o comportamento de $f$ perto dos valores de $x$ onde $f(x)$ toma
%valores grandes. Isto é, procurar \emph{assíntotas verticais} calculando os
%limites laterais $\lim_{x\to a^+}f(x)$ e $\lim_{x\to a^+}f(x)$ nos pontos $a$
%perto dos quais $f$ não é limitada.
%\item Estudar a derivabilidade de $f$ e da sua derivada. Em particular,
%procurar os \emph{pontos críticos de $f$}. Deduzir a sua \emph{variação} (via o
%estudo do sinal de $f'$).
%\item Estudar a convexidade de $f$, via o sinal da segunda derivada.
%Às vezes, o estudo do sinal de $f''$ nos pontos críticos (se tiver) pode
%permitir determinar se é mínimo ou máximo local.
%\item Juntando todas essas informações, montar o gráfico de $f$.
%Por exemplo, se $f$ é par, o gráfico é simétrico com respeito ao eixo $y$.
%Na hora de montar o gráfico, pode ser necessário calcular mais alguns limites.
%\end{itemize}
%
%
%
%
%O QUE NAO CAI NA PROVA: linearizacao.
%
%\begin{ex} Comecemos com $f(x)=\frac{x+1}{1-x}$, cujo domínio é $D=\bR\setminus
%\{1\}$. A função se anula no ponto $x=-1$, e o seu sinal é dado por:
%\begin{center}
%\begin{bmlimage}\begin{tikzpicture}
%\tkzTabInit[lgt=3, nocadre, espcl=2, colorC=red, colorV=blue]
%{Valores de $x$: /.6,  $x+1$ /.6, $1-x$ /.6, $f(x)$ /.8}%
%{,$-1$, $1$,}
%\tkzTabLine{,-,z,+, ,+,}
%\tkzTabLine{,+, ,+,z,-,}
%\tkzTabLine{,-,z,+,d,-,}
%\end{tikzpicture}\end{bmlimage}
%\end{center}
%(A dupla barra em $x=1$ é para indicar que $f$ não é definida em $x=1$.)
%A funçao não é nem par, nem ímpar.
%Como
%$$
%\lim_{x\to \pm\infty}\frac{x+1}{1-x}=\lim_{x\to
%\pm\infty}\frac{1+\frac{1}{x}}{\frac{1}{x}-1}=
%\frac{1}{-1}=-1\,,
%$$
%$f$ possui a reta $y=-1$ como assíntota horizontal.
%Por outro lado, como
%$$
%\lim_{x\to 1^+}\frac{x+1}{1-x}=-\infty\,,\quad 
%\lim_{x\to 1^-}\frac{x+1}{1-x}=+\infty\,,\quad 
%$$
%$f$ possui a reta $x=1$ como assíntota vertical.
%A derivada existe em todo $x\neq 1$, e vale
%$$
%f'(x)=\frac{(x+1)'(1-x)-(x+1)(1-x)'}{(1-x)^2}=
%\frac{1-x+(x+1)}{(1-x)^2}=
%\frac{2}{(1-x)^2}\,.
%$$
%O sinal de $f'$ dá logo a tabela de variação de $f$:
%\begin{center}
%\begin{bmlimage}\begin{tikzpicture}[scale=0.8]
%\tkzTabInit[nocadre, espcl=2,  color, colorV=lightgray!5, colorL=blue!15,
%colorC=blue!15]
%{$x$ /.6, $f'(x)$ /.6, Variaç. de $f$ /1.2}%
%{,$1$, }%
%\tkzTabLine{,+,d,+,}
%\tkzTabVar{-/,+D-/$\scriptscriptstyle{+\infty}$/$\scriptscriptstyle{-\infty}$,+/
%}
%%\tkzTabLine{,\searrow,\text{mín.},h,\text{mín.},\nearrow,}
%\end{tikzpicture}\end{bmlimage}
%\end{center}
%(Indicamos o fato de $x=1$ ser uma assíntota vertical.)
%Assim, $f$ não possui pontos críticos, e é crescente nos intervalos
% $]-\infty,1[$ e $]1.\infty[$.
%A segunda derivada se calcula facilmente (para $x\neq 0$):
%$$f''(x)=2((1-x)^{-2})'=2(-2)(1-x)^{-3}(-1)=\frac{4}{(1-x)^3}\,.$$
%Esta muda de sinal em $x=1$, e permite descrever a convexidade de $f$:
%\begin{center}
%\begin{bmlimage}\begin{tikzpicture}[scale=0.8]
%\tkzTabInit[nocadre, espcl=2,  color, colorV=lightgray!5, colorL=blue!15,
%colorC=blue!15]
%{$x$ /.6, $f''(x)$ /.6, Conv. de $f$ /1.2}%
%{,$1$, }%
%\tkzTabLine{,+,d,-,}
%\tkzTabLine{,\smile,d,\frown,}
%\end{tikzpicture}\end{bmlimage}
%\end{center}
%Isto é, $f$ é convexa em $]-\infty,1[$, côncava em $]1,\infty[$. Assim, o
%gráfico é da forma
%\begin{center}
%\begin{bmlimage}\begin{tikzpicture}[yscale=0.6]
%\draw[>=latex, ->] (-4,0)--(5,0);
%\draw[>=latex, ->] (0,-4)--(0,+2.5);
%\draw[dashed] (-4,-1)node[below right]{$y=-1$}--(5,-1);
%\draw[dashed] (1,-4)--(1,3)node[right]{$x=1$};
%\draw[thick, domain=-4:0.5, samples=50] plot (\x,{(\x+1)/(1-\x)});
%\draw[thick, domain=1.5:5, samples=50] plot (\x,{(\x+1)/(1-\x)});
%\draw (-1,0) node{$\shortmid$} node[above]{$-1$};
%\draw (1,0) node{$\shortmid$} node[above right]{$1$};
%\end{tikzpicture}\end{bmlimage}
%\end{center}
%\end{ex}
%
%\begin{exo} Faça um estudo completo das seguintes funções. 
%% \begin{multicols}{1}
%\begin{enumerate}
%\item\label{itexoEstudA1} $\bigl(\frac{x-1}{x}\bigr)^2$ (Segunda prova,
%primeiro semestre 2011)
%\item \label{itexoEstudA2} $\frac{x^2-1}{x^2+1}$
%\item \label{itexoEstudA3} $x(\ln x)^2$ (Segunda prova, 2010)
%\end{enumerate}
%% \end{multicols}
%\begin{sol}
%%%%%%%%%%%%%%%%%%%%%%%%%%%%%%%%%5
%\eqref{itexoEstudA1}
%O domínio de $\bigl(\frac{x-1}{x}\bigr)^2$ é $D=\bR\setminus \{0\}$, o sinal é
%sempre não-negativo, tem um zero
%em $x=1$. $f$ não é par, nem ímpar.
%Os limites relevantes são $\lim_{x\to 0^{\pm}}f(x)=+\infty$, logo $x=0$ é
%assíntota vertical, e
%$$\lim_{x\to \pm\infty}\bigl(\frac{x-1}{x}\bigr)^2=\Bigl(\lim_{x\to \pm
%\infty}\frac{x-1}{x}\Bigr)^2==\Bigl(\lim_{x\to \pm
%\infty}\bigl(1-\frac{1}{x}\bigr)\Bigr)^2=1^2=1\,.$$
%Logo, $y=1$ é assíntota horizontal. 
%$f$ é derivável em $D$, e $f'(x)=\frac{2(x-1)}{x^3}$.
%\begin{center}
%\begin{bmlimage}\begin{tikzpicture}
%\tkzTabInit[nocadre,espcl=2,  color, colorV=lightgray!5, colorL=blue!15,
%colorC=blue!15]
%{$x$ /.6,  $f'(x)$ /.6, Var. de $f$ /1.3}%
%{,$0$, $1$,}%
%%\tkzTabLine{,+,z,+,,+,}
%\tkzTabLine{,+,d,-,z,+,}
%\tkzTabVar{-/,+D+/$+\infty$/$+\infty$,-/mín,+/,}
%%\tkzTabLine{,\searrow,\text{mín.},h,\text{mín.},\nearrow,}
%\end{tikzpicture}\end{bmlimage}
%\end{center}
%$f$ possui um mínimo global em $(1,0)$.
%A segunda derivada é dada por $f''(x)=\frac{2(3-2x)}{x^4}$. Ela se anula em
%$x=\tfrac32$, e muda de sinal neste ponto:
%\begin{center}
%\begin{bmlimage}\begin{tikzpicture}
%\tkzTabInit[nocadre,espcl=2,  color, colorV=lightgray!5, colorL=blue!15,
%colorC=blue!15]
%{$x$ /.6,  $f''(x)$ /.7, Conv. de $f$ /1.2}%
%{,$0$, $\tfrac32$,}%
%\tkzTabLine{,+,d,+,z,-,}%
%\tkzTabLine{,\smile,d,\smile,z,\frown,}%
%\end{tikzpicture}\end{bmlimage}
%\end{center}
%Logo, $f$ é convexa em $]-\infty,0[$ e $]0,\frac32[$,  côncava em
%$]\frac32,\infty[$, e possui um ponto de inflexão em 
%$(\tfrac{3}{2},f(\tfrac{3}{2}))=(\tfrac{3}{2},\tfrac19)$.
%\begin{center}
%\begin{bmlimage}\begin{tikzpicture}
%\draw [thick, domain=-4:-1.2, samples=100] plot (\x,{(\x-1)^2/\x^2});
%\draw [thick, domain=0.4:4, samples=100] plot (\x,{(\x-1)^2/\x^2});
%\draw [>=latex, ->] (-4,0)--(4,0) node[right] {$x$};
%\draw [>=latex, ->] (0,-0.1)--(0,3) node[left] {$f(x)$};
%\draw [dotted] (-4,1)--(4,1) node[above] {$y=1$};
%\draw [dotted] (0,0)--(0,3.5) node[right] {$x=0$};
%\fill (1,0) circle (0.35mm);
%\draw (1,0) node[below] {$(1,0)$};
%\fill (1.5,0.1111) circle (0.35mm);
%\draw [>=latex, <-] (1.52,0.0911)--(2,-0.3) node[right]
%{$(\tfrac{3}{2},\tfrac{1}{9})$};
%\end{tikzpicture}\end{bmlimage}
%\end{center}
%\eqref{itexoEstudA2}
%Domínio: $D=\bR$.  Sinal: $f(x)$ é $\geq 0$ se $|x|\geq 1$, $<0$ caso contrário.
%Como $f(-x)=\frac{(-x)^2-1}{(-x)^2+1}=\frac{x^2-1}{x^2+1}=f(x)$, $f$ é par.
%Como 
%$$
%\lim_{x\to \pm \infty}\frac{x^2-1}{x^2+1}=\lim_{x\to \pm
%\infty}\frac{1-\tfrac{1}{x^2}}{1+\tfrac{1}{x^2}}=1\,,
%$$
%a reta $y=1$ é assíntota horizontal. Não tém assíntotas verticais.
%A derivada é dada por $f'(x)=\frac{4x}{(x^2+1)^2}$. Logo,
%\begin{center}
%\begin{bmlimage}\begin{tikzpicture}
%\tkzTabInit[nocadre,espcl=2,  color, colorV=lightgray!5, colorL=blue!15,
%colorC=blue!15]
%%{$x$ /.6,  $f'(x)$ /.9, Variação de $f$ /1.5}%
%{$x$ /.5,  $f'(x)$ /.5, Var. de $f$ /1}{,$0$,}
%%\tkzTabLine{,+,z,+,,+,}
%\tkzTabLine{,-,z,+,}
%\tkzTabVar{+/,-/\text{min.},+/,}
%%\tkzTabLine{,\searrow,\text{mín.},h,\text{mín.},\nearrow,}
%\end{tikzpicture}\end{bmlimage}
%\end{center}
%O mínimo tém coordenadas $(0,f(0))=(0,-1)$. A segunda derivada é dada por
%$f''(x)=\frac{4(1-3x^2)}{x^2+1}$, logo:
%\begin{center}
%\begin{bmlimage}\begin{tikzpicture}
%\tkzTabInit[nocadre,espcl=2,  color, colorV=lightgray!5, colorL=blue!15,
%colorC=blue!15]
%%{$x$ /.5,  $f''(x)$ /.7, Conc. de $f$ /1.3}
%{$x$ /.5,  $f''(x)$ /.5, Conc. de $f$ /1}
%{,$-1/\sqrt{3}$, $-1/\sqrt{3}$,}
%\tkzTabLine{,-,z,+,z,-,}
%\tkzTabLine{,{\frown},,\smile,,\frown,}
%\end{tikzpicture}\end{bmlimage}
%\end{center}
%Pontos de inflexão:
%$(\tfrac{-1}{\sqrt{3}},f(\tfrac{-1}{\sqrt{3}}))=(\tfrac{-1}{\sqrt{3}},-\tfrac{1}
%{2})$,
%$(\tfrac{+1}{\sqrt{3}},f(\tfrac{+1}{\sqrt{3}}))=(\tfrac{+1}{\sqrt{3}},-\tfrac{1}
%{2})$.
%\begin{center}
%\begin{bmlimage}\begin{tikzpicture}[scale=1.3]
%\draw [thick, domain=-4:4, samples=100] plot (\x,{(\x^2-1)/(\x^2+1)});
%\draw [>=latex, ->] (-4,0)--(4,0) node[right] {$x$};
%\draw [>=latex, ->] (0,-0.1)--(0,1.5) node[left] {$f(x)$};
%\draw [dotted] (-4,1)--(4,1) node[above left] {$y=1$};
%%\fill (1,0) circle (0.35mm);
%%\fill (-1,0) circle (0.35mm);
%\fill (0,-1) circle (0.35mm);
%\draw (0,-1) node[below] {$(0,-1)$};
%\fill (-0.577,-0.5) circle (0.35mm);
%\draw (-0.577,-0.6) node[left]{$(\tfrac{-1}{\sqrt{3}},-\half)$};
%\fill (+0.577,-0.5) circle (0.35mm);
%\draw (+0.577,-0.6) node[right]{$(\tfrac{+1}{\sqrt{3}},-\half)$};
%\end{tikzpicture}\end{bmlimage}
%\end{center}
%\eqref{itexoEstudA3}
%O domínio de  $f(x)=x(\ln x)^2$ é
%$D=(0,+\infty)$, e o seu sinal é: $f(x)\geq 0$ para todo $x\in D$.
%A função não é { par} nem { ímpar}.
%Como $\lim_{x\to \infty}f(x)=+\infty$, não tém assintota horizontal.
%Para ver se tém assíntota vertical em $x=0$, calculemos 
%$\lim_{x\to 0^+}f(x)=\lim_{x\to 0^+}\frac{(\ln x)^2}{1/x}$. Como ambas funções
%$(\ln x)^2$ e $1/x$ são deriváveis em $]0,1[$ e tendem a $+\infty$ quando $x\to
%0^+$, apliquemos a regra de B.H.:
%$$
%\lim_{x\to 0^+}\frac{(\ln x)^2}{1/x}=
%\lim_{x\to 0^+}\frac{2(\ln x)1/x}{-1/x^2}=
%-2\lim_{x\to 0^+}x\ln x\,.
%$$
%Usando a regra de B.H. de novo, pode ser mostrado que esse segundo limite é
%zero (ver Exemplo \ref{Ex:xlogxemzero}). Logo, $\lim_{x\to 0^+}f(x)=0$: não
%tém assíntota vertical em $x=0$.
%A derivada é dada por $f'(x)=\ln x(\ln x+2)$.
%\begin{center}
%\begin{bmlimage}\begin{tikzpicture}[scale=0.8]
%\tkzTabInit[nocadre, espcl=2,  color, colorV=lightgray!5, colorL=blue!15,
%colorC=blue!15]
%{$x$ /.6, $f'(x)$ /.6, Variaç. de $f$ /1.2}%
%{,$e^{-2}$, $1$, }%
%\tkzTabLine{,+,z,-,z,+}
%\tkzTabVar{-/,+/{máx.},-/{mín.},+/}
%%\tkzTabLine{,\searrow,\text{mín.},h,\text{mín.},\nearrow,}
%\end{tikzpicture}\end{bmlimage}
%\end{center}
%O máximo local está em
%$(e^{-2},f(e^{-2}))=(e^{-2},4e^{- 2})$, e o
%mínimo global em $(1,f(1))=(1,0)$.
%A {segunda derivada} de $f$ é dada por
%$f''(x)=\frac{2(\ln x+1)}{x}$.
%\begin{center}
%\begin{bmlimage}\begin{tikzpicture}[scale=0.8]
%\tkzTabInit[nocadre, espcl=2,  color, colorV=lightgray!5, colorL=blue!15,
%colorC=blue!15]
%{$x$ /.6, $f''(x)$ /.6, Conv. de $f$ /1.2}%
%{,$e^{-1}$, }%
%\tkzTabLine{,-,z,+,}
%\tkzTabLine{,\frown,,\smile,}
%\end{tikzpicture}\end{bmlimage}
%\end{center}
%Logo, $f$ é côncava em $(0,e^{-1})$, possui um ponto de inflexão em
%$(e^{-1},f(e^{-1}))=(e^{-1},e^{-1})$, e é convexa em $(e^{-1},+\infty)$.
%\begin{center}
%\begin{bmlimage}\begin{tikzpicture}[scale=1.3]
%\draw [thick, domain=0.001:2.5, samples=100] plot (\x,{\x*(ln(\x))^2});
% \draw [>=latex, ->] (0,0)--(2.5,0) node[right] {$x$};
% \draw [>=latex, ->] (0,-0.1)--(0,2);
%% \draw [dotted] (-4,1)--(4,1) node[above left] {Assíntota horiz.: $y=1$};
% \fill (1,0) circle (0.35mm);
% \draw (1,0) node[below] {$\scriptscriptstyle{(1,0)}$};
% \fill (0.367,0.367) circle (0.35mm);
% \draw[<-] (0.39,0.39)--(0.9,0.5) node[above]
%{$\scriptscriptstyle{(e^{-1},e^{-1})}$};
% \fill (0.1353,0.541) circle (0.35mm);
% \draw[<-] (0.14,0.58)--(0.9,1.5) node[above]
%{$\scriptscriptstyle{(e^{-2},4e^{-2})}$};
%\end{tikzpicture}\end{bmlimage}
%\end{center}
%Podemos também notar que $\lim_{x\to 0^+}f'(x)=+\infty$.
%\end{sol}
%\end{exo}
%
%
%\begin{exo}\label{Exo:EstudosBasicos}
%Faça um estudo completo das funções abaixo:
%\begin{multicols}{3}
%\begin{enumerate}
%\item\label{itEstBas1} $x+\frac{1}{x}$
%\item\label{itEstBas6} $x+\frac{1}{x^2}$
%\item\label{itEstBas9} $\frac{1}{x^2+1}$
%\item\label{itEstBas2} $\frac{x}{x^2-1}$
%\item\label{itEstBas3} $xe^{-x^2}$
%%\item\label{itEstBas5} $x^4-x^2$
%\item\label{itEstBas7} $\senh x$
%\item\label{itEstBas8} $\cosh x$
%\item\label{itEstBas8t} $\tanh x$
%\item\label{itEstBas13} $(x^3-1)/(x^3+1)$, 
%\item\label{itEstBas14} $\tfrac12\sen (2x)-\sen(x)$, 
%\item\label{itEstBas15} $x/\sqrt{x^2+1}$
%\item\label{itEstBas4} $\frac{(x+1)^3}{(x-1)^4}$
%\end{enumerate}
%\end{multicols}
%\begin{sol}
%OBS: Colocamos aqui somente um \emph{resumo} das soluções, na forma de um
%gráfico. Os detalhes são deixados para o leitor.
%
%\eqref{itEstBas1} 
%\begin{center}
%\begin{bmlimage}\begin{tikzpicture}[yscale=0.7]
%\draw [thick, domain=-3:-0.3, samples=100] plot (\x,{\x+1/\x});
%\draw [thick, domain=0.3:3, samples=100] plot (\x,{\x+1/\x});
% \draw [>=latex, ->] (-3,0)--(3,0) node[right] {$x$};
% \draw [>=latex, ->] (0,-3)--(0,3) node[left]{$x+\tfrac{1}{x}$};
% \fill (1,2) circle (0.45mm);
%  \draw (1,2) node[below] {$\scriptscriptstyle{(1,2)}$};
%  \fill (-1,-2) circle (0.45mm);
%  \draw (-1,-2) node[above] {$\scriptscriptstyle{(-1,-2)}$};
%\end{tikzpicture}\end{bmlimage}
%\end{center}
%
%
%\eqref{itEstBas6} 
%\begin{center}
%\begin{bmlimage}\begin{tikzpicture}[yscale=0.7]
%\draw [thick, domain=-3:-0.5, samples=100] plot
%(\x,{\x+1/(\x*\x)});
%\draw [thick, domain=0.6:3, samples=100] plot
%(\x,{\x+1/(\x*\x)})node[right]{$x+\tfrac{1}{x^2}$};
% \draw [>=latex, ->] (-3,0)--(3,0) node[right] {$x$};
% \draw [>=latex, ->] (0,-3)--(0,3);
% \fill (1.256,1.88) circle (0.45mm);
%  \draw (1.256,1.88) node[below]
%{$\scriptscriptstyle{(2^{1/3},2^{1/3}+2^{-2/3})}$};
%\draw (-1,0) node{$\shortmid$} node[above left]{$-1$};
%\end{tikzpicture}\end{bmlimage}
%\end{center}
%
%\eqref{itEstBas9} 
%\begin{center}
%\begin{bmlimage}\begin{tikzpicture}
%\draw [thick, domain=-3:3, samples=100] plot
%(\x,{1/(\x*\x+1)});
%\draw [>=latex, ->] (-3,0)--(3,0);
%\draw [>=latex, ->] (0,-0.5)--(0,1.5)node[right]{$\tfrac{1}{x^2+1}$};
%\fill (0.577,0.75) circle (0.45mm);
%\draw[<-] (0.6,0.8)--(1.3,1)
%node[right]{inflex: $(\tfrac{1}{\sqrt{3}},\tfrac34)$};
%\fill (-0.577,0.75) circle (0.45mm);
%\draw[<-] (-0.6,0.8)--(-1.3,1)
%node[left]{inflex: $(-\tfrac{1}{\sqrt{3}},\tfrac34)$};
%% {$\scriptscriptstyle{(2^{1/3},2^{1/3}+2^{-2/3})}$};
%% \draw (-1,0) node{$\shortmid$} node[above left]{$-1$};
%\end{tikzpicture}\end{bmlimage}
%\end{center}
%
%\eqref{itEstBas2} 
%\begin{center}
%\begin{bmlimage}\begin{tikzpicture}[yscale=0.7]
%\draw [>=latex, ->] (-3,0)--(3,0) node[right] {$x$};
%\draw [>=latex, ->] (0,-3)--(0,3) node[left]{$\frac{x}{x^2-1}$};
%\draw [thick, domain=-3:-1.2, samples=100] plot (\x,{\x/(\x^2-1)});
%\draw[dashed] (-1,-2.5)--(-1,2.5) node[below left]{$\scriptscriptstyle{x=-1}$};
%\draw [thick, domain=-0.8:0.8, samples=100] plot (\x,{\x/(\x^2-1)});
%\draw [thick, domain=1.2:3, samples=100] plot (\x,{\x/(\x^2-1)});
%\draw[dashed] (1,-2.5)node[right]{$\scriptscriptstyle{x=1}$}--(1,2.5) ;
%\end{tikzpicture}\end{bmlimage}
%\end{center}
%
%\eqref{itEstBas3}
%\begin{center}
%\begin{bmlimage}\begin{tikzpicture}
%\draw [>=latex, ->] (-3,0)--(3,0) node[right] {$x$};
%\draw [>=latex, ->] (0,-1)--(0,1) node[above]{$xe^{-x^2}$};
%\draw [thick, domain=-2.5:2.5, samples=100] plot (\x,{\x*exp(-\x*\x)});
%\fill (0.707,0.428) circle (0.45mm);
%  \draw[<-] (0.71,0.44)-- (0.9,1) node[right]
%{$\scriptstyle{(\tfrac{1}{\sqrt{2}},\tfrac{1}{\sqrt{2}}e^{-\tfrac12})}$};
%\fill (-0.707,-0.428) circle (0.45mm);
%  \draw[<-] (-0.71,-0.5)-- (-0.9,-1) node[left]
%{$\scriptstyle{(-\tfrac{1}{\sqrt{2}},-\tfrac{1}{\sqrt{2}}e^{-\tfrac12})}$};
%\draw[<-] (0.1,-0.1)--(0.5,-1.3) node[right]{pt. inflex. $\scriptstyle{(0,0)}$};
%\fill (1.225,0.273) circle (0.40mm);
%\fill (-1.225,-0.273) circle (0.40mm);
%\draw[<-] (1.225,0.24)--(1.5,-0.6)
%node[right]{pt. inflex.: $\scriptstyle{(\sqrt{3/2},f(\sqrt{3/2}))}$};
%\draw[<-] (-1.225,-0.24)--(-1.5,0.6)
%node[left]{pt. inflex.: $\scriptstyle{(-\sqrt{3/2},f(\sqrt{3/2}))}$};
%\end{tikzpicture}\end{bmlimage}
%\end{center}
%
%\eqref{itEstBas7}, \eqref{itEstBas8},
%\eqref{itEstBas8t}:
%\begin{center}
%\begin{bmlimage}\begin{tikzpicture}[scale=0.5]
%\draw [>=latex, ->] (0,-0.1)--(0,3);
%\pgfmathsetmacro{\a}{2};
%\draw [>=latex, ->] (-\a,0)--(\a,0);
%\draw [thick, domain=-\a:\a, samples=100] plot (\x,{(exp(\x)+exp(-\x))/2})
%node[right]{$\cosh x$};
%
%\begin{scope}[xshift=9cm, yshift=1cm]
%\draw [>=latex, ->] (0,-2)--(0,2);
%\pgfmathsetmacro{\a}{1.6};
%\draw [>=latex, ->] (-\a,0)--(\a,0);
%\draw [thick, domain=-\a:\a, samples=100] plot (\x,{(exp(\x)-exp(-\x))/2})
%node[right]{$\senh x$};
%\end{scope}
%
%\begin{scope}[xshift=18cm, yshift=1cm]
%\draw [>=latex, ->] (0,-1.5)--(0,1.5);
%\pgfmathsetmacro{\a}{3};
%\draw [>=latex, ->] (-\a,0)--(\a,0);
%\draw [thick, domain=-\a:\a, samples=100] plot
%(\x,{(exp(\x)-exp(-\x))/(exp(\x)+exp(-\x))})
%node[below right]{$\tanh x$};
%\draw[dashed] (0,1)--(\a,1) node[above]{$x=+1$};
%\draw[dashed] (0,-1)--(-\a,-1) node[below]{$x=-1$};
%\end{scope}
%
%\end{tikzpicture}\end{bmlimage}
%\end{center}
%
%\eqref{itEstBas13}
%\begin{center}
%\begin{bmlimage}\begin{tikzpicture}[yscale=0.7]
%\draw [>=latex, ->] (-3,0)--(3,0);
%\draw [>=latex, ->] (0,-3)--(0,3) node[right]{$\frac{x^3-1}{x^3+1}$};
%\draw [thick, domain=-3:-1.2, samples=100] plot (\x,{(\x^3-1)/(\x^3+1)});
%\draw [thick, domain=-0.8:3, samples=100] plot (\x,{(\x^3-1)/(\x^3+1)});
%\draw[dashed] (-1,-3)node[left]{$\scriptscriptstyle{x=-1}$}--(-1,3) ;
%\draw[dashed] (-3,1) node[below]{$\scriptscriptstyle{x=1}$}--(3,1) ;
%\fill (0,-1) circle (0.45mm);
%\fill (0.793, -0.3333) circle (0.45mm);
%\draw[<-] (0.1,-1.1)--(1,-3)node[right]{Pt. de inflexão e crítico: $(0,-1)$};
%\draw[<-] (0.8, -0.4)--(1.2,-1) node[right]{Pt. de inflexão:
%$(2^{1/3},f(2^{1/2}))$};
%\draw (1,0) node{$\shortmid$} node[above]{$1$};
%\end{tikzpicture}\end{bmlimage}
%\end{center}
%
%\eqref{itEstBas14}:
%\begin{center}
%\begin{bmlimage}\begin{tikzpicture}[yscale=0.7]
%\draw [>=latex, ->] (-3,0)--(3,0);
%\draw [>=latex, ->] (0,-3)--(0,3) node[right]{$\tfrac12\sen (2x)-\sen(x)$};
%\draw [thick, domain=0:2*pi, samples=100] plot (\x,{sin(2*\x r)-sin(\x r)});
%% \draw [thick, domain=-0.8:3, samples=100] plot (\x,{(\x^3-1)/(\x^3+1)});
%% \draw[dashed] (-1,-3)node[left]{$\scriptscriptstyle{x=-1}$}--(-1,3) ;
%% \draw[dashed] (-3,1) node[below]{$\scriptscriptstyle{x=1}$}--(3,1) ;
%% \fill (0,-1) circle (0.45mm);
%% \fill (0.793, -0.3333) circle (0.45mm);
%% \draw[<-] (0.1,-1.1)--(1,-3)node[right]{Pt. de inflexão e crítico: $(0,-1)$};
%% \draw[<-] (0.8, -0.4)--(1.2,-1) node[right]{Pt. de inflexão:
%% $(2^{1/3},f(2^{1/2}))$};
%% \draw (1,0) node{$\shortmid$} node[above]{$1$};
%\end{tikzpicture}\end{bmlimage}
%\end{center}
%
%
%\end{sol}
%\end{exo}
%
%
%\begin{exo}
%Faça um estudo completo das seguintes funções.
%\begin{multicols}{3}
%\begin{enumerate}
%\item $\ln |2-5x|$
%\item $\ln(e^{2x}-e^x+3)$
%\item $\ln(\ln x)$
%\item $\frac{\ln x}{\sqrt{x}}$
%\item $\frac{\ln x-2}{(\ln x)^2}$
%\item $\arcsen(2x^2-1)$
%\item (Gilcione, legal) $e^{-x}(x^2+2x)$.
%\item $\frac{\sqrt{x^2-1}}{x-2}$ 
%\item $(\ln x)^2+\ln x$.
%\end{enumerate}
%\end{multicols}
%(Pris dans \verb|exan_ln.pdf|, dans le dossier Coisasinternet,
%TRuc)
%\end{exo}
%
%\begin{exo} 
% Faça o esboço de uma função que tenha as 
%propriedades \ref{KKK1}-\ref{KKK4} abaixo: 
%\begin{enumerate}
%\item\label{KKK1} $f(0)=1$, $f'(0)=\half$,
%\item $x=-2$ e $x=2$ são assíntotas verticais,
%\item $y=1$ é assíntota horizontal,
%\item\label{KKK4} $f$ decresce no intervalo $[-3,-2[$, e cresce no intervalo
%$]2,3]$,
%\end{enumerate}
%\end{exo}
%
%
%\begin{exo}
%Mostre que $a^b=b^a$ n'a que deux solutions (Tikz pour l'impatient, page 61)
%\end{exo}
