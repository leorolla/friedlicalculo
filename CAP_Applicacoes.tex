
% !TeX spellcheck = pt_BR
% !TEX encoding = UTF-8 Unicode

\chapter{Aplicações}\label{CAP:Applicacoes}

\ifdefined\updateans
% Only need to run once in a lifetime, when the file ans.tex needs to be updated.
\Writetofile{ans}{\protect\section*{Capítulo \ref{CAP:Applicacoes}}}
\fi

\section{Comprimento de arco}
\index{comprimento de arco}
O procedimento usado na definição da integral de Riemann (cortar, somar, tomar
um limite) pode ser útil em outras situações.
As três próximas seções serão dedicadas ao uso de integrais para
calcular  quantidades geométricas associadas a funções.
Começaremos com o comprimento de arco.\\

Vimos acima que a integral de Riemann permite calcular a área debaixo 
do gráfico de uma função $f:[a,b]\to \bR$. 
Mostraremos agora como calcular o \emph{comprimento} 
do gráfico, via uma outra integral formada a partir da função.\\

Procederemos seguindo a mesma ideia, \emph{aproximando} o comprimento 
por uma soma. Escolhamos uma subdivisão do intervalo $[a,b]$ por intervalos $[x_i,x_{i+1}]$:

\begin{center}
\begin{bmlimage}\begin{tikzpicture}[scale=1.5]
\newcommand{\func}[1]{(#1)^2/4+1}
\pgfmathsetmacro{\a}{0};
\pgfmathsetmacro{\b}{2};
\pgfmathsetmacro{\n}{5};
\pgfmathsetmacro{\d}{(\b - \a )/\n};
\draw[>=latex, ->] (\a-0.2,0)--(\b+0.2,0);
\draw [thin, domain=\a:\b] plot (\x,{\func{\x}});

\foreach \j in {1,...,\n}{
\draw[color=blue] 
({\a+\j*\d},{\func{\a+\j*\d}})--({\a+(\j-1)*\d},{\func{\a+(\j-1)*\d}});
\fill ({\a+\j*\d},{\func{\a+\j*\d}}) circle (0.40mm);
%\pgfmathsetmacro{\x}{0};
}
\draw ({\a+2*\d},0) node[below]{\footnotesize{$x_{i}$}};
\draw[dotted] ({\a+2*\d},0)--({\a+2*\d},{\func{\a+2*\d}});

\draw ({\a+3*\d},0) node[below]{\footnotesize{$x_{i+1}$}};
\draw[dotted] ({\a+3*\d},0)--({\a+3*\d},{\func{\a+3*\d}});

\draw[dashed] (\a,0) node[below]{$a$}--(\a,{\func{\a}});
\draw[dashed] (\b,0) node[below]{$b$}--(\b,{\func{\b}});
\end{tikzpicture}\end{bmlimage}
\end{center}

Aproximaremos o comprimento do gráfico da função, 
em cada intervalo $[x_i,x_{i+1}]$, pelo comprimento do segmento que liga 
$(x_i,f(x_i))$ a $(x_{i+1},f(x_{i+1}))$, dado por 
\begin{align*}
 \sqrt{\Delta x_i^2+(f(x_{i+1})-f(x_i))^2}
&=\Delta x_i\sqrt{1+\Bigl(\frac{f(x_{i+1})-f(x_i)}{\Delta x_i}\Bigr)^2}\,,
\end{align*}
em que $\Delta x_i=x_{i+1}-x_i$. Quando $\Delta x_i\to 0$, o quociente 
$\frac{f(x_{i+1})-f(x_i)}{\Delta x_i}$ tende a $f'(x_i)$. Logo, o comprimento
do gráfico, $L$, é aproximado pela soma
$$\sum_{i=1}^n\sqrt{1+f'(x_i)^2}\Delta x_i\,,$$
que é uma soma de Riemann associada à função $\sqrt{1+f'(x)^2}$. Logo, 
tomando
um limite em que o número de intervalos cresce e o tamanho de cada intervalo
tende a zero,
obtemos uma expressão para $L$ via uma integral:
\eq{\label{eq:ComprimentoArco}\boxed{L=\int_a^b\sqrt{1+f'(x)^2}\,dx\,.}}

\begin{ex}
Calculemos o comprimento do gráfico da curva $y=\tfrac23 x^{3/2}$, entre $x=0$ e $x=1$.
Como $(\tfrac23 x^{3/2})'=\sqrt{x}$, 
$$
L=\int_0^1\sqrt{1+(\sqrt{x})^2}\,dx=\int_0^1\sqrt{1+x}\,dx=
\tfrac23(\sqrt{8}-1)\,.
$$
\end{ex}
Devido à raiz que apareceu na fórmula \eqref{eq:ComprimentoArco} (após o uso do
Teorema de Pitágoras), as integrais que aparecem para calcular comprimentos de
gráficos podem ser difíceis de calcular, isso mesmo quando a função $f$ é
simples:

\begin{ex}\label{ex:comprparabdifiss}
O comprimento da parábola $y=x^2$ entre $x=-1$ e $x=1$ é dado pela integral 
$$L=\int_{-1}^1\sqrt{1+4x^2}\,dx\,.$$
Vimos na Seção \ref{Sec:MetodoSubstitTrig} (ver o Exercício \ref{exo:comprparabola}) 
como calcular a primitiva de $\sqrt{1+4x^2}$ usando uma substituição
trigonométrica.
\end{ex}

\begin{exo}
Mostre, usando uma integral, que a circunferência de um disco de raio $R$ é $2\pi R$.
\begin{sol} Representando a metade superior do círculo de raio $R$ centrado na origem com a função $f(x)=\sqrt{R^2-x^2}$, podemos expressar o 
comprimento da circunferência como
$$
2\int_{-R}^R\sqrt{1+[(\sqrt{R^2-x^2})']^2}\,dx=2R\int_{-R}^R\frac{dx}{\sqrt{R^2-x^2}}=2R\int_{-1}^1\frac{du}{\sqrt{1-u^2}}=2\pi R\,.
$$
\end{sol}
\end{exo}

\begin{exo}
Calcule o comprimento da corda pendurada entre dois pontos $A$ e $B$, descrita pelo gráfico
 da função $f(x)=\cosh x$, entre $x=-1$ e $x=1$.  
\begin{sol}
Lembrando que $\cosh'(x)=\senh x$, que $\cosh^2 x-\senh^2x=1$, e que $\cosh x$ é par,
\begin{align*}
 L=\int_{-1}^1\sqrt{1+(\senh x)^2}\,dx=2\int_{0}^1\cosh x\,dx=2\senh (1)=e-e^{-1}\,.
\end{align*} 
\end{sol}
\end{exo}

\begin{exo}\label{Exo_finalmente}
Calcule o comprimento do gráfico da função exponencial $f(x)=e^x$, entre $x=0$
e $x=1$. (\emph{Dica}: $u=\sqrt{1+e^{2x}}$.)
\begin{sol}
O comprimento é dado por $L=\int_0^1\sqrt{1+e^{2x}}\,dx$.
Se $u=\sqrt{1+e^{2x}}$, então $dx=\frac{u}{u^2-1}du$, logo
$$L=\int_{\sqrt{2}}^{\sqrt{1+e^4}}\frac{u^2}{u^2-1}du
=\int_{\sqrt{2}}^{\sqrt{1+e^4}}1\,du+\int_{\sqrt{2}}^{\sqrt{1+e^4}}\frac{du}{
u^2-1}\,.
$$ 
Essa última integral pode ser calculada como no Exemplo \ref{Ex:unsurxdeuxmun}:
$\int\frac{du}{
u^2-1}=\tfrac{1}{2}\ln\Bigl|\frac{u-1}{u+1}\Bigr|+C$. Logo,
$$
L=\sqrt{1+e^4}-\sqrt{2}+\tfrac12\ln\Bigl[\frac{\sqrt{1+e^4}-1}{\sqrt{1+e^4}
+1}\cdot\frac{\sqrt{2}+1}{\sqrt{2}-1}\Bigr]\,.
$$
\end{sol}
\end{exo}

\section{Sólidos de revolução}\label{Sec_Solidos}
\index{sólidos de revolução}
Nesta seção usaremos a integral para calcular o volume de um 
tipo particular de região do
espaço, chamada de \emph{sólidos de revolução}. (Em Cálculo III, volumes de regiões mais
gerais serão calculados usando integral tripla.)\\

Considere uma função \emph{positiva} no intervalo $[a,b]$, $f:[a,b]\to
\bR_+$. Seja $R$ a região 
delimitada pelo gráfico de $f$, pelo eixo $x$ e pelas retas $x=a$,
$x=b$:
\begin{center}
\begin{bmlimage}\begin{tikzpicture}[scale=2]
\newcommand{\funcao}[1]{{1/((#1)^2+1)}}
\pgfmathsetmacro{\a}{0.2};
\pgfmathsetmacro{\b}{1.5};
\pgfmathsetmacro{\c}{(\a+\b)/2};
\draw[->] (0,-0.1)--(0,1.3);
\fill[areagrafico] (\a,0)--plot[domain=\a:\b](\x,\funcao{\x})
--(\b,0)--cycle;
\draw[->] (-0.1,0)--({\b+0.3},0) node[right]{$x$};
\draw[thick, domain=\a:\b] plot (\x,\funcao{\x});
\draw[dotted] (\a,0) node[below]{$a$} -- (\a,\funcao{\a});
\draw[dotted] (\b,0) node[below]{$b$} -- (\b,\funcao{\b});
\draw (\c,{(\funcao{\c})*0.5}) node{$R$};
\draw (\c,{(\funcao{\c})*1.2}) node[above]{$f(x)$};
\end{tikzpicture}\end{bmlimage}
\end{center}
Sabemos que a área de $R$ é dada pela integral de Riemann 
$$\text{área}(R)=\int_a^bf(x)\,dx\,.$$

Consideremos agora o
sólido $S$ obtido girando a região $R$ em torno do
eixo $x$, como na figura abaixo:

\begin{center}
\begin{bmlimage}\begin{tikzpicture}[scale=2]
\tdplotsetmaincoords{55}{115}

\newcommand{\funcao}[1]{(1/((#1)^2+1))}

\pgfmathsetmacro{\a}{0.2};
\pgfmathsetmacro{\b}{1.5};

\begin{scope}[tdplot_main_coords]
\pgfmathsetmacro{\nx}{20};
\pgfmathsetmacro{\Dx}{(\b-\a)/\nx};

\pgfmathsetmacro{\alphamin}{0};
\pgfmathsetmacro{\alphamax}{20};
\pgfmathsetmacro{\nalpha}{10};
\pgfmathsetmacro{\Dalpha}{(\alphamax-\alphamin)/\nalpha};


\draw[->] (0,0,0) -- (1.2,0,0);
\draw[->] (0,0,0) -- (0,{\b+0.3},0) node[anchor=north west]{$x$};
\draw[->] (0,0,0) -- (0,0,1.2);

%desenhar a flechinha de tras:
\draw[->, domain=0:330, variable=\alpha, samples=50] plot
({sin(\alpha)*(\funcao{\a})},\a,{cos(\alpha)*(\funcao{\a})}); 
%desenhar a linha pontilhada de tras:
\draw[thick, dotted] (0,\a,0)--(0,\a,{\funcao{\a}});

\fill[areagrafico]
(0,\a,0)--plot[domain=\a:\b](0,\x,{\funcao{\x}})--(0,\b,0)--cycle;
%desenhar o grafico de f em [a,b]
\draw[thick, domain=\a:\b, variable=\t] plot
(0,\t,{\funcao{\t}}); 
%desenhar a flechinha de frente:
\draw[->, domain=0:320, variable=\alpha] plot
({sin(\alpha)*\funcao{\b}},\b,{cos(\alpha)*\funcao{\b}}); 
%desenhar a linha pontilhada de frente:
\draw[thick,dotted] (0,\b,0)--(0,\b,{\funcao{\b}});
\end{scope}

\begin{scope}[xshift=2.5cm,tdplot_main_coords]
\pgfmathsetmacro{\nx}{15};
\pgfmathsetmacro{\Dx}{(\b-\a)/\nx};
\pgfmathsetmacro{\alphamin}{-40};
\pgfmathsetmacro{\alphamax}{150};
\pgfmathsetmacro{\nalpha}{35};
\pgfmathsetmacro{\Dalpha}{(\alphamax-\alphamin)/\nalpha};

\draw[->] (0,0,0) -- (1.2,0,0);
\draw[->] (0,0,0) -- (0,{\b+0.3},0) node[anchor=north west]{$x$};
\draw[->] (0,0,0) -- (0,0,1.2);
%desenhar o grande circulo de tras:
\draw[domain=0:360, variable=\alpha] plot
({sin(\alpha)*\funcao{\a}},\a,{cos(\alpha)*\funcao{\a}}); 
%desenhar a linha pontilhada de tras:
\draw[thick, dotted] (0,\a,0)--(0,\a,{\funcao{\a}});
%desenhar o grande circulo de frente:
\draw[domain=0:360, variable=\alpha] plot
({sin(\alpha)*\funcao{\b}},\b,{cos(\alpha)*\funcao{\b}}); 

\fill[areagrafico]
(0,\a,0)--plot[domain=\a:\b](0,\x,{\funcao{\x}})--(0,\b,0)--cycle;
\foreach \i in {0,...,\nalpha} {
\draw[thick, color=gray, domain=\a:\b, variable=\t] plot
({sin(\alphamin+\Dalpha*\i)*\funcao{\t}},\t,{cos(\alphamin+\Dalpha*\i)*\funcao{\t}}); 
}
\foreach \i in {0,...,\nx} {
\pgfmathsetmacro{\point}{\a+(\i*\Dx)};
\draw[thick, color=gray, domain=\alphamin:\alphamax, variable=\alpha] plot
({sin(\alpha)*\funcao{\point}},\point,{cos(\alpha)*\funcao{\point}}); 
 }
%desenhar o grafico de f em [a,b]
\draw[thick, thick, domain=\a:\b, variable=\t] plot
(0,\t,{\funcao{\t}}); 
%desenhar a linha pontilhada de frente:
\draw[thick,dotted] (0,\b,0)--(0,\b,{\funcao{\b}});
\end{scope}

\begin{scope}[xshift=5cm,tdplot_main_coords]
\pgfmathsetmacro{\nx}{20};
\pgfmathsetmacro{\Dx}{(\b-\a)/\nx};
\pgfmathsetmacro{\alphamin}{-40};
\pgfmathsetmacro{\alphamax}{150};
\pgfmathsetmacro{\nalpha}{40};
\pgfmathsetmacro{\Dalpha}{(\alphamax-\alphamin)/\nalpha};


\foreach \i in {0,...,\nalpha} {
\draw[thick, color=gray, domain=\a:\b, variable=\t] plot
({sin(\alphamin+\Dalpha*\i)*\funcao{\t}},\t,{cos(\alphamin+\Dalpha*\i)*\funcao{\t}}); 
}
\foreach \i in {0,...,\nx} {
\pgfmathsetmacro{\point}{\a+(\i*\Dx)};
\draw[thick, color=gray, domain=\alphamin:\alphamax, variable=\alpha] plot
({sin(\alpha)*\funcao{\point}},\point,{cos(\alpha)*\funcao{\point}}); 
 }

%Encher o disco de frente:
\filldraw[color=gray] plot[domain=0:360, variable=\alpha] 
({sin(\alpha)*\funcao{\b}},\b,{cos(\alpha)*\funcao{\b}}); 

\draw (0,\b,{\funcao{\b}+0.5}) node[left]{$S$};
\end{scope}
\end{tikzpicture}\end{bmlimage}
\end{center}

Sólidos que podem ser gerados dessa maneira, girando uma região em torno de
um eixo, são chamados de
\grasA{sólidos de revolução}. 
Veremos situações em que a região não precisa ser delimitada pelo 
gráfico de uma função, e que o eixo não precisa ser o eixo $x$.	

\begin{exo}
Quais dos seguintes 
corpos são sólidos de revolução? (Quando for o caso, dê a região e
o eixo)
\begin{enumerate}
\item\label{itexsolrev1} A esfera de raio $r$.
\item\label{itexsolrev2} O cilindro com base circular de raio $r$, e de altura $h$.
\item\label{itexsolrev3} O cubo de lado $L$.
\item\label{itexsolrev4} O cone de base circular de raio $r$ e de altura $h$.
\item\label{itexsolrev5} O toro de raios $0<r<R$.
\end{enumerate}
\begin{sol}
\eqref{itexsolrev1} A esfera pode ser obtida girando o semi-disco,
delimitado pelo gráfico da função
$f(x)=\sqrt{r^2-x^2}$, $x\in [-r,r]$, em torno do eixo $x$.
\eqref{itexsolrev2} O cilíndro pode ser obtido girando o gráfico da função
constante $f(x)=r$, no intervalo $[0,h]$.
\eqref{itexsolrev3} O cubo não é um sólido de revolução.
\eqref{itexsolrev4} O cone pode ser obtido girando o gráfico da função
$f(x)=\frac{r}{h}x$ (ou $f(x)=r-\frac{r}{h}x$), no intervalo $[0,h]$. 
\end{sol}
\end{exo}


Nesta seção desenvolveremos 
métodos para \emph{calcular o volume $V(S)$ de um sólido de
revolução $S$}. Antes de começar, consideremos um caso elementar, que será também usado para o caso geral.

\begin{ex}\label{Ex_vol_cilindro}
Suponha que $f$ é constante em $[a,b]$, isto é: $f(x)=r>0$ para todo $x\in [a,b]$:

%
\begin{center}
\begin{bmlimage}\begin{tikzpicture}[scale=2]
\tdplotsetmaincoords{55}{115}
\pgfmathsetmacro{\r}{0.6};
\newcommand{\funcao}[1]{(\r)}
\pgfmathsetmacro{\a}{0.2};
\pgfmathsetmacro{\b}{1.5};
\pgfmathsetmacro{\c}{(\a+\b)/2};
\draw[->] (0,-0.1)--(0,\r+0.3);
\fill[areagrafico]
(\a,0)--plot[domain=\a:\b](\x,{\funcao{\x}})--(\b,0)--cycle;
\draw[->] (-0.1,0)--({\b+0.3},0);
\draw[thick, domain=\a:\b] plot (\x,{\funcao{\x}});
\draw[dotted] (\a,0) node[below]{$a$} --
(\a,{\funcao{\a}});
\draw[dotted] (\b,0) node[below]{$b$} --
(\b,{\funcao{\b}});
%\draw (\c,{0.5*\funcao{\c}}) node{$R$};
\draw[dashed] (-0.1,\r)--(\b,\r);
\draw (-0.1,\r) node[left]{$r$};

\begin{scope}[xshift=3.5cm, yshift=0.5cm, tdplot_main_coords]
\pgfmathsetmacro{\nx}{20};
\pgfmathsetmacro{\Dx}{(\b-\a)/\nx};
\pgfmathsetmacro{\alphamin}{-40};
\pgfmathsetmacro{\alphamax}{150};
\pgfmathsetmacro{\nalpha}{40};
\pgfmathsetmacro{\Dalpha}{(\alphamax-\alphamin)/\nalpha};

\draw[->] (0,0,0) -- (1.2,0,0);
\draw[->] (0,0,0) -- (0,{\b+0.3},0) node[anchor=north west]{$x$};
\draw[->] (0,0,0) -- (0,0,1.2);
%desenhar o grande circulo de tras:
\draw[domain=0:360, variable=\alpha] plot
({sin(\alpha)*\funcao{\a}},{\a},{cos(\alpha)*\funcao{\a}}); 
%desenhar a linha pontilhada de tras:
\draw[thick, dotted] (0,\a,0)--(0,\a,{\funcao{\a}});
%desenhar o grande circulo de frente:
\draw[domain=0:360, variable=\alpha] plot
({sin(\alpha)*\funcao{\b}},{\b},{cos(\alpha)*\funcao{\b}}); 

\fill[areagrafico]
(0,\a,0)--plot[domain=\a:\b](0,\x,{\funcao{\x}})--(0,\b,0)--cycle;
\foreach \i in {0,...,\nalpha} {
\draw[thick, color=gray, domain=\a:\b, variable=\t] plot
({sin(\alphamin+\Dalpha*\i)*\funcao{\t}},\t,{cos(\alphamin+\Dalpha*\i)*\funcao{\t}}); 
}
\foreach \i in {0,...,\nx} {
\pgfmathsetmacro{\point}{\a+(\i*\Dx)};
\draw[thick, color=gray, domain=\alphamin:\alphamax, variable=\alpha] plot
({sin(\alpha)*\funcao{\point}},\point,{cos(\alpha)*\funcao{\point}}); 
 }
%desenhar o grafico de f em [a,b]
\draw[thick, thick, domain=\a:\b, variable=\t] plot
(0,\t,{\funcao{\t}}); 
%desenhar a linha pontilhada de frente:
\draw[thick,dotted] (0,\b,0)--(0,\b,{\funcao{\b}});
\draw[thick, ->] (0,\b,0)--({sin(120)*\funcao{\b}},\b,{cos(120)*\funcao{\b}})
node[midway, right]{$r$};
\end{scope}
\end{tikzpicture}\end{bmlimage}
\end{center}
%
Neste caso, o sólido gerado $S$ é um cilindro (deitado). 
\index{cilindro}
A sua base é circular de
raio $r$, e a sua altura é $b-a$. Pela fórmula bem conhecida do volume
de um cilíndro,
\begin{equation}
V(S)=\text{área da base }\times\text{ altura}=\pi r^2(b-a)\,.
\end{equation}
\end{ex}

Queremos agora calcular $V(S)$ 
para um sólido de revolução qualquer.\\

O procedimento será o mesmo que levou à propria definição
\index{integral de Riemann}da integral de Riemann:
\emph{aproximaremos $S$ por sólidos mais elementares}. Usaremos dois tipos
de sólidos elementares: cilíndros e cascas. 


\subsection{Aproximação por cilindros}
\index{aproximação!por cilindros}
Voltemos para 
o sólido de revolução da seção anterior. Um jeito de decompor o sólido $S$ 
é de aproximá-lo por uma união de fatias verticais, centradas no eixo
$x$:

\begin{center}
\begin{bmlimage}\begin{tikzpicture}[scale=2]
\tdplotsetmaincoords{55}{115}

\newcommand{\funcao}[1]{( 1/((#1)^2+1) )}
\pgfmathsetmacro{\a}{0.2};
\pgfmathsetmacro{\b}{1.5};
\pgfmathsetmacro{\c}{(\a+\b)/2};

%nombre de tranches:
\pgfmathsetmacro{\n}{10};

\draw[->] (0,-0.1)--(0,1.3);
\pgfmathsetmacro{\dx}{(\b-\a)/\n};

\foreach \i in {1,...,\n} {
\filldraw[corretangulos] ({\a+(\i-1)*\dx},0)
rectangle ({\a+\i*\dx},{\funcao{\a+\i*\dx}});
\draw ({\a+(\i-1)*\dx},0)
rectangle ({\a+\i*\dx},{\funcao{\a+\i*\dx}});
\fill ({\a+\i*\dx},{\funcao{\a+\i*\dx}}) circle (0.2mm);
}

\draw[->] (-0.1,0)--({\b+0.3},0);
\draw[thick, domain=\a:\b] plot (\x,{\funcao{\x}});
\draw[dotted] (\a,0) node[below]{$a$} --
(\a,{\funcao{\a}});
\draw[dotted] (\b,0) node[below]{$b$} --
(\b,{\funcao{\b}});

\begin{scope}[xshift=3cm, yshift=0.5cm, tdplot_main_coords]

\pgfmathsetmacro{\nx}{2};
\pgfmathsetmacro{\Dx}{\n/\nx};
\pgfmathsetmacro{\alphamin}{-40};
\pgfmathsetmacro{\alphamax}{150};
\pgfmathsetmacro{\nalpha}{40};
\pgfmathsetmacro{\Dalpha}{(\alphamax-\alphamin)/\nalpha};

%dessiner les axes x et z
\draw[->] (0,0,0) -- (1.2,0,0);
\draw[->] (0,0,0) -- (0,0,1.2);

%dessiner le traitille de derriere:
\draw[dotted] (0,\a,0)--(0,\a,{\funcao{\a}});


%dessiner les tranches:
\foreach \i in {1,...,\n} {
\pgfmathsetmacro{\ap}{(\a+(\i-1)*\dx)};
\pgfmathsetmacro{\bp}{(\a+(\i*\dx))};
\pgfmathsetmacro{\valf}{ \funcao{\bp} };
%remplir le premier cercle (derriere) de la tranche:
\filldraw[corretangulos, domain=0:360, variable=\alpha] plot
({sin(\alpha)*\valf},{\ap},{cos(\alpha)*\valf}); 
%prononcer un peu le bord:
\draw[color=gray, domain=0:360, variable=\alpha] plot
({sin(\alpha)*\valf},{\ap},{cos(\alpha)*\valf}); 
%dessiner les petits traits de la tranche:
\foreach \j in {0,...,\nalpha} {
\draw[color=gray, domain=\ap:\bp, variable=\t] plot
({sin(\alphamin+\Dalpha*\j)*\valf},\t,{cos(\alphamin+\Dalpha*\j)*\valf}); 
%remplir le disque (devant) de la tranche:
\filldraw[corretangulos, domain=0:360, variable=\alpha] plot
({sin(\alpha)*\valf},\bp,{cos(\alpha)*\valf}); 
%prononcer un peu le bord de ce disque
\draw[color=gray, domain=0:360, variable=\alpha] plot
({sin(\alpha)*\valf},\bp,{cos(\alpha)*\valf}); 
\fill (0,\bp,{\funcao{\bp}}) circle (0.2mm);
}
}

%dessiner le graphe de la fonction
\draw[thick, domain=\a:\b, variable=\t] plot
(0,\t,{\funcao{\t}});
%dessiner le petit traitille:
\draw[dotted] (0,\b,0)--(0,\b,{\funcao{\b}});

\draw[->] (0,\b,0) -- (0,{\b+0.3},0) node[right]{$x$};

\end{scope}
\end{tikzpicture}\end{bmlimage}
\end{center}
%
Cada fatia é obtida girando um retângulo cujo tamanho é determinado pela função $f$. Para
ser mais preciso, escolhemos pontos no intervalo $[a,b]$,
$x_0\equiv a<x_1<x_2<\dots<x_{n-1}<x_n\equiv b$, e a cada intervalo
$[x_{i-1},x_{i}]$ associamos o retângulo cuja base tem tamanho
$(x_i-x_{i-1})$ e cuja altura é de $f(x_i)$. Ao
girar em torno do eixo $x$, cada um desses retângulos gera uma
fatia cilíndrica $F_i$, como no Exemplo
\ref{Ex_vol_cilindro}:

\begin{center}
\begin{bmlimage}\begin{tikzpicture}[scale=2]
\tdplotsetmaincoords{55}{115}
\newcommand{\funcao}[1]{( 1/((#1)^2+1) )}
\pgfmathsetmacro{\a}{0.2};
\pgfmathsetmacro{\b}{1.5};
\pgfmathsetmacro{\c}{(\a+\b)/2};

%nombre de tranches:
\pgfmathsetmacro{\n}{10};

\draw[->] (0,-0.1)--(0,1.3);
%\fill[areagrafico] (\a,0)--plot[domain=\a:\b](\x,{\funcao{}})--(\b,0)--cycle;

\pgfmathsetmacro{\dx}{(\b-\a)/\n};

\foreach \i in {\n/2} {
%\draw ({\a + \i*\dx},0)--(1,1);
\filldraw[corretangulos] ({\a+(\i-1)*\dx},0)
rectangle ({\a+\i*\dx},{\funcao{\a+\i*\dx}});
\draw ({\a+(\i-1)*\dx},0)
rectangle ({\a+\i*\dx},{\funcao{\a+\i*\dx}});
\fill ({\a+\i*\dx},{\funcao{\a+\i*\dx}}) circle (0.2mm);
\draw[<-] ({\a+(\i-1)*\dx},-0.02)--
({\a+(\i-1)*\dx-0.2},-0.25) node[below]{$x_{i-1}$};
\draw[<-] ({\a+(\i)*\dx},-0.02)--({\a+(\i)*\dx+0.2},-0.25) node[below]{$x_{i}$}; 
}

\draw[->] (-0.1,0)--({\b+0.3},0);
\draw[thick, domain=\a:\b] plot (\x,{\funcao{\x}});
\draw[dotted] ({\a},0) node[below]{$a$} --
(\a,{\funcao{\a}});
\draw[dotted] (\b,0) node[below]{$b$} -- (\b,{\funcao{\b}});

\begin{scope}[xshift=3.5cm, yshift=0cm, tdplot_main_coords]

\pgfmathsetmacro{\nx}{2};
\pgfmathsetmacro{\Dx}{\n/\nx};
\pgfmathsetmacro{\alphamin}{-40};
\pgfmathsetmacro{\alphamax}{150};
\pgfmathsetmacro{\nalpha}{40};
\pgfmathsetmacro{\Dalpha}{(\alphamax-\alphamin)/\nalpha};

%dessiner les axes x et z
\draw[->] (0,0,0) -- (1.2,0,0);
\draw[->] (0,0,0) -- (0,0,1.2);

%dessiner le traitille de derriere:
\draw[dotted] (0,\a,0)--(0,\a,{\funcao{\a}});

\foreach \i in {\n/2} {
\pgfmathsetmacro{\ap}{(\a+(\i-1)*\dx)};
\draw (0,0,0)--(0,{\ap},0);
\pgfmathsetmacro{\bp}{\a+(\i*\dx)};
\pgfmathsetmacro{\valf}{\funcao{\bp}};
%remplir le premier cercle (derriere) de la tranche:

\filldraw[corretangulos, domain=0:360, variable=\alpha] plot
({sin(\alpha)*\valf},{\ap},{cos(\alpha)*\valf}); 

%prononcer un peu le bord:
\draw[color=gray, domain=0:360, variable=\alpha] plot
({sin(\alpha)*\valf},{\ap},{cos(\alpha)*\valf}); 

%dessiner les petits traits de la tranche:
\foreach \j in {0,...,\nalpha} {
\draw[color=gray, domain=\ap:\bp, variable=\t] plot
({sin(\alphamin+\Dalpha*\j)*\valf},{\t},{cos(\alphamin+\Dalpha*\j)*\valf}); 
%remplir le disque (devant) de la tranche:
\filldraw[corretangulos, domain=0:360, variable=\alpha] plot
({sin(\alpha)*\valf},{\bp},{cos(\alpha)*\valf}); 
%prononcer un peu le bord de ce disque
\draw[color=gray, domain=0:360, variable=\alpha] plot
({sin(\alpha)*\valf},{\bp},{cos(\alpha)*\valf}); 
\fill (0,\bp,{\funcao{\bp}}) circle (0.2mm);
}

%terminer le bout de l'axe des x:
\draw[thin] (0,{\bp},0)--(0,{\b},0);
\draw[thin] ({1.3*\valf},{\bp},0) node[left]{$F_i$};
}

%dessiner le graphe de la fonction
\draw[thick, domain=\a:\b, variable=\t] plot
(0,\t,{\funcao{\t}});
%dessiner le petit traitille:
\draw[dotted] (0,{\b},0)--(0,\b,{\funcao{\b}});
%terminer l'axe des x:
\draw[->] (0,{\b},0) -- (0,{\b+0.3},0) node[right]{$x$};

\end{scope}

\end{tikzpicture}\end{bmlimage}
\end{center}


Mas, como a fatia $F_i$ é um cilindro deitado de 
raio $f(x_i)$ e de altura $\Delta x_i=x_i-x_{i-1}$, o seu
volume é dado por $V(F_i)=\pi f(x_i)^2\Delta x_i$.
Logo, o volume do sólido $S$ pode ser aproximado 
pela soma dos volumes das fatias, que é 
uma soma de Riemann:
$$
\sum_{j=1}^nV(F_i)=\sum_{i=1}^n\pi f(x_i)^2\Delta x_i\,.
$$
Quando o número de retângulos
$n\to \infty$ e que todos os $\Delta x_i\to 0$, esta soma
converge (quando $f(x)^2$ pe contínua, por exemplo) 
para a uma integral de
Riemann que permite (em princípio) calcular o volume exato do sólido $S$:
\begin{equation}\label{eq_formula_volume}
\boxed{V(S)=\int_a^b\pi f(x)^2\,dx\,.}
\end{equation}

\begin{ex}\label{ex_giraseno}
Seja $R$ a região delimitada pela curva $y=\sen x$, pelo eixo $x$, e pelas duas retas
verticais $x=0$ e $x=\pi$.
Calculemos o volume do sólido $S$ obtido girando $R$ em torno do eixo $x$:

\begin{center}
\begin{bmlimage}\begin{tikzpicture}[scale=1.7]
\tdplotsetmaincoords{55}{115}

%\newcommand{\funcao}[1]{{(1-(#1)^2)}}
\newcommand{\funcao}[1]{(sin(#1 r))}

\pgfmathsetmacro{\a}{0};
\pgfmathsetmacro{\b}{pi};

\begin{scope}
\fill[areagrafico]
(\a,0)--plot[domain=\a:\b](\x,{\funcao{\x}})--(\b,0)--cycle;
\draw[thick] (\a,0) plot[domain=\a:\b](\x,{\funcao{\x}});
\draw[->] (\a-0.2,0)--(\b+0.2,0);
\draw[->] (0,-0.2)--(0,1.2);
\draw (pi,0) node[below]{$\pi$};
\end{scope}

\begin{scope}[xshift=5cm, yshift=0.7cm, tdplot_main_coords]
\pgfmathsetmacro{\nx}{20};
\pgfmathsetmacro{\Dx}{(\b-\a)/\nx};

\draw[->] (0,0,0) -- (1.2,0,0);
\draw[->] (0,0,0) -- (0,{\b+0.3},0) node[anchor=north west]{$x$};
\draw[->] (0,0,0) -- (0,0,1.2);
\fill[areagrafico]
(0,\a,0)--plot[domain=\a:\b](0,\x,{\funcao{\x}})--(0,\b,0)--cycle;
%desenhar a flechinha de frente:
\draw[->, domain=0:320, variable=\alpha] plot
({sin(\alpha)*\funcao{\b}},\b,{cos(\alpha)*\funcao{\b}}); 
%desenhar a linha pontilhada de frente:
\draw[thick,dotted] (0,\b,0)--(0,\b,{\funcao{\b}});
\draw (0,pi,0) node[above right]{$\pi$};

\pgfmathsetmacro{\alphamin}{-50};
\pgfmathsetmacro{\alphamax}{150};
\pgfmathsetmacro{\nalpha}{50};
\pgfmathsetmacro{\Dalpha}{(\alphamax-\alphamin)/\nalpha};

%Encher o grande disco de tras:
% \filldraw[color=gray!50] plot[domain=0:360, variable=\alpha] 
% ({sin(\alpha)*\funcao{\a}},\a,{cos(\alpha)*\funcao{\a}}); 

\foreach \i in {0,...,\nalpha} {
\draw[thick, color=gray, domain=\a:\b, variable=\t] plot
({sin(\alphamin+\Dalpha*\i)*\funcao{\t}},\t,{cos(\alphamin+\Dalpha*\i)*\funcao{\t}}); 
}
\foreach \i in {0,...,\nx} {
\pgfmathsetmacro{\point}{\a+(\i*\Dx)};
\draw[thick, color=gray, domain=\alphamin:\alphamax, variable=\alpha] plot
({sin(\alpha)*\funcao{\point}},\point,{cos(\alpha)*\funcao{\point}}); 
}

\draw[thick, domain=\a:\b, variable=\t] plot
(0,\t,{\funcao{\t}}); 
\end{scope}
\end{tikzpicture}\end{bmlimage}
\end{center}
%AQUI


Pela fórmula 
\eqref{eq_formula_volume}, o volume deste sólido é dado pela integral 
$$V=\int_{0}^\pi\pi(\sen x)^2\,dx=\pi \Bigl\{
\frac{x}{2}-\frac{\sen (2x)}{4}
\Bigr\}\Big|_{0}^\pi=\tfrac12 \pi^2\,.
$$

\end{ex}


O método permite calcular volumes clássicos da geometria.

\begin{ex}\label{ex_giradisco}
Seja $r>0$ fixo e $R$ a região delimitada pela semi-circunferência 
$y=\sqrt{r^2-x^2}$, entre $x=-r$ e $x=+r$, e pelo eixo $x$.
O sólido $S$ obtido girando $R$ em torno do eixo $x$ é uma esfera
\index{esfera} de raio $r$ centrada na origem:

\begin{center}
\begin{bmlimage}\begin{tikzpicture}[scale=1.7]
\tdplotsetmaincoords{55}{105}

%\newcommand{\funcao}[1]{{(1-(#1)^2)}}
\pgfmathsetmacro{\r}{1};
\newcommand{\funcao}[1]{(sqrt((\r)^2-(#1)^2))}
\pgfmathsetmacro{\a}{-\r};
\pgfmathsetmacro{\b}{\r};

\begin{scope}
\fill[areagrafico]
(\a,0)--plot[domain=\a:\b, samples=200](\x,{\funcao{\x}})--(\b,0)--cycle;
\draw[thick] (\a,0) plot[domain=\a:\b, samples=220](\x,{\funcao{\x}});
\draw[->] (\a-0.2,0)--(\b+0.2,0);
\draw[->] (0,-0.2)--(0,1.2);
\draw (\a,0) node[below]{$-r$};
\draw (\b,0) node[below]{$+r$};
\end{scope}

%\iffalse
\begin{scope}[xshift=4cm, yshift=0.2cm, tdplot_main_coords]
\pgfmathsetmacro{\nx}{40};
\pgfmathsetmacro{\Dx}{(\b-\a)/\nx};

\draw (0,0,0) -- (\r,0,0);
\draw[->] (0,0,0) -- (0,{\b+0.3},0) node[anchor=north west]{$x$};

% Dessiner la region R:
\fill[areagrafico]
(0,\a,0)--plot[domain=\a:\b](0,\x,{\funcao{\x}})--(0,\b,0)--cycle;

\pgfmathsetmacro{\alphamin}{-50};
\pgfmathsetmacro{\alphamax}{150};
\pgfmathsetmacro{\nalpha}{50};
\pgfmathsetmacro{\Dalpha}{(\alphamax-\alphamin)/\nalpha};

% Dessiner les cercles inclines:
\foreach \i in {0,...,\nalpha} {
\draw[thick, color=gray, domain=\a:\b, variable=\t] plot
({sin(\alphamin+\Dalpha*\i)*\funcao{\t}},\t,{cos(\alphamin+\Dalpha*\i)*\funcao{\t}}); 
}
%Dessiner les tranches verticales:
\foreach \i in {0,...,\nx} {
\pgfmathsetmacro{\point}{\a+(\i*\Dx)};
\draw[thick, color=gray, domain=\alphamin:\alphamax, variable=\alpha,
samples=220] plot
({sin(\alpha)*\funcao{\point}},\point,{cos(\alpha)*\funcao{\point}}); 
}

\draw[thick, domain=\a:\b, variable=\t, samples=220] plot
(0,\t,{\funcao{\t}}); 

% Le petit bout de l'axe z:
\draw[->] (0,0,\r) -- (0,0,\r+0.5);
% Le petit bout de l'axe x:
\draw[->] (\r,0,0) -- (\r+0.8,0,0);
\end{scope}
%\fi
\end{tikzpicture}\end{bmlimage}
\end{center}

Pela fórmula 
\eqref{eq_formula_volume}, o volume da esfera é dado pela integral 
\begin{align*}
V(\text)&=\int_{-r}^{+r}\pi\bigl(\sqrt{r^2-x^2}\bigr)^2\,dx\\
&=\pi\int_{-r}^{+r}(r^2-x^2)\,dx\\
&=\pi\Bigl\{r^2x-\tfrac{x^3}{3}\Bigr\}\Big|_{-r}^{+r}\\
&=\frac{4}{3}\pi r^3\,...
\end{align*}

\end{ex}


\begin{exo}
Um vaso \'e obtido rodando a curva $y=f(x)$ em torno do eixo
$x$, onde
$$
f(x)=
\begin{cases}
-x+3&\text{ se }0\leq x\leq 2,\\
x-1&\text{ se }2<x\leq 3\,.
\end{cases}
$$
Esboce o vaso obtido, em três dimensões, e calcule o seu volume.
\begin{sol} $11\pi$
 \end{sol}
\end{exo}

O importante, nesta seção, é de não tentar
\emph{decorar fórmulas}, e sim entender como montar uma nova
fórmula em cada situação. Vejamos como, no seguinte exemplo.

\begin{ex}\label{ex_gira_parabola}
Considere a região $R$ do primeiro quadrante, delimitada pelo gráfico
da função $f(x)=1-x^2$. Considere os sólidos $S_1$ e $S_2$, obtidos
rodando $R$ em torno, respectivamente, do eixo $x$ e $y$:

\begin{center}
\begin{bmlimage}\begin{tikzpicture}[scale=1.7]
\tdplotsetmaincoords{55}{115}

\newcommand{\funcao}[1]{(1-(#1)^2)}
%\newcommand{\funcao}[1]{{sin(#1 r)}}
\pgfmathsetmacro{\a}{0};
\pgfmathsetmacro{\b}{1};

\begin{scope}
\fill[areagrafico]
(\a,0)--plot[domain=\a:\b](\x,{\funcao{\x}})--(\b,0)--cycle;
\draw[thick] (\a,0) plot[domain=\a:\b](\x,{\funcao{\x}});
\draw[->] (\a-0.2,0)--(\b+0.2,0);
\draw[->] (0,-0.2)--(0,1.2);
\draw (\b,0) node[below]{$1$};
\end{scope}

\begin{scope}[xshift=3cm, yshift=0.6cm, tdplot_main_coords]
\pgfmathsetmacro{\nx}{20};
\pgfmathsetmacro{\Dx}{(\b-\a)/\nx};

%Encher o grande disco de tras:
 \filldraw[color=gray!50] plot[domain=0:360, variable=\alpha] 
 ({sin(\alpha)*\funcao{\a}},\a,{cos(\alpha)*\funcao{\a}}); 

\draw[->] (0,0,0) -- (1.2,0,0);
\draw[->] (0,0,0) -- (0,{\b+0.3},0) node[anchor=north west]{$x$};
\draw[->] (0,0,0) -- (0,0,1.2);

\fill[areagrafico]
(0,\a,0)--plot[domain=\a:\b](0,\x,{\funcao{\x}})--(0,\b,0)--cycle;

%desenhar a flechinha de frente:
\draw[->, domain=0:320, variable=\alpha] plot
({sin(\alpha)*\funcao{\b}},\b,{cos(\alpha)*\funcao{\b}}); 

%desenhar a linha pontilhada de frente:
\draw[thick,dotted] (0,\b,0)--(0,\b,{\funcao{\b}});
\draw (0,1,0) node[below right]{$1$};

\pgfmathsetmacro{\alphamin}{-50};
\pgfmathsetmacro{\alphamax}{150};
\pgfmathsetmacro{\nalpha}{50};
\pgfmathsetmacro{\Dalpha}{(\alphamax-\alphamin)/\nalpha};


\foreach \i in {0,...,\nalpha} {
\draw[thick, color=gray, domain=\a:\b, variable=\t] plot
({sin(\alphamin+\Dalpha*\i)*\funcao{\t}},\t,{cos(\alphamin+\Dalpha*\i)*\funcao{\t}}); 
}
\foreach \i in {0,...,\nx} {
\pgfmathsetmacro{\point}{\a+(\i*\Dx)};
\draw[thick, color=gray, domain=\alphamin:\alphamax, variable=\alpha] plot
({sin(\alpha)*\funcao{\point}},\point,{cos(\alpha)*\funcao{\point}}); 
}

\draw[thick, domain=\a:\b, variable=\t] plot
(0,\t,{\funcao{\t}}); 
\draw (0.9,0,1.2) node{$S_1$};
\end{scope}

\begin{scope}[xshift=6.3cm, yshift=0.5cm, tdplot_main_coords]
\pgfmathsetmacro{\nx}{20};
\pgfmathsetmacro{\Dx}{(\b-\a)/\nx};

%Encher o grande disco debaixo:
 \filldraw[color=gray!50] plot[domain=0:360, variable=\alpha] 
 ({sin(\alpha)*\funcao{\a}},{cos(\alpha)*\funcao{\a}},0); 

\draw[->] (0,0,0) -- (1.2,0,0);
\draw[->] (0,0,0) -- (0,{\b+0.3},0) node[anchor=north west]{$x$};
\draw[->] (0,0,0) -- (0,0,1.2);

\fill[areagrafico]
(0,\a,0)--plot[domain=\a:\b](0,\x,{\funcao{\x}})--(0,\b,0)--cycle;

\pgfmathsetmacro{\alphamin}{0};
\pgfmathsetmacro{\alphamax}{360};
\pgfmathsetmacro{\nalpha}{50};
\pgfmathsetmacro{\Dalpha}{(\alphamax-\alphamin)/\nalpha};

\foreach \i in {0,...,\nalpha} {
\draw[thick, color=gray, domain=\a:\b, variable=\t] plot
({sin(\alphamin+\Dalpha*\i)*\funcao{\t}},{cos(\alphamin+\Dalpha*\i)*\funcao{\t}},\t); 
}
\foreach \i in {0,...,\nx} {
\pgfmathsetmacro{\point}{\a+(\i*\Dx)};
\draw[thick, color=gray, domain=\alphamin:\alphamax, variable=\alpha] plot
({sin(\alpha)*\funcao{\point}},{cos(\alpha)*\funcao{\point}},\point); 
}

\draw[thick, domain=\a:\b, variable=\t] plot
(0,\t,{\funcao{\t}}); 
\draw (0,1.2,1.1) node{$S_2$};
\end{scope}
\end{tikzpicture}\end{bmlimage}
\end{center}

Calculemos, para começar, o volume do sólido $S_1$.
O raciocíno já descrito acima permite usar a fórmula:
$$
V(S_1)=\int_0^1 \pi
(1-x^2)^2\,dx=\pi\int_0^1\{1-2x^2+x^4\}\,dx=\tfrac{8\pi}{15}\,.
$$
Consideremos agora o sólido $S_2$.
Por ser um sólido de revolução em torno do eixo $y$, a aproximação mais
natural é de usar fatias horizontais, centradas no eixo $y$, 
como na figura a seguir:
\begin{center}
\begin{bmlimage}\begin{tikzpicture}[scale=2]
\tdplotsetmaincoords{55}{115}
\newcommand{\funcao}[1]{(sqrt(1-(#1)))}
\pgfmathsetmacro{\a}{0};
\pgfmathsetmacro{\b}{1};
\pgfmathsetmacro{\c}{(\a+\b)/2};

%nombre de tranches:
\pgfmathsetmacro{\n}{10};

%\fill[areagrafico]
%(\a,0)--plot[domain=\a:\b](\x,{\funcao{\x}})--(\b,0)--cycle;


\pgfmathsetmacro{\dx}{(\b-\a)/\n};


%%desenhar os retangulos no plano
\foreach \i in {1,...,\n} {
%\draw ({\a + \i*\dx},0)--(1,1);
\filldraw[corretangulos] (0,{\a+(\i-1)*\dx})
rectangle ({\funcao{\a+\i*\dx}},{\a+\i*\dx});
\draw (0,{\a+(\i-1)*\dx})
rectangle ({\funcao{\a+\i*\dx}},{\a+\i*\dx});
\fill ({\funcao{\a+\i*\dx}},{\a+\i*\dx}) circle (0.2mm);
}
\draw (0,1) node[left]{$1$};
\draw[->] (-0.1,0)--({\b+0.3},0);
\draw[thick, domain=\a:\b] plot ({\funcao{\x}},\x);
\draw[dotted] (\a,0) node[below]{$0$} --
(\a,{\funcao{\a}});
\draw[dotted] (\b,0) node[below]{$1$} --
(\b,{\funcao{\b}});

\begin{scope}[xshift=3cm, yshift=0.5cm, tdplot_main_coords]

\pgfmathsetmacro{\nx}{2};
\pgfmathsetmacro{\Dx}{\n/\nx};
\pgfmathsetmacro{\alphamin}{-40};
\pgfmathsetmacro{\alphamax}{150};
\pgfmathsetmacro{\nalpha}{40};
\pgfmathsetmacro{\Dalpha}{(\alphamax-\alphamin)/\nalpha};

%dessiner les axes x et z
\draw[->] (0,0,0) -- (1.2,0,0);
%\draw[->] (0,0,0) -- (0,0,1.2);

%dessiner le traitille de derriere:
\draw[dotted] (0,\a,0)--(0,\a,{\funcao{\a}});


%dessiner les tranches:
\foreach \i in {1,...,\n} {
\pgfmathsetmacro{\ap}{(\a+(\i-1)*\dx)};
\pgfmathsetmacro{\bp}{(\a+(\i*\dx))};
\pgfmathsetmacro{\valf}{\funcao{\bp}};
%remplir le premier cercle (derriere) de la tranche:
\filldraw[corretangulos, domain=0:360, variable=\alpha] plot
({sin(\alpha)*\valf},{cos(\alpha)*\valf},{\ap}); 
%prononcer un peu le bord:
\draw[color=gray, domain=0:360, variable=\alpha] plot
({sin(\alpha)*\valf},{cos(\alpha)*\valf},{\ap}); 
%dessiner les petits traits de la tranche:
\foreach \j in {0,...,\nalpha} {
\draw[color=gray, domain=\ap:\bp, variable=\t] plot
({sin(\alphamin+\Dalpha*\j)*\valf},{cos(\alphamin+\Dalpha*\j)*\valf},\t); 
%remplir le disque (devant) de la tranche:
\filldraw[corretangulos, domain=0:360, variable=\alpha] plot
({sin(\alpha)*\valf},{cos(\alpha)*\valf},\bp); 
%prononcer un peu le bord de ce disque
\draw[color=gray, domain=0:360, variable=\alpha] plot
({sin(\alpha)*\valf},{cos(\alpha)*\valf},\bp); 
\fill (0,{\funcao{\bp}},\bp) circle (0.2mm);
}
}
%dessiner le graphe de la fonction
\draw[thick, domain=\a:\b, variable=\t, samples=15] plot
(0,{\funcao{\t}},\t);
%dessiner le petit traitille:
\draw[dotted] (0,\b,0)--(0,\b,{\funcao{\b}});

\draw[->] (0,\b,0) -- (0,{\b+0.3},0) node[right]{$x$};

\draw[->] (0,1)--(0,1.3);
\end{scope}
\end{tikzpicture}\end{bmlimage}
\end{center}
%%
Neste caso, dividimos o intervalo $y\in [0,1]$ em
intervalos $[y_{i-1},y_{i}]$. Ao intervalo $[y_{i-1},y_i]$ 
associamos uma fatia horizontal
$F_i$ de altura $\Delta y_i=y_i-y_{i-1}$ de
de raio $\sqrt{1-y_i}$. De fato, já que $F_i$ está na altura $y_i$, o
seu raio é dado pelo \emph{inverso} da função $x\to 1-x^2$
(isto é
$y\mapsto \sqrt{1-y}$) no ponto $y_i$. Assim, $V(F_i)=\pi \sqrt{1-y_i}^2
\Delta y_i$, e o volume de $V(S_2)$ é aproximado pela soma das fatias:
$$
\sum_{i=1}^nV(F_i)=\sum_{i=1}^n\pi(1-y_i)\Delta y_i\,.
$$
Portanto, no limite $n\to \infty$, combinado com $\Delta y_i\to 0$,
obtemos:
$$
V(S_2)=\int_0^1\pi (1-y)\,dy=\tfrac{\pi}{2}\,.
$$
Na próxima seção mostraremos um outro jeito de calcular $V(S_2)$.
\end{ex}

\begin{exo}
Considere a região finita $R$ contida no primeiro quadrante, 
delimitada pelas curvas $y=x^2$, $y=x^4$.
Calcule o volume do sólido de revolução obtido girando $R$ em torno do
eixo $y$.
\begin{sol}
$\tfrac{\pi}{6}$.
\end{sol}
\end{exo}
(Haverá mais exercícios no fim da próxima seção.)

\subsection{Aproximação por cascas}
\index{aproximação!por cascas}
Os exemplos considerados na seção anterior partiam de uma decomposição
do sólido usando \emph{fatias cilíndricas}.
Vejamos agora um outro tipo de decomposição, usando \emph{cascas}.

\begin{ex}
Considere de novo a região $R$ do 
Exemplo \ref{ex_gira_parabola} (a área debaixo da parábola), e
o sólido $S_2$ gerado pela rotação de $R$ em torno do eixo $y$.
Lá, $V(S_2)$ foi calculado usando uma integral, que foi construida
a partir de uma soma de cilindros, obtidos
pela rotação de retangulos \emph{horizontais} em torno do
eixo $y$. Procuremos agora calcular o mesmo volume
$V(S_2)$, mas com uma integral obtida a partir de uma soma de
\emph{cascas}. Cascas são obtidas pela rotação de
retângulos \emph{verticais}, em torno do eixo $y$:

\begin{center}
\begin{bmlimage}\begin{tikzpicture}[scale=2.3]
\tdplotsetmaincoords{70}{110}
\newcommand{\funcao}[1]{{(1-(#1)^2)}}
\pgfmathsetmacro{\a}{0};
\pgfmathsetmacro{\b}{1};
%nombre de tranches:
\pgfmathsetmacro{\n}{6};
%la largeur d'un rectangle
\pgfmathsetmacro{\dx}{(\b-\a)/\n};


%la tranche que je vais faire tourner:
\pgfmathsetmacro{\num}{4}

%le xmin
\pgfmathsetmacro{\xmin}{\a+(\num-1)*\dx}
%le xmax
\pgfmathsetmacro{\xmax}{\a+\num*\dx}



%%%LE GRAPHE DE LA FONCTION ET LE RECTANGLE%%%%%
\draw[->] (0,-0.1)--(0,1.3);
%%desenhar um retangulo no plano
\filldraw[corretangulos] ({\a+(\num-1)*\dx},0)
rectangle ({\a+\num*\dx},\funcao{\a+\num*\dx});
\draw ({\a+(\num-1)*\dx},0)
rectangle ({\a+\num*\dx},\funcao{\a+\num*\dx});
\fill ({\a+\num*\dx},\funcao{\a+\num*\dx}) circle (0.2mm);
%desenhar a funcao
\draw[->] (-0.1,0)--({\b+0.3},0);
\draw[thick, domain=\a:\b] plot (\x, \funcao{\x});

\draw[<-, >=latex] (\xmin,0)--({\xmin-0.3},-0.3)
node[below]{$x_{i-1}$};
\draw[<-, >=latex] (\xmax,0)--({\xmax+0.3},-0.3)
node[below]{$x_{i}$};

\draw[decorate, decoration=brace] 
(\xmax+0.1,{\funcao{\xmax}})--(\xmax+0.1,0)node[midway,
right]{$f(x_i)$};

%3D%%% 
%dessin en trois dimension da casca
\begin{scope}[xshift=3.5cm, yshift=0cm, tdplot_main_coords]
\pgfmathsetmacro{\alphamin}{-40};
\pgfmathsetmacro{\alphamax}{150};
%dessiner les axes x et z
\draw (-\b+0.2,0,0) -- (\b-0.2,0,0);
\draw (0,-\b+0.2,0) -- (0,\b-0.2,0);
\draw[->] (0,0,0) -- (0,0,1.2);
%dessiner les deux cercles du bas:
\draw[domain=0:360, variable=\angle, samples=50, densely
dotted] 
plot ({\xmin*cos(\angle)},{\xmin*sin(\angle)},-0.005);
\draw[domain=0:360, variable=\angle, samples=50, densely
dotted] 
plot ({\xmax*cos(\angle)},{\xmax*sin(\angle)},-0.005);

%dessiner le rectangle en traitille:
\fill[corretangulos] (0,\xmin,0)--(0,\xmax,0)--
(0,\xmax,{\funcao{\xmax}})--(0,\xmin,{\funcao{\xmax}})--cycle;
\draw[thick] (0,\xmin,0)--(0,\xmax,0)--
(0,\xmax,{\funcao{\xmax}})--(0,\xmin,{\funcao{\xmax}})--cycle;
\pgfmathsetmacro{\tetaz}{50};
\fill[corretangulos]
({\xmin*cos(\tetaz)},{\xmin*sin(\tetaz)},0)--
({\xmax*cos(\tetaz)},{\xmax*sin(\tetaz)},0)--
({\xmax*cos(\tetaz)},{\xmax*sin(\tetaz)},{\funcao{\xmax}})--
({\xmin*cos(\tetaz)},{\xmin*sin(\tetaz)},{\funcao{\xmax}})--cycle;

\draw[dotted]
(0,0,0)--
({\xmax*cos(\tetaz)},{\xmax*sin(\tetaz)},0)--
({\xmax*cos(\tetaz)},{\xmax*sin(\tetaz)},{\funcao{\xmax}})--
(0,0,{\funcao{\xmax}})--cycle;
\draw[thick]
({\xmin*cos(\tetaz)},{\xmin*sin(\tetaz)},0)--
({\xmax*cos(\tetaz)},{\xmax*sin(\tetaz)},0)--
({\xmax*cos(\tetaz)},{\xmax*sin(\tetaz)},{\funcao{\xmax}})--
({\xmin*cos(\tetaz)},{\xmin*sin(\tetaz)},{\funcao{\xmax}})--cycle;

%dessiner les parois EXTERNE de DEVANT:
\pgfmathsetmacro{\valf}{ \funcao{\xmax} };
%ATENCAO: nangles eh o num de angulos em 180!!!
\pgfmathsetmacro{\nangles}{20};
\pgfmathsetmacro{\angleinit}{290};
\pgfmathsetmacro{\dalpha}{180/\nangles}

\foreach \i in {\nangles,...,1} {
\pgfmathsetmacro{\angleA}{\angleinit+(\i-1)*\dalpha};
\pgfmathsetmacro{\angleB}{\angleinit+\i*\dalpha};
\fill[corretangulos]({\xmax*cos(\angleA)},{\xmax*sin(\angleA)},0) --
 ({\xmax*cos(\angleB)},{\xmax*sin(\angleB)},0) --
 ({\xmax*cos(\angleB)},{\xmax*sin(\angleB)},{\valf}) --
 ({\xmax*cos(\angleA)},{\xmax*sin(\angleA)},{\valf}) -- cycle;
}


%dessiner le COUVERCLE
\pgfmathsetmacro{\xmin}{ \a + (\num-1)*\dx };
\pgfmathsetmacro{\xmax}{ \a + (\num*\dx) };
\pgfmathsetmacro{\valf}{\funcao{\xmax}};
%ATENCAO: nangles eh o num de angulos em 180!!!
\pgfmathsetmacro{\nangles}{20};
\pgfmathsetmacro{\angleinit}{0};
\pgfmathsetmacro{\dalpha}{360/\nangles}
\foreach \i in {\nangles,...,1} {
\pgfmathsetmacro{\angleA}{\angleinit+(\i-1)*\dalpha};
\pgfmathsetmacro{\angleB}{\angleinit+\i*\dalpha};
\fill[corretangulos]
({\xmin*cos(\angleA)},{\xmin*sin(\angleA)},{\valf}) --
({\xmax*cos(\angleA)},{\xmax*sin(\angleA)},{\valf}) --
({\xmax*cos(\angleB)},{\xmax*sin(\angleB)},{\valf}) --
({\xmin*cos(\angleB)},{\xmin*sin(\angleB)},{\valf}) 
-- cycle;
}


%dessiner les deux cercles du haut:
\draw[domain=0:360, variable=\angle, samples=50, densely dotted] 
plot ({\xmin*cos(\angle)},{\xmin*sin(\angle)},{\funcao{\xmax}});
\draw[domain=0:360, variable=\angle, samples=50, densely
dotted] 
plot ({\xmax*cos(\angle)},{\xmax*sin(\angle)},{\funcao{\xmax}});


%dessiner le graphe de la fonction
\draw[thick, domain=\a:\b, variable=\t, samples=15] plot
(0,\t, \funcao{\t});

\draw[->] (0,\xmax,0) -- (0,{\b+0.3},0) node[right]{$x$};

\fill (0,\xmax,{\funcao{\xmax}}) circle (0.2mm);
\draw (0,-0.6,0.85) node{$C_i$};

%Dessiner les fleches qui montrent que ca tourne:
\draw[domain=90:\tetaz, variable=\angle, samples=20, ->, thick] 
plot
({\xmax*cos(\angle)},{\xmax*sin(\angle)},{(\funcao{\xmax})*(0.5)});
\draw[domain=90:50, variable=\angle, samples=20, ->, thick] 
plot ({\xmax*cos(\angle)},{\xmax*sin(\angle)},{(\funcao{\xmax})});
\draw[domain=90:50, variable=\angle, samples=20, ->, thick] 
plot ({\xmax*cos(\angle)},{\xmax*sin(\angle)},0);
\end{scope}

\end{tikzpicture}\end{bmlimage}
\end{center}

O volume da casca $C_i$ pode ser calculado pela diferença dos
volumes de dois cilindros: o externo tem raio $x_i$, o
interno tem raio $x_{i-1}$, e ambos têm altura $f(x_i)$.
Logo, 
$$V(C_i)=\pi x_{i}^2\times f(x_i)-\pi x_{i-1}^2\times
f(x_i)=\pi(x_i^2-x_{i-1}^2)f(x_i)\,.$$
Fatorando,
$x_i^2-x_{i-1}^2=(x_i+x_{i-1})(x_i-x_{i-i})$. Quando 
$\Delta x_i=x_i-x_{i-1}$ for muito pequeno, isto é quando
$x_i$ e $x_{i-1}$ forem muito próximos, podemos
aproximar $x_i+x_{i+1}$ por $2x_i$. Logo, 
$$
V(C_i)\simeq 2\pi x_if(x_i)\Delta x_i\,.
$$
Obs: essa fórmula é facil de entender observando que a  
casca $C_i$ pode ser obtida torcendo um paralelepípedo
cuja base é o retângulo de base $(x_{i}-
x_{i-1})\times f(x_i)$ e de altura dada pela
circunferência do círculo de raio $x_i$, isto é $2\pi
x_i$. (Atenção: esse raciocíno é correto somente se a
base do retângulo é pequena em relação à sua distância ao
eixo de rotação!)\\

Portanto, o volume so sólido $S_2$ pode ser calculado via
a integral associada às somas de Riemann dos $V(C_i)$,
isto é:
$$
\boxed{
V(S_2)=\int_0^12\pi x f(x)\,dx\,.}
$$
Como era de se esperar, essa integral vale 
$$V(S_2)=\int_0^12\pi x(1-x^2)\,dx=\pisobredois\,.$$
\end{ex}


O último exemplo mostrou que o volume de um sólido pode ser calculado de
várias maneiras, usando cilindros ou cascas para o mesmo sólido
pode levar a integrar funções muito diferentes, e uma escolha
pode facilitar o cálculo da primitiva.

\begin{ex}\label{exemplo:solidos_girados}
Considere o triângulo $\cT$ determinado pelos pontos $A=(1,0)$,
$B=(1,1)$, $C=(2,0)$. 

\begin{center}
\begin{bmlimage}\begin{tikzpicture}[scale=2]
\coordinate (A) at (1,0);
\coordinate (B) at (1,1);
\coordinate (C) at (2,0);
\draw[->] (-0.2,0)--(3,0);
\draw[->] (0,-0.2)--(0,1);
\fill[areagrafico]  (A)--(B)--(C)--cycle;
\draw[thick] (A)--(B)--(C)--cycle;
\end{tikzpicture}\end{bmlimage}
\end{center}

Para começar, considere o cone $S_1$ obtido girando
$\cT$ em torno do eixo $x$:

\begin{center}
\begin{bmlimage}\begin{tikzpicture}[scale=2]
\tdplotsetmaincoords{55}{115}

\newcommand{\funcao}[1]{(2-(#1))}

\pgfmathsetmacro{\a}{1};
\pgfmathsetmacro{\b}{2};

\begin{scope}[tdplot_main_coords]
\pgfmathsetmacro{\nx}{20};
\pgfmathsetmacro{\Dx}{(\b-\a)/\nx};

\pgfmathsetmacro{\alphamin}{0};
\pgfmathsetmacro{\alphamax}{20};
\pgfmathsetmacro{\nalpha}{10};
\pgfmathsetmacro{\Dalpha}{(\alphamax-\alphamin)/\nalpha};


\draw[->] (0,0,0) -- (1.2,0,0);
\draw[->] (0,0,0) -- (0,{\b+0.3},0) node[anchor=north west]{$x$};
\draw[->] (0,0,0) -- (0,0,1.2);

%desenhar a flechinha de tras:
\draw[->, domain=0:330, variable=\alpha, samples=50] plot
({sin(\alpha)*(\funcao{\a})},\a,{cos(\alpha)*(\funcao{\a})}); 
%desenhar a linha pontilhada de tras:
\draw[thick, dotted] (0,\a,0)--(0,\a,{\funcao{\a}});

\fill[areagrafico]
(0,\a,0)--plot[domain=\a:\b](0,\x,{\funcao{\x}})--(0,\b,0)--cycle;
%desenhar o grafico de f em [a,b]
\draw[thick, domain=\a:\b, variable=\t] plot
(0,\t,{\funcao{\t}}); 
%desenhar a flechinha de frente:
\draw[->, domain=0:320, variable=\alpha] plot
({sin(\alpha)*\funcao{\b}},\b,{cos(\alpha)*\funcao{\b}}); 
%desenhar a linha pontilhada de frente:
\draw[thick,dotted] (0,\b,0)--(0,\b,{\funcao{\b}});
\end{scope}

\begin{scope}[xshift=3.5cm,tdplot_main_coords]
\pgfmathsetmacro{\nx}{15};
\pgfmathsetmacro{\Dx}{(\b-\a)/\nx};
\pgfmathsetmacro{\alphamin}{-50};
\pgfmathsetmacro{\alphamax}{155};
\pgfmathsetmacro{\nalpha}{35};
\pgfmathsetmacro{\Dalpha}{(\alphamax-\alphamin)/\nalpha};

\draw[->] (0,0,0) -- (1.2,0,0);
\draw[->] (0,0,0) -- (0,{\b+0.3},0) node[anchor=north west]{$x$};
\draw[->] (0,0,0) -- (0,0,1.2);
%desenhar o grande circulo de tras:
\draw[domain=0:360, variable=\alpha] plot
({sin(\alpha)*\funcao{\a}},\a,{cos(\alpha)*\funcao{\a}}); 
%desenhar a linha pontilhada de tras:
\draw[thick, dotted] (0,\a,0)--(0,\a,{\funcao{\a}});
%desenhar o grande circulo de frente:
\draw[domain=0:360, variable=\alpha] plot
({sin(\alpha)*\funcao{\b}},\b,{cos(\alpha)*\funcao{\b}}); 

\fill[areagrafico]
(0,\a,0)--plot[domain=\a:\b](0,\x,{\funcao{\x}})--(0,\b,0)--cycle;
\foreach \i in {0,...,\nalpha} {
\draw[thick, color=gray, domain=\a:\b, variable=\t] plot
({sin(\alphamin+\Dalpha*\i)*\funcao{\t}},\t,{cos(\alphamin+\Dalpha*\i)*\funcao{\t}}); 
}
\foreach \i in {0,...,\nx} {
\pgfmathsetmacro{\point}{\a+(\i*\Dx)};
\draw[thick, color=gray, domain=\alphamin:\alphamax, variable=\alpha] plot
({sin(\alpha)*\funcao{\point}},\point,{cos(\alpha)*\funcao{\point}}); 
 }
%desenhar o grafico de f em [a,b]
\draw[thick, thick, domain=\a:\b, variable=\t] plot
(0,\t,{\funcao{\t}}); 
%desenhar a linha pontilhada de frente:
\draw[thick,dotted] (0,\b,0)--(0,\b,{\funcao{\b}});
\draw (0,\b,{\funcao{\b}+0.8}) node[left]{$S_1$};
\end{scope}

\end{tikzpicture}\end{bmlimage}
\end{center}

Podemos calcular o volume de $S_1$ de duas maneiras. Primeiro,
girando retângulos verticais:

\begin{center}
\begin{bmlimage}\begin{tikzpicture}[scale=2]
\coordinate (A) at (1,0);
\coordinate (B) at (1,1);
\coordinate (C) at (2,0);
\draw[->] (0,0)--(2.5,0);
\draw[->] (0,-0.2)--(0,1);
%\fill[areagrafico]  (A)--(B)--(C)--cycle;
\draw (A)--(B)--(C)--cycle;

\pgfmathsetmacro{\s}{1.45};
\pgfmathsetmacro{\t}{1.55};
\fill[areagrafico] (\s,0)--(\s,{2-\t})--(\t,{2-\t})--(\t,0)--cycle;
\draw (\s,0)--(\s,{2-\t})--(\t,{2-\t})--(\t,0)--cycle;

\pgfmathsetmacro{\l}{0.3};
\draw[dotted] (\s,0)--(\s,{-\l});
\draw[dotted] (\t,0)--(\t,{-\l});
\draw[decorate, decoration=brace] 
(\t,{-\l})--(\s,{-\l})node[midway, below]{$dx$};

\draw (\t,0) node[below]{$x$};
\draw[thick] (1,0)node[below]{$1$}--(2,0)node[below]{$2$};

\pgfmathsetmacro{\g}{0.6};
\draw[decorate, decoration=brace] 
({\t+\g},{2-\t})--({\t+\g},{0})node[midway, right]{$f(x)$};
\draw[dotted] 
({\t+\g},{2-\t})--({\t},{2-\t});
\draw[dotted] 
({\t},{0})--({\t+\g},{0});

\end{tikzpicture}\end{bmlimage}
\end{center}

Seremos um pouco informais: 
o retângulo infinitesimal baseado em $x$ tem uma largura $dx$ e
uma altura $f(x)=2-x$ (que é a equação da reta que passa por $B$
e $C$). 
Ao girar em torno do eixo $x$, ele gera um
cilindro infinitesimal cuja base tem área igual a $\pi f(x)^2$, e altura 
$dx$.  Logo, o volume do cilindro é $\pi f(x)^2\times
dx=\pi(2-x)^2dx$, e o volume
de $S_1$ é obtido integrando todos os cilindros, quando $x$ varia de
$1$ até $2$:
\begin{equation}\label{eq_int_1}
V(S_1)=\int_1^2 \pi (2-x)^2\,dx\,.
\end{equation}

Mas é possível também calcular $V(S_1)$ girando retângulos
horizontais:

\begin{center}
\begin{bmlimage}\begin{tikzpicture}[scale=2]
\coordinate (A) at (1,0);
\coordinate (B) at (1,1);
\coordinate (C) at (2,0);
\draw[->] (0,0)--(2.5,0);
\draw[->] (0,-0.2)--(0,1.2);
%\fill[areagrafico]  (A)--(B)--(C)--cycle;
\draw (A)--(B)--(C)--cycle;

\pgfmathsetmacro{\s}{0.45};
\pgfmathsetmacro{\t}{0.55};
\fill[areagrafico] (1,\s)--({2-\t},\s)--({2-\t},\t)--(1,\t)--cycle;
\draw (1,\s)--({2-\t},\s)--({2-\t},\t)--(1,\t)--cycle;

\pgfmathsetmacro{\l}{0.3};
\draw[dotted] (1,\s)--({1-\l},\s);
\draw[dotted] (1,\t)--({1-\l},\t);
\draw[decorate, decoration=brace] ({1-\l},\s)--
({1-\l},\t) node[midway, left]{$dy$};

\draw (0,\t) node[left]{$y$};
\draw[dotted] (0,\t)--(1,\t);
\draw[dashed] (0,1)node[left]{$1$}--(1,1);
\draw (0,0)node[left]{$0$};
\draw[thick] (0,0)--(0,1);

\draw[dotted] (1,\s)--({1},{-\l});
\draw[dotted] ({2-\t},\s)--({2-\t},{-\l});
\draw[decorate, decoration=brace] ({2-\t},{-\l})-- 
(1,{-\l}) node[midway, below]{$h(y)$};

\end{tikzpicture}\end{bmlimage}
\end{center}

Um retângulo horizontal infinitesimal 
é definido pela sua posição com respeito
ao eixo $y$, pela sua altura, dada por $h(y)=(2-y)-1=1-y$ (aqui
calculamos a diferença entre a posição do seu ponto mais a
direita e do seu ponto mais a esquerda). 
Ao girar em torno do eixo $x$, esse retângulo gera uma casca
cujo raio é $y$, cuja altura é $h(y)$ e cuja espessura é $dy$,
logo, o seu volume é $2\pi y \times h(y)\times dy=2\pi y(1-y)dy$. Integrando
sobre todas as cascas, com $y$ variando entre $0$ e $1$:
\begin{equation}\label{eq_int_2}
V(S_1)=\int_0^12\pi y (1-y)\,dy\,.
\end{equation}


\begin{exo}\label{exo:verificar}
Verifique que os valores das integrais em \eqref{eq_int_1} e \eqref{eq_int_2}
são iguais.
\end{exo}

Consideremos agora o solído $S_2$ obtido girando $\cT$ em torno
da reta de equação $x=3$.

\begin{center}
\begin{bmlimage}\begin{tikzpicture}[scale=1.8]
\tdplotsetmaincoords{55}{115}

\newcommand{\funcao}[1]{(2-(#1))}

\pgfmathsetmacro{\a}{1};
\pgfmathsetmacro{\b}{2};

\begin{scope}[tdplot_main_coords]
\pgfmathsetmacro{\nx}{20};
\pgfmathsetmacro{\Dx}{(\b-\a)/\nx};

\pgfmathsetmacro{\alphamin}{0};
\pgfmathsetmacro{\alphamax}{20};
\pgfmathsetmacro{\nalpha}{10};
\pgfmathsetmacro{\Dalpha}{(\alphamax-\alphamin)/\nalpha};


\draw[->] (0,0,0) -- (1.2,0,0);
\draw[->] (0,0,0) -- (0,{5.5},0);
\draw[->] (0,0,0) -- (0,0,1.2);

%desenhar o eixo de rotacao:
\draw[dashed] (0,3,-0.4)--(0,3,1.6);

%desenhar a linha pontilhada de tras:
\draw[thick, dotted] (0,\a,0)--(0,\a,{\funcao{\a}});

%desenhar a flechinha de tras:
\draw[->, domain=180:-140, variable=\alpha, samples=50] plot
({2*sin(\alpha)},{3+2*cos(\alpha)},0); 

\draw (0,3,0.6) node[above right]{$x=3$};

\fill[areagrafico]
(0,\a,0)--plot[domain=\a:\b](0,\x,{\funcao{\x}})--(0,\b,0)--cycle;
%desenhar o grafico de f em [a,b]
\draw[thick, domain=\a:\b, variable=\t] plot
(0,\t,{\funcao{\t}}); 

\end{scope}
\end{tikzpicture}\end{bmlimage}
\end{center}

Comecemos girando retângulos verticais:
\begin{center}
\begin{bmlimage}\begin{tikzpicture}[scale=2]
\coordinate (A) at (1,0);
\coordinate (B) at (1,1);
\coordinate (C) at (2,0);
\draw[->] (0,0)--(4,0);
\draw[->] (0,-0.2)--(0,1);
%\fill[areagrafico]  (A)--(B)--(C)--cycle;
\draw (A)--(B)--(C)--cycle;

\pgfmathsetmacro{\s}{1.45};
\pgfmathsetmacro{\t}{1.55};
\fill[areagrafico] (\s,0)--(\s,{2-\t})--(\t,{2-\t})--(\t,0)--cycle;
\draw (\s,0)--(\s,{2-\t})--(\t,{2-\t})--(\t,0)--cycle;

\pgfmathsetmacro{\l}{0.3};
\draw[dotted] (\s,0)--(\s,{-\l});
\draw[dotted] (\t,0)--(\t,{-\l});
\draw[decorate, decoration=brace] 
(\t,{-\l})--(\s,{-\l})node[midway, below]{$dx$};

\draw (\t,0) node[below]{$x$};
\draw[thick] (1,0)node[below]{$1$}--(2,0)node[below]{$2$};

\pgfmathsetmacro{\g}{0.6};
\draw[decorate, decoration=brace] ({\g},{0})--
({\g},{2-\t})node[midway, left]{$f(x)$};
\draw[dotted] 
({\g},{2-\t})--({\t},{2-\t});
\draw[dotted] 
({\t},{0})--({\g},{0});

\draw[dashed] (3,-0.3)--(3,1.3);
\draw[decorate, decoration=brace] (\t,{2-\t})--
(3,{2-\t})node[midway, above]{$r(x)$};

\draw (3,0) node[below right]{$3$};


\end{tikzpicture}\end{bmlimage}
\end{center}

Ao girar o retângulo representado na figura em torno da reta $x=3$,
isto gera uma casca de raio
$r(x)=3-x$, de altura $f(x)=2-x$ e de espessura $dx$. Logo, o seu
volume é dado por $2\pi r(x)\times f(x)\times
dx=2\pi(3-x)(2-x)dx$. O volume de $S_2$ é obtido integrando com
respeito a $x$, entre $1$ e $2$:
\[ V(S_2)=\int_1^22\pi(3-x)(2-x)\,dx\,.  \]
Girando agora retângulos horizontais:


\begin{center}
\begin{bmlimage}\begin{tikzpicture}[scale=2]
\coordinate (A) at (1,0);
\coordinate (B) at (1,1);
\coordinate (C) at (2,0);
\draw[->] (0,0)--(4,0);
\draw[->] (0,-0.2)--(0,1.2);
%\fill[areagrafico]  (A)--(B)--(C)--cycle;
\draw (A)--(B)--(C)--cycle;


\draw[dashed] (3,-0.3)--(3,1.3);
\draw (3,0) node[above right]{$3$};

\pgfmathsetmacro{\s}{0.45};
\pgfmathsetmacro{\t}{0.55};
\fill[areagrafico] (1,\s)--({2-\t},\s)--({2-\t},\t)--(1,\t)--cycle;
\draw (1,\s)--({2-\t},\s)--({2-\t},\t)--(1,\t)--cycle;

\pgfmathsetmacro{\l}{0.3};
\draw[dotted] (1,\s)--({1-\l},\s);
\draw[dotted] (1,\t)--({1-\l},\t);
\draw[decorate, decoration=brace] ({1-\l},\s)--
({1-\l},\t) node[midway, left]{$dy$};

\draw (0,\t) node[left]{$y$};
\draw[dotted] (0,\t)--(1,\t);
\draw[dashed] (0,1)node[left]{$1$}--(1,1);
\draw (0,0)node[left]{$0$};
\draw[thick] (0,0)--(0,1);

\draw[dotted] (1,\s)--({1},{-1.5*\l});
\draw[dotted] ({2-\t},\s)--({2-\t},{-0.6*\l});
\draw[decorate, decoration=brace] ({3},{-1.5*\l})-- 
(1,{-1.5*\l}) node[midway, below]{$R(y)$};
\draw[decorate, decoration=brace] ({3},{-0.6*\l})-- 
({2-\t},{-0.6*\l}) node[midway, below]{$r(y)$};

\end{tikzpicture}\end{bmlimage}
\end{center}
Ao girar em torno da reta vertical $x=3$, o retângulo horizontal gera um
anel, de altura $dy$, de raio exterior $R(y)=2$, de raio
interior $r(y)=3-(2-y)=1+y$. O volume desse anel é dado por 
$\pi R(y)^2 \times dy-\pi r(y)^2\times dy$. Logo, o volume de
$S_2$ é dado pela integral
$$
V(S_2)=\int_0^1 (\pi 2^2-\pi (1+y)^2)\,dy\,.
$$

\end{ex}

\subsection{Exercícios}

\begin{exo}
Considere a região $R$ delimitada pelo gráfico da função $y=\sen x$,
pelo eixo $x$, e pelas duas retas $x=\pi/2$, $x=\pi$.
Calcule a área de $R$. Em seguida,
monte uma integral (não precisa calculá-la) cujo valor dê o volume so sólido
obtido girando $R$: 1) em torno do eixo $x$, 2) em torno da reta $x=\pi$.
\begin{sol}
A área é dada por 
$$\int_{\pi/2}^\pi\sen (x)dx=-\cos (x)|^{\pi}_{\pi/2}=-(-1)-0=1\,.$$
Girando em torno do eixo $x$:
$V_1=\int_{\pi/2}^{\pi}\pi(\sen x)^2\,dx$.
Ou, com as cascas: $V_1=\int_0^12\pi y (\pi/2-\arcsen y)\,dy$.
Em torno da reta $x=\pi$, usando as cascas:
$V_2=\int_{\pi/2}^\pi2\pi(\pi-x)\sen x\,dx$.
Sem usar as cascas:
$V_2=\pi(\tfrac{\pi}{2})^2\cdot 1-\int_0^1\pi (\arcsen y)^2\,.dy$.
\end{sol}
\end{exo}

\begin{exo}
Mostre que o volume de um cone de base circular de raio $R$ e de altura $H$ é
igual a $V=\frac{1}{3}\pi R^2H$.
\begin{sol}
O cone pode ser (tem vários jeitos, mas esse é o mais simples) 
obtido girando o gráfico da função $f(x)=\frac{R}{H}x$, $0\leq x\leq H$, em
torno do eixo $x$. Logo,
$$
V=\int_0^H\pi\big(\frac{R}{H}x\Big)^2dx=\pi\frac{R^2}{H^2}\int_0^Hx^2dx=
\pi\frac{R^2}{H^2}\frac{H^3}{3}=\frac{1}{3}\pi R^2H \,\,$$
Obs: pode também rodar o gráfico da função $f(x)=-\frac{H}{R}x+H$, $0\leq x\leq
R$, em torno do eixo $y$.
\end{sol}
\end{exo}

\begin{exo}(Prova 3, 2010, Turmas N)
Calcule o volume do sólido obtido girando a região $R=\{(x,y):1\leq x\leq
e\,,\,0\leq y\leq \sqrt{x}\ln x\}$ em torno da reta $y=0$.
\begin{sol}
O volume é dado por $V=\int_1^e\pi(\sqrt{x}\ln x)^2dx$. Integrando duas vezes
por partes, obtem-se
\begin{align*}
\int x(\ln x)^2dx&=\frac{x^2}{2}(\ln x)^2-\int \frac{x^2}{2}2(\ln
x)\frac{1}{x}dx\\
&=\frac{x^2}{2}(\ln x)^2-\int x\ln xdx\\
&=\frac{x^2}{2}(\ln x)^2-\big\{\frac{x^2}{2}\ln x-\int\frac{x^2}{2}\frac{1}{x}dx
\big\}\\
&=\frac{x^2}{2}(\ln x)^2-\frac{x^2}{2}\ln x+\frac{x^2}{4}+C
\end{align*}
Logo, $V=\pi\frac{e^2-1}{4}$.
\end{sol}
\end{exo}

\begin{exo}
Considere a região $R$ delimitada pela parábola $y=x^2$, 
pelo eixo $x$ e pela reta $x=1$,
contida no primeiro quadrante.
Para cada uma das retas abaixo, monte uma integral (sem calculá-la) que dê o volume do
sólido obtido girando $R$ em torno da reta $r$, usando a) cílindros, b) cascas.
\begin{multicols}{3}
\begin{enumerate}
\item\label{itexorotsol1}  $y=0$,
\item\label{itexorotsol2} $y=1$,
\item\label{itexorotsol3}   $y=-1$,
\item\label{itexorotsol4} $x=0$,
\item\label{itexorotsol5} $x=1$,
\item\label{itexorotsol6} $x=-1$.
\end{enumerate}
\end{multicols}
\vspace{0.01cm}
\begin{sol}
\eqref{itexorotsol1}
Cil.: $\int_0^1\pi (x^2)^2\,dx$, 
Casc.:
$\int_0^12\pi y(1-\sqrt{y})\,dy$.
\eqref{itexorotsol2}
Cil.: $\int_0^1\pi(1^2-(1-x^2)^2)\,dx$
Casc.: $\int_0^12\pi(1-y)(1-\sqrt{y})\,dy$,
\eqref{itexorotsol3}
Cil.: $\int_0^1\pi((1+x^2)^2-1^2)\,dx$
Casc.: $\int_0^1 2\pi(1+y)(1-\sqrt{y})\,dy$
\eqref{itexorotsol4}
Cil.: $\int_0^1\pi(1^2-\sqrt{y}^2)\,dy$
Casc.: $\int_0^12\pi x\cdot x^2\,dx$
\eqref{itexorotsol5}
Cil. $\int_0^1\pi(1-\sqrt{y})^2\,dy$
Casc.: $\int_0^12\pi(1-x)x^2\,dx$
\eqref{itexorotsol6}
Cil.: $\int_0^1\pi(2^2-(1+\sqrt{y})^2)\,dy$
Casc. $\int_0^12\pi(1+x)x^2\,dx$
\end{sol}
\end{exo}


\begin{exo}
Monte uma integral cujo valor seja igual ao volume do
sólido obtido girando a região $R$ (finita, delimitada pela curva $y=1-(x-2)^2$
e o eixo $x$) em torno da reta $y=2$.
\begin{sol}
Com o método dos cilíndros,
$$V=\int_1^3\pi 2^2dx-\int_1^3\pi\big(2-(1-(x-2)^2)\big)^2dx\,\,.$$
OU, usando o método das cascas, 
$$
V=\int_0^12\pi(2-y)2\sqrt{1-y}dy\,.
$$
OU, transladando o gráfico da função, e girando a nova região (finita,
delimitada pela nova curva $y=-1-x^2$ e o eixo $x$),
$$V=\int_{-1}^{+1}\pi 2^2dx-\int_{-1}^{+1}\pi(-1-x^2)^2dx\,.$$
\end{sol}
\end{exo}


\begin{exo}
Considere o sólido $S$ obtido girando o gráfico da função $f(x)=\cosh(x)$ em
torno da reta $y=0$, entre $x=-1$ e $x=+1$. Esboce $S$, e calcule o seu volume.
(Lembre que $\cosh(x)\pardef\frac{e^x+e^{-x}}{2}$.)
\begin{sol}
O volume é dado pela integral 
\begin{align*}
V=\int_{-1}^{+1}\pi \cosh^2xdx&=\pi\int_{-1}^{+1}\frac{e^{2x}+2+e^{-2x}}{4}dx\\
&=\frac{\pi}{4}\Big\{
\frac{e^{2x}}{2}+2x-\frac{e^{-2x}}{2}
\Big\}_{-1}^{+1}\\
&=\frac{\pi}{4}\big\{e^2+4-e^{-2}\big\}
\end{align*}
\end{sol}
\end{exo}

\begin{exo}
Considere a região $R$ delimitada pelo gráfico da função
$f(x)=\cos x$, pelas retas
$x=\frac{\pi}{2}$, $x=\pi$, e pelo eixo $x$.
Monte duas integrais, cujos valores dão o volume do
sólido de revolução obtido girando
$R$ em torno 1) da reta $x=\pi$, 2) da reta $y=-1$.
\begin{sol}
Em torno da reta $x=\pi$:
$$
V=\int_{\pi/2}^{\pi}2\pi(\pi-x)|\cos x|\,dx\,,\quad\text{ ou }
\quad V=\int_{-1}^0\pi(\pi-\arcos y)^2\,dy\,.
$$
Em torno da reta $y=-1$:
$$
V=\int_{\pi/2}^\pi \pi\cdot 1^2\,dx-\int_{\pi/2}^\pi\pi(\cos
x -(-1))^2\,dx\,,\quad\text{ ou }
\quad V=\int_{-1}^02\pi (y-(-1))(\pi-\arcos y)\,dy\,.
$$
\end{sol}
\end{exo}


\begin{exo}
Um toro é obtido girando um disco de raio $r$ em torno de
um eixo vertical, mantendo o centro do disco a distância
$R$ ($R>r$) do eixo.
Mostre que o volume desse toro é igual a $2\pi^2 r^2R$.
\end{exo}

\section{Áreas de superfícies de revolução}

Suponha que se queira calcular a \emph{área da superfície} do
sólido do início da Seção
\ref{Sec_Solidos} (sem os dois discos de frente
e de trás), denotada $A(S)$.
De novo, aproximaremos a área $A(S)$ por uma soma de áreas mais
simples.\\

Para decompor a área em áreas mais elementares, 
escolhamos uma divisão $a=x_0<x_1<\dots<x_n=b$,
e para cada intervalo $[x_{i-1},x_i]$, consideremos o anel $J_i$
obtido girando o segmento ligando $(x_{i-1},f(x_{i-1}))$ a 
$(x_i,f(x_i))$ em torno do eixo $x$:

\begin{center}
\begin{bmlimage}\begin{tikzpicture}[scale=2]
\tdplotsetmaincoords{55}{115}

\newcommand{\funcao}[1]{(1/((#1)^2+1))}

\pgfmathsetmacro{\a}{0.2};
\pgfmathsetmacro{\b}{1.5};

\pgfmathsetmacro{\nx}{7};
\pgfmathsetmacro{\len}{3}
\pgfmathsetmacro{\Dx}{(\b-\a)/\nx};

\pgfmathsetmacro{\point}{\a+(\len*\Dx)};
\pgfmathsetmacro{\pointb}{\a+((\len+1)*\Dx)};

\draw[->] (0,0) -- ({\b+0.3},0);
\draw[->] (0,0) -- (0,1.2);

\draw[dotted] (\a,0) -- (\a,{\funcao{\a}});
\draw[dotted] (\b,0) -- (\b,{\funcao{\b}});

\draw[densely dotted] (\point,0)--(\point,{\funcao{\point}});
\draw[densely dotted] (\pointb,0)--(\pointb,{\funcao{\pointb}});

\draw[color=gray!50, domain=\a:\b, variable=\t] plot
(\t,{\funcao{\t}}); 

\draw[thick] (\point,{\funcao{\point}})--
(\pointb,{\funcao{\pointb}});

\draw[<-,>=latex] (\point,0)--({\point-0.1},-0.2)
node[left]{$x_{i-1}$};


\draw[<-,>=latex] (\pointb,0)--({\pointb+0.1},-0.2)
node[right]{$x_{i}$};

\begin{scope}[xshift=4cm, yshift=0.5cm, tdplot_main_coords]

\pgfmathsetmacro{\alphamin}{0};
\pgfmathsetmacro{\alphamax}{360};
\pgfmathsetmacro{\nalpha}{40};
\pgfmathsetmacro{\Dalpha}{(\alphamax-\alphamin)/\nalpha};

\draw (0,\pointb,{\funcao{\pointb}+0.2}) node[above right]{$J_i$};

\draw[->] (0,0,0) -- (1.2,0,0);
\draw[->] (0,0,0) -- (0,{\b+0.3},0) node[anchor=north west]{$x$};
\draw[->] (0,0,0) -- (0,0,1.2);

\draw[thick, dotted] (0,\a,0)--(0,\a,{\funcao{\a}});

\draw[thick, color=gray, domain=\alphamin:\alphamax, variable=\alpha] plot
({sin(\alpha)*\funcao{\point}},\point,{cos(\alpha)*\funcao{\point}}); 
\draw[dotted] (0,\point,{\funcao{\point}})--(0,\point,0);
\draw[dotted] (0,\pointb,{\funcao{\pointb}})--(0,\pointb,0);
\foreach \j in {0,...,\nalpha} {
\draw ({sin(\alphamin+\Dalpha*\j)*\funcao{\point}},
\point,{cos(\alphamin+\Dalpha*\j)*\funcao{\point}})--
({sin(\alphamin+\Dalpha*\j)*\funcao{\pointb}},
\pointb,{cos(\alphamin+\Dalpha*\j)*\funcao{\pointb}})--
({sin(\alphamin+\Dalpha*(\j+1))*\funcao{\pointb}},
\pointb,{cos(\alphamin+\Dalpha*(\j+1))*\funcao{\pointb}});
}

\draw[thick, thick, domain=\a:\b, variable=\t] plot
(0,\t,{\funcao{\t}}); 
\draw[thick,dotted] (0,\b,0)--(0,\b,{\funcao{\b}});
\end{scope}

\end{tikzpicture}\end{bmlimage}
\end{center}

Pode ser verificado que o anel $J_i$ tem uma área dada por 
\begin{equation}\label{eq_area_anel}
A(J_i)=\pi\sqrt{(x_i-x_{i-1})^2+(f(x_i)-f(x_{i-1}))^2}(f(x_i)+f(x_{i-1}))\,.
\end{equation}

Quando $\Delta x_i=x_i-x_{i-1}$ for suficientemente pequeno, e se
$f$ for contínua, $f(x_i)+f(x_{i-1})$ pode ser aproximada por
$2f(x_i)$.
Logo, colocando $\Delta x_i$ em evidência dentro da raiz,
\begin{equation}
A(J_i)\simeq 2 \pi f(x_i)\sqrt{1+\Bigl(\frac{f(x_i)-f(x_{i-1})}{\Delta
x_i}\Bigr)^2)}\Delta x_i\,.
\end{equation}
Quando $\Delta x_i$ for pequeno, o quociente $(\frac{f(x_i)-f(x_{i-1})}{\Delta
x_i}$ pode ser aproximado por $f'(x_i)$. Logo, a 
área total pode ser aproximada pela soma de Riemann
$$
\sum_{i=1}^n A(J_i)\simeq \sum_{i=1}^n 
2\pi f(x_i)\sqrt{1+(f'(x_i))^2}\Delta x_i\,. 
$$
Quando $n\to \infty$ e todos os $\Delta x_i\to 0$, a soma de
Riemann acima converge para a integral
\begin{equation}
A(S)=\int_a^b2\pi f(x)\sqrt{1+(f'(x))^2}\,dx\,.
\end{equation}

\begin{ex}
Considere a superfície gerada pela rotação da curva
$y=\sqrt{x}$ em torno do eixo $x$, entre $x=0$ e $x=1$. A sua
área é dada pela integral 
\begin{align*}
A(S)&=\int_0^1
2\pi\sqrt{x}\sqrt{1+(\tfrac{1}{2\sqrt{x}})^2}\,dx=
\pi\int_0^1 \sqrt{1+4 x}\,dx=\tfrac{\pi}{6}(5^{3/2}-1)\,.
\end{align*}
\end{ex}

\begin{exo}
Prove \eqref{eq_area_anel}.
\begin{sol}
Se trata de mostrar que 
a área lateral de um cone truncado de raios $r\leq R$ 
e de altura $h$ é dada por 
$$
A=\pi (R+r)\sqrt{h^2+(R-r)^2}\,.
$$
De fato, fazendo o corte,
\begin{center}
\begin{bmlimage}\begin{tikzpicture}
\coordinate (A) at (0,0);
\coordinate (B) at (0,1);
\coordinate (C) at (0,3);
\coordinate (D) at (1.333,1);
\coordinate (E) at (2,0);
\draw (A)--(B) node[midway, left]{$h$}--
(C)--(E)--(A) node[midway, above]{$R$};
\draw (B)--(D) node[midway, above]{$r$};
\draw (C) node[left]{$C$};
\draw (D) node[right]{$D$};
\draw (E) node[right]{$E$};
\end{tikzpicture}\end{bmlimage}
\end{center}
Chamando a distância $CD$ de $l$, e a distância $CE$ de $L$, temos
$A=\pi R L-\pi rl$. Uma conta elementar mostra que
$l=\frac{r}{R-r}\sqrt{h^2+(R-r)^2}$, e que 
$L=\frac{R}{R-r}\sqrt{h^2+(R-r)^2}$.
Isso dá a fórmula desejada.
\end{sol}
\end{exo}

\begin{exo}
Mostre que a área da superfície de uma esfera de raio $R$ é igual a $4\pi R^2$.
\begin{sol}
Como a esfera é obtida girando o gráfico de
$f(x)=\sqrt{R^2-x^2}$, a sua área é dada por 
$$
A=2\pi\int_{-R}^R\sqrt{R^2-x^2}\sqrt{1+\bigl(\sqrt{R^2-x^2}'\bigr)^2}\,dx
=2\pi R\int_{-R}^R\,dx= 4\pi R^2\,.
$$
\end{sol}
\end{exo}

\section{Energia potencial}
(em construção)
\section{Resolvendo equações diferenciais}
(em construção)
