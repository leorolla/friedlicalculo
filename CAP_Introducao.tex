
% !TeX spellcheck = pt_BR
% !TEX encoding = UTF-8 Unicode

\chapter{Prefácio}
%\addcontentsline{toc}{chapter}{Prefácio}
%\chapter{Introdução}

Oriundo principalmente do estudo da mecânica e da astronomia, o 
\emph{Cálculo}, 
chamado também \emph{Cálculo infinitesimal}, nasceu no fim do século
XVII, com os trabalhos de Newton~\footnote{Sir Isaac Newton
(Woolsthorpe-by-Colsterworth, 4 de janeiro de 1643 — Londres, 31 de 
março de
1727).} e Leibniz~\footnote{Gottfried Wilhelm von Leibniz (Leipzig, 1 
de julho
de 1646 — Hanôver, 14 de novembro de 1716).}. Hoje em dia, ele é usado 
em todas
as áreas da ciência, e fundamental nas áreas da 
engenharia.\\

A presente apostila contém a ementa da matéria \emph{Cálculo I}, como 
ensinada 
no Departamento de Matemática da UFMG.
Ela tem como objetivo fornecer ao aluno um conhecimento básico dos 
conceitos 
principais do Cálculo que são: limites, derivadas e integral. Ela também 
prepara
o aluno para as outras matérias que usam Cálculo I nos cursos de 
ciências exatas
(física e matemática) e engenharia, 
tais como Cálculo II e III, EDA, EDB, EDC...\\

A apostila começa com um capítulo sobre fundamentos, fazendo uma revisão 
de 
vários conceitos básicos em princípio já conhecidos pelo aluno: equações,
inequações, plano cartesiano e trigonometria. A partir do Capítulo
\ref{Cap:Funcoes}, o conceito de função é introduzido. A noção central de
\emph{limite} é abordada no Capítulo \ref{Cap:Limites}, e a de 
\emph{derivada}
no Capítulo \ref{Cap:Derivacao}. O resto do texto é sobre o objeto 
central desse
curso: a noção de \emph{integral}, o \emph{Teorema Fundamental do 
Cálculo}, e
as suas aplicações.\\

O texto contém bastante exercícios, cuja compreensão é fundamental para a 
assimilação dos conceitos.
As soluções, às vezes detalhadas, se encontram num apêndice.\\

%Essa apostila está em fase de elaboração.
%Qualquer sugestão, crítica ou correção é bem vinda: 
%\verb|sacha@mat.ufmg.br|.\\

Agradeço às seguinte pessoas pelas suas contribuições:
Euller Tergis Santos Borges,
Felipe de Lima Horta Radicchi, 
Fernanda de Castro Maia, 
Hugo Freitas Reis,
Marina Werneck Ragozo,
Mariana Chamon Ladeira Amancio,
Pedro Silveira Gomes de Paiva,
Toufic Mahmed Pottier Lauar, 
Prof. Carlos Maria Carballo,
Prof. Fábio Xavier Penna (UNIRIO),
Prof. Francisco Dutenhefner,
Prof. Hamilton Prado Bueno,
Prof. Jorge Sabatucci,
Profa. Sylvie Oliffson Kamphorst Silva,
Profa. Viviane Ribeiro Tomaz da Silva,
Prof. Viktor Bekkert.\\

Alguns vídeos, criados uma vez para atender a uma classe
online, se encontram em 
\begin{center}
\verb|www.youtube.com/chachf|
\end{center}
Esses vídeos contêm uma boa
parte do conteúdo da presente apostila, mas alguns são de
qualidade baixa e precisam ser regravados....
\\

\hfill
Belo Horizonte, julho de 2015

