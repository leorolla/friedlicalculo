%&pdflatex
% !TeX spellcheck = pt_BR
% !TEX encoding = UTF-8 Unicode

\documentclass[12pt, reqno]{book}

%\def\updateans {}
% Only need to enable once in a lifetime, when the file ans.tex needs to be updated.

\synctex=1

\usepackage[brazil]{babel}
\usepackage[utf8]{inputenc}
\usepackage[T1]{fontenc}
\usepackage{lmodern}

\usepackage{bookml}

\title{Cálculo 1}

\usepackage{amssymb, framed, amsmath, amsthm, amscd, graphicx, amsfonts, color, xcolor}

\iflatexml
\else
\usepackage{polynom}
%%PACKAGES TIKZ
\usepackage{pgf, tikz}
\usepackage{tkz-tab}
%\usepackage{pgf}
%\usepackage{3dplot}
\usepackage{tikz-3dplot}
\usetikzlibrary{decorations.text}
\usetikzlibrary{calc}
%\usetikzlibrary{external}
%\tikzexternalize[prefix=tikzexfig/]
\usetikzlibrary{arrows,snakes,backgrounds,automata}
\usepackage{pgfplots}
%%%CORES NO TIKZ%%%
\tikzstyle{areagrafico}=[color=gray!20]
\tikzstyle{areafuncaoarea}=[color=gray!35]
\tikzstyle{corretangulos}=[color=gray!30, opacity=0.5]
\tikzstyle{corretangulos_back}=[color=gray!10, opacity=0.2]
\tikzstyle{corretangulos_front}=[color=gray!40, opacity=0.8]
%%%para os intervalos fechados:
\tikzstyle{intfechado}=[fill=black]
%%%e abertos:
\tikzstyle{intaberto}=[draw=black, fill=white]
%%CORES TRIGO
\newcommand{\coulseno}{gray!90}
\newcommand{\coulcoseno}{gray!70!black}
\newcommand{\coultang}{gray!50!black}
\fi

\usepackage{wrapfig}

\usepackage[top=3cm,left=2.5cm,right=2.5cm,bottom=3cm,a4paper]{geometry}

\usepackage{multicol}
\usepackage{caption}

\usepackage{verbatim}

\ifdefined\updateans
% Only need to run once in a lifetime, when the file ans.tex needs to be updated.
\usepackage{answers}
\Newassociation{sol}{Solution}{ans}
\Newassociation{resp}{Resposta}{ans}
\renewcommand{\Solutionlabel}[1]{\emph{{\bf #1}:}}
\renewcommand{\Respostalabel}[1]{\emph{Resposta do Exercício #1:}}
\else
\let\sol=\comment
\let\endsol=\endcomment
\let\ans=\comment
\let\endans=\endcomment
\fi

\newcommand{\questao}[3]{\item(#1pts) #2}

\newcommand{\pt}[1]{(\mathbf{#1}{\text{\bf pts}})}


\newcommand{\grasA}[1]{\textbf{#1}}
\newcommand{\eq}[1]{\begin{equation}#1\end{equation}}
\newcommand{\al}[1]{\begin{align}#1\end{align}}
\newcommand{\Z}{\mathbb{Z}}
\newcommand{\R}{\mathbb{R}}
\newcommand{\N}{\mathbb{N}}
\newcommand{\half}{\frac{1}{2}}

%LETTRES
\newcommand{\cA}{\mathcal{A}}
\newcommand{\cB}{\mathcal{B}}
\newcommand{\cC}{\mathcal{C}}
\newcommand{\cD}{\mathcal{D}}
\newcommand{\cE}{\mathcal{E}}
\newcommand{\cF}{\mathcal{F}}
\newcommand{\cG}{\mathcal{G}}
\newcommand{\cH}{\mathcal{H}}
\newcommand{\cI}{\mathcal{I}}
\newcommand{\cJ}{\mathcal{J}}
\newcommand{\cK}{\mathcal{K}}
\newcommand{\cL}{\mathcal{L}}
\newcommand{\cM}{\mathcal{M}}
\newcommand{\cN}{\mathcal{N}}
\newcommand{\cO}{\mathcal{O}}
\newcommand{\cP}{\mathcal{P}}
\newcommand{\cQ}{\mathcal{Q}}
\newcommand{\cR}{\mathcal{R}}
\newcommand{\cS}{\mathcal{S}}
\newcommand{\cT}{\mathcal{T}}
\newcommand{\cU}{\mathcal{U}}
\newcommand{\cV}{\mathcal{V}}
\newcommand{\cW}{\mathcal{W}}
\newcommand{\cX}{\mathcal{X}}
\newcommand{\cY}{\mathcal{Y}}
\newcommand{\cZ}{\mathcal{Z}}

\newcommand{\bA}{\mathbb{A}}
\newcommand{\bB}{\mathbb{B}}
\newcommand{\bL}{\mathbb{L}}
\newcommand{\bC}{\mathbb{C}}
\newcommand{\bR}{\mathbb{R}}
\newcommand{\bK}{\mathbb{K}}
\newcommand{\bP}{\mathbb{P}}
\newcommand{\bE}{\mathbb{E}}
\newcommand{\bQ}{\mathbb{Q}}
\newcommand{\bS}{\mathbb{S}}
\newcommand{\bF}{\mathbb{F}}
\newcommand{\bN}{\mathbb{N}}
\newcommand{\bT}{\mathbb{T}}
\newcommand{\bX}{\mathbb{X}}
\newcommand{\bZ}{\mathbb{Z}}


\usepackage{thmtools}
%\usepackage{ntheorem}

\theoremstyle{plain}

\declaretheorem[name=Teorema, numberwithin=chapter, 
shaded={bgcolor={RGB}{205,205,205}}]{teo}
\declaretheorem[name=Proposição, numberwithin=chapter, 
shaded={bgcolor={RGB}{210,210,210}}]{pro}
\declaretheorem[name=Corolário, numberwithin=chapter, 
shaded={bgcolor={RGB}{215,215,215}}]{cor}
\declaretheorem[name=Lema, numberwithin=chapter, 
shaded={bgcolor={RGB}{220,220,220}}]{lem}
\declaretheorem[name=Exercício, numberwithin=chapter, 
shaded={bgcolor={RGB}{235,235,235}}]{exo}
\declaretheorem[name=Definição, numberwithin=chapter, 
shaded={bgcolor={RGB}{225,225,225}}]{defin}

\theoremstyle{definition}
\newtheorem{regra}{Regra}

\declaretheoremstyle[
headfont=\bfseries,
notefont=\bfseries,
headpunct={.},
numbered=yes
]{estilo_exemplo}

\declaretheoremstyle[
headfont=\bfseries,
notefont=\bfseries,
headpunct={.},
numbered=yes
]{estilo_obs}

\iflatexml
\else
\theoremstyle{estilo_exemplo}
\fi

\declaretheorem[name=Exemplo, numberwithin=chapter, 
qed={$\diamond$}]{ex}
\declaretheorem[name=Observação, numberwithin=chapter, 
qed={$\bullet$}]{obs}

\newcommand{\detalhes}[1]{
{#1}
}

\newcommand{\citacao}[2]{
\begin{quotation}
 ``\emph{#1}''
\begin{flushright}
#2 
\end{flushright}
\end{quotation}
}


%% DEFINICOES FUNCOES

\newcommand{\pardef}{{:=}}
\newcommand{\sen}{\operatorname{sen}}
\newcommand{\senh}{\operatorname{senh}}
\renewcommand{\cot}{{\rm cotan\,}}

\newcommand{\arcsen}{\operatorname{arcsen}}
\newcommand{\arcos}{\operatorname{arcos}}
\newcommand{\imagem}{\operatorname{Im}}

\newcommand{\li}[2]{\lim_{{#1}\to {#2}}}
\newcommand{\pisobredois}{\tfrac{\pi}{2}}
\newcommand{\pisobretres}{\tfrac{\pi}{3}}
\newcommand{\pisobrequatro}{\tfrac{\pi}{4}}
\newcommand{\pisobreseis}{\tfrac{\pi}{6}}

\newcommand{\fracinfty}{\frac{\infty}{\infty}}

\usepackage{hyperref}

\usepackage{makeidx}
\makeindex



\begin{document}


\frontmatter

\author{Sacha Friedli
\\
\\
Departamento de Matemática
\\
Instituto de Ciências Exatas
\\
Universidade Federal de Minas Gerais
}
\maketitle

\chapter{Licença e créditos}

\copyright\ 2010-2015 Sacha Friedli.
Permitido o uso nos termos da licença CC BY-SA 4.0. 
\\


\noindent
\textbf{Créditos}.
\\
Para fazer atribuição em derivados deste material, incluir a referência:
\\
S. Friedli. Cálculo 1, 2015 (BY-SA).
\\
com um link para: https://github.com/leorolla/friedlicalculo

%\thispagestyle{empty}
%\vspace*{2cm}
%\begin{center}
%   {\Huge\bfseries Cálculo 1}\\[3mm]
%{\large S.  Friedli}\\ 
%{Departamento de Matemática}\\ 
%{Instituto de Ciências Exatas}\\ 
%{Universidade Federal de Minas Gerais}\\
%\vspace*{2cm}
%
%\vspace{10cm}
%\textbf{Versão 1.02}\\ \today\\
%\end{center}



% !TeX spellcheck = pt_BR
% !TEX encoding = UTF-8 Unicode

\chapter{Prefácio}
%\addcontentsline{toc}{chapter}{Prefácio}
%\chapter{Introdução}

Oriundo principalmente do estudo da mecânica e da astronomia, o 
\emph{Cálculo}, 
chamado também \emph{Cálculo infinitesimal}, nasceu no fim do século
XVII, com os trabalhos de Newton~\footnote{Sir Isaac Newton
(Woolsthorpe-by-Colsterworth, 4 de janeiro de 1643 — Londres, 31 de 
março de
1727).} e Leibniz~\footnote{Gottfried Wilhelm von Leibniz (Leipzig, 1 
de julho
de 1646 — Hanôver, 14 de novembro de 1716).}. Hoje em dia, ele é usado 
em todas
as áreas da ciência, e fundamental nas áreas da 
engenharia.\\

A presente apostila contém a ementa da matéria \emph{Cálculo I}, como 
ensinada 
no Departamento de Matemática da UFMG.
Ela tem como objetivo fornecer ao aluno um conhecimento básico dos 
conceitos 
principais do Cálculo que são: limites, derivadas e integral. Ela também 
prepara
o aluno para as outras matérias que usam Cálculo I nos cursos de 
ciências exatas
(física e matemática) e engenharia, 
tais como Cálculo II e III, EDA, EDB, EDC...\\

A apostila começa com um capítulo sobre fundamentos, fazendo uma revisão 
de 
vários conceitos básicos em princípio já conhecidos pelo aluno: equações,
inequações, plano cartesiano e trigonometria. A partir do Capítulo
\ref{Cap:Funcoes}, o conceito de função é introduzido. A noção central de
\emph{limite} é abordada no Capítulo \ref{Cap:Limites}, e a de 
\emph{derivada}
no Capítulo \ref{Cap:Derivacao}. O resto do texto é sobre o objeto 
central desse
curso: a noção de \emph{integral}, o \emph{Teorema Fundamental do 
Cálculo}, e
as suas aplicações.\\

O texto contém bastante exercícios, cuja compreensão é fundamental para a 
assimilação dos conceitos.
As soluções, às vezes detalhadas, se encontram num apêndice.\\

%Essa apostila está em fase de elaboração.
%Qualquer sugestão, crítica ou correção é bem vinda: 
%\verb|sacha@mat.ufmg.br|.\\

Agradeço às seguinte pessoas pelas suas contribuições:
Euller Tergis Santos Borges,
Felipe de Lima Horta Radicchi, 
Fernanda de Castro Maia, 
Hugo Freitas Reis,
Marina Werneck Ragozo,
Mariana Chamon Ladeira Amancio,
Pedro Silveira Gomes de Paiva,
Toufic Mahmed Pottier Lauar, 
Prof. Carlos Maria Carballo,
Prof. Fábio Xavier Penna (UNIRIO),
Prof. Francisco Dutenhefner,
Prof. Hamilton Prado Bueno,
Prof. Jorge Sabatucci,
Profa. Sylvie Oliffson Kamphorst Silva,
Profa. Viviane Ribeiro Tomaz da Silva,
Prof. Viktor Bekkert.\\

Alguns vídeos, criados uma vez para atender a uma classe
online, se encontram em 
\begin{center}
\verb|www.youtube.com/chachf|
\end{center}
Esses vídeos contêm uma boa
parte do conteúdo da presente apostila, mas alguns são de
qualidade baixa e precisam ser regravados....
\\

\hfill
Belo Horizonte, julho de 2015




\iflatexml
\else
\setcounter{tocdepth}{2}
\tableofcontents
\fi


\mainmatter


%\setlength{\parindent}{5pt}
\setlength{\parskip}{0cm}


% !TeX spellcheck = pt_BR
% !TEX encoding = UTF-8 Unicode

\chapter{Fundamentos}
\label{Cap_Fundam}

\citacao{A good course is a course with many stupid 
questions.}{Wendelin Werner, medalhista Fields 2006}

\citacao{Quem faz uma pergunta boba 
fica com vergonha $5$ segundos. Quem não pergunta nada fica 
bobo para sempre...}{Um faxineiro do ICEx, 2008}

\ifdefined\updateans
% Only need to run once in a lifetime, when the file ans.tex needs to be updated.
\Writetofile{ans}{\protect\section*{Capítulo \ref{Cap_Fundam}}}
\fi

\emph{Cálculo} lida com funções de uma ou mais variáveis reais. 
Portanto, ele necessita de uma compreensão boa das principais propriedades dos 
números reais, e suas manipulações na resolução de problemas elementares.\\

Esse capítulo contém lembretes sobre a aritmética elementar dos números reais, 
assim como a descrição de certos conjuntos do plano cartesiano, como retas e
círculos.
\emph{Não pretendemos dar uma exposição completa sobre esses assuntos}, mas 
apenas lembrar alguns fatos e estabelecer notações a respeito de coisas
elementares conhecidas pelo leitor.\\

A matéria desse capítulo será usada constantemente no restante da apostila: é 
importante o leitor verificar que ele consegue fazer todos os exercícios.

\section{Números reais}\index{números! reais $\bR$}
O conjunto dos números reais, $\bR$, pode ser visto como o conjunto dos pontos 
da linha real, que serão em geral denotados por letras minúsculas: $x,y,s,t,u$,
etc.
$\bR$ é munido de quatro operações aritméticas básicas: \grasA{adição} ($+$), 
\grasA{subtração} ($-$), \grasA{multiplicação} ($\times$ ou 
$\cdot$) e \grasA{divisão} ($\div$, ou simplesmente $/$).\\

Lembremos a importância de dois números com papel relevante com respeito à 
adição e multiplicação. Primeiro, o elemento $0$ (``zero'') é tal que $x+0=0+x=x$,
$x\cdot 0=0\cdot x=0$ para todo $x$. 
Um real $x$ diferente de zero será às vezes chamado de \grasA{não-nulo}.\\

Por outro lado, o elemento $1$ (``um'') é tal que $x\cdot 1=1\cdot x=x$ 
para todo $x\in \bR$.
É importante lembrar que \emph{a divisão por zero não é definida}\index{divisão por zero}.
Portanto, 
símbolos do tipo $x/0$ ou $0/0$ não fazem sentido. No entanto, $0/x=0$ para todo
$x\neq 0$.\\

Os subconjuntos de $\bR$ serão em geral denotados usando letras maiúsculas.
Por exemplo, $A=\{0,1,2\}$ é o conjunto que contém os três números reais $0,1$ 
e $2$, e $B=(0,2)$ é o intervalo aberto que contém todos os reais entre $0$ e
$2$ (ver abaixo).
O conjunto dos \grasA{números naturais é}~\footnote{Ao longo da apostila,
o símbolo ``$\pardef$'' será usado para definir um objeto. Por exemplo,
$A\pardef\{x\in \bR: x^2>1\}$ significa que $A$ é \emph{definido} como o conjunto
dos números reais cujo quadrado é maior que $1$.}\index{números! naturais $\bN$}
$$\bN\pardef \{1,2,3,\dots\}\,,$$
e o conjunto dos \grasA{inteiros}\index{números! inteiros $\bZ$} é
$$\bZ\pardef \{\dots,-3,-2,-1,0,1,2,3,\dots\}\,.$$


As operações entre conjuntos são: \grasA{interseção} ($\cap$), \grasA{união} 
($\cup$), \grasA{diferença} ($\setminus$). 
O \grasA{conjunto vazio} será denotado por $\varnothing$.

\subsection{Equações do primeiro e segundo grau}\label{SecEquacoes}
\index{equação! do primeiro grau}
Considere a equação do primeiro grau: 
\eq{\label{eqsimples}
1+4x=-7\,.}
\emph{Resolver} essa equação
significa achar o(s) valor(es) da variável $x$ para os quais a igualdade em \eqref{eqsimples} 
é verdadeira. Esse conjunto de valores será denotado por $S$ e chamado \grasA{conjunto de
soluções}\index{equação! conjunto de soluções}. 
A resolução é bem conhecida:
isolando $x$ obtemos 
uma única solução $x=-2$. Portanto, o conjunto das soluções de \eqref{eqsimples} é $S=\{-2\}$.\\


Considere em seguida a equação do segundo grau\index{equação! do segundo grau}:
\eq{\label{eqsimples2}
x^2=9\,.}
Aqui, sabemos que existem duas soluções, $x=\pm \sqrt{9}=\pm 3$, logo $S=\{+3,-3\}$. \\

Agora, já que um número negativo não possui raiz quadrada, a equação
$$
x^2=-4
$$
não possui nenhuma solução real: $S=\varnothing$. Finalmente, 
$$x^2=0$$
possui uma única solução: $S=\{0\}$.\\

 Um outro jeito de entender \eqref{eqsimples2} é de escrevê-la $x^2-9=0$ e de fatorar o
polinômio $x^2-9$, obtendo um produto de dois fatores:
$$(x-3)(x+3)=0\,.$$
 Para o produto de dois fatores (aqui, $x-3$ e $x+3$) ser zero, é necessário que pelo
menos um deles seja nulo. Se for o primeiro, $x-3=0$, então $x=3$. Se for o segundo,
$x+3=0$, logo $x=-3$. 
De modo geral, para $x$ ser solução de uma equação da forma 
\eq{\label{eq0}(x-\alpha)(x-\beta)=0\,,}
pelo menos um dos fatores, $(x-\alpha)$ ou $(x-\beta)$, deve ser igual a zero, o que 
 implica $x=\alpha$ ou $x=\beta$. Portanto, o conjunto das soluções de \eqref{eq0} é dado
por 
$S=\{\alpha,\beta\}$.\\

Olhemos agora para a equação do segundo grau da forma geral
\eq{\label{eq1}ax^2+bx+c=0\,.}
Se $a=0$, essa equação é do primeiro grau,
$$bx+c=0\,,$$
e a sua única solução é dada por $x=-\frac{c}{b}$ (supondo $b\neq 0$). Isto é,
$S=\{-\frac{c}{b}\}$. Por outro lado, se $a\neq 0$, então dividindo \eqref{eq1} por $a$, e
\emph{completando o quadrado}\index{completar um quadrado} obtemos:
\begin{align*}
0= x^2+\tfrac{b}{a}x+\tfrac{c}{a}&=(x+\tfrac{b}{2a})^2-(\tfrac{b}{2a})^2+\tfrac{c}{a}\,.
\end{align*}
Portanto,
$$(x+\tfrac{b}{2a})^2=(\tfrac{b}{2a})^2-\tfrac{c}{a}=\tfrac{b^2-4ac}{4a^2}\,.$$
Defina $\Delta\pardef b^2-4ac$. Se $\Delta<0$, não tem soluções:
$S=\varnothing$.
Se $\Delta\geq 0$, podemos tomar a raiz quadrada\index{raiz! quadrada} em ambos lados
dessa última expressão, e obter 
$$x+\tfrac{b}{2a}=\pm\tfrac{\sqrt{\Delta}}{2a}\,.$$
Isto é,
\eq{x=\tfrac{-b\pm\sqrt{\Delta}}{2a}\,.}
Resumindo: quando $a\neq 0$, o conjunto das soluções de \eqref{eq1} é dado por
$$S=
\begin{cases}
 \varnothing&\text{ se }\Delta<0\,\text{(zero soluções)}\\
\{\tfrac{-b}{2a}\}&\text{ se }\Delta=0\,\text{(uma solução)}\\
\{\tfrac{-b\pm \sqrt{\Delta}}{2a}\}&\text{ se }\Delta>0\, \text{(duas soluções)}\,.
\end{cases}
$$
\begin{exo}
Resolva as seguintes equações.
\begin{multicols}{3}
\begin{enumerate}
 \item\label{it0} $1-x=1$
 \item\label{it01} $x^2=1$
\item \label{it02} $\frac{1}{x}=x+1$
 \item\label{it1} $(x+1)(x-7)=0$
 \item\label{it2} $x=x$
 \item\label{it20} $x=x^2$
 \item\label{it3} $1=0$
\item \label{it4} $6x^3-1=3x(1+2x^2)$
\item \label{it500} $(x+6)(x+1)=1$ 
\end{enumerate}
\end{multicols}
\vspace{0.01cm}
\begin{sol}
\eqref{it0} $S=\{0\}$ 
\eqref{it01} $S=\{\pm 1\}$ 
 \eqref{it02} Observe primeiro que $0$ não é solução (a divisão por zero no lado esquerdo
não é nem definida). Assim, multiplicando por $x$ e rearranjando obtemos $x^2+x-1=0$. Como
$\Delta=5>0$, obtemos duas soluções: $S=\{\tfrac{-1\pm \sqrt{5}}{2}\}$. (Obs: o número
$\tfrac{-1+\sqrt{5}}{2}=0.618033989...$ é às vezes chamado de \grasA{razão áurea}. Veja 
$\verb|http://pt.wikipedia.org/wiki/Proporção_áurea|$)
 \eqref{it1} Para ter $(x+1)(x-7)=0$, é necessário que pelo menos um dos fatores, $(x+1)$
ou $(x-7)$, seja nulo. Isto é, basta ter $x=-1$ ou $x=7$. Assim, $S=\{-1,7\}$. Obs:
querendo aplicar a fórmula $x=\frac{-b\pm\sqrt{b^2-4ac}}{2a}$ de qualquer jeito, um aluno
com pressa pode querer expandir o produto $(x+1)(x-7)$ para ter $x^2-6x-7=0$, calcular
$\Delta=(-6)^2-4\cdot 1\cdot (-7)=64$, e obter 
$S=\{\frac{-(-6)\pm\sqrt{64}}{2\cdot 1}\}=\{-1,7\}$.
 Mas além de mostrar uma falta de compreensão (pra que expandir uma expressão já
fatorada!?), isso implica aplicar uma fórmula e fazer \emph{contas}, o que cria várias
oportunidades de errar!)
\eqref{it2} $S=\bR$ (qualquer $x$ torna a equação verdadeira!) 
 \eqref{it20} $S=\{0,1\}$
\eqref{it3} $S=\varnothing$
\eqref{it4} $S=\{-\tfrac13\}$
\eqref{it500} $S=\{\frac{-7\pm \sqrt{29}}{2}\}$.
\end{sol}
\end{exo}

\begin{exo}
Mostre que
se $\gamma$ e $\beta$ forem dois números positivos satisfazendo 
\[
2\gamma-\frac{\gamma^2}{2}+\frac{\beta^2}{2}=2\,,
\]
então ou $\gamma+\beta=2$, ou $\gamma-\beta=2$.
\end{exo}

\begin{exo}
Existe um triângulo retângulo de área $7$ e de perímetro $12$?
\begin{sol} Resposta: não.
Sejam $a$ e $b$ os catetos do triângulo. Para ter uma área de $7$, é preciso ter
$\frac{ab}{2}=7$. Para ter um perímetro de $12$, é preciso ter $a+b+\sqrt{a^2+b^2}=12$
(o comprimento da hipotenusa foi calculada com o Teorema de Pitágoras).
Essa última expressão é equivalente a $12-a-b=\sqrt{a^2+b^2}$, isto é (tomando o quadrado
em ambos lados) $144-24(a+b)+2ab=0$. Como $b=\frac{14}{a}$, esta equação se reduz a uma
equação da única incógnita $a$: $6a^2-43a+84=0$. Como essa equação tem $\Delta=-167<0$,
não existe triângulo retângulo com aquelas propriedades.
\end{sol}
\end{exo}


\subsection{Ordem e intervalos}
Existe em $\bR$ uma relação de ordem\index{ordem}: dois 
reais $x,y$ podem ser comparados usando os seguintes símbolos:
\begin{itemize}
\item
$x=y$: ``$x$ é \grasA{igual} a $y$'',
\item
$x\neq y$: ``$x$ é \grasA{diferente} de $y$'',
\item 
$x\geq y$: ``$x$ é \grasA{maior ou igual a} $y$'', 
\item 
$x> y$: ``$x$ é \grasA{estritamente maior} que $y$'', 
\item 
$x\leq y$: ``$x$ é \grasA{menor ou igual a} $y$'',
\item 
$x< y$: ``$x$ é \grasA{estritamente menor} que $y$''.
\end{itemize}

A ordem permite definir subconjuntos elementares de $\bR$. Por exemplo, os \grasA{reais
não-negativos $\bR_+$}\index{números! reais não-negativos $\bR_+$} são definidos por 
$$\bR_+\pardef \{x\in \bR:x\geq 0\}\,,$$
(leia-se: ``o conjunto dos números reais $x\in \bR$ tais que $x$ seja $\geq 0$)
e os \grasA{reais positivos}\index{números! reais positivos $\bR_+^*$} por 
$$\bR_+^*\pardef \{x\in \bR:x> 0\}\,.$$
Podem também ser definidos conjuntos particulares chamados 
\grasA{intervalos}. Começaremos com os intervalos \grasA{limitados}.
Se $a<b$ são dois números reais, o intervalo \grasA{fechado}\index{intervalo! fechado} é
definido como
$$[a,b]\pardef \{x\in \bR:a\leq x\leq b\}\,.$$ 
Leia-se: ``$[a,b]$ é definido como o conjunto dos números reais $x$ tais que $x$ seja
maior ou igual a $a$, e menor ou igual a $b$''.
O intervalo \grasA{aberto}\index{intervalo! aberto} é definido como 
$$(a,b)\pardef \{x\in \bR:a< x< b\}\,.$$ 
Observe que $(a,b)$ pode ser considerado como obtido a partir de $[a,b]$ retirando as
extremidades: $(a,b)=[a,b]\backslash \{a,b\}$.
Definam-se também os intervalos semi-abertos (ou semi-fechados)\index{intervalo!
semi-aberto/fechado}
$$[a,b)\pardef \{x\in \bR:a\leq x< b\}\,,\quad
(a,b]\pardef \{x\in \bR:a< x\leq  b\}\,.
$$
Graficamente, representaremos esses intervalos da seguinte maneira:
\begin{center}
\begin{bmlimage}\begin{tikzpicture}[scale=0.8]
\draw (-8,0)--(8,0);
\draw (8,0) node[right]{$\bR$};
%%%
\draw (-6,0) node[above] {$a$}; \draw (-6,0) node {$\shortmid$}; 
\draw (-4,0) node[above] {$b$}; \draw (-4,0) node {$\shortmid$};
\draw [very thick] (-6,0)--(-4,0);
\filldraw[intfechado] (-6,0) circle (0.75mm);
\filldraw[intaberto] (-4,0) circle (0.75mm);
\draw (-5,0) node[below]{$[a,b)$};
%%%
\draw (-2,0) node[above] {$c$}; 
\draw (-2,0) node {$\shortmid$}; 
\draw (1,0) node[above] {$d$}; 
\draw (1,0) node {$\shortmid$};
\draw [very thick] (-2,0)--(1,0);
\filldraw[intfechado] (-2,0) circle (0.75mm);
\filldraw[intfechado] (1,0) circle (0.75mm);
\draw (-0.5,0) node[below]{$[c,d]$};
%%%
\draw (3.5,0) node[above] {$e$}; \draw (3.5,0) node {$\shortmid$}; 
\draw (6.5,0) node[above] {$f$}; \draw (6.5,0) node {$\shortmid$};
%%CHECKER:
%\draw [very thick, o-)] (3.5,0)--(6.5,0);
\draw [very thick] (3.5,0)--(6.5,0);
\filldraw[intfechado] (6.5,0) circle (0.75mm);
\filldraw[intaberto] (3.5,0) circle (0.75mm);
\draw (5,0) node[below]{$(e,f]$};
\end{tikzpicture}\end{bmlimage}
\end{center}

Introduziremos também intervalos não-limitados: os \grasA{semi-infinitos
fechados}\index{intervalo! semi-infinito}
$$(-\infty,a]\pardef \{x\in \bR:x\leq a\}\,,\quad
[c,+\infty)\pardef \{x\in \bR:x\geq c\}\,,
$$
e os \grasA{semi-infinitos abertos}
$$(-\infty,a)\pardef \{x\in \bR:x< a\}\,,\quad
(c,+\infty)\pardef \{x\in \bR:x> c\}\,.
$$
Por exemplo,
\begin{center}
\begin{bmlimage}\begin{tikzpicture}[scale=0.8]
\draw (-8,0)--(8,0);
\draw (8,0) node[right]{$\bR$};
%%%
\draw (-2,0) node {$\shortmid$}; 
\draw (-2,0) node[above] {$a$}; 
\draw [very thick] (-8,0)--(-2,0);
\fill[intfechado] (-2,0) circle (0.75mm);
\draw (-5,0) node[below]{$(-\infty,a]$};
\draw (-7.5,0) node[above]{$\dots$};
%%%
\draw (1,0) node {$\shortmid$}; 
\draw (1,0) node[above] {$c$}; 
\draw [very thick] (1,0)--(8,0);
\filldraw[intaberto] (1,0) circle (0.75mm);
\draw (5,0) node[below]{$(c,+\infty)$};
\draw (7.5,0) node[above]{$\dots$};
\end{tikzpicture}\end{bmlimage}
\end{center}
 Observe que ``$+\infty$'' e ``$-\infty$'' \emph{não são números reais propriamente
ditos}, $+\infty$ (respectivamente $-\infty$) é somente um símbolo usado para representar
a idéia (meio abstrata) de um número maior (respectivamente menor) do que qualquer real
$x$.

\begin{exo}
Simplifique as expressões, usando as notações introduzidas acima.
\begin{multicols}{2}
\begin{enumerate}
\item $A=\{x\in \bR:x^2\leq 4\}$
\item $B=\{x:x\geq 0\}\cap\{x:x<1\}$ 
\item $C=\{x:x\leq 1\}\cap \{x:x<0\}$ 
\item $D=\{x:x\geq 1\}\cap\{x:x\leq -1\}$ 
\item $E=\{x:x\leq 2\}\cup [0,+\infty)$ 
\item $F=[1,2] \cap (-\infty, 1]$ 
\item $G=[0,1]\cap [0,\tfrac12]\cap [0,\tfrac13]\cap[0,\tfrac14]\cap\dots$
\item $H=[0,1]\cup [1,2]\cup [2,3]\cup[3,4]\cup\dots$
\end{enumerate}
\end{multicols}
\vspace{0.01cm}
\begin{sol}
$A=[-2,2]$, $B=[0,1)$, $C=(-\infty,0)$, $D=\varnothing$, $E=\bR$, $F=\{1\}$, $G=\{0\}$,
$H=\bR_+$.
\end{sol}
\end{exo}


\subsection{Inequações e sinal}\index{inequações}
Considere a inequação do primeiro grau:
\eq{\label{inequ00}2-2x\geq 1\,.}
Como antes, ``resolver'' essa inequação significa achar todos os valores de 
$x$ para os quais a expressão em \eqref{inequ00} se torne verdadeira.
Por exemplo, $x=0$ é solução, pois o lado esquerdo vale $2-2\cdot 0=2$, que é $\geq 1$.
Mas em geral uma inequação pode possuir mais de uma solução, às vezes possui 
infinitas soluções.
O conjunto de todas as soluções, também denotado por $S$, pode ser calculado 
da seguinte maneira. Primeiro, \emph{o conjunto $S$ das soluções não é
modificado ao adicionarmos (ou subtrairmos)
 expressões iguais em ambos lados de uma inequação}. Assim, adicionando $2x$ em cada lado
de \eqref{inequ00}, obtemos
$$2\geq 1+2x\,.$$
Podemos em seguida subtrair $1$ em ambos lados:
$$1\geq 2x\,.$$
Agora, \emph{o conjunto $S$ das soluções não é modificado ao multiplicarmos 
(ou dividirmos) ambos lados de uma inequação por um número {positivo}}.
Assim, dividindo ambos lados da inequação $1\geq 2x$ por $2$ obtemos 
$\tfrac12\geq x$, isto é $x\leq \tfrac12$. Assim,
qualquer real $x$ menor ou igual a $\tfrac12$ torna a desigualdade em 
\eqref{inequ00} verdadeira. Logo, $S=(-\infty,\tfrac12]$.\\

Observe que 
\eqref{inequ00} pode também ser resolvida subtraindo $2$ em ambos lados,
\eq{\label{desigcontra}
-2x\geq -1\,.}
 Passando $-2x$ para o lado direito e $-1$ para o lado esquerdo obtemos $1\geq 2x$, o que
equivale a 
\eq{\label{desigcontrb}2x\leq 1\,.}
 Vemos que \eqref{desigcontrb} é obtida a partir de \eqref{desigcontra} \emph{trocando os
sinais} (i.é. multiplicando ambos lados por $-1$), e \emph{trocando o sentido da
desigualdade}.

\begin{ex} Resolvamos agora uma inequação do segundo grau:
\eq{\label{inequ0}x^2-3x+2> 0\,.}
Primeiro, o polinômio do lado esquerdo da desigualdade em \eqref{inequ0} pode ser
fatorado\index{fatoração de polinômio}: 
$x^2-3x+2=(x-1)(x-2)$. Assim, \eqref{inequ0} é equivalente a
\eq{\label{inequ0b}(x-1)(x-2)> 0\,.}
 Observe agora que para o produto de dois números ser $> 0$, eles têm que ser ambos
não-nulos e ter o mesmo sinal. Portanto, a resolução de \eqref{inequ0b} passa pelo estudo
do sinal de $x-1$ e $x-2$. Isso pode ser feito como em \eqref{inequ00}.
Por um lado, $x-1<0$ se $x<1$, $x-1=0$ se $x=1$, e $x-1>0$ se $x>1$. 
Por outro lado, $x-2<0$ se $x<2$, $x-2=0$ se $x=2$, e $x-2>0$ se $x>2$. 
Isso pode ser resumido nas duas primeiras linhas da seguinte tabela:
\begin{center}
\begin{bmlimage}\begin{tikzpicture}
\tkzTabInit[lgt=3, nocadre, espcl=2]
{ /.6,  $x-1$ /.6, $x-2$ /.6, $(x-1)(x-2)$ /.8}%
{,$1$, $2$,}%
%\tkzTabLine{,+,z,+,,+,}
\tkzTabLine{,-,z,+,t,+,}
\tkzTabLine{,-,t,-,z,+,}
\tkzTabLine{,+,z,-,z,+,}
%\tkzTabVar{-/,+/\text{a.v.},-/$0$,+/,}
%\tkzTabLine{,\searrow,\text{mín.},h,\text{mín.},\nearrow,}
\end{tikzpicture}\end{bmlimage}
\end{center}
A terceira linha foi obtida multiplicando os sinais de $x-1$ e $x-2$: 
$(x-1)(x-2)>0$ se $x<1$, $(x-1)(x-2)=0$ se $x=1$,
$(x-1)(x-2)<0$ se $1<x<2$, $(x-1)(x-2)=0$ se $x=2$, e $(x-1)(x-2)>0$ se $x>2$.
Assim, $S=(-\infty,1)\cup (2,+\infty)$ dá todas as soluções de \eqref{inequ0}.
\end{ex}

\begin{exo}
Resolva as seguintes inequações.
\begin{multicols}{3}
\begin{enumerate}
 \item\label{itinequ1} $x>4-5$
 \item\label{itinequ2} $3x\leq x+1$
 \item\label{itinequ3} $-8x<3-4x$
 \item\label{itinequ4} $10>10-x$
 \item\label{itinequ5} $x^2\geq 1$
 \item\label{itinequ6} $-x^2>1+2x$
 \item\label{itinequ7} $x>x$
 \item\label{itinequ8} $x\geq x$
 \item\label{itinequ9} $x\leq x^2$
\item\label{itinequ10} $-2x^2+10x-12<0$
\item\label{itinequ10b} $x^2(x+7)\leq 0$
\item\label{itinequ11} $x^3-2x^2-x+2>0$
\item\label{itinequ12} $x^2-x(x+3)\leq 0$
\item\label{itinequ13} $x\leq \frac{x+3}{x-1}$
\item\label{itt6} $\frac{1}{x}-\frac{1}{x+2}\geq 0$
\item\label{itt7} $\frac{1}{x}+\frac{2}{2-x}<1$
\item\label{itt7a} $\frac{1-x}{2+x}\leq -\frac{2}{3x-4}$
\end{enumerate}
%PROVA: $(x^2-x-2)(2x-2)\geq 0$
\end{multicols}
\vspace{0.01cm}
\begin{sol}
 \eqref{itinequ1} $(-1,\infty)$
 \eqref{itinequ2} $(-\infty,\tfrac12]$
 \eqref{itinequ3} $(-\tfrac34,\infty)$
 \eqref{itinequ4} $(0,\infty)$
 \eqref{itinequ5} $(-\infty,-1]\cup [1,\infty)$
 \eqref{itinequ6} $\varnothing$
 \eqref{itinequ7} $\varnothing$
 \eqref{itinequ8} $\bR$
 \eqref{itinequ9} $(-\infty,0]\cup [1,\infty)$ Obs: aqui, um erro comum é de começar
dividindo ambos lados de $x\leq x^2$ por $x$, o que dá $1\leq x$. Isso dá somente uma
parte do conjunto das soluções, $[1,\infty)$, porque ao dividir por $x$, é preciso
considerar também os casos em que $x$ é negativo. Se $x$ é negativo, dividir por $x$ dá
$1\geq x$ (invertemos o sentido da desigualdade), o que fornece o outro pedaço das
soluções: $(-\infty,0]$.
 \eqref{itinequ10} $(-\infty,2)\cup (3,\infty)$
\eqref{itinequ10b} $(-\infty,-7]\cup \{0\}$
 \eqref{itinequ11} $(-1,+1)\cup (2,+\infty)$
\eqref{itinequ12} $[0,+\infty[$
\eqref{itinequ13} $S=(-\infty,-1]\cup (1,3]$. Cuidado: tem que excluir o valor
$x=1$ para evitar a divisão por zero\index{divisão por zero} e a inequação ser bem
definida. 
\eqref{itt6} Primeiro observemos que os
valores $x=0$ e $x=-2$ são proibidos. Em seguida, colocando no mesmo denominador,
queremos resolver $\frac{2}{x(x+2)}\geq 0$. Isso é equivalente a resolver $x(x+2)\geq 0$,
cujo conjunto de soluções é dado por $(-\infty,-2]\cup [0,\infty)$. Logo,
$S=(-\infty,-2)\cup (0,\infty)$ (tiramos os dois valores proibidos).
\eqref{itt7} $S=(-\infty,0)\cup(2,\infty)$.
\eqref{itt7a} $S=(-\infty,-2]\cup [0,\tfrac43]\cup [3,+\infty)$.
\end{sol}
\end{exo}

\begin{exo}
Quantos números inteiros $n$ existem tais que $3n-1\leq 5n-2<4$?
\begin{sol}
Um só: $n=1$.
\end{sol}
\end{exo}

\begin{exo}
Quantos números primos $p$ existem tais que $0\leq 2p-3\leq p+8$?
\begin{sol}
Resolvendo $0\leq 2x-3$ obtemos $S_1=[\tfrac32,\infty)$, e resolvendo $2x-3\leq
x+8$ obtemos $S_2=(-\infty,11]$. Logo, $S=S_1\cap S_2=[\tfrac32,11]$ é solução
das duas inequações no mesmo tempo. Mas esse intervalo contém os primos
$p=2,3,5,7,11$. Logo, a resposta é: $5$. 
\end{sol}
\end{exo}

\subsection{Valor absoluto}

Informalmente, o \grasA{valor absoluto} de um número real $x$, denotado por $|x|$,
representa o 
seu ``valor equivalente positivo''. Por exemplo, $|5|=5$, $|-3|=3$, e $|0|=0$.
Formalmente,\index{valor absoluto}

\eq{\label{eq:defvalorabs}
\boxed{|x|\pardef 
\begin{cases}
x&\text{ se }x> 0\,\\
0&\text{ se }x= 0\,\\
-x&\text{ se }x<0\,.
\end{cases}
}
}

Por exemplo, com essa definição, já que $-3<0$, temos $|-3|=-(-3)=3$. 
Observe que para qualquer número $a\geq 0$,
\begin{equation}\label{eq:consequvalabsol}
|x|\leq a\Longleftrightarrow -a\leq x\leq a\,.
\end{equation}
De fato, suponha primeiro que $x\geq 0$. Entao $|x|=x$, e $|x|\leq a$
é equivalente a $x\leq a$. Por outro lado, se $x\leq 0$, então $|x|=-x$, e 
$|x|\leq a$ é equivalente a $-x\leq a$, isto é a $-a\leq x$.
Juntando os dois casos, isso mostra que $|x|\leq a$ é equivalente a $-a\leq
x\leq a$. 

\begin{exo}\label{Exo:valorabscorreto}
Quais das expressões abaixo são verdadeiras (para qualquer $x$)?
Justifique.  
$$
\sqrt{x^2}=x\,,\quad \sqrt{x}^2=x\,,\quad\sqrt{x^2}=|x|\,.
$$
\begin{sol}
A expressão correta é a terceira, e vale para qualquer $x\in \bR$.
A primeira está certa quando $x\geq 0$, mas errada quando $x<0$ (por exemplo, 
$\sqrt{(-3)^2}=\sqrt{9}=3(\neq -3)$). A segunda também está certa quando $x\geq
0$, mas $\sqrt{x}$ não é nem definido quando $x<0$.
\end{sol}
\end{exo}
O valor absoluto para definir 
a \grasA{distância} entre dois números reais:
\begin{center}
\begin{bmlimage}\begin{tikzpicture}
\draw[->] (-5,0)--(5,0);
\draw (-3.5,0) node[above]{$x$};
\draw (-3.5,0) node{$\shortmid$};
\draw (1,0) node[above]{$y$};
\draw (1,0) node{$\shortmid$};
\draw (2.2,0.5) node[above right]{$d(x,y)\pardef |x-y|$};
\end{tikzpicture}\end{bmlimage}
\end{center}
De fato, se $x\leq y$, a distância é igual a $y-x=-(x-y)\equiv |x-y|$, e se
$x>y$ a distância é $x-y\equiv |x-y|$.\\

Podemos também resolver inequações que envolvem valores absolutos:
\begin{ex}\label{Ex:inequmodulo}
Resolvamos
\eq{\label{inequ2}|x-2|\geq 3\,.}
Sabemos que pela definição do valor absoluto,\index{inequações! com valores absolutos}
$$
|x-2|=
\begin{cases}
 x-2&\text{ se }x\geq 2\,,\\
-x+2&\text{ se }x< 2\,,
\end{cases}
$$
 Logo, a resolução de \eqref{inequ2} passa pela resolução de \emph{duas} inequações mais
simples. A primeira é
$$ x-2\geq 3\,, \text{ isto é } x\geq 5\,,$$
e deve ser considerada somente para os $x$ tais que $x\geq 2$. Isso 
dá um primeiro conjunto de soluções: $S_1=[5,+\infty)$ (os reais que são ao mesmo tempo
maiores ou iguais a  $5$ e maiores ou iguais a $2$). A segunda é
$$ -x+2\geq 3\,, \text{ isto é } x\leq -1\,,$$
 e deve ser considerada somente para os $x$ tais que $x\leq 2$, o que dá um segundo
conjunto de soluções $S_2=(-\infty,-1]$. Assim, o conjunto de todas as soluções de
\eqref{inequ2} é dado por $S=S_1\cup S_2$: $S=(-\infty,-1]\cup [5,+\infty)$.\\

Um jeito mais geométrico (mas equivalente) de resolver o problema é de escrever 
\eqref{inequ2} como: $d(x,2)\geq 3$. Assim, podemos
interpretar as soluções de \eqref{inequ2} como sendo os reais $x$ cuja distância ao ponto
$2$ é maior ou igual a $3$, que são todos os reais a {esquerda} de 
$-1$ ou a direita de $5$: $S=(-\infty,-1]\cup [5,+\infty)$.
\end{ex}

\begin{exo}
Resolva as seguintes inequações.
\begin{multicols}{3}
\begin{enumerate}
\item\label{itt1} $|x+27|\geq 0$
\item\label{itt2} $|x-2|<0$
\item\label{itt3} $|2x+3|>0$
\item\label{itt4} $3<|3-x|$
\item\label{itt4bis} $2x-3|x|-4\geq 0$
\item\label{itt5} $|x^2-1|\leq 1$
\item\label{itt7} $\frac{x}{|x-2|}>2$.
\end{enumerate}
\end{multicols}
\vspace{0.01cm}
\begin{sol}
\eqref{itt1} Observe que como um valor absoluto é sempre $\geq 0$, 
qualquer $x$ é solução de $|x+27|\geq 0$. Logo, $S=\bR$. \eqref{itt2} Como no
item anterior, $|x-2|\geq 0$ para qualquer $x$. Logo, não tem nenhum $x$ tal que
$|x-2|<0$, o que implica $S=\varnothing$. \eqref{itt3} Para ter $|2x+3|>0$, a
única possibilidade é de excluir $|2x+3|=0$. Como isso acontece se e somente se
$2x+3=0$, isto é se e somente se $x=-\tfrac32$, temos $S=\bR\setminus
\{-\tfrac32\}=(-\infty,-\tfrac32)\cup(-\tfrac32,+\infty)$.
\eqref{itt4} Considere primeiro o caso em que $3-x\geq 0$ (isto é $x\leq 3$). A inequação
se torna $3<3-x$, isto é $x<0$. Logo, $S_1=(-\infty,0)$. No caso em que $3-x\leq 0$ (isto
é $x\geq 3$), a inequação se torna $3<-(3-x)$, isto é $x>6$. Assim, $S_2=(6,+\infty)$.
Finalmente, $S=S_1\cup S_2=(-\infty,0)\cup ]6,+\infty)$.
\eqref{itt4bis} $S=\varnothing$
\eqref{itt5} $S=[-\sqrt{2},\sqrt{2}]$. Observe que por
\eqref{eq:consequvalabsol}, $|x^2-1|\leq 1$ se e somente se
$-1\leq x^2-1\leq 1$. Assim, resolvendo separadamente as inequações $-1\leq x^2-1$ e
$x^2-1\leq 1$ leva ao mesmo conjunto de soluções. 
\eqref{itt8} $S=(\tfrac43,2)\cup (2,4)$.
\end{sol}
\end{exo}

\grasA{Estudar o sinal de uma expressão} que depende de uma variável $x$ significa
determinar os valores de $x$ para os quais a expressão é positiva, negativa, ou
nula.

\begin{ex}
Estudemos o sinal da expressão $x^3+3x^2$.
Como $x^3+3x^2=x^2(x+3)$, o sinal da expressão inteira é obtido a partir dos sinais das
partes $x^2$ e $x+3$.
\begin{center}
\begin{bmlimage}\begin{tikzpicture}
\tkzTabInit[lgt=3, nocadre, espcl=2]
{ /.6,  $x^2$ /.6, $x+3$ /.6, $x^2(x+3)$ /.8}%
{,$-3$, $0$,}%
%\tkzTabLine{,+,z,+,,+,}
\tkzTabLine{,+,t,+,z,+,}
\tkzTabLine{,-,z,+,t,+,}
\tkzTabLine{,-,z,+,z,+,}
%\tkzTabVar{-/,+/\text{a.v.},-/$0$,+/,}
%\tkzTabLine{,\searrow,\text{mín.},h,\text{mín.},\nearrow,}
\end{tikzpicture}\end{bmlimage}
\end{center}
Assim vemos que $x^3+3x^2$ é $>0$ (\emph{estritamente positiva}) se $x\in (-3,0)\cup
(0,\infty)$, ela é $<0$ (\emph{estritamente negativa}) se $x<0$, e é $=0$ (\emph{nula})
se $x\in \{-3,0\}$.
\end{ex}

Mais tarde resolveremos inequações onde aparecem, e estudaremos o sinal de outras
expressões, como funções trigonométricas, raízes ou logaritmos.

\begin{exo}
Estude o sinal das seguintes expressões
\begin{multicols}{3}
\begin{enumerate}
\item\label{itexsinal1} $5+x$
\item\label{itexsinal2} $5+x^2$
\item\label{itexsinal21} $(x-5)^2$
\item\label{itexsinal3} $x^2-5$
\item\label{itexsinal4} $\frac{x^2+2x-48}{2-x}$
\item\label{itexsinal5} $(x+1)|2x-1-x^2|$
\end{enumerate}
\end{multicols}
\vspace{0.01cm}
\begin{sol}
\eqref{itexsinal1} $<0$ se $x<-5$, $>0$ se $x>-5$, nula se $x=-5$.
\eqref{itexsinal2} $>0$ para todo $x\in \bR$.
\eqref{itexsinal21} $>0$ se $x\in\bR\setminus \{5\}$, nula se $x=5$.
\eqref{itexsinal3} $>0$ se $x\in (-\infty,-\sqrt{5})\cup (\sqrt{5},\infty)$, $<0$ se
$x\in (-\sqrt{5},\sqrt{5})$, nula se $x=\pm \sqrt{5}$
\eqref{itexsinal4} $>0$ se $x\in (-\infty,-8)\cup (2,6)$, $<0$ se $x\in (-8,2)\cup
(6,\infty)$, nula se $x\in \{-8,6\}$. Observe que a expressão \emph{não é definida em
$x=2$}.
\eqref{itexsinal5} $>0$ se $x\in (-1,1)\cup(1,\infty)$, $<0$ se $x<-1$, nula se $x\in
\{-1,1\}$.
\end{sol}
\end{exo}


\section{O plano cartesiano}\index{plano Cartesiano}
O plano cartesiano, em geral denotado por $\bR^2$, é o conjunto dos pares $P=(x,y)$ de
reais,
$x$ e $y$, chamados respectivamente de \grasA{abscissa (ou primeira
coordenada)}\index{abcissa} e \grasA{ordenada (ou segunda coordenada)}\index{ordenada}. 

\begin{center}
\begin{bmlimage}\begin{tikzpicture}
\pgfmathsetmacro{\a}{2};
\pgfmathsetmacro{\b}{1};
\coordinate (P) at (\a,\b);
\draw [dotted] (0,\b)--(P)--(\a,0);
\fill (P) circle (0.45mm);
\draw (P) node[right]{$P=(x,y)$};
\draw [ ->] (0,-0.4)--(0,2);
\draw [ ->] (-0.2,0)--(4,0);
\draw (\a,0) node{$\shortmid$};
\draw (\a,0) node[below]{$x$};
\draw (0,\b) node{$-$};
\draw (0,\b) node[left]{$y$};
\end{tikzpicture}\end{bmlimage}
\end{center}

%\lipsum[1-10]
O conjunto dos pontos cuja primeira coordenada é nula, isto é, o conjunto dos pontos da
forma $P=(0,y)$, é chamado de \grasA{eixo $y$}, ou \grasA{eixo das ordenadas}. 
O conjunto dos pontos cuja segunda coordenada é nula, isto é, o conjunto dos pontos da
forma $P=(x,0)$, é chamado de \grasA{eixo $x$}, ou \grasA{eixo das abscissas}.
Os eixos $x$ e $y$ formam duas retas perpendiculares, e dividem o plano em quatro
\grasA{quadrantes}\index{quadrante}:
\begin{center}
\begin{bmlimage}\begin{tikzpicture}
\pgfmathsetmacro{\a}{1.5};
\pgfmathsetmacro{\b}{0.8};
\draw [ ->] (0,-\a)--(0,\a);
\draw [ ->] (-\a,0)--(\a,0);
\draw (\b,\b) node{$1^o$};
\draw (-\b,\b) node{$2^o$};
\draw (-\b,-\b) node{$3^o$};
\draw (\b,-\b) node{$4^o$};
\end{tikzpicture}\end{bmlimage}
\end{center}
Mais explicitamente, em termos das coordenadas, 
\begin{multicols}{2}
\begin{itemize}
\item $1^o=\{(x,y):x\geq 0, y\geq 0\}$,
\item $2^o=\{(x,y):x\leq 0, y\geq 0\}$,
\item $3^o=\{(x,y):x\leq 0, y\leq 0\}$,
\item $4^o=\{(x,y):x\geq 0, y\leq 0\}$.
\end{itemize}
\end{multicols}

Se $P=(x,y)$ e $Q=(x',y')$, a \grasA{distância Cartesiana}\index{distância Euclidiana}
entre $P$ e $Q$ é calculada usando o Teorema de Pitágoras:
\begin{center}
\begin{bmlimage}\begin{tikzpicture}
\pgfmathsetmacro{\a}{1};
\pgfmathsetmacro{\b}{2.7};
\pgfmathsetmacro{\c}{4.8};
\pgfmathsetmacro{\d}{1};
\coordinate (P) at (\a,\b);
\coordinate (Q) at (\c,\d);
\draw [dotted] (0,\b)--(P)--(\a,0);
\draw [dotted] (0,\d)--(Q)--(\c,0);
\draw [dashed, <->] (P)--(Q) node[midway, above, sloped]{$d(P,Q)$};
\draw [thin, <->] (\a-0.2,\b)--(\a-0.2,\d) node[midway, below, sloped]{\tiny{$|y-y'|$}};
\draw [thin, <->] (\a,\d-0.2)--(\c,\d-0.2) node[midway, below,
sloped]{\tiny{$|x-x'|$}};
\fill (P) circle (0.45mm);
\fill (Q) circle (0.45mm);
\draw (P) node[above]{$P$};
\draw (Q) node[right]{$Q$};
\draw [ ->] (0,-0.2)--(0,3.5);
\draw [ ->] (-0.2,0)--(6,0);
\draw (6,1.5) node[right]{$d(P,Q)\pardef \sqrt{(x-x')^2+(y-y')^2}\,.$};
\end{tikzpicture}\end{bmlimage}
\end{center}

\begin{exo}\label{Exo:subconjplano}
Descreva os seguintes subconjuntos do plano em termos das suas coordenadas cartesianas.
\begin{enumerate}
\item\label{itplano1}  Semi-plano acima do eixo $x$,
\item\label{itplano2}  semi-plano a esquerda do eixo $y$,
\item\label{itplano3}  quadrado de lado $1$ centrado na origem (com os lados
paralelos aos eixos),
\item\label{itplano4}  reta vertical passando pelo ponto $(2,0)$, 
\item\label{itplano5}  reta horizontal passando pelo ponto $(-3,-5)$, 
\item\label{itplano6}  reta horizontal passando pelo ponto $(13,-5)$,
 \item\label{itplano7}  faixa vertical contida entre o eixo $y$ e a reta do item
\eqref{itplano4},
\item\label{itplano8}  círculo de raio $1$ centrado na origem. 
\item\label{itplano9}  disco (cheio) de raio $2$ centrado em $(1,-2)$.
\end{enumerate}
\begin{sol}
\eqref{itplano1} $\{(x,y):y> 0\}$,
\eqref{itplano2} $\{(x,y):x< 0\}$,
\eqref{itplano3} $\{(x,y):|x|\leq \half, |y|\leq \half\}$,
\eqref{itplano4} $\{(x,y): x=2\}$,
\eqref{itplano5} $\{(x,y): y=-5\}$,
\eqref{itplano6} $\{(x,y): y=-5\}$,
\eqref{itplano7} $\{(x,y): 0\leq x\leq 2\}$,
\eqref{itplano8} $\{P=(x,y): d(P,(0,0))=1\}=\{(x,y):x^2+y^2=1\}$,
\eqref{itplano9} $\{P=(x,y): d(P,(1,-2))\leq 2\}=\{(x,y):(x-1)^2+(y+2)^2\leq 4\}$,
\end{sol}
\end{exo}

\subsection{Retas}\index{reta}
Já vimos, no Exercício \ref{Exo:subconjplano}, como expressar retas horizontais e 
verticais. Uma reta \emph{vertical} é o conjunto formado pelos pontos $(x,y)$ cuja
primeira coordenada $x$ é igual a um número fixo $a\in \bR$, a sua \grasA{equação} se
escreve: $x=a$.
\begin{center}
\begin{bmlimage}\begin{tikzpicture}[scale=1.3]
\draw [ ->] (0,-1.2)--(0,1.2) node[left]{$y$};
\draw [ ->] (-0.5,0)--(2,0) node[right]{$x$};
\draw [very thick] (1,-1)--(1,1);
\draw [dotted,  <-] (1.1,-0.1)--(1.5,-0.3) node[below right]{$(a,0)$};
\draw [dotted,  <-] (1.1,0.8)--(2,0.8) node[right]{equação da reta: $x=a$};
\fill (1,0) circle (0.35mm);
\end{tikzpicture}\end{bmlimage}
\end{center}
 Por outro lado, uma reta \emph{horizontal} é o conjunto formado pelos pontos $(x,y)$ cuja
segunda coordenada $y$ é igual a um número fixo $b\in \bR$, a sua \grasA{equação} se
escreve: $y=b$.
\begin{center}
\begin{bmlimage}\begin{tikzpicture}[scale=1.3]
\draw [ ->] (0,-0.5)--(0,1.2) node[left]{$y$};
\draw [ ->] (-0.5,0)--(2,0) node[right]{$x$};
\draw [very thick] (-0.5,0.7)--(2,0.7);
\draw [dotted,  <-] (-0.1,0.6)--(-1,0.2) node[left]{$(0,b)$};
\draw [dotted,  <-] (2.1,0.7)--(3,0.7) node[right]{equação da reta: $y=b$};
\fill (0,0.7) circle (0.35mm);
\end{tikzpicture}\end{bmlimage}
\end{center}

 As retas horizontais e verticais são descritas por somente \emph{um} parâmetro (o ``$a$''
para uma reta vertical, ou o ``$b$'' para uma reta horizontal). Para as outras retas do
plano, que não ficam necessariamente paralelas a um dos eixos, é preciso usar 
\emph{dois} parâmetros, $m$ e $h$, chamados respectivamente
\grasA{inclinação}\index{inclinação} (ou \grasA{coeficiente angular}\index{coeficiente
angular}) e \grasA{ordenada na origem}\index{ordenada! na origem}, para
especificar a dependência entre $x$ e $y$\index{equação! de reta}:
$$y=mx+h\,.$$
\begin{center}
\begin{bmlimage}\begin{tikzpicture}[scale=1.5]
\draw [ ->] (0,-0.1)--(0,1.5) node[left]{$y$};
\draw [ ->] (-0.5,0)--(2,0) node[right]{$x$};
\draw [very thick] (-0.5,0.5)--(2,1.5);
\draw [decorate, decoration=brace] (0.05,0.7)--(0.05,0) 
node[midway, right]{ordenada na origem: $h$};
\draw [dotted,  <-] (1.7,1.3)--(2.5,1) node[right]{equação da reta: $y=mx+h$};
\draw [dotted,  <-] (-0.4,0.6)--(-0.8,0.8) node[left]{inclinação: $m$};
\end{tikzpicture}\end{bmlimage}
\end{center}
 O significado da inclinação $m$ deve ser entendido da seguinte maneira: partindo de um
ponto qualquer da reta, ao andar horizontalmente uma distância $L$ para a direita, o
deslocamento vertical da reta é de $mL$. Por exemplo, para uma reta de inclinação
$\frac12$ (observe que todo os triângulos da seguinte figura são semelhantes),
\begin{center}
\begin{bmlimage}\begin{tikzpicture}[scale=1.5]
\draw [very thick] (0,0)--(4,2);
\draw [dashed] (0,0)--(4,0) node[midway, below]{$L$};
\draw [dashed] (4,0)--(4,2) node[midway, right]{$L/2$};
\draw [dashed] (0.8,0.4)--(1.8,0.4) node[midway, below]{$1$};
\draw [dashed] (1.8,0.4)--(1.8,0.9) node[midway, right]{$0.5$};
\draw [dashed] (2.6,1.3)--(3.2,1.3) node[midway, below]{$0.6$};
\draw [dashed] (3.2,1.3)--(3.2,1.6) node[midway, right]{$0.3$};
\end{tikzpicture}\end{bmlimage}
\end{center}
Se a inclinação é negativa, então o deslocamento vertical é para baixo.\\

 Se $P=(x_1,y_1)$ e $Q=(x_2,y_2)$ 
 são dois pontos de uma reta não vertical de inclinação $m$, então
\eq{\label{eq:exprinclin}
\frac{y_2-y_1}{x_2-x_1}=m\,.}
Essa relação pode ser usada também para \emph{calcular} a inclinação de uma reta. 

\begin{ex}\label{exemploreta}
 Procuremos a equação da reta $r$ que passa pelos pontos $P=(-1,3)$ e $Q=(3,0)$:
\begin{center}
\begin{bmlimage}\begin{tikzpicture}[scale=0.8]
\draw [ ->] (0,-0.1)--(0,3.2) node[right]{$y$};
\draw [ ->] (-1.5,0)--(3.5,0) node[right]{$x$};
\pgfmathsetmacro{\a}{-1.2}
\pgfmathsetmacro{\b}{3.2}
\draw [dotted] (\a,{(-3*\a)/4+9/4})--(\b,{(-3*\b)/4+9/4});
\draw (-1,3) node[below left]{$P$};
\fill (-1,3) circle (0.35mm);
\draw (3,0) node[above]{$Q$};
\fill (3,0) circle (0.35mm);
\foreach \k in {-1,...,3}
{\draw ({\k},0) node{$\shortmid$};}
\foreach \k in {0,...,3}
{\draw (0,{\k}) node{$-$};}
\end{tikzpicture}\end{bmlimage}
\end{center}
Como $r$ não é vertical, a sua equação é da forma $y=mx+h$. A 
inclinação pode ser
calculada usando \eqref{eq:exprinclin}:
$m=\frac{0-(3)}{3-(-1)}=-\frac{3}{4}$.
 (Pode também observar que para andar de $P$ até $Q$, é necessário andar 
$4$ passos para a
direita, e $3$ passos para baixo, logo $m=\tfrac{-3}{4}$.)
 Portanto, a equação é da forma $y=-\tfrac{3}{4}x+h$. Falta achar $h$, 
que pode ser
calculado usando o fato de $r$ passar pelo ponto $P$: 
$3=-\tfrac{3}{4}\cdot (-1)+h$ (daria
na mesma usando o ponto $Q$). 
Assim, $h=\tfrac{9}{4}$, e $r$ é descrita pela equação: $$y=-\tfrac{3}{4}x+\tfrac94\,.$$
 Ao multiplicarmos ambos lados por $4$ e rearranjando podemos a equação da reta
 da seguinte maneira:
$$
3x+4y-9=0\,.
$$
Essa é a \grasA{forma genérica} da reta. 
Em geral, qualquer reta pode ser
descrita na forma  générica,
\[
ax+by+c=0\,,
\]
em que $a,b,c$ são constantes. Se $a=0$ e $b\neq 0$, a reta é horizontal. Se
$a\neq 0$ e $b=0$, a reta é vertical. Se $a\neq 0$ e $b\neq 0$, a reta é oblíqua.

\end{ex}

\begin{exo}
Considere a reta $r$ do Exemplo \ref{exemploreta}.
Escolha alguns pares de pontos $P$ e $Q$ em $r$, 
e verifique a fórmula \eqref{eq:exprinclin}.
Ache os valores de $x$ e $y$ para que os pontos $R=(x,100)$ e $T=(6,y)$ pertençam a $r$.
\begin{sol}
$R=(-\frac{391}{3},100)$, $T=(6,-\frac{9}{4})$.
\end{sol}
\end{exo}

\begin{exo}
Determine a equação da reta que passa pelos pontos dados.
\begin{multicols}{2}
\begin{enumerate}
\item\label{itreta1} $(0,0)$, $(1,1)$
\item\label{itreta2} $(-2,1)$, $(100,1)$
\item\label{itreta3} $(-3,-21.57)$, $(-3,3)$ 
\item\label{itreta4} $(1,-2)$, $(-1,3)$ 
\item\label{itreta5} $(333,227)$, $(-402,-263)$ 
\end{enumerate}
\end{multicols}
\vspace{0.01cm}
\begin{sol}
\eqref{itreta1} $y=x$,
\eqref{itreta2} $y=1$,
\eqref{itreta3} $x=-3$,
\eqref{itreta4} $y=-\tfrac{5}{2}x+\tfrac12$,
\eqref{itreta5} $y=\tfrac{2}{3}x+5$.
\end{sol}
\end{exo}

\begin{exo}\label{ExoEsbocoretas}
Faça um esboço, no plano cartesiano, da reta descrita pela equação dada.
\begin{multicols}{3}
\begin{enumerate}
\item\label{itretta1} $r_1:\,x=4$
 \item\label{itretta2} $r_2:\,y=-3/2$
\item \label{itrett3} $r_3:\,x+2y=0$
\item \label{itrett4} $r_4:\,y=2x-3$
\end{enumerate}
\end{multicols}
\vspace{0.01cm}
\begin{sol}

\mbox{}
\begin{center}
\begin{bmlimage}\begin{tikzpicture}[scale=0.6]
\draw [ ->] (0,-3.3)--(0,2.4) node[left]{$y$};
\draw [ ->] (-5,0)--(5.4,0) node[right]{$x$};
\foreach \k in {-5,...,5}
{\draw ({\k},0) node{$\shortmid$};}
\foreach \k in {-3,...,2}
{\draw (0,{\k}) node{$-$};}
%%%%%%%%
\draw [very thick] (4,-2.5)--(4,2) node[right]{$r_1$};
%%%%%%%%%%%%
%\draw (4.2,0.2) node{$4$};
\draw [very thick] (4.8,-1.5)--(-4.5,-1.5) node[left]{$r_2$};
%\draw (0,-1.5) node[above left]{$-\tfrac{3}{2}$};
%%%%%%%%%%%%
%\draw (4.2,0.2) node{$4$};
\pgfmathsetmacro{\a}{-0.1}
\pgfmathsetmacro{\b}{2.3}
\draw [very thick] (\a,{(2*\a)-3})--(\b,{2*\b-3}) node[above]{$r_4$};
%%%
\pgfmathsetmacro{\c}{-3}
\pgfmathsetmacro{\d}{4.5}
\draw [very thick] (\d,{(-\d)/2})--(\c,{(-\c)/2}) node[left]{$r_3$};
\end{tikzpicture}\end{bmlimage}
\end{center}
\end{sol}
\end{exo}

Observe que retas paralelas têm a mesma inclinação.

\begin{exo}
Dê a equação da reta $r'$, paralela a $r$, que passa pelo ponto $P$.
\begin{multicols}{2}
 \begin{enumerate}
  \item\label{ittexreta1} $r:\,y=5x+2$, $P=(-1,5)$.
\item\label{ittexreta2} $r:\,4x-3y+6=0$, $P=(3,-5)$.
 \end{enumerate}
\end{multicols}
\vspace{0.01cm}
\begin{sol}
\eqref{ittexreta1} $r':\,y=5x+10$.
\eqref{ittexreta2} $r':\,y=\tfrac{4}{3}x-9$
\end{sol}
\end{exo}

\begin{exo}
Mostre que se $r_1$ tem inclinação $m_1\neq 0$, e $r_2$ tem inclinação
$m_2=-\frac{1}{m_1}$, então $r_1$ e $r_2$ são perpendiculares.
 \begin{sol} Comecemos com um exemplo: considere a reta $r_1$ de inclinação $m_1=\tfrac13$
que passa pela origem. Qual é a equação da reta $r_2$, perpendicular a $r_1$, que passa
pela origem?
\begin{center}
 \begin{bmlimage}\begin{tikzpicture}
  \draw [ ->] (0,-0.1)--(0,3.2) node[right]{$y$};
\draw [ ->] (-1.5,0)--(3.5,0) node[right]{$x$};
\draw[thick] (-1,-0.33)--(3.3,1.1) node[right]{$r_1$};
\draw[dashed] (0.2,-0.6)--(-1.1,3.3) node[left]{$r_2$};
\fill (3,1) circle (0.55mm);
\draw (3,1) node[above left]{$P_1$};
\fill (-1,3) circle (0.55mm);
\draw (-1,3) node[above right]{$P_2$};
\foreach \k in {-1,...,3}
{\draw ({\k},0) node{$\shortmid$};}
\foreach \k in {0,...,3}
{\draw (0,{\k}) node{$-$};}
 \end{tikzpicture}\end{bmlimage}
\end{center}
 Observe que se $P_1=(3,1)\in r_1$, então o ponto $P_2=(-1,3)\in r_2$, já que o segmento
$OP_1$ precisa ser perpendicular a $OP_2$. Logo, a inclinação de $r_2$ pode ser obtida
usando o ponto $P_2$:
$m_2=\frac{0-3}{0-(-1)}=-3$,
 o que prova $m_2=-\frac{1}{m_1}$. Escolhendo qualquer outro ponto $P_1=(x,y)$ em $r_1$,
obteríamos um ponto $P_2=(-y,x)$, e $m_2$ seria calculada da mesma maneira.\\

 Para uma reta de inclinação $m_1$ qualquer, podemos escolher $P_1=(1,m_1)$ e
$P_2=(-m_1,1)$, assim $m_2=\frac{0-1}{0-(-m_1)}=-\frac{1}{m_1}$ é sempre verificada.
\end{sol}
\end{exo}

\begin{exo}
Determine quais das seguintes retas são paralelas ou perpendiculares.
$$r_1:\,2x+y-1=0\,,\quad r_2:\,x+2y+1=0\,,\quad r_3:\,y=2x-3\,,\quad r_4:\,3x+6y-3=0\,.$$
Em seguida, esboce as retas e verifique.
\begin{sol}
$r_2$ e $r_4$ são paralelas, e ambas são perpendiculares a $r_3$.
\end{sol}
\end{exo}

\subsection{Círculos}\label{SecCirculos}\index{círculo}

Considere o círculo~\footnote{Às vezes, o que chamamos aqui de \emph{círculo} corresponde
a \emph{circunferência} em outros textos de matemática elementar.} $\gamma$ de centro
$C=(1,2)$ e de raio $R=2$:
\begin{center}
\begin{bmlimage}\begin{tikzpicture}[scale=0.6]
\draw [ ->] (0,-1)--(0,4.4) node[left]{$y$};
\draw [ ->] (-2,0)--(4,0) node[right]{$x$};
\foreach \k in {-2,...,3}
{\draw ({\k},0) node{$\shortmid$};}
\foreach \k in {-1,...,4}
{\draw (0,{\k}) node{$-$};}
\draw (3,2) arc (0:360:2);
\fill (1,2) circle (0.55mm);
\draw (1,2) node[above right]{$C$};
\draw (2.8,3.8) node{$\gamma$};
\end{tikzpicture}\end{bmlimage}
\end{center}
 Por definição (ver o Exercício \ref{Exo:subconjplano}), $\gamma$ é definido pelo conjunto
dos pontos $P$ cuja distância euclidiana a $C$ é igual a $2$:
 $d(P,C)=2$. Isso significa que as coordenadas $(x,y)$ de $P$ são ligadas pela seguinte
expressão:
$\sqrt{(x-1)^2+(y-2)^2}=2$. Equivalentemente, $\gamma$ é descrito pela seguinte equação:
$${(x-1)^2+(y-2)^2}=4\,.$$
 Observe que, expandindo os fatores $(x-1)^2$ e $(y-2)^2$, essa última expressão pode ser
escrita na \grasA{forma genérica}\index{círculo! forma genérica}:
$$x^2+y^2-2x-4y+1=0\,.$$

Em geral, um círculo de raio $R>0$ centrado em $C=(x_0,y_0)$ é descrito pela equação
\eq{(x-x_0)^2+(y-y_0)^2=R^2\,.}
Um problema clássico é de achar o centro e o raio a partir da forma genérica.
\begin{ex}
Considere o círculo $\gamma$ descrito pela sua equação genérica 
\eq{\label{eqcirgenerica}
x^2+y^2+6x-8y=0\,.}
Para achar o seu centro e o seu raio, completemos os quadrados\index{completar um
quadrado}: $x^2+6x=(x+3)^2-9$, $y^2-8y=(y-4)^2-16$. Logo, 
\eqref{eqcirgenerica} pode ser escrita como
$(x+3)^2-9+(y-4)^2-16=0$, isto é:
$$(x+3)^2+(y-4)^2=25\equiv 5^2\,.$$
Portanto, $\gamma$ é centrado em $C=(-3,4)$, de raio $R=5$.
\end{ex}

\begin{ex}
Considere $x^2+2x+y^2+2=0$. Completando o quadrado e rearranjando, obtemos
$(x+1)^2+y^2=-1$. Como ``$-1$'' não pode ser escrito como um quadrado, esta equação
não representa um círculo (e na verdade, não existe nenhum par $(x,y)$ que seja
solução).
\end{ex}

\begin{exo}
Determine quais das equações a seguir definem um círculo. Quando for o caso, calcule 
o centro e o raio.
\begin{multicols}{3}
\begin{enumerate}
\item\label{itexcirc1} $x^2+(y+1)^2=9$
\item\label{itexcirc2} $x^2+y^2=-1$
\item\label{itexcirc3} $x^2+y^2=6x$
\item\label{itexcirc4} $x^2+y^2+x+y+1=0$
\item\label{itexcirc5} $x^2+y^2+2x+1=0$
\item\label{itexcirc6} $x^2=y^2+1$
\end{enumerate}
\end{multicols}
\vspace{0.01cm}
\begin{sol}
\eqref{itexcirc1} $C=(0,-1)$, $R=3$.
\eqref{itexcirc2} não é círculo: $-1$ não é um quadrado.
\eqref{itexcirc3} $C=(3,0)$, $R=3$.
 \eqref{itexcirc4} não é círculo: depois de ter completado o quadrado obtemos
$(x+\half)^2+(y+\half)^2=-\half$, que não é um quadrado.
 \eqref{itexcirc5} não é círculo: depois de ter completado o quadrado obtemos
$(x+1)^2+y^2=0$ (que poderia ser interpretado como um círculo de raio $R=0$ centrado em
$(-1,0)$).
\eqref{itexcirc6} não é círculo ($x^2-y^2=1$ é
uma \emph{hipérbole}).
\end{sol}
\end{exo}

\section{Trigonometria}\index{trigonometria}                                  
A \emph{trigonometria} estabelece relações precisas entre os ângulos e os
lados de um triângulo.
Definiremos as três funções (mesmo se a própria noção de \emph{função}
será estudada no próximo capítulo) 
trigonométricas elementares, $\sen$ (seno), $\cos$ (cosseno) e $\tan$ (tangente), e
daremos as suas propriedades básicas. Nos próximos capítulos olharemos
mais de perto as propriedades analíticas dessas funções.
    
\subsection{Medir ângulos no plano}\index{ângulo}

 Para começar, é importante escolher uma \emph{unidade} (como ``metros'' para
comprimentos, ou ``litros'' para volumes) para medir um ângulo
determinado pela abertura entre duas retas.
Descreveremos as duas unidades mais usadas, \emph{graus} e \emph{radianos}.\\

Os ângulos serão medidos a partir de uma reta horizontal, em sentido
\emph{antihorário}.  A abertura mínima, naturalmente, é definida como
valendo zero, qualquer que seja a unidade.
O que precisa ser definido é o valor do \emph{ângulo total}.
Se o ângulo for medido em \grasA{graus}\index{ângulo! medido em graus}, 
esse ângulo total é definido como valendo \grasA{$360$ graus}:
\begin{center}
\begin{bmlimage}\begin{tikzpicture}[scale=1]
\draw (0,0)--(1.5,0) node[right]{$0^o$};
\draw[ ->] (1,0) arc (0:360:1);
\draw (-1,1) node{$360^o$};
\end{tikzpicture}\end{bmlimage}
\end{center}
 Uma vez que o ângulo total foi fixado, a medição dos outros se faz proporcionalmente: a
metade do ângulo total vale $180$ graus, o ângulo reto mede $90$ graus, etc.
A vantagem dessa unidade é que vários ângulos bastante usados em geometria tomam valores
inteiros: $30$, $60$, $90$, $180$, $270$, etc.
\begin{center}                                                                          
\begin{bmlimage}\begin{tikzpicture}[scale=1]
\foreach \k in {30,60,90,120,150,180,210,240,270,300,330,360}
{
\draw (0,0)--({\k}:1.5);
\draw ({\k}:1.9) node{$\k^o$};
}
\end{tikzpicture}\end{bmlimage}
\end{center}
Observe que apesar da posição do ângulo total coincidir com o ângulo
nulo, eles devem ser considerados como distintos.\\

Fixar o ângulo total como sendo igual a $360$ pode parecer arbitrário,
e um jeito mais natural de definir o ângulo total é de usar a
noção de comprimento usual na reta.
De fato, considere o 
círculo de raio $1$ centrado na origem e, partindo do ponto $(1,0)$ (que corresponde
a um ângulo de $0$), 
ande ao longo do círculo no sentido antihorário. Quando tiver
percorrido uma distância igual ao raio do círculo, o ângulo
correspondente é definido como sendo de \grasA{$1$ (um)
radiano}\index{ângulo! medido em radianos}: 

 \begin{center}
 \begin{bmlimage}\begin{tikzpicture}[scale=1]
\draw (0,0)--(2,0) node[right]{$0$};
\draw[dotted] (2,0) arc (0:73:2);
\draw(0,0)--(1 r:2);
\draw (0,0)--(2,0) node[midway, below]{$1$};
\draw (1.9,1.1) node{$1$};
\draw[thick]  (2,0) arc (0:1 r:2);
\draw[ ->] (0.5,0) arc (0:1 r:0.5);
\draw (0.4,0.4) node[right]{$1$ rad};
\end{tikzpicture}\end{bmlimage}
 \end{center}
Observe que o ângulo total corresponde à circunferência de um círculo de raio $1$:
$2\pi$.\\


 Em geral, nessa apostila, os ângulos serão medidos em radianos. Se a medida de um ângulo
em graus é $\alpha_g$ e em radianos é $\alpha_r$,
 a conversão se faz da seguinte maneira: como o ângulo total mede $360$ graus e $2\pi$
radianos, temos $\frac{360}{2\pi}=\frac{\alpha_g}{\alpha_r}$. Portanto,
\eq{
\alpha_g=\frac{180}{\pi}\alpha_r\,,\quad\text{ ou   }
\alpha_r=\frac{\pi}{180}\alpha_g\,.}
Assim, verifica-se por exemplo que um ângulo de $90$ graus corresponde a
$\frac{\pi}{180}90=\tfrac{\pi}{2}=1.57...$ radianos.

\begin{exo} O ponteiro dos segundos de um relógio mede $20$ 
centímetros. Qual distância a
ponta desse ponteiro percorreu depois de uma hora e 15 minutos?
 \begin{sol} Durante uma hora e quinze minutos, o ponteiro dos segundos 
dá $75$ voltas.
Como uma volta representa uma distância percorrida (pela ponta) de 
$2\times \pi\times
20\simeq 125.66$ centímetros, a distância total é de $\simeq 9424.5$ 
centímetros, o que corresponde a $\simeq 94.25$ metros.
\end{sol}
\end{exo}

\begin{exo}
Estime a velocidade (em km/s) com a qual a lua gira em torno 
da terra, sabendo que a distância média terra-lua fica é de 384'400km
e que uma volta dura aproximadamente um mês.
\end{exo}

Um ângulo \emph{negativo} será interpretado como medido no sentido horário:
\begin{center}
\begin{bmlimage}\begin{tikzpicture}[scale=1]
\pgfmathsetmacro{\a}{2.3};
\draw(0,0)--(1 r:1);
\draw(0,0)--(-1 r:1);
\draw (0,0)--(1,0);
\draw[ ->] (0.5,0) arc (0:1 r:0.5);
\draw (0.4,0.3) node[right]{$+\alpha$};
\draw[ ->] (0.5,0) arc (0:-1 r:0.5);
\draw (0.4,-0.3) node[right]{$-\alpha$};
\end{tikzpicture}\end{bmlimage}
\end{center}

\subsection{Seno, cosseno e tangente}
Para poder definir as ligações entre os ângulos e os lados de um triângulo, é necessário 
fazer umas simplificações. Trabalharemos com um \grasA{triângulo retângulo}, 
isto é, que possui um ângulo reto. Considere então o seguinte triângulo $ABC$, retângulo
em $C$:
\begin{center}
\begin{bmlimage}\begin{tikzpicture}[scale=1.5]
\draw (0,0)--(2,1) node[midway, above]{$c$} node[above right]{$B$}--(2,0) 
node[midway, right]{$a$} node[below right]{$C$} --(0,0) node[midway,
below]{$b$}node[below left]{$A$};
\draw (0.4,0) arc (0:26.56:0.4);
\draw (1.8,0)--(1.8,0.2)--(2,0.2);
\draw (0.37,0) node[above right]{$\alpha$};
%\draw (2.8,3.8) node{$\gamma$};
\end{tikzpicture}\end{bmlimage}
\end{center}
Com respeito a $\alpha$, $b$ é chamado de \grasA{cateto adjacente}, $a$ de 
\grasA{cateto oposto}, e $c$ de \grasA{hipotenusa}.\\

 Se dois lados forem conhecidos, o terceiro pode ser calculado usando o Teorema de
Pitágoras, e  o valor do ângulo $\alpha$ é determinado.
Como qualquer triângulo semelhante a $ABC$ tem os mesmos ângulos, $\alpha$ é
determinado uma vez que um dos quocientes $\tfrac{a}{c}$, $\tfrac{b}{c}$, ou
$\tfrac{a}{b}$ for conhecido. A ligação entre $\alpha$ e esses quocientes é
chamada respectivamente
\grasA{seno, cosseno e tangente de $\alpha$}, e
denotada\index{seno}\index{cosseno}\index{tangente} por
$$\boxed{\sen \alpha\pardef \frac{a}{c}\,,\quad 
\cos \alpha\pardef \frac{b}{c}\,,\quad 
\tan \alpha\pardef \frac{a}{b}\,.}$$
(Aqui escreveremos a tangente $\tan \alpha$, mas às vezes se encontra
também a notação $\mathrm{tg}\,\alpha$.)
Observe que a seguinte relação sempre vale:
\eq{\label{eqtrigo1}
\tan \alpha=\frac{\sen \alpha}{\cos \alpha}}
Em alguns casos simples, $\sen \alpha$, $\cos \alpha$ e $\tan \alpha$ podem ser
calculados ``manualmente''. 
\begin{ex} Considere $\alpha=\tfrac{\pi}{4}$ ($=45^o$). Para calcular 
$\sen \tfrac{\pi}{4}$, $\cos \tfrac{\pi}{4}$ e $\tan \tfrac{\pi}{4}$, 
consideremos o seguinte triângulo:
\begin{center}
\begin{bmlimage}\begin{tikzpicture}[scale=1.5]
 \draw (0,0)--(1,0) node[midway, below, color=\coulcoseno]{$1$} --(1,1) node[midway,
right, color=\coulseno]{$1$} --(0,0) node[midway, above left]{$\sqrt{2}$};
\draw (0.4,0) arc (0:45:0.4);
\draw (0.35,0) node[above right]{$\tfrac{\pi}{4}$};
 \draw (2,0.5) node[right]{$\Rightarrow\,\sen
\tfrac{\pi}{4}=\frac{\textcolor{\coulseno}{1}}{\sqrt{2}}\,, \quad
 \cos \tfrac{\pi}{4}=\frac{\textcolor{\coulcoseno}{1}}{\sqrt{2}}\,,\quad\tan
\tfrac{\pi}{4}=\tfrac{\textcolor{\coulseno}{1}}{\textcolor{\coulcoseno}{1}}=1$\,.};
\end{tikzpicture}\end{bmlimage}
\end{center}
\end{ex}

\begin{exo}\label{exo:calculsimple60} Montando em cada caso um triângulo apropriado,
calcule (sem calculadora) $\sen \tfrac{\pi}{3}$, $\cos \tfrac{\pi}{3}$,
$\tan \tfrac{\pi}{3}$, $\sen \tfrac{\pi}{6}$, $\cos \tfrac{\pi}{6}$,
$\tan \tfrac{\pi}{6}$. 
\begin{sol}
\mbox{}

\begin{center}
\begin{bmlimage}\begin{tikzpicture}[scale=2]
 \draw (0,0)--(0.5,0) node[midway, below]{$\tfrac12$} --(0.5,0.866)--(0,0) node[midway,
above left]{$1$};
\draw (0.6,0.35) node{$\frac{\sqrt{3}}{2}$};
\draw (0.225,0.15) node{$\tfrac{\pi}{3}$};
\draw (0.2,0) arc (0:60:0.2);
\draw (0.4,0.5) node{$\tfrac{\pi}{6}$};
\draw (0.4,0.7) arc (240:265:0.25);
\draw (1.5,0.7) node[right]{$\Rightarrow\,\sen \tfrac{\pi}{3}=\frac{\sqrt{3}}{{2}}\,, \quad
\cos \tfrac{\pi}{3}=\frac{1}{{2}}\,,\quad \tan \tfrac{\pi}{3}=\sqrt{3}$\,.};
\draw (1.5,0.2) node[right]{$\Rightarrow\,\sen \tfrac{\pi}{6}=\frac{1}{{2}}\,, \quad
 \cos \tfrac{\pi}{6}=\frac{\sqrt{3}}{{2}}\,,\quad \tan
\tfrac{\pi}{6}=\tfrac{1}{\sqrt{3}}$\,.};
\draw[dotted] (0.5,0)--(1,0)--(0.5,0.866)--cycle;
\end{tikzpicture}\end{bmlimage}
\end{center}
\end{sol}
\end{exo}

\begin{exo}
Para determinar a altura $H$ de uma torre, 
ficamos a uma distância qualquer dela, e medimos o ângulo $\alpha$
entre a horizontal e o topo da torre. Em seguida, andamos uma distância
$d$ em direção à base da torre, e medimos o ângulo $\beta$
entre a horizontal e o topo da torre. 
Expresse $H$ como função de $\alpha,\beta,d$.
\begin{sol}
$H=\frac{d(\tan\beta-\tan\alpha)}{\tan \alpha\tan\beta}$.
\end{sol}
\end{exo}

Faremos agora uma generalização, que permitirá 
\emph{enxergar} melhor os três números $\sen \alpha$, $\cos \alpha$ e 
$\tan \alpha$,  e
que será também útil para considerá-las como \emph{funções de uma 
variável real}, a partir do próximo capítulo.\\

Para tanto, usaremos um triângulo cuja hipotenusa é de tamanho $c=1$. 
Isto é, o ponto $B$
do triângulo da
figura acima é posicionado no círculo de raio $1$ centrado na origem, 
chamado
\grasA{círculo trigonométrico}\index{círculo trigonométrico}. As funções 
trigonométricas
podem então ser medidas efetivamente olhando para os comprimentos da 
seguinte figura:

\begin{center}
\begin{bmlimage}\begin{tikzpicture}[scale=1]
\pgfmathsetmacro{\a}{2.3};
\draw (-\a,0) -- (0,0);
\draw[ ->] (0,-\a) -- (0,\a);
\draw[ ->] (1.1,0)-- (\a,0);
 \draw [color=\coulseno, thick] (1.1,1.665)--(1.1,0) node[midway, above, sloped]{$\sen
\alpha$};
\draw [color=\coultang, thick] (2,3)--(2,0) node[midway, above, sloped]{$\tan \alpha$};
\draw [color=\coulcoseno, thick] (1.1,0)--(0,0) node[midway, below]{$\cos \alpha$};
\draw (1.1,1.665) node[above]{$B$};
\draw[dotted] (2,0) arc (0:360:2);
\draw(0,0)--(2,3);
%\draw (0,0)--(2,0);
\draw (0.5,1.1) node{$1$};
\draw[ ->] (0.5,0) arc (0:1 r:0.5);
\draw (0.4,0.3) node[right]{$\alpha$};
\fill (1.1,1.665) circle (0.45mm);
\end{tikzpicture}\end{bmlimage}
\end{center}

 Observe como $\sen \alpha$, $\cos \alpha$ e $\tan \alpha$ mudam à 
medida que $B$ se
movimenta ao longo do círculo. Em particular, $B$ pode dar uma volta 
completa no círculo,
o que permite estender as funções trigonométricas a qualquer
ângulo~\footnote{A tangente tem um problema nos múltiplos de
$\pisobredois$ (ver mais adiante).} 
$0\leq \alpha \leq 2\pi$, e também para valores maiores ou até negativos. 
Os sinais das funções trigonométricas mudam dependendo do quadrante ao qual $B$ pertence:

\begin{center}
\begin{bmlimage}\begin{tikzpicture}[scale=1]
\pgfmathsetmacro{\a}{2.1};
\draw[ ->] (-\a,0) -- (\a,0);
\draw[ ->] (0,-\a) -- (0,\a);
\draw[right] (0,1.8) node{$1^o:$};
\draw[color=\coulseno,right] (0,1.3) node{$\sen \alpha\geq 0$};
\draw[color=\coulcoseno,right] (0,0.8) node{$\cos \alpha\geq 0$};
\draw[color=\coultang,right] (0,0.3) node{$\tan \alpha\geq 0$};

\pgfmathsetmacro{\a}{-2.1};
\draw[right] (\a,1.8) node{$2^o:$};
\draw[color=\coulseno,right] (\a,1.3) node{$\sen \alpha\geq 0$};
\draw[color=\coulcoseno,right] (\a,0.8) node{$\cos \alpha\leq 0$};
\draw[color=\coultang,right] (\a,0.3) node{$\tan \alpha\leq 0$};

\pgfmathsetmacro{\h}{2.1}
\draw[right] (\a,1.8-\h) node{$3^o:$};
\draw[color=\coulseno,right] (\a,1.3-\h) node{$\sen \alpha\leq 0$};
\draw[color=\coulcoseno,right] (\a,0.8-\h) node{$\cos \alpha\leq 0$};
\draw[color=\coultang,right] (\a,0.3-\h) node{$\tan \alpha\geq 0$};

\draw[right] (0,1.8-\h) node{$4^o:$};
\draw[color=\coulseno,right] (0,1.3-\h) node{$\sen \alpha\leq 0$};
\draw[color=\coulcoseno,right] (0,0.8-\h) node{$\cos \alpha\geq 0$};
\draw[color=\coultang,right] (0,0.3-\h) node{$\tan \alpha\leq 0$};
\end{tikzpicture}\end{bmlimage}
\end{center}

Várias propriedades podem ser obtidas a partir do círculo trigonométrico. Por exemplo, 
 observe que $\alpha$ e $-\alpha$ têm o mesmo cosseno, mas que ao transformar $\alpha$ em
$-\alpha$, o seno muda de sinal. Portanto,
\eq{\label{eqtrigo0}
\cos(-\alpha)=\cos\alpha\,,\quad
\sen(-\alpha)=-\sen\alpha\,,\quad
\tan(-\alpha)=-\tan\alpha\,.}

Todas as identidades do seguinte exercício podem ser obtidas de maneira
parecida, olhando simplesmente para o círculo trigonométrico.

\begin{exo}\label{exorelattrigo}\index{identidades trigonométricas}
Prove as identidades:
\eq{\label{eqtrigo00}
\cos(\pi-\alpha)=-\cos\alpha\,,\quad
\sen(\pi-\alpha)=\sen\alpha\,,\quad
\tan(\pi-\alpha)=-\tan\alpha\,.}
\eq{\label{eqtrigo000}
\cos(\pi+\alpha)=-\cos\alpha\,,\quad
\sen(\pi+\alpha)=-\sen\alpha\,,\quad
\tan(\pi+\alpha)=\tan\alpha\,.}
\eq{\label{eqtrigo0000}
\cos(\tfrac{\pi}{2}-\alpha)=\sen\alpha\,,\quad
\sen(\tfrac{\pi}{2}-\alpha)=\cos\alpha\,,\quad
\tan(\tfrac{\pi}{2}-\alpha)={\cot\alpha}\,.}
\eq{\label{eqtrigo00000}
\cos(\tfrac{\pi}{2}+\alpha)=-\sen\alpha\,,\quad
\sen(\tfrac{\pi}{2}+\alpha)=\cos\alpha\,,\quad
\tan(\tfrac{\pi}{2}+\alpha)=-{\cot\alpha}\,.}
 A \grasA{cotangente}, definida por $\cot \alpha\pardef \frac{1}{\tan \alpha}$, apareceu
naturalmente.
\begin{sol} 
Todas essas identidades seguem da observação do círculo trigonométrico. Por exemplo, 
o desenho
\begin{center}
\begin{bmlimage}\begin{tikzpicture}[scale=3]
\pgfmathsetmacro{\a}{1};
\draw[dotted] (\a,0) arc (0:180:\a);
\draw[ ->, color=gray!70] (-1.1*\a,0) -- (1.1*\a,0);
\draw[ ->, color=gray!70] (0,0) -- (0,1.1*\a);

\pgfmathsetmacro{\alf}{35};

%DESSINER LES ANGLES:
\draw[ ->] ({0.4*\a},0) arc (0:\alf:{0.4*\a});
%\draw ({(\alf)/2}:{(\a)*(0.45)}) node{$\alpha$};
\draw ({\alf/2}:{\a*0.45}) node{$\alpha$};
\draw[ ->] ({0.3*\a},0) arc (0:180-\alf:{0.3*\a});
\draw ({(180-\alf)/2}:{0.35*\a}) node[above]{$\pi-\alpha$};

%DEFINIR LES POINTS:
\coordinate (B) at ({\a*cos(\alf)},{\a*sin(\alf)});
\draw (B) node[above right]{$B$};
\fill (B) circle (0.15 mm);
\draw (0,0)--(B);

\coordinate (C) at ({\a*cos(180-\alf)},{\a*sin(180-\alf)});
%\draw (C) node[above left]{$C$};
\fill (C) circle (0.15 mm);
\draw (0,0)--(C);

\draw[dotted] (C)--(B);

 \draw [color=\coulseno, thick] (B)--({\a*cos(\alf)},0) node[midway, above, sloped]{$\sen
\alpha$};
 \draw [color=\coulcoseno, thick] ({\a*cos(\alf)},0)--(0,0) node[midway, below]{$\cos
\alpha$};
 \draw [color=\coulseno, thick] (C)--({\a*cos(180-\alf)},0) node[midway, above,
sloped]{$\sen (\pi-\alpha)$};
 \draw [color=\coulcoseno, thick] ({\a*cos(180-\alf)},0)--(0,0) node[midway, below]{$\cos
(\pi-\alpha)$};

\end{tikzpicture}\end{bmlimage}
\end{center}
mostra que $\cos(\pi-\alpha)=-\cos\alpha$ e $\sen(\pi-\alpha)=\sen\alpha$.
 Como consequência, 
$$\tan(\pi-\alpha)=\frac{\sen(\pi-\alpha)}{\cos(\pi-\alpha)}=-\tan \alpha\,.$$
Deixemos o leitor provar as identidades parecidas com $\pi+\alpha$.
Por outro lado, o desenho
\begin{center}
\begin{bmlimage}\begin{tikzpicture}[scale=3]
\pgfmathsetmacro{\a}{1};
\draw[dotted] (\a,0) arc (0:90:\a);
\draw[ ->, color=gray!70] (0,0) -- (1.1*\a,0);
\draw[ ->, color=gray!70] (0,0) -- (0,1.1*\a);

\pgfmathsetmacro{\alf}{35};

%DESSINER LES ANGLES:
\coordinate (P) at ({0.4*\a*cos(\alf)},{0.4*\a*sin(\alf)});
\draw[ <-] (P) arc (\alf:90:{0.4*\a});
\draw[ ->] ({0.4*\a},0) arc (0:\alf:{0.4*\a});
\draw ({\alf/2}:{0.45*\a}) node{$\alpha$};
\draw ({\alf+(90-\alf)/2}:{0.35*\a}) node[above right]{$\tfrac{\pi}{2}-\alpha$};

%DEFINIR LES POINTS:
\coordinate (B) at ({\a*cos(\alf)},{\a*sin(\alf)});
\coordinate (Bx) at ({\a*cos(\alf)},0);
\coordinate (By) at (0,{\a*sin(\alf)});

\draw (B) node[above right]{$B$};
\draw (0,0)--(B);
\draw [color=\coulseno, thick] (B)--(Bx) node[midway, above, sloped]{$\sen \alpha$};
\draw [color=\coulcoseno, thick] (Bx)--(0,0) node[midway, below]{$\cos \alpha$};
 \draw [color=\coulseno, thick] (By)--(B) node[midway, above,
sloped]{$\sen(\tfrac{\pi}{2}- \alpha)$};
 \draw [color=\coulcoseno, thick] (By)--(0,0) node[midway, below,
sloped]{$\cos(\tfrac{\pi}{2}- \alpha)$};

\fill (B) circle (0.15 mm);
\end{tikzpicture}\end{bmlimage}
\end{center}
 mostra que $\cos(\tfrac{\pi}{2}-\alpha)=\sen\alpha$ e
$\sen(\tfrac{\pi}{2}-\alpha)=\cos\alpha$.
Como consequência,
$$
 \tan
(\tfrac{\pi}{2}-\alpha)=\frac{\sen(\tfrac{\pi}{2}-\alpha)}{\cos(\tfrac{\pi}{2}-\alpha)}=
\frac{\cos\alpha}{\sen\alpha}\equiv \frac{1}{\tan \alpha}=\cot \alpha\,.
$$
\end{sol}
\end{exo}

\begin{exo}
Complete a seguinte tabela
\begin{center}
\begin{tabular}{|c|c|c|c|c|c|c|c|c|c|c|c|c|c|c|}
\hline
graus & $0$ & $30$&$45$&$60$&$90$&$120$&$150$&$180$&$210$&$240$&$270$&$300$&$330$&$360$\\
\hline
 rad &$0$ &$\tfrac{\pi}{6}$
&$\tfrac{\pi}{4}$&$\tfrac{\pi}{3}$&$\tfrac{\pi}{2}$&$\tfrac{2\pi}{3}$&$\tfrac{5\pi}{6}
$&$\pi$&$\tfrac{7\pi}{6}$&$\tfrac{4\pi}{3}$&$\tfrac{3\pi}{2}$&$\tfrac{5\pi}{3}$&$\tfrac{
11\pi}{6}$&$2\pi$\\ \hline
$\sen$ &$0$ & &$\frac{{1}}{\sqrt{2}}$& &$1$&&&$0$&&&&&&$0$\\  \hline
$\cos$ &$1$ & &$\frac{{1}}{\sqrt{2}}$&&$0$&&&$-1$&&&&&&$1$\\ \hline
$\tan$ &$0$ & &$1$&& $\varnothing$ &&&$0$&&&&&&$0$\\
\hline
\end{tabular}
\end{center}
\end{exo}

\subsection{Identidades trigonométricas}\index{identidades trigonométricas}
As identidades do Exercício \ref{exorelattrigo} deram algumas ligações entre 
seno, cosseno e tangente.
O Teorema de Pitágoras dá também a relação
\eq{\label{eqtrigo2}\cos^2\alpha
+\sen^2\alpha=1\,.}
Provaremos agora a identidade
\begin{align}
\sen(\alpha+\beta)&=\sen\alpha\cos\beta+\cos \alpha\sen\beta\,.\label{eqsensoma}
\end{align}
Apesar desta valer para ângulos $\alpha$ e $\beta$ quaisquer, suporemos que 
$\alpha,\beta\in (0,\pisobrequatro)$, 
e usaremos o seguinte desenho:
\begin{center}
\begin{bmlimage}\begin{tikzpicture}[scale=4]
\pgfmathsetmacro{\a}{1};
\draw[dotted] (\a,0) arc (0:90:\a);
\draw[ ->] (0,0) -- (1.1*\a,0);
\draw[ ->] (0,0) -- (0,1.1*\a);

\pgfmathsetmacro{\alf}{20};
\pgfmathsetmacro{\bet}{35};

\draw[ ->] ({0.2*\a},0) arc (0:\alf:{0.2*\a});
\draw ({\alf/2}:{\a/4}) node{$\alpha$};
\draw[ ->] (\alf:{0.2*\a}) arc (\alf:{\alf+\bet}:{0.2*\a});
\draw ({\alf+\bet/2}:{\a/4}) node{$\beta$};

\coordinate (A) at ({\a*cos(\alf+\bet)},{\a*sin(\alf+\bet)});
\coordinate (B) at ({\a*cos(\alf+\bet)},{cos(\alf+\bet)*tan(\alf)});
\coordinate (C) at ({\a*cos(\alf+\bet)},0);
\coordinate (D) at ({\a*cos(\alf)},{\a*sin(\alf)});
\coordinate (E) at ({\a*cos(\alf+\bet)+(sin(\alf)*sin(\bet)/(sqrt(1+(sin(\alf))^2)))},{\a*sin(\alf+\bet)-((sin(\bet))/(sqrt(1+(sin(\alf))^2)))});

\draw (0,0) node[below, left]{$O$};
\pgfmathsetmacro{\ra}{0.15};
\draw (A) node[above]{$A$};
\fill (A) circle (\ra mm);

\fill (B) circle (\ra mm);
\draw (B) node[above left]{$B$};

\fill (C) circle (\ra mm);
\draw (C) node[below]{$C$};

\fill (D) circle (\ra mm);
\draw (D) node[right]{$D$};

\fill (E) circle (\ra mm);
\draw (E) node[below]{$E$};

\draw[ ->] ({\a*cos(\alf+\bet)},{\a*sin(\alf+\bet)-0.2*\a}) arc
(270:{270+\alf}:{0.2*\a});
\draw  ({\a*cos(\alf+\bet)-0.03*\a},{\a*sin(\alf+\bet)-0.24*\a})
node[below, right]{$\alpha$};

\draw (0,0)--(A) node[midway, above, left]{$1$};
\draw (0,0)--(B);
\draw (0,0)--(C);
\draw (A)--(C);
\draw[dotted] (B)--(D);
\draw[dotted] (A)--(E);
\end{tikzpicture}\end{bmlimage}
\end{center}
Observe que $\sen(\alpha+\beta)=d(A,C)=d(A,B)+d(B,C)$.
 Usando o ponto $E$ (projeção ortogonal de $A$ no segmento $OD$) e olhando para o
triângulo $OEA$, temos $d(O,E)=\cos \beta$ e $d(A,E)=\sen \beta$.
Observe também que o ângulo $BAE$ vale $\alpha$.
Portanto, $d(A,B)=d(A,E)/\cos \alpha=\sen\beta/\cos\alpha$ e $d(B,E)=d(A,B)\sen \alpha$.
Por outro lado, $d(B,C)=d(O,B)\sen \alpha$, mas como
\begin{align*}
 d(O,B)&=d(O,E)-d(B,E)\\
&=\cos \beta-d(A,B)\sen \alpha\\
&=\cos \beta-\frac{\sen\beta}{\cos\alpha}\sen \alpha=\cos \beta-\sen  \beta\tan \alpha\,,
\end{align*}
temos
\begin{align*}
 \sen(\alpha+\beta)&=\frac{\sen \beta}{\cos \alpha}+\sen \alpha\bigl(
\cos \beta-\sen  \beta\tan \alpha\bigr)\\
 &=\frac{\sen \beta}{\cos \alpha}+\sen \alpha\cos\beta-\sen \beta\frac{\sen^2\alpha}{\cos
\alpha}\\
&=\sen\alpha\cos\beta+\sen\beta\cos\alpha\,,
\end{align*}
o que prova \eqref{eqsensoma}.\\

\begin{exo} Prove as identidades (dica: todas podem se deduzir a partir de
\eqref{eqsensoma} e de algumas identidades do Exercício \ref{exorelattrigo}):
\begin{align}
\sen(\alpha-\beta)&=\sen\alpha\cos\beta-\cos \alpha\sen\beta\label{eqsensomabis}\\
\cos(\alpha+\beta)&=\cos\alpha\cos\beta-\sen \alpha\sen\beta\label{eqcossoma}\\
 \tan(\alpha+\beta)&=\frac{\tan \alpha+\tan \beta}{1-\tan\alpha \tan
\beta}\label{eqtansoma}\\
\cos(\alpha-\beta)&=\cos\alpha\cos\beta+\sen \alpha\sen\beta\label{eqcossomabis}\\
 \tan(\alpha-\beta)&=\frac{\tan \alpha-\tan \beta}{1+\tan\alpha \tan
\beta}\label{eqtansomabis}\,.
\end{align}
\begin{sol}
 \eqref{eqsensomabis} segue de \eqref{eqsensoma} trocando $\beta$ por $-\beta$ e usando
\eqref{eqtrigo0}. Para provar 
\eqref{eqcossoma}, basta usar  \eqref{eqsensomabis} da seguinte maneira:
\begin{align*}
 \cos(\alpha+\beta)&=\sen\bigl(\tfrac{\pi}{2}-(\alpha+\beta)\bigr)\\
&=\sen\bigl((\tfrac{\pi}{2}-\alpha)-\beta)\bigr)\\
&=\sen(\tfrac{\pi}{2}-\alpha)\cos\beta-\cos (\tfrac{\pi}{2}-\alpha)\sen \beta\\
&=\cos \alpha\cos\beta-\sen\alpha\sen\beta\,.
\end{align*}
Para \eqref{eqtansoma},
\begin{align*}
 \tan(\alpha+\beta)=\frac{\sen(\alpha+\beta)}{\cos(\alpha+\beta)}=
\frac{\sen\alpha\cos\beta+\cos \alpha\sen\beta}{
\cos\alpha\cos\beta-\sen \alpha\sen\beta}
=\frac{\tan \alpha+\tan\beta}{1-\tan\alpha\tan\beta}\,.
\end{align*}
 A última igualdade foi obtida dividindo o numerador e o denominador por
$\cos\alpha\cos\beta$.
\end{sol}
\end{exo}

\begin{exo}
Prove as identidades:
\begin{align}
 \sen(2\alpha)&=2\sen\alpha\cos\alpha\label{eqidensendoisalpha}\\
\cos(2\alpha)&=\cos^2\alpha-\sen^2\alpha=2\cos^2\alpha-1
=1-2\sen^2\alpha\,,\label{eqidencosdoisalpha}\\
\tan\tfrac{\alpha}{2}&=\frac{\sen \alpha}{1+\cos \alpha}\,,\label{eqidentandoisalpha}\\
\cos\alpha\cdot \cos\beta&=\tfrac12(\cos(\alpha+\beta)+\cos(\alpha-\beta))\,.\label{eqidentantreixx}
\end{align}
\begin{sol}
As duas primeiras seguem das identidades anteriores, com $\beta=\alpha$. 
A terceira obtem-se escrevendo:
$$
\sen\alpha=\sen(2\tfrac{\alpha}{2})=2\sen\tfrac{\alpha}{2}\cos\tfrac{\alpha}{2}=
2\tan\tfrac{\alpha}{2}\cos^2\tfrac{\alpha}{2}=\tan\tfrac{\alpha}{2}(\cos\alpha+1)\,.
$$
 Será que você consegue provar \eqref{eqidentandoisalpha} somente a partir do círculo
trigonométrico?

A última, \eqref{eqidentantreixx}, se obtem facilmente a partir de $\cos(\alpha\pm \beta)$. Observe que
a relação \eqref{eqidentantreixx} é a base da técnica chamada \emph{ring modulation} em música
eletrônica.
\end{sol}
\end{exo}


\begin{exo}
 Calcule a equação da reta $r$ que passa pelo ponto $(2,-1)$, cujo ângulo com a horizontal
é igual a $60^o$. 
\begin{sol}
 A inclinação é dada por $\tan 60^o=\tan \frac{\pi}{3}=\sqrt{3}$ (Exercício
\ref{exo:calculsimple60}). Logo, a equação é $y=\sqrt{3}x-1-2\sqrt{3}$.
\end{sol}
\end{exo}

\begin{exo}\label{exoequacoestrigo}
 Resolva:
\begin{multicols}{3}
\begin{enumerate}
 \item\label{itOinequ1} $\cos x=0$
 \item\label{itOinequ10} $\sen x=\half$
\item \label{itOinequ11} $\sen x=\cos x$
\item \label{itOinequ2} $\sen x=\sen^2 x$
\item \label{itOinequ3} $\sen^2x+\tfrac{3}{2}\sen x=1$
\item \label{itOinequ4} $\sen x\geq \tfrac12$
\item \label{itOinequ5} $|\cos x|< \tfrac{1}{\sqrt{2}}$
\item \label{itOinequ6} $(\cos x+\sen x)^2=\tfrac12$
\item \label{itOinequ7} $\sen (2x)=\sen x$.
\end{enumerate}
\end{multicols}
\vspace{0.01cm}
\begin{sol}
Observe que boa parte das equações desse exercício possuem \emph{infinitas} soluções!
As soluções obtêm-se essencialmente olhando para o círculo trigonométrico.
\eqref{itOinequ1} $S=\{\pisobredois\pm k\pi,\,k\in \bZ\}$.
\eqref{itOinequ10} $S=\{\pisobreseis\pm k2\pi\}\cup \{\tfrac{5\pi}{6}\pm k2\pi\}$
\eqref{itOinequ11} $S=\{\pisobrequatro\pm k\pi,\,k\in \bZ\}$.
\eqref{itOinequ2} $S=\{\pm k\pi\}\cup \{\pisobredois+2k\pi\}$.
\eqref{itOinequ3}  Observe que $z\pardef \sen x$ satisfaz $z^2+\tfrac{3}{2}z-1=0$, isto é
$z=\tfrac{1}{2}$ ou $-2$. Como o seno somente toma valores entre $-1$ e $1$, $\sen x=-2$
não possui soluções. Por outro lado, $\sen x=\half$ possui as soluções $\{\pisobreseis\pm
k2\pi\}\cup \{\tfrac{5\pi}{6}\pm k2\pi\}$, como visto em \eqref{itOinequ10}.
Portanto, $S=\{\pisobreseis\pm k2\pi\}\cup \{\tfrac{5\pi}{6}\pm k2\pi\}$.
\eqref{itOinequ4}  $S=[\tfrac{\pi}{6},\tfrac{5\pi}{6}]$ e as suas translações de $\pm
2k\pi$.
\eqref{itOinequ5} 
$S=(\tfrac{\pi}{4},\tfrac{3\pi}{4})\cup(\tfrac{5\pi}{4},\tfrac{7\pi}{4})$ e as suas
translações de $\pm 2k\pi$.
 \eqref{itOinequ6} Rearranjando obtemos $\sen (2x)=-\tfrac12$, o que significa $2x\in
\{\frac{7\pi}{6}\pm 2k\pi\}\cup \{\frac{11\pi}{6}\pm 2k\pi\}$. Logo,
$S= \{\frac{7\pi}{12}\pm k\pi\}\cup \{\frac{11\pi}{12}\pm k\pi\}$
 \eqref{itOinequ7} $S=\{k\pi,k\in\bZ\}\cup \{\pisobretres+2k\pi,k\in\bZ\}\cup
\{\tfrac{5\pi}{3}+2k\pi,k\in\bZ\}$.
\end{sol}
\end{exo}




% !TeX spellcheck = pt_BR
% !TEX encoding = UTF-8 Unicode

\chapter{Funções}\label{Cap:Funcoes}

\ifdefined\updateans
% Only need to run once in a lifetime, when the file ans.tex needs to be updated.
\Writetofile{ans}{\protect\section*{Capítulo \ref{Cap:Funcoes}}}
\fi

O conceito de \emph{função}\index{função} será o principal assunto 
tratado neste curso.
Neste capítulo daremos algumas definições elementares, e consideraremos
algumas das funções mais usadas na prática, que são as funções 
trigonométricas e as
potências (exponenciais e logaritmos serão estudadas no próximo capítulo).
Também começaremos a falar de \emph{gráfico de uma função} desde a Seção
\ref{Sec:Graficos}.\\

A noção de função aparece quando uma grandeza depende de 
uma outra. Por exemplo:

\begin{itemize}
\item Uma partícula evolui na reta. A \emph{trajetória} é uma função que 
dá a sua posição em função do tempo:
$$t\mapsto x(t)\,.$$

\item 
O \emph{volume} e a \emph{superfície} de uma esfera são duas funções que 
dependem ambas do raio:
$$r\mapsto \tfrac{4}{3}\pi r^3\,,\quad r\mapsto 4\pi r^2\,.$$

\item 
Um gás está contido num recipiente hermeticamente fechado, de 
temperatura fixa mas de volume variável. A \emph{pressão} no recipiente 
é função do volume:
$$v\mapsto p(v)\,.$$
\end{itemize}

\section{Definição e exemplos}

Como visto acima, uma \grasA{função} $f$ (de uma variável real) é um 
mecanismo que, a um número real 
$x$, chamado \grasA{entrada} (ou \grasA{variável}), associa um
único número real construído a partir de 
$x$, denotado $f(x)$ e chamado \grasA{saída} (ou \grasA{imagem}). Essa associação costuma
ser denotada: 
$$x\mapsto f(x)\,.$$
Neste curso, a entrada e a saída serão ambos números reais.
Veremos em breve que cada função precisa ser definida com um \emph{domínio}.

\begin{ex}
 A função ``multiplicação por dois'' $x\mapsto 2x$ (por exemplo $3\mapsto 6$, $-13\mapsto
-26$),
 a função ``valor absoluto'' $x\mapsto |x|$ (por exemplo $3\mapsto 3$, $-13 \mapsto 13$),
a função ``quadrado'' $x\mapsto x^2$ (por exemplo $3\mapsto 9$, $-13\mapsto 169$), e a
função ``valor inteiro'' $x\mapsto \lfloor x\rfloor$, onde $\lfloor x\rfloor$ é o maior
número inteiro menor ou igual a $x$ (por exemplo $3\mapsto 3$, $1.5\mapsto 1$,
$-3.1415\mapsto -4$), são todas bem definidas para qualquer real $x\in \bR$.
\end{ex}

\begin{ex}\label{Ex:Umsobrex}
%\emph{Excluir a divisão por zero.}
Para definir a função ``inverso'', $x\mapsto \frac{1}{x}$, é
preciso evitar uma divisão por zero, isto é, somente tomar uma entrada $x\in \bR\setminus
\{0\}$. 
Assim, a função $f(x)=\tfrac{1}{x}$ é bem definida uma vez que escrita da seguinte maneira:
\begin{align*}
 f:\bR\setminus \{0\}&\to \bR\\
x&\mapsto \tfrac{1}{x}\,.
\end{align*}
Do mesmo jeito, para definir $f(x)=\tfrac{x}{x^2-1}$, é preciso excluir os valores em que
o denominador é zero\index{divisão por zero}:
\begin{align*}
 f:\,\bR\setminus\{-1,+1\}&\to \bR\\
x&\mapsto \tfrac{x}{x^2-1}\,.
\end{align*}
\end{ex}

Os dois últimos exemplos mostram que em geral, uma função deve ser definida {junto} com o
seu \grasA{domínio}\index{domínio}, que dá os valores de $x$ para os quais $f(x)$ é
definida. O domínio será em geral denotado por $D$:
\begin{align*}
f:\,D&\to \bR\\
x&\mapsto f(x)\,.
\end{align*}
O domínio será importante para garantir que $f(x)$ seja bem definida. Mas às
vezes, poderemos escolher um domínio particular somente por razões específicas, ou pelas
exigências de um problema.

\begin{ex}
 As funções trigonométricas encontradas no Capítulo \ref{Cap_Fundam} podem ser
consideradas como \emph{funções} no sentido acima. 
O seno, por exemplo, associa ao ângulo $\alpha$ de um triângulo retângulo a razão do lado
oposto sobre a hipotenusa: $\alpha\mapsto\sen \alpha$.\index{seno! função}
Aqui vemos que, pela origem geométrica do problema, é necessário especificar os valores
possíveis 
de $\alpha$: para o triângulo ser bem definido, o ângulo precisa tomar valores entre $0$ e 
 $\tfrac{\pi}{2}$ (de fato, é delicado falar de ``lado oposto'' para um ângulo nulo ou
maior que 
$\tfrac{\pi}{2}$). 
 Para indicar que a função assim definida pega a sua entrada no intervalo
$(0,\tfrac{\pi}{2})$, escreveremos
\begin{align*}
 \sen:(0,\tfrac{\pi}{2})&\to \bR\\
\alpha&\mapsto \sen \alpha\,.
\end{align*}

% \begin{center}
% \begin{bmlimage}\begin{tikzpicture}[scale=1]
% \pgfmathsetmacro{\a}{2.3};
% %\draw (-\a,0) -- (0,0);
% % \draw[thick, gray] (0,0) grid[step=0.5] (6,3); 
% % \draw[very thin, gray] (0,0) grid[step=0.1] (6,3);
% \draw[ ->] (0,0) -- (0,\a);
% \draw[ ->] (0,0)-- (\a,0);
% \draw [color=\coulseno, thick] (1.1,1.665)--(1.1,0) node[midway, above, sloped]{$\sen \alpha$};
% \draw[dotted] (2,0) arc (0:90:2);
% \draw(0,0)--(1.1,1.665);
% \draw (0.5,1.1) node{$1$};
% \draw[ ->] (0.5,0) arc (0:1 r:0.5);
% \draw (0.4,0.3) node[right]{$\alpha$};
% \fill (1.1,1.665) circle (0.45mm);
% \pgfmathsetmacro{\h}{8};
% \draw (3-\h,1) node[right]{$\sen:(0,\tfrac{\pi}{2})\to \bR$};
% \draw (4.65-\h,0.5) node[right]{$\alpha\mapsto \sen \alpha$};
% \end{tikzpicture}\end{bmlimage}
% \end{center}

No entanto vimos que, usando o círculo trigonométrico\index{círculo trigonométrico}, o
seno
de qualquer ângulo (mesmo negativo) pode ser definido, o que permite estender ele à reta
real inteira:
\begin{align*}
 \sen:\bR&\to \bR\\
\alpha&\mapsto {\sen \alpha}\,.
\end{align*}
A função cosseno\index{cosseno! função} se define de maneira análoga. Mas, com a
tangente, uma restrição é necessária. 
 De fato, $\tan \alpha=\frac{\sen \alpha}{\cos \alpha}$ e, a divisão por zero sendo
proibida, a tangente não é definida para ângulos $\alpha\in \bR$ tais que $\cos\alpha=0$.
Logo (veja o Exercício \ref{exoequacoestrigo}), 
\begin{align*}
 \tan:\bR\setminus\{\pisobredois k\pi,k\in\bZ\}&\to \bR\\
\alpha&\mapsto {\tan \alpha}\,.
\end{align*}
\end{ex}

\begin{ex}
\emph{A função raiz.}\index{raiz! função} Seja $a\in \bR$, e considere a equação
\eq{\label{eqparadefraiz}z^2=a\,.}
 Sabemos (ver Seção \ref{SecEquacoes}) que se $a<0$, essa equação não possui soluções, se
$a=0$ ela possui a única solução $z=0$, e se $a> 0$, ela possui duas soluções:
$z=+\sqrt{a}$ e $z=-\sqrt{a}$. Nesses dois últimos casos, quando $a\geq 0$, definiremos 
a \grasA{função raiz de $a$} como sendo a solução positiva de \eqref{eqparadefraiz}, isto
é, $+\sqrt{a}$.
 Quando $a<0$, a função raiz de $a$ não é definida. Assim, a função raiz $x\mapsto
f(x)=\sqrt{x}$
é bem definida somente quando $x\geq 0$, o que se escreve da seguinte maneira:
\begin{align*}
 f:\,\bR_+&\to \bR\\
x&\mapsto \sqrt{x}\,.
\end{align*}
\end{ex}

Por exemplo, para achar o domínio da função $\sqrt{1-x}$, é necessário que $1-x\geq 0$, 
isto é, que $x\leq 1$. Logo,
\begin{align*}
 f:\,(-\infty,1]&\to \bR\\
x&\mapsto \sqrt{1-x}\,.
\end{align*}

\begin{exo}
Determine os domínios das seguintes funções:
\begin{multicols}{4}
\begin{enumerate}
\item\label{itte1} $\frac{1}{x^2+3x-40}$
\item\label{itte2} $\frac{x}{x}$
\item\label{itte21} $|x-1|$
\item\label{itte3} $\frac{x+1}{x^2+1}$
\item\label{itte4} $\frac{1}{1-\frac{1-x}{x}}$
\item\label{itte45} $\sqrt{x-1}$
\item\label{itte46} $\sqrt{x^2-1}$
\item\label{itte5} $\frac{1}{1-\sqrt{x-1}}$
\item\label{itdominio1} $\frac{8x}{1-x^2}$
\item\label{itdominio2} $\frac{8x}{\sqrt{1-x^2}}$
\item\label{itdominio3} $\sqrt{2x-1-x^2}$
\item\label{itdominio4} $\frac{\sqrt{2x-x^2}}{\sqrt{2-x-x^2}}$
\item\label{itdominio41} $\frac{1}{\cos x}$
\item\label{itdominio5} $\sqrt{{\sen x}}$
\item\label{itdominio6} $\sqrt{x}-\sqrt{x}$
\item\label{itdominio7} $\sqrt{1-\sqrt{1+x^2}}$
\end{enumerate}
\end{multicols}
\vspace{0.01cm}
\begin{sol}
\eqref{itte1} $D=\bR\setminus\{-8,5\}$ 
\eqref{itte2} $D=\bR\setminus\{0\}$ 
\eqref{itte21} $D=\bR$
\eqref{itte3} $D=\bR$ 
\eqref{itte4} $D=\bR\setminus\{0,\tfrac12\}$ 
\eqref{itte45} $D=[1,\infty)$
\eqref{itte46} $D=(-\infty,-1]\cup [1,\infty)$ 
\eqref{itte5} $D=[1,\infty ) \setminus \{2\}$
\eqref{itdominio1} $D=\bR\setminus\{\pm 1\}$
\eqref{itdominio2} $D=(-1,+1)$
\eqref{itdominio3} $D=\{1\}$
 \eqref{itdominio4} $D=[0,1)$ (Atenção: é necessário que o numerador \emph{e} o
denominador sejam bem definidos.)
\eqref{itdominio41} $D=\bR\setminus \{\pisobredois+k\pi, k\in \bZ\}$ 
\eqref{itdominio5} $D=$união dos intervalos $[k2\pi,\pi +k2\pi]$, para $k\in \bZ$.
 \eqref{itdominio6} $D=\bR_+$. Observe que apesar da função ser identicamente nula, o seu
domínio não é a reta toda.
\eqref{itdominio7} $D=\{0\}$ (e não $D=\varnothing$!).
\end{sol}
\end{exo}

\subsection{Limitação}\index{função! limitada}
Vimos que a função $f(x)=\tfrac{1}{x}$ é bem definida quando $x\neq 0$, mas observemos
agora o que acontece com $f(x)$ para os valores de $x$ perto de $0$. Por exemplo, para os
valores de $x$ positivos $x=0.1$, $x=0.01$, ...
$$
 \tfrac{1}{0.1}=10\,,\quad \tfrac{1}{0.01}=100\,,\quad
\tfrac{1}{0.001}=1000\,,\quad\dots\quad\,,\,\tfrac{1}{0.0000001}=10000000\,\dots.
$$
Assim, vemos que a medida que $x>0$ se aproxima de zero, $\tfrac{1}{x}$ atinge valores
positivos arbitrariamente grandes. O mesmo fenômeno acontece para os valores de $x<0$:
$\tfrac{1}{x}$ atinge valores negativos arbitrariamente grandes.
Diz-se que a função é \emph{não-limitada}.\\

Uma função $f$ com domínio $D$ é dita \grasA{limitada superiormente} se existir um 
número finito $M_+$
tal que
$$f(x)\leq M_+\quad \forall x\in D\,.$$
Por outro lado, $f$ é dita \grasA{limitada inferiormente} se existir um número
finito $M_-$ tal que
$$f(x)\geq M_-\quad \forall x\in D\,.$$
Se $f$ for limitada inferiormente \emph{e} superiormente, então ela é
\grasA{limitada}.\\

\begin{ex}
A função seno é limitada. De fato, pela definição (olhe para o círculo trigonométrico),
$-1\leq \sen x\leq 1$. Aqui podemos tomar $M_+=1$, $M_-=-1$.
\end{ex}

\begin{ex} Como visto acima, a função $\tfrac{1}{x}$ não é limitada, nem
inferiormente nem superiormente. 
Por outro lado, $\tfrac{1}{x^2}$ não é limitada superiormente, pois
pode tomar valores arbitrariamente grandes a medida que $x$ se aproxima de zero.
No entanto, como $\tfrac{1}{x^2}\geq 0$, ela é limitada inferiormente
(podemos escolher $M_-=0$, ou $M_-=-3$, ou qualquer outro número negativo).

Do mesmo jeito, a função $f(x)=\frac{x}{x^2-1}$ (Exemplo
\ref{Ex:Umsobrex}) é não-limitada, pois toma valores arbitrariamente grandes
(negativos ou positivos) quando $x$ se aproxima de $+1$ ou $-1$.
\end{ex}

\begin{ex}
 Considere $f(x)=\frac{x^2}{x^2+1}$. Observe que $f$ é sempre não-negativa, e que o
numerador é menor do que o denominador para qualquer $x$: $x^2\leq x^2+1$. Logo,
$$0\leq f(x)=\frac{x^2}{x^2+1}\leq \frac{x^2+1}{x^2+1}=1\,,$$
o que prova que $f$ é limitada (por exemplo com $M_-=0$, $M_+=1$).
\end{ex}

\begin{exo}
Determine quais das funções abaixo são limitadas.
\begin{multicols}{3}
\begin{enumerate}
\item\label{itlimitacao1} $x^2$
 \item\label{itlimitacao2} $\tan x$
\item\label{itlimitacao3} $\frac{1}{x^2+1}$
\item\label{itlimitacao4} $\frac{1}{\sqrt{1-x}}$
\item\label{itlimitacao5} $\frac{x-1}{x^3-x^2+x-1}$
\item\label{itlimitacao6} $x+\sen x$
\end{enumerate}
\end{multicols}
\vspace{0.01cm}
\begin{sol}
\eqref{itlimitacao1} $x^2$ é limitada inferiormente ($M_-=0$) mas não
superiormente: toma valores arbitrariamente grandes quando
$x$ toma valores grandes. \eqref{itlimitacao2} Não-limitada. De fato, $\tan x=\frac{\sen
x}{\cos x}$, e quando $x$ se aproxima por exemplo de $\pisobredois$, $\sen x$ se aproxima
de $1$ e $\cos x$ de $0$, o que dá uma divisão por zero. (Dê uma olhada no gráfico da
função tangente mais longe no capítulo.) \eqref{itlimitacao3} É limitada: 
$\tfrac{1}{x^2+1}\geq 0\equiv M_-$, e como $x^2+1\geq
1$, temos 
$\frac{1}{x^2+1}\leq \tfrac11=1\equiv M_+$. \eqref{itlimitacao4} Limitada
inferiormente ($M_-=0$), mas não superiormente: o domínio
dessa função é $(-\infty,1)$, e quando $x<1$ se aproxima de $1$, $\sqrt{1-x}$ se aproxima
de zero, o que implica que $\frac{1}{\sqrt{1-x}}$ toma valores arbitrariamente grandes.
\eqref{itlimitacao5} Observe que o denominador $x^3-x^2+x-1$ se anula em $x=1$. Logo, o
domínio da função é $\bR\setminus \{1\}$. Fatorando (ou fazendo a divisão),
$x^3-x^2+x-1=(x-1)(x^2+1)$. Portanto, quando $x\neq 1$,
$\frac{x-1}{x^3-x^2+x-1}=\frac{x-1}{(x-1)(x^2+1)}=\frac{1}{x^2+1}$. Como $\frac{1}{x^2+1}$
é limitada (item \eqref{itlimitacao3}), $\frac{x-1}{x^3-x^2+x-1}$ é limitada.
\eqref{itlimitacao6} Não-limitada. Apesar de $\sen x$ ser limitado por $-1$ e $+1$, 
o ``$x$'' pode tomar valores arbitrariamente grandes.
\end{sol}
\end{exo}

\section{Gráfico}\label{Sec:Graficos}\index{gráfico}
Um dos nossos objetivos é de entender, pelo menos de maneira
qualitativa, a dependência de uma função $f(x)$ em relação à sua
variável $x$. Uma jeito de
proceder é representar a função no plano cartesiano, via o seu
\emph{gráfico}. O gráfico permite extrair 
a informação essencial contida na função, de maneira intuitiva, pois
\emph{geométrica}.\\

Seja $f$ uma função com domínio $D$. 
\grasA{Esboçar o gráfico de $f$} consiste em traçar todos os pontos do plano cartesiano
da forma $(x,f(x))$, 
onde $x\in D$. Por exemplo, se $f$ tem um domínio $D=[a,b]$, 
\begin{center}
\begin{bmlimage}\begin{tikzpicture}
\coordinate (A) at (0,0);
\coordinate (B) at (3,0);
\pgfmathsetmacro{\x}{1.8}
\coordinate (X) at (\x,0);
\draw [->] (-0.5,0)--(3.5,0);
\draw (A) node{$\shortmid$};
%\fill (A) circle (0.35mm);
\draw (A) node[below]{$a$};
%\draw[thick] (A)--(B);
\draw (B) node{$\shortmid$};
%\fill (B) circle (0.35mm);
\draw (B) node[below]{$b$};
\draw (X) node{$\shortmid$};
\draw (X) node[below]{$x$};
\draw[dashed] (X)--(\x,{0.05*\x^3-0.2*(\x)^2+2}) node[above right]{$(x,f(x))$};
\draw [thick, domain=0:3] plot (\x,{0.05*\x^3-0.2*(\x)^2+2});
\fill (\x,{0.05*\x^3-0.2*(\x)^2+2}) circle (0.45mm);
\draw[dotted] (A)--(0,2);
\draw[dotted] (B)--(3,{0.05*3^3-0.2*3^2+2});
\fill (0,2) circle (0.45mm);
\fill (3,{0.05*3^3-0.2*3^2+2}) circle (0.45mm);
\end{tikzpicture}\end{bmlimage}
\end{center}
Ao $x$ percorrer o seu domínio $[a,b]$, o ponto $(x,f(x))$ traça o gráfico de $f$.

\begin{ex}\label{Ex:retaegrafico}\index{reta}
\emph{Retas não-verticais} são gráficos de um tipo particular. Por exemplo, 
se $f(x)=\tfrac{x}{2}+1$ é considerada com o domínio $D=[0,2)$, o seu gráfico 
é um pedaço da reta de inclinação\index{inclinação} $\tfrac12$ com
ordenada na origem igual a $1$: 
\begin{center}
\begin{bmlimage}\begin{tikzpicture}
\draw [ ->] (0,-0.1)--(0,2) node[left]{$y$};
\draw [ ->] (-0.3,0)--(2.5,0) node[right]{$x$};
\draw [thick] (0,1)--(2,2);
\draw (0,0) node{$\shortmid$};
\fill[intfechado] (0,1) circle (0.55mm);
\fill[intaberto] (2,2) circle (0.55mm);
\pgfmathsetmacro{\x}{0.8};
\draw (\x,0) node{$\shortmid$};
\draw (\x,0) node[below]{$x$};
\draw[dashed] (\x,0)--(\x,{1+\x/2});
\fill (\x,{1+\x/2}) circle (0.45mm);
\draw (0,0) node[below]{$0$};
\draw (2,0) node[below]{$2$};
\draw[dotted] (2,0)--(2,2);
\draw [decorate, decoration=brace] (-0.05,0)--(-0.05,1) node[midway, left]{$1$};
\end{tikzpicture}\end{bmlimage}
\end{center}
Observe que uma reta vertical \emph{não define o gráfico de uma
função}.
\end{ex}

\begin{ex}
Façamos o esboço da função $f(x)=|x|$, com domínio $D=[-1,2]$. Lembre
que pela definição de valor absoluto em \eqref{eq:defvalorabs},
$|x|=x$ se $x\geq 0$, e $|x|=-x$ se $x<0$.
Portanto, o gráfico de $f$ é: 1) entre $-1$ e $0$, a reta de
inclinação $-1$ passando pela origem, 2) entre $0$ e $2$, a reta de
inclinação $1$ passando pela origem:
\begin{center}
\begin{bmlimage}\begin{tikzpicture}
\draw [ ->] (-1.5,0)--(2.5,0) node[right]{$x$};
\draw [ ->] (0,-0.4)--(0,2) node[left]{$f(x)$};
\draw[thick] (-1,1)--(0,0)--(2,2);
\draw[dotted] (-1,1)--(-1,-0.05) node[below]{$-1$};
\draw[dotted] (2,2)--(2,-0.05) node[below]{$2$};
%%%%%%%
\pgfmathsetmacro{\x}{1};
\pgfmathsetmacro{\y}{abs(\x)};
\draw (\x,0) node{$\shortmid$};
\draw (\x,0) node[below]{$x$};
\draw[dashed] (\x,0)--(\x,\y);
\fill (\x,\y) circle (0.45mm);
%%%%%%%%%
\fill (-1,1) circle (0.45mm);
\fill (2,2) circle (0.45mm);
\end{tikzpicture}\end{bmlimage}
\end{center}
\end{ex}

Os dois gráficos acima eram compostos essencialmente de retas. Vejamos agora um exemplo
um pouco diferente.

\begin{ex}\label{Ex:Grafxdois}
Considere $f(x)=x^2$ com $D=[-2,2]$. Como esboçar o gráfico? Por exemplo,
os pontos $(0,f(0))=(0,0)$, $(1,f(1))=(1,1)$, e $(-\half,f(-\half))=(-\half,\tfrac14)$
pertecem ao gráfico. Traçando o gráfico completo:
\begin{center}
\begin{bmlimage}\begin{tikzpicture}\label{pagefuncaoquadratica}
\pgfmathsetmacro{\a}{2}
\fill (-\a,{\a^2}) circle (0.35mm);
\fill (\a,{\a^2}) circle (0.35mm);
\draw [thick, domain=-\a:\a, samples=50] plot (\x,{(\x)^2});
\draw [ ->] (-\a-0.2,0)--(\a+0.2,0) node[right]{$x$};
\draw [ ->] (0,-0.4)--(0,{\a^2+0.2}) node[left]{$f(x)$};
% \draw[thick] (-1,1)--(0,0)--(2,2);
\draw[dotted] ({-\a},{\a^2})--(-\a,-0.05) node[below]{$-2$};
\draw[dotted] (\a,{\a^2})--(\a,-0.05) node[below]{$2$};
%%%%%%%%%%
\pgfmathsetmacro{\x}{-0.9};
\pgfmathsetmacro{\y}{(\x)^2};
\draw (\x,0) node{$\shortmid$};
\draw (\x,0) node[below]{$x$};
\draw[dashed] (\x,0)--(\x,\y);
\fill (\x,\y) circle (0.35mm);
%%%%%%%%%%%
\end{tikzpicture}\end{bmlimage}
\end{center}
A curva obtida, chamada \grasA{parábola},\index{parábola} será usada inúmeras vezes nesse
curso.
\end{ex}

\begin{obs}
Um dos objetivos desse curso é de poder entender 
as principais propriedades de uma
função pelo estudo do seu gráfico. A noção de \emph{derivada} (ver Capítulo
\ref{Cap:Derivacao}) será de importância central nesse desenvolvimento.

No entanto, o gráfico da função $x^2$ acima foi feito com um computador. Primeiro,
o computador escolhe pontos entre $-2$ e $+2$, digamos $-2<x_1<\dots<x_n<2$, e calcula as  
posições $(x_j,f(x_j))$. Em seguida, ele traça a linha poligonal formada pelos segmentos
ligando $(x_j,f(x_j))$ a $(x_{j+1},f(x_{j+1}))$. Esse procedimento é chamado
\emph{interpolação}\index{interpolação}.
Por exemplo, escolhendo $n=3$, $5$ ou $9$ pontos no intervalo $[-2,2]$:
\begin{center}
\begin{bmlimage}\begin{tikzpicture}[scale=0.7]
\pgfmathsetmacro{\a}{2}

\begin{scope}
\draw [thin, color=gray!20, domain=-\a:\a] plot (\x,{(\x)^2});
\newcommand{\funcao}[1]{((#1)^2)}
\draw [ ->] (-\a-0.2,0)--(\a+0.2,0);
\draw [ ->] (0,-0.4)--(0,{\a^2+0.2});
% \draw[thick] (-1,1)--(0,0)--(2,2);
\draw[dotted] ({-\a},{\a^2})--(-\a,-0.05);
\draw[dotted] (\a,{\a^2})--(\a,-0.05) ;
\pgfmathsetmacro{\l}{4};
\pgfmathsetmacro{\incr}{((2*\a)/(\l))};
\foreach \i in {1,...,\l} {
\fill ({(\i-1)*\incr-2},{\funcao{-2+(\i-1)*\incr}}) circle (0.40mm);
\draw ({-2+(\i-1)*(\incr)},{(-2+(\i-1)*(\incr))^2})--
({-2+(\i)*(\incr)},{(-2+(\i)*(\incr))^2});
}
\fill (2,4) circle (0.40mm);
\end{scope}

\begin{scope}[xshift=7cm]
\draw [thin, color=gray!20, domain=-\a:\a] plot (\x,{(\x)^2});
\draw [ ->] (-\a-0.2,0)--(\a+0.2,0);
\draw [ ->] (0,-0.4)--(0,{\a^2+0.2});
% \draw[thick] (-1,1)--(0,0)--(2,2);
\draw[dotted] ({-\a},{\a^2})--(-\a,-0.05) ;
\draw[dotted] (\a,{\a^2})--(\a,-0.05) ;
\pgfmathsetmacro{\k}{8};
\pgfmathsetmacro{\incr}{(2*\a/\k)};
\foreach \i in {1,...,\k} {
\fill ({-2+((\i)-1)*\incr},{(-2+(\i-1)*\incr)^2}) circle (0.40mm);
\draw ({-2+(\i-1)*\incr},{(-2+(\i-1)*\incr)^2})--
({-2+\i*\incr},{(-2+\i*\incr)^2});
}
\fill (2,4) circle (0.40mm);
\end{scope}

\begin{scope}[xshift=14cm]
\draw [thin, color=gray!20, domain=-\a:\a] plot (\x,{(\x)^2});
\draw [ ->] (-\a-0.2,0)--(\a+0.2,0);
\draw [ ->] (0,-0.4)--(0,{\a^2+0.2});
% \draw[thick] (-1,1)--(0,0)--(2,2);
\draw[dotted] ({-\a},{\a^2})--(-\a,-0.05) ;
\draw[dotted] (\a,{\a^2})--(\a,-0.05) ;
\pgfmathsetmacro{\k}{10};
\pgfmathsetmacro{\incr}{2*\a/\k};
\foreach \i in {1,...,\k} {
\fill ({-2+((\i)-1)*\incr},{(-2+(\i-1)*\incr)^2}) circle (0.40mm);
\draw ({-2+(\i-1)*\incr},{(-2+(\i-1)*\incr)^2})--
({-2+\i*\incr},{(-2+\i*\incr)^2});
}
\fill (2,4) circle (0.40mm);
\end{scope}

\end{tikzpicture}\end{bmlimage}
\end{center}
Quando o número de pontos escolhidos é grande e  
 $|x_{j+1}-x_j|$ é pequeno, a linha poligonal dá uma idéia do que deve ser o verdadeiro
esboço (o gráfico do Exemplo \ref{Ex:Grafxdois} foi feito com $n=50$, e já não dá mais
para perceber que a curva é na verdade uma linha poligonal).
O mesmo método permite (em princípio, tomando às vezes um certo cuidado) usar o computador
para esboçar o gráfico de qualquer função $f:D\mapsto \bR$.
Todos os gráficos dessa apostila foram feitos com esse método de interpolação. 
Enfatizemos que as ferramentas matemáticas desenvolvidas mais longe no curso 
permitirão extrair informações a respeito do
gráfico de uma função dada, \emph{sem} usar o computador. Isso será o objetivo do
\emph{estudo de funções}.
Lá, o computador poderá ser usado somente como meio de \emph{verificação}.
\end{obs}


Um problema inverso é de procurar uma função cujo esboço tenha características
específicas.
\begin{ex}
 Procuremos agora a função cujo gráfico é a metade superior do círculo de raio $R=4$
centrado na origem:\index{círculo! equação}
\begin{center}
\begin{bmlimage}\begin{tikzpicture}[scale=0.5]
\pgfmathsetmacro{\a}{4}
%\draw [domain=-\a:\a] plot (\x,{sqrt(16-(\x)^2)});
\draw[thick] (4,0) arc (0:180:4);
\draw [ ->] (-\a-0.2,0)--(\a+0.5,0) node[right]{$x$};
\draw [ ->] (0,-0.4)--(0,{\a+0.5}) node[left]{};
% \draw[thick] (-1,1)--(0,0)--(2,2);
\draw (-\a,0) node[below]{$-4$};
\draw (\a,0) node[below]{$4$};
\fill (\a,0) circle (0.95mm);
\fill (-\a,0) circle (0.95mm);
\end{tikzpicture}\end{bmlimage}
\end{center}

 Lembre (Seção \ref{SecCirculos}) que o círculo \emph{completo} de raio $4$ centrado na
origem, $\gamma$, é formado pelos pontos $(x,y)$ tais que $x^2+y^2=16$. 
 A função procurada será obtida isolando $y$ nessa última relação. Para $y^2=16-x^2$ ter
soluções (aqui, $y$ é a incógnita), 
é preciso impor que $16-x^2\geq 0$, o que implica $-4\leq x\leq 4$. 
 Assim, o domínio da função procurada é $D=[-4,4]$ (como podia se adivinhar olhando para a
figura acima). 
 Assim, quando $x\in D$, a equação acima possui duas soluções $y=+\sqrt{16-x^2}$ e
$y=-\sqrt{16-x^2}$. Para selecionar o semi-círculo \emph{superior}, escolhamos a solução
positiva. Portanto, a função cujo gráfico é dado pelo semi-círculo acima é:
\begin{align*}
f:[-4,4]&\to\bR\\
x&\mapsto \sqrt{16-x^2}\,.
\end{align*}
\end{ex}

\begin{ex} Como a função ``valor absoluto'', funções podem ser definidas \emph{por
trechos}. Por exemplo, com $D=[-1,1)$, o gráfico da função
\begin{align*}
f(x)=
\begin{cases}
-x&\text{ se }-1\leq x< 0\,,\\
\sqrt{1-x^2}&\text{ se }0\leq x< 1\,,
\end{cases}
\end{align*}
é formado pela reta de inclinação $m=-1$ que passa pela origem
entre $x=-1$ e $x=0$, e pela parte do semi-círculo de raio $1$ centrado na origem 
entre $x=0$ e $x=1$:
\begin{center}
\begin{bmlimage}\begin{tikzpicture}
\pgfmathsetmacro{\a}{1}
%\draw [domain=-\a:\a] plot (\x,{sqrt(16-(\x)^2)});
\draw[thick] (0,1) arc (90:0:1);
\draw [->] (-\a-0.2,0)--(\a+0.3,0) node[right]{$x$};
\draw [->] (0,-0.4)--(0,{\a+0.3}) node[left]{};
\draw[thick] (-1,1)--(0,0);
\fill (0,1) circle (0.45mm);
\fill (-1,1) circle (0.45mm);
\fill[intaberto] (0,0) circle (0.45mm);
\fill[intaberto] (1,0) circle (0.45mm);
\draw (-\a,0) node{$\shortmid$};
\draw (-\a,0) node[below]{$-1$};
\draw (\a,0) node[below]{$1$};
\end{tikzpicture}\end{bmlimage}
\end{center}
 Observe que essa função possui uma \emph{descontinuidade em $x=0$}: ao variar $x$ entre
pequenos valores $x<0$ e pequenos valores $x>0$, $f(x)$ pula de valores perto de zero para
valores perto de $1$. 
\end{ex}

\begin{exo}
Dê uma função (e o seu domínio) cujo gráfico seja:
\begin{enumerate}
\item \label{itgrafunc0} a reta horizontal que passa pelo ponto $(-21,-1)$
\item\label{itgrafunc1} a parte inferior do círculo de raio $9$ centrado 
em $(5,-4)$ 
\item \label{itgrafunc2} a parte do círculo de raio $5$ centrado na 
origem que fica
estritamente acima da reta de equação $y=3$
\item \label{itgrafunc3} a parte do círculo de raio $5$ centrado na 
origem contida no quarto quadrante
\end{enumerate}
\vspace{0.01cm}
\begin{sol}
\eqref{itgrafunc0} $f(x)=-1$, $D=\bR$
\eqref{itgrafunc1} $f(x)=-\sqrt{81-(x-5)^2}-4$, $D=[-4,14]$.
\eqref{itgrafunc2} $f(x)=\sqrt{25-x^2}$, $D=(-4,4)$
\eqref{itgrafunc3} $f(x)=-\sqrt{25-x^2}$, $D=[0,5]$
\end{sol}
\end{exo}

\begin{exo}\label{ExoEsbocosElementares}
Esboce os gráficos das seguintes funções (todas com $D=\bR$):
\begin{enumerate}
\item $f(x)= 1$ se $x\leq 1$, $f(x)=x^2$ caso contrário,
\item $g(x)=-|x-1|$,
\item $h(x)=\lfloor x\rfloor$,
\item $i(x)=x-\lfloor x\rfloor$,
\item $j(x)=||x|-1|$.
\end{enumerate}
\vspace{0.01cm}
\begin{sol}\mbox{}

\begin{bmlimage}\begin{tikzpicture}
\begin{scope}[xshift=-0.5cm]
\draw[thick] (-0.7,1)--(1,1);
\fill (1,1) circle (0.45mm);
\draw [thick, domain=1:1.5] plot (\x,{(\x)^2});
\draw [ ->] (-1.2,0)--(1.2,0);
\draw [ ->] (0,-0.4)--(0,{1.2});
\draw (0,1) node[right]{$1$};
\draw (-0.5,0.5) node{$f(x)$};
\end{scope}

\begin{scope}[xshift=1.8cm, yshift=1cm]
\draw [ ->] (-0.2,0)--(2.2,0);
\draw [ ->] (0,-1)--(0,0.5);
\draw (0,0.4) node[right]{$g(x)$};
\draw (1,0) node[above]{$1$};
\draw[thick] (0.2,-0.8)--(1,0)--(2,-1);
\end{scope}

\begin{scope}[xshift=5.8cm, yshift=0.5cm]
\draw [ ->] (-1.5,0)--(1.5,0);
\draw [ ->] (0,-1)--(0,1);
\draw (0,0.8) node[left]{$h(x)$};
\foreach \k in {-3,...,3}{
\pgfmathsetmacro{\a}{\k/3};
\fill (\a,\a) circle (0.45mm);
\draw[thick] (\a,\a)--({\a+0.333},\a);
\fill[intaberto] ({\a+0.333},\a) circle (0.45mm);
}
\end{scope}

\begin{scope}[xshift=9cm, yshift=0.5cm]
\draw [ ->] (-1.3,0)--(1.3,0);
\draw [ ->] (0,-1)--(0,1);
\draw (0,0.8) node[left]{$i(x)$};
\foreach \k in {-3,...,3}{
\pgfmathsetmacro{\a}{\k/3};
\fill (\a,0) circle (0.45mm);
\draw[thick] (\a,0)--({\a+0.333},0.333);
\fill[intaberto] ({\a+0.333},0.333) circle (0.45mm);
}
\end{scope}

\begin{scope}[xshift=12.5cm, yshift=0.1cm, scale=0.7]
\draw [ ->] (-1.3,0)--(1.3,0);
\draw [ ->] (0,-0.3)--(0,1.3) node[left]{$j(x)$};
%\draw [thick, domain=-1.5:1.5, samples=20] plot (\x,{abs(abs(\x)-1)});
\draw[thick] (-2.2,1.2)--(-1,0)--(0,1)--(1,0)--(2.2,1.2);
\end{scope}

\end{tikzpicture}\end{bmlimage}
\end{sol}
\end{exo}


\begin{exo}
Determine quais curvas abaixo são (ou não são) gráficos de funções. Quando for um gráfico,
dê a função associada.
\begin{center}
\begin{bmlimage}\begin{tikzpicture}
 \begin{scope}
\draw[thick] (-1.2,-1)--(1,-1);
\fill (1,-1) circle (0.45mm);
\draw [thick] (1,1)--(2.5,-0.5);
\fill[intaberto] (1,1) circle (0.45mm);
%\draw [thick,  <-, domain=0:1] plot (\x,{(\x)^2});
\draw [ ->] (-1.2,0)--(3,0);
\draw [ ->] (0,-1.2)--(0,{1.2});
\draw (1,0) node{$\shortmid$};
\draw (1,0) node[above left]{$1$};
\end{scope}

\begin{scope}[xshift=5.5cm]
\fill (0,-1) circle (0.45mm);
\draw [thick, domain=-1.2:0] plot (\x,{-sqrt(1-\x)});
\draw [thick, domain=-1.2:0] plot (\x,{-sqrt(1-\x)});
\draw [->] (-1.2,0)--(1.2,0);
\draw [thick, domain=-0.5:1.2] plot (\x,{sqrt(\x+1)});
\draw[dotted] (-0.5,0)--(-0.5,{sqrt(-0.5+1)});
\draw [->] (0,-1.3)--(0,{1.3});
\fill[intaberto] (-0.5,0.707) circle (0.45mm);
\draw (-0.5,0) node[below]{$\scriptstyle{-\tfrac12}$};
\end{scope}

\begin{scope}[xshift=10cm]
\draw [ ->] (-2.4,0)--(2.4,0);
\draw [ ->] (0,-0.8)--(0,{1.3});
\foreach \i in {-2,-1,0,1} {
\draw[thick] (\i,1)--(\i+1,1);
\fill[intaberto] (\i,1) circle (0.45mm);
}
\foreach \i in {-2,-1,0,1,2} {
\draw[dotted] (\i,0)--(\i,1);
\fill (\i,0) circle (0.45mm);
\draw (\i,0) node[below]{$\i$};
}
\end{scope}

\end{tikzpicture}\end{bmlimage}
\end{center}
\begin{sol}
A primeira curva é o gráfico da função $f(x)=-1$ se $x\leq 1$, $f(x)=2-x$ se $x>1$. 
 A segunda não é um gráfico, pois os pontos $-\tfrac12 <x\leq 0$ têm duas saídas, o que
não é descrito por uma função (lembra que uma função é um mecanismo que a um entrada $x$
do domínio associa \emph{um (único)} número $f(x)$). No entanto, seria possível
interpretar aquela curva como a união dos gráficos de duas funções distintas: uma função 
$f$ com domínio $(-\infty,0]$, e uma outra função $g$ com domínio $(-\tfrac12,\infty)$.
A terceira é o gráfico da função $f(x)=0$ se $x\in \bZ$, $f(x)=1$ caso contrário.
\end{sol}
\end{exo}

\subsection{Potências inteiras: $x^p$}\label{Sec:GraficosPotencias}
 Já esboçamos o gráfico da função $f(x)=x^2$ no Exemplo \ref{Ex:Grafxdois}. Vejamos agora
o caso mais geral de uma potência $f(x)=x^p$, onde $p\in \bZ$ (excluiremos o caso $p=0$,
que corresponde a $f(x)=1$). 

\subsubsection{Potências positivas}\index{potência! inteira, positiva}
Para potências positivas \emph{inteiras}, $p>0$, temos $x^p=x\cdot x\cdots x$ ($p$ vezes),
logo o domínio de $x^p$ é sempre $D=\bR$.
Quando $p$ é positiva e \grasA{par}, isto é, $p\in \{ 2,4,6, \dots\}$, então 
$x^p\geq 0$ para todo $x$, e os gráficos são da forma:
\begin{center}
\begin{bmlimage}\begin{tikzpicture}[scale=1]
\pgfmathsetmacro{\a}{1.41}
\draw [thick, dotted, domain=-\a:\a, samples=100] plot
(\x,{(\x)^4});
\pgfmathsetmacro{\a}{1.26}
\draw [thick, dashed, domain=-\a:\a, samples=100] plot
(\x,{(\x)^6});
\pgfmathsetmacro{\a}{2}
\draw [thick, domain=-\a:\a, samples=100] plot (\x,{(\x)^2});
\fill (-1,1) circle (0.35mm);
\fill (1,1) circle (0.35mm);
\draw [ ->] (-2.2,0)--(2.2,0) node[right]{$x$};
\draw [ ->] (0,-0.4)--(0,{4}) node[left]{$x^p$};
\pgfmathsetmacro{\b}{4};
\pgfmathsetmacro{\c}{3.5};
\draw (\b,\c) node[left]{$p=2:$};
\draw[thick] (\b,\c)--(\b+1,\c);
\draw (\b,\c-0.5) node[left]{$p=4:$};
\draw[thick, dotted] (\b,\c-0.5)--(\b+1,\c-0.5);
\draw (\b,\c-1) node[left]{$p=6:$};
\draw[thick, dashed] (\b,\c-1)--(\b+1,\c-1);
\end{tikzpicture}\end{bmlimage}
\end{center}
 Observe que todos os gráficos passam pela origem e pelos pontos 
 $(-1,1)$ e $(1,1)$, e que as funções correspondentes não são 
 limitadas superiormente: tomam 
valores arbitrariamente grandes longe da 
origem (no entanto, todas são limitadas inferiormente por $M_-=0$). 
Vemos
também que quanto maior o $p$, mais rápido $x^p$ cresce quando $x$
cresce.\\

 Quando a potência $p$ é positiva e \grasA{ímpar}, isto é, $p\in \{ 1,3,5, \dots\}$, então
há uma mudança de sinal:
$x^p\geq 0$ para $x\geq 0$, $x^p\leq 0$ para $x\leq 0$. Os gráficos são da forma:
\begin{center}
\begin{bmlimage}\begin{tikzpicture}[scale=1]
\pgfmathsetmacro{\a}{1.15}
\draw [thick, dotted, domain=-\a:\a, samples=100] plot (\x,{\x^3});
\pgfmathsetmacro{\a}{1.1}
\draw [thick, dashed, domain=-\a:\a, samples=100] plot (\x,{\x^5});
\pgfmathsetmacro{\a}{1.4}
\draw [thick, domain=-\a:\a, samples=100] plot (\x,{\x});
\fill (1,1) circle (0.35mm);
\fill (-1,-1) circle (0.35mm);
\draw [ ->] (-1.5,0)--(1.5,0) node[right]{$x$};
\draw [ ->] (0,-0.4)--(0,{2}) node[left]{$x^p$};
\pgfmathsetmacro{\b}{4};
\pgfmathsetmacro{\c}{1.5};
\draw (\b,\c) node[left]{$p=1:$};
\draw[thick] (\b,\c)--(\b+1,\c);
\draw (\b,\c-0.5) node[left]{$p=3:$};
\draw[thick, dotted] (\b,\c-0.5)--(\b+1,\c-0.5);
\draw (\b,\c-1) node[left]{$p=5:$};
\draw[thick, dashed] (\b,\c-1)--(\b+1,\c-1);
\end{tikzpicture}\end{bmlimage}
\end{center}
Observe que nenhuma dessas funções é limitada em $\bR\backslash\{0\}$, 
nem inferiormente nem superiormente.

\subsubsection{Potências negativas}\label{Subsec:graficpotnegativ}\index{potência!
inteira, negativa}
A potência negativa $p={-1}$ já foi encontrada no Exemplo \ref{Ex:Umsobrex}.
Se $p<0$, escreveremos $p=-q$ com $q>0$. Assim, $x^p=\tfrac{1}{x^q}$, que não é definida em $x=0$:
\begin{align*}
 f:\bR\setminus\{0\}&\to\bR\\
x&\mapsto \tfrac{1}{x^q}
\end{align*}
Quando a potência $q$ é \grasA{par}, isto é, $q\in \{ 2,4,6, \dots\}$, então 
$\tfrac{1}{x^{q}}\geq 0$ para todo $x\neq 0$, e os gráficos são da forma:
\begin{center}
\begin{bmlimage}\begin{tikzpicture}[scale=1]
\pgfmathsetmacro{\a}{4}

\draw [thick, domain=-\a:-0.5, samples=100] plot (\x,{1/((\x)^2)});
\draw [thick, domain=0.5:\a, samples=100] plot  (\x,{1/((\x)^2)});

\draw [thick, dotted, domain=-\a:-0.72, samples=100] plot
(\x,{1/((\x)^4)});
\draw [thick, dotted, domain=0.72:\a, samples=100] plot (\x,{1/((\x)^4)});

\draw [thick, dashed, domain=-\a:-0.8, samples=100] plot
(\x,{1/((\x)^6)});
\draw [thick, dashed, domain=0.8:\a, samples=100] plot (\x,{1/((\x)^6)});

\draw [ ->] (-4,0)--(4,0) node[right]{$x$};
\draw [ ->] (0,-0.4)--(0,{3}) node[above]{$\tfrac{1}{x^q}$};
 \pgfmathsetmacro{\b}{4};
 \pgfmathsetmacro{\c}{3.5};
 \draw (\b,\c) node[left]{$q=2:$};
 \draw[thick] (\b,\c)--(\b+1,\c);
 \draw (\b,\c-0.5) node[left]{$q=4:$};
 \draw[thick, dotted] (\b,\c-0.5)--(\b+1,\c-0.5);
 \draw (\b,\c-1) node[left]{$q=6:$};
 \draw[thick, dashed] (\b,\c-1)--(\b+1,\c-1);
\end{tikzpicture}\end{bmlimage}
\end{center}
 Observe que para cada uma dessas funções, ao $x$ se aproximar de $0$, $f(x)$ cresce e
toma valores arbitrariamente \emph{grandes}: é não-limitada. Diremos (mais tarde) que
 há uma \emph{assíntota vertical} em $x=0$. 
Também, quando $x$ toma valores grandes, $f(x)$ decresce e toma 
valores arbitrariamente \emph{pertos de zero}. Diremos (mais tarde) que a função
\emph{tende a zero no infinito}, e que a reta horizontal $y=0$
é \emph{assíntota horizontal}.\\

 Quando a potência é \grasA{ímpar}, a mesma mudança de sinal acontece, e os gráficos têm
propriedades parecidas:
\begin{center}
\begin{bmlimage}\begin{tikzpicture}[scale=1]
\pgfmathsetmacro{\a}{4}

\draw [thick, domain=-\a:-0.32, samples=100] plot (\x,{1/((\x))});
\draw [thick, domain=0.32:\a, samples=100] plot (\x,{1/((\x))});

\draw [thick, dotted, domain=-\a:-0.72, samples=100] plot
(\x,{1/((\x)^3)});
\draw [thick, dotted, domain=0.72:\a, samples=100] plot (\x,{1/((\x)^3)});

\draw [thick, dashed, domain=-\a:-0.8, samples=100] plot
(\x,{1/((\x)^5)});
\draw [thick, dashed, domain=0.8:\a, samples=100] plot (\x,{1/((\x)^5)});

\draw [ ->] (-4,0)--(4,0) node[right]{$x$};
\draw [ ->] (0,-3)--(0,{3}) node[above]{$\tfrac{1}{x^q}$};
 \pgfmathsetmacro{\b}{4};
 \pgfmathsetmacro{\c}{3.5};
 \draw (\b,\c) node[left]{$q=1:$};
 \draw[thick] (\b,\c)--(\b+1,\c);
 \draw (\b,\c-0.5) node[left]{$q=3:$};
 \draw[thick, dotted] (\b,\c-0.5)--(\b+1,\c-0.5);
 \draw (\b,\c-1) node[left]{$q=5:$};
 \draw[thick, dashed] (\b,\c-1)--(\b+1,\c-1);
\end{tikzpicture}\end{bmlimage}
\end{center}

\subsection{Paridade}

Observemos algumas simetrias nos gráficos das funções $x^p$ da seção anterior.
 Primeiro, para os valores de $p$ \emph{pares}, o gráfico de $x^p$ é \emph{simétrico com
respeito ao eixo $y$}, o que segue do seguinte fato: $(-x)^p=x^p$. 
 Por outro lado, para os valores de $p$ \emph{ímpares}, o gráfico de $x^p$ é
\emph{simétrico com respeito à origem} (por uma rotação de $180^o$), o que segue do fato
seguinte: $(-x)^p=-x^p$. \\

Esses fatos \index{função! par}\index{função! par} 
levam a introduzir duas noções gerais. Por um lado, diremos que
$$\boxed{\text{$f$ é \grasA{par} se }f(-x)=f(x)\,,\quad\forall x\text{
do seu domínio.}}$$
Por outro lado,
$$\boxed{\text{$f$ é \grasA{impar} se }f(-x)=-f(x)\,,\quad\forall x\text{
do seu domínio.}}$$ 
\begin{ex}
A função $f(x)=\frac{x^2}{1-x^4}$ é par. De fato, como as potências
envolvidas são pares, $(-x)^2=x^2$, $(-x)^4=x^4$, assim:
$$
f(-x)=\frac{(-x)^2}{1-(-x)^4}=\frac{x^2}{1-x^4}=f(x)\,.
$$
\end{ex}

\begin{ex}
Considere $g(x)=\frac{x^2}{\sen (x)}$.
Vimos que o seno é uma função ímpar: $\sen(-x)=-\sen x$. 
Como consequência, a função $g$ é ímpar, já que
\[
g(-x)=\frac{(-x)^2}{\sen(-x)}=\frac{x^2}{-\sen x}=
-\frac{x^2}{\sen x}=-g(x)\,.
\]
\end{ex}
Mas uma função, em geral, não precisa ser par ou ímpar.
Para mostrar que uma função $f$ não é par, basta achar um ponto
$x$ em que $f(-x)\neq f(x)$. Do mesmo jeito, para mostrar que $f$ não é ímpar, basta achar um ponto em
que $f(-x)\neq -f(x)$.
\begin{ex}
Mostremos que $f(x)=x+1$ não é par. 
De fato, olhando para o ponto $x=-1$, temos $f(-1)=0$, e $f(1)=2$. Logo,
$f(-1)\neq f(1)$. Mas como $f(-1)\neq -f(1)$, $f$ também não é ímpar.
\end{ex}

\begin{exo}
Determine quais das funções $f$ abaixo são pares ou ímpares (justificando a sua
resposta). Quando não for nem par nem ímpar, dê um contra-exemplo.
\begin{multicols}{4}
\begin{enumerate}
\item\label{itparidade1} $\tfrac{x}{x^3-x^5}$
\item\label{itparidade2} $\sqrt{1-x^2}$
\item\label{itparidade3} $x^2\sen x$
\item\label{itparidade4} $\sen (\cos x)$
\item\label{itparidade41} $\sen (\sen x)$
\item\label{itparidade5} $\sen^2 x-\cos x$
\item\label{itparidade6} $\sen x+\cos x$
\item\label{itparidade7} $\sqrt{x^2}-|x|$
\end{enumerate}
\end{multicols}
\vspace{0.01cm}
\begin{sol}
\eqref{itparidade1} É par: $f(-x)=\tfrac{(-x)}{(-x)^3-(-x)^5}=\tfrac{-x}{-(x^3-x^5)}=f(x)$.
\eqref{itparidade2} É par: $f(-x)=\sqrt{1-(-x)^2}=\sqrt{1-x^2}=f(x)$.
\eqref{itparidade3} É ímpar: $f(-x)=(-x)^2\sen (-x)=x^2(-\sen x)=-f(x)$.
\eqref{itparidade4} É par: $f(-x)=\sen (\cos(-x))=\sen(\cos x)=f(x)$.
\eqref{itparidade41} É ímpar: $f(-x)=\sen (\sen(-x))=\sen(-\sen x)=-\sen(\sen x)=-f(x)$.
\eqref{itparidade5} É par: $f(-x)=(\sen(-x))^2-\cos(-x)=(-\sen x)^2-\cos x=f(x)$.
 \eqref{itparidade6} Não é par nem ímpar, pois $f(\pisobrequatro)=\sqrt{2}$,
$f(-\pisobrequatro)=0$.
\eqref{itparidade7} Como $f(x)\equiv 0$, ela é par \emph{e} ímpar.
\end{sol}
\end{exo}


\subsection{Crescimento e decrescimento}\label{sec_Funcoes_variacao}
O que mais nos interessará, no estudo de uma função $f$ dada,
será distinguir as regiões em que ela \emph{cresce/decresce}:

\begin{defin} Seja $I$ um intervalo.
Uma função $f$ é \index{função!crescente}
\begin{itemize}
 \item \grasA{crescente em $I$} se $f(x)\leq f(x')$ para todo $x,x'\in
 I$, $x<x'$.
\item \grasA{estritamente crescente em $I$} se $f(x)<  f(x')$ para todo $x,x'\in
 I$, $x< x'$.\index{função!decrescente}
 \item \grasA{decrescente em $I$} se $f(x)\geq f(x')$ para todo $x,x'\in
 I$, $x< x'$.
\item \grasA{estritamente decrescente em $I$} se $f(x)>  f(x')$ para todo $x,x'\in
 I$, $x< x'$.
\end{itemize}
\end{defin}

Por exemplo, o gráfico de uma função estritamente crescente:
\begin{center}
\begin{bmlimage}\begin{tikzpicture}
\pgfmathsetmacro{\a}{1}
\pgfmathsetmacro{\b}{3}
\newcommand{\mafonc}[1]{4-0.2*( #1 -5)^2}
\draw [thick, domain=0:4.5, samples=100] plot (\x,{\mafonc{\x}});
\draw [ ->] (-0.1,0)--(4,0);
\draw [ ->] (0,-0.5)--(0,{4.5});
\draw[dashed] (\a,0) node[below]{$x$}--(\a,{\mafonc{\a}})--(0,{\mafonc{\a}}) node[left]{$f(x)$};
\draw[dashed] (\b,0) node[below]{$x'$}--(\b,{\mafonc{\b}})--(0,{\mafonc{\b}}) node[left]{$f(x')$};
\end{tikzpicture}\end{bmlimage}
\end{center}
Pela definição acima, uma função constante é ao mesmo tempo crescente e decrescente.\\

\grasA{Estudar a variação} de uma função $f$ será entendido como \emph{procurar
os intervalos em que $f$ cresce ou decresce}. 

\begin{exo}
Estude a variação das funções abaixo.
\begin{multicols}{4}
\begin{enumerate}
\item\label{itexodecr1} $x$
\item \label{itexodecr2}$|x|$
\item \label{itexodecr3}$x^2$
\item \label{itexodecr4}$x^3$
\item \label{itexodecr5}$\frac{1}{x}$
\item \label{itexodecr6}$\frac{1}{x^2}$
\item \label{itexodecr7}$x-x^2$
\item \label{itexodecr8}$||x|-1|$
\end{enumerate}
\end{multicols}
\vspace{0.05cm}
\begin{sol}
\eqref{itexodecr1} cresce na reta toda.
\eqref{itexodecr2} decrescce (estritamente) em $(-\infty,0]$, cresce (estritamente) em $[0,\infty)$.
\eqref{itexodecr3} decrescce (estritamente) em $(-\infty,0]$, cresce (estritamente) em $[0,\infty)$.
\eqref{itexodecr4}  cresce (estritamente) na reta toda.
\eqref{itexodecr5} decrescce (estritamente) em $(-\infty,0)$, decresce (estritamente) em $(0,\infty)$.
\eqref{itexodecr6} crescce (estritamente) em $(-\infty,0)$, decresce (estritamente) em $(0,\infty)$.
\eqref{itexodecr7} crescce (estritamente) em $(-\infty,\tfrac12]$, decresce (estritamente) em
$[\tfrac12,\infty$. (Será mais fácil resolver esse item depois de saber esboçar o gráfico de $x-x^2$,
veja o Exemplo \ref{exemplo_Funcoes_grafparabdesloc}.)
\eqref{itexodecr8} decrescce (estritamente) em $(-\infty,-1]$ e em $[0,1]$, 
cresce (estritamente) em $[-1,0]$ e $[1,\infty)$.
\end{sol}
\end{exo}

Mais tarde introduziremos uma ferramenta fundamental (a \emph{derivada}) para o estudo da variação.

\subsection{Funções Trigonométricas}\label{Sec:GraficosTrigo}
Começemos com o gráfico de $\sen x$, para $x\in [0,2\pi]$:\index{seno! gráfico}

\begin{center}
\begin{bmlimage}\begin{tikzpicture}

\begin{scope}[scale=0.7]
\pgfmathsetmacro{\a}{2.3};
\draw (-\a,0) -- (0,0);
\draw[ ->] (0,-\a) -- (0,\a);
\draw[ ->] (1.1,0)-- (\a,0);
\draw (0,0)--(1 r:2);
\draw [color=\coulseno, thick] (1.1,1.665)--(1.1,0) node[midway, above, sloped]{$\sen x$};
\draw[dotted] (2,0) arc (0:360:2);
\draw (0,0)--(2,0);
\draw (0.5,1.1) node{$1$};
\draw[->] (0.5,0) arc (0:1 r:0.5);
\draw (0.4,0.3) node[right]{$x$};
\fill (1.1,1.665) circle (0.45mm);
\draw (4,0) node{$\Rightarrow$};
\end{scope}

\begin{scope}[scale=1.2, xshift=3.5cm]
\draw [thick, domain=0:6.28, samples=100] plot (\x,{sin(\x r)});
\draw [ ->] (-0.2,0)--(7,0) node[right]{$x$};
\draw [ ->] (0,-1.3)--(0,{1.3}) node[right]{$\sen x$};
\draw[thick, color=\coulseno] (1,{sin(1 r)})--(1,0) node[midway, above, sloped]{$\sen x$};
\fill (1,{sin(1 r)}) circle (0.28mm);
\draw (1,0) node{$\shortmid$};
\draw (1,0) node[below]{$x$};
\draw[dotted] (0,1)--(6.28,1);
\draw (0,1) node[left]{$1$};
\draw[dotted] (0,-1)--(6.28,-1);
\draw (0,-1) node[left]{$-1$};
\draw (3.14,0) node{$\shortmid$};
\draw (3.15,0) node[above]{$\pi$};
\draw (6.28,0) node{$\shortmid$};
\draw (6.28,0) node[above]{$2\pi$};
\end{scope}
\end{tikzpicture}\end{bmlimage}
\end{center}
Se o seno for considerado na reta real toda, obtemos:
\begin{center}
 \begin{bmlimage}\begin{tikzpicture}[scale=0.5]
\fill[color=gray!10] (0,-1) rectangle (6.28,1);
  \draw [thick, domain=-14:14, samples=100] plot (\x,{sin(\x r)});
\draw [ ->] (-15,0)--(15,0) node[right]{$x$};
\draw [ ->] (0,-1.3)--(0,{1.3}) node[left]{$\sen x$};
\pgfmathsetmacro{\Pi}{3.1415};
\draw ({-4*\Pi},0) node[above]{$\scriptstyle{ -4\pi}\,\,\,$};
\draw ({-4*\Pi},0) node{$\shortmid$};
\draw ({-2*\Pi},0) node[above]{$\scriptstyle{ -2\pi}\,\,\,$};
\draw ({-2*\Pi},0) node{$\shortmid$};
\draw ({2*\Pi},0) node[above]{$\scriptstyle{ 2\pi}$};
\draw ({2*\Pi},0) node{$\shortmid$};
\draw ({4*\Pi},0) node[above]{$\scriptstyle{4\pi}$};
\draw ({4*\Pi},0) node{$\shortmid$};
 \end{tikzpicture}\end{bmlimage}
\end{center}
 Observemos que esse gráfico é simétrico em torno da origem (por uma
 rotação de $\pi$), o que
 reflete o fato do seno ser uma função ímpar. Vemos também que $\sen$ é
\grasA{periódica}\index{função! periódica}, de período $2\pi$\index{período}:
$$\boxed{\sen (x+2\pi)=\sen x\,,\quad \forall x\in \bR\,.}$$
Geometricamente: o gráfico completo (para $x\in \bR$) 
é obtido usando translações do gráfico da figura anterior (hachurado, feito para $x\in
[0,2\pi]$).
Essa propriedade pode ser provada analiticamente, usando \eqref{eqtrigo000}:
$\sen(x+2\pi)=\sen(\pi+(x+\pi))=-\sen(x+\pi)=\sen x$.\\

Considerações análogas se aplicam ao cosseno:\index{cosseno! gráfico}

\begin{center}
\begin{bmlimage}\begin{tikzpicture}

\begin{scope}[scale=0.7]
\pgfmathsetmacro{\a}{2.3};
\draw (-\a,0) -- (0,0);
\draw[ ->] (0,-\a) -- (0,\a);
\draw[ ->] (1.1,0)-- (\a,0);
\draw (0,0)--(1 r:2);
\draw [color=\coulcoseno, thick] (1.1,0)--(0,0) node[midway, below]{$\cos x$};
\draw[dotted] (1.1,0)--(1.1,1.665);
\draw[dotted] (2,0) arc (0:360:2);
\draw (1.1,0)--(2,0);
\draw (0.5,1.1) node{$1$};
\draw[->] (0.5,0) arc (0:1 r:0.5);
\draw (0.4,0.3) node[right]{$x$};
\fill (1.1,1.665) circle (0.45mm);
\draw (4,0) node{$\Rightarrow$};
\end{scope}

\begin{scope}[scale=1.2, xshift=3.5cm]
\draw [thick, domain=0:6.28, samples=100] plot (\x,{cos(\x r)});
% \draw[thick] (1,0) arc (0:90:1);
\draw [ ->] (-0.2,0)--(7,0) node[right]{$x$};
\draw [ ->] (0,-1.3)--(0,{1.3}) node[right]{$\cos x$};
% \fill (0,1) circle (0.45mm);
\draw[thick, color=\coulcoseno] (1,{cos(1 r)})--(1,0) node[midway, below, sloped]{$\cos x$};
\fill (1,{cos(1 r)}) circle (0.28mm);
\draw (1,0) node{$\shortmid$};
\draw (1,0) node[below]{$x$};
\draw[dotted] (0,1)--(6.28,1);
\draw (0,1) node[left]{$1$};
\draw[dotted] (0,-1)--(6.28,-1);
\draw (0,-1) node[left]{$-1$};
\draw (3.14,0) node{$\shortmid$};
\draw (3.15,0) node[above]{$\pi$};
\draw (6.28,0) node{$\shortmid$};
\draw (6.28,0) node[above]{$2\pi$};
\end{scope}
\end{tikzpicture}\end{bmlimage}
\end{center}
Quando considerado na reta real, o cosseno é par, e também tem período $2\pi$:
\begin{center}
 \begin{bmlimage}\begin{tikzpicture}[scale=0.5]
\fill[color=gray!10] (0,-1) rectangle (6.28,1);
  \draw [thick, domain=-14:14, samples=100] plot (\x,{cos(\x r)});
\draw [ ->] (-15,0)--(15,0) node[right]{$x$};
\draw [ ->] (0,-1.3)--(0,{1.4}) node[left]{$\cos x$};
\pgfmathsetmacro{\Pi}{3.1415};
\draw ({-4*\Pi},0) node[below]{$\scriptstyle{ -4\pi}\,\,\,$};
\draw ({-4*\Pi},0) node{$\shortmid$};
\draw ({-2*\Pi},0) node[below]{$\scriptstyle{ -2\pi}\,\,\,$};
\draw ({-2*\Pi},0) node{$\shortmid$};
\draw ({2*\Pi},0) node[below]{$\scriptstyle{ 2\pi}$};
\draw ({2*\Pi},0) node{$\shortmid$};
\draw ({4*\Pi},0) node[below]{$\scriptstyle{4\pi}$};
\draw ({4*\Pi},0) node{$\shortmid$};
 \end{tikzpicture}\end{bmlimage}
\end{center}

O esboço da função tangente é um pouco mais delicado. Como foi visto no início do capítulo, 
$\tan x=\frac{\sen x}{\cos x}$ é bem definida somente se $x$ é diferente de $\tfrac{\pi}{2}\pm k\pi$.
Isso implica a presença de \emph{assíntotas verticais} no gráfico:\index{tangente!
gráfico}

\begin{center}
\begin{bmlimage}\begin{tikzpicture}

\begin{scope}[scale=0.7]
\pgfmathsetmacro{\a}{2.3};
\draw (-\a,0) -- (0,0);
\draw[ ->] (0,-\a) -- (0,\a);
\draw[ ->] (1.1,0)-- (\a,0);
%\draw (0,0)--(1 r:2);
%\draw [color=\coulseno, thick] (1.1,1.665)--(1.1,0) node[midway, above, sloped]{$\sen x$};
 \draw [color=\coultang, thick] (2,3)--(2,0) node[midway, above, sloped]{$\tan x$};
%\draw [color=coulcoseno, thick] (1.1,0)--(0,0) node[midway, below]{$\cos x$};
% \draw (1.1,1.665) node[above]{$B$};
%\draw[dotted] (1.1,0)--(1.1,1.665);
\draw[dotted] (2,0) arc (0:360:2);
\draw(0,0)--(2,3);
\draw (0,0)--(2,0);
\draw (0.5,1.1) node{$1$};
\draw[->] (0.5,0) arc (0:1 r:0.5);
\draw (0.4,0.3) node[right]{$x$};
\fill (1.1,1.665) circle (0.45mm);
\draw (4,0) node{$\Rightarrow$};
\end{scope}

\pgfmathsetmacro{\eps}{0.3};
\pgfmathsetmacro{\pisd}{1.5707};
\pgfmathsetmacro{\h}{3};
\begin{scope}[scale=1.2, xshift=3.5cm]
\draw [thick, domain=0:{\pisd-\eps}, samples=100] plot (\x,{(sin(\x r))/(cos(\x r))});
\draw[dashed] (\pisd,-\h)--(\pisd,+\h);
\draw [thick, domain={\pisd+\eps}:{3*\pisd-\eps}, samples=100] plot (\x,{(sin(\x r))/(cos(\x r))});
\draw[dashed] (3*\pisd,-\h)--(3*\pisd,+\h);
\draw [thick, domain={3*\pisd+\eps}:{4*\pisd}, samples=100] plot (\x,{(sin(\x r))/(cos(\x r))});
%\draw[dashed] ({4*\pisd},-\h)--({4*\pisd},\h);
% \draw[thick] (1,0) arc (0:90:1);
\draw [ ->] (-0.2,0)--(7,0) node[right]{$x$};
\draw [ ->] (0,-1.3)--(0,{1.3}) node[above]{$\tan x$};
% \fill (0,1) circle (0.45mm);
\draw[thick, color=\coultang] (1,{tan(1 r)})--(1,0) node[midway, above, sloped]{$\tan x$};
\fill (1,{tan(1 r)}) circle (0.28mm);
\draw (1,0) node{$\shortmid$};
\draw (1,0) node[below]{$x$};
\draw (3.14,0) node{$\shortmid$};
\draw (3.15,0) node[above]{$\pi$};
\draw (6.28,0) node{$\shortmid$};
\draw (6.28,0) node[above]{$2\pi$};
\end{scope}
\end{tikzpicture}\end{bmlimage}
\end{center}

Quando considerado na reta real,
\begin{center}
\begin{bmlimage}\begin{tikzpicture}[scale=0.5]
\pgfmathsetmacro{\eps}{0.3};
\pgfmathsetmacro{\pisd}{1.5707};
\pgfmathsetmacro{\h}{3};

\pgfmathsetmacro{\Pi}{3.1415};

\fill[color=gray!10] (0,-3.4) rectangle (\Pi,3.4);
\draw [ ->] (0,-3.5)--(0,{3.6}) node[above]{$\tan x$};
\foreach \k in {-2,-1,0,1} {
\draw [ ->] (-14,0)--(14,0) node[right]{$x$};
 \draw [thick, domain={\k*2*\Pi}:{\k*2*\Pi+\pisd-\eps}, samples=100] plot (\x,{(sin(\x
r))/(cos(\x r))});
\draw[dashed] (\k*2*\Pi+\pisd,-\h)--(\k*2*\Pi+\pisd,+\h);
 \draw [thick, domain={\k*2*\Pi+\pisd+\eps}:{\k*2*\Pi+3*\pisd-\eps}, samples=100] plot
(\x,{(sin(\x r))/(cos(\x r))});
\draw[dashed] (\k*2*\Pi+3*\pisd,-\h)--(\k*2*\Pi+3*\pisd,+\h);
 \draw [thick, domain={\k*2*\Pi+3*\pisd+\eps}:{\k*2*\Pi+4*\pisd}, samples=100] plot
(\x,{(sin(\x r))/(cos(\x r))});
}

\draw (\Pi,0) node{$\shortmid$};
\draw (\Pi,0) node[above]{$\scriptstyle{\pi}$};
\draw ({2*\Pi},0) node{$\shortmid$};
\draw ({2*\Pi},0) node[above]{$\scriptstyle{2\pi}$};

\end{tikzpicture}\end{bmlimage}
\end{center}
Observemos que o período da tangente é $\pi$ (e não $2\pi$!), como foi 
visto em \eqref{eqtrigo000}:
$$\boxed{\tan(x+\pi)=\tan x\,,\quad \forall x\in \bR\,.}$$
 
\subsection{Transformações}\index{gráfico! transformação de}
O gráfico de uma função $f$ permite obter os gráficos de outras funções, via
\emph{transformações elementares}.
Para simplificar, nesta seção consideraremos somente funções cujo domínio é a reta toda.
\begin{ex}
Considere o gráfico da função $f(x)=x^2$, a parábola do Exemplo \ref{Ex:Grafxdois}. Qual é a 
função ${g}$ cujo gráfico é o gráfico de $f$ 
\emph{transladado de $3$ unidades para a direita?}
\begin{center}
\begin{bmlimage}\begin{tikzpicture}
\draw [color=gray!50, dashed, domain=-1:1] plot (\x,{(\x)^2});
\draw (-1,1) node[above]{$x^2$};
\draw [thick, domain=2:4] plot (\x,{(\x-3)^2});
%\draw (2,1) node[above]{$?$};
\draw (3,0) node[below]{$3$};
\draw [ ->] (-1.2,0)--(4.5,0);
\draw [ ->] (0,-0.4)--(0,1.3);
\pgfmathsetmacro{\a}{0.9};
\draw (\a,0) node[below]{$\scriptstyle{x}$};
\draw (3,0) node{$\shortmid$};
\draw (\a+3,0) node[below]{$\scriptstyle{\tilde{x}}$};
\draw[dotted] (\a,0)--(\a,{\a^2});
\draw[dotted, ->] (\a,{\a^2})--(\a+3,{\a^2}) node[midway, above]{$+3$};
\draw[dotted] (\a+3,{\a^2})--(\a+3,0);
\end{tikzpicture}\end{bmlimage}
\end{center}
 Vemos que o valor tomado por ${{g}}$ em $\tilde{x}=x+3$ deve ser o mesmo que o valor tomado por
$f$ em $x$: ${{g}}(\tilde{x})=f(x)$. Como $x=\tilde{x}-3$, ${{g}}(\tilde{x})=f(\tilde{x}-3)$. Logo, a função procurada
é ${{g}}(x)=(x-3)^2$.
\end{ex}

 De modo geral, suponha $f(x)$ definida para todo $x$, e $a\neq 0$ um número fixo. Defina
a função $g$ por
$$g(x)\pardef f(x-a)\,.$$
 Então o gráfico de $g$ é obtido \emph{transladando horizontalmente o gráfico de $f$ de
$a$ unidades.}\index{translação! horizontal} Apesar do sinal ``$-$'', 
a translação é \emph{para a direita se $a>0$, e para a esquerda se $a<0$}.\\

Por outro lado, se $b\in \bR$, 
$$h(x)\pardef f(x)+b\,$$
é uma função cujo gráfico é o gráfico de $f$ \emph{transladado verticalmente de $b$
unidades.}\index{translação! vertical} A translação é \emph{para cima se $b>0$, para baixo
se $b<0$}.

\begin{ex}\label{exemplo_Funcoes_grafparabdesloc}
 Esbocemos o gráfico da função $f(x)=x^2+2x$. Completando o quadrado\index{completar um
quadrado}, $f(x)=(x+1)^2-1$. Portanto, o gráfico de $f$ é obtido a partir da
parábola\index{parábola} $x^2$ pela composição de uma translação horizontal de uma
unidade para a esquerda, e em seguida uma translação vertical de uma unidade para
baixo:
\begin{center}
\begin{bmlimage}\begin{tikzpicture}
\draw [gray=!30, dashed, domain=-1.2:1.2] plot (\x,{(\x)^2}) node[right]{$x^2$};
%\draw (-1,1) node[above]{$x^2$};
\draw [thick, domain=-2.2:0.2] plot (\x,{(\x+1)^2-1}) ;
\draw (-2.2,{(-2.2+1)^2-1}) node[above]{$x^2+2x$};
%\draw (-1,0) node[above]{$-1$};
\draw (-1,0) node{$\shortmid$};
\draw [gray=!30,  ->] (-2.6,0)--(1.5,0);
\draw [gray=!30,  ->] (0,-0.4)--(0,1.3);
\draw[dotted, ->] (0,0)--(-1,0);
\draw[dotted, ->] (-1,0)--(-1,-1);
\draw (-1,-1) node[below]{$(-1,-1)$};
\end{tikzpicture}\end{bmlimage}
\end{center}
\end{ex}

É claro que o gráfico de $g(x)\pardef -f(x)$ é obtido fazendo a reflexão\index{reflexão}
do gráfico em relação ao eixo $x$, e que o gráfico de $h(x)\pardef f(-x)$ é obtido fazendo
a reflexão do gráfico em relação ao eixo $y$. Portanto, se $f$ é par, $h$ e $f$ têm o
mesmo gráfico. 

\begin{exo}
Considere uma função $f$ definida na reta toda, e a reta vertical $r:$ $x=a$.
Dê a função $g$ cujo gráfico é obtido pelo gráfico de $f$ por \grasA{reflexão em relação à
reta $r$}. Faça a mesma coisa com uma reta horizontal.
\begin{sol}
Se a reta for vertical ($x=a$): $g(x)\pardef f(2a-x)$.
Se a reta for horizontal ($y=b$): $g(x)\pardef 2b-f(x)$.
\end{sol}
\end{exo}

Finalmente, estudemos o que acontece com $g(x)\pardef |f(x)|$. 
Sabemos que o gráfico de $g$ é o mesmo que o de $f$ em todos os pontos $x$ onde $f(x)\geq
0$. Por outro lado, quando $f(x)<0$, então $g(x)=-f(x)$, isto é, o gráfico de $g$ em $x$ é
o de $f$ refletido em relação ao eixo $x$. Em outras palavras: o gráfico 
de $|f|$ é obtido
refletindo todas as partes do gráfico de $f$ negativas, tornando-as 
positivas.
\begin{ex}\label{Ex:modulodografico}
Como $x^2-1$ é a parábola transladada de uma unidade para baixo, o gráfico de $|x^2-1|$ é dado por:
\begin{center}
\begin{bmlimage}\begin{tikzpicture}
\pgfmathsetmacro{\a}{1.6};
\draw [gray=!30, dashed, domain=-\a:\a] plot (\x,{(\x)^2-1});
\draw [thick, domain=-\a:\a, samples=100] plot (\x,{abs((\x)^2-1)}) node[right]{$|x^2-1|$};
\draw[gray=!30] (0.5,-0.8) node[right]{$x^2-1$};
\draw [ ->] (-2,0)--(2,0);
\draw [ ->] (0,-0.4)--(0,1.8);
\end{tikzpicture}\end{bmlimage}
\end{center}
\end{ex}

\begin{exo}
 Interprete todas as identidades trigonométricas\index{identidades trigonométricas} do
Exercício \ref{exorelattrigo} como tranformações dos gráficos de $\sen$, $\cos$ e $\tan$.
\end{exo}

\begin{exo}\label{Ex:graficosbasicos}
Esboce os gráficos das seguintes funções:
\begin{multicols}{3}
 \begin{enumerate}
  \item $f(x)=1-|\sen x|$
  \item $g(x)=x+1-x^2$
  \item $h(x)=||x|-1|$
\item $i(x)=2\sen x$
\item $j(x)=\half\sen x$
\item $k(x)=\frac{2x-x^2}{(x-1)^2}$
 \end{enumerate}
\end{multicols}
\vspace{0.01cm}
\begin{sol}\mbox{}
\begin{center}
 \begin{bmlimage}\begin{tikzpicture}[scale=0.5]
\draw [thick, domain=-8:8, samples=200] plot (\x,{1-abs(sin(\x r))});
\draw [ ->] (-9,0)--(9,0) node[right]{$x$};
\draw [ ->] (0,-0.3)--(0,{1.3}) node[left]{$f(x)$};
\pgfmathsetmacro{\Pi}{3.1415};
\draw ({-2*\Pi},0) node[below]{$\scriptstyle{ -2\pi}\,\,\,$};
\draw ({-2*\Pi},0) node{$\shortmid$};
\draw ({2*\Pi},0) node[below]{$\scriptstyle{ 2\pi}$};
\draw ({2*\Pi},0) node{$\shortmid$};
 \end{tikzpicture}\end{bmlimage}
\end{center}
Observe que o período de $f$ é $\pi$. Completando o quadrado\index{completar um quadrado},
$g(x)=-(x-\tfrac12)^2+\tfrac{5}{4}$:
\begin{center}
\begin{bmlimage}\begin{tikzpicture}
\draw [thick, domain=-0.8:1.8] plot (\x,{1.25-(\x-0.5)^2}) ;
\draw[dotted] (0,1.25)--(0.5,1.25)--(0.5,0);
\draw (0.5,1.25) node[above]{$\scriptstyle{(\tfrac12,\tfrac54)}$};
\fill (0.5,1.25) circle (0.45mm);
\draw [ ->] (-1,0)--(1.9,0);
\draw [ ->] (0,-0.4)--(0,1.7) node[left]{$g(x)$};
\end{tikzpicture}\end{bmlimage}
\end{center}
 Observe que a parábola corta o eixo $x$ nos pontos solução da equação $g(x)=0$, que são
$\frac{1\pm \sqrt{5}}{2}$.
 O gráfico da função $h$ já foi esboçado no Exercício \ref{Ex:graficosbasicos}. Mas aqui
vemos que ele pode ser obtido a partir do gráfico de $|x|$ por uma translação de $1$ para
baixo, composta por uma reflexão das partes negativas.
Como $i(x)$ é igual ao dobro de $\sen x$ e $j(x)$ à metade de $\sen x$, temos:
\begin{center}
 \begin{bmlimage}\begin{tikzpicture}[scale=0.5]
 \draw [color=gray!30, domain=-14:14, samples=100] plot (\x,{sin(\x r)})
node[color=gray!30, right]{$\sen x$};
\draw [thick, domain=-14:14, samples=100] plot (\x,{2*sin(\x r)}) node[right]{$i(x)$};
\draw [thick, domain=-14:14, samples=100] plot (\x,{0.5*sin(\x r)}) node[right]{$j(x)$};
\draw [ ->] (-15,0)--(15,0) node[right]{$x$};
\draw [ ->] (0,-1.3)--(0,{1.3});
\pgfmathsetmacro{\Pi}{3.1415};
 \end{tikzpicture}\end{bmlimage}
\end{center}
Completando o quadrado do numerador:
$k(x)=\frac{1-(x-1)^2}{(x-1)^2}=\frac{1}{(x-1)^2}-1$. Portanto, o gráfico pode ser obtido
a partir do gráfico de $\frac{1}{x^2}$:
\begin{center}
\begin{bmlimage}\begin{tikzpicture}[scale=0.7]
\pgfmathsetmacro{\a}{3.5}
\draw [thick, domain=-\a:-0.5, samples=100] plot (\x,{1/((\x)^2)});
\draw [thick, domain=0.5:\a, samples=100] plot (\x,{1/((\x)^2)});
\draw [ ->] (-1,-1)--(-1,2) node[left]{$y$};
\draw [ ->] (-3,1)--(2,1) node[right]{$x$};
\draw[dotted] (-3,0)--(4,0);
\draw[dotted] (0,-1)--(0,3);
\fill (0,0) circle (0.45mm);
\draw (0,0) node[below]{$(1,-1)$};
\end{tikzpicture}\end{bmlimage}
\end{center}
\end{sol}
\end{exo}

\begin{exo}\label{Exo:TrajetPartic}
Uma partícula de massa $m$ 
é lançada da origem com uma velocidade $\vec{v}=\binom{v_{\textsf{h}}}{v_{\textsf{v}}}$. 
A resolução da segunda equação de Newton mostra que a sua trajetória é dada pela função 
$$x\mapsto y(x)=-\frac12 g\Big(
\frac{x}{v_{\textsf{h}}}\Big)^2+\frac{v_\textsf{v}}{v_{\textsf{h}}}x\,,$$
onde $g$ é o campo de gravitação.
Descreva essa trajetória. Em particular, calcule 
 1) a qual distância a partícula vai cair no chão, e compare essa distância quando $g$ é a
constante de gravitação na superfície da terra ($g=9.81m/s^2$), ou na superfície da lua
($g=1.63m/s^2$, seis vezes menor do que na terra), 2) as coordenadas $(x_*,y_*)$ do ponto
mais alto da trajetória.
\begin{sol}
A trajetória é uma \emph{parábola}.
Resolvendo $y(x)=0$ para $x$, obtemos os pontos onde a parábola toca o chão: $x_1=0$
(ponto de partida), e 
$x_2=\frac{2v_{\textsf{v}}v_{\textsf{h}}}{g}$ (distância na qual a partícula vai cair no
chão).
É claro que se o campo de gravitação é mais fraco (na lua por exemplo), $g$ é menor, logo
$x_2$ é maior: o objeto vai mais longe.
Por simetria sabemos que a abcissa do ponto mais alto da trajetória é 
$x_*=\frac{x_2}{2}=\frac{v_{\textsf{v}}v_{\textsf{h}}}{g}$, e a sua abcissa é dada por
$y_*=y(x_*)=\tfrac12 \frac{v_{\textsf{v}}^2}{g}$. Observe que $y_*$ \emph{não depende de
$v_{\textsf{h}}$}.
O ponto $(x_*,y_*)$ pode também ser calculado a partir da trajetória $y(x)$, completando
o quadrado.
\end{sol}
\end{exo}

Um gráfico permite (em princípio) resolver uma inequação graficamente\index{inequação!
resolução gráfica}. 
\begin{ex}
Considere a inequação do Exemplo \ref{Ex:inequmodulo} (último capítulo),
$$|x-2|>3\,.$$
Com $f(x)=|x-2|$ e $g(x)=3$, o conjunto das soluções da inequação, $S$, pode ser
interpretado como o conjunto dos pontos onde o gráfico de $f$ fica
\emph{estritamente acima} do gráfico de $g$: $f(x)>g(x)$. Como o gráfico de $g$
é uma reta horizontal e 
o de $f$ é o gráfico de $|x|$ transladado de duas unidades para a direita,
\begin{center}
\begin{bmlimage}\begin{tikzpicture}[scale=0.3]
\draw [ ->] (-4.5,0)--(8.5,0) node[right]{$x$};
\draw [ ->] (0,-0.4)--(0,7) node[left]{$y$};
\draw[thick] (8,6)--(2,0)--(-4,6) node[left]{$f$};
\draw[color=gray!70, thick] (-4,3)--(8,3) node[right]{$g$};
\pgfmathsetmacro{\x}{-1};
\pgfmathsetmacro{\y}{abs(\x-2)};
\draw (\x,0) node{$\shortmid$};
\draw (\x,0) node[below]{$\scriptstyle{-1}$};
\draw[dashed] (\x,0)--(\x,\y);
\draw (2,0) node{$\shortmid$};
\draw (2,0) node[below]{$\scriptstyle{2}$};
\pgfmathsetmacro{\x}{5};
\pgfmathsetmacro{\y}{abs(\x-2)};
\draw (\x,0) node{$\shortmid$};
\draw (\x,0) node[below]{$\scriptstyle{5}$};
\draw[dashed] (\x,0)--(\x,\y);
\end{tikzpicture}\end{bmlimage}
\end{center}
vemos que todos os pontos em $(-\infty,-1)\cup (5,\infty)$ satisfazem a essa condição, que
é o que tinha sido encontrado anteriormente.
\end{ex}

\begin{exo}
 Resolva graficamente: 
\begin{multicols}{3}
\begin{enumerate}
\item\label{itineqgraf1}  $1-|x-1|\geq |x|$
\item\label{itineqgraf2}  $1-|x-1|> |x|$
\item\label{itineqgraf3}  $|x^2-1|<1$
\end{enumerate}
\end{multicols}
\vspace{0.01cm}
\begin{sol} \eqref{itineqgraf1} Se $f(x)=1-|x-1|$, $g(x)=|x|$,
\begin{center}
\begin{bmlimage}\begin{tikzpicture}[scale=0.7]
\draw [ ->] (-1.7,0)--(2,0) node[right]{$x$};
\draw [ ->] (0,-1)--(0,2) node[left]{$y$};
\draw[thick, dashed] (-1.5,1.5)--(0,0)--(1.5,1.5) node[right]{$g$};
\draw[thick] (-1,-1)--(1,1)--(3,-1) node[right]{$f$};
\pgfmathsetmacro{\x}{-1};
\pgfmathsetmacro{\y}{abs(\x-2)};
\draw (1,0) node{$\shortmid$};
\draw (1,0) node[below]{$\scriptstyle{1}$};
\end{tikzpicture}\end{bmlimage}
\end{center}
Logo, $S=[0,1]$. Para \eqref{itineqgraf2}, $S=\varnothing$.
\eqref{itineqgraf3} Se $f(x)=|x^2-1|$ (veja o gráfico do Exemplo
\ref{Ex:modulodografico}), vemos que $S=(-\sqrt{2},0)\cup(0,\sqrt{2})$.
\end{sol}
\end{exo}

\section{Montar funções}\index{montar funções}

Será sempre necessário, no estudo de certos problemas, \emph{montar} uma função que
satisfaça a algumas condições.

\begin{exo}
 Uma esfera\index{esfera} é pintada com uma tinta cujo custo é de $\mathrm{R}\$10,00$ por
metro quadrado. 
Expresse o custo total da tinta necessária em função do raio (medido em metros) da
esfera, $T(r)$.
Em seguida, a esfera é enchida de concreto, a $\mathrm{R}\$30,00$ o metro cúbico. Expresse
o 
custo total de concreto necessário em função da superfície (medida em metros quadrados)
da esfera, $C(s)$.
\begin{sol}
Tinta: Como a esfera tem superfície igual a $4\pi r^2$, temos $T(r)=40\pi r^2$ 
(onde $r$ é medido em metros).
Concreto: Como o volume é dado por $V=\tfrac43\pi r^3$, o custo de concreto em 
função do raio é $C(r)=40\pi r^3$. Como a superfície $s=4\pi r^2$ temos 
$r=\sqrt{s/4\pi}$. Portanto, 
$C(s)=40\pi(\tfrac{s}{4\pi})^{3/2}$.
\end{sol}
\end{exo}

\begin{exo}
 Considere um ponto $P=(a,b)$ na reta $2y+x=2$.
Expresse $d(a)$ (respectivamente $d(b)$), 
a distância de $P$ ao ponto $Q=(1,-2)$ em função de $a$ (respectivamente $b$).
\begin{sol}
 Por definição, $d(P,Q)=\sqrt{(a-1)^2+(b+2)^2}$.
Como $2a+b=2$, temos $d(a)=\sqrt{\tfrac54a^2-5a+10}$, e $d(b)=\sqrt{5b^2+5}$.
\end{sol}
\end{exo}

\begin{exo}
Um polígono regular (isto é, com todos os seus lados iguais) com $n$ lados é inscrito em um
disco de raio $r$. Calcule o seu perímetro e a sua área em função de $n$ e $r$.
\begin{sol}
Perímetro: $P(n,r)=2nr\sen (\tfrac{\pi}{n})$.
Área: $A(n,r)=\tfrac12 nr^2\sen(\tfrac{2\pi}{n})$.
\end{sol}
\end{exo}

\begin{exo}
Um recipiente cônico\index{cone} é criado girando o gráfico da função $|x|$ em torno do
eixo $y$.
 O objetivo é usar esse recipiente para criar um medidor de volumes (digamos, em metros
cúbicos). Explique como que a marcação do eixo $y$ deve ser feita: $1m^3$, $2m^3$, ...
Faça um esboço desse medidor.
\begin{sol}
Suponha que o cone fique cheio de água, até uma altura de $h$ metros. Isso representa um
volume de 
$V(h)=\tfrac13 (\pi h^2)\times h$ metros cúbicos. Logo, $h(V)=(\tfrac{3 V}{\pi})^{1/3}$.
Assim, a marca para $1m^3$ deve ficar na altura $h(1)\simeq 0.98$, para $2$ metros
cúbicos, $h(2)\simeq 1.24$, etc.
\begin{center}
\begin{bmlimage}\begin{tikzpicture}
 \draw (-3,3)--(0,0)--(3,3)--cycle;
 \fill[areagrafico] (-3,3)--(0,0)--(3,3)--cycle;
\draw (0,0)--(0,3);
\foreach \k in {1,...,5}{
\pgfmathsetmacro{\h}{(((3*\k)/3.141)^(0.333333))};
\draw (0,{\h}) node{$-$};
\draw[dotted] ({-\h},\h)--(\h,\h) node[right]{$\scriptscriptstyle{\k m^3}$};
}
\foreach \k in {6,...,28}{
\pgfmathsetmacro{\h}{(3*\k/3.1414)^(0.333333)};
\draw (0,{\h}) node{$-$};
\draw[dotted] ({-\h},{\h})--({\h},{\h});
}
\end{tikzpicture}\end{bmlimage}
\end{center}

\end{sol}
\end{exo}

\begin{exo}
Uma corda\index{corda} de tamanho $L$ é cortada em dois pedaços. Com o primeiro pedaço,
faz-se um quadrado, e com o segundo, um círculo. Dê a área total (quadrado $+$ círculo) em
função do tamanho do primeiro pedaço. Dê o domínio dessa função.
\begin{sol}
Seja $x$ o tamanho do primeiro pedaço. Como os lados do quadrado medem
$\tfrac{x}{4}$, a área do quadrado é $\tfrac{x^2}{16}$. O círculo tem
circunferência igual a $L-x$, logo o seu raio vale $\tfrac{L-x}{2\pi}$, e a sua
área 
$\pi(\tfrac{L-x}{2\pi})^2=\tfrac{(L-x)^2}{4\pi}$. Portanto a área total é dada por
$A(x)=\tfrac{x^2}{4}+\tfrac{(L-x)^2}{4\pi}$, e o seu domínio é $D=[0,L]$.
\end{sol}
\end{exo}

\begin{exo}\label{Exo:airesousladroite}
Um triângulo $ABC$ é isósceles em $A$, com $|AB|=|AC|=1$.
Dê a área do triângulo em função do ângulo entre $AB$ e $AC$.
Em seguida, esboce essa função no seu domínio, e ache o ângulo para o qual a área é máxima.
\begin{sol}
Seja $\alpha$ o ângulo entre $AB$ e $AC$.
Área: $A(\alpha)=\sen \tfrac{\alpha}{2}\cos\tfrac{\alpha}{2}=\tfrac12\sen \alpha$, com
$D=(0,\pi)$. 
Logo, (olhe para a função $\sen \alpha$), a área é máxima para $\alpha=\tfrac{\pi}{2}$.
\end{sol}
\end{exo}

\begin{exo}\label{Exo:primeiraarea}
 Considere a reta $r:\, y=x+1$, e os pontos $P=(1,0)$, $Q=(t,0)$, $t>1$. 
Seja $R_t$ a região delimitada pela reta $r$, pelo eixo $x$, 
e pelas retas verticais passando por $P$ e $Q$. Esboce $R_t$, e expresse a sua área
$A(t)$ em  função de $t$.
\begin{sol}
A área pode ser calculada via uma diferença de dois triângulos:
\begin{center}
\begin{bmlimage}\begin{tikzpicture}
\fill[areagrafico] (1,0)--(1,2)--plot[domain=1:2.5](\x,{\x+1})--(2.5,0)--cycle;
\draw[dotted] (1,0)--(1,2);
\draw (1,0) node{$\shortmid$};
\draw (1,0) node[below]{$1$};
\draw[dotted] (2.5,0)--(2.5,3.5);
\draw (2.5,0) node{$\shortmid$};
\draw (2.5,0) node[below]{$t$};
\draw (2.5,-0.1) node[below left]{$\leftarrow$};
\draw (2.5,-0.1) node[below right]{$\rightarrow$};
\draw [ ->] (0,-0.2)--(0,3);
\draw [ ->] (-0.2,0)--(3.5,0);
\draw[thick] (-0.5,0.5)--(3,4) node[right]{$r:\,y=x+1$};
\draw (5,2) node[right]{$A(t)=\tfrac{t^2}{2}+t-\tfrac32$};
\draw (1.7,1.3) node{$R_t$};
\end{tikzpicture}\end{bmlimage}
\end{center}
%\caption{Truc}
%\end{figure}
\end{sol}
\end{exo}


\begin{exo}

Considere uma pirâmide\index{pirâmide} $\Pi$ de altura $H$, cuja base é um quadrado de
lado $L$ ($H$ e $L$ são constantes). Considere em seguida a pirâmide truncada $\Pi'$
obtida cortando $\Pi$ horizontalmente, na altura de um ponto 
$P$ na aresta lateral, como na ilustração. 
\begin{center}
\begin{bmlimage}\begin{tikzpicture}[scale=1.5]
\pgfmathsetmacro{\L}{1};
\pgfmathsetmacro{\H}{2};
\coordinate (A) at (\L,0,\L);
\coordinate (B) at (\L,0,-\L);
\coordinate (C) at (-\L,0,-\L);
\coordinate (D) at (-\L,0,\L);
\coordinate (S) at (0,\H,0);
\coordinate (P) at ({\L/2},{\H/2},{-\L/2});
\coordinate (Q) at ({\L/2},{\H/2},{\L/2});
\coordinate (R) at ({-\L/2},{\H/2},{\L/2});
\coordinate (T) at ({-\L/2},{\H/2},{-\L/2});

\draw[thin] (S)--(D)--(C)--cycle;
\draw[thin] (S)--(C)--(B)--cycle;
\draw (S) node[left]{$S$};
\fill[areagrafico, opacity=0.5] (P)--(B)--(A)--(Q)--cycle;
\draw (B) node[right]{$B$};
\fill[areagrafico, opacity=0.5] (P)--(Q)--(R)--(T)--cycle;
\draw (P) node[above right]{$P$};
\fill[areagrafico, opacity=0.5] (Q)--(R)--(D)--(A)--cycle;
\draw[thick] (P)--(Q)--(R)--(T)--cycle;
\draw[thick] (P)--(B)--(A)--(Q)--cycle;
\draw[thick] (Q)--(R)--(D)--(A)--cycle;
\draw[thin] (S)--(D)--(A)--cycle;
\draw[thin] (S)--(A)--(B)--cycle;
\fill (P) circle (0.5mm);
\end{tikzpicture}\end{bmlimage}
\end{center}
Expresse o volume e a área da superfície de $\Pi'$ em função da distância $x=|PB|$.
\end{exo}

\section{Composição, contradomínio e imagem}

Suponha que se queira obter o valor de $\sen (\pi^2)$ com uma
calculadora. Como uma calculadora possui em geral as duas funções
$(\cdot)^2$ e $\sen(\cdot)$, calculemos primeiro o quadrado de $\pi$, e
em seguida tomemos o seno do resultado:
\[
\pi=3.1415...\stackrel{(\cdot)^2}{\longmapsto} \pi^2=9,8696...
\stackrel{\sen(\cdot)}{\longmapsto} 
\sen(\pi^2)=-0.4303...
\]
O que foi feito foi \emph{compor} duas funções.\\

Sejam $f$ e $g$ duas funções reais. Definemos a \grasA{composição de $f$ com
$g$}\index{função! composição de } como a nova função $f\circ g$ definida por
$$\boxed{(f\circ g)(x)\pardef f(g(x))\,.}$$
Isto significa que para calcular $x\mapsto (f\circ g)(x))$, 
calculamos \emph{primeiro} $g(x)$,
$$x\mapsto g(x)\,,$$
e \emph{em seguida} aplicamos $f$:
$$x\mapsto g(x)\mapsto f(g(x))\,.$$

\begin{exo}
Sejam $f(x)=x^2$, $g(x)=\frac{1}{x+1}$, $h(x)=x+1$. Calcule 
$$(f\circ g)(0)\,, (g\circ f)(0)\,, (f\circ g)(1)\,, (g\circ f)(1)\,, f(g(h(-1)))\,,
h(f(g(3)))\,.$$
\begin{sol} Como $f(g(x))=\frac{1}{(x+1)^2}$, $g(f(x))=\frac{1}{x^2+1}$, temos
$(f\circ g)(0)=1$, $(g\circ f)(0)=1$, $(f\circ g)(1)=\frac14$, $(g\circ f)(1)=\frac12$.
Como $f(g(h(x)))=\frac{1}{(x+2)^2}$ e $h(f(g(x)))=\frac{1}{(x+1)^2}+1$, 
 $f(g(h(-1)))=1$,
$h(f(g(3)))=\frac{17}{16}$.
\end{sol}
\end{exo}

Como foi observado no exercício anterior, $f\circ g$ é em geral diferente de $g\circ f$.\\

Às vezes será necessário considerar uma função complicada como sendo uma composta de
funções mais elementares:
\begin{ex}
A função $x\mapsto \sqrt{1+x^2}$ pode ser vista como a composta
$$x\mapsto 1+x^2\mapsto\sqrt{1+x^2}\,,$$
que significa que $\sqrt{1+x^2}=f(g(x))$, com $g(x)=1+x^2$ e $f(x)=\sqrt{x}$.
Observe que podia também escrever 
$$x\mapsto x^2\mapsto 1+x^2\mapsto \sqrt{1+x^2}\,,$$
que dá a decomposição $\sqrt{1+x^2}=f(g(h(x)))$, onde $h(x)=x^2$, $g(x)=x+1$,
$f(x)=\sqrt{x}$.
\end{ex}

\begin{exo}\label{Exo_elem_decomp_compos}
Para cada função $f$ a seguir, dê uma decomposição de $f$ como composição de funções mais
simples. 
\begin{multicols}{3}
\begin{enumerate}
\item\label{itexcompos1} $\sen (2x)$
\item\label{itexcompos2} $\frac{1}{\sen x}$
\item\label{itexcompos3} $\sen (\tfrac1x)$
\item\label{itexcompos4} $\sqrt{\frac{1}{\tan (x)}}$
\end{enumerate}
\end{multicols}
\vspace{0.01cm}
\begin{sol}
\eqref{itexcompos1} $\sen (2x)=f(g(x))$, onde $g(x)=2x$, $f(x)=\sen x$.
\eqref{itexcompos2} $\frac{1}{\sen x}=f(g(x))$, onde $g(x)=\sen x$, $f(x)=\frac1x$.
\eqref{itexcompos3} $\sen(\frac{1}{x})=f(g(x))$, onde $f(x)=\sen x$, $g(x)=\frac1x$.
\eqref{itexcompos4} $\sqrt{\frac{1}{\tan (x)}}=f(g(h(x)))$, onde $f(x)=\sqrt{x}$,
$g(x)=\frac{1}{x}$, $h(x)=\tan x$.
\end{sol}

\end{exo}

\begin{exo}
Considere
$$
f(x)\pardef 
\begin{cases}
 x+3&\text{ se }x\geq 0\,,\\
x^2&\text{ se }x<0\,,
\end{cases}
\quad\quad
g(x)\pardef 
\begin{cases}
 2x+1&\text{ se }x\geq 3\,,\\
x&\text{ se }x<3\,.
\end{cases}
$$
Calcule $f\circ g$ e $g\circ f$.
\begin{sol}
$$
(g\circ f)(x)=
\begin{cases}
 2x+7&\text{ se }x\geq 0\,,\\
x^2&\text{ se }-\sqrt{3}<x<0\,,\\
2x^2+1&\text{ se }x\leq -\sqrt{3}\,.
\end{cases}
\quad\quad
(f\circ g)(x)=
\begin{cases}
 2x+4&\text{ se }x\geq 3\,,\\
x+3&\text{ se }0\leq x<3\,,\\
x^2&\text{ se }x<0\,.
\end{cases}
$$
\end{sol}
\end{exo}

Lembramos que uma função é sempre definida junto com o seu domínio:
\begin{align*}
f:D&\to \bR\\
x&\mapsto f(x)\,.
\end{align*}
Em ``$f:D\to \bR$'', o ``$\bR$'' foi colocado para indicar que qualquer 
que seja $x$,
$f(x)$ é sempre um número real. Em outras palavas: a imagem de qualquer 
$x\in D$ por $f$ é um número real.
Vejamos em alguns exemplos que esse conjunto ``$\bR$'' pode ser mudado 
por um conjunto que represente melhor a função.

\begin{ex}
Considere
\begin{align*}
f:\bR&\to \bR\\
x&\mapsto x^2\,.
\end{align*}
Como $x^2\geq  0$ qualquer que seja $x\in \bR$, 
vemos que a imagem de qualquer $x\in \bR$ por $f$ é positiva.
Logo, podemos rescrever a função da seguinte maneira:
\begin{align*}
f:\bR&\to [0,\infty)\\
x&\mapsto x^2\,.
\end{align*}
\end{ex}

Quando uma função for escrita na forma 
\begin{align*}
f:D&\to C\\
x&\mapsto f(x)\,,
\end{align*}
para indicar que qualquer $x$ em $D$ tem a sua imagem em $C$, diremos que um
\grasA{contradomínio} foi escolhido para $f$. Em geral, não existe uma escolha única para
o contradomínio.

\begin{ex}
Como, $x\mapsto \sen x$ é uma função limitada, podemos escrever 
\begin{align}
\sen:\bR&\to [-10,+10]\label{eq:contradomsen}\\
x&\mapsto \sen x\,.\nonumber
\end{align}
Mas podemos também escolher um contradomínio menor:
\begin{align}
\sen:\bR&\to [-1,+1]\label{eq:contradomsen}\\
x&\mapsto \sen x\,.\nonumber
\end{align}
Acontece que $[-1,+1]$ é o menor contradomínio possível (ver abaixo).
\end{ex}

% \begin{ex} Para voltar à discussão anterior, considere
% $f(x)=\sqrt{x}$ e $g(x)={1-x}$. Para $(f\circ g)(x)=f(g(x))$ ser bem definida, vemos 
% $x$ precisa ser tal que
% $g(x)1-x\geq 0$. Em outras
% Isto é, $x\in (-\infty,1]$. 
% \end{ex}

 Seja $f:D\to C$. Para cada $x\in D$, lembremos que $f(x)\in C$ é chamado de \emph{imagem
de $x$}\index{imagem}, e o
\grasA{conjunto imagem} de $f$ é definido como
$$\boxed{\imagem(f)\pardef \{f(x):x\in D\}\,.}$$
Por definição, $\imagem(f)\subset C$ é um contradomínio, e é também o menor possível.
Para cada $y\in \imagem(f)$, existe pelo menos um $x\in D$ tal que $f(x)=y$, cada $x$ com
essa propriedade é chamado de \grasA{preimagem}\index{preimagem} de $y$. 
 Cada ponto $x\in D$ possui uma única imagem em $C$, um $y\in C$ pode possuir uma
preimagem, mais de uma preimagem, ou nenhuma preimagem.

\begin{ex}
Considere a função seno na reta.
 Ao $x$ percorrer a reta real, $\sen x$ atinge todos os pontos do intervalo $[-1,1]$. Logo,
$\imagem(\sen)=[-1,1]$. Qualquer $y\in [-1,1]$ possui infinitas preimagens, por exemplo,
todos os pontos de $\{k\pi,k\in \bZ\}$ são preimagens de 
 $y=0$. O ponto $y=2$, por sua vez, não possui nenhuma preimagem (não existe $x\in \bR$
tal que $\sen x=2$).
\end{ex}

\begin{exo} Calcule o conjunto imagem das seguintes funções: 
\begin{multicols}{3}
\begin{enumerate}
\item\label{itconjimagem1} $-2x+1$, $D=\bR$
\item\label{itconjimagem2} $-2x+1$, $D=[-1,1]$
\item\label{itconjimagem21} $x^p$ ($p$ ímpar) 
\item\label{itconjimagem22} $x^p$ ($p$ par)
\item\label{itconjimagem3} $\tfrac{1}{x}$, $D=\bR\setminus \{0\}$
\item\label{itconjimagem4} $\tfrac{1}{x}$, $D=(0,\infty)$
\item\label{itconjimagem5a} $x^2+1$, $D=\bR$
\item\label{itconjimagem5} $1-x^2$, $D=\bR$
\item\label{itconjimagem51} $x^2+2x$, $D=(-\infty,0)$
\item\label{itconjimagem6} $\tan x$,
\item\label{itconjimagem7} $\sen x$, $D=[-\pisobredois,\pisobredois]$
\item\label{itconjimagem8} $\cos x$, $D=(-\pisobredois,\pisobredois)$
\item\label{itconjimagem801} $\tfrac13\sen x$, $D=\bR$
\item\label{itconjimagem81} $\sen (\pisobrequatro\sen x)$, $D=\bR$
\item\label{itconjimagem9} $\frac{1}{x^2+1}$, $D=\bR$ 
\item\label{itconjimagem10} $\begin{cases}x+1&\text{ se }x\geq0\\\tfrac12(x-1)&\text{ se } x<0\end{cases}$
\end{enumerate}
\end{multicols}
\vspace{0.01cm}
Faça a mesma coisa com as funções do Exercício \eqref{ExoEsbocosElementares}.
\begin{sol}
\eqref{itconjimagem1} $\imagem(f)=\bR$,
\eqref{itconjimagem2} $\imagem(f)=[-1,3]$,
\eqref{itconjimagem21} Se $p>0$ então $D=\bR$ e $\imagem(f)=\bR$. Se 
$p<0$ então $D=\bR\setminus\{0\}$ e $\imagem(f)=\bR\setminus\{0\}$
\eqref{itconjimagem22} $\imagem(f)=[0,\infty)$ se $p>0$, $\imagem(f)=(0,\infty)$ se $p<0$,
\eqref{itconjimagem3} $\imagem(f)=\bR\setminus \{0\}$,
\eqref{itconjimagem4} $\imagem(f)=(0,\infty)$,
\eqref{itconjimagem5a} $\imagem(f)=[1,\infty)$,
\eqref{itconjimagem5} $\imagem(f)=(-\infty,1]$,
\eqref{itconjimagem51} $\imagem(f)=[-1,\infty)$,
\eqref{itconjimagem6} $\imagem(f)=\bR$,
\eqref{itconjimagem7} $\imagem(f)=[-1,1]$,
\eqref{itconjimagem8} $\imagem(f)=(0,1]$,
\eqref{itconjimagem801} $\imagem(f)=[-\tfrac13,\tfrac13]$,
\eqref{itconjimagem81} $\imagem(f)=[-\tfrac{1}{\sqrt{2}},\tfrac{1}{\sqrt{2}}]$,
\eqref{itconjimagem9} $\imagem(f)=(0,1]$. De fato, $0<\frac{1}{1+x^2}\leq 1$. Melhor:
se $y\in (0,1]$ então $y=\frac{1}{1+x^2}$ possui uma única solução, dada por
$x=\sqrt{\frac{1-y}{y}}$.
\eqref{itconjimagem10} $\imagem(f)=(-\infty,-\tfrac12)\cup [1,\infty)$.

Para as funções do Exercício \ref{ExoEsbocosElementares}:
$\imagem(f)=(0,\infty)$,
$\imagem(g)=(-\infty,0]$,
$\imagem(h)=\bZ$,
$\imagem(i)=[0,1)$,
$\imagem(j)=[0,\infty)$.
\end{sol}
\end{exo}

\begin{exo}
Se $f(x)=\frac{2x}{x^2+25}$, calcule $\imagem(f)$. Para cada $y\in \imagem(f)$, determine
se $y$ possui uma única preimagem ou mais.
\begin{sol}
Se trata de achar todos os $y\in \bR$ para os quais existe pelo menos um $x\in
\bR$ tal que $f(x)=y$. Isso corresponde a resolver a equação do segundo grau em
$x$: $yx^2-2x+25y=0$. Se $y=0$, então $x=0$. Se $y\neq 0$,
$x=\frac{1\pm\sqrt{1-25y^2}}{y}$, que tem solução se e somente se
$|y|\leq\tfrac15$.
Logo, $\imagem(f)=[-\tfrac15,\tfrac15]$. O ponto $y=0$ é o único que possui uma única preimagem, qualquer outro ponto de $\imagem(f)$ possui duas preimagens. Isso pode ser verificado no gráfico:
\begin{center}
 \begin{bmlimage}\begin{tikzpicture}[scale=0.5]
\draw[->] (-13,0)--(13,0);
\draw[->] (0,-3)--(0,3);
\draw[thick, domain=-12.5:12.5, samples=100] plot (\x,{(60*\x)/(9*(\x)^2+25)});
\draw[dotted] (0,2)node[left]{$\tfrac15$}--(5/3,2);
\draw[dotted] (0,-2)node[right]{$\tfrac15$}--(-5/3,-2);
\draw[dotted,->] (0,1)node[left]{$y$}--(0.5,1)--(0.5,0);
\draw (0,1) node{$-$};
\draw[dotted,->] (0,1)--(0.5,1)--(0.5,0);
\draw[dotted,->] (0,1)--(0.5,1)--(0.5,0);
\draw[dotted,->] (0,1)--(6,1)--(6,0);
 \end{tikzpicture}\end{bmlimage}
\end{center}
\end{sol}
\end{exo}

\subsection{Bijeção, função inversa}

Diremos que uma função $f:D\to C$ é \grasA{bijetiva}\index{função! bijetiva} (ou
simplesmente: $f$ é uma \grasA{bijeção}) 
se
\begin{enumerate} 
\item $\imagem(f)=C$ (isto é, se $f$ atinge cada
ponto do seu contradomínio), e se 
\item 
qualquer $y\in C$ possui uma única preimagem, i.e.
existe um único $x\in D$ tal que
\eq{\label{eq:definverse}f(x)=y\,.} 
\end{enumerate}

Quando uma função é bijetiva, é possivel definir a sua \grasA{função
inversa}\index{função! inversa},
$f^{-1}:C\to D$, onde para todo $y\in C$, $f^{-1}(y)$ é definido como a única
solução $x$ de \eqref{eq:definverse}. A função inversa tem as seguintes
propriedades:
$$\boxed{\forall x\in D, 
\,(f^{-1}\circ f)(x)=x\,, \quad \text{ e } \forall y\in C,\, (f\circ
f^{-1})(y)=y\,.}$$
\begin{ex}\label{exemploinversao}
Considere a função do Exemplo \ref{Ex:retaegrafico}: $f(x)=\tfrac{x}{2}+1$ com $D=[0,2)$.
Então $\imagem(f)=[1,2)$, e $f:[0,2)\to [1,2)$ é uma bijeção:
\begin{center}
\begin{bmlimage}\begin{tikzpicture}
\begin{scope}
\draw [ ->] (0,-0.1)--(0,2.5) node[left]{$y$};
\draw [ ->] (-0.3,0)--(2.5,0) node[right]{$x$};
\draw [thick] (0,1)--(2,2);
\fill[intaberto] (2,2) circle (0.45mm);
%%
\draw[dotted] (0,2)--(2,2);
\draw (0,2) node[left]{$2$};
\draw (0,1) node[left]{$1$};
\draw (0,2) node{$-$};
\draw (0,1) node{$-$};
%%
\draw (0,0) node{$\shortmid$};
\fill (0,1) circle (0.45mm);
\pgfmathsetmacro{\x}{0.8};
\draw (\x,0) node{$\shortmid$};
\draw (\x,0) node[below]{$\scriptstyle{x}$};
\draw[dashed, ->] (\x,0)--(\x,{1+\x/2})--(0,{1+\x/2}) node[left]{$\scriptstyle{f(x)}$};
\draw (0,0) node[below]{$0$};
\draw (2,0) node[below]{$2$};
\draw[dotted] (2,0)--(2,2);
\end{scope}

\begin{scope}[xshift=7cm]
\draw [ ->] (0,-0.1)--(0,2.5) node[left]{$y$};
\draw [ ->] (-0.3,0)--(2.5,0) node[right]{$x$};
\draw [thick] (0,1)--(2,2);
\draw (0,0) node{$\shortmid$};
\fill (0,1) circle (0.45mm);
\fill[intaberto] (2,2) circle (0.45mm);
%%
\draw[dotted] (0,2)--(2,2);
\draw (0,2) node[left]{$2$};
\draw (0,1) node[left]{$1$};
\draw (0,2) node{$-$};
\draw (0,1) node{$-$};
%%
\pgfmathsetmacro{\x}{0.8};
\draw (\x,0) node{$\shortmid$};
\draw (\x,0) node[below]{$\scriptstyle{f^{-1}(y)}$};
\draw[dashed, color=gray!60, <-] (\x,0)--(\x,{1+\x/2})--(0,{1+\x/2}) node[left]{$\scriptstyle{y}$};
\draw (0,0) node[below]{$0$};
\draw (2,0) node[below]{$2$};
\draw[dotted] (2,0)--(2,2);
\end{scope}
\end{tikzpicture}\end{bmlimage}
\end{center}
Como $y=\tfrac{x}{2}+1$, a função inversa obtém-se isolando $x$: $x=2(y-1)$. Logo, 
$f^{-1}:[1,2)\to[0,2)$, 
$f^{-1}(y)=2(y-1)$. 
Para esboçar o gráfico da função inversa no plano cartesiano, é mais natural
\emph{renomear} a variável usada para representar $f^{-1}$, da seguinte maneira:
\begin{align*}
 f^{-1}:[1,2)&\to[0,2)\\
x&\mapsto 2(x-1)\,.
\end{align*}
Podemos agora esbocar $f^{-1}$:
\begin{center}
\begin{bmlimage}\begin{tikzpicture}
\draw [ ->] (0,-0.1)--(0,2.5);
\draw [ ->] (-0.3,0)--(2.5,0) node[right]{$x$};
%\draw [color=gray!30, ->] (0,1)--(2,2);
\draw [thick] (1,0)--(2,2);
\fill[intaberto] (2,2) circle (0.45mm);
%%
\draw[dotted] (0,2)--(2,2);
\draw (0,2) node[left]{$2$};
\draw (1,0) node[below]{$1$};
%\draw (0,2) node{$-$};
\draw (1,0) node{$\shortmid$};
%%
\draw (0,0) node{$\shortmid$};
\fill (1,0) circle (0.45mm);
\pgfmathsetmacro{\x}{1.4};
\draw (\x,0) node{$\shortmid$};
\draw (\x,0) node[below]{$\scriptstyle{x}$};
\draw[dashed, ->] (\x,0)--(\x,{2*(\x-1)})--(0,{2*(\x-1)}) node[left]{$\scriptstyle{f^{-1}(x)}$};
\draw (0,0) node[below]{$0$};
\draw (2,0) node[below]{$2$};
\draw[dotted] (2,0)--(2,2);
\end{tikzpicture}\end{bmlimage}
\end{center}
É importante observar que o gráfico da função inversa obtém-se a partir do
gráfico de $f$ por uma \emph{simetria através da diagonal do primeiro quadrante}:
\begin{center}
\begin{bmlimage}\begin{tikzpicture}
\draw [ ->] (0,-0.1)--(0,2.5);
\draw [ ->] (-0.3,0)--(2.5,0);
\draw [color=gray!30, ->] (0,1)--(2,2);
\draw [thick] (1,0)--(2,2);
\fill[intaberto] (2,2) circle (0.45mm);
%%
\draw[dotted] (0,2)--(2,2);
\draw (0,2) node[left]{$2$};
\draw (1,0) node[below]{$1$};
%\draw (0,2) node{$-$};
\draw (1,0) node{$\shortmid$};
%%
\draw (0,0) node{$\shortmid$};
\fill (1,0) circle (0.45mm);
\fill[color=gray!30] (0,1) circle (0.45mm);
\pgfmathsetmacro{\x}{1.4};
%\draw (\x,0) node{$\shortmid$};
\draw[dashed] (-0.2,-0.2)--(2.2,2.2);
%\draw (\x,0) node[below]{$\scriptstyle{x}$};
%\draw[dashed, ->] (\x,0)--(\x,{2*(\x-1)})--(0,{2*(\x-1)})
%node[left]{$\scriptstyle{f^{-1}(x)}$};
\draw (0,0) node[below]{$0$};
\draw (2,0) node[below]{$2$};
\draw[dotted] (2,0)--(2,2);
\end{tikzpicture}\end{bmlimage}
\end{center}
\end{ex}

Vimos no último exemplo 
que o gráfico de $f^{-1}$ é obtido a partir do
gráfico de $f$ 
por uma {simetria através da diagonal do primeiro quadrante}. Isso vale
em geral. De fato, se um ponto
$(x,y=f(x))$ pertence ao gráfico de $f$, então $(y,x=f^{-1}(y))$ pertence ao gráfico de
$f^{-1}$.

\begin{ex}
Considere $f(x)=1-x^2$. 
\begin{center}
\begin{bmlimage}\begin{tikzpicture}
\draw [thick, domain=-1:1] plot (\x,{1-(\x)^2});
\draw (-1,1) node{$1)$};
\fill (-1,0) circle (0.45mm);
\fill (1,0) circle (0.45mm);
\draw (-1,0) node{$\shortmid$};
\draw [ ->] (-1.5,0)--(1.5,0);
\draw [ ->] (0,-0.1)--(0,1.3);
\pgfmathsetmacro{\x}{0.6};
\draw[dotted, ->] (\x,0)--(\x,{1-(\x)^2})--(0,{1-(\x)^2});
\draw (\x,0) node[below]{$\scriptstyle{x}$};
\draw[dotted, ->] (-\x,0)--(-\x,{1-(\x)^2})--(0,{1-(\x)^2});
\draw (-\x,0) node[below]{$\scriptstyle{-x}$};
\begin{scope}[xshift=7cm]
 \draw [color=gray!30, domain=-1:0] plot (\x,{1-(\x)^2});
\draw (-1,1) node{$2)$};
\draw [thick, domain=0:1] plot (\x,{1-(\x)^2});
\fill (0,1) circle (0.45mm);
\fill (1,0) circle (0.45mm);
\draw (-1,0) node{$\shortmid$};
\draw [ ->] (-1.5,0)--(1.5,0);
\draw [ ->] (0,-0.1)--(0,1.3);
\pgfmathsetmacro{\x}{0.6};
\draw[dotted, ->] (\x,0)--(\x,{1-(\x)^2})--(0,{1-(\x)^2});
\draw (\x,0) node[below]{$\scriptstyle{x}$};
\end{scope}
\end{tikzpicture}\end{bmlimage}
\end{center}
1) Com $D=[-1,1]$, temos $\imagem(f)=[0,1]$. Mas 
 como $1-(-x)^2=1-x^2$, cada ponto do contradomínio (diferente de zero) possui exatamente
\emph{duas}
preimagens, logo $f:[-1,1]\to [0,1]$ não é bijetiva. 2) Mas, ao \emph{restringir o
domínio}, $D=[0,1]$, então $f:[0,1]\to [0,1]$, $f$ se torna bijetiva. O seu inverso se
acha resolvendo $y=1-x^2$: $x=\sqrt{1-y}$. Assim, a sua função inversa é dada por 
$f^{-1}:[0,1]\to [0,1]$, $f^{-1}(y)=\sqrt{1-y}$.
\end{ex}



\begin{exo}
Mostre que a função 
\begin{align*}
 f:(-1,0)&\to (0,1)\\
x&\mapsto \sqrt{1-x^2}
\end{align*}
é bijetiva, e calcule $f^{-1}$. Esboce o gráfico de $f^{-1}$.
\begin{sol}
Observe que se $x\in (-1,0)$, então $f(x)\in (0,1)$. Por outro lado, se $y\in (0,1)$,
então existe um único $x\in (-1,0)$ tal que $f(x)=y$: $x=-\sqrt{1-y^2}$.
Logo, $f^{-1}:(0,1)\to(-1,0)$, $f^{-1}(x)=-\sqrt{1-x^2}$.
\begin{center}
\begin{bmlimage}\begin{tikzpicture}
 \draw[ ->] (-1.1,0)--(1.1,0) node[right]{$\scriptstyle{x}$};
 \draw[ ->] (0,-0.1)--(0,1.2) node[right]{$\scriptstyle{f(x)}$};
\draw[thick, <->] (0,1) arc (90:180:1);
\pgfmathsetmacro{\x}{-0.8};
\draw (\x,0) node[below]{$\scriptstyle{f^{-1}(y)}$};
\draw[dotted, <-] (\x,0)--(\x,{sqrt(1-(\x)^2)})--(0,{sqrt(1-(\x)^2)}) node[right]{$\scriptstyle{y}$};

\begin{scope}[xshift=5cm, yshift=1cm]
  \draw[ ->] (-1.1,0)--(1.3,0)node[right]{$\scriptstyle{x}$};
  \draw[ ->] (0,-1.2)--(0,0.4)node[left]{$\scriptstyle{f^{-1}(x)}$};
\draw[thick, <->] (0,-1) arc (270:360:1);
\pgfmathsetmacro{\x}{0.6};
\draw (\x,0) node[above]{$\scriptstyle{x}$};
\draw[dotted, ->] (\x,0)--(\x,{-sqrt(1-(\x)^2)})--(0,{-sqrt(1-(\x)^2)}) node[left]{$\scriptstyle{f^{-1}(x)}$}; 
\end{scope}
 \end{tikzpicture}\end{bmlimage}
\end{center}
\end{sol}
\end{exo}

\begin{exo}
Considere $f:(-1,\infty)\to \bR$, $f(x)=\frac{1}{x+1}$. A partir do gráfico de $f$, dê o
seu conjunto imagem, e mostre que $f:(-1,\infty)\to \imagem(f)$ 
é uma bijeção. Em seguida, dê a sua função inversa.
\begin{sol}
O gráfico de $\frac{1}{x+1}$ é o de $\tfrac1x$ transladado de uma unidade para a esquerda.
O conjunto imagem é $(0,\infty)$. De fato, para todo $y\in (0,\infty)$, a equação $y=\frac{1}{x+1}$
possui uma solução dada por $x=\frac{1-y}{y}$. Logo, $f^{-1}:(0,\infty)\to (-1,\infty)$,
$f^{-1}(x)=\frac{1-x}{x}$.
\end{sol}
\end{exo}

\begin{exo}
Seja $f:\bR\to \bR$ uma bijeção ímpar. Mostre que a sua função inversa
$f^{-1}:\bR\to\bR$ é ímpar também.  
\begin{sol}
Para verificar que $f^{-1}(-y)=-f^{-1}(y)$, usemos a definição: seja $x$ o único
$x$ tal que $f^{-1}(-y)=x$. Pela definição de função inversa ($(f\circ
f^{-1})(y)=y$), aplicando
$f$ temos $-y=f(x)$. Portanto, $y=-f(x)=f(-x)$ (pela imparidade de $f$).
Aplicando agora $f^{-1}$ obtemos $f^{-1}(y)=-x$, isto é, $x=-f^{-1}(y)$. Isso
mostra que
$f^{-1}(-y)=-f^{-1}(y)$.
\end{sol}
\end{exo}

\begin{exo}
%(do Prof. Michel Spira)
%Michel
Para cada um dos contradomínios $C$ a seguir, dê um exemplo explícito de bijeção
$f:(0,1)\to C$.
\begin{multicols}{3}
\begin{enumerate}
 \item\label{itexobijecoes1} $(0,b)$, onde $b>0$.
\item\label{itexobijecoes2} $(a,b)$, onde $a<b$.
\item\label{itexobijecoes3} $(0,\infty)$
\item\label{itexobijecoes4} $(-\infty,\infty)$
\item\label{itexobijecoes5} $(0,1)$
%\item $[0,1]$ (difícil!)
\end{enumerate}
\end{multicols}
\vspace{0.01cm}
\begin{sol}
Exemplos:
\eqref{itexobijecoes1} $f(x)=bx$
\eqref{itexobijecoes2} $f(x)=a+(b-a)x$
\eqref{itexobijecoes3} $f(x)=\tan \pisobredois{x}$, ou $f(x)=\tfrac{1}{(x-1)^2}-1$
\eqref{itexobijecoes4} $f(x)=\tan (\tfrac{2}{\pi}(x-\tfrac12))$
\end{sol}
\end{exo}

\begin{exo}
Sejam $f(x)$ e $g(x)$, $x\in \bR$, definidas por 
$$
f(x)\pardef \lfloor x \rfloor+(x-\lfloor x\rfloor)^2\,,\quad
g(x)\pardef \lfloor x \rfloor+\sqrt{x-\lfloor x\rfloor}\,.
$$
Mostre que $g=f^{-1}$.
\end{exo}

\subsection{Inversos das potências}\label{Sec:InversoPotencias}\index{potência! inverso
de}
Vimos que se $p$ é par, então a função $f(x)=x^p$ é par, e 
$\imagem(f)=[0,\infty)$ ou $(0,\infty)$ (dependendo de $p$ ser $>0$ ou $<0$). 
Logo, para serem invertidas, o domínio delas precisa ser restringido. 
Escolheremos (para $p$ par)
\begin{align*}
 f:[0,\infty)&\to [0,\infty)\\
x&\mapsto x^p\,.
\end{align*}
Vemos que com essa restrição, $f$ se torna 
bijetiva: para cada $y\in [0,\infty)$ existe um único $x\in [0,\infty)$ tal que $x^p=y$.
Esse $x$ costuma ser denotado por $x=y^{1/p}$:
\begin{align*}
 f^{-1}: [0,\infty)&\to[0,\infty)\\
y&\mapsto y^{1/p}\,.
\end{align*}
No caso $p=2$, $y^{1/2}= \sqrt{y}$ é a função \grasA{raiz quadrada}. 
\begin{center}
\begin{bmlimage}\begin{tikzpicture}[scale=1]
\pgfmathsetmacro{\a}{1.2}
\draw [color=gray!30, dotted, domain=0:\a, samples=100] plot (\x,{\x^4});
\pgfmathsetmacro{\a}{1.12}
\draw [color=gray!30, dashed, domain=0:\a, samples=100] plot (\x,{\x^6});
\pgfmathsetmacro{\a}{1.4}
\draw [color=gray!30, domain=0:\a, samples=100] plot (\x,{(\x)^2});
%\fill (-1,1) circle (0.35mm);
%\fill (1,1) circle (0.35mm);
\pgfmathsetmacro{\a}{1.41}
\draw [thick, dotted, domain=0:\a, samples=100] plot ({\x^4},\x);
\pgfmathsetmacro{\a}{1.26}
\draw [thick, dashed, domain=0:\a, samples=100] plot ({\x^6},\x);
\pgfmathsetmacro{\a}{2}
\draw [thick, domain=0:\a, samples=100] plot ({(\x)^2},\x) node[right]{$\sqrt{x}$};
\draw [ ->] (-0.2,0)--(4.2,0) node[right]{$x$};
\draw [ ->] (0,-0.2)--(0,{2}) node[left]{$x^{1/p}$};
\pgfmathsetmacro{\b}{7};
\pgfmathsetmacro{\c}{1.5};
\draw (\b,\c) node[left]{$p=2:$};
\draw[thick] (\b,\c)--(\b+1,\c);
\draw (\b,\c-0.5) node[left]{$p=4:$};
\draw[thick, dotted] (\b,\c-0.5)--(\b+1,\c-0.5);
\draw (\b,\c-1) node[left]{$p=6:$};
\draw[thick, dashed] (\b,\c-1)--(\b+1,\c-1);
\end{tikzpicture}\end{bmlimage}
\end{center}

Se $p>0$ for ímpar, $\imagem(f)=\bR$ e não é preciso restringir o seu domínio: 
\begin{align*}
 f:\bR &\to \bR\\
x&\mapsto x^p
\end{align*}
é bijetiva, e o seu inverso tem o seguinte gráfico:
\begin{center}
\begin{bmlimage}\begin{tikzpicture}[scale=1]
\pgfmathsetmacro{\a}{1.15}
\draw [color=gray!30, dotted, domain=-\a:\a, samples=100] plot (\x,{\x^3});
\pgfmathsetmacro{\a}{1.1}
\draw [color=gray!30, dashed, domain=-\a:\a, samples=100] plot (\x,{\x^5});
\pgfmathsetmacro{\a}{1.3}
\draw [thick, dotted, domain=-\a:\a, samples=100] plot ({\x^3},\x);
\pgfmathsetmacro{\a}{1.2}
\draw [thick, dashed, domain=-\a:\a, samples=100] plot ({\x^5},\x);
\draw [ ->] (-2.5,0)--(2.5,0) node[right]{$x$};
\draw [ ->] (0,-1.5)--(0,1.5) node[left]{$x^{1/p}$};
\pgfmathsetmacro{\b}{5};
\pgfmathsetmacro{\c}{1.5};
%\draw (\b,\c) node[left]{$p=1:$};
%\draw[thick] (\b,\c)--(\b+1,\c);
\draw (\b,\c-0.5) node[left]{$p=3:$};
\draw[thick, dotted] (\b,\c-0.5)--(\b+1,\c-0.5);
\draw (\b,\c-1) node[left]{$p=5:$};
\draw[thick, dashed] (\b,\c-1)--(\b+1,\c-1);
\end{tikzpicture}\end{bmlimage}
\end{center}


\begin{exo}
Complete essa discussão, incluindo os valores negativos de $p$.
\end{exo}

\begin{exo}
Resolva:
\begin{multicols}{3}
\begin{enumerate}
\item\label{itexoresolvagraf1} $x>\sqrt{x+2}$
\item\label{itexoresolvagraf2} $(x-1)^2\leq \sqrt{1-x}$
\item\label{itexoresolvagraf3} $\sqrt{-x^2-x+6}=-(x+1)$
\end{enumerate}
\vspace{0.04cm}
\end{multicols}
\begin{sol}
\ref{itexoresolvagraf1} $S=(\frac{1+\sqrt{5}}{2},+\infty)$
\ref{itexoresolvagraf2} $S=[0,1]$
\ref{itexoresolvagraf2} $S=\{-\tfrac52\}$
\end{sol}
\end{exo}

\subsection{Funções trigonométricas inversas}\label{Sec:Functriginversas}
Vimos que para a função $\sen: \bR\to [-1,1]$, 
um $y\in [-1,1]$ possui infinitas preimagens, logo não é bijeção. 
Portanto, para \emph{inverter} a função seno, é necessário restringir o seu domínio. A restringiremos ao intervalo  $[-\pisobredois,\pisobredois]$:
\begin{center}
\begin{bmlimage}\begin{tikzpicture}
\draw [->] (-3.5,0)--(3.5,0) node[right]{$x$};
\draw [->] (0,-1.3)--(0,1.3) node[right]{$\sen x$};
%\draw [color=gray!30, domain=-3.5:3.5] plot (\x,{sin(\x r)}); 
\draw [thick, domain=-1.57:1.57, samples=50] plot (\x,{sin(\x r)}); 
\draw (-1.57,0) node[above]{$\scriptstyle{-\pisobredois}$};
\draw (1.57,0) node[below]{$\scriptstyle{\pisobredois}$};
\draw [dashed] (-1.57,0)--(-1.57,-1);
\draw [dashed] (1.57,0)--(1.57,1);
\draw [dashed] (0,1) node[left]{$\scriptstyle{1}$}--(1.57,1);
\draw [dashed] (0,-1) node[right]{$\scriptstyle{-1}$}--(-1.57,-1);
\pgfmathsetmacro{\a}{0.8};
\draw [dotted, <->] (\a,0) node[below]{$\scriptscriptstyle{x}$}--(\a,{sin(\a r)})--(0,{sin(\a r)}) node[left]{$\scriptstyle{y}$};
\fill (1.57,1) circle (0.45mm);
\fill (-1.57,-1) circle (0.45mm);
\end{tikzpicture}\end{bmlimage}
\end{center}
De fato, com essa restrição,
\begin{align*}
\sen :[-\pisobredois,\pisobredois]&\to [-1,1]\\
x&\mapsto \sen x
\end{align*}
é uma bijeção, pois cada $y\in [-1,1]$ é atingido e possui uma única preimagem. A função
inversa é chamada \grasA{arcseno}\index{$\arcsen$}, e denotada
\begin{align*}
\arcsen :[-1,1]&\to [-\pisobredois,\pisobredois]\\
y&\mapsto \arcsen y\,.
\end{align*}
Pela sua definição, ela satisfaz:
\begin{equation}
\boxed{ \forall y\in [-1,1]:\,\sen (\arcsen y)=y\,\,,\quad\text{ e } 
\quad\forall x\in [-\pisobredois,\pisobredois]:\,\arcsen(\sen
x)=x\,\,.}
\end{equation}

 O gráfico de $\arcsen$ pode ser obtido por uma reflexão do gráfico de $\sen$ pela
diagonal do primeiro quadrante:
\begin{center}
\begin{bmlimage}\begin{tikzpicture}
\draw [->] (-1.3,0)--(1.3,0) node[right]{$x$};
\draw [->] (0,-1.7)--(0,1.7) node[above]{$\arcsen x$};
%\draw [color=gray!30, domain=-3.5:3.5] plot ({sin(\x r)},\x); 
\draw [thick, domain=-1.57:1.57, samples=50] plot ({sin(\x r)},\x); 
\draw [dashed] (1,0) node[below]{$\scriptstyle{1}$}--(1,1.57)--(0,1.57) node[left]{$\scriptstyle{\pisobredois}$};
\draw [dashed] (-1,0) node[above]{$\scriptstyle{-1}$}--(-1,-1.57)--(0,-1.57) node[right]{$\scriptstyle{-\pisobredois}$};
\pgfmathsetmacro{\a}{0.8};
% \draw [dotted, <->] (0,\a) node[below]{$\scriptscriptstyle{x}$}--(\a,{sin(\a r)})--(0,{sin(\a r)}) node[left]{$\scriptstyle{y}$};
\fill (1,1.57) circle (0.45mm);
\fill (-1,-1.57) circle (0.45mm);
\end{tikzpicture}\end{bmlimage}
\end{center}
\begin{obs} (Já fizemos esse comentário no Exemplo \ref{exemploinversao}.)
 Como $\arcsen$ é definida como a função inversa de $x\mapsto \sen x$ (no intervalo
$[-\pisobredois, \pisobredois]$), o mais correto é escrevê-la $y\mapsto \arcsen y$.
 Mas para esboçar o seu gráfico, faz mais sentido usar a notação habitual, em que o eixo
das abscissas é chamado de ``$x$''. Por isso, esse último gráfico representa o gráfico da
função $\arcsen$, mas chamando a sua variável $x$ (em vez de $y$): 
$x\mapsto \arcsen x$.
Faremos a mesma modificação nos próximos gráficos.
\end{obs}

\begin{exo}
Seja $y\in (0,\pisobredois)$ tal que $y=\arcsen\tfrac35$. Calcule $\sen y$, $\cos y$, e
$\tan y$.
\begin{sol}
Por definição, $\sen y=\tfrac35$. Logo, $\cos y=+\sqrt{1-\sen^2 y}=\tfrac45$ (a raiz
positiva é escolhida, já que $y\in (0,\pisobredois)$). Portanto, $\tan y=\tfrac34$.
\end{sol}
\end{exo}

O cosseno pode ser invertido também, uma vez que o seu domínio é bem escolhido:
\begin{align*}
\cos :[0,\pi]&\to [-1,1]\\
x&\mapsto \cos x
\end{align*}
\begin{center}
\begin{bmlimage}\begin{tikzpicture}
\draw [->] (-0.7,0)--(3.7,0) node[right]{$x$};
\draw [->] (0,-1.3)--(0,1.3) node[right]{$\cos x$};
%\draw [color=gray!30, domain=-1:3.7] plot (\x,{cos(\x r)}); 
\draw [thick, domain=0:3.14, samples=50] plot (\x,{cos(\x r)}); 
%\draw (-1.57,0) node[above]{$\scriptstyle{-\pisobredois}$};
\draw (3.14,0) node[above right]{$\scriptstyle{\pi}$};
\draw [dashed] (0,1) node[left]{$\scriptstyle{1}$}--(3.14,1)--(3.14,0);
\draw [dashed] (0,-1) node[left]{$\scriptstyle{-1}$}--(3.14,-1)--(3.14,0);
\pgfmathsetmacro{\a}{0.8};
 \draw [dotted, <->] (\a,0) node[below]{$\scriptscriptstyle{x}$}--(\a,{cos(\a
r)})--(0,{cos(\a r)}) node[left]{$\scriptstyle{y}$};
\fill (0,1) circle (0.45mm);
\fill (3.14,-1) circle (0.45mm);
\end{tikzpicture}\end{bmlimage}
\end{center}
A função inversa é chamada \grasA{arcosseno}\index{$\arcos$}, e denotada
\begin{align*}
\arcos :[-1,1]&\to [0,\pi]\\
y&\mapsto \arcos y\,.
\end{align*}
Ela possui as propriedades:
\begin{equation}\label{eq:identarseno1}
\boxed{ \forall y\in [-1,1]:\,\cos (\arcos y)=y\,\,,\quad\text{ e } 
\quad\forall x\in [0,\pi]:\,\arcos(\cos x)=x\,\,.}
\end{equation}

O gráfico de $\arcos$ pode ser obtido por uma reflexão pela diagonal do primeiro quadrante:

\begin{center}
\begin{bmlimage}\begin{tikzpicture}
\draw [->] (-1.3,0)--(1.3,0) node[right]{$x$};
\draw [->] (0,-0.3)--(0,3.5) node[above]{$\arcos x$};
%\draw [color=gray!30, domain=-3.5:3.5] plot ({sin(\x r)},\x); 
\draw [thick, domain=0:3.14, samples=50] plot ({cos(\x r)},\x); 
 \draw  (1,0) node[below]{$\scriptstyle{1}$};
\draw [dashed] (-1,0) node[below]{$\scriptstyle{-1}$}--(-1,3.14)--(0,3.14) node[right]{$\scriptstyle{\pi}$};
\pgfmathsetmacro{\a}{0.8};
\fill (1,0) circle (0.45mm);
\fill (-1,3.14) circle (0.45mm);
\end{tikzpicture}\end{bmlimage}
\end{center}


Para inverter a tangente, faremos a restrição 
\begin{align*}
 \tan :(-\pisobredois,\pisobredois)&\to \bR\\
x&\mapsto \tan x\,,
\end{align*}
obtendo assim uma bijeção.
\begin{center}
\begin{bmlimage}\begin{tikzpicture}
\pgfmathsetmacro{\eps}{0.3};
\pgfmathsetmacro{\pisd}{1.5707};
\pgfmathsetmacro{\h}{3};
\begin{scope}
%\draw[color=gray!70, dashed] (-3*\pisd,-\h)--(-3*\pisd,+\h);
%\draw [color=gray!70, domain={-3*\pisd+\eps}:{-\pisd-\eps}, samples=100] plot
%(\x,{(sin(\x r))/(cos(\x r))});
\draw[dashed] (-\pisd,-\h)--(-\pisd,+\h);
\draw [thick, domain={-\pisd+\eps}:{\pisd-\eps}, samples=100] plot (\x,{(sin(\x r))/(cos(\x r))});
\draw[dashed] (\pisd,-\h)--(\pisd,+\h);
%\draw [color=gray!70, domain={\pisd+\eps}:{3*\pisd-\eps}, samples=100] plot (\x,{(sin(\x
%r))/(cos(\x r))});
%\draw[color=gray!70, dashed] (3*\pisd,-\h)--(3*\pisd,+\h);
\draw [ ->] (-6,0)--(6,0) node[right]{$x$};
\draw [ ->] (0,-2.3)--(0,{2.3}) node[above]{$\tan x$};
\draw[dotted, <->] (1,0) node[below]{$x$}--(1,{tan(1 r)})--(0,{tan (1 r)}) node[left]{$y$};
\fill (1,{tan(1 r)}) circle (0.28mm);
\end{scope}
\end{tikzpicture}\end{bmlimage}
\end{center}
A função inversa é chamada de \grasA{arctangente}\index{$\arctan$}:
\begin{align*}
 \arctan :\bR&\to (-\pisobredois,\pisobredois)\\
y&\mapsto \arctan y\,.
\end{align*}
Como antes,
\begin{equation}
\boxed{\forall x\in (-\pisobredois,\pisobredois):\,\arctan (\tan 
x)=x\,\,,\quad\text{ e }\quad \forall
y\in \bR:\,\tan(\arctan y)=y\,\,.}
\end{equation}
 O seu gráfico possui duas \emph{assíntotas horizontais}: quando $x$ é 
positivo e grande, o gráfico de $\arctan x$ se aproxima da reta de equação 
$y=\pisobredois$, e quando $x$ é
negativo e grande, ele  se aproxima da reta de equação $y=-\pisobredois$:
\begin{center}
\begin{bmlimage}\begin{tikzpicture}
\pgfmathsetmacro{\eps}{0.2};
\pgfmathsetmacro{\pisd}{1.5707};
\pgfmathsetmacro{\h}{5};
\begin{scope}
\draw[dashed] (-\h,-\pisd)--(\h,-\pisd);
 \draw [thick, domain={-\pisd+\eps}:{\pisd-\eps}, samples=100] plot ({(sin(\x r))/(cos(\x
r))},\x);
\draw[dashed] (-\h,\pisd)--(\h,\pisd);
\draw [->] (-\h,0)--(\h,0) node[right]{$x$};
\draw [ ->] (0,-2)--(0,2) node[right]{$\arctan x$};
%\draw[dotted, <->] (0,1) node[below]{$x$}--({tan(1 r)},1)--({tan (1 r)},0) node[left]{$y$};
\end{scope}
\end{tikzpicture}\end{bmlimage}
\end{center}
Observemos também que $\arctan$ é uma função ímpar, limitada por $\pisobredois$. 

\begin{obs}
 É importante notar que as três funções trigonométricas inversas, $\arcsen$ $\arcos$ e
$\arctan$, foram definidas a partir de uma \emph{escolha} de uma restrição para cada uma
das funções $\sen$, $\cos$ e $\tan$.
Essa escolha pode parecer arbitrária, mas é a mais comum usada nos livros de matemática.
Continuaremos usando as funções inversas assim definidas, até o fim do curso. 
\end{obs}

\begin{exo}
Determine os domínios das seguintes funções.
\begin{multicols}{2}
\begin{enumerate}
\item\label{itdomintriginv1} $\arcos x -\arcsen x$
\item\label{itdomintriginv2} $\arcos (2x)$
\item\label{itdomintriginv3} $\tan (\arcsen x)$
\item\label{itdomintriginv4} $\arcos(\tfrac{2-x^2}{1+x^2})$
\end{enumerate}
\end{multicols}
\vspace{0.1mm}
\begin{sol}
\eqref{itdomintriginv1} $[-1,1]$,
\eqref{itdomintriginv2} $[-\tfrac12,\tfrac12]$,
\eqref{itdomintriginv3} $(-1,1)$,
\eqref{itdomintriginv4} $(-\infty,-\tfrac{1}{\sqrt{2}}]\cup
[\tfrac{1}{\sqrt{2}},+\infty)$.
\end{sol}
\end{exo}

\begin{exo}\label{Exo:Telao}
Uma tela de cinema de $5$ metros de altura está pregada numa parede, $3$ metros acima do chão.
a) Se $P$ é um ponto no chão a distância $x$ da parede, calcule o ângulo $\theta$ sob o qual 
$P$ vê a tela, em função de $x$. 
b) Mesma coisa se $P$ é a $2$ metros do chão.
(Obs: no Exercício \ref{Exo:TelaoBIS} calcularemos onde colocar o ponto $P$ de modo tal que o ângulo 
seja máximo.)
\begin{sol}
Seja $A$ a posição do topo da tela, $B$ a sua base, e $Q$ o ponto onde a parede toca o chão.
Seja $\alpha$ o ângulo $APQ$ e $\beta$ o ângulo $BPQ$.
Temos $\tan \alpha=\tfrac8x$, $\tan \beta=\tfrac3x$. Logo, em a): 
$\theta(x)=\arctan\tfrac8x-\arctan\tfrac3x$. Em
b), $\theta(x)=\arctan\tfrac6x-\arctan\tfrac1x$. 
\end{sol}
\end{exo}

\begin{exo}
Resolva:
\begin{multicols}{3}
\begin{enumerate}
\item\label{itexoinvtrig1} $3\arcsen x=\pisobredois$
\item\label{itexoinvtrig2} $\arctan (x-1)=\tfrac{\pi}{3}$
\item\label{itexoinvtrig3} $2\sen (\arcsen x)=\tfrac13$
\item\label{itexoinvtrig4} $\arctan(\tan (x^2))=\tfrac{\pi}{9}$
\end{enumerate}
\end{multicols}
\vspace{0.01cm}
\begin{sol}
\eqref{itexoinvtrig1} $x=\frac{1}{2}$
\eqref{itexoinvtrig2} $x=\sqrt{3}+1$
\eqref{itexoinvtrig3} $x=\tfrac16$
\eqref{itexoinvtrig4} $x=\tfrac{\sqrt{\pi}}{3}$
\end{sol}
\end{exo}

As funções trigonométricas inversas têm identidades associadas. Somente consideraremos algumas:
\begin{ex}\label{Ex:identidadesenoinverso}
Provemos, por exemplo, a identidade
\eq{\cos(\arcsen x)=\sqrt{1-x^2}\,,\quad\forall x\in [-1,1]\,.}
Primeiro, como
$\sen^2\alpha+\cos^2\alpha=1$, temos, usando \eqref{eq:identarseno1}, 
$$\cos^2(\arcsen
x)=1-\sen^2(\arcsen x)=1-x^2\,.$$
Mas como $-\pisobredois\leq \arcsen x\leq \pisobredois$, vale 
$\cos(\arcsen x)\in [0,1]$, logo, tomando a raiz quadrada dá a 
idendidade desejada.
Um outro jeito de entender a identidade é de escrevê-la como 
$\cos(\arcsen x)=\cos \alpha$, onde 
 $\alpha=\arcsen x$. Logo, $\sen \alpha=x$, o que pode ser representado 
num triângulo:

\begin{center}
\begin{bmlimage}\begin{tikzpicture}[scale=2]
\draw (0,0)--(1,0)--(1,0.5) node[midway, right]{$\scriptstyle{x}$};
\draw (1,0.5)--(0,0) node[midway, above]{$\scriptstyle{1}$};
\draw[ ->] (0.3,0) arc (0:28:0.3);
\draw (0.43,0.1) node{$\scriptstyle{\alpha}$};
\end{tikzpicture}\end{bmlimage}
\end{center}
Nesse triângulo vemos que $\cos \alpha=\tfrac{\sqrt{1-x^2}}{1}=\sqrt{1-x^2}$.
\end{ex}

\begin{exo} Simplifique:
\begin{multicols}{3}
 \begin{enumerate}
\item\label{itidenttriginv1} $\cos(2\arcos x)$
\item\label{itidenttriginv2} $\cos(2\arcsin x)$
\item\label{itidenttriginv3} $\sen(2\arcos x)$
\item\label{itidenttriginv4} $\cos(2\arctan x)$
\item\label{itidenttriginv5} $\sen (2\arctan x)$
\item\label{itidenttriginv6} $\tan (2\arcsen x)$
 \end{enumerate}
\end{multicols}
\vspace{0.01cm}
\begin{sol}
\eqref{itidenttriginv1} $\cos(2\arcos x)=2\cos^2(\arcos x)-1=2x^2-1$
\eqref{itidenttriginv2} $\cos(2\arcsin x)=1-2\sen^2(\arcsen x)=1-2x^2$
\eqref{itidenttriginv3} $\sen(2\arcos x)=2\sen (\arcos x)\cos (\arcos x)=2x\sqrt{1-x^2}$
\eqref{itidenttriginv4} $\cos(2\arctan x)=2\cos^2(\arctan x)-1=\tfrac{1-x^2}{1+x^2}$
\eqref{itidenttriginv5} $\sen (2\arctan x)=\frac{2x}{1+x^2}$
\eqref{itidenttriginv6} $\tan (2\arcsen x)=\frac{2x\sqrt{1-x^2}}{1-2x^2}$
\end{sol}
\end{exo}

\begin{exo}
Mostre que para todo $x\in [-1,1]$, 
$$
\arcsen x+\arcos x=\tfrac{\pi}{2}\,.
$$
\begin{sol}
Chamando $\alpha=\arcsen x$, $\beta=\arcos x$, temos $x=\sen \alpha$, $x=\cos \beta$:
\begin{center}
\begin{bmlimage}\begin{tikzpicture}[scale=2]
\pgfmathsetmacro{\a}{1};
\draw[dotted] (\a,0) arc (0:90:\a);
\draw[ ->, color=gray!70] (0,0) -- (1.1*\a,0);
\draw[ ->, color=gray!70] (0,0) -- (0,1.1*\a);
\pgfmathsetmacro{\alf}{35};
\coordinate (P) at ({0.4*\a*cos(\alf)},{0.4*\a*sin(\alf)});
\draw[<-] (P) arc (\alf:90:{0.4*\a});
\draw[->] ({0.4*\a},0) arc (0:\alf:{0.4*\a});
\draw ({\alf/2}:{0.45*\a}) node{$\alpha$};
\draw ({\alf+(90-\alf)/2}:{0.35*\a}) node[above right]{$\beta$};
\coordinate (B) at ({\a*cos(\alf)},{\a*sin(\alf)});
\coordinate (Bx) at ({\a*cos(\alf)},0);
\coordinate (By) at (0,{\a*sin(\alf)});
\draw (0,0)--(B);
\draw [thick] (B)--(Bx) node[midway, above, sloped]{$x$};
\draw [thick] (By)--(B);
\draw [thick] (By)--(0,0) node[midway, below, sloped]{$x$};
\end{tikzpicture}\end{bmlimage}
\end{center}
\end{sol}
\end{exo}

%VER EXOS DOUCHET p.54









% !TeX spellcheck = pt_BR
% !TEX encoding = UTF-8 Unicode

\chapter{Exponencial e logaritmo}\label{CAP:ExponLog}

\ifdefined\updateans
% Only need to run once in a lifetime, when the file ans.tex needs to be updated.
\Writetofile{ans}{\protect\section*{Capítulo \ref{CAP:ExponLog}}}
\fi

%\citacao{$S=k_B\log W$}{L. Boltzmann}

O objetivo nesse capítulo é definir e descrever as principais
propriedades de uma das funções mais importantes da matemática, a
\grasA{exponencial de base $a$},\index{função!
exponencial}\index{exponencial! na base $e$} 
%\inputimage
e da sua função inversa, o \grasA{logaritmo na base $a$},\index{função!
logaritmo}\index{logaritmo! na base $a$}
%inputimage

Os exemplos de uso dessas duas funções em ciências são inúmeros. Vejamos dois
exemplos onde elas aparecem nos axiomas de uma teoria:

\begin{ex}\label{Ex:Fisstat}\index{física estatística}
Em \emph{física estatística}, estudam-se sistemas em equilíbrio termodinâmico. 
Suponha que um sistema pode estar, no equilíbrio, em 
um dos $N$ microestados $x_1,\dots,x_N$ de energias respectivas $E_1,\dots,E_N$.
Se a temperatura é $T$, a \grasA{probabilidade do sistema estar no estado $i$} é
dada por 
$$p_i=\frac{e^{-\tfrac{E_i}{k_BT}}}{Z}\,,$$
onde $e^x$ é a função exponencial na base $e=2.718...$ (ver Seção
\ref{Sec:ExpoLog}), $k_B$ é a \grasA{constante de Boltzmann} e $Z$ a
\grasA{função de partição}.
\end{ex}

\begin{ex}
Em \emph{Teoria da Informação,}\index{informação! teoria da} estudam-se sequências
infinitas de símbolos
aleatórios. Com um alfabeto binário $\bA=\{0,1\}$,
$$01101001000011011011001001101010011001000000111010101100110....$$
Com um alfabeto $\bA=\{0,1,2,\dots,8,9\}$,
$$43895612031468275092781059463897360142581974603522706194583...$$
Se cada algarismo $a_i$ de um alfabeto 
$\bA=\{a_1,a_2,\dots,a_k\}$ aparece com uma probabilidade $p_i$, onde
$\sum_{j=1}^kp_j=1$, então a \grasA{Entropia de Shannon} de uma sequência
aleatória com essa propriedade é definida por
$$S=-\sum_{j=1}^kp_j\log_2 p_j\,,$$
onde o logaritmo é na base $2$ (mas pode ser tomado numa base qualquer).
$S$ dá um limite para a maior \grasA{taxa de compactação} para essa sequência.
\end{ex}

Uma construção completa das funções $\exp_ax$, $\log_a x$, para todo $x\in \bR$,
como se encontra nos livros de análise, requer um conhecimento detalhado das
propriedades dos números reais. Aqui daremos uma construção que, apesar de não
ser completamente rigorosa, tem a vantagem de ser intuitiva (espera-se) e permitirá
usar essas funções já desde o próximo capítulo.

\section{Exponencial}\label{Sec:Exponencial}

Seja $a>0$ um número positivo, fixo, chamado \grasA{base}. Definamos primeiro,
para todo número natural $n\in\bN$,
$$\exp_a(n)\pardef a^n= a\cdot a\cdots a\,\quad \text{($n$ vezes)}\,.$$
(Em particular, $a^1=a$.)
Assim obtemos uma função
\begin{align*}
 \exp_a:\bN&\to (0,\infty)\\
n&\mapsto a^n\,,
\end{align*}
que satisfaz às seguintes propriedades: para todo $m,n\in\bN$,
\begin{align}
a^ma^n&=a^{m+n}\,,\label{Eq:Proprexp1}\\
(a^m)^n&=a^{m\cdot n}\,.\label{Eq:Proprexp2}
\end{align}
Se $b>0$ for uma outra base,
\begin{equation}
(a\cdot b)^n=a^nb^n\,.\label{Eq:Proprexp3}
\end{equation}

O nosso objetivo é de estender essa função à reta real toda:
\begin{align*}
 \exp_a:\bR&\to (0,\infty)\\
x&\mapsto a^x\,.
\end{align*}
Faremos essa extensão passo a passo, com o seguinte objetivo em mente: \emph{que
as relações \eqref{Eq:Proprexp1}-\eqref{Eq:Proprexp3} sejam sempre satisfeitas,
também para variáveis reais.}\\

Por exemplo, como definir $a^0$? Para \eqref{Eq:Proprexp1} ser satisfeita com
$m=0$, $n=1$,
$$a=a^1=a^{1+0}=a^1\cdot a^0=a\cdot a^0\,.$$
Daí, simplificando por $a$ na última expressão, 
vemos que é preciso definir $$a^0\pardef 1\,.$$ 
Podemos em seguida definir a exponencial dos inteiros {negativos}, $a^{-n}$.
Usando de novo \eqref{Eq:Proprexp1} com $m=-n$, temos
$$a^{n}a^{-n}=a^{n-n}=a^0=1\,.$$
Logo, vemos que $a^{-n}$ precisa ser definida como:
$$
a^{-n}\pardef \frac{1}{a^n}\,.
$$
O mesmo raciocínio pode ser aplicado em geral: se $a^x$ já foi definido para
$x>0$, então o único jeito de definir $a^{-x}$ é como: 
$$a^{-x}\pardef \tfrac{1}{a^x}\,.$$ 
Estamos por enquanto com uma função 
\begin{align*}
 \exp_a:\bZ&\to (0,\infty)\\
n&\mapsto a^n\,.
\end{align*}
Façamos um primeiro esboço, isto é,
representemos alguns pontos de coordenadas $(n,a^n)$, $n\in \bZ$, no plano
cartesiano
(nessa figura, $a=2$):
\begin{center}
\begin{bmlimage}\begin{tikzpicture}[scale=0.8]
\begin{scope}
\draw[->] (-4,0)--(2.5,0) node[right]{$\bZ$};
\draw[->] (0,-0.2)--(0,5) node[left]{$a^n$};
\draw[color=gray!30, domain=-4:2.1]  plot (\x,{exp(\x*ln(2)}) node[right]{$a^x$};
\draw (0,1) node[left]{$1$};
\foreach \k in {-4,...,2} {
\pgfmathsetmacro{\y}{2^(\k)};
\fill (\k,\y) circle (0.45mm);
\draw (\k,0) node{$\shortmid$};
\draw (\k,0) node[below]{$\k$};
\draw[dotted] (\k,0)--(\k,\y);
}
\end{scope}
\end{tikzpicture}\end{bmlimage}
\end{center}
Já podemos observar que para valores de $n$ positivos grandes (aqui $a=2$), 
$$2^1=2\,\quad 2^2=4\,,\quad 2^3=8\,\quad 2^4=16\,,\quad 2^5=32\,,\quad
2^6=64\,,...$$
Como cada elemento dessa sequência é o \emph{dobro} do anterior, ela
\emph{diverge exponencialmente}\index{exponencial! divergência} rápido.
Por outro lado, para valores de $n$ negativos grandes, a sequência
\emph{converge exponencialmente} rápido para zero: 
$$
2^{-1}=0.5\,,\quad 2^{-2}=0.25\,,\quad 2^{-3}=0.125\,,\quad
2^{-4}=0.0625\,,\quad 2^{-5}=0.03125\,...
$$

Agora que $a^x$ foi definida para os valores de $x$ inteiros, vejamos como 
definir $a^x$ para os semi-inteiros
$x\in\{\dots,-\frac52,-\frac32,-\frac12, \frac12, \frac32,\frac52,\dots\}$.
Por exemplo, se $x=\tfrac{1}{2}$, já que $(a^{\frac12})^2=a$ por
\eqref{Eq:Proprexp2}, vemos que
$a^{\frac12}=\sqrt{a}$. Para definir $a^x$ para $x=\frac{m}{2}$, $m\in\bZ$,
usemos também \eqref{Eq:Proprexp2}. Quando $m>0$,
$$
a^{\frac{m}{2}}\pardef (a^{\frac{1}{2}})^m=\sqrt{a}^m\,,$$
e quando $m<0$,
$$\quad a^{-\frac{m}{2}}\pardef \frac{1}{a^{\frac{m}{2}}}\,.
$$
Assim, o gráfico anterior pode ser acrescentado dos pontos da forma
$(\tfrac{m}{2},a^{\frac{m}{2}})$:

\begin{center}
\begin{bmlimage}\begin{tikzpicture}[scale=0.8]
\begin{scope}
\draw[->] (-4,0)--(2.5,0);
\draw[->] (0,-0.2)--(0,5);
\draw[color=gray!30, domain=-4:2.1]  plot (\x,{exp(\x*ln(2))}) node[right]{$a^x$};
\foreach \k in {-8,...,4} {
\pgfmathsetmacro{\y}{exp(\k*ln(2)/2)};
\fill (\k/2,\y) circle (0.45mm);
}
\foreach \k in {-4,...,2} {
\draw (\k,0) node{$\shortmid$};
\draw (\k,0) node[below]{$\scriptstyle{\k}$};
\pgfmathsetmacro{\y}{exp(\k*ln(2))};
\draw[dotted] (\k,0)--(\k,\y);
}
\foreach \k in {-7,-5,-3,-1,1,3} {
\draw (\k/2,0) node{$\shortmid$};
%\draw (\k/2,-0.1) node[below]{$\scriptscriptstyle{\tfrac{\k}{2}}$};
}
\end{scope}
\end{tikzpicture}\end{bmlimage}
\end{center}

Repetindo esse processo, $a^x$ pode ser definido para os pontos da forma
$\tfrac{m}{4}$, $\tfrac{m}{8}$, $\tfrac{m}{16}$, etc, obtendo assim uma função
definida para qualquer $x$ da forma $\tfrac{m}{2^k}$. Esses reais são chamados
de \grasA{racionais diádicos}.

\begin{center}
\begin{bmlimage}\begin{tikzpicture}
\begin{scope}[scale=0.6]
\draw[->] (-4,0)--(2.5,0);
\draw[->] (0,-0.2)--(0,5);
\draw[color=gray!30, domain=-4:2.1] plot (\x,{exp(\x*ln(2))}) node[right]{$a^x$};
\foreach \k in {-4,...,2} {
\draw (\k,0) node{$\shortmid$};
\draw (\k,0) node[below]{$\scriptstyle{\k}$};
\pgfmathsetmacro{\y}{2^(\k)};
\fill (\k,\y) circle (0.45mm);
\draw[dotted] (\k,0)--(\k,\y);
}
\foreach \i in {-4,...,2} {
\draw (\i,0) node[below]{$\scriptstyle{\i}$};
}
%%%%%%%%%%%
\draw (-3,3) node{$k=1$:};
\pgfmathsetmacro{\l}{2};
\pgfmathsetmacro{\n}{6*\l};
\foreach \k in {0,...,\n} {
\pgfmathsetmacro{\x}{-4+\k/\l};
\pgfmathsetmacro{\y}{exp(\x*ln(2))};
\draw (\x,0) node{$\shortmid$};
\fill (\x,\y) circle (0.45mm);
}
\end{scope}
\begin{scope}[xshift=5cm, scale=0.6]
\draw[->] (-4,0)--(2.5,0);
\draw[->] (0,-0.2)--(0,5);
\draw[color=gray!30, domain=-4:2.1]  plot (\x,{exp(\x*ln(2))}) node[right]{$a^x$};
\foreach \k in {-4,...,2} {
\draw (\k,0) node{$\shortmid$};
\draw (\k,0) node[below]{$\scriptstyle{\k}$};
\pgfmathsetmacro{\y}{2^(\k)};
\fill (\k,\y) circle (0.45mm);
\draw[dotted] (\k,0)--(\k,\y);
}
\foreach \i in {-4,...,2} {
\draw (\i,0) node[below]{$\scriptstyle{\i}$};
}
%%%%%%%%%%%
\draw (-3,3) node{$k=2$:};
\pgfmathsetmacro{\l}{4};
\pgfmathsetmacro{\n}{6*\l};
\foreach \k in {0,...,\n} {
\pgfmathsetmacro{\x}{-4+\k/\l};
\pgfmathsetmacro{\y}{exp(\x*ln(2))};
\draw (\x,0) node{$\shortmid$};
\fill (\x,\y) circle (0.45mm);
}
\end{scope}

\begin{scope}[xshift=10cm, scale=0.6]
\draw[->] (-4,0)--(2.5,0);
\draw[->] (0,-0.2)--(0,5);
\draw[color=gray!30, domain=-4:2.1]  plot (\x,{exp(\x*ln(2))}) node[right]{$a^x$};
\foreach \k in {-4,...,2} {
\draw (\k,0) node{$\shortmid$};
\draw (\k,0) node[below]{$\scriptstyle{\k}$};
\pgfmathsetmacro{\y}{2^(\k)};
\fill (\k,\y) circle (0.45mm);
\draw[dotted] (\k,0)--(\k,\y);
}
\foreach \i in {-4,...,2} {
\draw (\i,0) node[below]{$\scriptstyle{\i}$};
}
%%%%%%%%%%%
\draw (-3,3) node{$k=3$:};
\pgfmathsetmacro{\l}{8};
\pgfmathsetmacro{\n}{6*\l};
\foreach \k in {0,...,\n} {
\pgfmathsetmacro{\x}{-4+\k/\l};
\pgfmathsetmacro{\y}{exp(\x*ln(2))};
\draw (\x,0) node{$\shortmid$};
\fill (\x,\y) circle (0.45mm);
}
\end{scope}

\end{tikzpicture}\end{bmlimage}
\end{center}

Observe que a medida que $k$ aumenta, 
os racionais diádicos $\tfrac{m}{2^k}$\index{números! racionais diádicos} vão enchendo a
reta real: diz-se que eles
formam um conjunto \emph{denso}\index{conjunto! denso} na reta.\\

Mas todos os racionais diádicos são racionais, e existem muitos (!) reais que
não são racionais...
Daremos a idéia da última (e mais delicada) etapa da construção de $a^x$ para
qualquer real $x$. Procederemos por \emph{aproximação}, observando que
qualquer real $x$ pode ser cercado por dois diádicos, digamos $z_-$ e $z_+$,
arbitrariamente próximos um do outro.
\begin{ex}
Por exemplo, $\pi=3.141592\dots$ é irracional~\footnote{A irracionalidade de
$\pi$ foi provada pela primeira vez por Johann Heinrich Lambert em $1761$.}, 
e é possível obter
aproximações pegando sucessivamente diádicos com $k=0,1,2,\dots$,etc.:
\begin{align*}
3=\frac{3}{2^0}& <\pi < \frac{4}{2^0}=4\\
3=\frac{6}{2^1}& <\pi < \frac{7}{2^1}=3.5\\
3=\frac{12}{2^2}& <\pi < \frac{13}{2^2}=3.250\\
3.125=\frac{25}{2^3}& <\pi < \frac{26}{2^3}=3.250\,\quad \text{etc.}
\end{align*}
Assim, podemos sempre achar um diádico, ou maior ou menor do que $\pi$, cujo valor
numérico é arbitrariamente perto de $\pi$.
\end{ex}

Assim, para qualquer irracional $x$, é possível escolher uma
sequência decrescente de diádicos $z_n^+$ que \emph{tende a $x$}, 
$z_n^+\searrow x$, e uma sequência crescente de diádicos $z_n^-$ 
\emph{que tende a $x$}, $z_n^-\nearrow x$. 

\begin{center}
\begin{bmlimage}\begin{tikzpicture}
 \draw[ ->] (-1.2,0)--(2.5,0);
\draw[->] (0,-0.2)--(0,3.3);
\draw[color=gray!60, domain=-0.5:1.6]  plot (\x,{exp(\x*ln(2))}) node[right]{$a^x$};
\pgfmathsetmacro{\x}{0.75};
\draw[dotted, color=blue!100] (\x,0)
node[below]{$\scriptstyle{x}$}--(\x,{exp(\x*ln(2))})--(0,{exp(\x*ln(2))})
node[left]{$\scriptstyle{?}$};
\draw (\x,{exp(\x*ln(2))}) node[fill=white]{$\scriptstyle{?}$};
\pgfmathsetmacro{\x}{0.3};
\draw[->, domain=\x:\x+0.3]  plot (\x,{exp(\x*ln(2))});
\draw[dashed] (\x,0) node[below]{$\scriptstyle{z_n^-}$}--(\x,{exp(\x*ln(2))})--(0,{exp(\x*ln(2))})
node[left]{$\scriptstyle{a^{z_n^-}}$};
\fill (\x, {exp(\x*ln(2))}) circle (0.45mm);
\pgfmathsetmacro{\x}{1.4};
\draw[<-, domain=\x-0.5:\x]  plot (\x,{exp(\x*ln(2))});
\fill (\x, {exp(\x*ln(2))}) circle (0.45mm);
\draw[dashed] (\x,0)
node[below]{$\scriptstyle{z_n^+}$}--(\x,{exp(\x*ln(2))})--(0,{exp(\x*ln(2))})
node[left]{$\scriptstyle{a^{z_n^+}}$};
\end{tikzpicture}\end{bmlimage}
\end{center}
Vemos então que os valores de $a^{z_n^-}$ e $a^{z_n^+}$ se aproximam de um
valor comum, que será usado para definir o valor de $a^x$.

Observe que essa construção usa implicitamente, pela primeira vez, a idéia
sutil de \emph{limite}\index{limite}, que será apresentada no próximo capítulo: qualquer
real
$x$ pode ser \emph{aproximado por uma sequência}\index{aproximação! por racionais} $z_n$
de racionais diádicos: 
$$x=\lim_{n\to \infty}z_n\,.$$ 
Como $a^{z_n}$ já foi definida para cada $z_n$ da sequência, $a^x$ é definida como 
$$a^x\pardef \lim_{n\to \infty}a^{z_n}\,.$$
Pode ser mostrado que a função $x\mapsto a^x$ obtida satisfaz às propriedades
\eqref{Eq:Proprexp1}-\eqref{Eq:Proprexp3}. Por exemplo, se $y$ é um outro real,
aproximado pela sequência $w_n$, $y=\lim_{n\to \infty}w_n$, então $x+y$ pode ser
aproximado pela sequência $(z_n+w_n)$, logo
$$
a^{x+y}=\lim_{n\to \infty}a^{z_n+w_n}
=\lim_{n\to \infty}a^{z_n}a^{w_n}
=(\lim_{n\to \infty}a^{z_n})(\lim_{n\to \infty}a^{w_n})=a^xa^y\,.
$$
Todas as operações acima são corretas, mas precisam ser justificadas.

AQUI
Assim conseguimos definir a função \grasA{exponencial  na base $a>0$}\index{exponencial!
na base $a$|textbf} como uma
função definida na reta real inteira:
\begin{align*}
 \exp_a:\bR&\to(0,\infty)\\
x&\mapsto a^x\,.
\end{align*}
Ela foi construida de maneira tal que as seguintes propriedades sejam
satisfeitas: $a^0=1$,\index{exponencial! propriedades}
\begin{align}
a^xa^y&=a^{x+y}\label{Eq:PropExpon1}\\
(a^x)^y&=a^{xy}\label{Eq:PropExpon2}\\
\frac{a^x}{a^y}&=a^{x-y}\label{Eq:PropExpon3}\\
(ab)^x&=a^xb^x\label{Eq:PropExpon4}\,.
\end{align}
Todas as funções exponenciais com base $a>1$ têm gráficos parecidos: 

\begin{center}
\begin{bmlimage}\begin{tikzpicture}\label{Fig:graficosdifbases}
\begin{scope}[scale=0.8]
\draw[->] (-4,0)--(2.5,0) node[right]{$x$};
\draw[->] (0,-0.2)--(0,5) node[left]{$a^x$};
\draw[domain=-4:2.1]  plot (\x,{exp(\x*ln(2))}) node[right]{$a=2$};
\draw[dashed, domain=-4:2.1]  plot (\x,{exp(\x*ln(1.5))}) node[right]{$a=\tfrac32$};
\draw[dotted, domain=-4:1.5]  plot (\x,{exp(\x*ln(3))}) node[right]{$a=3$};
\foreach \k in {-4,...,2} {
\draw (\k,0) node{$\shortmid$};
\draw (\k,0) node[below]{$\scriptstyle{\k}$};
}
\end{scope}
\end{tikzpicture}\end{bmlimage}
\end{center}
Observe que todos os gráficos passam pelo ponto $(0,1)$, e que $x\mapsto a^x$
é \emph{estritamente crescente}:
$$x<y\quad \Leftrightarrow \quad a^x<a^y\,.$$


Para os valores $a<1$, basta usar uma simetria:
Para $a=\frac12$ por exemplo, podemos observar que 
$$\exp_{{\frac12}}(x)=(\tfrac12)^x=2^{-x}=\exp_2(-x)\,.$$
Portanto, o gráfico de $x\mapsto (\frac12)^x$ é obtido a partir do gráfico de
$x\mapsto 2^x$ por uma simetria pelo eixo $y$. Em geral, o gráfico de $x\mapsto
(\frac1a)^x$ é obtido a partir do gráfico de $x\mapsto a^x$ por uma simetria
pelo eixo $y$:
\begin{center}
\begin{bmlimage}\begin{tikzpicture}
\begin{scope}[scale=0.8]
\draw[->] (-2.5,0)--(4.3,0) node[right]{$x$};
\draw[->] (0,-0.2)--(0,5) node[right]{$a^x$};
\draw[domain=-4:2.1]  plot (-\x,{exp(\x*ln(2))}) node[left]{$a=\tfrac12$};
\draw[dashed, domain=-4:2.1]  plot (-\x,{exp(\x*ln(1.5))}) node[left]{$a=\tfrac23$};
\draw[dotted, domain=-4:1.5]  plot (-\x,{exp(\x*ln(3))}) node[left]{$a=\tfrac13$};
\foreach \k in {-2,...,4} {
\draw (\k,0) node{$\shortmid$};
\draw (\k,0) node[below]{$\scriptstyle{\k}$};
}
\end{scope}
\end{tikzpicture}\end{bmlimage}
\end{center}
Temos também que  quando $0<a<1$, $x\mapsto a^x$ 
é \emph{estritamente decrescente}:
$$x<y \quad \Leftrightarrow \quad  a^x>a^y\,.$$

\begin{exo}
Esboce os gráficos das funções $1-2^{-x}$, $3^{x-1}$, $(\tfrac32)^{-x}$,
$-(\frac32)^{|x|}$.
\begin{sol} Todos os gráficos podem ser obtidos por transformações de
$2^x$,\index{gráfico! transformação de}
$3^x$ e $(\frac32)^x$:
\begin{center}
 \begin{bmlimage}\begin{tikzpicture}[scale=0.7]
\draw[->] (-4.5,0)--(4.3,0) node[right]{$x$};
\draw[->] (0,-3)--(0,5);
\draw[domain=-2.1:4.1]  plot (\x,{1-exp(-\x*ln(2))}) node[right]{$1-2^{-x}$};
\draw[domain=-3.4:2.5]  plot (\x,{exp((\x-1)*ln(3))}) node[above]{$3^{x-1}$};
\draw[domain=-3.7:2.4]  plot (-\x,{exp(\x*ln(1.5)))})
node[above]{$(\tfrac{3}{2})^{-x}$};
\draw[domain=-3:3]  plot (\x,{(-1)*(exp(abs(\x)*ln(1.5)))})
node[right]{$-(\tfrac{3}{2})^{|x|}$};
\draw[dotted] (-3,1)--(4,1);
\draw (0,1) node[above right]{$1$};
 \end{tikzpicture}\end{bmlimage}
\end{center}
\end{sol}
\end{exo}

Com mais funções, resolvem-se mais (in)equações:
\begin{ex}
Resolvamos $$3^x+3^{-x}=2\,.$$
Multiplicando por $3^x$ em ambos lados e agrupando os termos obtemos
$(3^x)^2-2\cdot 3^x+1=0$. Chamando $z=3^x$, essa equação se torna $z^2-2z+1=0$,
cuja única solução é $z=1$, isto é, $3^x=1$. Logo, $S=\{0\}$.
\end{ex}

\begin{exo}
Resolva:
\begin{multicols}{3}
 \begin{enumerate}
  \item\label{itexoresolexp1} $5^x+25\cdot 5^{-x}=26$
  \item\label{itexoresolexp2} $(2^x)^2=16$
  \item\label{itexoresolexp3} $2^{x+1}-16\leq 0$
  \item\label{itexoresolexp6} $3^x>\tfrac19$
 \item\label{itexoresolexp4} $(2^x-2)(\tfrac{1}{5^x}-1)<0$
 \item\label{itexoresolexp5} $2^{2x+1}\geq 5^{1-x}$
 \end{enumerate}
\end{multicols}
\vspace{0.01cm}
\begin{sol}
\eqref{itexoresolexp1} $S=\{0,2\}$.
\eqref{itexoresolexp2}  Tomando a raiz: $2^x=\pm 4$, mas como a função
exponencial somente toma valores positivos, $2^x=-4$ não possui soluções. Logo,
$S=\{2\}$.
\eqref{itexoresolexp3} Escrevendo a inequação como $2^{x+1}\leq 2^4$, vemos que
$S=\{x:x+1\leq 4\}=(-\infty,3]$.
\eqref{itexoresolexp6} $S=(-2,\infty)$.
\eqref{itexoresolexp4} $S=(-\infty,0)\cup (1,\infty)$.
\eqref{itexoresolexp5} $S=(\log_{20}\tfrac52,\infty)$.
\end{sol}
\end{exo}

\begin{obs}
Para se acostumar com a as mudanças de escala entre os valores de $10^{n}$ para
$n$ grande positivo e $n$ grande negativo, sugiro assistir o pequeno filme
clássico de Charles e Bernice Ray Eames de $1968$: \emph{Powers of Ten}
(\emph{Potências de dez})\index{potência! Potências de dez (filme)}. Se encontra por
exemplo em:
\verb|http://www.youtube.com/watch?v=0fKBhvDjuy0|.
\end{obs}

\begin{obs}
Lembramos que a base de uma exponencial é sempre \emph{estritamente positiva}.
De fato, definir a exponencial para bases negativas, por exemplo $a=-1$, daria 
problemas já para definir $(-1)^{1/2}$, que corresponde a $\sqrt{-1}$, que
não é definido nos reais.
Observe também que a base $a=0$ não é interessante, mas mesmo assim daremos um 
sentido a ``$0^0$'' no Capítulo~\ref{Cap:Derivacao}.
\end{obs}

\section{Logaritmo}\index{logaritmo|textbf}
Como a exponencial $x\mapsto \exp_ax$ é estritamente crescente (ou decrescente
se $0<a<1$), é uma bijeção de $\bR$ para $(0,\infty)$, e
a sua função inversa é bem definida, chamada \grasA{logaritmo na base $a$}:
\begin{align*}
\log_a:(0,\infty)&\to \bR \\
y&\mapsto \log_ay\,.
\end{align*}
Como $a^0=1$, temos $\log_a1=0$, e como $a^1=a$ temos $\log_aa=1$.  
O gráfico do logaritmo, dependendo da base, é da forma\index{logaritmo! gráfico}:
\begin{center}
\begin{bmlimage}\begin{tikzpicture}[scale=0.7]
\begin{scope}
\draw (-1,3) node[left]{$a>1:$};
\draw[->] (-0.2,0)--(4,0) node[right]{$x$};
\draw[->] (0,-3)--(0,3) node[left]{$y$};
\draw (1,0) node{$\shortmid$} node[above]{$1$};
\draw[thick, domain=-3:1.7]  plot ({exp(\x)},\x) node[right]{$\log_ax$};
\draw[dotted] (2.718,0) node[below]{$a$}--(2.718,1)--(0,1) node[left]{$1$}
node{$-$};
\end{scope}
\begin{scope}[xshift=12cm]
\draw (-1,3) node[left]{$0<a<1:$};
\draw[->] (-0.2,0)--(4,0) node[right]{$x$};
\draw[->] (0,-3)--(0,3) node[left]{$y$};
\draw (1,0) node{$\shortmid$} node[above]{$1$};
\draw[thick, domain=-3:1.7]  plot ({exp(\x)},-\x) node[right]{$\log_ax$};
\end{scope}
\end{tikzpicture}\end{bmlimage}
\end{center}
O logaritmo é estritamente crescente se $a>1$, estritamente decrescente se
$0<a<1$.
Por definição,
\eq{
\label{eq_ExpLog_debase}
\forall x>0\,:\, a^{\log_ax}=x\,,\quad \text{ e } \forall x\in
\bR\,:\,\log_a(a^x)=x\,.}

A definição do logaritmo deve ser lembrada pela seguinte equivalência:
\eq{
z=\log_ax\quad\Leftrightarrow\quad a^z=x\,.
}
Por exemplo, para calcular $\log_28$, basta chamar $z=\log_28$, que é
equivalente a 
$2^z=8$, cuja única solução é $z=3$.

\begin{obs}
O logaritmo foi inventado por Napier~\footnote{John Napier, Merchiston (Escócia)
1550 - 1617.}\index{Napier, John} no século $XVI$, numa época em que ainda não existiam
calculadoras. 
Suponha que se queira calcular, \emph{na mão}, uma potência de um número grande.
Por exemplo: $9846^6$. A conta, apesar de não ser difícil, requer um certo
trabalho: primeiro calcula $9846^2=9846\times 9846=\cdots=96943716$. Depois,
calcula $9846^3=96943716 \times 9846=954507827736$, etc. Até obter $9846^6$, que
é um número de $23$ dígitos...

Suponha agora que seja conhecido um número $x$ tal que $9846= 10^x$. Então, pela
propriedade \eqref{Eq:PropExpon2} 
da exponencial, tomar a sexta potência se reduz a multiplicar $x$ por $6$:
$$9846^6= (10^x)^6=10^{6x}\,!$$
O número procurado $x$ não é nada mais do que o logaritmo de $9846$ na  base
$10$: $x=\log_{10}9846$ (com a minha calculadora: $x\sim 3,9932$).
No fim do século $XVI$ já existiam tabelas dando $\log_{10}n$ para todos os
inteiros $n$ entre $1$ e $90000$, com uma precisão de quatorze decimais.

Dando assim um novo jeito de calcular, 
logaritmos se tornaram uma ferramenta indispensável nas ciências e na
engenharia. 
O Kepler~\footnote{Johannes Kepler, Weil der Stadt (Alemanha) 1571 - Regensburg
1630.}\index{Kepler, Johannes} usou logaritmos sistematicamente no seu estudo do movimento
dos planetas.
% Antes dos computadores, 
% eles eram calculados usando tabelas. 
\end{obs}

O logaritmo satisfaz às seguinte identidades (supondo $x,y>0$, menos na segunda,
onde $y\in \bR$): \index{logaritmo! propriedades}

\begin{align}
\log_a(xy)&=\log_ax+\log_ay\label{Eq:PropLog1}\\
\log_a(x^y)&=y\log_ax\label{Eq:PropLog2}\\
\log_a\tfrac{x}{y}&=\log_ax-\log_ay\label{Eq:PropLog3}
\end{align}

Para provar a primeira, chamemos $z=\log_a(xy)$, o que significa $a^z=xy$. 
Escrevendo $x=a^{\log_ax}$, $y=a^{\log_a y}$ e usando a propriedade
\eqref{Eq:PropExpon1} 
da exponencial, temos $$a^z=a^{\log_ax}a^{\log_a y}=a^{\log_ax+\log_a y}\,.$$ 
Assim vemos que $z=\log_ax+\log_a y$, o que prova \eqref{Eq:PropLog1}.

\begin{exo}
Prove \eqref{Eq:PropLog2} e \eqref{Eq:PropLog3}.
\begin{sol}
Se $z=\log_a(x^y)$, então $z$ satisfaz $a^z=x^y$.
Por~\eqref{eq_ExpLog_debase}, 
podemos sempre escrever $x$ como $x=a^{\log_a x}$, o que permite
escrever $x^y=(a^{\log_ax})^y=a^{y\log_a x}$. Assim temos $a^z=a^{y\log_a x}$, o que implica
$z=y\log_ax$.
Se $z=\log_a\frac{x}{y}$, então 
$$
a^z=\frac{x}{y}=\frac{a^{\log_ax}}{a^{\log_ay}}=a^{\log_ax-\log_ay},
$$
logo $z=\log_ax-\log_ay$.
\end{sol}
\end{exo}

\begin{exo}
Calcule, sem usar calculadora,
\[ 
\log_4 16\,,\quad
\log_\pi 1\,,\quad
\log_2\tfrac{1}{16}\,,\quad
\log_{1/2}8\,,\quad
7^{2\log_75}\,.
\]
\begin{sol}
$\log_4 16=2$,
$\log_\pi 1=0$,
$\log_2\frac{1}{16}=-4$,
$\log_{\tfrac12}8=-3$,
$7^{2\log_75}=25$.
\end{sol}
\end{exo}

\begin{exo}
Suponhamos que o tamanho de uma população de baratas numa casa dobre a cada mês,
e que no fim do mês de dezembro de $2010$, foram registradas $3$ baratas. 
Dê o número de baratas em função do número de meses passados ($n=1$: fim de
janeiro, etc.)
Quantas baratas vivem na casa no fim do mês de julho de $2011$? No fim de
agosto?
 Quando que será ultrapassado o milhão de baratas?
\begin{sol}
Se $N(n)$ é o número de baratas depois de $n$ meses, temos $N(1)=3\cdot 2$,
$N(2)=3\cdot 2\cdot 2$, etc. Logo, $N(n)=3\cdot 2^n$. No fim de julho se
passaram $7$ meses, logo são $N(7)=3\cdot 2^7=384$ baratas. No fim do mês
seguinte são $384\times 2=768$ baratas. 
Para saber quando a casa terá mais de um milhão de baratas, é preciso resolver
$N(n)>1000000$, isto é, $3\cdot 2^n>1000000$, que dá
$n>\log_2(1000000/3)=18,34...$, 
isto é, no fim do $19$-ésimo mês, o que significa julho de $2012$...
\end{sol}
\end{exo}




\begin{exo} Dê o domínio de cada função abaixo.\index{domínio}
\begin{multicols}{3}
\begin{enumerate}
\item \label{iteqdomlog2}
$\log_5(2+x)$
\item\label{iteqdomlog3} $\log_2(2-x)$ 
\item\label{iteqdomlog1}
$\frac{8x}{\log_6(1-x^2)}$
\item\label{iteqdomlog4} $\sqrt{1-\log_7(x)}$
\item\label{iteqdomlog5} $\frac{1}{\sqrt{1-\log_8(x)}}$

\item \label{iteqdomlog6} $\log_2(|2x+1|+3x)$
\item \label{iteqdomlog7} $3^{\log_3 x}$
\end{enumerate}
\end{multicols}
\vspace{0.01cm}
\begin{sol}
\eqref{iteqdomlog2} $D=(-2,\infty)$
\eqref{iteqdomlog3} $D=(-\infty,2)$
\eqref{iteqdomlog1} Para $\log_6(1-x^2)$ ser definido, precisa $1-x^2>0$, que dá
$(-1,1)$. Por outro lado, 
para evitar uma divisão por zero, precisa $\log_6(1-x^2)\neq 0$, isto é,
$1-x^2\neq 1$, isto é, $x\neq 0$. Logo, $D=(-1,0)\cup(0,1)$. 
\eqref{iteqdomlog4} $D=(0,7]$
\eqref{iteqdomlog5} $D=(0,8)$
\eqref{iteqdomlog6} $D=(-\tfrac15,\infty)$
\eqref{iteqdomlog7} $D=\bR_+^*$
\end{sol}
\end{exo}

Suponha que o logaritmo de $x>0$ seja conhecido na base $a$: $\log_a x$. Como
calcular o logaritmo numa outra base $b>0$, $\log_bx$? Chamando $z=\log_bx$,
temos $b^z=x$. Mas $b$ pode ser escrito como $b=a^{\log_ab}$, assim temos
$a^{z\log_ab}=x$. Portanto, $z\log_ab=\log_ax$. Obtemos assim a fórmula de
\grasA{mudança de base}:\index{logaritmo! fórmula de mudança de base}
\eq{\label{eq:mudancabaselog}
\boxed{
\log_bx=\frac{\log_ax}{\log_ab}\,.}
}

\begin{ex}
Resolvamos: 
$$2^x\cdot 3^{-x}=4\,.$$
Coloquemos cada termo na mesma base, por exemplo na base $a=5$:
$$5^{x\log_52}\cdot 5^{-x\log_53}=5^{\log_54}\,.$$
Logo, $x$ satisfaz $x\log_52-x\log_53=\log_54$, isto é:
$x=\frac{\log_54}{\log_52-\log_53}$.
Observe que por \eqref{eq:mudancabaselog}, essa resposta não depende da base
escolhida para calcular o logaritmo. 
De fato, ao escolher $b=3$ em vez de $a=5$,
teríamos obtido $x=\frac{\log_34}{\log_32-\log_33}$, que por
\eqref{eq:mudancabaselog} é igual a
$$\frac{\frac{\log_54}{\log_53}}{\frac{\log_52}{\log_53}-\frac{\log_33}{\log_53}
}
\equiv \frac{\log_54}{\log_52-\log_53}\,.
$$
\end{ex}

\begin{exo}
Resolva.
\begin{multicols}{3}
\begin{enumerate}
\item\label{itlogeq_1} $2^x=\frac18$
\item \label{itlogeq_2}$\log_{10}(x+3)=3$
\item\label{itlogeq_3} $3^{x^2}=\frac{1}{3^x}$
\item\label{itlogeq_4} $2^x=5^{1-x}$
\item\label{itlogeq_5} $\log_2(x^2+1)=-2$
\item\label{itlogeq_6} $3+\log_2(\tfrac12-x)=\log_2(\tfrac{x-9}{x+1})$
\item\label{itlogeq_7} $\log_8(-x)>0$
\item\label{itlogeq_8} $\log_3(x^2-2x)<1$
\item\label{itlogeq_9} 
\end{enumerate}
\end{multicols}
\vspace{0.1mm}
\begin{sol}
\eqref{itlogeq_1} $S=\{-3\}$,
\eqref{itlogeq_2} $S=\{997\}$,
\eqref{itlogeq_3} $S=\{0,1\}$,
\eqref{itlogeq_4} $S=\{\frac{\log_25}{1+\log_25}\}$,
\eqref{itlogeq_5} $S=\varnothing$,
\eqref{itlogeq_6} $S=\{-\tfrac{13}{8}\}$.
\eqref{itlogeq_7} $S=(-\infty,-1)$,
\eqref{itlogeq_8} $S=(-1,0)\cup(2,3)$,
\end{sol}
\end{exo}

\begin{exo} Considere duas colônias de bactérias, de tipos $A$ e $B$,
originalmente com $N_A=123456$ e $N_B=20$ indivíduos.
As bactérias do tipo $A$ triplicam (em número) a cada dia, enquanto as do tipo
$B$ dobram a cada hora.
Quanto tempo demora para as duas colônias terem populações iguais em tamanho?
A longo prazo, qual colônia cresce mais rápido?
\begin{sol}
As populações respectivas de bactérias depois de $n$ horas são:
$N_A(n)=123456\cdot 3^{\tfrac{n}{24}}$, $N_B(n)=20\cdot 2^n$.
Procuremos o $n_*$ tal que $N_A(n)=N_B(n)$, isto é (o logaritmo pode ser em
qualquer base):
$$n_*=\frac{\log_{10}123456-\log_{10}
20}{\log_{10}2-\tfrac{1}{24}\log_{10}3}=13.48...\,.$$
Isto é, depois de aproximadamente $13$ horas e meia, as duas colônias têm o
mesmo número de indivíduos.
Depois desse instante, as bactérias do tipo $B$ são sempre maiores em número.
De fato (verifique!), $N_A(n)<N_B(n)$ para todo $n>n_*$.
\end{sol}
\end{exo}

\begin{exo}
%DOUCHET 53
Mostre que a função abaixo é uma bijeção, e calcule $f^{-1}$.
\begin{align*}
 f:\bR&\to \bR_+^*\\
x&\mapsto \frac{3^x+2}{3^{-x}}
\end{align*}
\begin{sol}
Se $y\in \bR_+^*$, procuremos uma solução de  
$y=\frac{3^x+2}{3^{-x}}$. Essa equação se reduz a $(3^x)^2+2\cdot 3^x-y=0$, isto
é $3^x=-1\pm \sqrt{1+y}$. Como $y>0$, vemos que a solução positiva dá uma única 
preimagem $x=\log_3(-1+\sqrt{1+y})\in \bR$. Logo $f$ é uma bijeção e 
$f^{-1}:\bR_+^*\to \bR$ é dada por $f^{-1}(y)=\log_3(-1+\sqrt{1+y})$.
\end{sol}
\end{exo}

\begin{exo}\label{Exo:Banco}
Deixar uma quantidade $C_0$ no banco numa poupança\index{juros! taxa de} com taxa de
juros de $r\%$
significa que em um ano, 
essa quantidade gerou um lucro de $\frac{r}{100}C_0$. Assim, depois de 
um ano, a quantidade inicial acrescentada do lucro é de:
$C_1=C_0+\frac{r}{100}C_0=(1+\frac{r}{100})C_0$. Se essa nova quantidade for
deixada por mais um ano, a nova quantidade no fim do segundo ano será de
$C_2=C_1+\frac{r}{100}C_1=(1+\frac{r}{100})^2C_0$. Assim, a quantidade de
dinheiro em função do número de anos é exponencial de base $a=1+\frac{r}{100}$: 
$$C_n=C_0\big(1+\tfrac{r}{100}\big)^n\,.$$
\begin{enumerate}
 \item\label{itbanco1} Suponha que a taxa é de $5\%$.
Se eu puser $\mathrm{RS}1000$ no banco hoje, quanto que eu terei daqui a 5 anos?
Quanto que eu preciso por no banco hoje, para ter $\mathrm{RS}2000$ daqui a dois
anos?
Se eu puser $\mathrm{RS}1$ hoje, quantos anos que eu preciso esperar para eu ter
$\mathrm{RS}1.000.000$?
\item\label{itbanco2} Qual deve ser a taxa se eu quiser investir
$\mathrm{RS}1000$ hoje e ter um lucro de $\mathrm{RS}600$ em $5$ anos?
\end{enumerate}
\begin{sol}
\eqref{itbanco1} Se $r=5\%$, $C_n=C_0\cdot 1,05^n$. 
Logo, seu eu puser $1000$ hoje, daqui a $5$ anos terei 
$C_5\simeq 1276$, e 
para ter $2000$ daqui a $5$ anos, preciso por hoje $C_0\simeq 1814$.
Para por $1$ hoje e ter um milhão, preciso esperar
$n=\log_{1,05}(1000000/1)\simeq 283$ anos.
\eqref{itbanco2} Para ter um lucro de $600$ em $5$ anos, começando de $1000$,
preciso achar o $r$ tal que 
$1000+600=1000(1+r/100)^5$. Isto é, $r=100\times
(10^{\frac{\log_{10}1,6}{5}}-1)\simeq 9,8\%$.
\end{sol}
\end{exo}

\begin{exo}
Uma folha de papel é dobrada em dois, para ter a metade do tamanho inicial mas
uma espessura duas vezes maior, pra depois ser dobrada de novo em dois, etc. 
\begin{enumerate}
 \item\label{itDobrafolha1}
Estime a espessura de uma folha de papel $A4$ comum, e calcule a espessura total
depois de $6$, respectivamente  $7$ dobras.
\item\label{itDobrafolha2} Quantas dobras são necessárias para que a espessura
final seja 
a) de $1.80m$?
b) do tamanho da distância terra-lua? 
\end{enumerate}
\begin{sol}
\eqref{itDobrafolha1} 
Um pacote de $500$ folhas $A4$ para impressora tem uma espessura de
aproximadamente $5$ centímetros. Logo, uma folha tem uma espessura de
$E_0=5/500=0,01$ centrímetros. Como a espessura dobra a cada dobra, a espessura
depois de $n$ dobras é de $E_n=E_02^n$. Assim, $E_6=0,64$cm, $E_7=1.28$cm
\eqref{itDobrafolha1} a) Para ter $E_n=180$, são necessárias
$n=\log_{2}\frac{180}{0,01}\simeq 14$ dobras.
b) A distância média da terra à lua é de $D=384'403$km. Em centímetros:
$D=3,84403\times 10^{10}$cm. Assim, depois da $41$-ésima dobra, a distância
terra-lua já é ultrapassada.
Observe que depois desse tanto de dobras, o a largura do pacote de papel é
microscópica.
\end{sol}
\end{exo}

\begin{obs}
O uso de logaritmos é comum quando se quer comparar certas grandezas físicas que
tomam valores grandes.
Como exemplo, considere a nossa percepção do volume sonoro, 
que depende da potência exercida pela pressão do ar nos nossos tímpanos. 
%A nossa percepção do volume de um som 
%depende da \emph{potência} produzida pelo som no tímpano.
Por um lado, a potência mínima que um tímpano consegue detectar fica em torno
de $P_{min}=10^{-12}W/m^2$. 
Por outro lado, a potência máxima que ele consegue
aguentar (isto é sem soffrer danos irreversíveis) 
fica em torno de $P_{max}=100W/m^2$. 
Para $P\simeq P_{min}$ a sensação é de que não há som nenhum (volume nulo).
Quando $P\simeq P_{max}$, a sensação é de um volume altíssimo.

Acontece que a nossa percepção do volume associado a uma
potência intermediária $P_{min}\leq P\leq P_{max}$ 
é proporcional não a $P$ mas ao \emph{logaritmo} de $P$.
%~\footnote{Para se
%convencer de que a percepção do volume não é proporcional a $P$, podemos pensar
%da seguinte maneira: se eu escuto um som num rádio, e que eu colocar do lado }
Por isso, define-se o \grasA{número de decibeis} (unidades: dB) 
\index{decibel}
associado à potência $P$ como
\begin{equation}
L(P)= 10\cdot \log_{10}\Bigl(
\frac{P}{P_{min}}
\Bigr)\,.
\end{equation}
Assim, $L$ varia entre um mínimo de 
\[L_{min}=10\cdot
\log_{10}\Bigl(\tfrac{P_{min}}{P_{min}}\Bigr)=0\,\text{dB}\,,\]
e um máximo de 
\[ 
L_{max}=10\cdot \log_{10}\Bigl(\frac{P_{max}}{P_{min}}  \Bigr)=140\,\text{dB}\,.
\]
\end{obs}
\begin{exo}
Qual é o volume total produzido por duas fontes de $120$dB cada?
\begin{sol}
Se uma fonte é de $120$dB, a potência $P$ que ela produz se acha
isolando $P$ em $120=10\cdot \log_{10}(\tfrac{P}{P_{min}})$, o que dá 
$P=10^{-2}W/m^2$. 
Como duas fontes produzem o dobro da potência, isto é $2P$, o que representa 
\[L=10\cdot
\log_{10}\Bigl(\frac{2P}{P_{min}}\Bigr)=120+\log_{10}2\simeq 120.3\text{dB}\]
\end{sol}
\end{exo}

\section{A base $e=2,718...$}\label{Sec:ExpoLog}\index{exponencial! na base $e$}

A exponencial $a^x$ foi definida para qualquer base $a>0$. 
A escolha de uma base específica depende em geral da situação. Por exemplo, num
problema 
de bactérias cuja população \emph{dobra} a cada unidade de tempo, a base será
$a=2$. 
Vimos também que a base não precisa ser inteira: no Exercício 
\ref{Exo:Banco}, $a=1+\frac{r}{100}$.\\

A priori, qualquer base pode ser escolhida para estudar um problema. Por
exemplo, se tivermos alguma preferência para a base $3$, qualquer exponencial
pode ser transformada na base $3$:
$$
2^x=3^{(\log_32)x}\,,\quad 5^x=3^{(\log_35)x}\,,\quad
17^x=3^{(\log_317) x}
$$

Existe uma base, denotada por $e$, cuja importância será vista nos próximos
capítulos, mas que será introduzida aqui:
$$e=2.718281828459045235360287471352...$$
Como $\pi$, o número $e$ é uma constante fundamental da matemática. Ele pode ser
definido de várias maneiras. Por exemplo, 
\emph{geometricamente}, $e$ é o único número $>1$ tal que a área delimitada pelo
gráfico da função $x\mapsto \frac1x$, pelo eixo $x$ e pelas retas verticais
$x=1$, $x=e$, seja igual a $1$:
\begin{center}
\begin{bmlimage}\begin{tikzpicture} 
\draw[ ->] (0,-0.1)--(0,2.3) node[left]{$\tfrac{1}{x}$};
\fill[areagrafico] (1,0)--plot[domain=1:2.718](\x,{1/\x})--(2.718,0)--cycle;
\draw[<-] (1.8,0.3)--(2.5,1.2) node[above right]{área$=1$};
\draw [domain=0.5:3.5] plot (\x,{1/\x});
\draw[dotted] (1,0) node[below]{$\scriptstyle{1}$}--(1,1);
\draw[dotted] (2.718,0) node[below]{$e$}--(2.718,{1/2.718});
\draw[ ->] (-0.1,0)--(4,0) node[right]{$x$};
\end{tikzpicture}\end{bmlimage}
\end{center}
(Mais tarde veremos como calcular a área debaixo de um gráfico.)
\emph{Analiticamente}, $e$le pode ser obtido calculando o valor da soma infinita
(chamada \emph{série}, ver \emph{Cálculo 2})
$$e=1+\frac{1}{1!}+\frac{1}{2!}+\frac{1}{3!}+\frac{1}{4!}+\frac{1}{5!}+\dots\,,
$$
ou como o valor do limite
\begin{equation}\label{eq_def_e}
e=\lim_{n\to \infty}\bigl(1+\tfrac{1}{n}\bigr)^n\,.
\end{equation}
Foi mostrado por Euler~\footnote{Leonard Euler, Basileia (Suiça) $1707$ -
São-Petersburgo (Rússia) $1783$.}\index{Euler, Leonard} que $e$ é irracional.\\

Não mostraremos aqui porque que as três definições acima são equivalentes, mas a partir
de agora admitiremos que o limite em \eqref{eq_def_e} existe, e o usaremos para definir a
base $e$.\\

A exponencial associada á base $e$ costuma ser escrita $\exp(x)$ (em vez de 
$\exp_e(x)$), ou simplesmente $e^x$. O logaritmo na base $e$ escreve-se $\ln(x)$
(em vez de $\log_e(x)$), e chama-se \grasA{logaritmo neperiano}\index{logaritmo!
neperiano} (devido a Napier), ou \grasA{logaritmo natural}\index{logaritmo! natural}.
Por serem a exponencial e o logaritmo de uma base específica, as funções $e^x$ e
$\ln x$ possuem todas as propriedades das funções $\log_ax$ descritas acima para
$a>1$. Em particular, elas são ambas estritamente crescentes:

\begin{center}
\begin{bmlimage}\begin{tikzpicture}
\begin{scope}[scale=0.7]
\draw[->] (-4,0)--(2,0) node[right]{$x$};
\draw[->] (0,-0.2)--(0,5) node[left]{$y$};
\draw (0,1) node{$-$} node[left]{$1$};
\draw[dotted] (1,0) node[below]{$1$}--(1,2.718)--(0,2.718) node[left]{$e$};
\draw[thick, domain=-4:1.7]  plot (\x,{exp(\x)}) node[right]{$e^x$};
\end{scope}
\begin{scope}[scale=0.7, xshift=7cm, yshift=2cm]
\draw[->] (-0.2,0)--(4,0) node[right]{$x$};
\draw[->] (0,-3)--(0,3) node[left]{$y$};
\draw (1,0) node{$\shortmid$} node[below]{$1$};
\draw[thick, domain=-3:1.7]  plot ({exp(\x)},\x) node[right]{$\ln x$};
\draw[dotted] (0,1) node[left]{$1$}--(2.718,1)--(2.718,0) node[below]{$e$};
\end{scope}
\end{tikzpicture}\end{bmlimage}
\end{center}

Veremos que é mais fácil manusear 
exponencial e logaritmos quando esses são na base $e$. Por exemplo, sera
visto que a função $e^x$ é a única função cujo valor em $x=0$ é $1$, e que é
igual a sua própria derivada: $(e^x)'=e^x$.

\begin{obs}
Uma boa referência para aprender mais sobre o número $e$, sobre a invenção do
logaritmo e sobre o seu papel no desenvolvimento do Cálculo é o livro de Eli
Maor, \emph{$e$: a história de um número} (se encontra na Biblioteca Central).
\end{obs}

Daremos mais dois exemplos em que a constante $e$ tem um papel fundamental:

\begin{ex}
A \grasA{curva de Gauss}\index{Gauss, curva de Gauss}, ou \grasA{Gaussiana} é uma
distribuição de
probabilidade universal, que rege o \emph{desvio padrão} de um grande número de
variáveis aleatórias independentes:
%$$\varphi(x)=\frac{1}{\sqrt{2\pi}}e^{-\frac{x^2}{2}}\,.$$
\begin{center}
\begin{bmlimage}\begin{tikzpicture}
\draw (-6,1.5) node{$\rho(x)=\frac{1}{\sqrt{2\pi}}e^{-\frac{x^2}{2}}$};
\pgfmathsetmacro{\n}{2.5};
\draw[->] (-\n-0.5,0)--(\n+0.5,0) node[right]{$x$};
\draw[->] (0,-0.2)--(0,2.5) node[right]{$\rho(x)$};
\draw[thick, domain=-\n:\n, samples=100] plot (\x,{2*exp(-1*(\x)^2)});
\end{tikzpicture}\end{bmlimage}
\end{center}
\end{ex}

\begin{ex}
Em física nuclear, uma substância radioativa se desintegra\index{desintegração}
naturalmente com uma
taxa $0<\lambda<1$, o que significa que 
a quantidade de substância em função do tempo $t$ decresce como 
\eq{\label{eq:desintegra}N_t=N_0e^{-\lambda t}\,,t\geq 0\,,}
onde $N_0$ é a quantidade de substância inicial e $t$ o tempo.
\begin{center}
\begin{bmlimage}\begin{tikzpicture}
\begin{scope}[scale=0.7]
\pgfmathsetmacro{\n}{2};
\draw[->] (-0.2,0)--(4.5,0) node[right]{$t$ (anos)};
\draw[->] (0,-0.2)--(0,\n+0.5) node[right]{$N_t$};
\draw[thick, domain=0:3.7] plot (\x,{\n*exp(-\x)});
\draw (0,\n) node{$-$} node[left]{$N_0$};
\end{scope}
\end{tikzpicture}\end{bmlimage}
\end{center}
\end{ex}

\begin{exo} Considere \eqref{eq:desintegra}.
\begin{enumerate} 
\item Calcule o tempo de \grasA{meia-vida}\index{tempo de meia-vida} $T$, isto é, o tempo
necessário para a
quantidade de substância ser igual à metade da sua quantidade inicial.
Qual é a quantidade de substância sobrando depois de duas meia-vidas? Quatro? 
Existe um tempo em que a substância toda se desintegrou?
\item Sabendo que o urânio $235$ possui uma taxa de desintegração
$\lambda_U=9.9\cdot 10^{-10}$, calcule o seu tempo de meia-vida.
\end{enumerate}
\begin{sol}
Para ter $N_T=\tfrac{N_0}{2}$, significa que $e^{-\alpha T}=\tfrac12$. Isto é:
$T=\tfrac{\ln 2}{\lambda}$.
Depois de duas meia-vidas, $N_{2T}=N_0e^{-\lambda\tfrac{2 \ln
2}{\lambda}}=\frac{N_0}{4}$ ($>0$: logo, duas meia-vidas não são suficientes
para acabar com a substância!). 
Para quatro, $N_{4T}=\frac{N_0}{16}$. Depois de $k$ meia-vidas,
$N_{kT}=\frac{N_0}{2^k}$:
depois de um número qualquer de meia-vidas, sempre sobre alguma coisa...
Para o uranio $235$, a meia-vida vale $T=\frac{\ln 2}{9.9\cdot 10^{-10}}$, isto
é aproximadamente: $700$ milhões de anos.
\end{sol}
\end{exo}

\begin{exo}
Resolva:
\begin{multicols}{2}
 \begin{enumerate}
  \item\label{iteqln0} $\ln (-x)=2$
  \item\label{iteqln1} $\ln (x^2)=0$
  \item\label{iteqln11} $\ln (x+1)+\tfrac{1}{5}=0$
  \item\label{iteqln2} $\ln (1+x^2)=-\tfrac12$
  \item\label{iteqln3} $e^{x}+e^{-x}=4$
\item\label{iteqln4}$e^{2x-1}<\sqrt{e}$
\item\label{iteqln5}$e^{\tfrac{2x-1}{3x+1}}>\tfrac{1}{e^2}$
\item\label{iteqln6}$\ln(\frac{2x-1}{5x+1})<0$
\item\label{iteqln7} $\ln |x+4|+\ln |x-1|=\ln 6$
\item\label{iteqln8} $(\ln x)^2+\ln x\geq 0$
 \end{enumerate}
\end{multicols}
\vspace{0.01cm}
\begin{sol}
\eqref{iteqln0} $S=\{-e^2\}$
\eqref{iteqln1} $S=\{\pm 1\}$ Obs: aqui, se escrever $\ln(x^2)=2\ln x$, perde-se
a solução negativa! Lembre que $\ln (x^y)=y\ln x$ vale se $x$ é positivo! Então
aqui poderia escrever $\ln(x^2)=\ln (|x|^2)=2\ln |x|$.
\eqref{iteqln11} $S=\{e^{-\tfrac15}-1\}$
\eqref{iteqln2} $S=\varnothing$
\eqref{iteqln3} $S=...$
\eqref{iteqln4} $S=(-\infty,\tfrac34)$
\eqref{iteqln5} $S=(-\infty,-\tfrac13)\cup (-\tfrac18,\infty)$
\eqref{iteqln6} $S=(-\infty,-\tfrac23)\cup (\tfrac12,\infty)$
\eqref{iteqln7} $S=\{-5,-2,-1,2\}$
\eqref{iteqln8} $S=(0,e^{-1}]\cup [1,+\infty)$
\end{sol}
\end{exo}

\begin{exo}
Determine quais das funções abaixo são pares, ímpares, ou nem par e nem
ímpar\index{função! par}.
\begin{multicols}{4}
\begin{enumerate}
\item\label{itparidadelog1} $e^x$
\item\label{itparidadelog2} $\ln x$
\item\label{itparidadelog3} $e^{x^2-x^4}$
\item\label{itparidadelog4} $e^x+e^{-x}$
\item\label{itparidadelog5} $e^x-e^{-x}$
\item\label{itparidadelog6} $\ln (1-|x|+x^2)$
\item\label{itparidadelog7} $\frac{e^{x^2}+e^{|x|}}{x^4+x^6+1}$
\end{enumerate}
\end{multicols}
\vspace{0.01cm}
\begin{sol}
\eqref{itparidadelog1} Nem par nem ímpar.
\eqref{itparidadelog2} Nem par nem ímpar (aqui, tem um problema de domínio: o
domínio do $\ln$ é $(0,\infty)$, então nem faz sentido verificar se 
$\ln (-x)=\ln (x)$).
\eqref{itparidadelog3} Par: $e^{(-x)^2-(-x)^4}=e^{x^2-x^4}$.
\eqref{itparidadelog4} Par.
\eqref{itparidadelog5} Ímpar.
\eqref{itparidadelog6} Par (cuidado com o domínio: $\bR\setminus\{0\}$)
\eqref{itparidadelog7} Par.
\end{sol}
\end{exo}

\begin{exo} 
Esboce o gráfico da função $g(x)\pardef \frac{1}{(x-1)^2}$.
Em seguida, esboce o gráfico da função $f(x)\pardef (\ln \circ g)(x)$ somente a
partir das propriedades do gráfico de $g$ e das propriedades do $\ln$.
\begin{sol}
Sabemos que o gráfico de $\frac{1}{(x-1)^2}$ é obtido transladando o de
$\frac{1}{x^2}$ de uma unidade para direita.
\begin{center}
\begin{bmlimage}\begin{tikzpicture}[scale=0.7]
\draw [ ->] (0,-1)--(0,2) node[left]{$y$};
\draw [ ->] (-3,0)--(4,0) node[right]{$x$};
\pgfmathsetmacro{\a}{3.5}
\pgfmathsetmacro{\e}{0.5}
\draw [thick, domain=-2:1-\e, samples=100] plot (\x,{1/((\x-1)^2)});
\draw [thick, domain=1+\e:4, samples=100] plot (\x,{1/((\x-1)^2)});
%\draw [thick, domain=0.5:\a, samples=100] plot (\x,{\x^{-2}});
%\draw[dotted] (-3,0)--(4,0);
\draw[dotted] (1,-1)--(1,3);
\fill (0,1) circle (0.55mm);
\fill (2,1) circle (0.55mm);
\end{tikzpicture}\end{bmlimage}
\end{center}
Ao tomar o logaritmo de $g(x)$, $f(x)=\ln(g(x))$, é bom ter o gráfico da função
$\ln x$ debaixo dos olhos. 
Quando $x$ é grande (positivo ou negativo),
$g(x)$ é próximo de zero, logo $f(x)$ vai tomar valores grandes e negativos.
Quando $x$ cresce, $g(x)$ cresce até atingir o valor $1$ em $x=0$, logo $f(x)$
cresce até atingir o valor $0$ em $0$. Entre $x=0$ e $x=1$ ($x<1$), $g(x)$
diverge, logo $f(x)$ diverge também.
Entre $x=1$ ($x>1$) e $x=2$, $g(x)$ decresce até atingir o valor $1$ em $x=2$,
logo $f(x)$ decresce até atingir o valor $0$ em $x=2$. 
Para $x>2$, $g(x)$ continua decrescendo, e toma valores que se aproximam de $0$,
logo $f(x)$ se toma valores negativos, e decresce para tomar valores
arbitrariamente grandes negativos.
\begin{center}
\begin{bmlimage}\begin{tikzpicture}[scale=0.7]
\draw [ ->] (0,-1)--(0,2) node[left]{$y$};
\draw [ ->] (-3,0)--(4,0) node[right]{$x$};
\pgfmathsetmacro{\a}{3.5}
\pgfmathsetmacro{\e}{0.5}
\draw [color=gray!60, domain=-2:1-\e, samples=100] plot (\x,{1/((\x-1)^2)});
\draw [color=gray!60, domain=1+\e:4, samples=100] plot (\x,{1/((\x-1)^2)});
\pgfmathsetmacro{\e}{0.2}
\draw [thick, domain=-1.8:1-\e, samples=100] plot (\x,{ln(1/((\x-1)^2))});
\draw [thick, domain=1+\e:3.8, samples=100] plot (\x,{ln(1/((\x-1)^2))});
\draw[dotted] (1,-1)--(1,3);
\fill (0,0) circle (0.55mm);
\fill (2,0) circle (0.55mm);
\end{tikzpicture}\end{bmlimage}
\end{center}
Observe que é também possível observar que $f(x)=-2\ln|x-1|$, e obter o seu
gráfico a partir do gráfico da função $\ln |x|$!

\end{sol}
\end{exo}

\begin{exo}
Determine o conjunto imagem da função $f(x)\pardef
\frac{e^{x}}{e^{x}+1}$.
\begin{sol}
Lembramos que $y\in \bR$ pertence ao conjunto imagem de $f$ se e somente
se existe um $x$ (no domínio de $f$) tal que $f(x)=y$.
Ora $\frac{e^{x}}{e^{x}+1}=y$ implica $e^x=\frac{y}{1-y}$. Para ter uma
solução, é necessário ter $\frac{y}{1-y}>0$. É fácil ver que
$\frac{y}{1-y}>0$ se e somente se $y\in (0,1)$. Logo, 
$\mathrm{Im}(f)=(0,1)$. 
\end{sol}
\end{exo}

\section[Funções hiperbólicas]{Funções trigonométricas hiperbólicas}\label{Sec:FuncTrigHiperb}
\index{funções trigonométricas! hiperbólicas}
A exponencial na base $e$ 
permite definir três funções fundamentais chamadas respectivamente 
\grasA{seno
hiperbólico}\index{seno! hiperbólico}, \grasA{cosseno 
hiperbólico}\index{cosseno!
hiperbólico} e \grasA{tangente hiperbólica}\index{tangente! hiperbólica}:
\eq{
\boxed{
\senh x\pardef \frac{e^x-e^{-x}}{2}\,,\quad 
\cosh x\pardef \frac{e^x+e^{-x}}{2}\,,\quad 
\tanh x\pardef
\frac{e^x-e^{-x}}{e^x+e^{-x}}\,.}
}

Para entender a origem da mistura de terminologia (nada óbvia a priori!) 
usada para definir essas funções, ``trigonometria'' e 
``hipérbole''\index{hipérbole}, o leitor
interessado poderá consultar o texto da Professora Sônia Pinto de
Carvalho~\footnote{\texttt{www.mat.ufmg.br/$\sim$sonia/pubensino.htm}}.
Estudaremos mais a fundo as propriedades dessas funções nos próximos 
capítulos; por enquanto faremos somente alguns comentários.\\

Observe primeiro que
$$\tanh x=\frac{\senh x}{\cosh x}\,, $$
Também,
$$
\Bigl( \frac{e^x+e^{-x}}{2}\Bigr)^2-\Bigl(
\frac{e^x-e^{-x}}{2}\Bigr)^2=\frac{e^{2x}+2+e^{-2x}}{4}-
\frac{e^{2x}-2+e^{-2x}}{4}=1\,,
$$
portanto vale a seguinte identidade,
\eq{\label{eqtrigohiperb2}\cosh^2x-\senh^2x=1\,,}
que tem uma semelhança com \eqref{eqtrigo2}: $\cos^2x +\sen^2x=1$. 

\begin{exo}
Mostre que $\cosh x$ é uma função par, e que $\senh x$ e $\tanh x$ são ímpares.
\begin{sol}
Por exemplo, $\senh (-x)=\frac{e^{(-x)}-e^{-(-x)}}{2}=\frac{e^{-x}-e^{x}}{2}
=-\frac{e^{x}-e^{-x}}{2}=-\senh (x).$
\end{sol}
\end{exo}

Os gráficos das funções hiperbólicas serão estudados em detalhes nos próximos
capítulos. Mencionaremos somente o seguinte fato:
o gráfico da 
função $\cosh x$ aparece a cada vez que uma corda\index{corda} é pendurada entre 
dois pontos $A$ e $B$:
\begin{center}
\begin{bmlimage}\begin{tikzpicture}
\pgfmathsetmacro{\a}{-1.3};
\pgfmathsetmacro{\b}{1};
\coordinate (A) at (\a,{(exp(\a)+exp(-1*\a))/2});
\coordinate (B) at (\b,{(exp(\b)+exp(-1*\b))/2});
\fill[color=gray!30] (\a-1,0) rectangle (A);
\fill[color=gray!30] (\b+1,0) rectangle (B);
\draw (\a-1,0) rectangle (A);
\draw (\b+1,0) rectangle (B);
\draw [thick, domain=\a:\b, samples=100] plot (\x,{(exp(\x)+exp(-1*\x))/2});
%\coordinate (pB) at (\b,0);
\fill (A) circle (0.50mm);
\fill (B) circle (0.50mm);
\draw (A) node[above]{$A$};
\draw (B) node[above]{$B$};
\draw[thick] (\a-1.5,0)--(\b+1.5,0);
%\draw[thick] (\b,0)--(B);
\end{tikzpicture}\end{bmlimage}
\end{center}



%%%NAO USADO:
%\section{Diagramas $\log-\log$}




% !TeX spellcheck = pt_BR
% !TEX encoding = UTF-8 Unicode

\chapter{Limites}\label{Cap:Limites}

\ifdefined\updateans
% Only need to run once in a lifetime, when the file ans.tex needs to be updated.
\Writetofile{ans}{\protect\section*{Capítulo \ref{Cap:Limites}}}
\fi

Nesse capítulo começaremos o estudo do conceito fundamental do Cálculo:
\emph{limite}.\\

A ordem na qual a matéria é apresentada aqui é um pouco diferente 
da ordem usual. Na Seção \ref{Sec:Limnoinf} 
começaremos descrevendo os limites \emph{no infinito}, isto é, estudaremos 
o comportamento dos valores de uma função $f(x)$ 
quando $x$ é grande (positivo ou negativo).
Depois, na Seção \ref{Sec:Lim}, olharemos o que acontece 
quando $x\to a$, onde $a$ é um ponto da reta real.
A noção de \emph{continuidade} será considerada na Seção 
\ref{Sec:Continuidade}.

\section{Limites $\lim_{x\to \pm \infty}f(x)$}\label{Sec:Limnoinf}

A primeira informação que será extraída sobre uma função será o seu
comportamento no infinito. 
Portanto, 
\index{limite!$x\to \infty$}
começaremos estudando os valores de uma função $f(x)$, quando $x$
fica
arbitrariamente grande e positivo, ou arbitrariamente grande e
negativo.
%O nosso primeiro objetivo será de ver se, em cada um desses casos, os 
%valores de $f(x)$
%se aproximam de algum valor específico. 

\subsection{Introdução}
Apesar de elementar, o nosso primeiro exemplo será um dos mais
importantes, pois ele nós permite introduzir pela primeira vez a
ideia de \emph{tender a zero}.

\begin{ex}\label{expl_LIM_unsurx}
Já montamos o gráfico da função $\tfrac{1}{x}$ no Capítulo
\ref{Cap:Funcoes}. Consideremos aqui o que acontece com $\frac1x$ quando
$x$ toma valores grandes, positivos ou negativos:

\begin{center}
\begin{bmlimage}\begin{tikzpicture}[scale=0.9]
\pgfmathsetmacro{\a}{5.5}
\draw [thick, domain=-\a:-0.32, samples=100] plot (\x,{1/(\x^1)});
\draw [thick, domain=0.32:\a, samples=100] plot (\x,{1/(\x^1)});
\draw [ ->] (-\a-0.3,0)--(\a+0.3,0) node[right]{$x$};
\draw [ ->] (0,-3)--(0,{3}) node[left]{$\tfrac{1}{x}$};
\pgfmathsetmacro{\x}{4.5};
\draw[dotted] (\x,0) node[below]{$x$} --(\x,{1/\x})--(0,{1/\x})
node[left]{$\scriptstyle{{1}/{x}}$};
\fill (\x,{1/\x}) circle (0.45mm);
\pgfmathsetmacro{\x}{4.5};
\draw[dotted] (-\x,0) node[above]{$x$} --(-\x,{-1/\x})--(0,{-1/\x})
node[right]{$\scriptstyle{{1}/{x}}$};
\fill (-\x,{-1/\x}) circle (0.45mm);
\end{tikzpicture}\end{bmlimage}
\end{center}

Quando $x$ se afasta da origem, tomando valores grandes e
positivos, o que será denotado $x\to+\infty$, 
vemos que \emph{os valores de $\tfrac{1}{x}$ tendem a
zero}. Para ilustrar isso podemos observar os valores da função quando a variável
toma por exemplo os valores $x=10$, $x=100$, $x=1000$, ...:
\begin{center}
\begin{tabular}{c|c|c|c|c}
$x=$&10&100&1000&10'000\\
\hline
$\frac{1}{x}=$&$0.1$&$0.01$&$0.001$&$0.0001$
\end{tabular}
\end{center}
Na verdade, pegando uma outra seqüência de números, por exemplo $x=4$,
$x=8$, $x=16$, $x=32$, ..., observaríamos também que os valores se
aproximam de zero.
O fato de $\frac{1}{x}$ se aproximar de zero à medida
que $x$ aumenta é obviamente devido
ao fato da divisão de $1$ por um número grande resultar em 
um número pequeno.

Vamos ser agora um pouco mais precisos, e render \emph{quantitativa} a
seguinte afirmação:
\emph{tomar $x$ grande o suficiente permite tornar $\frac1x$ 
arbitrariamente pequeno}. 
Vamos proceder da seguinte maneira. Primeiro
escolhamos
um número positivo arbitrário, pequeno, que chamaremos de
\grasA{tolerância}. Por exemplo:
$0.000002$. Em seguida, façamos a pergunta:
quão grande $x$ precisa ser tomado
para tornar
$\frac{1}{x}$ menor que a tolerância escolhida, isto é
\begin{equation}\label{eq_LIM_menorquepequ} 
0\leq \frac{1}{x}\leq 0.000002\,\quad ?
\end{equation}
Para responder, basta resolver a desigualdade acima.
Multiplicando ambos lados por $x$ (pode ser feito sem
mudar o sentido da
desigualdade, já que $x$ é positivo), e dividindo ambos lados por
$0.000002$,
\[ \frac{1}{0.000002}\leq x\,.
\]
Como $\frac{1}{0.000002}=500000$, isso significa que 
qualquer número $x$ que satisfaz 
\[x\geq 500000,,
\]
também satisfaz \eqref{eq_LIM_menorquepequ}.
Isto é, tomar um número $x$ qualquer maior ou igual a $5000000$
garante que a sua
imagem (pela função $\frac1x$) será contida entre $0$ e $0.000002$ (a
tolerância que fixamos).

O importante é que o mesmo raciocíno pode ser feito com qualquer
tolerância, mesmo muito pequena. Por exemplo, podemos escolher uma
tolerância igual a
$0.00000000123$, e verificar que todos os $x$ grandes, dessa vez
$x\geq 813008131$, satisfazem
\[ 
0\leq \frac{1}{x}\leq 0.00000000123\,.
\]

Vemos que o mesmo argumento funcionará com qualquer tolerância.
Logo, em vez de tomar valores particulares
para a tolerância, podemos simplesmente dar um nome a ela: $\epsilon$. 
Seja então $\epsilon>0$ uma tolerância qualquer (subentendido: tão pequena
quanto quisermos, mas \emph{fixa}). 
Podemos então procurar os $x>0$ que satisfazem
\[0\leq \frac{1}{x}\leq \epsilon\,.\]
Resolvendo essa desigualdade obtemos: 
\[x\geq \tfrac1\epsilon\,.\]

\begin{center}
\begin{bmlimage}\begin{tikzpicture}
\pgfmathsetmacro{\e}{0.4};
\pgfmathsetmacro{\N}{1/\e};
\draw[dashed] (0,\e)--(8,\e);
\draw[<->] (-0.2,0)--(-0.2,\e) node[midway, left]{$\epsilon$};
\fill[color=gray!20] (\N,0) rectangle (8,\e);
\pgfmathsetmacro{\a}{2}
\draw [color=gray!80, domain=0.5:6, samples=100] plot
(\x,{1/(\x^1)});
\draw [ ->] (-0.1,0)--(8.5,0);
\draw [ ->] (0,-0.3)--(0,2) node[left]{$\tfrac{1}{x}$};
\draw [thick, domain=\N:8, samples=100] plot
(\x,{1/(\x^1)});
\draw[dotted] (\N,0) node[below]{$\tfrac{1}{\epsilon}$}--(\N,\e);
\draw[dotted] (6,0)node[below]{$x$}--(6,{1/6})--(0,{1/6});
\end{tikzpicture}\end{bmlimage}
\end{center}
O fato de ser possível mostrar que
para uma tolerância arbitrariamente pequena,
existe sempre um intervalo infinito de valores de $x$ para os quais a desigualdade
$0\leq \frac1x\leq \epsilon$
é verdadeira é o que define rigorosamente o 
\grasA{limite}. A seguinte notação costuma ser
usada:
$$
\boxed{\lim_{x\to +\infty}\frac{1}{x}=0\,.}$$
Leia-se: \emph{o limite de $\frac1x$, quando $x$ tende a $+\infty$, é
igual a $0$}, ou \emph{
$\frac{1}{x}$ tende a zero quando $x$ tende a $+\infty$}. 
Enfatizemos que isso não significa, de forma alguma,
que $\tfrac{1}{x}$ é \emph{igual} a zero quando $x$ é grande, mas somente que
\emph{se aproxima arbitrariamente perto} de zero à medida que $x$ vai
crescendo.\\

Consideremos agora o que acontece com $\tfrac1x$ quando $x\to-\infty$. 
Dessa vez a função tende a
zero também mas com valores negativos, já que $\tfrac1x<0$ se $x<0$
(dê uma olhada na figura do início do exemplo). 
Logo, gostaríamos de fixar uma tolerância $\epsilon>0$, e achar os $x$ que satisfazem 
\[ 
-\epsilon\leq \frac1x\leq 0\,.
\]
Desta vez, essa desigualdade é satisfeita para qualquer 
$x\leq -\frac1\epsilon$. Escreveremos também:
$$\boxed{\lim_{x\to -\infty}\frac{1}{x}=0\,.}$$
Poderiamos ter calculado os dois limites de uma vez, $x\to -\infty$ e
$x\to+\infty$, 
observando simplesmente que para um $\epsilon>0$ fixo,
é possivel garantir 
\[\Bigl|\frac{1}{x}\Bigr|\leq \epsilon\] 
para todo $x$ a distância maior que $\frac1\epsilon$ da origem, isto é
$|x|\geq \frac1\epsilon$.
\end{ex}

O Exemplo \ref{expl_LIM_unsurx} levou a definir precisamente o que
significa \emph{$\tfrac1x$ tender a zero quando $x\to\infty$}. 
Podemos agora considerar o caso geral:

\begin{defin}\label{defin_tender_a_zero}
Diremos que $f(x)$ \grasA{tende a zero
quando $x\to \infty$}
se para qualquer tolerância $\epsilon>0$, é possível garantir que 
\begin{equation}\label{eq_LIM_tendeazeroo}
|f(x)|\leq \epsilon\quad \text{ para todo $x>0$ suficientemente grande.} 
\end{equation}
Escreve-se:
\[ 
\lim_{x\to\infty}f(x)=0\,.
\]
\end{defin}
O valor absoluto foi usado em
\eqref{eq_LIM_tendeazeroo}, pois $f(x)$ pode tender a zero sem que o
seu sinal seja sempre $\geq 0$ ou $\leq 0$ (como foi visto no caso de
$\tfrac1x$).
Para ver um caso que em uma função tende a zero com o seu sinal
oscilando, veja a figura do 
Exemplo \ref{ex_LIM_sinxsurx} na página \pageref{ex_LIM_sinxsurx}.

\begin{exo}
Usando a definição acima, mostre que 
\[ 
\lim_{x\to\infty}\frac{500}{x}=0\,,\quad
\lim_{x\to\infty}\frac{9}{x^2}=0\,,\quad
\lim_{x\to\infty}\frac{2}{3-x}=0\,.
\]
\begin{sol}
Em cada caso, fixemos uma tolerância $\epsilon>0$ e procuremos resolver uma
desigualdade elementar.
(1) Observe que $\frac{500}{x}>0$ para todo $x>0$. Seja $\epsilon>0$. Procuremos
quais são os $x>0$ grandes, positivos, para os quais
$0<\frac{500}{x}\leq \epsilon$.
Resolvendo a desigualdade achamos: $x\geq \frac{500}{\epsilon}$.
(2) Seja $\epsilon>0$. Procuremos resolver $0<\frac{9}{x^2}\leq \epsilon$, que dá $x\geq
\frac{3}{\sqrt{\epsilon}}$.
(3) Observe que $\frac{2}{3-x}<0$ quando $x$ for grande, positivo. 
Fixemos $\epsilon>0$, e procuremos resolver
$-\epsilon\leq \frac{2}{3-x}<0$, e achamos $x\geq 3+\frac{2}{\epsilon}$.
\end{sol}
\end{exo}
\begin{ex}\label{Ex:razaomuitosimples}
Consideremos em seguida o comportamento de
$$\frac{x}{x+2}\,,\quad \text{quando $x\to + \infty$}\,.$$
Para ver o que está acontecendo, 
calculemos primeiro a função para alguns valores de $x$, grandes e
positivos:
\begin{center}
\begin{tabular}{c|c|c|c|c}
$x=$&10&100&1000&10'000\\
\hline
$\frac{x}{x+2}\simeq $&$0.8333$&$0.9803$&$0.9980$&$0.9998$
\end{tabular}
\end{center}
Isso parece indicar que $\frac{x}{x+2}$ se
aproxima de $1$ quando $x\to+\infty$. 
Esse fato pode ser observado no traço do gráfico da função (feito com um
computador):

\begin{center}
\begin{bmlimage}\begin{tikzpicture}[scale=1.5]
\pgfmathsetmacro{\a}{7}
\draw[dashed] (0,1) node[left]{$\scriptstyle{1}$}--(\a,1);
\draw [thick, domain=-1:\a, samples=100] plot
(\x,{(\x)/(\x+2)});
\draw [ ->](-1,0)--(\a+0.3,0);
\draw [->](0,-1)--(0,1.3);
\pgfmathsetmacro{\x}{4.5};
\draw[dotted] (\x,0) node[below]{$x$}
--(\x,{(\x)/(\x+2)})--(0,{(\x)/(\x+2)})
node[left]{$\scriptstyle{\frac{x}{x+2}}$};
\fill (\x,{(\x)/(\x+2)}) circle (0.45mm);
\end{tikzpicture}\end{bmlimage}
\end{center}

Gostaríamos então de dar um sentido ao seguinte símbolo:
\begin{equation}\label{eq_ftendeaUM}
\lim_{x\to \infty}\frac{x}{x+2}=1\,.
\end{equation}
A dificuldade, aqui, é que quando $x$ toma valores grandes, $\frac{x}{x+2}$
é uma divisão de dois números grandes, o que representa uma forma de
\emph{indeterminação} (falaremos mais sobre isso depois). 
\index{indeterminação} No entanto, mostraremos que $\frac{x}{x+2}$ tende a $1$,
mostrando que $\frac{x}{x+2}-1$ tende a zero no sentido da Definição
\ref{eq_LIM_tendeazeroo}.

Fixemos uma tolerância 
$\epsilon>0$, e procuremos saber se dá para garantir que 
\begin{equation}\label{eq_distaunpet} 
\Bigl|\frac{x}{x+2}-1\Bigr|\leq \epsilon\,,
\end{equation}
para todo $x$ suficientemente grande.
Comecemos explicitando a diferença:
\begin{equation}\label{eq_calculdifff}\Bigl|\frac{x}{x+2}-1\Bigr|=
\Bigl| \frac{-2}{x+2}\Bigr|=\frac{2}{|x+2|}=\frac{2}{x+2}\,.
\end{equation}
Os valores absolutos foram removidos na última igualdade,
já que $x$ será tomado grande, positivo, o que
implica $x+2>0$.
%Quando escrita dessa forma, parece claro que a diferença tende a zero quando
%$x\to+\infty$.
Agora, \eqref{eq_distaunpet} será satisfeita se
saber se dá para garantir que 
\[ 
\frac{2}{x+2}\leq \epsilon\,.
\]
Resolvendo a desigualdade obtemos: 
\[x\geq \frac{2}{\epsilon}-2\,.\]
Como isso pode ser feito com qualquer tolerância,
conseguimos provar \eqref{eq_ftendeaUM}.
\end{ex}

Vejamos agora um exemplo em que o comportamento quando $x\to\infty$ pode
ser diferente do comportamento quando $x\to-\infty$.

\begin{ex}\label{ex_LIM_deuxassymp}
Considere $f(x)\pardef \frac{|x|}{x+1}$.
Usando a definição do valor absoluto, vemos que essa função é dada por
\[ 
f(x)=
\begin{cases}
\frac{x}{x+1}&\text{ se }x\geq 0\,,\\
0&\text{ se }x= 0\,,\\
\frac{-x}{x+1}&\text{ se }x< 0\,.
\end{cases}
\]
Logo,
$$\lim_{x\to \infty}f(x)=\lim_{x\to\infty}\frac{x}{x+1}\,,
$$
e é fácil mostrar que esse limite vale $1$.
Por outro lado,
$$\lim_{x\to -\infty}f(x)=\lim_{x\to-\infty}\frac{-x}{x+1}\,,
$$
e esse limite se calcula facilmente, e é igual a $-1$.
\begin{center}
\begin{bmlimage}\begin{tikzpicture}
\pgfmathsetmacro{\a}{5.5}
 \draw[dashed] (0,1) node[left]{$1$}--(\a,1);
 \draw[dashed] (-\a,-1)--(0,-1) node[right]{$-1$} ;
\draw [thick, domain=-\a:-1.8, samples=100] plot
(\x,{(abs(\x))/(\x+1)});
\draw [thick, domain=-0.7:\a, samples=100] plot
(\x,{(abs(\x))/(\x+1)});
\draw [ ->] (-\a-0.3,0)--(\a+0.3,0) node[right]{$x$};
\draw [ ->] (0,-1.8)--(0,{3})
node[right]{$f(x)=\tfrac{|x|}{x+1}$};
\pgfmathsetmacro{\x}{4.5};
\end{tikzpicture}\end{bmlimage}
\end{center}
\end{ex}

\begin{obs}
Podemos ver, graças aos gráficos montados acima com um computador, 
que a existência dos limites $\lim_{x\to
\pm\infty}f(x)$ implica que o gráfico da função se aproxima,
longe da origem, de uma reta horizontal (que será chamada de
\emph{assíntota horizontal}).
Mas é claro que \emph{aprender a esboçar gráficos} 
é um dos objetivos desse curso, 
então o uso de gráficos até agora deve ser considerado somente como
uma ajuda para entender a definição de limite.
\end{obs}

\begin{obs}
Em geral, um limite \emph{nem sempre existe}. 
Por exemplo,
``$\lim_{x\to \infty}\sen x$'' não existe, pois à medida que $x$
cresce, $\sen x$ oscila em torno de $0$, sem \emph{tender} a nenhum
valor.
Um limite pode também \emph{ser infinito}, como veremos
mais adiante.  
\end{obs}

\begin{exo}
Explique porque que $\sen x$ \emph{não} tende a zero quando $x\to \infty$
no sentido da Definição \ref{defin_tender_a_zero}.
Dica:
distinguir os casos $\epsilon\geq 1$ e $0<\epsilon<1$.
\end{exo}


\subsection{A definição de limite}

Mostramos no Exemplo \ref{Ex:razaomuitosimples} que
$f(x)=\frac{x}{x+2}$ tende a $1$ quando $x\to\infty$, provando que a
\emph{diferença} $|\frac{x}{x+2}-1|$ se torna sempre menor a medida que
$x$ cresce. 
Em geral, dizer que os
valores de uma função $f(x)$ se aproximam arbitrariamente perto de um
valor $\ell$ quando $x$ é grande, é equivalente a dizer que
\emph{$|f(x)-\ell|$ se torna arbitrariamente pequeno desde que $x$ seja 
grande o suficiente}. Em outras palavras,

\begin{defin}\label{Def:limiteinfinito}
\index{limite!$x\to \infty$}
Diz-se que $f(x)$ \grasA{tende a $\ell$ quando $x\to \infty$}, e
escreve-se 
\[\lim_{x\to \infty}f(x)=\ell\,,\] 
(ou às vezes $f(x)\to \ell$ se não tiver
ambiguidade) se $f(x)-\ell$ tende a zero, isto é se 
para todo $\epsilon>0$ (subentendito: arbitrariamente pequeno, mas fixo)
existir um $N$ tal que se $x\geq N$, então 
$$|f(x)-\ell|\leq \epsilon\,.$$
A definição de $\lim_{x\to -\infty}f(x)=\ell$ é parecida, mas
``$x\geq N$'' é trocado por ``$x\leq -N$''.
\end{defin}

\begin{obs}
É sempre subentendido, ao escrever ``$\lim_{x\to \infty}f(x)$'', que
$f(x)$ é bem definida para todo $x$ suficientemente grande.
\end{obs}

\begin{obs}
Em geral, o número $N$ associado a um
$\epsilon>0$ não é único. De fato, suponha que foi mostrado que para
um certo $\epsilon>0$, existe um $N>0$ tal que $|f(x)-\ell|\leq
\epsilon$ para todos os $x\geq N$. Então, definindo por exemplo $N'=3N$, 
a desigualdade $|f(x)-\ell|\leq \epsilon$ vale também se $x\geq
N'$, obviamente. O que importa é ser capaz \emph{achar pelo menos um
$N$}, não importa quão grande for.
\end{obs}

\begin{exo}\label{Ex:razaosimples}
Usando o método acima, mostre que
$$\lim_{x\to+\infty}\frac{2x-1}{3x+5}=\frac23\,,
\quad \lim_{x\to-\infty}\frac{2x-1}{3x+5}=\frac23
\,.$$
\begin{sol}
%Primeiro, precisamos decidir qual deve ser o valor do limite. Podemos por exemplo
%observar os valores da função para alguns valores de $x$, grandes e
%positivos:
%\begin{center}
%\begin{tabular}{c|c|c|c|c}
%$x=$&10&100&1000&10'000\\
%\hline
%$\frac{2x-1}{3x+5}\simeq $&$0.5428$&$0.6524$&$0.6652$&$0.6665$
%\end{tabular}
%\end{center}
%Esses números parecem indicar que $\frac{2x-1}{3x+5}$ se
%aproxima de $0.6666\dots=\tfrac23$. 
%Podemos também argumentar da seguinte maneira: na 
%fração $\tfrac{2x-1}{3x+5}$, quando $x$ é grande, 
%o numerador $2x-1$ e o denominador $3x+5$ são ambos grandes.
%No entanto, o ``$-1$'' no
%numerador se torna desprezível comparado com $2x$ (que é
%\emph{grande}!), logo $2x-1$ pode ser aproximado por $2x$. No denominador,
%o ``$5$'' é desprezível comparado com o ``$3x$'', logo
%$3x+5$ pode ser aproximado por $3x$. Portanto, para $x$ grande, 
%$$
%\frac{2x-1}{3x+5}\quad\text{ pode ser aproximado por } 
%\quad\frac{2x}{3x}=\frac23\,.
%$$
%Atenção: esse tipo de raciocínio ajuda a adivinhar qual deve ser o valor do
%limite (no caso
%$\tfrac23$) quando $x\to \infty$, mas não sempre funciona, e é ainda preciso
%\emph{mostrar} que o limite é $\tfrac23$ mesmo.

%Para tornar o argumento rigoroso, basta colocar $x$ em evidência no
%numerador e denominador, e \emph{simplificar por $x$}:
%$$
%\frac{2x-1}{3x+5}=\frac{x(2-\frac{1}{x})}{x(3+\frac{5}{x})}=
%\frac{2-\frac{1}{x}}{3+\frac{5}{x}}\,.
%$$
%Agora vemos que quando $x\to\infty$, o numerador dessa fração, 
%$2-\frac{1}{x}$, tende a
%$2$ (pois já sabemos que $\frac{1}{x}$ tende a zero) e que o 
%denominador, $3+\frac{5}{x}$, tende a $3$. 
%Assim podemos escrever (as operações com limites serão justificadas mais 
%tarde)
%$$\lim_{x\to\infty}f(x)=
%\lim_{x\to\infty}\frac{2-\frac{1}{x}}{3+\frac{5}{x}}
%=\frac{\lim_{x\to\infty}(2-\frac{1}{x})}{\lim_{x\to\infty}(3+\frac{5}{
%x})}
%=\frac{2-\lim_{x\to\infty}\frac{1}{x}}{3+5\lim_{x\to\infty}\frac{1}{x}
%}=\frac{2-0}{3+5\cdot 0}=\frac{2}{3}\,.
%$$
%
Vamos mostrar que
\begin{equation}\label{eq_ftendeadoistercos}
\lim_{x\to \infty}\frac{2x-1}{3x+5} =\frac23\,.
\end{equation}
Para isso fixemos uma tolerância $\epsilon>0$ (arbitrariamente pequena), 
e verifiquemos que
\[ 
\Bigl|\frac{2x-1}{3x+5}-\frac23\Bigr|\leq \epsilon
\]
vale sempre que $x$ for tomado suficientemente grande.
Para começar, calculemos o valor absoluto da diferença:
\begin{equation}\label{eq_calculdifff}
\Bigl|\frac{2x-1}{3x+5}-\frac23\Bigr|=
\Bigl|\frac{-13}{3(3x+5)}\Bigr|=\frac{13}{3} 
\frac{1}{3x+5}\,.
\end{equation}
Agora resolvemos a desigualdade (para $x$ grande, positivo)
\[ \frac{13}{3} \frac{1}{3x+5}
\leq \epsilon\,,
\]
e achamos a solução: 
$x\geq 13\epsilon-15$. Assim, provamos
\eqref{eq_ftendeadoistercos}.
Deixamos o leitor tratar o limite
$x\to-\infty$.
Usando um computador, podemos verificar que de fato, os valores 
de $\frac{2x-1}{3x+5}$, longe da origem, se aproximam
de $\tfrac23$:

\begin{center}
\begin{bmlimage}\begin{tikzpicture}
\pgfmathsetmacro{\a}{5.5}
\draw[dashed] (-\a,0.6666)--(\a,0.6666) node[above]{$y=\tfrac{2}{3}$};
\draw [thick, domain=-\a:-2.2, samples=100] plot
(\x,{(2*\x-1)/(3*\x+5)});
\draw [thick, domain=-0.8:\a, samples=100] plot
(\x,{(2*\x-1)/(3*\x+5)});
\draw [ ->] (-\a-0.3,0)--(\a+0.3,0) node[right]{$x$};
\draw [ ->] (0,-1)--(0,{3})
node[right]{$f(x)=\tfrac{2x-1}{3x+5}$};
\pgfmathsetmacro{\x}{4.5};
\draw[dotted] (\x,0) node[below]{$x$}
--(\x,{(2*\x-1)/(3*\x+5)})--(0,{(2*\x-1)/(3*\x+5)})
node[left]{$\scriptstyle{f(x)}$};
\fill (\x,{(2*\x-1)/(3*\x+5)}) circle (0.45mm);
\pgfmathsetmacro{\x}{-4.5};
\draw[dotted] (\x,0) node[below]{$x$}
--(\x,{(2*\x-1)/(3*\x+5)})--(0,{(2*\x-1)/(3*\x+5)})
node[right]{$\scriptstyle{f(x)}$};
\fill (\x,{(2*\x-1)/(3*\x+5)}) circle (0.45mm);
\end{tikzpicture}\end{bmlimage}
\end{center}
\end{sol}
\end{exo}

Em termos do gráfico de $f$, $f(x)\to \ell$ 
deve ser interpretado dizendo que a medida que $x$ aumenta, 
a \emph{distância} entre o gráfico de $f$ e a reta de equação 
$y=\ell$ tende a zero:
$$d(f(x),\ell)\to 0\,.$$
Se pelo menos um dos limites $\lim_{x\to \infty}f(x)$,
$\lim_{x\to -\infty}f(x)$, existe e vale $\ell$, 
diz-se então que \grasA{a reta $y=\ell$ é assíntota horizontal de
$f$}.
\index{assíntota!horizontal}
Por exemplo, a função $f(x)=\frac{x}{x+2}$ do Exemplo
\ref{Ex:razaomuitosimples} tem uma assíntota horizontal $y=1$, que
descreve o comportamento quando $x\to-\infty$ e
$x\to+\infty$. 
A função $f(x)=\frac{|x|}{x+1}$ do Exemplo \ref{ex_LIM_deuxassymp},
por sua vez, tem a assíntota $y=-1$ que descreve o comportamento
quando $x\to-\infty$, e
a assíntota $y=+1$ que descreve o comportamento quando $x\to+\infty$.

%\begin{ex}
%Justifiquemos agora o valor limite do Exemplo \ref{Ex:razaosimples}, 
%usando a definição.
%Se $x>0$, podemos usar \eqref{eq_calculdifff} para calcular 
%$$|f(x)-\tfrac23|=
%\Bigl|\frac{2x-1}{3x+5}-\frac23\Bigr|=\frac{13}{3} 
%\frac{1}{|3x+5|}=\frac{13}{3} 
%\frac{1}{3x+5}\,.$$
%O valor absoluto foi retirado, já que $3x+5>0$ para todo $x$
%suficientemente grande. 
%Agora, é claro 
%que $\frac{13}{3}\frac{1}{3x+5}$ se torna arbitrariamente pequeno a
%medida que $x$ cresce.
%Fixemos então um $\epsilon>0$ e façamos a pergunta: \emph{quão grande
%$x$ precisa ser para garantir que 
%\begin{equation}\label{eqrteze}
%\frac{13}{3}\frac{1}{3x+5}\leq
%\epsilon\,?
%\end{equation}
%}
%Podemos resolver essa última inequação, isolando $x$, e obtemos
%\[
%x\geq \frac{1}{3}\Bigl(\frac{13}{3\epsilon}-5\Bigr)\,.
%\]
%Agora, chamando 
%$N\pardef\frac{1}{3}(\frac{13}{3\epsilon}-5)$, temos
%que se 
%$x\geq N$, então
%$|f(x)-\frac23|\leq \epsilon$. Isso pode ser repetido para qualquer
%$\epsilon>0$, e podemos ver que a medida que $\epsilon$ fica menor, o
%$N$ fica maior.
%%$$
%%\Bigl|\frac{2x-1}{3x+5}-\frac23\Bigr|\leq \epsilon\,.
%%$$
%Assim conseguimos provar que  
%$\lim_{x\to \infty}\frac{2x-1}{3x+5}=\frac23$.
%
%\end{ex}


\begin{exo}\label{exo_LIM_fasssiu} Calcule os limites abaixo, 
usando a Definição \ref{Def:limiteinfinito}:
\begin{enumerate}
\item\label{itexoformallim2} $\lim_{x\to
\pm\infty}\frac{x^2-1}{5x^2}$.
\item\label{itexoformallim4} $\lim_{x\to\infty}f(x)$, em que 
$f(x)=
\begin{cases}
-\cos x&\text{ se }x< 0\,,\\
1&\text{ se }x\geq 0\,.
\end{cases}$
\item\label{itexoformallim3} $\lim_{x\to \infty}\frac{1}{x^3+\sen^2
x}$.
\end{enumerate}
\begin{sol}
\eqref{itexoformallim2}
Vamos mostrar que o limite é $\tfrac15$.
Calculemos então
$\bigl|\frac{x^2-1}{5x^2}-\tfrac15\bigr|=\frac{1}{5x^2}$.
Seja $\epsilon>0$. Para ter $\tfrac{1}{5x^2}\leq \epsilon$, podemos tomar 
$x\geq N$, onde $N=\frac{1}{\sqrt{5\epsilon}}$.
Logo, como isso pode ser feito com qualquer $\epsilon>0$, isso mostra que
$\lim_{x\to \pm\infty}\frac{x^2-1}{5x^2}=\tfrac15$.
\eqref{itexoformallim4}
Como a função é \emph{constante e igual a $1$} nos positivos, temos
$\lim_{x\to\infty}f(x)=1$. Observe aqui que para qualquer tolerância $\epsilon>0$,
podemos sempre tomar o mesmo $N$, por exemplo $N=0$. De fato, para todo $x\geq 0$,
$|f(x)-1|=0\leq \epsilon$, qualquer que seja a tolerância. 
Esse exemplo mostra que uma função pode coincidir com a sua assíntota.
\eqref{itexoformallim3} 
Como a função é a divisão de $1$ por um número grande, o limite deve ser zero.
De fato, seja $\epsilon>0$. Precisamos mostrar que 
\[ \Bigl|\frac{1}{x^3+\sen^2 x}\Bigr|\leq \epsilon
\]
para todo $x$ suficientemente grande. Mas como não dá para resolver essa desigualdade
(isto é: isolar o $x$), podemos começar observando que
$\bigl|\frac{1}{x^3+\sen^2 x}\bigr|\leq	\frac{1}{x^3}$, e procurar
resolver $\frac{1}{x^3}\leq \epsilon$.
Vemos que se $x\geq N\pardef \epsilon^{-1/3}$, então essa desigualdade será
verificada, e $\bigl|\frac{1}{x^3+\sen^2 x}\bigr|\leq \epsilon$.
Isso mostra que $\lim_{x\to \infty}\frac{1}{x^3+\sen^2 x}=0$.
\end{sol}
\end{exo}


Mencionemos algumas propriedades básicas que decorrem da Definição
\index{limite! propriedades}
\ref{Def:limiteinfinito}:

\begin{pro}\label{Prop:SomasProdLimInfinito}
Suponha que duas funções, $f$ e $g$,
possuam limites quando $x\to\infty$: 
$$\lim_{x\to\infty}f(x)=\ell_1\,,\quad
\lim_{x\to\infty}g(x)=\ell_2\,,$$
onde $\ell_1$ e $\ell_2$ são ambos \emph{finitos}. Então 
\begin{gather}
\lim_{x\to\infty}\{f(x)+g(x)\}=
\lim_{x\to\infty}f(x)+\lim_{x\to\infty}g(x)=\ell_1+\ell_2\,,
\label{eq:proprliminfty1} \\
\lim_{x\to\infty}f(x)g(x)=
\bigl(\lim_{x\to\infty}f(x)\bigr)\cdot
\bigl(\lim_{x\to\infty}g(x)\bigr)=\ell_1\cdot
\ell_2\,.\label{eq:proprliminfty2}
\end{gather}
Além disso, se $\ell_2\neq 0$, então
\eq{
\lim_{x\to\infty}\frac{f(x)}{g(x)}=
\frac{\lim_{x\to\infty}f(x)}{\lim_{x\to\infty}g(x)}=\frac{\ell_1}{
\ell_2}\,.\label{eq:proprliminfty3}
}
As mesmas propriedades valem no caso $x\to-\infty$.
\end{pro}

\begin{proof} Provaremos somente \eqref{eq:proprliminfty1}. 
Seja $\epsilon>0$. Definamos $\epsilon_1\pardef \epsilon/2$.
Por definição, $\lim_{x\to\infty}f(x)=\ell_1$ implica 
que existe $N_1$ tal que se $x>N_1$ então 
$|f(x)-\ell_1|\leq \epsilon_1$.
Por outro lado, se $\epsilon_2\pardef \epsilon/2$, então 
$\lim_{x\to\infty}g(x)=\ell_2$ implica,
por definição, que existe $N_2$ tal que se $x>N_2$ então 
$|g(x)-\ell_2|\leq \epsilon_2$.
Logo, se $x$ é maior que $N_1$ e $N_2$ ao mesmo tempo, temos
\begin{align*}
\bigl|(f(x)+g(x))-(\ell_1+\ell_2)\bigr|&=
\bigl|(f(x)-\ell_1)+(g(x)-\ell_2)\bigr|\\
&\leq |f(x)-\ell_1|+|g(x)-\ell_2|\leq
\epsilon_1+\epsilon_2=\epsilon\,.
\end{align*}
\end{proof}


A identidade \eqref{eq:proprliminfty2} implica em particular que se 
$\lambda$ é uma constante (isto é, um número que não depende de $x$), 
então 
\eq{\label{eq:proprliminfty2cte}\lim_{x\to \infty}(\lambda 
f(x))=\lambda\lim_{x\to \infty} f(x)\,.}

A maior parte do tempo não precisaremos passar pelo uso de
tolerâncias para calcular limites. Em vez disso, 
usaremos as propriedades acima, e alguns
limites conhecidos, para calcular outros limites mais complicados.
Por exemplo, tendo feito o Exercício \ref{exo_LIM_fasssiu}, podemos
calcular o
seguinte limite, usando somente as propriedades básicas da proposição, sem
passar pela escolha de tolerâncias arbitrariamente pequenas, etc.:
\begin{align*}
 \lim_{x\to\infty}\frac{2x^2-2}{5x^2(x^3+\sen^2x)}
 &=
 \lim_{x\to\infty}2\cdot\frac{x^2-1}{5x^2}\cdot\frac{1}{x^3+\sen^2x}\\
 &=
2\cdot\Bigl\{\lim_{x\to\infty}\frac{x^2-1}{5x^2}\Bigr\}
\cdot\Bigl\{\lim_{x\to\infty}
\frac{1}{x^3+\sen^2x}\Bigr\}\\
&=2\cdot \tfrac15\cdot 0\\
&=0\,.
\end{align*}

\subsection{Limites infinitos}
\index{limite!infinito}
Em geral, uma função qualquer $f(x)$ não precisa possuir limites no infinito.
Isto é, $f(x)$ pode não se aproximar de nenhum valor finito
quando $x$ toma valores grandes. Por exemplo, já mencionamos que as funções
trigonométricas, por serem periódicas, não possuem limites quando
$x\to \pm\infty$.\\

Mas já sabemos que várias funções
não-limitadas, como $x^2$, tomam valores arbitrariamente grandes ao
$x$ se afastar da origem. Neste caso, o limite não existe no sentido
de \emph{ser finito}. No entanto, gostaríamos de poder escrever:
$$
\lim_{x\to \infty}x^2=+\infty\,.
$$
Aqui não se trata de usar tolerâncias, mas de definir precisamente o
que significa \emph{ultrapassar qualquer valor finito a medida que
$x$ cresce}. 
Por exemplos, $f(x)=x^2$ ultrapassa o valor $100$, a partir de $x=10$ em
diante, isto é para todos 
os $x\geq 10$. Mas ela também ultrapassa o valor $10'000$, para
todos os $x\geq 100$, etc.

\begin{center}
\begin{bmlimage}\begin{tikzpicture}
\draw[->] (-0.5,0)--(2.5,0);
\draw[->] (0,-0.5)--(0,4.2);
\draw [domain=0:2, samples=100] plot (\x,{\x^2});
\draw[dotted] (0.5,0) node[below]{$10$}--(0.5,0.25);
\draw[dotted] (0,0.25) node[left]{$100$}--(2,0.25);
\draw [->, very thick, domain=0.5:2, samples=100] plot (\x,{\x^2});

\begin{scope}[xshift=6cm]
\draw[->] (-0.5,0)--(2.5,0);
\draw[->] (0,-0.5)--(0,4.2);
\draw [domain=0:2, samples=100] plot (\x,{\x^2});
\draw[dotted] (1.5,0) node[below]{$100$}--(1.5,2.25);
\draw[dotted] (0,2.25) node[left]{$10'000$}--(2,2.25);
\draw [->, very thick, domain=1.5:2, samples=100] plot (\x,{\x^2});
\end{scope}
\end{tikzpicture}\end{bmlimage}
\end{center}

\begin{defin}
Diz-se que \grasA{$f(x)$ tende a $+\infty$ quando $x\to\infty$} se
para qualquer $A>0$ (subentendido: arbitrariamente grande, fixo) 
existe um $N$ tal que $f(x)\geq A$ para todo
$x\geq N$. 
Diz-se que \grasA{$f(x)$ tende a $-\infty$ quando $x\to\infty$} se
para qualquer $A<0$ existe um $N$ tal que $f(x)\leq A$ para todo
$x\geq N$. 
(Limites infinitos no caso $x\to-\infty$ se definem de maneira
parecida, trocando ``$x\geq N$'' por ``$x\leq -N$''.)
\end{defin}

Vejamos primeiro alguns exemplos de funções fundamentais que tem
limites infinitos.

Começaremos com potências inteiras, $x^p$, $p>0$,
\eq{\label{eq:infiniUM}
\lim_{x\to\infty}x^p=+\infty\,,\quad
\lim_{x\to-\infty}x^p=
\begin{cases}
 +\infty&\text{ se $p$ é par,}\\
 -\infty&\text{ se $p$ é ímpar.}
\end{cases}
}

\begin{ex}
Calculemos o limite
\[ 
\lim_{x\to\infty}\{x^2+\sen(10 x)\}\,.
\]
Sabemos que $x^2$ tende a $+\infty$, mas que o $\sen (\cdot)$ não tem limite. 
No entanto, o $\sen (10 x)$ é limitado por $1$ em valor absoluto.
Logo,
parece que a soma acima deve também tender a
$+\infty$. Para provar isso, fixemos um $A>0$ qualquer. Para mostrar
que $x^2+\sen (10x)\geq A$ para todos os $x$ suficientemente grandes,
comecemos observando que $\sen (10x)\geq -1$, o que permite escrever (veja a
figura abaixo):
\begin{equation}\label{eq_LIM_lemachindusink} 
x^2+\sen (10x)\geq x^2-1
\end{equation}
Mas, observe que $x^2-1\geq A$ quando $x\geq N$, onde
$N=\sqrt{A+1}$.
Agora, é claro que por \eqref{eq_LIM_lemachindusink}
temos também $x^2+\sen (10x)\geq A$ quando $x\geq N$.
Como o $A$ era arbitrário, isso mostra que
\[ 
\lim_{x\to\infty}\{x^2+\sen (10x)\}=+\infty\,.
\]
\begin{center}
\begin{bmlimage}\begin{tikzpicture}
\draw[->] (-0.5,0)--(10,0);
\draw[->] (0,-0.5)--(0,4.2);
%\draw[dotted] (0.5,0) node[below]{$10$}--(0.5,0.25);
%\draw[dotted] (0,0.25) node[left]{$100$}--(2,0.25);
\draw [very thick, domain=0:10.2, samples=500] plot
(\x,{0.04*\x^2+0.3*sin(10*\x r)}) node[right]{$x^2+\sen(10x)$};
\draw [color=gray, domain=0:10.2, samples=500] plot
(\x,{0.04*\x^2-0.3}) node[right]{$x^2-1$};
\end{tikzpicture}\end{bmlimage}
\end{center}
\end{ex}

Vimos que, dependendo da base, as exponenciais e os
logaritmos possuem comportamentos diferentes no infinito. Se a base
for $a>1$,
\begin{equation}\label{eq_Lim_expinf_a} 
\boxed{
\lim_{x\to\infty}a^x=+\infty\,,\quad \quad
\lim_{x\to-\infty}a^x=0\,.
}
\end{equation}
Em particular,
\begin{equation}\label{eq_Lim_expinf_b} 
\boxed{
\lim_{x\to\infty}e^x=+\infty\,,\quad \quad
\lim_{x\to-\infty}e^x=0\,.
}
\end{equation}
Por outro lado, se a base
for $a<1$,
\begin{equation}\label{eq_Lim_expinf_c} 
\boxed{
\lim_{x\to\infty}a^x=0\,,\quad \quad
\lim_{x\to-\infty}a^x=+\infty\,.
}
\end{equation}

Os logaritmos, por sua vez,
\eq{
\boxed{
\lim_{x\to\infty}\log_ax=
\begin{cases}
 +\infty&\text{ se $a>1$,}\\
-\infty&\text{ se $a<1$.}
\end{cases}
}
}
Observe que ``$\lim_{x\to-\infty}\log_ax$'' não faz sentido, já que
o domínio de $\log_a$ é $(0,\infty)$!

\begin{exo}
Mostre que se $\lim_{x\to\infty}f(x)=\pm\infty$, então
\[ 
\lim_{x\to\infty}\frac{1}{f(x)}=0\,.
\]
\begin{sol}
Seja $\epsilon>0$. Queremos mostrar que $|\frac{1}{f(x)}|\leq \epsilon$ para todo
$x$ suficientemente grande. Como $\lim_{x\to\infty}f(x)=\pm\infty$, sabemos que
se $A=\tfrac1\epsilon$, então existe $N$ tal que $f(x)\geq A$ para todo $x\geq
N$ (em particular, $f(x)>0$ para esses $x$). Mas isso implica também
$\frac{1}{f(x)}\leq \frac{1}{A}=\epsilon$, o que queríamos. 
\end{sol}
\end{exo}

A propriedade provada no último exercício permite obter o
comportamento no infinito para as potências negativas: 
$x^{-q}=\tfrac1{x^q}$, com $q>0$. Como
$\lim_{x\to\infty}x^q=+\infty$, 
temos 
\[ 
\lim_{x\to \infty}\frac{1}{x^q}=0\,.
\]
O limite $x\to -\infty$ se calcula da mesma maneira.

\begin{exo}\label{utilparaimproprias}
Mostre que 
\[
\lim_{L\to\infty}\frac{1}{L^{1-p}}=
\begin{cases}
0&\text{ se }p<1\,,\\
1&\text{ se }p=1\,,\\
\infty&\text{ se }p>1\,.
\end{cases}
\]
%\begin{sol}
%Se $p=1$ então $\frac{1}{L^{1-p}}=1$ para todo $L>0$. 
%Os casos $p<1$ e $p>1$ se deduzem de \eqref{eq:infiniUM} e
%\eqref{eq:infiniDOIS}.
%\end{sol}
\end{exo}

É importante notar que em geral, as propriedades descritas na
Proposição \ref{Prop:SomasProdLimInfinito} 
\emph{não se aplicam} quando os limites envolvidos
são infinitos. 
Aparece frequentemente de ter que lidar com quocientes
$\frac{f(x)}{g(x)}$ ou diferenças $f(x)-g(x)$, em que ambos
$f(x)\to\infty$ e $g(x)\to\infty$. Neste caso, as identidades da
Proposição \ref{Prop:SomasProdLimInfinito} não se aplicam, e um estudo 
caso a caso é preciso.

\subsubsection{Produtos de números grandes}
Na propriedade \eqref{eq:proprliminfty2},
insistimos sobre o fato dos dois limites
$\lim_{x\to\infty}f(x)$  e $\lim_{x\to\infty}g(x)$ existirem e serem
\emph{finitos} para poder escrever
\begin{equation}\label{eq_mmjj}
\lim_{x\to\infty}\{f(x)\cdot g(x)\}=
\bigl(\lim_{x\to\infty}f(x)\bigr)\cdot\bigl(\lim_{x\to\infty}g(x)\bigr)\,.
\end{equation}
É importante entender que existem casos em que essa relação \emph{não}
pode ser usada.\\

%\begin{ex} 
Considere  $f(x)=x$, $g(x)=\frac{1}{x}$. Neste caso, $f(x)g(x)=1$,
portanto o lado esquerdo de 
\eqref{eq_mmjj} é igual a
\[ 
\lim_{x\to\infty}\{f(x)\cdot g(x)\}=
\lim_{x\to\infty}1=1\,.
\]
Mas o lado direito  é igual a
\[ 
\bigl(\lim_{x\to\infty}f(x)\bigr)\cdot\bigl(\lim_{x\to\infty}g(x)\bigr)=\infty\cdot
0\,.
\]
Portanto, se \eqref{eq_mmjj} fosse verdadeira, teríamos 
\[``1=\infty\cdot 0"\,,\]
o que já mostra que há um problema: zero multiplicado por outra coisa
dificilmente pode dar $1$...
Mas, se agora $f(x)=2x$, $g(x)=\frac1x$, então $f(x)g(x)=2$, e 
o mesmo raciocíno leva a 
\[``2=\infty\cdot 0"\,.  \]
Ou, com $f(x)=x^2$ e $g(x)=\frac1x$,
\[``\infty=\infty\cdot 0"\,.\]
Sabemos que qualquer número multiplicado por zero dá zero, mesmo se o número for
grande:
\[ 
0\cdot 10=0\,,\quad\quad
0\cdot 100=0\,,\quad\quad
0\cdot 10000000=0\,,\quad\quad
\text{ etc.}
\]
Mas os exemplos acima mostram que há um problema com 
``$0\cdot\infty$'', e 
lembram que ``$\infty$'' não pode ser manuseado como os
outros números reais: em geral ``$0\cdot \infty$'' não vale zero, e
pode valer qualquer coisa. 
É por isso que será sempre escrito usando aspas.
A gente chama ``$\infty\cdot 0$'' (ou ``$0\cdot \infty$'') 
de \grasA{forma indeterminada}.
%\end{ex}

Em termos de limites, o exemplo acima 
mostra que não se pode aplicar \eqref{eq_mmjj}
quando um dos limites é infinito e o outro zero. No entanto, 

\begin{pro}\label{prop_LIM_produtouminfinito}
Se $\lim_{x\to\infty}f(x)=+\infty$ e $\lim_{x\to\infty}g(x)=\ell$, 
$\ell\neq 0$, então 
\[ 
\lim_{x\to\infty}\{f(x)\cdot g(x)\}=
\begin{cases}
+\infty&\text{ se }\ell>0\,,\\
-\infty&\text{ se }\ell<0\,.\\
\end{cases}
\]
\end{pro}

\begin{ex}
Por exemplo, 
já que $\lim_{x\to\infty}\frac{x}{x+2}=1>0$ e $\lim_{x\to\infty}e^x=+\infty$, 
\[ 
\lim_{x\to\infty}\frac{xe^x}{x+2}=
\lim_{x\to\infty}\frac{x}{x+2}\cdot e^x=+\infty\,.
\]
\end{ex}

\subsubsection{Quocientes de números grandes}

No Exemplo \ref{Ex:razaosimples} calculamos 
$\lim_{x\to \infty}\frac{x}{x+2}=1$. Observe que 
este limite é da forma
\[ 
\lim_{x\to \infty}\frac{f(x)}{g(x)}\,,
\]
em que $\lim_{x\to\infty}f(x)=\infty$ e 
$\lim_{x\to\infty}g(x)=\infty$.
Portanto, podemos dizer que é uma \grasA{forma indeterminada}
\[ ``\frac{\infty}{\infty}"\,.
\]
Em geral, ter uma \emph{indeterminação} (qualquer que seja) não significa
que o limite considerado não existe ou que ele não pode ser calculado, mas
que um estudo mais minucioso é necessário.
De fato, os exemplos a seguir 
são todos limites da forma ``$\frac{\infty}{\infty}$'',
mas todos podem ser calculados explicitamente e dar valores
diferentes:
\[ 
\lim_{x\to\infty}\frac{x}{x^2}=0\,,\quad
\lim_{x\to\infty}\frac{x^2}{x^2}=1\,,\quad
\lim_{x\to\infty}\frac{x^3}{x^2}=\infty\,,\,\text{ etc.}
\]
\begin{obs}
Na verdade as indeterminações da forma ``$\frac{\infty}{\infty}$'' são
equivalentes às indeterminações da forma ``$\infty\cdot 0$''. De fato, se 
$\lim_{x\to \infty}\frac{f(x)}{g(x)}$ é
``$\frac{\infty}{\infty}$'', podemos escrever \[ 
\lim_{x\to \infty}\frac{f(x)}{g(x)}=
\lim_{x\to \infty}\Bigl\{f(x)\cdot \frac{1}{g(x)}\Bigr\}\,.
\]
Como 
$\lim_{x\to\infty}f(x)=\infty$ e 
$\lim_{x\to\infty}\frac{1}{g(x)}=0$, o limite acima é também da forma
``$\infty\cdot 0$''.
\end{obs}

No próximos exemplos mostraremos várias técnicas que permitem resolver
indeterminações do tipo ``$\frac{\infty}{\infty}$''.
Começaremos com razões de polinômios, em que os polinômios têm o mesmo grau.
\begin{ex} Calcularemos um limite parecido com o 
do Exemplo \ref{Ex:razaomuitosimples}:
\[\lim_{x\to\infty}\frac{3x-5}{x+2}\,.\]
É fácil mostrar que esse limite é igual a $3$, mas paremos para pensar
de uma maneira diferente.  Na fração acima, quando $x$ é grande, 
o numerador $3x-5$ e o denominador $x+2$ são ambos grandes.
No entanto quando $x$ for grande, 
no numerador o ``$-5$'' se torna desprezível comparado com o ``3x'', e 
no denominador o ``$+2$'' 
se torna desprezível comparado com o ``$x$''.
Portanto, para $x$ grande, gostaríamos de pensar que
$$
\frac{3x-5}{x+2}\quad\text{ pode ser aproximado por } 
\quad\frac{3x}{x}\,, \quad \text{ que (após simplificação) é igual a }3\,.
$$
Esse argumento não é perfeitamente rigoroso, mas 
sugere que o limite é $3$. Para tornar ele 
mais rigoroso, colocamos $x$ em evidência no
denominador, e \emph{simplificamos por $x$}:
\begin{equation}\label{eq_simplicacao}
\frac{3x-5}{x+2}=\frac{x(3-\frac5x)}{x(1+\frac{2}{x})}
=\frac{3-\frac5x}{1+\frac{2}{x}}\,.
\end{equation}
Isso é só um outro jeito de reescrever a fração, mas agora observe que
quando $x\to \infty$, o limite desta última fração
não é mais da forma ``$\frac{\infty}{\infty}$''! 
Assim, usando \eqref{eq_simplicacao}, 
\eqref{eq:proprliminfty1} e \eqref{eq:proprliminfty3}:
$$
\lim_{x\to\infty}\frac{3x-5}{x+2}=
\lim_{x\to\infty}\frac{3-\frac5x}{1+\frac{2}{x}}=
\frac{\lim_{x\to\infty}(3-\frac5x)}{\lim_{x\to\infty}(1+\frac{2}{
x})}
=\frac{3-0}{1+ 0}=3\,.
$$
\end{ex}
Neste último exemplo aprendemos a extrair, em uma fração, as partes
mais importantes. Vejamos mais um exemplo.

\begin{ex}
Considere 
\[ 
\lim_{x\to\infty}\frac{x^3+1000x}{2x^3+1}\,.
\]
Vemos que tem dois termos de grau $3$, um termo de grau $1$ e um termo de
grau $0$ (aquele $+1$). O que importa, aqui, é que no limite
$x\to\infty$, os termos de grau $3$ vão ser os mais importantes.
De fato, quando $x$ for grande, $x^3$ sendo $x\cdot x\cdot x$, será muito maior
que $x$. Logo, vamos
\emph{extrair os termos de grau $3$ no numerador e
denominador}, simplificar, e 
usar \eqref{eq:proprliminfty1}-\eqref{eq:proprliminfty3}:
\[ 
\lim_{x\to\infty}\frac{x^3+1000x}{2x^3+1}=
\lim_{x\to\infty}\frac{x^3(1+\frac{1000}{x^2})}{x^3(2+\frac{1}{x^3})}=
\lim_{x\to\infty}\frac{1+\frac{1000}{x^2}}{2+\frac{1}{x^3}}=
\frac{\lim_{x\to\infty}(1+\frac{1000}{x^2})}{\lim_{x\to\infty}(2+\frac{1}{x^3})}=
\frac{1+0}{2+0}=\frac{1}{2}\,.
\]
\end{ex}

Extrair os termos de grau maior no numerador e
denominador pode ser feito em outras situações, com potências que não são
inteiras.
\begin{ex}
Considere
\[
\lim_{x\to\infty}\frac{x^5-x^2+1}{2x^5+3\sqrt{x}}\,.
\]
O limite é da forma ``$\frac{\infty}{\infty}$'', e a fração contem termos de
grau $5$, $2$, $\tfrac12$ e $0$.
Extraindo o termo de grau maior,
\[
\frac{x^5-x^2+1}{2x^5+3\sqrt{x}}
=\frac{x^5(1-\frac{1}{x}+\frac{1}{x^5})}{x^5(2+\frac{3}{x^{9/2}})}
=\frac{1-\frac{1}{x}+\frac{1}{x^5}}{2+\frac{3}{x^{9/2}}}\,,
\]
o limite do novo quociente \emph{não é mais indeterminado}. De fato, 
o novo numerador satisfaz
$\lim_{x\to\infty}(1-\frac{1}{x}+\frac{1}{x^5})=1-0+0=1$, e o
denominador 
$\lim_{x\to\infty}(2+\frac{3}{x^{9/2}})=2+0=2$, que é diferente de zero.
Logo, por \eqref{eq:proprliminfty3},
\[
\lim_{x\to\infty}\frac{x^5-x^2+1}{2x^5+3\sqrt{x}}
=\frac{\lim_{x\to\infty}(1-\frac{1}{x}+\frac{1}{x^5})}{\lim_{x\to\infty}(2+\frac{3}{x^{9/2}})}
=\frac{1}{2}\,.
\]
\end{ex}

Vejamos agora dois exemplos em que o denominador e o numerador tem graus
diferentes.

\begin{ex}
Considere o seguinte limite, da forma ``$\fracinfty$'':
\[ \lim_{x\to\infty}\frac{x+2x^2}{1+x^4}\,.
\]
Extraindo os termos de grau maior em cima e em baixo,
\[ 
\frac{x+2x^2}{1+x^4}=
\frac{x^2(2+\frac{1}{x})}{x^4(1+\frac{1}{x^4})}=
\frac{1}{x^2}\cdot \frac{2+\frac1{x}}{1+\frac{1}{x^4}}
\]
Logo, como $\lim_{x\to\infty}\frac{1}{x^2}=0$ e
$\lim_{x\to\infty}\frac{2+\frac1x}{1+\frac{1}{x^4}}=2$, 
\eqref{eq:proprliminfty2} implica
\[ 
 \lim_{x\to\infty}\frac{x+2x^2}{1+x^4}=0\cdot 2=0\,.
\]
\end{ex}

\begin{ex}
Estudemos agora 
\[\lim_{x\to \infty}\frac{x^2+2}{x+1}\,,\]
que representa também uma indeterminação do tipo ``$\fracinfty$''.
Mas, pondo os termos de grau $2$ em evidência,
$$\frac{x^2+2}{x+1}=\frac{x^2(1+\frac{2}{x^2})}{x(1+\frac{1}{x})}=
x\cdot \frac{1+\frac{2}{x^2}}{1+\frac{1}{x}}\,.$$
Observe agora que o primeiro fator, $x$, tende a $+\infty$, e que o 
segundo fator,
$\frac{1+\frac{2}{x^2}}{1+\frac{1}{x}}$, tende a $1$. Logo, 
pela Proposição \ref{prop_LIM_produtouminfinito},
$$\lim_{x\to\infty}\frac{x^2+2}{x+1}=
\lim_{x\to\infty}x\cdot \frac{1+\frac{2}{x^2}}{1+\frac{1}{x}}=
+\infty\,.$$
\end{ex}


\begin{exo}\label{Exo:limitesinfini} Calcule os limites abaixo,
evitando o uso da 
definição formal. Abaixo, $x\to \pm\infty$ significa que são dois limites
para calcular: $x\to+\infty$ e $x\to-\infty$.
\begin{multicols}{3}
\begin{enumerate}
\item\label{itexliminfini2} $\lim_{x\to
\pm\infty}\{\frac{1}{x}+\frac{1}{x^2}+\frac{1}{x^3}\}$
\item\label{itexliminfini3} $\lim_{x\to \pm\infty}\frac{x^2-1}{x^2}$
\item\label{itexliminfini6} $\lim_{x\to\pm\infty}\frac{1-x^2}{x^2-1}$
\item\label{itexliminfini7} $\lim_{x\to
\pm\infty}\frac{2x^3+x^2+1}{x^3+x}$
\item\label{itexliminfini8} $\lim_{x\to
\pm\infty}\frac{2x^3-2}{x^4+x}$
\item\label{itexliminfini9} $\lim_{x\to\infty}\frac{1+x^4}{x^2+4}$
\item\label{itexliminfini10} $\lim_{x\to
\pm\infty}\frac{\sqrt{x+1}}{\sqrt{x}}$
\item\label{itexliminfini15}
$\lim_{x\to\pm\infty}\frac{\sqrt{4x^2+1}}{x}$
\item\label{itexliminfini11}
$\lim_{x\to\infty}\frac{3x+2}{\sqrt{x^2+3}-4}$
\item\label{itexliminfini12} $\lim_{x\to\pm\infty}
\frac{\sqrt{x+\sqrt{x+\sqrt{x}}}}{\sqrt{x+1}}$
\item\label{itexliminfini13} $\lim_{x\to\pm\infty}\frac{|x|}{x^2+1}$
\item\label{itexliminfini14} $\lim_{x\to\pm\infty}\sqrt{x^2+1}$
\item\label{itexliminfini141} $\lim_{x\to\pm\infty}\frac{1}{2^x}$
\item\label{itexliminfini16}
$\lim_{x\to\pm\infty}\frac{e^x+100}{e^{-x}-1}$
\item\label{itexliminfini161} $\lim_{x\to \pm \infty}\ln(1+\frac{x+1}{x^2})$
\item\label{itexliminfini162} $\lim_{x\to \pm \infty}\frac{\ln(1+e^x)}{x}$
\item\label{itexliminfini5} $\lim_{x\to\pm \infty}e^{\frac{1}{x}}$
\item\label{itexliminfini17} $\lim_{x\to \pm\infty}\sen^2x$
\item\label{itexliminfini19} $\lim_{x\to \pm\infty}\arctan x$
\item\label{itexliminfini20} $\lim_{x\to \pm\infty}\senh x$
\item\label{itexliminfini21} $\lim_{x\to \pm\infty}\cosh x$
\item\label{itexliminfini22} $\lim_{x\to \pm\infty}\tanh x$
\end{enumerate}
\end{multicols}
\vspace{0.01cm}
\begin{sol}
\eqref{itexliminfini2} Como $\lim_{x\to\pm\infty}\frac{1}{x^q}=0$ para
qualquer $q>0$, usando \eqref{eq:proprliminfty1} dá 
$\lim_{x\to
\pm\infty}\{\frac{1}{x}+\frac{1}{x^2}+\frac{1}{x^3}\}=0$.
\eqref{itexliminfini3} $\lim_{x\to \pm\infty}\frac{x^2-1}{x^2}=1$
\eqref{itexliminfini6} $\lim_{x\to\pm\infty}\frac{1-x^2}{x^2-1}=-1$.
\eqref{itexliminfini7} Colocando $x^3$ em evidência e usando
\eqref{eq:proprliminfty3}, 
$$\lim_{x\to\pm\infty}\frac{2x^3+x^2+1}{x^3+x}=\lim_{x\to
\pm\infty}\frac{x^3(2+\frac{1}{x}+\frac{1}{x^3})}{x^3(1+\frac{1}{x^2})
} =\lim_{x\to
\pm\infty}\frac{2+\frac{1}{x}+\frac{1}{x^3}}{1+\frac{1}{x^2}}=\frac{2}
{1}=2\,.$$
\eqref{itexliminfini8} $\lim_{x\to
\pm\infty}\frac{2x^3-2}{x^4+x}=0$
\eqref{itexliminfini9} Colocando $x^4$ em
evidência no denominador, $x^2$ no numerador,
$\frac{1+x^4}{x^2+4}=x^2\frac{\frac{1}{x^4}+1}{1+\frac{4}{x^2}}$.
Como $x^2\to\infty$ e que a fração tende a $1$, temos 
$\lim_{x\to\pm\infty}\frac{1+x^4}{x^2+4}=\infty$.
\eqref{itexliminfini10} ``$\lim_{x\to
-\infty}\frac{\sqrt{x+1}}{\sqrt{x}}$'' não é definido.
Por outro lado, colocando $\sqrt{x}$ em evidência, 
$$\lim_{x\to+\infty}\frac{\sqrt{x+1}}{\sqrt{x}}=
\lim_{x\to +\infty}\frac{\sqrt{1+\frac{1}{{x}}}}{1}=1\,.
$$
\eqref{itexliminfini15}
Lembrando que $\sqrt{x^2}=|x|$ (Exercício
\ref{Exo:valorabscorreto}!), temos
$\frac{\sqrt{4x^2+1}}{x}=\frac{\sqrt{x^2(4+\frac{1}{x^2})}}{x}=\frac{
|x|}{x}\sqrt{4+\frac{1}{x^2}}$.
Como $\frac{|x|}{x}=+1$ se $x>0$, $=-1$ se $x<0$, temos $\lim_{x\to\pm\infty}\frac{|x|}{x}=\pm 1$. Como 
$\lim_{x\to\pm\infty}\sqrt{4+\frac{1}{x^2}}=\sqrt{4}=2$, temos
$\lim_{x\to\pm\infty}\frac{\sqrt{4x^2+1}}{x}=\pm 2$.
\eqref{itexliminfini11} 
Do mesmo jeito,
$\sqrt{x^2+3}=|x|\sqrt{1+\frac{3}{x^2}}$. Assim,
$$
\frac{3x+2}{\sqrt{x^2+3}-4}=\frac{x}{|x|}\frac{3+\frac{2}{x}}{\sqrt{1+
\frac{3}{x^2}}-\frac{4}{|x|}}
$$
Como $\lim_{x\to\pm\infty}\frac{x}{|x|}=\pm 1$, e que a razão tende a
$3$, temos 
$$\lim_{x\to+\infty}\frac{3x+2}{\sqrt{x^2+3}-4}=+3\,,\quad 
\lim_{x\to-\infty}\frac{3x+2}{\sqrt{x^2+3}-4}=-3\,.
$$
\eqref{itexliminfini12} O limite $x\to-\infty$
não é definido, e $\lim_{x\to+\infty}
\frac{\sqrt{x+\sqrt{x+\sqrt{x}}}}{\sqrt{x+1}}=1$.
\eqref{itexliminfini13} $\lim_{x\to\pm\infty}\frac{|x|}{x^2+1}=0$
\eqref{itexliminfini14} $\lim_{x\to\pm\infty}\sqrt{x^2+1}=+\infty$
\eqref{itexliminfini141} Como $\frac{1}{2^x}=2^{-x}$, temos $\lim_{x\to+\infty}\frac{1}{2^x}=0$, 
$\lim_{x\to-\infty}\frac{1}{2^x}=+\infty$.
\eqref{itexliminfini16}
$\lim_{x\to+\infty}\frac{e^x+100}{e^{-x}-1}=-\infty$,
$\lim_{x\to-\infty}\frac{e^x+100}{e^{-x}-1}=0$.
\eqref{itexliminfini161}  Primeiro mostre (usando os mesmos métodos do que os que foram
usados nos outros itens) que $\lim_{x\to \pm \infty}(1+\frac{x+1}{x^2})=1$. Em seguida,
observe que 
se $z$ se aproxima de $1$,  então $\ln(z)$ se aproxima de $\ln(1)=0$. Logo, $\lim_{x\to
\pm \infty}\ln(1+\frac{x+1}{x^2})=0$. Obs: dizer que ``se $z$ se aproxima de $1$, então
$\ln(z)$ se aproxima de $\ln(1)$'' presupõe que a função $\ln$ é \emph{contínua} em $1$.
Continuidade será estudada no fim do capítulo.
\eqref{itexliminfini162}  Escreve $(1+e^x)=e^x(1+e^{-x})$, logo
$\frac{\ln(1+e^x)}{x}=\frac{\ln
e^x}{x}+\frac{\ln(1+e^{-x})}{x}=1+\frac{\ln(1+e^{-x})}{x}$. Mas $\lim_{x\to
\infty}\frac{\ln(1+e^{-x})}{x}=0$, logo
$\lim_{x\to \infty}\frac{\ln(1+e^{x})}{x}=1$.
Por outro lado, $\ln(1+e^x)\to 0$  quando $x\to-\infty$, logo $\lim_{x\to
-\infty}\frac{\ln(1+e^{x})}{x}=0$.
\eqref{itexliminfini5} Como $\lim_{x\to\pm \infty}\frac{1}{x}=0$
temos, $\lim_{x\to\pm \infty}e^{\frac{1}{x}}=e^0=1$.
\eqref{itexliminfini17} $``\lim_{x\to \pm\infty}\sen^2x''$ não existe.
\eqref{itexliminfini19} $\lim_{x\to \pm\infty}\arctan
x=\pm\pisobredois$.
\eqref{itexliminfini20} 
Por definição, $\senh x=\frac{e^x-e^{-x}}{2}$. Para estudar
$x\to\infty$, coloquemos $e^x$ em evidência:
$\frac{e^x-e^{-x}}{2}=e^x\frac{1-e^{-2x}}{2}$. Como $e^x\to\infty$ e
$1-e^{-2x}\to 1$ temos $\lim_{x\to \infty}\senh x=+\infty$. Como
$\senh x$ é ímpar, temos $\lim_{x\to -\infty}\senh x=-\infty$.
\eqref{itexliminfini21} $\lim_{x\to \pm\infty}\cosh x=+\infty$
\eqref{itexliminfini22} Para estudar, $x\to\infty$:
$\tanh x=\frac{e^x-e^{-x}}{e^x+e^{-x}}=\frac{e^x}{e^x}\frac{1-e^{-2x}}{1+e^{
-2x }}=\frac{1-e^{-2x}}{1+e^{
-2x }}$, logo $\lim_{x\to +\infty}\tanh x=+1$. Como $\tanh$ é ímpar,
$\lim_{x\to -\infty}\tanh x=-1$.
\end{sol}
\end{exo}



\begin{exo}
% http://limite.cours-de-math.eu/solution-probleme2.php
Um tempo $t$ depois de ter pulado do avião, a velocidade
vertical de um paraquedista em queda livre é dada por:
$$V(t)=\sqrt{\frac{m g}{k}}\tanh\Bigl(\sqrt{\frac{gk}{m}}t\Bigr)\,,$$
onde $m$ é a massa do paraquedista, $g=9,81m/s^2$, e $k$ é um
coeficiente de resistência (atrito) do ar (em $kg/m$). Esboce
$t\mapsto V(t)$,
e calcule o limite de velocidade $V_{\rm lim}$ (que ele nunca
atingirá).
Dê uma estimativa de $V_{\rm lim}$ quando $m=80kg$, $k=0.1kg/m$.
\begin{sol}
Pelo gráfico de $x\mapsto \tanh x$, vemos que $V(t)$ cresce e tende a
um valor limite, dado por 
$$
V_{\rm lim}=\lim_{t\to\infty}V(t)=\sqrt{\frac{m
g}{k}}\lim_{t\to\infty}\tanh\Bigl(\sqrt{\frac{gk}{m}}t\Bigr)
$$
Vimos no Exercício \ref{Exo:limitesinfini} que
$\lim_{x\to\infty}\tanh x=1$. Portanto, 
$$V_{\rm lim}=\sqrt{\frac{m
g}{k}}\,.$$
Observe que $V(t)<V_{\rm lim}$ para todo $t$, então o paraquedista
nunca atinge a velocidade limite, mesmo se ele cair um tempo infinito!
Com os valores propostos,
$V_{\rm lim}=\sqrt{80\cdot 9,81/0.1}\simeq 89m/s\simeq 318km/h$.
\end{sol}
\end{exo}

\subsubsection{Somas e diferenças de números grandes}

Na propriedade \eqref{eq:proprliminfty1},
insistimos sobre o fato dos dois limites
$\lim_{x\to\infty}f(x)$  e $\lim_{x\to\infty}g(x)$ serem
\emph{finitos} para poder escrever
$$\lim_{x\to\infty}\{f(x)+g(x)\}=
\lim_{x\to\infty}f(x)+\lim_{x\to\infty}g(x)\,.$$

Quando os dois limites são infinitos, com o mesmo sinal, então o
limite da soma pode também ser calculado:

\begin{ex}
Considere $x+x^3$. Como $\lim_{x\to\infty}x=+\infty$ e
$\lim_{x\to\infty}x^3=+\infty$ (aqui, ambos tem o sinal ``$+$''), temos 
$\lim_{x\to\infty}\{x+x^3\}=+\infty$.
\end{ex}

Agora, para estudar $\lim_{x\to\infty}\{f(x)-g(x)\}$, com
$\lim_{x\to\infty}f(x)=\infty$ e
$\lim_{x\to\infty}g(x)=\infty$, 
leva a um caso de \grasA{indeterminação do
tipo ``$\infty-\infty$''}. Vejamos exemplos que ilustram que de
fato, ``$\infty-\infty$'' pode tomar qualquer valor.

\begin{ex}
Considere $x^3-x^2$, em que $\lim_{x\to\infty}x^3=+\infty$ e
$\lim_{x\to\infty}x^2=+\infty$. Como o termo de grau
maior deve ser mais importante, escrevamos
$x^3-x^2=x^3(1-\frac{1}{x})$. Como $x^3\to\infty$ e
$1-\frac{1}{x}\to 1$, a Proposição \ref{prop_LIM_produtouminfinito} 
garante que 
\[\lim_{x\to\infty}\{x^3-x^2\}=+\infty\,.\]
O que aconteceu aqui se resume assim: $x^3$ e $x^2$ ambos tendem a
$+\infty$, mas $x^3$ \emph{cresce mais rápido} que $x^2$, e isso
implica que a diferença $x^3-x^2$ é regida (quando $x$ é grande)
pelo termo $x^3$.
\end{ex}

\begin{ex}
A diferença $x^2-x^4$ no limite $x\to\infty$ pode ser estudada da
mesma maneira: $x^2-x^4=x^4(\frac{1}{x^2}-1)$, e como $x^4\to\infty$,
$(\frac{1}{x^2}-1)\to-1$, temos que $x^2-x^4\to -\infty$.
Aqui, é o termo $-x^4$ que rege o comportamento para $x$ grande.
\end{ex}


\begin{ex}\label{Ex:conjugadobasico}
Considere $\sqrt{x+1}-\sqrt{x}$. Quando $x\to\infty$, os dois termos
$\sqrt{x+1}$ e $\sqrt{x}$ tendem a $+\infty$, mas eles são do mesmo grau. 
\index{indeterminação!do tipo ``$\infty-\infty$''}
Como calcular o limite dessa diferênça? O método usado aqui consiste em
\emph{multiplicar e
dividir pelo \index{conjugado}conjugado}, isto é, escrever ``$1$'' como
$$
1=\frac{\sqrt{x+1}+\sqrt{x}}{
\sqrt{x+1}+\sqrt{x}}\,.
$$
Lembrando que $(a-b)(a+b)=a^2-b^2$,
$$
\sqrt{x+1}-\sqrt{x}=(\sqrt{x+1}-\sqrt{x})\frac{\sqrt{x+1}+\sqrt{x}}{
\sqrt{x+1}+\sqrt{x}}=\frac{\sqrt{x+1}^2-\sqrt{x}^2}{\sqrt{x+1}+\sqrt{
x}}=\frac{1}{\sqrt{x+1}+\sqrt{x}}\,.
$$
Mas como $\sqrt{x+1}+\sqrt{x}\to\infty$, temos
$$\lim_{x\to\infty}\{\sqrt{x+1}-\sqrt{x}\}=\lim_{x\to\infty}\frac{1}{
\sqrt{x+1}+\sqrt{x}}=0\,.$$ 
\end{ex}

\begin{exo}
Calcule os limites quando $x\to\infty$ das seguintes funções.
\begin{multicols}{3}
\begin{enumerate}
\item\label{itexliminfini1} $7-x$
\item\label{itexliminfini4} $\sqrt{1-x}$
\item\label{itexliminfini18} $x+\cos x$
\item\label{itexoinfinf1} $100x-x^2$
\item\label{itexoinfinf2} $x^7-x^7$
\item\label{itexoinfinf3} $x^4-\tfrac12 x^4$
\item\label{itexoinfinf311} $(x-1)^2-x^2$
\item\label{itexoinfinf31} $x-\sqrt{x}$
\item\label{itexoinfinf5} $\sqrt{x^2+1}-\sqrt{x^2-x}$
\item\label{itexoinfinf51} $\sqrt{x^2+1}-\sqrt{x^2-3x}$
\item\label{itexoinfinf4} $\sqrt{2x}-\sqrt{x+1}$
\item\label{itexoinfinf6} $e^x-e^{2x}$
\item\label{itexoinfinf7} $\ln(x)-\ln(2x)$
\item\label{itexoinfinf8} $\ln(x)-\ln(x+1)$
\end{enumerate}
\end{multicols}
\vspace{0.01cm}
\begin{sol}
\eqref{itexliminfini1} $\lim_{x\to\infty}(7-x)=-\infty$,
$\lim_{x\to-\infty}(7-x)=+\infty$.
\eqref{itexliminfini4} ``$\lim_{x\to +\infty}\sqrt{1-x}$'' não é
definida, pois o domínio de $\sqrt{1-x}$ é $(-\infty,1]$. 
$\lim_{x\to -\infty}\sqrt{1-x}=+\infty$.
\eqref{itexliminfini18} Como $\lim_{x\to\pm\infty}
x=\pm\infty$, e que $\cos x$ é limitado por $-1\leq \cos x\leq 1$,
temos $\lim_{x\to \pm\infty}x+\cos x=\pm\infty$.
\eqref{itexoinfinf1} $-\infty$.
\eqref{itexoinfinf2} $0$.
\eqref{itexoinfinf3} $+\infty$.
\eqref{itexoinfinf311} $-\infty$
\eqref{itexoinfinf31} $+\infty$
\eqref{itexoinfinf5} $\frac12$.
Esse ítem (e o próximo) mostram que argumentos informais do tipo
``$x^2+1\simeq x^2$
quando $x$ é grande'' não sempre são eficazes! De fato, aqui daria
$\sqrt{x^2+1}-\sqrt{x^2-x}\simeq \sqrt{x^2}-\sqrt{x^2}=0$...
\eqref{itexoinfinf51} $\frac32$. 
\eqref{itexoinfinf4} Aqui não precisa multiplicar pelo conjugado: pode
simplesmente colocar $\sqrt{x}$ em evidência:
$\sqrt{2x}-\sqrt{x+1}=\sqrt{x}(\sqrt{2}-\sqrt{1+\frac1x})$. Como
$\sqrt{x}\to\infty$ e $\sqrt{2}-\sqrt{1+\frac1x}\to \sqrt{2}-1>0$,
temos $\sqrt{x}(\sqrt{2}-\sqrt{1+\frac1x})\to +\infty$.
\eqref{itexoinfinf6} $-\infty$ (Obs: pode observar que $e^x-e^{2x}=z-z^2$, em que $z=e^x$. Como $z\to \infty$ 
quando $x\to\infty$, temos $z-z^2\to \infty$, como no item \eqref{itexoinfinf1}.)
\eqref{itexoinfinf7} Como $\ln x-\ln(2x)=-\ln 2$, o limite é $-\ln 2$.
\eqref{itexoinfinf8} $\lim_{x\to \infty}\{\ln x-\ln(x+1)\}=
\lim_{x\to \infty}\ln (\frac{x}{x+1})=\ln 1=0$.
\end{sol}
\end{exo}

\subsubsection{O ``sanduiche''}
\index{``sanduíche''}
\begin{ex}\label{ex_LIM_sinxsurx}
Considere o limite
$\lim_{x\to \infty}\frac{\sen x}{x}$.
Sabemos que o denominador tende a $+\infty$, mas
$\sen x$ não possui limite quando $x\to\infty$.
Apesar de tudo, sabemos que $\sen x$ é uma função \emph{limitada}:
\index{função!limitada}
para todo $x$, $-1\leq \sen x\leq +1$. Portanto, quando $x>0$, 
$$
-\frac{1}{x}\leq \frac{\sen x}{x}\leq +\frac{1}{x}\,.
$$
Mas como a cota superior $+\frac1x$ tende a zero, e que a cota
inferior $-\frac1x$ também tende a zero, a função $\frac{\sen x}{x}$ 
também deve tender a zero:
\begin{center}
\begin{bmlimage}\begin{tikzpicture}
\pgfmathsetmacro{\a}{2}
\draw [color=gray!80, domain=0.7:6, samples=100] plot
(\x,{1/(\x^1)});
\draw [color=gray!80, domain=0.7:6, samples=100] plot
(\x,{-1/(\x^1)});
\draw [ ->] (-0.1,0)--(6.5,0);
\draw [ ->] (0,-1.4)--(0,1.4);
\draw [thick, domain=0.7:6, samples=100] plot
(\x,{sin(3*\x r)/\x});
\pgfmathsetmacro{\b}{2};
\draw[color=gray!90,<-] (\b,{1/\b+0.1})--(\b+0.7,{1/\b+0.7})
node[right]{$+\tfrac{1}{x}$};
\draw[color=gray!90,<-] (\b,{-1/\b-0.1})--(\b+0.7,{-1/\b-0.7})
node[right]{$-\tfrac{1}{x}$};
\draw[<-] (3.1,{0.1})--(3.7,0.8)
node[right]{$\tfrac{\sen x}{x}$};
\draw (11,0.5) node[left]{$\displaystyle{\Rightarrow \lim_{x\to
\infty}\frac{\sen x}{x}=0}$};
\end{tikzpicture}\end{bmlimage}
\end{center}
\end{ex}

Esse método vale em geral:

\begin{teo}\label{Teo:Sanduicheinfinito}
 Suponha que $f$, $g$ e $h$ seja três funções que satisfazem
$$
g(x)\leq f(x)\leq h(x)\,,\text{ para todo $x$ suficientemente grande.}
$$
Suponha também que
$\lim_{x\to\infty}g(x)=\lim_{x\to\infty}h(x)=\ell$. Então 
$\lim_{x\to\infty}f(x)=\ell$.
\end{teo}



\begin{exo} Calcule:
\begin{multicols}{3}
\begin{enumerate}
\item\label{itexosanduiche2}
$\lim_{x\to\infty}\frac{1+\cos(x^2+3x)}{x^2}$
\item\label{itexosanduiche1} $\lim_{x\to\infty}\frac{x+\sen x}{x-\cos
x}$
\item\label{itexosanduiche3} $\lim_{x\to\infty}e^{-x}\sen x$
\item\label{itexosanduiche4}
$\lim_{x\to\infty}\frac{x-\lfloor x\rfloor}{x}$
\item\label{itexosanduiche5}
$\lim_{x\to \infty}\frac{\arctan(\sen x)}{\ln x}$
\item\label{itexosanduiche6}
$\lim_{x\to\infty}\bigl\{1+\frac{\sen x}{x^2+4}\bigr\}$
\end{enumerate}
\end{multicols}
\vspace{0.01cm}
\begin{sol}
\eqref{itexosanduiche2} Para todo $x$, $-1\leq
\cos(x^2+3x)\leq +1$, logo
$0\leq \frac{1+\cos(x^2+3x)}{x^2}\leq \frac{2}{x^2}$.
Como $\frac{2}{x^2}$ tende a zero,
$\lim_{x\to\infty}\frac{1+\cos(x^2+3x)}{x^2}=0$.
\eqref{itexosanduiche1} Como $\frac{x+\sen x}{x-\cos
x}=\frac{1+\frac{\sen x}{x}}{1-\frac{\cos x}{x}}$, e como
$\lim_{x\to\infty}\frac{\sen x}{x}=0$, $\lim_{x\to\infty}\frac{\cos
x}{x}=0$ (mesmo método), temos que $\lim_{x\to\infty}\frac{x+\sen
x}{x-\cos x}=1$.
\eqref{itexosanduiche3} 
Como $-e^{-x}\leq e^{-x}\sen x\leq e^{-x}$ e 
$\lim_{x\to\infty}-e^{-x}=\lim_{x\to\infty}e^{-x}=0$, o limite
procurado vale $0$.
\eqref{itexosanduiche4} Como $0\leq x-\lfloor x\rfloor\leq 1$, temos
$\lim_{x\to\infty}\frac{x-\lfloor x\rfloor}{x}=0$.
\eqref{itexosanduiche5} Como
$-\frac{\pi/2}{\ln x}\leq \frac{\arctan(\sen x)}{\ln x}\leq
\frac{\pi/2}{\ln x}$, e $\lim_{x\to\infty}\frac{1}{\ln x}=0$, o
limite procurado é $0$.
\eqref{itexosanduiche6} Para todo $x$,
$-\frac{1}{x^2+4}\leq \frac{\sen x}{x^2+4}\leq
\frac{1}{x^2+4}$. Como 
$\lim_{x\to \infty}\frac{1}{x^2+4}=0$, o limite procurado vale
$1$.
\end{sol}
\end{exo}


\begin{obs}
Alguns limites no infinito, tais como
$\lim_{x\to \infty}\frac{e^x}{x}$  ou  $\lim_{x\to \infty} \frac{\ln x}{x}$,
não podem ser calculados com os métodos desenvolvidos até agora; serão estudados mais tarde.
\end{obs}

% \subsubsection{Assíntotas oblíquas}
% 
% Quando uma função possui uma assíntota, é uma especie de
% simplificação do comportamento: é essencialmente uma reta...

\section{Limites laterais: $\lim_{x\to a^\pm}f(x)$}\label{Sec:Lim}
\index{limite!lateral}
Na seção anterior estudamos o comportamento de uma função $f$
longe da origem, $x\to\infty$ ou $x\to-\infty$. 
Agora observaremos
o comportamento de uma função $f(x)$ a medida que $x$ 
se aproxima de um ponto da reta, que denotaremos por $a\in \bR$.
\\

Como um $x$ pode estar ou à esquerda de $a$ ($x<a$), ou à direita
de $a$ ($x>a$), começaremos com dois tipos de limites, chamados de
\grasA{laterais}: escreveremos $x\to a^+$ (ou $x\searrow a$) para
indicar que $x$ se aproxima de $a$ pela direita, e 
$x\to a^-$ (ou $x\nearrow a$) para indicar que $x$ se aproxima de $a$ pela esquerda.

Comecemos com um exemplo bem simples.
\begin{ex}\label{ExemploLimitesimples}
Considere a função \[f(x)=\frac{x}{2}+1\,,\] 
em uma vizinhança do ponto $a=1$. Olhemos primeiro os
valores de $f(x)$ quando $x\searrow 1$, isto é, quando $x$ toma valores maiores mas perto
de $1$:
\begin{center}
\begin{tabular}{c|c|c|c|c}
$x=$&$1.5$&$1.1$&$1.01$&$1,0001$\\
\hline
$f(x)= $&$1.75$&$1.55$&$1.505$&$1,50005$
\end{tabular}
\end{center}
Vemos que
estes valores decrescem, se aproximando de $1.5=\tfrac32$. Gostaríamos de escrever
\[\lim_{x\searrow 1}f(x)=\tfrac32\,.\]
Ao olharmos
os valores de $f(x)$ quando $x\nearrow 1$, isto é, quando $x$ toma valores menores mas
perto de $1$, 
vemos que estes crescem para o mesmo valor $\tfrac32$: 
\begin{center}
\begin{tabular}{c|c|c|c|c}
$x=$&$0.5$&$0.9$&$0.99$&$0.9999$\\
\hline
$f(x)= $&$1.25$&$1.45$&$1.495$&$1,49995$
\end{tabular}
\end{center}
Gostariamos então de escrever
\[\lim_{x\nearrow 1}f(x)=\tfrac32\,.\]
Essas propriedades se tornam óbvias olhando para o gráfico, que é uma simples reta:
\begin{center}
\begin{bmlimage}\begin{tikzpicture}[scale=1.3]
\draw [ ->] (0,-0.1)--(0,2.3) node[right]{$f(x)$};
\draw [ ->] (-0.3,0)--(2.5,0)  node[right]{$x$};
\draw [thick, domain=-0.2:2.2] plot (\x,{1+\x/2});
\pgfmathsetmacro{\e}{0.2};
\pgfmathsetmacro{\x}{1.6};
\draw[dotted] (\x,0) --(\x,{1+\x/2})--(0,{1+\x/2});
\draw[thick, <-] (\x-\e,0)--(\x,0);
\draw[thick, <-] (0,{1+\x/2-\e})--(0,{1+\x/2});
\pgfmathsetmacro{\x}{0.4};
\draw[dotted] (\x,0) --(\x,{1+\x/2})--(0,{1+\x/2});
\draw[thick, ->] (\x,0)--(\x+\e,0);
\draw[thick, ->] (0,{1+\x/2})--(0,{1+\x/2+\e});
\pgfmathsetmacro{\x}{1};
\draw[densely dashed] (\x,0) node[below]{$\scriptstyle{1}$}--(\x,{1+\x/2})--(0,{1+\x/2})
node[left]{$\scriptstyle{3/2}$};
\end{tikzpicture}\end{bmlimage}
\end{center}
Para entender um pouco melhor o que está acontecendo, vamos 
estudar a diferença:
\[|f(x)-\tfrac32|=\bigl|(\tfrac{x}{2}+1)-\tfrac32\bigr|=\tfrac12|x-1|\,.\]
Assim, vemos que quando $x$ fica perto de $1$, isto é quando a distância 
$|x-1|$ é
pequena, então a diferença $|f(x)-\tfrac32|$ é pequena também.
Poderíamos ser um pouco mais precisos, fixar uma \grasA{tolerância} $\epsilon>0$
(subentendido pequena) e perguntar:
\emph{quão próximo de $1$ $x$ precisa estar para garantir que }
\[|f(x)-\tfrac32|\leq
\epsilon\,\quad ?\]
Como $|f(x)-\tfrac32|=\tfrac12|x-1|$, vemos que $x$ precisa satisfazer
$\tfrac12|x-1|\leq \epsilon$, isto é
\[|x-1|\leq 2\epsilon\,.  \]
Logo, a resposta à pergunta acima é: \emph{a distância menor que $2\epsilon$.}
\end{ex}

Na verdade, 
pode parecer óbvio que a medida que $x$ se aproxima de $1$, a função
$f(x)=\frac{x}{2}+1$ se aproxima de $f(1)=\tfrac{1}{2}+1=\tfrac32$. Isto
é: colocando o valor $x=1$ na função, a
gente já sabe qual será o valor limite. 
Mas isso funciona porque a função do exemplo é simples o suficiente. Às 
vezes, teremos que trabalhar mais, como no próximo exemplo.

%É importante mencionar 
%que os limites estudados no exemplo anterior \emph{não dependem do valor da função no ponto $a=1$!}
%De fato, se $g$ for uma modificação de $f$ em $1$, por exemplo
%$$
%g(x)\pardef
%\begin{cases}
% \frac{x}{2}+1 &\text{ se }x\neq 1,\\
%0&\text{ se }x=1\,,
%\end{cases}
%$$
%então os limites laterais de $g$ em $x=1$ são os \emph{mesmos} que os de $f$, isto é:
%$\lim_{x\searrow 1}g(x)=\lim_{x\nearrow 1}g(x)=\tfrac32$.

% Vimos que a função $f(x)=\tfrac{x}{2}+1$ \emph{tende a} $\tfrac32$, isto é, que
% $|f(x)-\tfrac32|$ tende a zero, 
% quando $x$ tende a 1. 

\begin{ex}\label{Ex:primeironaotrivial} Consideremos agora 
\[f(x)=\frac{x^3-1}{x-1}\,,\]
também na vizinhança de $a=1$.
Observe que essa função \emph{não é definida em $1$}. Logo, 
para saber o que acontece quando $x$ tende a $1$, não
temos como adivinhar qual será o limite trocando simplesmente $x$ por $1$.

Mas isso não significa que ele não pode ser calculado. Calculemos alguns valores 
de $f(x)$, com $x\searrow 1$,
\begin{center}
\begin{tabular}{c|c|c|c|c}
$x=$&1.1&1.02&1.002&1.0002\\
\hline
$f(x)\simeq $&$3,310$&$3,060$&$3.006$&$3,001$
\end{tabular}
\end{center}
e quando $x\nearrow 1$:
\begin{center}
\begin{tabular}{c|c|c|c|c}
$x=$&0,9&0,99&0.999&0.9999\\
\hline
$f(x)\simeq $&$2,710$&$2,970$&$2,997$&$2,999$
\end{tabular}
\end{center}
Esses números sugerem que
\[\lim_{x\to 1^+}f(x)=\lim_{x\to 1^-}f(x)=3\,.\] 
Não falaremos de tolerância aqui, mas podemos 
fazer uma conta simples que mostra porque que o limite é $3$.
De fato, o polinômio $x^3-1$ possui a raiz $x=1$, sabemos então 
que ele pode ser fatorado da seguinte maneira:
\[ x^3-1=(x-1)(\dots)\,.
\]
O fator $(\dots)$ pode ser calculado pela divisão de $x^3-1$ por $x-1$,
que dá:
\[
\begin{array}{r|l}
x^3 \phantom{+x^2+x}-1 & x-1 \\ \cline{2-2}
x^3-x^2\phantom{+x+1} & x^2+x+1 \\ \cline{1-1} \\
x^2\phantom{+x}-1 \\
   x^2-x\phantom{-1} \\ \cline{1-1} \\
                       x - 1\\
		       x-1 \\\cline{1-1}\\
		       0
  \end{array}
\]
Isto mostra que o nosso quociente na verdade pode ser escrito como
$$\frac{x^3-1}{x-1}=x^2+x+1\,.$$
Agora fica claro que se $x$ tende a $1$, não importa de qual lado, 
\eq{\label{eq:limsombesta}
\lim_{x\to 1^\pm}\frac{x^3-1}{x-1}=
\lim_{x\to 1^\pm}(x^2+x+1)= 1^2+1+1=3\,.} 
\end{ex}

\begin{obs}
No exemplo anterior, a função $\frac{x^3-1}{x-1}$ 
não é definida \emph{no} ponto $a=1$, mas 
em qualquer outro ponto da sua vizinhança, e à medida que $x$ se aproxima de $a=1$, o
numerador e o denominador \emph{ambos} tendem a $0$. 
Foi o nosso primeiro exemplo de resolução de uma \grasA{indeterminação do tipo}
\index{indeterminação!do tipo ``$\frac00$''}
\[``\frac{0}{0}"\,.\]
\end{obs}

\begin{exo}\label{Exo_derivpotpasencore}
Calcule $\lim_{x\to 1^{\pm}}\frac{x^4-1}{x-1}$,
$\lim_{x\to 1^{\pm}}\frac{x^5-1}{x-1}$, ...
\begin{sol}
 A divisão dá $\frac{x^4-1}{x-1}=x^3+x^2+x+1$. Logo, como cada termo tende a $1$,
$\lim_{x\to 1^{\pm}}\frac{x^4-1}{x-1}=4$.
 No caso geral, $\frac{x^n-1}{x-1}=x^{n-1}+\dots+x+1$. Como são $n$ termos e que cada um
tende a $1$, temos $\lim_{x\to 1^{\pm}}\frac{x^n-1}{x-1}=n$.
\end{sol}
\end{exo}

Eis agora a definição geral de limite lateral:

\begin{defin} Seja $a\in \bR$. 
\begin{enumerate}
\item  Diz-se que \grasA{$f(x)$ tende a $\ell$ quando $x$ tende a $a$ pela direita} se
para todo $\epsilon>0$ existe um $\delta>0$
tal que se $a<x\leq a+\delta$, então $|f(x)-\ell|\leq \epsilon$.
Escreve-se $\lim_{x\to a^+}f(x)=\ell$.
\item  Diz-se que \grasA{$f(x)$ tende a $\ell$ quando $x$ tende a $a$ pela esquerda} se
para todo $\epsilon>0$ existe um $\delta>0$
tal que se $a-\delta\leq x< a$, então $|f(x)-\ell|\leq \epsilon$.
Escreve-se $\lim_{x\to a^-}f(x)=\ell$.
\end{enumerate}
\end{defin}

\begin{ex}
Usando a definição, mostremos que 
\[\lim_{x\to 1}x^2=1\,.\]
Observe primeiro que $|x^2-1|=|x+1|\cdot |x-1|$. 
Quando $x>1$ fica perto de $1$, digamos a distância menor que $\tfrac12$, temos
$|x-1|=x-1$, e $|x+1|=x+1\leq \tfrac52$. 
Quando $x$ tende a $1$, $|x-1|$ tende a
$0$. 
Seja agora $\epsilon>0$.
Para garantir que
$|x^2-1|\leq \epsilon$, podemos escrever primeiro $|x^2-1|\leq \tfrac52(x-1)$, e procurar
primeiro 
resolver $\tfrac52(x-1)\leq \epsilon$, que dá $x\leq 1+\tfrac25\epsilon$. Assim,
mostramos que se $1<x\leq 1+\delta$, com  
$\delta\pardef \frac{2\epsilon}{5}$, 
 teremos
$|x^2-1|=|x+1|\cdot |x-1|\leq \tfrac32(x-1)\leq \tfrac32\delta=\epsilon$.
\end{ex}

Foi usado implicitamente em \eqref{eq:limsombesta} que se cada termo de uma soma possui
limite, então a soma possui limite também, e este vale a soma dos limites; segue do
seguinte resultado, que é o análogo da Proposição \ref{Prop:SomasProdLimInfinito}:

\begin{pro}\label{Prop:ProprLimitesparaa}
\index{limite!propriedades}
Suponha que duas funções, $f$ e $g$,
possuam limites quando $x\to a^+$: 
$$\lim_{x\to a^+}f(x)=\ell_1\,,\quad
\lim_{x\to a^+}g(x)=\ell_2\,,$$
onde $\ell_1$ e $\ell_2$ são ambos \emph{finitos}. Então 
\begin{gather}
\lim_{x\to a^+}\{f(x)+g(x)\}=
\lim_{x\to a^+}f(x)+\lim_{x\to a^+}g(x)=\ell_1+\ell_2\,,
\label{eq:proprlima1} \\
\lim_{x\to a^+}f(x)g(x)=
\bigl(\lim_{x\to a^+}f(x)\bigr)\cdot
\bigl(\lim_{x\to a^+}g(x)\bigr)=\ell_1\cdot
\ell_2\,.\label{eq:proprlima2}
\end{gather}
Além disso, se $\ell_2\neq 0$, então
\eq{\lim_{x\to a^+}\frac{f(x)}{g(x)}=
\frac{\lim_{x\to a^+}f(x)}{\lim_{x\to a^+}g(x)}=\frac{\ell_1}{
\ell_2}\,.\label{eq:proprlima3}
}
As mesmas propriedades valem no caso $x\to a^-$.
\end{pro}

Nos exemplos anteriores, os limites laterais $x\to a^+$ e $x\to a^-$ eram iguais.
Vejamos um exemplo onde eles são diferentes.

\begin{ex}\label{Ex:funcaodescontinua}
Considere $f(x)=\tfrac{x}{3}+\frac{x}{2|x|}$ na vizinhança de $a=0$ (em que ela
nem é definida).
Usando a definição de $|x|$, podemos reescrever $f$ da seguinte maneira:
\[ f(x)=
\begin{cases}
\tfrac{x}{3}+\tfrac12&\text{ se }x>0\,,\\
\tfrac{x}{3}-\tfrac12&\text{ se }x<0\,.\\
\end{cases}
\]
Logo,
$$
\lim_{x\to 0^+}f(x)=
\lim_{x\to 0^+}\bigl\{ \tfrac{x}{3}+\tfrac12 \bigr\}=
+\tfrac12\,,\quad \quad 
\text{ e }
\quad
\quad
\lim_{x\to 0^-}f(x)=
\lim_{x\to 0^-}\bigl\{ \tfrac{x}{3}-\tfrac12 \bigr\}=
-\tfrac12\,.
$$
 Isso significa que o gráfico de $f(x)$, ao $x$ crescer de $<0$ para $>0$ e atravessar
$0$, dá um \emph{pulo} de valores pertos de $-\tfrac12$ para valores perto de $+\tfrac12$.
Diz-se que essa função é \emph{descontínua} em $x=0$:\index{descontinuidade}
\begin{center}
\begin{bmlimage}\begin{tikzpicture}[scale=1.5]
\draw [ ->] (0,-1.1)--(0,1.1) node[left]{$\scriptstyle{f(x)}$};
\draw [ ->] (-1.5,0)--(1.5,0)  node[right]{$\scriptstyle{x}$};
\draw [thick, domain=0:1] plot (\x,{0.5+\x/3});
\draw [thick, domain=-1:0] plot (\x,{-0.5+\x/3});
\pgfmathsetmacro{\e}{0.2};
\pgfmathsetmacro{\x}{0.8};
\draw[dotted] (\x,0) --(\x,{0.5+\x/3})--(0,{0.5+\x/3});
\draw[thick, <-] (\x-\e,0)--(\x,0);
\pgfmathsetmacro{\x}{-0.8};
\draw[dotted] (\x,0) --(\x,{-0.5+\x/3})--(0,{-0.5+\x/3});
\draw[thick, ->] (\x,0)--(\x+\e,0);
\filldraw[intaberto] (0,0.5) circle (0.45mm);
\filldraw[intaberto] (0,-0.5) circle (0.45mm);
\end{tikzpicture}\end{bmlimage}
\end{center}
\end{ex}

\begin{exo}
Seja
$$f(x)\pardef
\begin{cases}
5-x&\text{ se }x\geq 2\\
\frac{x}{2}&\text{ se }x< 2\,.
\end{cases}$$
Calcule os limites laterais $\lim_{x\to a^{\pm}}f(x)$ para $a=0$, $a=2$, $a=5$.
\begin{sol}
$\lim_{x\to 0^+}f(x)=\lim_{x\to 0^+}\frac{x}{2}=0$,
$\lim_{x\to 0^-}f(x)=\lim_{x\to 0^-}\frac{x}{2}=0$.
$\lim_{x\to 2^+}f(x)=\lim_{x\to 2^+}5-x=3$.
$\lim_{x\to 2^-}f(x)=\lim_{x\to 2^-}\frac{x}{2}=1$, logo $f$ é descontínua em
$x=2$.
$\lim_{x\to 5^+}f(x)=\lim_{x\to 5^+}5-x=0$,
$\lim_{x\to 5^-}f(x)=\lim_{x\to 5^-}5-x=0$.
\end{sol}
\end{exo}

Às vezes, limites laterais não existem: 
\begin{ex}
Por exemplo, o limite lateral $x\to 0^+$ da 
função $\sen \tfrac1x$ (que obviamente não é definida em $x=0$) 
não existe:
\begin{center}
\begin{bmlimage}\begin{tikzpicture}
\draw[->] (-0.3,0)--(6,0) node[right]{$x$};
\draw[->] (0,-1.1)--(0,1.25) node[left]{$\sen \tfrac1x$};
\pgfmathsetmacro{\e}{0.1};
\draw [dotted, domain=\e/4:\e, samples=1000] plot (\x,{sin(4/\x r)});
\draw [thick, domain=\e:0.4, samples=1000] plot (\x,{sin(4/\x r)});
\draw [thick, domain=0.4:6.5, samples=500] plot (\x,{sin(4/\x r)});
\end{tikzpicture}\end{bmlimage}
\end{center}
Observe, no entanto, que $\lim_{x\to \infty}\sen \tfrac1x=0$.
\end{ex}

% \begin{exo} Considere $f(x)=\sen \tfrac1x$, com $D=]0,\frac{1}{\pi}]$. 
% Fixe qualquer $y\in [-1,1]$ e calcule as suas preimagens. 
% \begin{sol}
% Para resolver $\sen \tfrac1x=y$
% \end{sol}
% \end{exo}

\begin{exo}\label{Ex:semoulediadique}
Considere a função definida por 
$$f(x)=
\begin{cases}
 +1&\text{ se $x$ é racional diádico}\,,\\
0&\text{ caso contrário}.
\end{cases}
$$
Estude os limites laterais de $f(x)$ num ponto qualquer $a$.
\begin{sol}
Escolha um ponto $a\in \bR$ qualquer. 
Como os racionais diádicos \index{racionais diádicos}
são densos em $\bR$, existem infinitos diádicos $x_D>a$, 
 arbitrariamente próximos de $a$, tais que $f(x_D)=1$. Mas existem também infinitos
irracionais $x_I>a$ arbitrariamente próximos de $a$ tais que $f(x_I)=0$. Portanto, $f(x)$
não pode tender a um valor quando $x\to a^+$. O mesmo raciocínio vale para $x\to a^-$.
Logo, a função $f$ não possui limites laterais em nenhum ponto da reta.
\end{sol}
\end{exo}

\begin{exo}
Seja $f(x)\pardef \lfloor x\rfloor$.
Calcule 
$\lim_{x\to \half^+}f(x)$, $\lim_{x\to \half^-}f(x)$, 
$\lim_{x\to \frac{1}{3}^+}f(x)$, $\lim_{x\to \frac{1}{3}^-}f(x)$. 
Calcule
$\lim_{x\to 1^+}f(x)$, $\lim_{x\to 1^-}f(x)$. Calcule, para qualquer número
inteiro $n$, $\lim_{x\to n^+}f(x)$, $\lim_{x\to n^-}f(x)$.
\begin{sol}
$\lim_{x\to \half^+}f(x)=\lim_{x\to \half^-}f(x)=0$, 
$\lim_{x\to \frac{1}{3}^+}f(x)=\lim_{x\to \frac{1}{3}^-}f(x)=0$. 
$\lim_{x\to 1^+}f(x)=1$, $\lim_{x\to 1^-}f(x)=0$. Para 
$n\in \bZ$, $\lim_{x\to n^+}f(x)=n$, $\lim_{x\to n^-}f(x)=n-1$.
(Pode verificar essas afirmações também no seu esboço do
Exercício \ref{ExoEsbocosElementares}!)
\end{sol}
\end{exo}

\section{Limites $\lim_{x\to a}f(x)$}
\index{limite!bilateral}

\begin{defin}
Se uma função $f$ possui limites laterais iguais em $a\in \bR$, isto é, se $\lim_{x\to
a^+}f(x)=\lim_{x\to a^-}f(x)=\ell$, então 
diremos que \grasA{$f(x)$ tende a $\ell$ quando $x$ tende a $a$}, e escreveremos
simplesmente\index{limite}
$$\lim_{x\to a}f(x)=\ell\,.$$
\end{defin}

 Observe que nesse caso, $f(x)$ tende a $\ell$ à medida que $x$ tende a $a$,
\emph{qualquer que seja o lado}: para todo $\epsilon>0$, existe 
$\delta>0$ tal que se
$|x-a|\leq \delta$, $x\neq a$, então $|f(x)-\ell|\leq \epsilon$. O 
limite $\lim_{x\to
a}f(x)$ será às
vezes chamado de \grasA{bilateral}.\\

Por definição, o limite bilateral satisfaz às mesmas propriedades que 
aquelas para os
limites laterais descritas na Proposição \ref{Prop:ProprLimitesparaa}.

\begin{exo}\label{Exo:Limiteselementares}
Estude os limites abaixo. (Em particular, comece verificando 
se o tipo de limite considerado é compatível com o domínio da função.)
\begin{multicols}{3}
\begin{enumerate}
\item\label{itlimbasic1} $\lim_{x\to 7}(7-x)$
\item\label{itlimbasic2} $\lim_{x\to 0^+}\sqrt{x}$
\item\label{itlimbasic3} $\lim_{x\to 0}\cos x$
\item\label{itlimbasic4} $\lim_{x\to 3}\frac{x^2-1}{x^2+1}$
\item\label{itlimbasic5} $\lim_{x\to 4}\frac{x-4}{x-4}$
\item\label{itlimbasic6} $\lim_{x\to 4}\frac{|x-4|}{x-4}$
\item\label{itlimbasic61} $\lim_{x\to -5}\frac{x-5}{|x-5|}$
\item\label{itlimbasic7} $\lim_{x\to 1}\frac{1-x}{x^2-1}$
\item\label{itexolimelem20} $\lim_{x\to 1}\sqrt{\ln x}$
\item\label{itexolimelem201} $\lim_{x\to -2}\frac{2-x}{\sqrt{x-2}}$
\end{enumerate}
\end{multicols}
\vspace{0.01cm}
\begin{sol}
\eqref{itlimbasic1} $0$
\eqref{itlimbasic2} $0$ (O limite é bem definido, no seguinte
sentido: como $\sqrt{x}$ é definida para $x>0$, o limite
somente pode ser do tipo $x\to 0^+$.)
\eqref{itlimbasic3} $1$
\eqref{itlimbasic4} $\frac45$
\eqref{itlimbasic5} $1$
\eqref{itlimbasic6} Sabemos que $\frac{|x-4|}{x-4}=+1$ se $x>4$, e
$=-1$ se $x<4$. Logo, $\lim_{x\to 4^+}\frac{|x-4|}{x-4}=+1$,
$\lim_{x\to 4^-}\frac{|x-4|}{x-4}=-1$, mas $\lim_{x\to
4}\frac{|x-4|}{x-4}$ não existe.
\eqref{itlimbasic61} $-1$
\eqref{itlimbasic7} $-\frac12$
\eqref{itexolimelem20} Como $\ln x$ muda de sinal em $1$, é preciso
que $x$ tenda a $1$ pela direita para $\sqrt{\ln x}$ ser bem definida,
e escrever esse limite como $\lim_{x\to 1^+}\sqrt{\ln x}=0$.
$\lim_{x\to 1^-}\sqrt{\ln x}$ não é definido.
\eqref{itexolimelem201} Não definido pois $\sqrt{x-2}$ não é definido perto de $x=-2$.
\end{sol}
\end{exo}

 Vejamos agora o análogo do Teorema \ref{Teo:Sanduicheinfinito} para limites laterais e
bilaterais.
\index{``sanduíche''}
\begin{teo}\label{Teo:Sanduichefinito}
 Suponha que $f$, $g$ e $h$ sejam três funções que satisfazem
$$g(x)\leq f(x)\leq h(x)\,,\text{ para todo $x$ numa vizinhança de $a$}\,.
$$
Suponha também que
$\lim_{x\to a^+}g(x)=\lim_{x\to a^+}h(x)=\ell$. Então 
$\lim_{x\to a^+}f(x)=\ell$.
(O mesmo resultado vale trocando todos os $x\to a^+$ por $x\to a^-$ ou por $x\to a$.)
\end{teo}

\begin{ex}\label{Ex:sanduicheseno}
O limite $\lim_{x\to 0}x^2\sen \tfrac1x$ pode ser calculado, observando que
 $-1 \leq \sen \tfrac1x\leq +1$ para todo $x\neq 0$. Logo, multiplicando por $x^2$ (que é
$>0$),
$$-x^2\leq x^2\sen \tfrac1x\leq x^2\,.$$
 Quando $x\to 0$, $-x^2$ e $x^2$ ambos tendem a zero. 
\begin{center}
\begin{bmlimage}\begin{tikzpicture}
\pgfmathsetmacro{\a}{2}
\draw [color=gray!80, domain=0:4, samples=100] plot
(\x,{0.1*\x^2}) node[right]{$+x^2$};
\draw [color=gray!80, domain=0:4, samples=100] plot
(\x,{-0.1*\x^2}) node[right]{$-x^2$};
\draw [ ->] (-0.1,0)--(6.5,0);
\draw [ ->] (0,-1.4)--(0,1.4);
\draw [thick, domain=0.1:3.85, samples=200] plot
(\x,{0.1*\x^2*sin(10/\x r)}) node[right]{$x^2\sen\tfrac1x$};
\pgfmathsetmacro{\b}{2};
%\draw[color=gray!90,<-] (\b,{1/\b+0.1})--(\b+0.7,{1/\b+0.7})
%node[right]{$+\tfrac{1}{x}$};
%\draw[color=gray!90,<-] (\b,{-1/\b-0.1})--(\b+0.7,{-1/\b-0.7})
%node[right]{$-\tfrac{1}{x}$};
%\draw[<-] (3.1,{0.1})--(3.7,0.8)
%node[right]{$\tfrac{\sen x}{x}$};
%\draw (11,0.5) node[left]{$\displaystyle{\Rightarrow \lim_{x\to
%\infty}\frac{\sen x}{x}=0}$};
\end{tikzpicture}\end{bmlimage}
\end{center}
 Pelo Teorema
\ref{Teo:Sanduichefinito},
concluimos que $\lim_{x\to 0}x^2\sen \tfrac1x=0$.
\end{ex}

\begin{exo}\label{Ex:semoulediadiquebis}
Determine se o limite ${x\to 0}$ da função existe. Se for o caso, dê o
seu valor.
\index{racionais diádicos}
$$
f(x)=
\begin{cases}
x^2&\text{ se $x$ é racional diádico}\,,\\
0&\text{ caso contrário}\,,
\end{cases}
\quad\quad 
g(x)=
\begin{cases}
\frac{1+x}{1+x^2}&\text{ se }x<0\,,\\
-1&\text{ se }x=0\,,\\
\sen(\frac{\pi}{2}+x)&\text{ se }x>0\,.
\end{cases}
$$
\begin{sol}
No primeiro caso, podemos comparar $0\leq f(x)\leq x^2$ para todo $x$.
Logo, 
pelo Teorema \ref{Teo:Sanduichefinito},
$\lim_{x\to 0}f(x)$ existe e vale $0$.
No segundo caso, 
$\lim_{x\to 0^-}g(x)=\lim_{x\to 0^-}\frac{1+x}{1+x^2}=1$, e
$\lim_{x\to 0^+}g(x)=\lim_{x\to 0^+}\sen(\frac{\pi}{2}+x)=\sen
\pisobredois=1$. Logo, $\lim_{x\to 0}g(x)$ existe e vale $1$.
\end{sol}

\end{exo}

\section{Indeterminações do tipo ``$\tfrac00$''}
\index{indeterminação!do tipo ``$\frac00$''}
Na seção anterior encontramos, quando $x\to \infty$ ou $x\to -\infty$, indeterminações do
tipo ``$\infty-\infty$'', ``$\frac{\infty}{\infty}$''.
Já encontramos (ver o Exemplo \ref{Ex:primeironaotrivial}, e alguns 
dos limites do Exercício \ref{Exo:Limiteselementares}) 
limites de quocientes, em que numerador e denominador ambos 
tendem a zero. Tais quocientes não podem ser estudados 
usando \eqref{eq:proprlima3}, e representam a uma {indeterminação
do tipo ``$\tfrac{0}{0}$''}.

Será visto no próximo capítulo que a \emph{derivada}, que 
fornece informações úteis a respeito de uma função, é \emph{por definição} 
um limite que leva a uma indeterminação do tipo ``$\tfrac{0}{0}$''. Por isso, indeterminações ``$\frac00$''
serão os limites mais estudados a partir de agora.
Nos próximos exemplos veremos algumas técnicas para lidar com essas
indeterminações.

\begin{ex}\label{Ex:derivxissdoisemum}
$\lim_{h\to 0}\frac{(1+h)^2-1}{h}$ é do tipo ``$\frac00$'', já 
que $(1+h)^2-1\to 0$ quando $h\to 0$.
Mas o limite pode ser calculado facilmente, observando 
que $(1+h)^2-1=2h+h^2$:
$$
\lim_{h\to 0}\frac{(1+h)^2-1}{h}=\lim_{h\to 0}
\frac{2h+h^2}{h}=\lim_{h\to 0}2+h=2\,.
$$
\end{ex}

\begin{ex}
Considere $\lim_{x \to 2}\frac{x^2+x-6}{x^2-9x+14}$.
Observe que aqui, $\lim_{x \to 2}(x^2+x-6)=0$ e 
$\lim_{x \to 2}(x^2-9x+14)=0$, logo o limite é do tipo ``$\tfrac00$''.
Mas o polinômio $x^2+x-6$ tender a zero quando $x\to 2$, significa que 
ele se anula em $x=2$. Portanto, ele pode ser fatorado, com
um fator $(x-2)$:
$x^2+x-6=(x-2)(x+3)$. Do mesmo jeito,
$x^2-9x+14=(x-2)(x-7)$. Portanto,
$$
\lim_{x \to 2}\frac{x^2+x-6}{x^2-9x+14}=\lim_{x \to 2}\frac{(x-2)(x+3)}{(x-2)(x-7)}
=\lim_{x \to 2}\frac{x+3}{x-7}=\frac{5}{-5}=-1\,.
$$
O que foi feito aqui, com a fatoração e simplificação 
por $(x-2)$, foi de \emph{extrair} a origem comum da 
anulação do
numerador e denominador em $x=2$.
\end{ex}

\begin{ex}
O método da multiplicação e divisão pelo conjugado, 
vista no Exemplo \ref{Ex:conjugadobasico}, serve também 
para estudar
alguns limites do tipo ``$\frac00$''.
Por exemplo,
\begin{align*}\lim_{h\to 0}\frac{\sqrt{1+h}-1}{h}
&=\lim_{h\to 0}
\frac{\sqrt{1+h}-1}{h}\cdot\frac{\sqrt{1+h}+1}{\sqrt{1+h}+1}\\
&=\lim_{h\to 0}
\frac{\sqrt{1+h}^2-1^2}{h(\sqrt{1+h}+1)}\\
&=\lim_{
h\to 0}\frac{1}{\sqrt{1+h}+1}\\
&=\frac{1}{2}\,.
\end{align*}
\end{ex}

\begin{exo}  Calcule os limites
\begin{multicols}{3}
\begin{enumerate}
\item\label{itlimzerozero1} $\lim_{x\to 2}\frac{(x-2)(4-x^2)}{x^2-4x+4}$
\item\label{itlimzerozero2} $\li{t}{9}\frac{9-t}{3-\sqrt{t}}$
\item\label{itlimzerozero24} $\li{x}{4}\frac{\sqrt{x}-3}{x-2}$
\item\label{itlimzerozero4} $\lim_{t\to 0}\frac{\sqrt{a^2+bt}-a}{t}$
\item\label{itlimzerozero44} $\lim_{t\to \infty}\frac{\sqrt{a^2+bt}-a}{t}$
\item\label{itlimzerozero3} $\li{x}{2}\frac{\sqrt{6-x}-2}{\sqrt{3-x}-1}$
\end{enumerate}
\end{multicols}
\vspace{0.01cm}
\begin{sol}
\eqref{itlimzerozero1} $-4$.
\eqref{itlimzerozero2} $6$.
\eqref{itlimzerozero24} $-\tfrac12$.
\eqref{itlimzerozero4} $\frac{b}{2a}$. 
\eqref{itlimzerozero44} $0$. 
\eqref{itlimzerozero3} $\tfrac12$. 
\end{sol}
\end{exo}



\begin{exo}
Existe um n\'umero $a$ tal que 
$$
\lim_{x\to -2}\frac{3x^2+ax+a+3}{x^2+x-2}
$$
exista e seja finito? Caso afirmativo, encontre $a$ e o valor do limite.
\begin{sol}
Observe que quando $x\to -2$, o denominador tende a $0$. 
Para o limite existir, a única possibilidade é do numerador também
tender a zero quando $x\to -2$. Mas como $3x^2+ax+a+3$ tende a $15-a$
quando $x\to -2$, $a$ precisa satisfazer $15-a=0$, isto é: $a=15$.
Neste caso (e somente neste caso), o limite existe e vale 
$$
\lim_{x\to -2}\frac{3x^2+15x+18}{x^2+x-2}
\lim_{x\to -2}\frac{(3x+9)(x+2)}{(x-1)(x+2)}=
\lim_{x\to -2}\frac{3x+9}{x-1}=-1\,.
$$
\end{sol}
\end{exo}

%\begin{exo}
%Quais expressões abaixo representam uma indeterminação?
%$$
%(+\infty)\cdot(-\infty)\,,\quad -\infty+\infty\,,\quad 0\cdot \infty\,,\quad
%\frac{0}{\infty}\,,\quad\frac{\infty}{-\infty}\,,\quad
%\frac{+\infty}{0}\,,\quad 0-\infty
%$$
%\begin{sol}
%``$-\infty+\infty$'', ``$0\cdot \infty$'', ``$\frac{\infty}{-\infty}$'' e 
%``$\frac{+\infty}{0}$''
%são indeterminações.
%\end{sol}
%\end{exo}

\subsection{O limite $\lim_{x\to 0}\tfrac{\sen x}{x}$}
\index{limite!$\lim_{x\to 0}\frac{\sen x}{x}$}
Aqui provaremos o limite mais fundamental para funções trigonométricas:
\eq{\label{eq_limsenxsurx}\boxed{\lim_{x\to 0}\frac{\sen x}{x}=1\,.}}
É importante mencionar que $x$ é medido em \emph{radianos}.
Consideremos primeiro $\tfrac{\sen x}{x}$ no limite lateral $x\to 0^+$.

Considere um ângulo $0<x<\pisobredois$ no círculo trigonométrico:
\begin{center}
\begin{bmlimage}\begin{tikzpicture}
\pgfmathsetmacro{\r}{3};
\pgfmathsetmacro{\t}{33}
\coordinate (B) at ({\r*cos(\t)},{\r*sin(\t)});
\coordinate (Bp) at ({\r*cos(\t)},0);
\coordinate (C) at (\r,{(\r*sin(\t))/cos(\t)});
\coordinate (Cp) at (\r,0);
\draw[dotted] (B)--(Bp);
\draw[dotted] (C)--(Cp);
\draw (0,0)--(C);
\draw (0,0)--(\r+0.2,0);
\draw[thick] (\r,0) arc (0:\t:\r);
\fill (B) circle (0.45mm);
\fill (Bp) circle (0.45mm);
\fill (C) circle (0.45mm);
\fill (Cp) circle (0.45mm);
\draw (B) node[above]{$B$};
\draw (Bp) node[below]{$B'$};
\draw (C) node[above]{$C$};
\draw (Cp) node[below]{$C'$};
\fill (0,0) circle (0.45mm);
\draw (0,0) node[above]{$O$};
\pgfmathsetmacro{\b}{0.8};
\draw[->] (\b,0) arc (0:\t:\b);  
\draw ({\t/2}:{1.3*\b}) node{$x$};
\end{tikzpicture}\end{bmlimage}
\end{center}

Temos $|OC'|=|OB|=1$, $|B'B|=\sen x$, $|OB'|=\cos x$ e $|C'C|=\tan x$.
Observe que 
\[ 
\text{área do triângulo }OB'B\leq
\text{área do setor }OC'B\leq
\text{área do triângulo }OC'C.
\]
A área $\sigma$ do setor $OC'B$ se
calculada observando que por proporcionalidade:
$\frac{x}{2\pi}=\frac{\sigma}{\pi\cdot 1^2}$. Logo,
$\sigma=\frac{x}{2}$. Assim, reescrevendo as três desigualdades acima em termos
de $x$,
$$\tfrac12 \sen x\cos x\leq \tfrac12 x\leq \tfrac12 \tan x\,.$$
A primeira desigualdade implica $\sen x\cos x\leq x$, isto é,
$\frac{\sen x}{x}\leq \frac{1}{\cos x}$. A segunda implica 
$x\leq \tan x=\frac{\sen x}{\cos x}$,
isto é, $\cos x\leq \frac{\sen x}{x}$. Logo,
$$\cos x\leq \frac{\sen x}{x}\leq \frac{1}{\cos x}\,,\quad\forall 0<x<\pisobredois\,.
$$
Como $\lim_{x\to 0^+}\cos x=\lim_{x\to 0^+}\frac{1}{\cos x}=1$, O Teorema \ref{Teo:Sanduichefinito} implica
$\lim_{x\to 0^+}\frac{\sen x}{x}=1$.
Como $\frac{\sen x}{x}$ é par, temos também $\lim_{x\to
0^-}\frac{\sen x}{x}=1$. Portanto, provamos \eqref{eq_limsenxsurx}.

\begin{exo}\label{Exo:variantessinxsurx}
%\eqref{itexosinxx5}
Usando \eqref{eq_limsenxsurx}, calcule os limites 
\begin{multicols}{4}
\begin{enumerate}
\item\label{itexosinxx1} $\lim_{x\to 0}\tfrac{\tan x}{x}$
\item\label{itexosinxx2} $\lim_{x\to 0}\tfrac{\sen x}{\tan x}$
\item\label{itexosinxx3} $\lim_{x\to 0}\frac{\sen 2x}{\cos x}$
\item\label{itexosinxx4} $\lim_{x\to 0}\frac{\sen 2x}{x\cos x}$
\item\label{itexosinxx5} $\lim_{x\to 0}\tfrac{1-\cos x}{x^2}$
\item\label{itexosinxx6} $\lim_{x\to 0^+}\tfrac{\cos x}{x}$
\item\label{itexosinxx7} $\lim_{x\to 0^+}\tfrac{\sen (x^2)}{x}$
\end{enumerate}
\end{multicols}
\vspace{0.01cm}
\begin{sol} \eqref{itexosinxx1}
Como $\frac{\tan x}{x}=\frac{\sen x}{x}\frac{1}{\cos x}$,
temos $\lim_{x\to 0}\tfrac{\tan x}{x}=1$.
\eqref{itexosinxx2}
Como $\frac{\sen x}{\tan x}=\cos x$, temos $\lim_{x\to 0}\tfrac{\sen
x}{\tan x}=1$.
\eqref{itexosinxx3} Como ${\sen 2x}\to 0$ e ${\cos x}\to 1$, temos $\lim_{x\to 0}\frac{\sen 2x}{\cos x}=\frac{0}{1}=0$ (não é um limite do tipo ``$\frac00$'').
\eqref{itexosinxx4}
Como $\frac{\sen 2x}{x\cos x}=2\frac{\sen x}{x}$,
temos $\lim_{x\to 0}\frac{\sen 2x}{x\cos x}=2$.
\eqref{itexosinxx5} Como 
$$\frac{1-\cos x}{x^2}=\frac{1-\cos x}{x^2}\frac{1+\cos
x}{1+\cos x}=\frac{1-\cos^2x}{x^2}\cdot \frac{1}{1+\cos x}=\Bigl(\frac{\sen x}{x}\Bigr)^2\cdot \frac{1}{1+\cos x}\,,$$
temos 
$\lim_{x\to 0}\tfrac{1-\cos x}{x^2}=(1)^2\cdot \frac12=\frac12$.
\eqref{itexosinxx6} $+\infty$
\eqref{itexosinxx7}  $\lim_{x\to 0^+}\tfrac{\sen (x^2)}{x}=\lim_{x\to
0^+}x\cdot\tfrac{\sen(x^2)}{x^2}=0\cdot 1=0$.
\end{sol}
\end{exo}



\section{Limites laterais infinitos, assíntotas verticais}
\index{assíntota!vertical}
Vimos casos em que limites laterais são iguais, casos em que eles são 
diferentes, e casos em que eles nem existem. Vejamos agora casos em que
eles são \emph{infinitos}.

\begin{ex}
Considere primeiro $f(x)=\frac{1}{x}$. Já vimos que a $f$ não é
limitada, e à medida que $x>0$ tende a zero,
$\frac{1}{x}$ cresce e toma valores positivos arbitrariamente grandes.
Por outro lado se $x<0$ tende a zero, $\frac{1}{x}$ decresce e toma
valores negativos arbitrariamente grandes:
$$
\lim_{x\to 0^+}\frac{1}{x}=+\infty\,,\quad\quad \lim_{x\to
0^-}\frac{1}{x}=-\infty\,.
$$ 
De modo geral, qualquer $x^p$ com potência inteira negativa
$p=-q$, $q>0$:
$$
\lim_{x\to 0^+}\frac{1}{x^q}=+\infty\,,\quad\quad
\lim_{x\to 0^-}\frac{1}{x^q}=
\begin{cases}
+\infty&\text{ se $q$ é par\,,}\\
-\infty&\text{ se $q$ é ímpar\,.}
\end{cases}
$$
\end{ex}

\begin{exo}
Tente definir 
rigorosamente $\lim_{x\to a^+}f(x)=+\infty$, $\lim_{x\to a^+}f(x)=-\infty$.
\begin{sol}
``$\lim_{x\to a^+}f(x)=+\infty$'' significa que $f(x)$ ultrapassa
qualquer valor dado (arbitrariamente grande), desde que $x>a$ esteja
suficientemente perto de $a$. Isto é: para todo $M>0$ (arbitrariamente
grande), existe um $\delta>0$ tal que se $a<x\leq a+\delta$, então
$f(x)\geq M$.
Por outro lado, $\lim_{x\to a^+}f(x)=-\infty$ significa que 
para todo $M>0$ (arbitrariamente grande), 
existe um $\delta>0$ tal que se $a<x\leq a+\delta$, então $f(x)\leq
-M$.
\end{sol}
\end{exo}

\begin{defin}
 Se pelo menos um dos limites $\lim_{x\to a^+}f(x)$ ou $\lim_{x\to
a^-}f(x)$ é $\pm \infty$, diremos que a reta vertical de equação
$x=a$ é \grasA{assíntota vertical} da função $f$.
\end{defin}

\begin{ex} Como $\lim_{x\to 0^+}\log_ax=-\infty$ se $a>1$, $=+\infty$
se $0<a<1$, $x=0$ é assíntota vertical da função $\log_a$.
\end{ex}

\begin{ex}
A função tangente possui infinitas assíntotas verticais, 
de equações $x=\pisobredois+k\pi$, $k\in \bZ$, 
já que para todo $k\in \bZ$,
$$
\lim_{x\to (\pisobredois+k\pi)^-}\tan x=+\infty\,,\quad\quad
\lim_{x\to (\pisobredois+k\pi)^+}\tan x=-\infty\,.
$$
\end{ex}

\begin{exo}
Calcule os limites.
\begin{multicols}{3}
\begin{enumerate}
\item\label{itlimbasic9a} $\li{x}{2^+}\frac{x^2+5x+6}{x+2}$
\item\label{itlimbasic9b} $\li{x}{-2^+}\frac{x^2+5x+6}{x+2}$
\item\label{itlimbasic9c} $\li{x}{-2^\pm}\frac{x^2+5x-6}{x+2}$
\item\label{itlimbasic10} $\lim_{x\to 2^+}\frac{x-2}{(\sqrt{x^2-4})^2}$
\item\label{itlimbasic12} $\lim_{x\to -2^-}\frac{x-2}{(\sqrt{x^2-4})^2}$
\item\label{itlimbasic14} $\li{t}{0^+}\ln t-t$
\item\label{itlimbasic14b} $\li{t}{0^-}\ln t-t$
\item\label{itlimbasic15a} $\lim_{t\to 0^\pm}\frac{1}{\sen t}$
\item\label{itlimbasic15b} $\lim_{t\to 0^\pm}\frac{t}{\sen t}$
\item\label{itlimbasic15} $\lim_{t\to 0^+}\frac{\sen \frac1t}{t}$
\item\label{itlimbasic16} $\li{z}{0^\pm}9^{\frac{1}{z}}$
\item\label{itlimbasic18b} $\lim_{x\to 0^+}\ln \frac{1}{x}$
\item\label{itlimbasic19b} $\li{x}{0}\log(x^2)$
\item\label{itlimbasic13} $\lim_{x\to 0}\frac{e^x-1}{x}$
\end{enumerate}
\end{multicols}
\vspace{0.01cm}
\begin{sol}
\eqref{itlimbasic9a} $5$
\eqref{itlimbasic9b} $1$
\eqref{itlimbasic9c} $\mp \infty$
\eqref{itlimbasic10} Observe que enquanto $x^2-4>0$, $\frac{x-2}{(
\sqrt{x^2-4})^2}=\frac{1}{x+2}$. Logo, $\lim_{x\to
2^+}\frac{x-2}{(\sqrt{x^2-4})^2}=\frac14$, e
\eqref{itlimbasic12} $\lim_{x\to -2^-}\frac{x-2}{(\sqrt{x^2-4})^2}=
-\infty$
\eqref{itlimbasic14} $-\infty$
\eqref{itlimbasic14b} Não é definido.
\eqref{itlimbasic15a}   $\lim_{t\to 0^+}\frac{1}{\sen t}=+\infty$, 
$\lim_{t\to 0^-}\frac{1}{\sen t}=-\infty$
\eqref{itlimbasic15b}   $\lim_{t\to 0^\pm}\frac{t}{\sen t}=\lim_{t\to 
0^\pm}\frac{1}{\frac{\sen t}{t}}=1$.
\eqref{itlimbasic15} Não existe, porqué quando $t\to 0^+$, $\sen
\frac1t$ 
oscila entre $+1$ e $-1$, enquanto $\frac1t$ tende a $+\infty$:
\begin{center}
\begin{bmlimage}\begin{tikzpicture}[scale=0.3]
\draw[->] (0,0)--(3,0) node[right]{$t$};
\draw[->] (0,-3)--(0,3) node[left]{$\frac{\sen \frac1t}{t}$};
\pgfmathsetmacro{\e}{0.2};
\draw[thick, domain=\e:{3}, samples=1000] plot (\x,{(sin(5/\x r))/\x});
\end{tikzpicture}\end{bmlimage}
\end{center}
\eqref{itlimbasic16} $\li{z}{0^+}9^{\frac{1}{z}}=+\infty$, 
$\li{z}{0^-}9^{\frac{1}{z}}=0$.
\eqref{itlimbasic18b} $+\infty$
\eqref{itlimbasic19b} $-\infty$
\eqref{itlimbasic13} $1$ (veremos mais tarde como calcular esse
limite...) 
\end{sol}
\end{exo}

\begin{exo}
Na Teoria da Relatividade Restrita (ou Especial), cujo principal 
postulado é que a velocidade da luz é uma constante $c>0$ para
qualquer observador, 
é provado que a massa efetiva de uma partícula em movimento uniforme 
depende da sua velocidade. Se a massa no repouso é $m_0$, então a sua
massa efetiva quando a partícula tem uma
velocidade constante $v$ é dada por
$$m_v=\frac{m_0}{\sqrt{1-\frac{v^2}{c^2}}}\,.$$
Estude $m_v$ quando $v$ se aproxima da velocidade da luz.
\begin{sol}
A função $v\mapsto m_v$ tem domínio $[0,c)$, e a reta $v=c$ é 
assíntota vertical: 
\begin{center}
\begin{bmlimage}\begin{tikzpicture}
\draw[->] (0,0)--(4,0) node[right]{$v$};
\draw[->] (0,0)--(0,3) node[left]{$m_v$};
\pgfmathsetmacro{\c}{3.2};
\pgfmathsetmacro{\m}{1};
\fill (0,\m) circle (0.45mm);
\draw (0,\m) node[left]{$m_0$};
\draw[thick, domain=0:{\c-0.2}] plot (\x,{\m/(sqrt(1-(\x/\c)^2))});
\draw[dotted] (\c,0) node[below]{$c$}--(\c,3);
\draw (3.8,2.5) node[right]{$\displaystyle{\lim_{v\to c^-}m_v}=
+\infty$};
\end{tikzpicture}\end{bmlimage}
\end{center}
\end{sol}
\end{exo}

\begin{exo}
Considere $f(x)=\frac{x+1}{x-1}$. Estude os limites relevantes e 
ache as assíntotas (horizontais e verticais) de $f$. A partir dessas
informações, monte o gráfico de $f$.
\begin{sol}
Observe que $\lim_{x\to \pm\infty}f(x)=+1$, logo $y=1$ é assíntota 
horizontal. 
Por outro lado, $\lim_{x\to 1^+}f(x)=+\infty$ e $\lim_{x\to 1^-}f(x)
=-\infty$. Portanto, $x=1$ é assíntota vertical. 
Temos então: 1) o gráfico se aproxima da sua assintota horizontal em 
$-\infty$, e ele tende a $-\infty$ quando $x\to 1^-$,
2) o gráfico se aproxima da sua assintota horizontal em $+\infty$, e 
ele tende a $+\infty$ quando $x\to 1^+$.
Somente com essas informações, um esboço razoável pode ser montado: 
\begin{center}
\begin{bmlimage}\begin{tikzpicture}[scale=0.5]
\draw[->] (-3,0)--(3,0) node[right]{$x$};
\draw[->] (0,-2)--(0,3) node[left]{$y$};
\pgfmathsetmacro{\e}{0.6};
%\pgfmathsetmacro{\m}{1};
%\fill (0,\m) circle (0.45mm);
%\draw (0,\m) node[left]{$m_0$};
\draw[thick, domain=-4:1-\e] plot (\x,{(\x+1)/(\x-1)});
\draw[thick, domain=1+\e:4] plot (\x,{(\x+1)/(\x-1)});
\draw[dotted] (1,-2) node[right]{$\scriptstyle{x=1}$}--(1,3);
\draw[dotted] (-4,1) node[above]{$\scriptstyle{y=1}$}--(4,1);
%\draw (3.8,2.5) node[right]{$\displaystyle{\lim_{v\to c^-}m_v}=+\infty$};
\end{tikzpicture}\end{bmlimage}
\end{center}
Observe que pode também escrever $\frac{x+1}{x-1}=\frac{2}{x-1}+1$, 
logo o gráfico pode ser obtido a partir de transformações elementares
do gráfico de $\frac1x$...
\end{sol}
\end{exo}

\begin{exo}
Dê o domínio e ache as assíntotas (horizontais e verticais), caso
existam, das funções 
\begin{multicols}{4}
\begin{enumerate}
\item\label{itexassint1} $2x+1$
\item\label{itexassint2} $\frac{1}{x+1}$
\item\label{itexassint3} $\frac{x^2-9}{x-3}$
\item\label{itexassint4} $\frac{2x-3}{x}$
\item\label{itexassint5} $\frac{1-x}{x+3}$
\item\label{itexassint6} $\frac{x}{x}$
\item\label{itexassint6b} $\log_5(2-x)$
\item\label{itexassint7} $x^3+\frac{1}{x}$
\item\label{itexassint8} $\frac{\sen x}{x}$
\item\label{itexassint9} $\frac{\cos x}{x}$
\item\label{itexassint10} $\frac{x^2+4x-21}{x^2-x+6}$
\item\label{itexassint11} $\ln(1-x^2)$
\item\label{itexassint12} $\frac{1}{2+x}+\ln(1-x^2)$
\item\label{itexassint13} $\frac{6-2x}{(1-x^2)(x-3)}$
\item\label{itexassint14} $\frac{1}{\ln(1-x^2)}$
\item\label{itexassint17} $\frac{\sqrt{x^2+1}}{x}$
\item\label{itexassint15b} $\frac{1}{\sqrt{1-x^2}}$
\item\label{itexassint18} $\frac{\ln(1+e^x)}{x}$
\end{enumerate}
\end{multicols}
\vspace{0.01cm}
\begin{sol}
\eqref{itexassint1} $D=\bR$, sem assíntotas.
\eqref{itexassint2} $D=\bR\setminus\{-1\}$. Horiz: $y=0$, Vertic: $x=-1$.
\eqref{itexassint3} $D=\bR\setminus\{3\}$. sem assíntotas.
\eqref{itexassint4} $D=\bR\setminus\{0\}$. Horiz: $y=2$, Vertic: $x=0$.
\eqref{itexassint5} $D=\bR\setminus\{-3\}$. Horiz: $y=-1$, Vertic: $x=-3$.
\eqref{itexassint6} $D=\bR\setminus\{0\}$. Horiz: $y=1$, Vertic: não tem.
\eqref{itexassint6b} $D=(-\infty,2)$. Horiz: não tem, Vertic: $x=2$.
\eqref{itexassint7} $D=\bR\setminus\{0\}$. Horiz: não tem, Vertic: $x=0$.
\eqref{itexassint8} $D=\bR\setminus\{0\}$. Horiz: $y=0$, Vertic: não tem.
\eqref{itexassint9} $D=\bR\setminus\{0\}$. Horiz: $y=0$, Vertic: $x=0$.
\eqref{itexassint10} $D=\bR$. Horiz: $y=1$, Vertic: não tem.
\eqref{itexassint11} Para garantir $1-x^2>0$, $D=(-1,1)$ Horiz: não 
tem (já que o domínio é $(-1,1)$...), Vertic: $x=-1$ (porqué
$\lim_{x\to -1^+}\ln (1-x^2)=-\infty$), $x=+1$ (porqué $\lim_{x\to
+1^-}\ln (1-x^2)=-\infty$).
\eqref{itexassint12} $D=(-1,1)$. Horiz: não tem, Vertic: $x=-1$, $x=+1$.
\eqref{itexassint13} $D=\bR\setminus\{\pm 1, 3\}$. Horiz: $y=0$, 
Vertic: $x=+1$, $x=-1$.
\eqref{itexassint14} $D=(-1,+1)\setminus\{ 0\}$. Horiz: não tem, 
Vertic: $x=0$.
\eqref{itexassint17} $D=\bR\setminus\{0\}$. Horiz: $y=+1$, $y=-1$, 
Vertic: $x=0$.
\eqref{itexassint15b} $D=(-1,1)$. Horiz: não tem, Vertic: $x=-1$, $x=+1$.
\eqref{itexassint18} $D=\bR\setminus\{0\}$. Horiz: $y=1$ (a direita), $y=0$ (a esquerda), 
Vertic: $x=0$.
\end{sol}
\end{exo}

\begin{exo}(Primeira prova, Turmas D, 15 de abril de 2011)
Defina \emph{assíntota horizontal/vertical} de uma função $f$, e 
ache as assíntotas das funções 
$$
\frac{|x-\pi|}{\pi+x}\,,\quad 
\frac{2+\sen x-3 x^2}{x^2-x+20}
\,,\quad \frac{\sqrt{x(x-1)}}{x-1}\,.$$
\end{exo}

\begin{exo}
Dê exemplos de funções $f$ que tenham $x=-1$ e
$x=3$ como assíntotas verticais, e $y=-1$ como assíntota
horizontal.
\begin{sol}
Por exemplo: $f(x)=\frac{1-x^2}{(x+1)(x-3)}$, ou $f(x)=\frac{1}{x+1}
+\frac{1}{x-3}-\frac{x^2}{x^2+1}$.
\end{sol}
\end{exo}

\section{Mudar de variável}
\index{mudança de variável}
O cálculo de um limite pode ser às vezes simplificado transformando
ele em outro limite, via uma \emph{mudança de variável}.

\begin{ex}
Suponha que se queira calcular o limite de $\frac{\sen 2x}{x}$ quando
$x\to 0$. Um jeito possível é de usar a identidade 
$\sen 2x=2\sen x\cos x$, escrevendo 
$$
\lim_{x\to 0}\frac{\sen 2x}{x}
=\lim_{x\to 0}2\frac{\sen x}{x}\cos x=2\cdot
1\cdot{1}=2\,.
$$
Um outro jeito de proceder é de introduzir a nova variável $y\pardef
2x$. Ao fazer essa mudança, é preciso reescrever o limite
$\lim_{x\to 0}\frac{\sen 2x}{x}$ somente usando a variável $y$. Como
$x\to 0$ implica $y\to 0$, e como $x=y/2$, 
$$
\lim_{x\to 0}\frac{\sen 2x}{x}
=\lim_{y\to 0}\frac{\sen y}{y/2}=2\lim_{y\to 0}\frac{\sen
y}{y}=2\cdot 1=2.$$
\end{ex}

\begin{ex} Considere o limite 
$\lim_{x\to 0}\frac{\cos^3x-1}{\cos x-1}$. Chamando $z\pardef \cos
x$, ao $x\to 0$ temos $z\to 1$. Logo, 
$$
\lim_{x\to 0}\frac{\cos^3x-1}{\cos x-1}=\lim_{z\to
1}\frac{z^3-1}{z-1}=3\quad\text{(ver Exemplo
\ref{Ex:primeironaotrivial}).}
$$
\end{ex}

Vejamos também como um limite lateral pode ser transformado em um
limite no infinito:

\begin{ex}
Considere os  limites laterais calculados no Exercício
\ref{Exo:Limiteselementares}: $\lim_{x\to 0^+}9^{\frac{1}{x}}$,
$\lim_{x\to 0^-}9^{\frac{1}{x}}$.
Chamemos $z\pardef \frac{1}{x}$. Se $x\to 0^+$, então $z\to
+\infty$. Logo,
$$\lim_{x\to 0^+}9^{\frac{1}{x}}
=\lim_{z\to\infty}9^z=+\infty\,.
$$
Por outro lado, se $x\to 0^-$, então $z\to -\infty$, e
$$\lim_{x\to 0^-}9^{\frac{1}{x}}
=\lim_{z\to-\infty}9^z=0\,.
$$
\end{ex}

\begin{exo}\label{Exo:mudvarlimites}
Calcule os limites fazendo uma mudança de variável.
\begin{multicols}{3}
\begin{enumerate}
\item\label{itmudvarlim1} $\lim_{x\to 1}\frac{\sen (x-1)}{3x-3}$
\item\label{itmudvarlim11} $\lim_{x\to 0}\frac{\sen (3x)}{\sen (5x)}$
\item\label{itmudvarlim2} $\lim_{x\to -1}\frac{\sen(x+1)}{1-x^2}$
\item\label{itmudvarlim3} $\lim_{x\to a}\frac{x^n-a^n}{x-a}$
\item\label{itmudvarlim31} $\lim_{x\to 4}\frac{x-4}{x-\sqrt{x}-2}$
\item\label{itmudvarlim4} $\lim_{x\to 0^{\pm}}\tanh \frac{1}{x}$
\item\label{itmudvarlim5} $\lim_{x\to 0^{\pm}}x\tanh \frac{1}{x}$
\end{enumerate}
\end{multicols}
\vspace{0.01cm}
\begin{sol}
\eqref{itmudvarlim1} Com $z\pardef x-1$, $\lim_{x\to 1}\frac{\sen
(x-1)}{3x-3}=\lim_{z\to 0}\frac{\sen
z}{3z}=\frac13$.
\eqref{itmudvarlim11} $\frac35$ (Escreve $\frac{\sen (3x)}{\sen (5x)}=\frac{\sen (3x)}{3x}\frac{1}{\frac{\sen (5x)}{5x}}\frac{3x}{5x}$.) 
\eqref{itmudvarlim2} Com $z\pardef x+1$, $\lim_{x\to
-1}\frac{\sen(x+1)}{1-x^2}=\lim_{z\to
0}\frac{\sen z}{z}\frac{1}{2-z}=\frac12$.
\eqref{itmudvarlim3} Com $h\pardef x-a$, 
$\lim_{x\to a}\frac{x^n-a^n}{x-a}=\lim_{h\to
0}\frac{(a+h)^n-a^n}{h}=na^{n-1}$.
\eqref{itmudvarlim31} Chamando $t\pardef \sqrt{x}$, 
$$\lim_{x\to 4}\frac{x-4}{x-\sqrt{x}-2}=\lim_{t\to 2}\frac{t^2-4}{t^2-t-2}=\lim_{t\to 2}\frac{(t-2)(t+2)}{(t-2)(t+1)}=\lim_{t\to 2}\frac{(t+2)}{(t+1)}=\tfrac43\,.$$
\eqref{itmudvarlim4} Com $z\pardef \frac{1}{x}$, temos (lembre o
item \eqref{itexliminfini22} do Exercício \ref{Exo:limitesinfini})
 $\lim_{x\to 0^+}\tanh \frac{1}{x}=\lim_{z\to +\infty}\tanh z=+1$,
$\lim_{x\to 0^-}\tanh \frac{1}{x}=\lim_{z\to -\infty}\tanh z=-1$.
\eqref{itmudvarlim5} Com a mesma mudança,
$\lim_{x\to 0^\pm}x\tanh \frac{1}{x}=\lim_{z\to \pm
\infty}\frac{1}{z}\tanh{z}=0\cdot (\pm 1)=0$.
\end{sol}
\end{exo}

\section{O limite $e=\lim_{x\to
\infty}\bigl(1+\tfrac1x\bigr)^x$}\label{Sec:Fundam_numero_e}

Mencionamos, no último capítulo, que uma das definições possíveis do
número $e=2,718...$ é via o limite de $(1+\tfrac1x)^x$
quando $x\to \infty$. De fato,
\begin{center}
\begin{tabular}{c|c|c|c|c}
$x=$&10&100&1000&10'000\\
\hline
$(1+\tfrac1x\bigr)^x=$&$2,59374...$&$2,70481...$&$2,
71692...$&$2.71814...$
\end{tabular}
\end{center}
Pode ser mostrado que o limite quando $x\to \infty$
existe, e tomamos o valor do limite como definição da base do logaritmo natural:
$$\boxed{e\pardef \lim_{x\to \infty}\bigl(1+\tfrac1x\bigr)^x\,.}$$

Essa caracterização de $e$ permite calcular vários limites importantes,
como por exemplo
$\lim_{h\to 0^+}\frac{\ln(1+h)}{h}$.
De fato, com a mudança de variável $z=\frac{1}{h}$, $h\to 0^+$ implica $z\to +\infty$:
\eq{\label{eq:usoimpliccontinlog}\lim_{h\to
0^+}\frac{\ln(1+h)}{h}=\lim_{z\to+\infty}\frac{\ln(1+\tfrac{1}{z})}{\frac1z}=
\lim_{z\to+\infty}\ln\bigl((1+\tfrac{1}{z})^z\bigr)
=\ln e=1\,.
}

Um outro limite que pode ser calculado é $\lim_{x\to 0^+}\frac{e^x-1}{x}$. Dessa
vez, chamando $z=e^x$, $x\to 0^+$ implica $z\to 1^+$:
$$
\lim_{x\to 0^+}\frac{e^x-1}{x}=\lim_{z\to 1^+}\frac{z-1}{\ln
z}=\lim_{z\to
1^+}\frac{1}{\frac{\ln z}{z-1}}$$
Mas agora se $h\pardef z-1$, então $z\to 1^+$ implica $h\to 0^+$, e 
por \eqref{eq:usoimpliccontinlog},
$$
\lim_{z\to 1^+}\frac{\ln z}{z-1}=\lim_{h\to 0^+}\frac{\ln (1+h)}{h}=1\,.
$$
Portanto, 
\eq{\label{Eq:derivexponenzero}
\lim_{x\to 0^+}\frac{e^x-1}{x}=1\,.}
Observe que o limite lateral a esquerda se obtém facilmente: chamando $y\pardef -x$,
\begin{align*}
\lim_{x\to 0^-}\frac{e^x-1}{x}
=\lim_{y\to 0^+}\frac{e^{-y}-1}{-y}&=\lim_{y\to 0^+}\frac{e^{y}-1}{y}e^{-y}\\
&=\Bigl(\lim_{y\to 0^+}\frac{e^{y}-1}{y}\Bigr)\bigl(\lim_{y\to 0^+}e^{-y}\bigr)=1\cdot 1=1\,.
\end{align*}


\begin{exo}
Mostre que  para todo $a>0$,
\eq{\label{eq:derivlnenun}\lim_{h\to
0}\frac{\log_a(1+h)}{h}=\frac{1}{\ln a}\,,\quad\quad \lim_{x\to
0}\frac{a^x-1}{x}=\ln a\,.}
\begin{sol}
Pela fórmula \eqref{eq:mudancabaselog} de mudança de base para o
logaritmo, $\log_a(1+h)=\frac{\ln(1+h)}{\ln a}$. Logo, por
\eqref{eq:derivlnenun},
$$\lim_{h\to 0}\frac{\log_a(1+h)}{h}=\frac{1}{\ln a}\lim_{h\to
0}\frac{\ln(1+h)}{h}=\frac{1}{\ln a}\,.$$
Por outro lado, chamando $z\pardef a^x$, $x\to 0$ implica $z\to 1$.
Mas $x=\log_az$, logo
$$
\lim_{x\to
0}\frac{a^x-1}{x}=\lim_{z\to 1}\frac{z-1}{\log_a
z}=\frac{1}{\lim_{z\to 1}\frac{\log_az}{z-1}}\,.
$$
Definindo $h\pardef z-1$ obtemos $\lim_{z\to
1}\frac{\log_az}{z-1}=\lim_{h\to 0}\frac{\log_a(1+h)}{h}=\frac{1}{\ln
a}$, o que prova a identidade desejada.
\end{sol}
\end{exo}

\section{O limite 
$\lim_{x\to\infty}\frac{a^x}{x}$}\label{sec_Lim_parenteseAldo}
Nesta seção
%~\footnote{Essa seção foi escrita após a proposta
%do Prof.\ Aldo Procacci.} 
veremos como calcular alguns limites que envolvem
exponenciais e logaritmos, do tipo 
\[ 
\lim_{x\to\infty}\frac{e^x}{x}\,, \quad
\lim_{x\to\infty}\frac{(\ln x)^3}{x}\,, \dots\quad
\]
Esses limites costumam ser estudados usando a regra de
Bernoulli-l'Hôpital, que será vista no
Capítulo~\ref{Cap:Derivacao}. 
Será suficiente considerar um caso.
\begin{ex}
Mostraremos aqui que
\begin{equation}\label{eq_Lim_ideiaaldo} 
\boxed{
\lim_{x\to\infty}\frac{a^x}{x}=
\begin{cases}
+\infty&\text{ se }a>1\,,\\
0&\text{ se }0<a\leq 1\,.
\end{cases}
}
\end{equation}
Quando $a=1$, o limite é simplesmente
$\lim_{x\to\infty}\frac{a^0}{x}=
\lim_{x\to\infty}\frac{1}{x}=0$.
Quando $0<a<1$, temos $\lim_{x\to\infty}a^x=0$ (lembre
de~\eqref{eq_Lim_expinf_c}), o que implica
$\lim_{x\to\infty}\frac{a^x}{x}=
\lim_{x\to\infty}a^x\cdot\frac{1}{x}=0\cdot 0=0$.
Portanto, falta tratar o caso $a>1$.
Observe que nesse caso, podemos escrever $a=1+\beta$, com
$\beta>0$. 
\[ 
\frac{a^x}{x}=\frac{(1+\beta)^x}{x}\,.
\]
Suponhamos agora que $x>0$ seja grande. 
Como $x\geq \lfloor x \rfloor$, temos $(1+\beta)^x\geq
(1+\beta)^{\lfloor x \rfloor}$.
Agora, como $\lfloor x\rfloor$ é \emph{inteiro}. 
Assim podemos usar a fórmula do binômio de
Newton~\footnote{$(A+B)^n=\sum_{k=0}^n\binom{n}{k}A^kB^{n-k}$. Aqui 
usamos essa fórmula com $A=1$, $B=\beta$.}: 
\begin{align*}
(1+\beta)^{\lfloor x\rfloor}&=1+\beta
{\lfloor x\rfloor}+\beta^2\frac{{\lfloor x\rfloor}({\lfloor x\rfloor}-1)}{2}+\dots+\frac{{\lfloor x\rfloor}({\lfloor x\rfloor}-1)}{2}\beta^{\lfloor x\rfloor}\\
&\geq \beta^2\frac{{\lfloor x\rfloor}({\lfloor x\rfloor}-1)}{2}\,.
\end{align*}
Nesta última desigualdade observamos que 
todos os termos da soma são positivos, 
e mantivemos somente o termo de ordem $2$.
Portanto,
$\frac{a^x}{x} \geq \beta^2\frac{{\lfloor x\rfloor}^2-{\lfloor x\rfloor}}{2{x}}$.
Mas, como
\[ 
\lim_{x\to \infty}\frac{{\lfloor x\rfloor}^2-{\lfloor
x\rfloor}}{2{ x}}=+\infty\,, \]
provamos o resultado desejado:
$\lim_{x\to\infty}\frac{a^x}{x}=+\infty$.
\end{ex}

\begin{ex}
Podemos usar o último exemplo para mostrar que 
se $a>1$, então para todo inteiro $p>0$,
\begin{equation}\label{eq_Limlimlimlog} 
\boxed{
\lim_{x\to\infty}\frac{a^x}{x^p}=+\infty\,.
}
\end{equation}
De fato, podemos sempre escrever
$\frac{a^x}{x^p}=(\frac{b^x}{x})^p$, em que $b=a^{1/p}$. Como
$a>1$, vale $b>1$ também. Logo, por~\eqref{eq_Lim_ideiaaldo},
\[ 
\lim_{x\to\infty}\frac{a^x}{x^p}=
\lim_{x\to\infty}\bigl(\frac{b^x}{x}\bigr)^p=
\Bigl(\lim_{x\to\infty}\frac{b^x}{x}\Bigr)^p=+\infty\,.
\]
\end{ex}
A partir dos exemplos anteriores podemos calcular outros
limites: para todo $a>1$ e todo inteiro $p>0$,
\[ 
\boxed{
\lim_{x\to\infty}\frac{(\log_a x)^p}{x}=0
}
\]
De fato, com a mudança $y=\log_a x$, $x\to\infty$ implica
$y\to \infty$, logo por~\eqref{eq_Limlimlimlog}
\[ 
\lim_{x\to\infty}\frac{(\log_a x)^p}{x}=
\lim_{y\to\infty}\frac{y^p}{a^y}=
\lim_{y\to\infty}\frac{1}{\frac{a^y}{y^p}}=0\,.
\]
\begin{exo}
Calcule os limites abaixo.  
\begin{multicols}{3}
\begin{enumerate}
\item\label{it_exsuplAldo_1}
$\lim_{x\to\infty}\frac{e^{x}}{x^3}$
\item\label{it_exsuplAldo_2}
$\lim_{x\to\infty}\frac{0.5^x(\tfrac{18}{8})^x}{x^{16}}$
\item\label{it_exsuplAldo_3}
$\lim_{x\to\infty}\frac{e^{-2x}}{x}$
\item\label{it_exsuplAldo_4}
$\lim_{x\to\infty}\frac{e^{5x}}{x^{-2}}$
\item\label{it_exsuplAldo_5}
$\lim_{x\to\infty}\frac{(\log_3x)^{7}}{4x}$
\item\label{it_exsuplAldo_6}
$\lim_{x\to\infty}\frac{e^{2x}+3x^5}{(e^x+1)^2+2x^3}$
\end{enumerate}
\end{multicols}
\vspace{0.1mm}
\begin{sol}
\eqref{it_exsuplAldo_1} $\infty$
\eqref{it_exsuplAldo_2} $\infty$
\eqref{it_exsuplAldo_3} $0$
\eqref{it_exsuplAldo_4} $\infty$
\eqref{it_exsuplAldo_5} $0$
\eqref{it_exsuplAldo_6} $1$
\end{sol}
\end{exo}

\section{Exercícios de revisão}

\begin{exo}
Considere a função 
$$f(x)=
\begin{cases}
2x+2&\text{ se }x<0\,,\\
x^2-2&\text{ se }0\leq x<2\,,\\
2&\text{ se }x\geq 2\,.\\
\end{cases}
$$
Calcule os limites 
$\lim_{x\to 0^-}f(x)$,
$\lim_{x\to 0^+}f(x)$, $\lim_{x\to 0}f(x)$, $\lim_{x\to 2^-}f(x)$, $\lim_{x\to 2^+}f(x)$, 
$\lim_{x\to 2}f(x)$. Em seguida, interprete esses limites no gráfico de $f$.
\begin{sol}
$\lim_{x\to 0^-}f(x)=\lim_{x\to 0^-}(2x+2)=2$,
$\lim_{x\to 0^+}f(x)=\lim_{x\to 0^+}(x^2-2)=-2$,
Já que esses dois limites laterais são diferentes, $\lim_{x\to 0}f(x)$ não
existe.
$\lim_{x\to 2^-}f(x)=\lim_{x\to 2^-}(x^2-2)=2$.
$\lim_{x\to 2^+}f(x)=\lim_{x\to 2^+}2=2$. Como
$\lim_{x\to 2^-}f(x)=\lim_{x\to 2^+}f(x)$, $\lim_{x\to 2}f(x)$ existe e vale
$2$.
\begin{center}
\begin{bmlimage}\begin{tikzpicture}[scale=0.7]
\draw[->] (-1.5,0)--(3,0)node[right]{$x$};
\draw[->] (0,-2)--(0,2.4);
\draw[->, thick] (-2,-2)--(0,2);
\fill (0,-2) circle (0.45mm);
\draw[thick, domain=0:2] plot (\x,{\x^2-2});
\fill (2,2) circle (0.45mm);
\draw[thick] (2,2)--(3,2);
\end{tikzpicture}\end{bmlimage}
\end{center}
\end{sol}
\end{exo}

\begin{exo}
Considere um ponto $Q$ na parábola $y=x^2$.
Seja $M$ o ponto meio do segmento $OQ$ ($O$ é a origem) e seja
$r$ a reta perpendicular ao segmento $OQ$, passando por $M$. Seja $R$
a interseção de $r$ com o eixo $y$. Estude o que acontece com $R$ quando
$Q$ varia. O que acontece com $R$ no limite $Q\to O$?
\begin{sol}
O ponto $Q$ é da forma $Q=(\lambda,\lambda^2)$, e $Q\to O$
corresponde a $\lambda\to 0$.
Temos $M=(\frac{\lambda}{2},\frac{\lambda^2}{2})$. 
É fácil ver que a equação da reta $r$ é 
$y=-\frac{1}{\lambda}x+\frac{\lambda^2}{2}+\frac12$. Logo,
$R=(0,\frac{\lambda^2}{2}+\frac12)$. Quando $Q$ se aproxima da origem,
isto é, quando $\lambda$ se aproxima de $0$, $\lambda^2$ decresce,
o que significa que $R$ \emph{desce}. Quando $\lambda\to 0$, $R\to
(0,\frac12)$. (Pode parecer contra-intuitivo, já que o segmento $OQ$ tende a
ficar sempre mais horizontal, logo o segmento $MR$ fica mais vertical, à medida
que $Q\to O$.)
\end{sol}
\end{exo}

\begin{exo}\label{Exo:DecomporCircemTriang}
Considere um círculo $C$ de raio $r>0$. Considere a divisão de $C$ em $n$
setores de aberturas iguais. 
Aproxime a área de cada setor pela área de um triângulo,
escreva a área $A_n$ do polígono definido pela união dos $n$ triângulos, e
calcule $\lim_{n\to \infty}A_n$.
\begin{sol}\mbox{}
\begin{center}
\begin{bmlimage}\begin{tikzpicture}
\pgfmathsetmacro{\r}{2};
\pgfmathsetmacro{\n}{10};
\pgfmathsetmacro{\incrang}{2*3.14152/\n};
\draw (0,0) circle (\r);
\foreach \k in {0,...,\n} {
\coordinate (Pk) at ({\r*cos(\k*\incrang r)},{\r*sin(\k*\incrang r)});
\coordinate (Pkm) at ({\r*cos((\k-1)*\incrang r)},{\r*sin((\k-1)*\incrang
r)});
\fill[color=gray!10] (0,0)--(Pk)--(Pkm)--cycle;
\draw (0,0)--(Pk)--(Pkm);
}
\coordinate (Pk) at ({\r*cos(\incrang r)},{\r*sin(\incrang r)});
\coordinate (Pkm) at ({\r*cos(0 r)},{\r*sin(0 r)});
\coordinate (M) at ($(Pk)!(0,0)!(Pkm)$);
\fill[color=gray!30] (0,0)--(Pk)--(Pkm)--cycle;
\draw[thick] (0,0)--(Pk)--(Pkm)--cycle;
\draw[thick] (0,0)--(M);
%\draw (M) node[above right]{$M$};
\end{tikzpicture}\end{bmlimage}
\end{center}
Como um setor tem abertura $\alpha_n=\frac{2\pi}{n}$, 
a área de cada triângulo se calcula facilmente:
$$2\times \frac12
\times(r\cos\tfrac{\alpha_n}{2})\times
(r\sen\tfrac{\alpha_n}{2})=\frac{r^2}{2}\sen
{\alpha_n}=\frac{r^2}{2}\sen \tfrac{2\pi}{n}\,.$$
Logo, a área do polígono é dada por $A_n=n\times \frac{r^2}{2}\sen
\frac{2\pi}{n}$. No limite $n\to \infty$ obtemos
$$
\lim_{n\to \infty}A_n=r^2\lim_{n\to \infty}\frac{n}{2}\sen \frac{2\pi}{n}
=\pi r^2\lim_{n\to \infty}\frac{1}{\frac{2\pi}{n}}\sen \tfrac{2\pi}{n}
=\pi r^2\lim_{t\to 0^+}\frac{\sen t}{t}=\pi r^2\,.
$$
\end{sol}
\end{exo}

\begin{exo}
Calcule o limite, se existir.
\begin{multicols}{3}
\begin{enumerate}
\item\label{itrevisaolimites1} $\lim_{x\to 2}\frac{x^4-16}{x-2}$
\item\label{itrevisaolimites11} $\lim_{x\to \frac13}\frac{3x^2-x}{3x-1}$
\item\label{itrevisaolimites2} $\lim_{x\to 3}\frac{x^2+4x-21}{x^2-x-6}$
\item\label{itrevisaolimites3} $\lim_{x\to 3}\frac{x^2+4x-21}{x^2-x+6}$
\item\label{itrevisaolimites31} $\lim_{x\to \infty}\frac{x^2+4x-21}{x^2-x+6}$
\item\label{itrevisaolimites4} $\lim_{x\to\infty}\frac{x^3+1}{x^3+x^2-2x^3}$
\item\label{itrevisaolimites5} $\lim_{x\to -1}\frac{\sen(x+1)}{1-x^2}$
\item\label{itrevisaolimites6} $\lim_{x\to 0}\frac{\sen x}{(\cos x)^2}$
\item\label{itrevisaolimites7} $\lim_{x\to 0^+}\log_9(\sen(x))$
\item\label{itrevisaolimites8} $\lim_{x\to 0^+}\log_9(\cos(x))$
\item\label{itrevisaolimites9} $\lim_{x\to 0}\frac{1-\cos x}{x}$
\item\label{itrevisaolimites10} $\lim_{x\to 0}(\frac1x-\frac{1}{e^x-1})$ 
\item\label{itrevisaolimites12} $\scriptstyle{
\li{x}{\pm \infty}\sqrt{x^2-\pi
x}-\sqrt{x^2-1}}$
\item\label{itrevisaolimites13}
$\li{x}{+\infty}\sen(\frac{\pi}{2}+\frac{1}{1+x^2})$
\item\label{itrevisaolimites14} $\li{x}{+\infty}\frac{x^2+3}{5+x^3}$
\item\label{itrevisaolimites15} $\li{x}{+\infty}\frac{1-x^7}{10x^7+1}$
\item\label{itrevisaolimites16} $\lim_{h\to 0}\frac{\sqrt{3+3h}-\sqrt{3}}{h}$
\item\label{itrevisaolimites161} $\lim_{h\to 1}\frac{\sqrt[3]{h}-1}{\sqrt{h}-1}$
\item\label{itrevisaolimites17} $\lim_{x\to
-\infty}\frac{5x^2+8x-3}{7x^3-4x-17}$
\item\label{itrevisaolimites20} $\lim_{x\to 0}\frac{x\sen x}{2-2\cos x}$
\item\label{itrevisaolimites21} $\lim_{x\to
0}\frac{1-\sqrt{1-4x^2}}{2 x}$
\end{enumerate}
\end{multicols}
\vspace{0.01cm}
\begin{sol}
\eqref{itrevisaolimites1} $32$
\eqref{itrevisaolimites11} $\frac13$
\eqref{itrevisaolimites2} $2$
\eqref{itrevisaolimites3} $0$
\eqref{itrevisaolimites31} $1$
\eqref{itrevisaolimites4} $-1$
\eqref{itrevisaolimites5} Com a mudança $y=x+1$, $\frac12$
\eqref{itrevisaolimites6} $0$
\eqref{itrevisaolimites7} $-\infty$
\eqref{itrevisaolimites8} $0$
\eqref{itrevisaolimites9} $0$
\eqref{itrevisaolimites10} $\tfrac12$ (Pois é, esse limite é
um pouco mais difícil. Calcularemos ele no
Capítulo~\ref{Cap:Derivacao} usando a regra de
Bernoulli-l'Hôpital.)
\eqref{itrevisaolimites12} $\mp \pisobredois$
\eqref{itrevisaolimites13} Como $\sen$ é contínua em $\pisobredois$,
$\li{x}{+\infty}\sen(\frac{\pi}{2}+\frac{1}{1+x^2})=\sen(\frac{
\pi}{2}+\li{x}{+\infty}\frac{1}{1+x^2})=\sen \pisobredois =1$.
\eqref{itrevisaolimites14} $0$
\eqref{itrevisaolimites15} $-\frac{1}{10}$
\eqref{itrevisaolimites16} $\frac{\sqrt{3}}{2}$
\eqref{itrevisaolimites161} $\frac{2}{3}$
\eqref{itrevisaolimites17} $0$
\eqref{itrevisaolimites20} $1$ 
\eqref{itrevisaolimites21} $1$ 
\end{sol}
\end{exo}

\begin{exo} Prove o Teorema \ref{Teo:Sanduicheinfinito}.
\begin{sol}
Seja $\epsilon>0$ e $N$ grande o suficiente, tal que $|g(x)-\ell|\leq \epsilon$ e $|h(x)-\ell|\leq \epsilon$ para todo $x\geq N$.
Para esses $x$, podemos escrever $f(x)-\ell \leq h(x)-\ell\leq |h(x)-\ell|\leq \epsilon$, e $f(x)-\ell\geq g(x)-\ell\geq -|g(x)-\ell|\geq -\epsilon$. Logo, $|f(x)-\ell|\leq \epsilon$.
\end{sol}
\end{exo}

\begin{exo}\label{Exo:LimitescomDicas}
Calcule os limites 
%\begin{multicols}{2}
\begin{enumerate}
\item\label{itlimdificeis0} $\lim_{x\to 0}\frac{\sqrt{1-\cos x}}{|x|}$ (Dica:
$1-\cos^2 x=\dots$) 
\item\label{itlimdificeis1} $\lim_{h\to 0}\frac{\sen (a+h)-\sen a}{h}$ (Dica:
$\sen (a+b)=\dots$)
\item\label{itlimdificeis2} $\lim_{x\to \alpha}\frac{x^3-\alpha^3}{\sen
(\tfrac{\pi}{\alpha}x)}$ (Dica: $\lim_{x\to
\alpha}\frac{x^3-\alpha^3}{x-\alpha}=\dots$)
\item\label{itlimdificeis3} Para $a,b>0$, $\lim_{x\to \pisobretres}\frac{1-2\cos
x}{\sen(\pi-3x)}$ (Dica: $\pi-3x=3t$, $\cos (a+b)=\dots$)
\item\label{itlimdificeis34} $\lim_{x\to\infty}\frac1x\ln(a^x+b^x)$ (Dica:
distinguir $a\geq b$, $a<b$)
\item\label{itlimdificeis37}
Para $n\in \bN$, $x_0\in \bR$, 
$\lim_{h\to 0}\frac{(x+h)^n-x^n}{h}$ 
(Dica: use uma mudança de variável $z=x+h$ e faça uma
divisão, ou então
use a fórmula do binômio de Newton para expandir $(x_0+h)^n$.)
\end{enumerate}
%\end{multicols}
\begin{sol}
\eqref{itlimdificeis0} Como $\sqrt{1-\cos^2x}=\sqrt{\sen^2x}=|\sen x|$ e
$x\mapsto |x|$ é contínua,
$$\lim_{x\to 0}\frac{\sqrt{1-\cos x}}{|x|}=\lim_{x\to
0}\frac{1}{\sqrt{1+\cos x}}\frac{|\sen x|}{|x|}
=\Bigl(\lim_{x\to 0}\frac{1}{\sqrt{1+\cos x}}\Bigr)
\cdot \Bigl|
\lim_{x\to 0}\frac{\sen x}{x}\Bigr|=\frac{1}{\sqrt{2}}\,.
$$
\eqref{itlimdificeis1} Como $\sen (a+h)=\sen a\cos h+\sen h\cos a$, temos
$$
\lim_{h\to 0}\frac{\sen (a+h)-\sen a}{h}=\sen a \Bigl(\lim_{h\to 0}\frac{\cos
h-1}{x}\Bigr)+\cos a\Bigl(\lim_{h\to 0}\frac{\sen h}{h}\Bigr)=\cos a\,.
$$
\eqref{itlimdificeis2} Escrevendo 
$$
\frac{x^3-\alpha^3}{\sen (\tfrac{\pi}{\alpha}x)}=
\frac{x^3-\alpha^3}{x-\alpha}\frac{1}{\frac{\sen
(\tfrac{\pi}{\alpha}x)}{x-\alpha}}\,.
$$
Já calculamos $\lim_{x\to \alpha}\frac{x^3-\alpha^3}{x-\alpha}= 3\alpha^2$, e 
chamando $y\pardef \tfrac{\pi}{\alpha}x$ seguido por $y'\pardef y-\pi$,
$$\lim_{x\to \alpha}\frac{\sen(\tfrac{\pi}{\alpha}x)}{x-\alpha}
=\lim_{y\to \pi}\frac{\sen(y)}{\frac{\alpha}{\pi}(y-\pi)}
=\frac{\pi}{\alpha}\lim_{y'\to 0}\frac{\sen(y'+\pi)}{y'}
=-\frac{\pi}{\alpha}\lim_{y'\to 0}\frac{\sen(y')}{y'}=-\frac{\pi}{\alpha}\,.$$
Logo,
$$
\lim_{x\to \alpha}\frac{x^3-\alpha^3}{\sen
(\tfrac{\pi}{\alpha}x)}=(3\alpha^2)/ (-\frac{\pi}{\alpha})=-3\alpha^3/ \pi\,.
$$
\eqref{itlimdificeis3} Comecemos definindo $t$ tal que $\pi-3x=3t$, isto é:
$t\pardef \pisobretres-x$:
$$\lim_{x\to \pisobretres}\frac{1-2\cos
x}{\sen(\pi-3x)}=\lim_{t\to 0}\frac{1-2\cos (\pisobretres-t)}{\sen (3t)}\,.$$
Mas $\cos (\pisobretres-t)=\cos \pisobretres\cos t+\sen \pisobretres\sen
t=\frac12\cos t+\frac{\sqrt{3}}{2}\sen t$,
\begin{align*}
\lim_{t\to 0}\frac{1-2\cos (\pisobretres-t)}{\sen (3t)}&=
\lim_{t\to 0}\frac{1-\cos t}{\sen (3t)}-\sqrt{3}\lim_{t\to 0}\frac{\sen
(t)}{\sen (3t)}\\
&=\lim_{t\to 0}\frac{1-\cos t}{t}\frac{1}{
3\frac{\sen (3t)}{3t}}-
\sqrt{3}\lim_{t\to
0}\frac{\sen (t)}{t}\frac{1}{3\frac{\sen
(3t)}{3t}}=0-\sqrt{3}\frac{1}{3}=-\frac{1}{\sqrt{3}}\,.
\end{align*}
\eqref{itlimdificeis34}
Se $a\geq b$, é melhor escrever $a^x+b^x=a^x(1+(b/a)^x)$, logo
\[ 
\lim_{x\to\infty}\frac{1}{x}\ln(a^x+b^x)
=\ln a+ 
\lim_{x\to\infty}\frac{\ln(1+(b/a)^x)}{x}
=\ln a\,.
\]
O caso $a<b$ se trata da mesma maneira. Obtemos:
\[ 
\lim_{x\to\infty}\frac{1}{x}\ln(a^x+b^x)
=
\begin{cases}
\ln a&\text{ se }a\geq b\,,\\
\ln b&\text{ se }a< b\,.\\
\end{cases}
\]
\eqref{itlimdificeis37}
O caso $n=1$ é trivial: $(x_0+h)^1=x_0+h$. Quando $n=2$, 
$(x_0+h)^2=x_0^2+2x_0h+h^2$, logo (veja o Exemplo
\ref{Ex:derivxissdoisemum})
$$
\lim_{h\to 0}\frac{(x_0+h)^2-x_0^2}{h}=
\lim_{h\to 0}(2x_0+h)=2x_0\,.
$$
Para $n=3,4,\dots$, usaremos a fórmula do binômio de
Newton:
$$(x_0+h)^n=x_0^n+\binom{n}{1}x_0^{n-1}h+\binom{n}{2}x_0^{n-2}
h^2+\dots+\binom{n}{k}x_0^{n-k} h^k+\dots+h^n\,,
$$
onde $\binom{n}{k}=\frac{n!}{(n-k)!k!}$. Portanto,
$$
\frac{(x_0+h)^n-x_0^n}{h}=\binom{n}{1}x_0^{n-1}+\binom{n}{2}x_0^{n-2}
h+\dots+\binom{n}{k}x_0^{n-k} h^{k-1}+\dots+h^{n-1}\,.
$$
Observe que cada termo dessa soma, a partir do segundo, contém
uma potência de $h$. Logo, quando $h\to 0$, só sobra 
o primeiro termo: $\binom{n}{1}x_0^{n-1}=nx_0^{n-1}$. Logo,
\[
\lim_{h\to 0}\frac{(x_0+h)^n-x_0^n}{h}=nx_0^{n-1}\,.
\]
Esse limite será usado para \emph{derivar} polinômios, no próximo capítulo.
\end{sol}
\end{exo}





% !TeX spellcheck = pt_BR
% !TEX encoding = UTF-8 Unicode

\chapter{Continuidade}\label{Cap:Continuidade}

\ifdefined\updateans
% Only need to run once in a lifetime, when the file ans.tex needs to be updated.
\Writetofile{ans}{\protect\section*{Capítulo \ref{Cap:Continuidade}}}
\fi

%\section{Continuidade}\label{Sec:Continuidade}
\index{continuidade}\index{função!contínua}

\emph{Continuidade} é o conceito fundamental da análise. Sem saber, já
nos  deparamos com continuidade
em vários lugares ao longo desse capítulo.

\begin{ex}
No Exemplo \ref{ExemploLimitesimples} estudamos a função $f(x)=\frac{x}{2}+1$
na vizinhança de $a=1$.
Lá, vimos que
\[ 
\lim_{x\to 1}f(x)=\tfrac32\,.
\]
Já tínhamos observado que esse fato parecia óbvio, já que \emph{no ponto}
$a=1$, a função $f$ toma o valor $f(1)=\frac32$. Logo, o que
acontece para essa função no ponto $a=1$ é que
\[ 
\lim_{x\to 1}f(x)=f(1)\,.
\]
Diremos que $f$ é \emph{contínua} em $a=1$.
Em palavras, isso significa que \emph{nos pontos $x$ perto de $1$, a função
toma valores $f(x)$ perto de $f(1)$}. 
Acontece que essa função é contínua em qualquer ponto da reta $a\in \bR$:
\[ \lim_{x\to a}f(x)=f(a)\,.  \]
\end{ex}

Mas essa propriedade não vale para todas as funções. 

\begin{ex}
Considere a seguinte modificação do Exemplo \ref{Ex:funcaodescontinua}:
\[ f(x)=
\begin{cases}
\tfrac{x}{3}+\tfrac12&\text{ se }x>0\,,\\
\tfrac{x}{3}-\tfrac12&\text{ se }x\leq 0\,.\\
\end{cases}
\]
cujo gráfico na vizinhança de $a=0$ é fácil de esboçar:
\begin{center}
\begin{bmlimage}\begin{tikzpicture}[scale=1.5]
\draw [ ->] (0,-1.1)--(0,1.1) node[left]{$\scriptstyle{f(x)}$};
\draw [ ->] (-1.5,0)--(1.5,0)  node[right]{$\scriptstyle{x}$};
\draw [thick, domain=0:1] plot (\x,{0.5+\x/3});
\draw [thick, domain=-1:0] plot (\x,{-0.5+\x/3});
\pgfmathsetmacro{\e}{0.2};
\pgfmathsetmacro{\x}{0.8};
%\draw[dotted] (\x,0) --(\x,{0.5+\x/3})--(0,{0.5+\x/3});
%\draw[thick, <-] (\x-\e,0)--(\x,0);
\pgfmathsetmacro{\x}{-0.8};
%\draw[dotted] (\x,0) --(\x,{-0.5+\x/3})--(0,{-0.5+\x/3});
%\draw[thick, ->] (\x,0)--(\x+\e,0);
\filldraw[intaberto] (0,0.5) circle (0.45mm);
\filldraw[intfechado] (0,-0.5) circle (0.45mm);
\end{tikzpicture}\end{bmlimage}
\end{center}
Aqui temos $f(0)=-\tfrac12$, 
$$
\lim_{x\to 0^-}f(x)=
-\tfrac12\,,\quad \quad 
\text{ e }
\quad
\quad
\lim_{x\to 0^+}f(x)=
+\tfrac12\,.
$$
Logo,
\[ 
\lim_{x\to 0^-}f(x)=f(0)\,,\quad \quad 
\text{ mas }
\quad
\quad
\lim_{x\to 0^+}f(x)\neq f(0)\,.
\]
Diremos que $f$ é \emph{contínua a esquerda em $a=1$}, mas ela \emph{não é}
contínua a direita.
Diz-se que essa função é \emph{descontínua} em $a=0$.\index{descontinuidade}
\end{ex}

%\begin{center}
%\begin{bmlimage}\begin{tikzpicture}
%\draw[ ->] (1.3,0)--(5,0) node[right]{$x$};
%%\draw[ ->] (0,-0.1)--(0,3);
%\pgfmathsetmacro{\a}{3.6};
%\pgfmathsetmacro{\r}{-0.05};
%\pgfmathsetmacro{\m}{2};
%\pgfmathsetmacro{\d}{3};
%\pgfmathsetmacro{\h}{2};
%\draw [thick, dotted, domain=2:\a+0.5] plot
%(\x,{\r*((\m*(\x-\d))^2)+\h}) node[right]{?};
%\draw [thick, domain=1.5:\a] plot (\x,{\r*((\m*(\x-\d))^2)+\h});
%\fill (\a,{\r*((\m*(\a-\d))^2)+\h}) circle (0.45mm);
%\draw[dotted] (\a,0) node[below]{$a$}--(\a,{\r*((\m*(\a-\d))^2)+\h});
%\end{tikzpicture}\end{bmlimage}
%\end{center}

\begin{defin}
Uma função $f$ é 
\begin{enumerate}
\item \grasA{contínua a direita em $a$} se $\lim_{x\to
a^+}f(x)=f(a)$.
\item \grasA{contínua a esquerda em $a$} se $\lim_{x\to
a^-}f(x)=f(a)$.
\end{enumerate}
Se $f$ é ao mesmo tempo contínua a esquerda e a direita em $a$, então
\[
\lim_{x\to a}f(x)=f(a)\,,
\]
e $f$ é dita \grasA{contínua em }$a$. 
Se os limites laterais $\lim_{x\to a^+}f(x)$ $\lim_{x\to a^-}f(x)$ forem
diferentes, ou se eles forem iguais mas diferentes de $f(a)$, 
então $f$ é dita \grasA{descontínua} em $a$.
\end{defin}

Diremos, em geral, que uma função $f$ é \grasA{contínua} se ela é contínua em cada ponto
do seu domínio.

\begin{obs}
Informalmente: $f$ é contínua em $a$ se uma pequena
variação de $x$ em torno de $a$ implica uma pequena variação de
$f(x)$ em torno de $f(a)$. 
Em particular, o gráfico de $f$ não ``dá pulo'' num ponto de
continuidade.
% Observe que para falar de continuidade em $a$,
% $f$ \emph{precisa} ser definida em $a$!
\end{obs}

A maioria das funções fundamentais consideradas até agora são funções 
contínuas. 

\begin{ex}
Qualquer \emph{polinômio} define uma função contínua.
Por exemplo, considere $f(x)=x^2-2x^3$, e $a\in \bR$ um real qualquer. Quando $x$
tende a $a$, então $x^2\to a^2$, e $-2x^3\to -2a^3$. Logo
$f(x)\to f(a)$, portanto $f$ é contínua em $a$. O mesmo raciocínio
pode ser adaptado para qualquer polinômio.
\end{ex}

\begin{ex}
As \emph{funções trigonométricas} são contínuas. 
Por exemplo, por definição do seno e do cosseno via o círculo trigonométrico, parece
claro (e será mostrado analiticamente mais tarde) 
que $\sen x$ e $\cos x$ variam \emph{continuamente} em função de $x$. 
Portanto, sendo um quociente de duas funções contínuas, a tangente é
contínua também (no seu domínio).\\
\end{ex}

\begin{ex}
As funções \emph{exponencial e logaritmo}, $a^x$ e $\log_a(x)$ (em particular, $e^x$ e
$\ln x$), são funções contínuas~\footnote{Apesar de parecer uma afirmação elementar,
provar a
continuidade de $x\mapsto a^x$ implica usar a sua definição precisa. Uma prova pode ser
encontrada nos livros de análise.}.
\end{ex}

\begin{pro}\label{Prop:continuidadechiante}
Se $f$ e $g$ são contínuas em $a$, então $\lambda f$ (onde $\lambda$ é uma constante),
$f+g$, e $f\cdot g$ 
são contínuas em $a$ também. 
Se $g(a)\neq 0$, então $\frac{f}{g}$ é contínua em $a$ também.
Se $g$ é contínua em $a$ e se $f$ é contínua em $g(a)$, então $f\circ g$ é contínua em
$a$.
\end{pro}


\begin{ex}
Considere (lembre o  Exemplo \ref{Ex:funcaodescontinua})
$$f(x)=
\begin{cases}
\tfrac{x}{3}+\frac{x}{2|x|}&\text{ se }x\neq 0\,,\\
\tfrac12&\text{ se }x=0\,.
\end{cases}
$$
Se $a\neq 0$, então $\lim_{x\to a}f(x)=f(a)$, logo $f$ é contínua em  $a\neq 0$. 
Como $\lim_{x\to 0^+}f(x)=\tfrac12=f(0)$, $f$ é contínua a direita em $a=0$. Mas, como
$\lim_{x\to 0^-}f(x)=-\tfrac12\neq f(0)$, $f$ é descontínua em $a=0$.
\end{ex}

\begin{ex}
A função $f$ do Exercício \ref{Ex:semoulediadique} é descontínua em \emph{todo} $a\in
\bR$.
\end{ex}


\begin{exo}
Determine os pontos $a\in \bR$ em que a primeira função $f$ do Exercício
\ref{Ex:semoulediadiquebis} é contínua.
\begin{sol}
Em qualquer ponto $a\neq 0$, os limites laterais nem existem, então
$f$ é descontínua. Por outro lado vimos que $\lim_{x\to
0^+}f(x)=\lim_{x\to 0^- }f(x)=0$. Logo, 
$\lim_{x\to 0}f(x)=f(0)$: $f$ é contínua em $0$.
\end{sol}
\end{exo}

\begin{exo}
Considere $f(x)=
\begin{cases}
x-\frac{x}{|x|}&\text{ se }x\neq 0\\
-1&\text{ se }x=0\,.
\end{cases}$.
Dê o domínio  $D$ de $f$, assim como o conjunto $C$ dos pontos em que $f$ é
contínua.
\begin{sol}
$D=\bR$, $C=\bR_*$.
\end{sol}
\end{exo}

\begin{exo}
Estude a continuidade da função
$$
f(x)\pardef
\begin{cases}
\frac{x^2-3x+2}{x-2}&\text{ se }x\neq 2\,,\\
0&\text{ se }x=2\,.
\end{cases}
$$
Como que $f$ pode ser modificada para se tornar contínua na reta toda?
\begin{sol}
Considere um $a\neq 2$. $f$ sendo uma razão de polinómios, e como o denumerador não se
anula em $a$, a Proposição
\ref{Prop:continuidadechiante} implica que $f$ é contínua em $a$.
Na verdade, quando $x\neq 2$, $f(x)=\frac{x^2-3x+2}{x-2}=\frac{(x-1)(x-2)}{x-2}=x-1$.
Logo, $\lim_{x\to 2}f(x)=\lim_{x\to 2}(x-1)=1$. Como $1\neq f(2)=0$, $f$ é descontínua em
$2$.
Para tornar $f$ contínua na reta toda, é so redefiní-la em $x=2$, da seguinte maneira:
$$
\tilde{f}(x)\pardef
\begin{cases}
\frac{x^2-3x+2}{x-2}&\text{ se }x\neq 2\,,\\
1&\text{ se }x=2\,.
\end{cases}
$$
Agora, $\tilde{f}(x)=x-1$ para todo $x\in \bR$.
\end{sol}
\end{exo}

\begin{exo}
Ache o valor da constante $a$ tal que a seguinte função seja contínua em todo
$x\in \bR$:
$$f(x)\pardef
\begin{cases}
\frac{x^2-(a+1)x+a}{x-1}&\text{ se }x\neq 1\,,\\
5+a&\text{ se }x=1\,.
\end{cases}
$$
\begin{sol}
Como $\lim_{x\to 1}f(x)=1-a$ e que $f(1)=5+a$, é preciso ter $1-a=5+a$, o que implica
$a=-2$.
\end{sol}
\end{exo}



\begin{exo}
%(Courant and John)
Estude a continuidade das funções 
$$
f(x)\pardef
\begin{cases}
\tanh \tfrac1x &\text{ se } x\neq 0\,,\\
0  & \text{ se } x=0\,,
\end{cases}
\quad\quad
g(x)\pardef
\begin{cases}
x\tanh \tfrac1x &\text{ se } x\neq 0\,,\\
0  & \text{ se } x=0\,.
\end{cases}
$$
\begin{sol}
Por um lado, como $\tanh \tfrac1x$ é a composição de duas funções contínuas, ela é
contínua em todo $a\neq 0$.
Um raciocínio parecido implica que $g$ é contínua em todo $a\neq 0$.
Por outro lado, 
vimos no item \eqref{itmudvarlim4} do Exercício \ref{Exo:mudvarlimites} que $\lim_{x\to
0^{\pm}}\tanh \frac{1}{x}=\pm 1$, o que implica que $f$ é descontínua em $a=0$.
Vimos no item \eqref{itmudvarlim5} do mesmo exercício que $\lim_{x\to
0^{\pm}}x\tanh \frac{1}{x}=0$, logo $\lim_{x\to 0}g(x)$ existe e vale $g(0)$.
Logo, $g$ é contínua em $a=0$.
\begin{center}
\begin{bmlimage}\begin{tikzpicture}
\pgfmathsetmacro{\a}{3};
\pgfmathsetmacro{\e}{0.12};
\draw[ ->] (-\a,0)--(\a,0)node[right]{$x$};
\draw[ ->] (0,-1.3)--(0,1.3)node[left]{$\tanh\frac1x$};
\draw[->, thick, domain=-\a:-\e] plot (\x,
{(exp(1/\x)-exp(-1/\x))/(exp(1/\x)+exp(-1/\x))});
\draw[<-, thick, domain=\e:\a] plot (\x, {(exp(1/\x)-exp(-1/\x))/(exp(1/\x)+exp(-1/\x))});
\fill (0,0) circle (0.50mm);

\begin{scope}[xshift=7cm]
\draw[ ->] (-\a,0)--(\a,0)node[right]{$x$};
\draw[ ->] (0,-1.3)--(0,1.3)node[left]{$x\tanh\frac1x$};
\draw[->, thick, domain=-\a:-\e] plot (\x,
{\x*(exp(1/\x)-exp(-1/\x))/(exp(1/\x)+exp(-1/\x))});
\draw[<-, thick, domain=\e:\a] plot (\x,
{\x*(exp(1/\x)-exp(-1/\x))/(exp(1/\x)+exp(-1/\x))});
\fill (0,0) circle (0.50mm);
\end{scope}
\end{tikzpicture}\end{bmlimage}
\end{center}
\end{sol}
\end{exo}


\section{O Teorema do valor intermediário}
\index{Teorema!do valor intermediário}
Funções contínuas possuem propriedades muito particulares.
Considere por exemplo
uma função contínua num intervalo fechado, $f:[a,b]\to \bR$. 
Então, ao $x$ variar entre $a$ e $b$, \emph{o gráfico de $f$ corta 
qualquer reta horizontal
intermediária, de altura $h$ entre $f(a)$ e $f(b)$, pelo menos uma
vez}:

\begin{center}
\begin{bmlimage}\begin{tikzpicture}
\pgfmathsetmacro{\taille}{6};
\pgfmathsetmacro{\N}{200};
\pgfmathsetmacro{\incrx}{\taille/\N};
\pgfmathsetmacro{\incry}{3/sqrt(\N)};
\pgfmathsetmacro{\a}{1};
\pgfmathsetmacro{\b}{6};


\pgfmathsetmacro{\m}{-40};
\pgfmathsetmacro{\n}{20};
\pgfmathsetmacro{\o}{30};
\pgfmathsetmacro{\p}{65};
\pgfmathsetmacro{\q}{18};
\pgfmathsetmacro{\r}{-10};

\coordinate (A) at (1,0.5);
\coordinate (B) at (6,3);
\coordinate (C) at (4,2);

\draw[dotted] (1,0)node[below]{$a$}--(A)--(0,0.5)
node[left]{$f(a)$};
\draw[dotted] (6,0)node[below]{$b$}--(B)--(0,3)
node[left]{$f(a)$};
\draw (0,2)node[left]{$h$} -- (6,2);
\draw[dashed] (4,0)node[below]{$c$} -- (C);

\fill (A) circle (0.50mm);
\fill (B) circle (0.50mm);
\fill (C) circle (0.50mm);

\draw[thick] 
(A) to[out=\m, in={180+\n}] 
(2,0.2) to[out=\n, in={180+\o}] 
(3,1.2) to[out=\o, in={180+\p}] 
(C) to[out=\p, in={180+\q}] 
(5,3.7) to[out=\q, in={180+\r}] 
(B);


%\draw (A) .. controls (-1,-1) .. (B);
%\draw (A) .. controls (1,1) .. (B);
%\draw (0,0) .. controls (1,1) .. (4,0)
%      (5,0) .. controls (6,0) and (6,1) .. (5,2);
%\coordinate (origin) at (\x,\y);
%\coordinate (current) at (origin);

%\foreach \k in {1,...,\N} {
%\draw[thick] (current)-- ++ (\incrx,{(2*rnd-0.91)*\incry}) coordinate (current); 
%\pgfmathtruncatemacro{\NN}{\N/2};
%\ifnum \k=\NN  
%\coordinate (currentVV) at ($(0,-10)!(current)!(0,10)$);
%\coordinate (currentVVV) at ($({\x+\taille},-10)!(current)!({\x+\taille},10)$);
%\coordinate (currentHH) at ($(0,0)!(current)!(2*\taille,0)$);
%\draw[dashed] (currentVV)node[left]{$h$}--(currentVVV);
%\draw[dotted] (current)--(currentHH)node[below]{$c$};
%\fill (current) circle (0.60mm);
%\fi
%}
%
%\coordinate (currentV) at ($(\x,-10)!(current)!(\x,10)$);
%\coordinate (currentVV) at ($(0,-10)!(current)!(0,10)$);
%\coordinate (currentH) at ($(origin)!(current)!(2*\taille,\y)$);
%
%\fill (origin) circle (0.50mm);
%\fill (current) circle (0.50mm);
%
%\draw[dotted] (\x,0)node[below]{$a$}--(origin);
%\draw[dotted] (\x+\taille,0)node[below]{$b$}--(current);
%\draw[dashed] (0,\y)node[left]{$\scriptstyle{f(a)}$}--({\x+\taille},\y);
%\draw[dashed] (currentVV)node[left]{$\scriptstyle{f(b)}$}--(current);
\draw[ ->] (-0.5,0)--(\taille+2,0)node[right]{$x$};
\draw[ ->] (0,-0.5)--(0,3.5) node[right]{$f(x)$};

\end{tikzpicture}\end{bmlimage}
\end{center}

\begin{teo}[Teorema do Valor Intermediário]\label{Teo:ValInterm}
Seja  $f:[a,b]\to \bR$ uma função contínua, tal que $f(a)<f(b)$. 
Então para todo $h\in [f(a),f(b)]$, existe $c\in [a,b]$ tal que
$f(c)=h$. Uma afirmação parecida vale quando $f(a)>f(b)$
\end{teo}

\begin{exo}
Para cada função abaixo, estude a propriedade do valor
intermediário (isto é, fixe uma reta de altura $h$ e vê se o gráfico de $f$
corta a reta).
%\begin{multicols}{2}
\begin{enumerate}
\item $f:[-1,2]\to \bR$, $f(x)\pardef x^2$.
\item $g:[-1,1]\to \bR$, 
$g(x)\pardef
\begin{cases}
\frac{|x|}{x}&\text{ se }x\neq 0\,,\\
0&\text{ se }x=0\,.\\
\end{cases}
$
\item $h:[0,2]\to \bR$, 
$h(x)\pardef
\begin{cases}
2x-1&\text{ se }0\leq x<1\,,\\
2x-3&\text{ se }1\leq x\leq 2\,.\\
\end{cases}
$
\end{enumerate}
%\end{multicols}
\begin{sol}
(Esboçar os gráficos de $f,g,h$ ajuda a compreensão do exercício).

Temos $f(-1)=1$, $f(2)=4$.
Como $f$ é contínua, o Teorema \eqref{Teo:ValInterm} se aplica:
se $1\leq h\leq 4$, o gráfico de $f$ corta a reta horizontal de
altura $y=h$ pelo menos uma vez. Na verdade, ele corta a reta
exatamente uma vez se $1<h\leq 4$, e duas vezes se $h=1$.

Temos $g(-1)=-1$, $g(1)=1$. 
Como $g$ é descontínua em $x=0$, o teorema não se aplica. Por
exemplo, o gráfico de $g$ nunca corta a reta horizontal $y=\frac12$.

Temos $h(0)=-1$, $h(2)=1$. Apesar de $h$ não ser contínua, ela
satisfaz à propriedade do valor intermediário. De fato, o gráfico de
$h$ corta a reta $y=h_*$ duas vezes se $-1\leq h_*<1$, e uma vez se $h_*=1$.
\end{sol}
\end{exo}

O Teorema do valor intermediário pode ser usado para 
a resolução numérica de equações:

\index{resolução numérica}
\begin{ex}
Considere a função 
$f(x)\pardef \tfrac12-x^2-x^5$,
no intervalo $[-1,1]$. Como $f$ é contínua e muda de
sinal entre $-1$ e $+1$,
$f(-1)=\frac12>0$, 
$f(+1)=-\frac32<0$, 
o Teorema do Valor Intermediário implica que 
deve existir pelo menos um ponto $x_*\in [-1,1]$ tal que $f(x^*)=0$.
\begin{center}
\begin{bmlimage}\begin{tikzpicture}[scale=1.7]
%\draw[very thin, gray] (0,0) grid[step=0.1] (5,2);
\newcommand{\funcao}[1]{0.5-(#1)^2-(#1)^5}
\draw[ ->] (-1.5,0)--(1.5,0) node[right]{$x$};
\draw[ ->] (0,-1.2)--(0,1.3) node[right]{$f(x)$};
\draw[thick, domain=-1:1, samples=100] plot (\x,{\funcao{\x}});
\draw[dotted] (-1,0)node[below]{$-1$}--(-1,{\funcao{-1}});
\fill (-1,{\funcao{-1}}) circle (0.45mm);
\draw[dotted] (1,0)node[above]{$+1$}--(1,{\funcao{1}});
\fill (1,{\funcao{1}}) circle (0.45mm);
\coordinate (a) at (0.622,0);
\fill (a) circle (0.45mm);
\draw (a) node[below left]{$x_*$};
\end{tikzpicture}\end{bmlimage}
\end{center}

Como calcular $x_*$? Por definição, $x_*\in [-1,1]$ é solução da
equação do quinto grau:
$$x^5+x^2-\tfrac12=0\,,$$
Como não existe um método geral para a resolução de tais equações, vejamos um
método que, sem ser \emph{exato}, fornece pelo menos uma \emph{aproximação}
de $x_*$.\\

A ideia é de localizar $x_*$ usando recursivamente o Teorema do Valor intermediário. 
Para começar, observemos que como $f(0)>0$, $f(1)<0$, 
$f$ muda de sinal também no intervalo $[0,1]$, o que implica que 
$x_*\in [0,1]$.\\

Calculemos então o valor de $f$ no meio do
intervalo $[0,1]$ e observemos que $f(\frac12)>0$. Portanto, $f$ muda
de sinal entre $\frac12$ e $1$, o que implica que $x_*\in
[\frac12,1]$. 
Em seguida, $f(\frac34)<0$ implica que $f$ muda de sinal entre
$\frac12$ e $\frac34$, isto é, $x_*\in [\frac12,\frac34]$. Continuando
assim, obtemos uma sequência decrescente de intervalos encaixados,
cada um contendo $x_*$:
$$[0,1]\supset [\tfrac12,1]\supset [\tfrac12,\tfrac34]\supset \cdots
$$
Os tamanhos dos intervalos decrescem exponencialmente rápido: o
primeiro tem tamanho $1$, o segundo tamanho $\frac12$, etc., o 
$n$-ésimo tem tamanho $2^{-n}$.
Logo, qualquer ponto do $n$-ésimo intervalo dá uma aproximação 
de $x_*$ com uma precisão de $2^{-n}$.\\

O método descrito acima, que consiste em usar o Teorema do Valor
intermediário a cada etapa, é chamado de \emph{método da bisseção}.

\end{ex}




\section{Limites e funções contínuas}\label{Sec:FuncConteLim}
\index{limites!de funções contínuas}
%Uma função $f$ ser contínua implica que se $x\to a$, então $f(x)\to f(a)$.
Como visto na Proposição \ref{Prop:continuidadechiante}, se 
$g$ é contínua em $a$, e se $f$ é contínua em $g(a)$, então $f\circ g$ é
contínua em $a$. Isso pode ser dito da seguinte maneira: \emph{se $g(x)\to L$
quando $x\to a$ e se $f$ é contínua em $L$, então $f(g(x))\to f(L)$ quando
$x\to a$.} Isto é,
$$\lim_{x\to a}f(g(x))= f\bigl(\lim_{x\to a}g(x)\bigr)\,.$$
Esse fato foi usado, sem sequer ser mencionado, em vários lugares nas seções
anteriores. Por exemplo apareceu, no item \eqref{itexosinxx5}
do Exercício \ref{Exo:variantessinxsurx}, o limite de $(\frac{\sen
x}{x})^2$ quando $x\to 0$. Aqui, a função é da forma $f(g(x))$, com
$g(x)=\frac{\sen x}{x}$, $f(x)=x^2$. Ora, como $g(x)\to 1$ e como $f$ é
contínua em $1$, podemos ``entrar o limite dentro do $(\cdot)^2$'':
$$
\lim_{x\to 0}\Bigl(
\frac{\sen x}{x}
\Bigr)^2=\Bigl(\lim_{x\to 0}
\frac{\sen x}{x}
\Bigr)^2=(1)^2=1\,.
$$
Também, no item \eqref{itexolimelem20} do Exercício
\ref{Exo:Limiteselementares}, como $\sqrt{x}$ é contínua a direita em $0$
$$
\lim_{x\to 1^+}\sqrt{\ln x}=\sqrt{\lim_{x\to 1^+}{\ln x}}=\sqrt{0}=0\,.
$$
Um resultado parecido vale para limites no infinito:  \emph{se $g(x)\to L$
quando $x\to \infty$ e se $f$ é contínua em $L$, então $f(g(x))\to f(L)$
quando $x\to \infty$.} Em outras palavras:
$$\lim_{x\to \infty}f(g(x))= f\bigl(\lim_{x\to \infty}g(x)\bigr)\,.$$
Por exemplo, em \eqref{eq:usoimpliccontinlog}, 
\eq{\label{eq:usoimpliccontinlogBIS}
\lim_{z\to+\infty}\ln\bigl((1+\tfrac{1}{z})^z\bigr)
=\ln\Bigl(\lim_{z\to+\infty}(1+\tfrac{1}{z})^z\Bigr)
=\ln e=1\,.
}

%\newpage
\section{Exercícios de revisão}

\begin{exo}
Considere 
$$
f(x)\pardef
\begin{cases}
\sqrt{x^4+1}-(ax^2+b)+\frac{1-\cos (cx)}{x^2} &\text{ se }x\neq 0 \,,\\
0  & \text{ se }x=0\,.
\end{cases}
$$
Ache $a,b,c$ de modo tal que $f$ seja contínua em $0$, e que $\lim_{x\to \infty}=-3$.
\begin{sol}
$a=1$, $b=3$, $c=\pm 2$.
\end{sol}
\end{exo}


\begin{exo}
Seja $f:\bR\to \bR$ contínua tal que $\lim_{x\to +\infty}f(x)=+\infty$,
$\lim_{x\to -\infty}f(x)=-\infty$. Mostre que $\imagem(f)=\bR$.
\begin{sol}
Seja $y\in \bR$ fixo, qualquer. Como $\lim_{x\to
+\infty}f(x)=+\infty$, existe $b>0$ grande o suficiente tal que
$f(b)>y$. 
Como $\lim_{x\to -\infty}f(x)=-\infty$, existe $a<0$ grande o
suficiente tal que $f(a)<y$. 
Pelo Teorema do Valor Intermediário, existe $c\in [a,b]$ tal que
$f(c)=y$. Isto implica que $y\in \imagem(f)$.
\end{sol}
\end{exo}


\begin{exo}
Se $f$ é par (respectivamente ímpar), qual é a relação entre 
$\lim_{x\to 0^+}f(x)$ e $\lim_{x\to 0^-}f(x)$?
Seja $f$ uma função ímpar tal que $\lim_{x\to 0^+}f(x)$ existe e vale $L>0$.
Essa função é contínua?
\begin{sol}
Considere $\lim_{x\to 0^-}f(x)$. Chamando $y\pardef -x$, $x\to 0^-$ corresponde
a $y\to 0^+$.
Logo, $$\lim_{x\to 0^-}f(x)=\lim_{y\to 0^+}f(-y)=-\lim_{y\to 0^+}f(y)\equiv-
\lim_{x\to 0^+}f(x)\,.$$
Portanto, para uma função ímpar ser contínua em $0$, é preciso ter 
$\lim_{x\to 0^+}f(x)=f(0)=0$ (não pode ser $L>0$).
\end{sol}
\end{exo}

\begin{exo} (Aplicação do Teorema do valor Intermediário).
Se $I$ é um intervalo e se $f:I\to \bR$ é contínua, então
$\imagem(f)$ é um
intervalo.
\end{exo}



\begin{exo}
Estude a continuidade das seguintes funções:
$$
f(x)\pardef 
\begin{cases}
 e^{\arctan 1/x}&\text{se }x\neq 0\,,\\
e^{\frac{\pi}{2}}&\text{ se }x=0\,,
\end{cases}
\quad\quad
g(x)\pardef 
\begin{cases}
 e^{\frac{x}{x^2-1}}&\text{se }x\not\in \{\pm 1\}\,,\\
0&\text{se }x=-1\,,\\
1&\text{se }x=+1\,.
\end{cases}
$$
% \begin{sol}
% Como $\frac{1}{x}$ é contínua em qualquer $a\neq 0$, que $\arctan x$ é
%contínua 
% em $\frac{1}{a}$, e que $e^x$ é contínua em $\arctan \frac1a$, temos
%que 
% $f(x)$ é contínua em qualquer $a\neq 0$. Em $a=0$ temos:
% $$
% \lim_{x\to 0^\pm }f(x)=\lim_{x\to 0^\pm}e^{\arctan 1/x}
% =e^{\lim_{x\to 0^\pm}\arctan 1/x}=e^{\arctan(\lim_{x\to 0^\pm}
% 1/x)}=e^{\pm\frac{\pi}{2}}\,.
% $$
% Logo $f$ é contínua a direita (mas descontínua) em $0$.
% Por outro lado, $g$ é contínua em qualquer $x\neq \pm 1$, e descontínua em
% $x=\pm 1$.
% \end{sol}
\end{exo}

\begin{exo}
Sejam $f,g$ duas funções contínuas na reta, tais que $f(x)=g(x)$ para todo
racional diádico $x$. Mostre que $f=g$.
\end{exo}





% !TeX spellcheck = pt_BR
% !TEX encoding = UTF-8 Unicode

\chapter{Derivada}\label{Cap:Derivacao}
  
\ifdefined\updateans
% Only need to run once in a lifetime, when the file ans.tex needs to be updated.
\Writetofile{ans}{\protect\section*{Capítulo \ref{Cap:Derivacao}}}
\fi

A \emph{derivada} será o nosso principal uso da noção de limite.
Veremos primeiro, na Seção \ref{Sec:RetasGraf}, como ela aparece 
naturalmente na
procura da equação da reta tangente a um gráfico. 
Em seguida, a derivada será
tratada como uma nova função e as suas propriedades serão descritas.
Estudaremos a \emph{segunda derivada} e o seu sentido geométrico na Seção 
\ref{Sec:Segundaderivada}.
Mais tarde abordaremos o estudo de problemas concretos de otimização
no Capítulo~\ref{cap:MineMax}, e no Capítulo~\ref{cap:Estudos},
derivada e derivada segunda serão
usadas para estudos detalhados de funções.

\section{Retas e gráficos de funções}\label{Sec:RetasGraf}

Para começar, consideraremos retas do plano
associadas \emph{ao gráfico de uma função}.
Isto é, escolheremos um ponto \emph{fixo} $P$, um ponto \emph{móvel} $Q$, e 
consideraremos a inclinação da reta que passa por $P$ e $Q$.
Será interessante estudar como que essa inclinação evolui em função da 
posição de $Q$, quando $Q$ se mexe ao longo do gráfico de uma função.

\begin{ex}
Considere o ponto {fixo} $P=(0,-1)$ e a reta horizontal $r$ de
equação $y=1$. Consideremos agora um ponto móvel $Q$ em $r$. Isto é, $Q$ 
é da forma
$Q=(\lambda,1)$, onde $\lambda$ varia em $\bR$, e estudemos
\emph{a inclinação da reta passando por $P$ e $Q$}, dada por 
$$m(\lambda)=\frac{1-(-1)}{\lambda-0}=\frac{2}{\lambda}\,.$$
\begin{center}
\begin{bmlimage}\begin{tikzpicture}
\draw[ ->] (-3,0)--(3,0);
\draw[ ->] (0,-1.5)--(0,1.5);
\draw[thick] (-2.8,1)--(2.8,1);
\coordinate (P) at (0,-1);
\pgfmathsetmacro{\l}{1.5};
\coordinate (Q) at (\l,1);
\draw[dashed, domain=-0.3:\l+0.5] plot (\x,{2*\x/\l-1})
node[right]{$\leftarrow$ inclinação: $m(\lambda)$};
\fill (P) circle (0.50mm);
\fill (Q) circle (0.50mm);
\draw[dotted] (\l,0)node[below]{$\lambda$}--(Q);
\draw (P) node[left]{$P$};
\draw (Q) node[above]{$Q$};
\end{tikzpicture}\end{bmlimage}
\end{center}
Vemos que quando $Q$ pertence ao primeiro quadrante ($\lambda>0$),
$m(\lambda)$ é positiva, e quando $Q$ pertence ao segundo quadrante
($\lambda<0$), $m(\lambda)$ é negativa.
Observemos também que a
medida que $Q$ se afasta pela direita ou pela esquerda, a reta tende a ficar
mais
horizontal. Em termos da sua inclinação: 
$$\lim_{\lambda\to-\infty}m(\lambda)=0\,,\quad
\quad\lim_{\lambda\to +\infty}m(\lambda)=0\,.$$
Por outro lado, quando $Q$ se aproximar de $(0,1)$, a reta se aproxima
de uma vertical, e a sua inclinação toma valores
arbitrariamente grandes:
$$\lim_{\lambda\to 0^-}m(\lambda)=-\infty\,,\quad \quad\lim_{\lambda\to
0^+}m(\lambda)=+\infty\,.$$
\end{ex}


\begin{ex}
Considere agora o ponto fixo $P=(-1,0)$ e um
ponto móvel $Q$ no gráfico da função $f(x)=\frac1x$, contido no primeiro
quadrante. Isto é,
$Q$ é da forma $Q=(\lambda,\frac{1}{\lambda})$, com $\lambda>0$. Como no
exemplo anterior, 
estudemos {a inclinação da reta passando por $P$ e $Q$},
dada por 
$$m(\lambda)=\frac{\frac{1}{\lambda}-0}{\lambda-(-1)}=\frac{1}{
\lambda(\lambda+1)}\,.$$
\begin{center}
\begin{bmlimage}\begin{tikzpicture}
\draw[ ->] (-1.5,0)--(3,0);
\draw[ ->] (0,-0.5)--(0,2.5);
\draw[thick, domain=0.4:3] plot (\x,{1/\x});
\coordinate (P) at (-1,0);
\pgfmathsetmacro{\l}{1.5};
\coordinate (Q) at (\l,{1/\l});
\draw[dashed, domain=-1.6:\l+0.6] plot (\x,{(\x+1)/(\l*(\l+1))});
\fill (P) circle (0.50mm);
\fill (Q) circle (0.50mm);
\draw[dotted] (\l,0)node[below]{$\lambda$}--(Q);
\draw (P) node[above]{$P$}; 
\draw (Q) node[above]{$Q$};
\end{tikzpicture}\end{bmlimage}
\end{center}
Aqui vemos que 
$$\lim_{\lambda\to 0^+}m(\lambda)=+\infty\,,\quad \quad\lim_{\lambda\to
+\infty}m(\lambda)=0\,.$$
\end{ex}

Finalmente, consideremos um exemplo em que \emph{ambos} pontos pertencem ao
gráfico de uma mesma função.

\begin{ex}\label{Ex:primeiraretatangente}
Considere a parábola\index{parábola}, gráfico da função $f(x)=x^2$. Consideremos , de novo, um
ponto fixo nessa parábola, $P=(-1,1)$, e um ponto móvel $Q=(\lambda,\lambda^2)$.
\begin{center}
\begin{bmlimage}\begin{tikzpicture}
\newcommand{\funcao}[1]{(#1)^2}
\draw[ ->] (-3,0)--(3,0);
\draw[ ->] (0,-0.5)--(0,2.5);
\draw[thick, domain=-1.5:1.5] plot (\x,{\funcao{\x}});
\coordinate (P) at (-1,1);
\pgfmathsetmacro{\l}{1.4};
\coordinate (Q) at (\l,{\funcao{\l}});
\draw[dashed, domain=-1.6:\l+0.4] plot (\x,{(\l-1)*\x+\l});
\fill (P) circle (0.50mm);
\fill (Q) circle (0.50mm);
\draw[dotted] (\l,0)node[below]{$\lambda$}--(Q);
\draw (P) node[above right]{$P$};
\draw[dotted] (P)--(-1,0) node[below]{$-1$};
\draw (Q) node[above left]{$Q$};
\end{tikzpicture}\end{bmlimage}
\end{center}
Aqui, 
$$
m(\lambda)=\frac{\lambda^2-1}{\lambda-(-1)}=\frac{\lambda^2-1}{\lambda+1}\,.
$$
Quando $Q$ se afasta de $P$,
$$
\lim_{\lambda\to -\infty}m(\lambda)=-\infty\,,\quad \quad\lim_{\lambda\to
+\infty}m(\lambda)=+\infty\,.
$$
Vejamos agora algo mais interessante: \emph{o que acontece quando $Q$
se aproxima arbitrariamente perto de $P$, isto é, quando $\lambda\to -1$?}
\begin{center}
\begin{bmlimage}\begin{tikzpicture}[scale=1.5]
\newcommand{\funcao}[1]{(#1)^2}
\draw[ ->] (-3,0)--(3,0);
\draw[ ->] (0,-0.5)--(0,2.3);
\draw[ domain=-1.5:1.5] plot (\x,{\funcao{\x}});
\coordinate (P) at (-1,1);
\foreach \l in {0.3,0,-0.3,-0.7} {
\coordinate (Q) at (\l,{\funcao{\l}});
\draw[dashed, domain=-1.6:\l+0.4] plot (\x,{(\l-1)*\x+\l});
\fill (Q) circle (0.40mm);
}
\pgfmathsetmacro{\l}{-1+0.01};
\draw[thick, domain=\l+0.4:-1.6] plot (\x,{(\l-1)*\x+\l})
node[left]{$r_t^P\,\rightarrow$};
\draw (-1,1) node[below left]{$P$};
\fill (P) circle (0.45mm);
\end{tikzpicture}\end{bmlimage}
\end{center}
Vemos que a medida que $Q$ se aproxima de $P$, a reta $r$ se aproxima da
\grasA{reta tangente à parábola no ponto $P$}, denotada $r_t^P$.
Em particular, a inclinação de $r_t^P$ pode ser calculada pelo limite 
$$
m_t^P=\lim_{\lambda\to -1}m(\lambda)=\lim_{\lambda\to
-1}\frac{\lambda^2-1}{\lambda+1}\,.$$
Esse limite é indeterminado, da forma ``$\tfrac00$'', mas pode ser calculado:
$$\lim_{\lambda\to
-1}\frac{\lambda^2-1}{\lambda+1}=\lim_{\lambda\to
-1}\frac{(\lambda-1)(\lambda+1)}{\lambda+1}
=\lim_{\lambda\to -1}(\lambda-1)=-2\,.
$$
\index{reta!tangente}
Portanto, a equação da reta tangente $r_t^P$ é da forma $y=-2x+h$, e a ordenada
na origem pode ser calculada usando o fato de $r_t^P$ passar por $P$. Obtém-se:
\begin{center}
\begin{bmlimage}\begin{tikzpicture}[scale=1.5]
\newcommand{\funcao}[1]{(#1)^2}
\draw[ ->] (-3,0)--(3,0);
\draw[ ->] (0,-0.5)--(0,2.3);
\draw[thick, domain=-1.5:1.5] plot (\x,{\funcao{\x}});
\coordinate (P) at (-1,1);
\pgfmathsetmacro{\l}{-1+0.01};
\draw[thick,  domain=\l+0.4:-1.6] plot (\x,{(\l-1)*\x+\l})
node[below left]{$r_t^P:\, y=-2x-1$};
\draw (-1,1) node[below left]{$P$};
\fill (P) circle (0.50mm);
\draw[dotted] (-1,0)node[below]{$-1$}--(-1,1);
\draw[dotted] (0,1)node[right]{$1$}--(-1,1);
\end{tikzpicture}\end{bmlimage}
\end{center}
\end{ex}

Na verdade, a mesma conta permite calcular a inclinação da reta tangente a
qualquer ponto do gráfico:
\begin{exo}\label{Exo:retatangxisdois}
Considere um ponto $P$ da parábola, cuja primeira coordenada é um número
$a\in\bR$ qualquer, fixo.
Escolha um ponto $Q$ da parábola (com primeira coordenada $\lambda$), e
calcule
a equação da reta $r$ que passa por $P$ e $Q$.
Estude o que acontece com a equação dessa reta quando $\lambda\to a$?
\begin{sol}
Se $P=(a,a^2)$, $Q=(\lambda,\lambda^2)$, a equação da reta $r^{PQ}$ é dada por
$y=(\lambda+a)x-a\lambda$. Quando $\lambda\to a$ obtemos 
a equação da reta tangente à parábola em $P$: $y=2a x-a^2$.
Por exemplo, se $a=0$, a equação da reta tangente é $y=0$, se $a=2$, é
$y=4x-4$, 
$a=-1$, é $y=-2x-1$ (o que foi calculado no Exemplo
\ref{Ex:primeiraretatangente}).
\end{sol}
\end{exo}

\section{Reta tangente e derivada}
\index{reta!tangente}
O procedimento descrito no Exemplo \ref{Ex:primeiraretatangente} acima pode ser
generalizado, e fornece um método para calcular a reta tangente ao gráfico de
uma função $f$ num ponto $P=(a,f(a))$.
Escolhamos
um ponto vizinho de $P$, também no gráfico de $f$, denotado $Q=(x,f(x))$, e
consideremos a reta $r$ que passa por $P$ e $Q$.

\begin{center}
\begin{bmlimage}\begin{tikzpicture}[scale=1.5]
\newcommand{\funcao}[1]{(#1)^2/4+0.4}
\newcommand{\dfuncao}[2]{ (\funcao{#1+#2})/(#2)-(\funcao{#1})/(#2)}
\draw[ ->] (0,-0.2)--(0,1.5)node[left]{$f(x)$};
 \draw[ ->] (-0.5,0)--(3,0);
 \draw[thick, domain=-0.5:2.7] plot (\x,{\funcao{\x}});
 
 \pgfmathsetmacro{\a}{0.5};
 \coordinate (P) at (\a,{\funcao{\a}});
 \pgfmathsetmacro{\l}{2.5};
 \coordinate (Q) at (\l,{\funcao{\l}});
\draw[->, very thick, domain=\l+0.1:\l-0.2] plot (\x,{\funcao{\x}-0.2});
\draw[thick,  domain={\a-0.4}:{\l+0.4}] plot
(\x,{(\dfuncao{\a}{0.03})*(\x-\a)+\funcao{\a}});
\draw[dashed, domain={\a-0.4}:{\l+0.6}] plot
(\x,{(\dfuncao{\a}{(\l-\a)})*(\x-\a)+\funcao{\a}}) node[above
right]{$r$};
\draw (P) node[above]{$P$};
\fill (P) circle (0.50mm);
\draw (Q) node[above]{$Q$};
\fill (Q) circle (0.50mm);
\draw[dotted] (\a,0)node[below]{$a$}--(\a,{\funcao{\a}});
\draw[dotted] (\l,0)node[below]{$x$}--(\l,{\funcao{\l}});
\draw[dotted] 
(\a,{\funcao{\a}})--({\l+0.4},{\funcao{\a}})node[right]{$f(a)$};
 \draw[dotted] 
(\l,{\funcao{\l}})--({\l+0.4},{\funcao{\l}})node[right]{$f(x)$};
 \end{tikzpicture}\end{bmlimage}
\end{center}
A inclinação da reta $r$ é dada por
$$\frac{f(x)-f(a)}{x-a}\,,$$
e a inclinação da reta tangente em $P$ é obtida pegando $Q\to P$,
isto é, $x\to a$.


\begin{defin}
Considere uma função $f$ definida num ponto $a$ e na sua vizinhança.
Se o limite 
\eq{\boxed{
f'(a)\pardef \lim_{x\to a}\frac{f(x)-f(a)}{x-a}\,,}}
existir e for finito, diremos que \index{função ! derivável num ponto}
$f$ \grasA{é derivável (ou diferenciável) em $a$}. O valor de $f'(a)$
é chamado de \grasA{derivada de $f$ no ponto $a$}, e 
\index{diferenciabilidade}\index{reta!inclinação de}
representa a \grasA{inclinação da reta tangente ao
gráfico de $f$ no ponto $P=(a,f(a))$}.
\end{defin}

%\begin{obs}
\begin{wrapfigure}{r}{5cm}
\begin{center}
\begin{bmlimage}\begin{tikzpicture}
\newcommand{\funcao}[1]{2-(#1-2)^2/3}
\draw[ ->] (-0.4,0)--(2,0) node[right]{$x$};
\draw[ ->] (-0.3,-0.2)--(-0.3,1.9) node[left]{$f(x)$};
\draw[thick, domain=0:2.3] plot (\x,{\funcao{\x}});
\coordinate (P) at (0.3,{\funcao{0.3}});
\coordinate (Q) at (1.8,{\funcao{1.8}});
\draw[dashed] (P)--(1.8,{\funcao{0.3}}) node[midway, below]{$\Delta
x$}--(Q)node[midway, right]{$\Delta f$};
\fill (P) circle (0.45mm);
\fill (Q) circle (0.45mm);
\draw (0.3,0) node[below]{$a$}--(P);
\end{tikzpicture}\end{bmlimage}
\end{center}
%\vspace{-1cm}
\end{wrapfigure}
Veremos mais tarde que a derivada deve ser interpretada como \emph{taxa local de
\index{taxa!de variação}
crescimento da função}: $f'(a)$ dá a taxa com a qual $f(x)$ cresce em relação a $x$, na
vizinhança de $a$. Considerando o gráfico na forma de uma curva $y=f(x)$, e
chamando $\Delta x\pardef x-a$ e $\Delta f\pardef
f(x)-f(a)$, vemos que uma notação natural para a derivada, bastante
usada na literatura é:
$${\frac{df}{dx}=\lim_{\Delta x\to
0}\frac{\Delta f}{\Delta x}}
$$
%\end{obs}

% \lipsum[1-4]
% BUD
% 
%  \begin{wrapfigure}{r}{5cm}
%     \centering
%     \begin{bmlimage}\begin{tikzpicture}
%       \draw[style=help lines] (0,0) grid (3,8);
%     \end{tikzpicture}\end{bmlimage}
% %    \caption{Tall and narrow figure}\label{fig:tnfigure}
%   \end{wrapfigure}
% DUB
% \lipsum[5-7]

\begin{obs}
\index{indeterminação!do tipo ``$\frac00$''}
Em geral, $f'(a)$ é um limite indeterminado da forma
$\frac00$. De fato, se $f$ é contínua em $a$ então quando $x\to a$,
o numerador $f(x)-f(a)\to 0$ e o denominador $x-a\to 0$.
Por isso, os métodos estudados no último capítulo serão usados constantemente
para calcular derivadas.
\end{obs}

\begin{obs}
Observe que com a mudança de variável $h\pardef x-a$, $x\to a$ implica $h\to
0$, logo a derivada pode ser escrita também como
\eq{\boxed{
f'(a)\pardef \lim_{h\to 0}\frac{f(a+h)-f(a)}{h}\,,}}
\end{obs}


\begin{exo}
Considere  $f(x)\pardef x^2-x$. Esboce o gráfico de $f$.
Usando a definição de derivada, calcule a derivada de $f$ nos pontos
$a=0$, $a=\frac12$, $a=1$. 
Interprete o seu resultado graficamente.
\begin{sol} 
Como $x^2-x=(x-\frac12)^2-\frac14$, o gráfico obtém-se a partir do gráfico de
$x\mapsto x^2$ por duas translações.
Usando a definição de derivada, podemos calcular para todo $a$:
$$f'(a)=\lim_{x\to a}\frac{f(x)-f(a)}{x-a}
=\lim_{x\to a}\frac{(x^2-x)-(a^2-a)}{x-a}=
\lim_{x\to a}\Bigl\{\frac{x^2-a^2}{x-a}-1\Bigr\}=2a-1\,.$$
Aplicando essa fórmula para $a=0,\frac12,1$, obtemos $f'(0)=-1$,
$f'(\frac12)=0$, $f'(1)=+1$. 
Esses valores correspondem às inclinações das retas
tangentes ao gráfico nos pontos $(0,f(0))=(0,0)$,
$(\frac12,f(\frac12))=(\frac12,-\frac14)$ e $(1,f(1))=(1,0)$:
\begin{center}
\begin{bmlimage}\begin{tikzpicture}[scale=1.7]
\newcommand{\funcao}[1]{((#1)^2-(#1))}
\newcommand{\dfuncao}[2]{ ((\funcao{#1+#2})/(#2)-(\funcao{#1})/(#2)) }
\draw[ ->] (0,-0.2)--(0,1)node[right]{$\scriptstyle{x^2-x}$};
\draw[ ->] (-0.5,0)--(1.5,0);
\draw[thick, domain=-0.5:1.5] plot (\x,{\funcao{\x}});
\foreach \a in {0,0.5,1} {
\draw[thick,  domain={\a-0.3}:{\a+0.3}] plot
(\x,{(\dfuncao{\a}{0.01})*(\x-\a)+\funcao{\a}});
\fill (\a,{\funcao{\a}}) circle (0.40mm);
}
\draw[dotted]
(0.5,{\funcao{0.5}})--(0.5,0)node[above]{$\scriptstyle{\tfrac12}$};

\draw (1,0) node[above]{$\scriptstyle{1}$};
\end{tikzpicture}\end{bmlimage}
\end{center}
\end{sol}
\end{exo}




\begin{exo}
Usando a {definição}, calcule a derivada de $f$ no ponto
dado.
\begin{multicols}{2}
\begin{enumerate}
\item\label{itderivelem11} $f(x)=\sqrt{x}$,  $a=1$
\item\label{itderivelem1} $f(x)=\sqrt{1+x}$, $a=0$
\item\label{itderivelem2} $f(x)=\frac{x}{x+1}$, $a=0$
\item\label{itderivelem3} $f(x)=x^4$, $a=-1$
\item\label{itderivelem4} $f(x)=\frac{1}{x}$, $a=2$.
\end{enumerate}
\end{multicols}
\vspace{0.01cm}
\begin{sol}
\eqref{itderivelem11} $f'(1)=\half$,
\eqref{itderivelem1} $f'(0)=\half$ (a mesma do item anterior, pois o
gráfico de $\sqrt{1+x}$ é o de $\sqrt{x}$ transladado de $1$ para a esquerda!),
\eqref{itderivelem2} $f'(0)=1$,
\eqref{itderivelem3} $f'(-1)=-4$,
\eqref{itderivelem4} $f'(2)=-\frac{1}{4}$.
\end{sol}
\end{exo}


\begin{exo}
Dê a equação da reta tangente ao gráfico da função no(s)
ponto(s) dado(s):
\begin{multicols}{2}
\begin{enumerate}
\item\label{iteqretang1} $3x+9$, $(4,21)$
\item\label{iteqretang2} $x-x^2$, $(\half,\frac{1}{4})$
\item\label{iteqretang3} $\sqrt{1+x}$, $(0,1)$
\item\label{iteqretang4} $\frac{1}{x}$, $(-1,-1)$, $(1,1)$
\item\label{iteqretang5} $\sqrt{1-x^2}$, $(-1,0)$, $(1,-1)$
$(0,1)$, $(1,0)$
\item\label{iteqretang6} $\sen x$, $(0,0)$, $(\frac{\pi}{2},1)$
\end{enumerate}
\end{multicols}
\vspace{0.01cm}
\begin{sol}
\eqref{iteqretang1} $y=3x+9$,
\eqref{iteqretang2} $y=\frac{1}{4}$,
\eqref{iteqretang3} $y=\half x+1$,
\eqref{iteqretang4} $y=-x-2$, $y=-x+2$
\eqref{iteqretang5} Observe que a função descreve a metade superior de um
circulo de raio $1$ centrado na origem. As retas tangentes são, em $(-1,0)$:
$x=-1$, em $(1,-1)$: não existe (o ponto nem pertence ao círculo!), em $(0,1)$:
$y=1$, e em $(1,0)$: $x=1$.
\eqref{iteqretang6} Mesmo sem saber ainda como calcular a derivada da
função seno: $y=x$, $y=1$.
\end{sol}
\end{exo}

\begin{exo}\label{Exo:tangenteaucercle}
\index{círculo}
Calcule a equação da reta tangente ao círculo $x^2+y^2=25$ nos pontos
$P_1=(3,4)$, $P_2=(3,-4)$, $P_3=(5,0)$.
\begin{sol}
Primeiro é preciso ter uma função para representar o círculo na vizinhança de
$P_1$: $f(x)\pardef \sqrt{25-x^2}$. A inclinação da tangente em $P_1$ é dada por
\begin{align*}
f'(3)=\lim_{x\to 3}\frac{f(x)-f(3)}{x-3}&=
\lim_{x\to 3}\frac{\sqrt{25-x^2}-\sqrt{16}}{x-3}\\
&=\lim_{x\to 3}\frac{(25-x^2)-{16}}{(x-3)(\sqrt{25-x^2}+\sqrt{16})}
=\lim_{x\to 3}\frac{-(3+x)}{\sqrt{25-x^2}+\sqrt{16}}=-\tfrac34\,.
\end{align*}
(Essa inclinação poderia ter sido obtido observando que a reta
procurada é perpendicular ao segmento $OP$, cuja inclinação é
$\frac43$...)
Portanto, a equação da reta tangente em $P_1$ é $y=-\frac34
x+\frac{25}{4}$.  No ponto $P_2$, é preciso tomar a função 
$f(x)\pardef -\sqrt{25-x^2}$. Contas parecidas dão a equação
da tangente ao círculo em $P_2$: $y=\frac34 x-\frac{25}{4}$.
\begin{center}
\begin{bmlimage}\begin{tikzpicture}[scale=1]
\draw[ ->] (0,-1.2)--(0,1.2);
\draw[ ->] (-1.2,0)--(1.2,0);
\draw[dotted]
(0.6,0)node[below]{$\scriptstyle{3}$} -- (0.6,0.8) -- (0,0.8)node[left]
{$\scriptstyle{4}$};
\pgfmathsetmacro{\a}{0.6};
\draw[very thick, domain={\a-0.4}:{\a+0.4}] plot
(\x,{-0.75*(\x-\a)+0.8});
\draw[very thick,  domain={\a-0.4}:{\a+0.4}] plot
(\x,{+0.75*(\x-\a)-0.8});
\draw (0,0) circle (1cm);
\fill (0.6,0.8) circle (0.40mm);
\fill (0.6,-0.8) circle (0.40mm);
\draw (0.6,0.8) node[above right]{$P_1$};
\draw (0.6,-0.8) node[below right]{$P_2$};
\draw[very thick] (1,-0.4)--(1,0.4);
\fill (1,0) circle (0.40mm);
\draw (1,0) node[above right]{$P_3$};
\end{tikzpicture}\end{bmlimage}
\end{center}
A reta tangente ao círculo no ponto $P_3$ é vertical, e tem equação $x=5$.
Aqui podemos observar que a derivada de $f$ em $a=5$ \emph{não existe}, porqué
a inclinação de uma reta vertical não é definida (o que não impede achar a sua
equação...)!
\end{sol}
\end{exo}

\begin{exo}
Determine o ponto $P$ da curva $y = \sqrt{x}$, $x\geq 0$, no qual a reta
tangente
$r_P$ \`a curva \'e paralela \`a 
reta $r$ de equa\c c\~ao 
$8x-y- 1 = 0$. Esboce a curva e as duas retas $r_P$, $r$.
\begin{sol}
Se $f(x)=\sqrt{x}$, temos que para todo $a>0$,
$f'(a)=\frac{1}{2\sqrt{a}}$.
Como a reta $8x-y- 1 = 0$ tem inclinação $8$, precisamos achar um $a$ tal que
$f'(a)=8$, isto é, tal que $\frac{1}{2\sqrt{a}}=8$: $a=\frac{1}{256}$.
Logo, o ponto procurado é $P=(a,f(a))=(\frac{1}{256},\frac{1}{16})$.
\end{sol}
\end{exo}


\begin{exo}
Calcule o valor do parâmetro $\beta$ para que a reta $y=x-1$ seja tangente
ao gráfico da função $f(x)=x^2-2x+\beta$. Em seguida, faça o esboço de 
$f$ e da reta.
\begin{sol}
Para a reta $y=x-1$ (cuja inclinação é $1$) poder ser tangente ao gráfico de
$f$ em algum ponto $(a,f(a))$, esse $a$ deve satisfazer $f'(a)=1$. Ora, é fácil
ver que para um $a$ qualquer, $f'(a)=2a-2$. Logo, $a$ deve satisfazer $2a-2=1$,
isto é: $a=\frac32$. Ora, a reta e a função devem ambas passar pelo ponto
$(a,f(a))$, logo $f(a)=a-1$, isto é:
$(\frac32)^2-2\cdot\frac32+\beta=\frac32-1$. Isolando:
$\beta=\frac{5}{4}$.
\begin{center}
\begin{bmlimage}\begin{tikzpicture}
\newcommand{\funcao}[1]{(#1)^2-2*(#1)+1.25}
\newcommand{\dfuncao}[2]{ (\funcao{#1+#2})/(#2)-(\funcao{#1})/(#2)}
\draw[ ->] (0,-0.2)--(0,2.5) node[right]{$y$};
\draw[ ->] (-1,0)--(3,0) node[right]{$x$};
\draw[thick, domain=-0.5:2.5] plot (\x,{\funcao{\x}})
node[right]{$y=x^2-2x+\frac54$};
\pgfmathsetmacro{\a}{1.5};
\draw[thick,  domain={\a-1}:{\a+1}] plot
(\x,{(\dfuncao{\a}{0.01})*(\x-\a)+\funcao{\a}}) node[right]{$y=x-1$};
\fill (\a,{\funcao{\a}}) circle (0.40mm);
\end{tikzpicture}\end{bmlimage}
\end{center}
Esse problema pode ser resolvido sem usar derivada:
para a parábola $y=x^2-2x+\beta$ ter $y=x-1$ como reta tangente, a única
possibilidade é que as duas se intersectem em um ponto só, isto é, que a
equação $x^2-2x+\beta=x-1$ possua uma única solução. Rearranjando:
$x^2-3x+\beta+1=0$. Para essa equação ter uma única solução, é preciso que o
seu $\Delta=5-4\beta=0$. Isso implica $\beta=\frac{5}{4}$.
\end{sol}
\end{exo}

\begin{exo}
Considere o gráfico de $f(x)=\frac{1}{x}$. Existe um ponto $P$ do gráfico
de $f$ no qual a reta tangente ao gráfico
passa pelo ponto $(0,3)$?
\begin{sol}
Seja $P=(a,\frac1a)$ um ponto qualquer do gráfico. Como 
$f'(a)=-\frac{1}{a^2}$, a reta tangente ao gráfico em $P$ é 
$y=f'(a)(x-a)+f(a)=-\frac{1}{a^2}(x-a)+\frac1a$. Para essa reta passar pelo
ponto $(0,3)$, temos $3=-\frac{1}{a^2}(0-a)+\frac1a$, o que
significa que $a=\frac{2}{3}$.
Logo, a reta tangente ao gráfico de $\frac1x$ no ponto $P=(\frac23,\frac32)$
passa pelo ponto $(0,3)$.
\end{sol}
\end{exo}

\begin{exo}
Determine o ponto $P$ do gráfico da função $f(x)=x^3-2x+1$ 
no qual a equação da tangente é $y=x+3$.
\begin{sol}
$P=(-1,2)$.
\end{sol}
\end{exo}


\subsection{Pontos de não-diferenciabilidade}
A derivada nem sempre existe, por razões geométricas particulares: a reta
tangente não é sempre bem definida. Vejamos alguns exemplos:
\begin{ex}\label{Ex:derivracine}
Considere $f(x)\pardef x^{1/3}$, definida para todo $x\in \bR$ (veja Seção
\ref{Sec:InversoPotencias}).
Para um $a\neq 0$ qualquer, calculemos (com a mudança $t=x^{1/3}$)
$$f'(a)=\lim_{x\to a}\frac{x^{1/3}-a^{1/3}}{x-a}=
\lim_{t\to a^{1/3}}\frac{t-a^{1/3}}{t^3-a}=\lim_{t\to
a^{1/3}}\frac{1}{t^2+a^{1/3}t+a^{2/3}}=\frac{1}{3a^{2/3}}\,.
$$
Se $a=0$, é preciso calcular:
$$
f'(0)=\lim_{x\to 0}\frac{x^{1/3}-0^{1/3}}{x-0}=
\lim_{x\to 0}\frac{1}{x^{2/3}}=+\infty\,.
$$
De fato, a reta tangente ao gráfico em $(0,0)$ é vertical:
\begin{center}
\begin{bmlimage}\begin{tikzpicture}
\draw[->] (-2.5,0)--(2.5,0)node[right]{$x$};
\draw[->] (0,-1)--(0,1)node[left]{$x^{1/3}$};
\draw[thick, domain=-1.2:1.2] plot ({\x^3},\x); 
\draw[very thick] (0,-0.8)--(0,0.8);
\fill (0,0) circle (0.40mm);
\end{tikzpicture}\end{bmlimage}
\end{center} 
Assim, $x^{1/3}$ é derivável em qualquer $a\neq 0$, mas não em $a=0$. 
\end{ex}



\begin{ex}
Considere agora $f(x)=|x|$, também definida para todo $x\in \bR$. Se $a>0$,
então
$$f'(a)=\lim_{x\to a}\frac{|x|-|a|}{x-a}=
\lim_{x\to a}\frac{x-a}{x-a}=+1\,.$$
Por outro lado, se $a<0$,
$$f'(a)=\lim_{x\to a}\frac{|x|-|a|}{x-a}=
\lim_{x\to a}\frac{-x-(-a)}{x-a}=-1\,.$$
Então $|x|$ é derivável em qualquer $a\neq 0$.
Mas observe que em $a=0$, 
$$\lim_{x\to 0^+}\frac{|x|-|0|}{x-0}=+1\,,\quad \quad
\lim_{x\to 0^-}\frac{|x|-|0|}{x-0}=-1\,.$$
Como os limites laterais não coincidem, o limite bilateral não existe, o que
significa que $f(x)=|x|$ \emph{não é derivável (apesar de ser contínua) em
$a=0$.} De fato, o gráfico mostra que na origem $(0,0)$, a reta tangente não é
bem definida:

\begin{center}
\begin{bmlimage}\begin{tikzpicture}[scale=1.5]
\draw[->] (-1.4,0)--(1.4,0)node[right]{$x$};
\draw[->] (0,-0.3)--(0,1)node[right]{$|x|$};
\draw[thick] (-1,1)--(0,0)--(1,1); 
\pgfmathsetmacro{\r}{0.8};
\foreach \alf in {-30, 15, 35} {
\coordinate (A) at ({\r*cos(\alf)},{\r*sin(\alf)});
\coordinate (B) at ({-\r*cos(\alf)},{-\r*sin(\alf)});
\draw (A) node[right]{$?$}--(B);
}
\fill (0,0) circle (0.40mm);
\end{tikzpicture}\end{bmlimage}
\end{center}

\end{ex}

\begin{exo}
 Dê um exemplo de uma função contínua $f:\bR\to \bR$ que seja
derivável em qualquer ponto da reta, menos em $-1,0,1$.
\begin{sol}
Por exemplo, $f(x)\pardef |x+1|/2-|x|+|x-1|$.
Mais explicitamente,
\begin{center}
\begin{bmlimage}\begin{tikzpicture}
\draw (-8,0.8) node{$\displaystyle{
f(x)=
\begin{cases}
\frac{1-x}{2}&\text{ se }x\leq -1\\
\frac{x+3}{2}&\text{ se }-1\leq x\leq 0\\
\frac{3-3x}{2}&\text{ se }0\leq x\leq 1\\
\frac{x-1}{2}&\text{ se }x\geq 1\,.
\end{cases}
}$};
\draw [thick, domain=-2:2, samples=200]plot
(\x,{abs(\x+1)/2-abs(\x)+abs(\x-1)});
\draw [->](-2,0)--(2,0) ;
\draw (2.2,0) node {$x$};
\draw [->](0,-0.5)--(0,2);
\draw (-0.5,1.8) node {$f(x)$};
\draw (-1,0) node {$\shortmid$};
\draw (-1,-0.4) node {$-1$};
\draw (1,0) node {$\shortmid$};
\draw (1,-0.4) node {$1$};
\end{tikzpicture}\end{bmlimage}
\end{center}
$f$ não é derivável em $x=1$, porqué 
$\lim_{x\to 1^+}\frac{f(x)-f(1)}{x-1}=\lim_{x\to
1^+}\frac{\frac{x-1}{2}-0}{x-1}=\frac12$,
enquanto 
$\lim_{x\to 1^-}\frac{f(x)-f(1)}{x-1}=\lim_{x\to
1^-}\frac{\frac{3-3x}{2}-0}{x-1}=-\frac32\neq \frac12$.
A não-derivabilidade nos pontos $-1$ e $0$ obtem-se da mesma maneira.
\end{sol}
\end{exo}


Apesar da função $|x|$ não ser derivável em $a=0$, vimos que é possível 
``derivar pela esquerda ou pela direita'', usando limites laterais.
Para uma função $f$, as
\grasA{derivadas laterais em $a$}, $f'_+(a)$ e $f'_-(a)$,
\index{derivada!lateral}
são definidas pelos limites (quando eles existem)
\eq{f'_\pm(a)\pardef \lim_{x\to a^\pm}\frac{f(x)-f(a)}{x-a}=
\lim_{h\to 0^\pm}\frac{f(a+h)-f(a)}{h}\,.}



\subsection{Derivabilidade e continuidade}
\index{diferenciabilidade! e continuidade}
Vimos casos (como $|x|$ ou $x^{1/3}$ em $a=0$) em que uma função pode ser
contínua num ponto sem ser derivável nesse ponto. 
Mas o contrário sempre vale:

\begin{teo}\label{Teo:derivimplicacontin}
 Se $f$ é derivável em $a$, então ela é contínua em $a$.
\end{teo}
\begin{proof}
 De fato, dizer que $f$ é derivável em $a$ implica que o limite $f'(a)=\lim_{x\to
a}\frac{f(x)-f(a)}{x-a}$ existe e é finito. Logo, 
\begin{align*}
\lim_{x\to a}(f(x)-f(a))&=
\lim_{x\to a}\Bigl\{\frac{f(x)-f(a)}{x-a}(x-a)
\Bigr\}\\
&=
\Bigl\{\lim_{x\to a}\frac{f(x)-f(a)}{x-a}
\Bigr\}\cdot \{\lim_{x\to a}(x-a)\}=0\,,
\end{align*}
o que implica $f(x)\to f(a)$ quando $x\to a$. Isto é: $f$ é contínua em $a$.
\end{proof}

\section{A derivada como função}
\index{derivada!como função}

\begin{ex}
\emph{Será que existe um ponto $P$ da parábola $f(x)=x^2$ em que a
reta tangente tem inclinação igual a $2975$?}

O que sabemos fazer, até agora, é fixar um ponto, por exemplo $a=1$, e calcular a inclinação da reta
tangente à parábola no ponto $(1,f(1))$, que é dada por $f'(1)$.
Para responder à pergunta acima, poderíamos calcular a derivada em vários 
pontos da reta, um a um, até achar um
em que a inclinação à igual a $2975$.

Mas é mais fácil reformular a pergunta acima diretamente em termos da derivada: \emph{Será
que existe um ponto $a$ em que}
\[ f'(a)=2975\quad ?
\]
Para isto, é preciso ter a \emph{função} $f'(\cdot)$, que associa a cada $a$ a inclinação
da reta tangente ao gráfico de $f$ no ponto $(a,f(a))$.
Logo, vamos supor que $a$ é um ponto fixo da reta, sem
especificar o seu valor, e calcular
\[ 
f'(a)
=\lim_{x\to a}\frac{f(x)-f(a)}{x-a}
=\lim_{x\to a}\frac{x^2-a^2}{x-a}
=\lim_{x\to a}\frac{(x-a)(x+a)}{x-a}
=\lim_{x\to a}(x+a)
=2a\,.
\]
Agora, a \emph{equação} que precisamos resolver, $f'(a)=2975$, é simplesmente
\[ 
2a=2975\,,\quad \Rightarrow \quad 
a=1487.5\,.
\]
O ponto procurado é $P(1487.5,1487.5^2)$.
\end{ex}

O exemplo acima mostrou a utilidade de ver a derivada como uma \emph{função}
$a\mapsto f'(a)$.
Quando se fala em função, é mais
natural a escrever usando a letra $x$ em
vez da letra $a$: $$x\mapsto f'(x)\,.$$
Assim, a derivada pode também ser vista como um jeito de definir, a partir de
uma função $f$, uma outra função $f'$, chamada \grasA{derivada de $f$},
definida (quando o limite existe) por 
$$\boxed{f'(x)\pardef \lim_{h\to 0}\frac{f(x+h)-f(x)}{h}\,.}$$
Observe que nessa expressão, $h$ tende a zero enquanto $x$ é \emph{fixo}.

\begin{obs}
É importante mencionar que o domínio de $f'$ é em geral menor que o de $f$.
Por exemplo, $|x|$ é bem  definida para todo $x\in \bR$, mas
vimos que a sua derivada é definida
somente quando $x\neq 0$.
\end{obs}

\begin{exo}
Se $f$ é par (resp. ímpar), derivável, mostre que a sua derivada é ímpar (resp.
par).
\begin{sol}
De fato, se $f$ é par, 
\begin{align*}
f'(-x)=\lim_{h\to 0}\frac{f(-x+h)-f(-x)}{h}
&=\lim_{h\to 0}\frac{f(x-h)-f(x)}{h}\\
&=-\lim_{h'\to 0}\frac{f(x+h')-f(x)}{h'}=-f'(x)\,.
\end{align*}
\end{sol}
\end{exo}

\begin{exo}
Se $f$ é derivável em $a$, calcule o limite $\lim_{x\to
a}\frac{af(x)-xf(a)}{x-a}$
\begin{sol}
$af'(a)-f(a)$
\end{sol}
\end{exo}

Derivadas serão usadas extensivamente no resto do curso.
Nas três próximas seções calcularemos as derivadas de algumas funções
fundamentais. Em seguida provaremos as regras de derivação, que permitirão
calcular a derivada de qualquer função a partir das derivadas das funções
fundamentais. Em seguida comecaremos a usar derivadas na resolução de problemas
concretos.

\subsection{Derivar as potências inteiras: $x^{p}$}
\index{derivada! de potências}
Mostraremos aqui que para as potências inteiras de $x$, 
$x^p$ com $p\in \bZ$,
\eq{\label{eq:derivpotenciainteira}\boxed{(x^p)'=px^{p-1}\,.}}
O caso $p=2$ já foi tratado no Exemplo
\ref{Ex:primeiraretatangente} e no Exercício \ref{Exo:retatangxisdois}:
$$(x^2)'=\lim_{h\to 0}\frac{(x+h)^2-x^2}{h}=\lim_{h\to 0}\frac{2xh+h^2}{h}=
\lim_{h\to 0} (2x+h)=2x\,.
$$
Na verdade, para $x^n$ com $n\in \bN$ qualquer, já calculamos no Exercício 
\ref{Exo:LimitescomDicas}:
\eq{(x^n)'=\lim_{h\to 0}\frac{(x+h)^n-x^n}{h}=nx^{n-1}\,.}
Por exemplo, $(x^4)'=4x^3$, $(x^{17})'=17x^{16}$.
Daremos uma prova alternativa da fórmula 
\eqref{eq:derivpotenciainteira} no Exercício \ref{exo_DERIV_potenciasalternat} abaixo.
\begin{obs}
 O caso $p=0$ corresponde a $x^0=1$. Ora, a derivada de qualquer constante
$C\in \bR$ é zero (o seu gráfico corresponde a uma reta horizontal, portanto de
inclinação $=0$!):
$$\boxed{(C)'=0\,.}$$
\end{obs}


Para as potências negativas, $x^{-p}\equiv \frac{1}{x^q}$ obviamente não é
derivável em $0$, mas se $x\neq 0$,
$$
\bigl(\tfrac{1}{x^q}\bigr)'=\lim_{h\to
0}\frac{\frac{1}{(x+h)^q}-\frac{1}{x^q}}{h}=
\lim_{h\to
0}\frac{-1}{(x+h)^qx^q}\frac{(x+h)^q-x^q}{h}=
\frac{-1}{x^qx^q}q x^{q-1}=-qx^{-q-1}\,.
$$
Isso prova \eqref{eq:derivpotenciainteira} para qualquer $p\in \bZ$.
Veremos adiante que \eqref{eq:derivpotenciainteira} vale para
qualquer $p$, \emph{mesmo não inteiro}. Por exemplo,
$(x^{\sqrt{2}})'=\sqrt{2}x^{\sqrt{2}-1}$. Para alguns casos simples, uma conta
explícita pode ser feita. Por exemplo, se $p=\pm \frac12$,
\begin{exo}
Calcule $(\sqrt{x})'$, $(\frac{1}{\sqrt{x}})'$.
\begin{sol}
$(\sqrt{x})'=\lim_{h\to 0}\frac{\sqrt{x+h}-\sqrt{x}}{h}=
\lim_{h\to 0}\frac{1}{\sqrt{x+h}+\sqrt{x}}=\frac{1}{2\sqrt{x}}$.
O outro limite se calcula de maneira parecida:
$$(\frac{1}{\sqrt{x}})'=\lim_{h\to
0}\frac{\frac{1}{\sqrt{x+h}}-\frac{1}{\sqrt{x}}}{h}=
\lim_{h\to
0}\frac{\sqrt{x}-\sqrt{x+h}}{h\sqrt{x}\sqrt{x+h}}=\cdots=-\frac{1}{2\sqrt{x^3}}
\,.
$$
\end{sol}
\end{exo}



\subsection{Derivar as funções trigonométricas}
\index{derivada! de funções trigonométricas}
A derivada da função seno já foi calculada no Exercício
\ref{Exo:LimitescomDicas}. Por definição,
$$
(\sen)'( x) 
=\lim_{h\to 0}\frac{\sen (x+h)-\sen x}{h}\,.
$$
Usando a fórmula \eqref{eqsensoma}, $\sen (x+h)=\sen x\cos h+\sen h\cos x$,
obtemos
\begin{align*}
\lim_{h\to 0}\frac{\sen (x+h)-\sen x}{h}
&=\lim_{h\to 0}\frac{\sen x\cos h+\sen h\cos x-\sen x}{h}\\
&=\sen x\Bigl\{\lim_{h\to 0}\frac{\cos h-1}{h}\Bigr\}+\cos x\Bigl\{\lim_{h\to
0}\frac{\sen h}{h}\Bigr\}\,.
\end{align*}
Ora, sabemos que $\lim_{h\to 0}\frac{\sen h}{h}=1$, e que 
$\lim_{h\to 0}\frac{\cos h-1}{h}=\lim_{h\to 0}h\frac{\cos h-1}{h^2}=0$ (lembre o
item \eqref{itexosinxx5} do Exercício \ref{Exo:variantessinxsurx}).
Portanto, provamos que
\eq{\label{eq:derivseno}\boxed{(\sen)'( x) =\cos x}\,.}
Pode ser provado (ver o exercício abaixo) que
\eq{\label{eq:derivcosseno}\boxed{(\cos)'( x)=-\sen x\,.}}
Para calcular a derivada da tangente, $\tan x=\frac{\sen x}{\cos x}$,
precisaremos de uma regra de derivação que será provada na Seção
\ref{Sec:regrasdederivacao}; obteremos 
\eq{\label{eq:derivtan}\boxed{(\tan)'( x)
=1+\tan^2x=\frac{1}{\cos^2x}\,.}}
\begin{exo}\label{Exo:retastangentesseno}
Calcule a equação da reta tangente ao gráfico da função $\sen x$, nos pontos 
$P_1=(0,0)$, $P_2=(\pisobredois, 1)$, $P_3=(\pi,0)$. Confere no gráfico.
\begin{sol}
Como $(\sen)'(x)=\cos x$, a inclinação da reta tangente em $P_1$ é $\cos(0)=1$,
em $P_2$ é $\cos(\pisobredois)=0$, e em $P_3$ é $\cos(\pi)=-1$. Logo, as
equações das respectivas retas tangentes são $r_1$: $y=x$, $r_2$: $y=1$, $r_3$:
$y=-(x-\pi)$:
\begin{center}
\begin{bmlimage}\begin{tikzpicture}[scale=1]
\newcommand{\funcao}[1]{sin(#1 r)}
\newcommand{\dfuncao}[2]{ (\funcao{#1+#2})/{#2}-(\funcao{#1})/{#2} }
\draw[ ->] (0,-0.2)--(0,1)node[left]{$\scriptstyle{\sen x}$};
\draw[ ->] (-2,0)--(5,0);
\draw[thick, domain=-1.8:4.8] plot (\x,{\funcao{\x}});

\pgfmathsetmacro{\e}{0.75};
\pgfmathsetmacro{\a}{0};
\draw[thick,  domain={\a*1.5707-\e}:{\a*1.5707+\e}] plot
(\x,{\x});
\pgfmathsetmacro{\a}{1};
\draw[thick,  domain={\a*1.5707-\e}:{\a*1.5707+\e}] plot
(\x,{1});
\pgfmathsetmacro{\a}{2};
\draw[thick,  domain={\a*1.5707-\e}:{\a*1.5707+\e}] plot
(\x,{3.1415-\x});

\foreach \a in {0,1,2} {
\fill ({\a*1.5707},{sin(\a*1.5707 r)}) circle (0.50mm);
}
% \draw[dotted]
% (0.5,{\funcao{0.5}})--(0.5,0)node[above]{$\scriptstyle{\tfrac12}$};
% 
% \draw (1,0) node[above]{$\scriptstyle{1}$};
\end{tikzpicture}\end{bmlimage}
\end{center}
\end{sol}
\end{exo}

\begin{exo}
Prove \eqref{eq:derivcosseno}.
\end{exo}

\subsection{Derivar exponenciais e logaritmos}
\index{derivada!de exponencial e logaritmo}
Na Seção \ref{Sec:Fundam_numero_e} calculamos
\eq{\label{Eq:derivexponenzerobis}\lim_{h\to 0}\frac{e^h-1}{h}=1\,,\quad\quad
\lim_{h\to 0}\frac{\ln(1+h)}{h}=1\,.}
Lembre que esses limites seguem diretamente da definição do número
$e$, como o limite $e\pardef \lim_{n\to \infty}(1+\frac1n)^n$.
Usaremos agora o primeiro desses limites para calcular
a derivada de $e^x$: para $x\in \bR$,
$$
(e^x)'\pardef\lim_{h\to 0}\frac{e^{x+h}-e^x}{h}=
\lim_{h\to 0}\frac{e^{x}e^h-e^x}{h}=
e^{x}\Bigl\{\lim_{h\to 0}\frac{e^h-1}{h}\Big\}=e^x\,.
$$
Portanto, está provado que a função exponencial é igual a sua derivada!
Por outro lado, para derivar o logaritmo, observe que 
para todo $x>0$,
$\ln(x+h)-\ln (x)=\ln(\frac{x+h}{x})=\ln(1+\frac{h}{x})$. Logo,
$$(\ln x)'\pardef \lim_{h\to 0}\frac{\ln(x+h)-\ln(x)}{h}=\lim_{h\to
0}\frac{\ln(1+\frac{h}{x})}{h}\,.$$
Chamando $\alpha\pardef\frac{h}{x}$ temos, usando \eqref{Eq:derivexponenzerobis},
$$(\ln x)'=\tfrac1x\Bigl\{\lim_{\alpha\to 
0}\frac{\ln(1+\alpha)}{\alpha}\Bigr\}=\tfrac1x\,.
$$
Calculamos assim duas derivadas fundamentais:
$$
\boxed{
(e^x)'=e^x\,,\quad\quad (\ln x)'=\frac{1}{x}\,.
}
$$

\begin{obs}
A interpretação geométrica dos limites em
\eqref{Eq:derivexponenzerobis} é a seguinte: a inclinação da reta
tangente ao gráfico de $e^x$ no ponto $(0,1)$ e a inclinação da reta
tangente ao gráfico de $\ln x$ no ponto $(1,0)$ ambas valem $1$ (lembre que o
gráfico do logaritmo é a reflexão do gráfico da exponencial pela bisetriz do
primeiro quadrante):
\begin{center}
\begin{bmlimage}\begin{tikzpicture}
\draw[ ->] (-2,0)--(1.5,0);
\draw[ ->] (0,-2)--(0,2);
\draw[thick, domain=-1.8:1] plot (\x,{exp(\x)}) node[left]{$e^x$}; 
\draw[thick,  domain=-0.8:1] plot (\x,{\x+1});
\draw[thick, domain=-1.8:1] plot ({exp(\x)},\x) node[right]{$\ln x$};
\draw[thick,  domain=1:-0.8] plot ({\x+1},\x);
%\draw[<-] (0.1,1)--(1,0.7)node[right]{inclinação=$1$}; 
\fill (0,1) circle (0.50mm);
\fill (1,0) circle (0.50mm);
\draw[dashed] (-1.5,-1.5)--(2,2);
% 
% \begin{scope}[xshift=5cm]
% \draw[ ->] (-2,0)--(1.5,0);
% \draw[ ->] (0,-0.2)--(0,2);
% \end{scope}
\end{tikzpicture}\end{bmlimage}
\end{center}
Uma olhada nos esboços das funções $a^x$ na página
\pageref{Fig:graficosdifbases} mostra que $e^x$ é a única com essa propriedade.
Às vezes, livros \emph{definem} ``$e$'' como sendo a única base $a$ que
satisfaz a essa propriedade: a inclinação da reta tangente a $a^x$ na origem é
igual a $1$.
\end{obs}

\section{Regras de derivação}\label{Sec:regrasdederivacao}
\index{regras de derivação}
Antes de começar a usar derivadas, é necessário estabelecer algumas 
\emph{regras
de derivação}, que respondem essencialmente à seguinte pergunta: se $f$ e $g$
sáo deriváveis, $f'$ e $g'$ conhecidas, como calcular $(f+g)'$, $(f\cdot 
g)'$,
$(\frac{f}{g})'$, $(f\circ g)'$? 
Nesta seção, será sempre subentendido que as funções consideradas são
deriváveis nos pontos considerados. Comecemos com o caso mais fácil:

\begin{regra}
$\boxed{(\lambda f(x))'=\lambda f'(x)}$ para toda constante $\lambda\in \bR$.
\end{regra}
\begin{proof} Usando a definição de $(\lambda f(x))'$  e colocando $\lambda$ em
evidência,
$$(\lambda f(x))'\pardef\lim_{h\to 0}\frac{\lambda f(x+h)-\lambda f(x)}{h}=
\lambda\lim_{h\to 0}\frac{ f(x+h)-f(x)}{h}\equiv \lambda f'(x)\,.
$$ 
\end{proof}
Por exemplo, $(2x^5)'=2(x^5)'=2\cdot 5x^4=10 x^4$.

\begin{regra}
 $\boxed{(f(x)+g(x))'=f'(x)+g'(x).}$
\end{regra}
\begin{proof}
Aplicando a definição e rearranjando os termos,
 \begin{align*}
(f(x)+g(x))'&\pardef \lim_{h\to
0}\frac{\bigl(f(x+h)+g(x+h)\bigr)-\bigl(f(x)+g(x)\bigr)}{h}\\
&=\lim_{h\to
0}\Bigl\{\frac{f(x+h)-f(x)}{h}+\frac{g(x+h)-g(x)}{h}\Bigr\}\\
&=\lim_{h\to
0}\frac{f(x+h)-f(x)}{h}+\lim_{h\to
0}\frac{g(x+h)-g(x)}{h}=f'(x)+g'(x)\,.
\end{align*}
\end{proof}
Por exemplo, $(2x^5+\sen x)'=(2x^5)'+(\sen x)'=10x^4+\cos x$.

\begin{regra}
 $\boxed{(f(x)g(x))'=f'(x)g(x)+f(x)g'(x)}$ (Regra do produto de Leibniz).
\index{regra de Leibniz}
\end{regra}
\begin{proof}
 Por definição, 
\begin{align*}
(f(x)g(x))'&\pardef \lim_{h\to
0}\frac{f(x+h)g(x+h)-f(x)g(x)}{h}\,.
\end{align*}
Para fazer aparecer as derivadas respectivas de $f$ e $g$, escrevamos o
quociente como
\begin{align*}
\frac{f(x+h)g(x+h)-f(x)g(x)}{h}
={\frac{f(x+h)-f(x)}{h}}g(x+h)+f(x){\frac{g(x+h)-g(x)}{
h}}
\end{align*}
Quando $h\to 0$, temos $\frac{f(x+h)-f(x)}{h}\to
f'(x)$ e $\frac{g(x+h)-g(x)}{ h}\to g'(x)$.
Como $g$ é derivável em $x$, ela é também contínua em $x$
(Teorema \ref{Teo:derivimplicacontin}), logo $\lim_{h\to 0}g(x+h)=g(x)$. Assim,
quando $h\to 0$, o quociente inteiro tende a $f'(x)g(x)+f(x)g'(x)$.
\end{proof}
Por exemplo, 
\[(x^2\sen x)'=(x^2)'\sen x+x^2(\sen x)'=2x\sen x+x^2\cos x\,.\]

\begin{exo}
Dê contra-exemplos para mostrar que em geral, $(fg)'\neq f'g'$.
\begin{sol}
Por exemplo, se $f(x)=g(x)=x$, temos $(f(x)g(x))'=(x\cdot x)'=(x^2)'=2x$,
e $f'(x)g'(x)=1\cdot 1=1$. Isto é, $(f(x)g(x))'\neq f'(x)g'(x)$.
\end{sol}
\end{exo}
\index{regra!da cadeia}

\begin{exo}\label{exo_DERIV_potenciasalternat}
Mostre a fórmula $(x^n)'=nx^{n-1}$ usando indução e a regra de Leibniz. (Dica:
$x^{n+1}=x\cdot x^{n}$.)
\begin{sol}
Já sabemos que $(x)'=1$, e que $(x^2)'=2x$, o que prova a fórmula para $n=1$ e $n=2$.
Supondo que a fórmula foi provada para $n$, provaremos que ela vale para $n+1$
também.
De fato, usando a regra de Leibniz e a hipótese de indução,
\[ 
(x^{n+1})'=
(x\cdot x^n)'=1\cdot x^n+x\cdot nx^{n-1}=x^n+nx^n=(n+1)x^n\,.
\]
\end{sol}
\end{exo}

Estudemos agora a derivação de funções \emph{compostas}:
\begin{regra}
\index{regra!da cadeia}
 $\boxed{(f(g(x)))'=f'(g(x))g'(x)}$ (Regra da cadeia).
\end{regra}
\begin{proof} Fixemos um ponto $x$.
Suporemos, para simplificar, que $g(x+h)-g(x)\neq 0$ para todo $h$ suficientemente
pequeno~\footnote{Sem essa hipótese, a prova precisa ser ligeiramente modificada.}.
Podemos escrever
\begin{align}
(f(g(x)))'&\pardef \lim_{h\to 0}\frac{f(g(x+h))-f(g(x))}{h}\nonumber\\
&=\lim_{h\to
0}\frac{f(g(x+h))-f(g(x))}{g(x+h)-g(x)}{\frac{g(x+h)-g(x)}{h}}\,.\label{multdivgzero}
\end{align}
Sabemos que o segundo 
termo $\frac{g(x+h)-g(x)}{h}\to g'(x)$ quando $h\to 0$.
Para o
primeiro termo chamemos $a\pardef g(x)$ e $z\pardef g(x+h)$. Quando $h\to
0$, $z\to a$, logo
$$
\lim_{h\to
0}\frac{f(g(x+h))-f(g(x))}{g(x+h)-g(x)}=\lim_{z\to
a}\frac{f(z)-f(a)}{z-a}\equiv f'(a)=f'(g(x))\,.
$$
\end{proof}

Para aplicar a regra da cadeia, é importante saber identificar quais são as funções
envolvidas, e em qual ordem elas são aplicadas (lembre do Exercício
\ref{Exo_elem_decomp_compos}).

\begin{ex}
Suponha por exemplo que queira calcular a derivada da função $\sen(x^2)$, que é
a composta de $f(x)=\sen x$ com $g(x)=x^2$: $\sen(x^2)=f(g(x))$. 
Como $f'(x)=\cos x$ e $g'(x)=2x$ temos, pela regra da
cadeia,
$$(\sen(x^2))'=f(g(x))'=f'(g(x))g'(x)=\cos(x^2)\cdot
(2x)=2x\cos (x^2)\,.$$
Para calcular $e^{x^2}$, que é
a composta de $f(x)=e^x$ com $g(x)=x^2$, e como $f'(x)=e^x$,
temos
$$(e^{x^2})'=e^{x^2}\cdot (x^2)'=2xe^{x^2}\,.$$
\end{ex}

\begin{ex}
Para calcular a derivada de $\frac{1}{\cos x}$, que é a composta de
$f(x)=\frac{1}{x}$ com $g(x)=\cos x$, e como $f'(x)=-\frac{1}{x^2}$,
$g'(x)=-\sen x$, temos
$$
(\tfrac{1}{\cos x})'=-\frac{1}{(\cos x)^2}\cdot (-\sen x)=\frac{\sen x}{(\cos
x)^2}\,.
$$
De modo geral, deixando $g(x)$ ser uma função qualquer, derivável e não-nula em
$x$,
\eq{\label{eq:derivumsobreg}
\Big(\frac{1}{g(x)}\Big)'=-\frac{g'(x)}{g(x)^2}\,.}
\end{ex}

\begin{regra}\label{Regraquociente}
$\boxed{\Bigl(\frac{f(x)}{g(x)}\Bigr)'=\frac{f'(x)g(x)-f(x)g'(x)}{g(x)^2}}$ (Regra do
quociente).
\end{regra}
\begin{proof}
Aplicando a Regra de Leibniz e \eqref{eq:derivumsobreg},
$$
\Big(\frac{f(x)}{g(x)}\Big)'=
\Big(f(x)\cdot\frac{1}{g(x)}\Big)'=f'(x)\cdot\frac{1}{g(x)}+f(x)\cdot\Big(-\frac
{ g'(x) } { g(x)^2 }\Big)
=\frac{f'(x)g(x)-f(x)g'(x)}{g(x)^2}\,.
$$ 
\end{proof}
\begin{ex}
Usando a regra do quociente, podemos agora calcular:
$$(\tan x)'=\Big(
\frac{\sen x}{\cos x}\Big)'=\frac{(\sen x)'\cos x-\sen x(\cos x)'}{\cos^2x}
=\frac{\cos^2x+\sen^2x}{\cos^2x}
$$
Essa última expressão pode ser escrita de dois jeitos:
\[
\boxed{
(\tan x)'=
\begin{cases}
1+\tan^2x\,,\\
\text{ ou }\frac{1}{\cos^2x}\,.\
\end{cases}
}
\]
\end{ex}

\begin{exo}
Use as regras de derivação para 
calcular as derivadas das seguintes funções. Quando for possível,
simplifique a expressão obtida.
\begin{multicols}{3}
\begin{enumerate}
\item\label{itderivbas0} $-5x$
\item\label{itderivbas1} $x^3-x^7$
\item\label{itderivbas111} $1+x+\frac{x^2}{2}+\frac{x^3}{3}$
\item\label{itderivbas2} $\frac{1}{1-x}$
\item\label{itderivbas15} $x\sen x$
\item\label{itderivbas151} $(x^2+1)\sen x\cos x$
\item\label{itderivbas3} $\frac{\sen x}{x}$
\item\label{itderivbas4} $\frac{x+1}{x^2-1}$
\item\label{itderivbas1111} $(x+1)^5$
\item\label{itderivbas6} $\big(3+\frac{1}{x}\big)^2$
\item\label{itderivbas5} $\sqrt{1-x^2}$
\item\label{itderivbas7} $\sen^3 x-\cos^7 x$
\item\label{itderivbas8} $\frac{1}{1-\cos x}$
\item\label{itderivbas8meio} $\frac{1}{\cos (2x-1)}$
\item\label{itderivbas9} $\frac{1}{\sqrt{1+x^2}}$
\item\label{itderivbas10} $\frac{(x^2-1)^2}{\sqrt{x^2-1}}$
\item\label{itderivbas11} $\frac{x}{x+\sqrt{9+x^2}}$
\item\label{itderivbas12} $\sqrt{1+\sqrt{x}}$
\item\label{itderivbas16} $\frac{x}{\cos x}$
\item\label{itderivbas17} $\cos\sqrt{1+x^2}$
\item\label{itderivbas18} $\sen (\sen x)$
\end{enumerate}
\end{multicols}
\vspace{0.01cm}
\begin{sol}
\eqref{itderivbas0} $-5$
\eqref{itderivbas1} $(x^3-x^7)'=(x^3)'-(x^7)'=3x^2-7x^6$.
\eqref{itderivbas111}
$(1+x+\frac{x^2}{2}+\frac{x^3}{3})'=(1)'+(x)'+(\frac{x^2}{2})'+(\frac{x^3}{3})'
=1+x+x^2$.
\eqref{itderivbas2}
$(\frac{1}{1-x})'=-\frac{1}{(1-x)^2}\cdot(1-x)'=\frac{1}{(1-x)^2}$
\eqref{itderivbas15} $\sen x+x\cos x$
\eqref{itderivbas151} Usando duas vezes a regra de Leibniz: 
$((x^2+1)\sen x\cos x)'=2x\sen x\cos x+(x^2+1)(\cos^2x-\sen^2x)$
\eqref{itderivbas3} $\frac{x\cos x-\sen x}{x^2}$
\eqref{itderivbas4} $(\frac{x+1}{x^2-1})'=(\frac{1}{x-1})'=\frac{-1}{(x-1)^2}$.
\eqref{itderivbas1111} $(x+1)^5=f(g(x))$ com $f(x)=x^5$ e $g(x)=x+1$.
Logo, $((x+1)^5)'=f'(g(x))g'(x)=5(x+1)^4$. Obs: poderia também expandir
$(x+1)^5=x^5+\cdots$, derivar termo a termo, mas é muito mais longo, e a
resposta não é fatorada.
\eqref{itderivbas6} Como $(3+\frac{1}{x})^2=f(g(x))$ com
$f(x)=x^2$ e $g(x)=3+\frac{1}{x}$, e que $f'(x)=2x$,
$g'(x)=(3+\frac{1}{x})'=0-\frac{1}{x^2}$, temos
$((3+\frac{1}{x})^2)'=2(3+\frac{1}{x})\cdot(\frac{-1}{x^2})=-2\frac{3+\frac{1}
{x}}{x^2}$.
\eqref{itderivbas5} Como $\sqrt{1-x^2}=f(g(x))$, com $f(x)=\sqrt{x}$,
$g(x)=1-x^2$, e que $f'(x)=\frac{1}{2\sqrt{x}}$, $g'(x)=-2x$,
temos $(\sqrt{1-x^2})'=\frac{-x}{\sqrt{1-x^2}}$-
\eqref{itderivbas7} $3\sen^2x\cos x+7\cos^6x\sen x$
\eqref{itderivbas8} $\frac{\sen x}{(1-\cos x)^2}$
\eqref{itderivbas8meio} $\frac{2\sen (2x-1)}{(\cos(2x-1))^2}$
\eqref{itderivbas9}
$(\frac{1}{\sqrt{1+x^2}})'=((1+x^2)^{-\frac12})'=-\frac12(1+x^2)^{-\frac32}
\cdot (2x)=-\frac{x}{(1+x^2)^{\frac32}}=\frac{-x}{\sqrt{(1+x^2)^3}}$. 
\eqref{itderivbas10}
$(\frac{(x^2-1)^2}{\sqrt{x^2-1}})'=((x^2-1)^{\frac32})'=\frac{3}{2}(x^2-1)^{
\frac12}
\cdot(2x)=3x\sqrt{x^2-1}$ Obs: vale a pena simplificar a fração antes de
usar a regra do quociente!
\eqref{itderivbas11} $\frac{9}{\sqrt{9+x^2}(x+\sqrt{9+x^2})^2}$
\eqref{itderivbas12} $\frac{1}{4\sqrt{x}\sqrt{1+\sqrt{x}}}$
\eqref{itderivbas16} $\frac{\cos x+x\sen x}{(\cos x)^2}$
\eqref{itderivbas17} Usando duas vezes a regra da cadeia:
$(\cos\sqrt{1+x^2})'=(-\sen \sqrt{1+x^2})(\sqrt{1+x^2})'=\frac{-x\sen
\sqrt{1+x^2}}{\sqrt{1+x^2}}$ 
\eqref{itderivbas18} $\cos(\sen x)\cdot\cos x$
\end{sol}
\end{exo}


\begin{exo}
Calcule a derivada da função dada.
\begin{multicols}{4}
\begin{enumerate}
\item\label{itderivexpon1} $2e^{-x}$
\item\label{itderivexpon2} $\ln (1+x)$
\item\label{itderivexpon3} $\ln (e^{3x})$ 
\item\label{itderivexpon61} $e^x\sen x$
\item\label{itderivexpon4} $e^{\sen x}$
\item\label{itderivexpon5} $e^{e^x}$
\item\label{itderivexpon6} $\ln(1+e^{2x})$
\item\label{itderivexpon7} $x\ln x$
\item\label{itderivexpon8} $e^{\frac1x}$
%\item\label{itderivexpon9} $\senh x$
%\item\label{itderivexpon10} $\cosh x$
%\item\label{itderivexpon11} $\tanh x$
\item\label{itderivexpon12} $\ln(\cos x)$
\item\label{itderivexpon13} $\ln(\frac{1+\cos x}{\sen x})$
\end{enumerate}
\end{multicols}
\vspace{0.01cm}
\begin{sol}
\eqref{itderivexpon1} $(2e^{-x})'=2(e^{-x})'=2(e^{-x}\cdot(-x)')=-2e^{-x}$.
\eqref{itderivexpon2} $\frac{1}{x+1}$
\eqref{itderivexpon3} $(\ln (e^{3x}))'=(3x)'=3$
\eqref{itderivexpon61} $e^x(\sen x+\cos x)$
\eqref{itderivexpon4} $\cos x \cdot e^{\sen x}$
\eqref{itderivexpon5} $e^{e^x}\cdot e^x$
\eqref{itderivexpon6} $\frac{2e^{2x}}{1+e^{2x}}$
\eqref{itderivexpon7} $\ln x+x\frac{1}{x}=\ln x+1$
\eqref{itderivexpon8} $\frac{-e^{\frac1x}}{x^2}$
\eqref{itderivexpon12} $-\tan x$
\eqref{itderivexpon13} $\frac{-1}{\sen x}$
\end{sol}
\end{exo}

\begin{exo}
Verifique que as derivadas das funções trigonométricas hiperbólicas são
dadas por 
\[
\boxed{(\senh x)'=\cosh x\,,\quad (\cosh x)'=\senh x\,,\quad (\tanh x)'=
\begin{cases}
1-\tanh^2 x\,,\\
\text{ou }\frac{1}{\cosh^2 x}\,.
\end{cases}
}
\]
\begin{sol}
$(\senh
x)'=(\frac{e^x-e^{-x}}{2})'=\frac{e^x+e^{-x}}{2}\equiv \cosh x$.
Do mesmo jeito, $(\cosh x)'=\senh x$.
Para $\tanh$, basta usar a regra do quociente.
Observe as semelhanças entre as derivadas das funções trigonométricas
hiperbólicas e as funções trigonométricas.
\end{sol}
\end{exo}

Às vezes, um limite pode ser calculado uma vez que interpretado como uma
derivada.
\begin{ex}
Considere o limite $\lim_{x\to 1}\frac{\ln x}{x-1}$, que é indeterminado da
forma $\frac00$.
Como $\frac{\ln x}{x-1}=\frac{\ln x-\ln 1}{x-1}$, vemos que o limite pode ser
interpretado como a derivada da função $f(x)=\ln x$ no ponto $a=1$:
$$\lim_{x\to 1}\frac{\ln x-\ln 1}{x-1}=\lim_{x\to 1}\frac{f(x)-f(1)}{x-1}\equiv
f'(1)\,.$$ 
Ora, como $f'(x)=\frac{1}{x}$, temos $f'(1)=1$. Isto é: $\lim_{x\to 1}\frac{\ln
x}{x-1}=1$.
\end{ex}

\begin{exo}
Calcule os seguintes limites, interpretando-os como derivadas.
\begin{multicols}{3}
\begin{enumerate}
\item\label{itlimbargeaotviaderiv1} $\lim_{x\to 1}\frac{x^{999}-1}{x-1}$
\item\label{itlimbargeaotviaderiv2} $\lim_{x\to \pi}\frac{\cos x+1}{x-\pi}$
\item\label{itlimbargeaotviaderiv3} $\lim_{x\to \pi}\frac{\sen (x^2)-\sen
(\pi^2)}{x-\pi}$
\item\label{itlimbargeaotviaderiv4} $\lim_{x\to 2}\frac{\ln x-\ln 2}{x-2}$
\item\label{itlimbargeaotviaderiv5} $\lim_{t\to 0}\frac{e^{\lambda t}-1}{t}$
\end{enumerate}
\end{multicols}
\vspace{0.01cm}
\begin{sol}
\eqref{itlimbargeaotviaderiv1}
Sabemos que o limite $\lim_{x\to 1}\frac{x^{999}-1}{x-1}$ dá a inclinação da
reta tangente ao gráfico da função $f(x)=x^{999}$ no ponto $a=1$, isto é:
$\lim_{x\to 1}\frac{x^{999}-1}{x-1}=f'(1)$. Mas como 
$f'(x)=999x^{998}$, temos $f'(1)=999$.
\eqref{itlimbargeaotviaderiv2}
Da mesma maneira, $\lim_{x\to \pi}\frac{\cos x+1}{x-\pi}=
\lim_{x\to \pi}\frac{\cos x-\cos(\pi)}{x-\pi}$ dá a inclinação da reta tangente
ao gráfico do $\cos$ no ponto $\pi$. Como $(\cos x)'=-\sen x$, o limite vale
$0$.
\eqref{itlimbargeaotviaderiv3} $2\pi \cos(\pi^2)$
\eqref{itlimbargeaotviaderiv4} $\frac12$
\eqref{itlimbargeaotviaderiv5} $\lambda$
\end{sol}
\end{exo}

\begin{exo}
Considere as funções 
$$
f(x)\pardef
\begin{cases}
x\sen \frac1x &\text{ se } x\neq 0\,,\\
0  & \text{ se }x=0\,,
\end{cases}
\quad\quad
g(x)\pardef
\begin{cases}
x^2\sen \frac1x &\text{ se } x\neq 0\,,\\
0  & \text{ se }x=0\,.
\end{cases}
$$
Mostre que $g$ é derivável (logo, contínua) em todo $x\in \bR$.
Mostre que $f$ é contínua em todo $x\in \bR$ e derivável em todo $x\in
\bR\setminus\{0\}$, mas não é derivável em $x=0$.
\begin{sol}
Fora de $x=0$, $g$ é derivável e a sua derivada se calcula facilmente:
$g'(x)=(x^2\sen \frac1x)'=2x\sen \frac1x-\cos\frac1x$. 
Do mesmo jeito $f$ é derivável fora de $x=0$.
Em $x=0$, 
$$
g'(0)=\lim_{h\to 0}\frac{g(h)-g(0)}{h}=\lim_{h\to 0} h\sen \tfrac1h=0\,.
$$
(O último limite pode ser calculado como no Exemplo \ref{Ex:sanduicheseno},
escrevendo
$-h\leq h\sen \tfrac1h\leq +h$.)
Assim, $g$ é derivável também em $x=0$. No entanto, como
$$\lim_{h\to 0}\frac{f(h)-f(0)}{h}=\lim_{h\to 0} \sen \tfrac1h\,,$$
$f'(0)$ não existe: $f$ não é derivável em $x=0$.
\end{sol}
\end{exo}

\subsection{Derivar as potências $x^\alpha$; exponenciação}
\index{exponenciação}
Definir uma potência $x^p$ para $p\in \bZ$ é imediato. Por exemplo,
$x^3\pardef x\cdot x\cdot x$. Mas como definir $x^\alpha$ para uma potência
não-inteira, por exemplo $x^{\sqrt{2}}=x^{1,414...}$? \\

Um jeito de fazer é de se lembrar que qualquer $x>0$ pode ser
\emph{exponenciado}: $x=e^{\ln x}$. Como $(e^{\ln x})^\alpha=e^{\alpha \ln x}$,
é natural definir
\eq{\boxed{x^\alpha\pardef e^{\alpha \ln x}\,.}}
Observe que com essa definição, as regras habituais são satisfeitas. Por
exemplo, para qualquer $\alpha, \beta\in \bR$,
$$
x^\alpha x^\beta=e^{\alpha \ln x}e^{\beta \ln x}=e^{\alpha \ln x+\beta \ln x}
=e^{(\alpha+\beta)\ln x}=x^{\alpha+\beta}\,.
$$
Mas a definição dada acima permite também derivar $x^\alpha$, usando
simplesmente a regra da cadeia:
$$(x^\alpha)'=(e^{\alpha \ln x})'=(\alpha \ln x)'e^{\alpha \ln
x}=\frac{\alpha}{x}x^\alpha=\alpha x^{\alpha-1}\,.$$
Assim foi provado que a fórmula $(x^p)'=px^{p-1}$, inicialmente provada para 
$p\in \bZ$, vale também para expoentes não-inteiros.\\

O que foi usado acima é que
se $g$ é derivável, então pela regra da cadeia,
\eq{\label{eq:derivexponcadeia}(e^{g(x)})'=e^{g(x)}g'(x)\,.}

\begin{ex}
Considere uma exponencial numa base qualquer, $a^x$, $a>0$. Exponenciando
a base $a=e^{\ln a}$, temos $a^x=e^{x\ln a}$. Logo, 
\eq{\boxed{(a^x)'=(e^{x\ln a})'=(x\ln a)'e^{x\ln a}=(\ln a)a^x\,.}}
\end{ex}

Essa expressão permite calcular as derivadas das funções da forma 
$f(x)^{g(x)}$. De fato, se $f(x)$, sempre podemos escrever
$f(x)=e^{\ln f(x)}$, transformando $f(x)^{g(x)}=e^{g(x)\ln f(x)}$. 
Por exemplo, 

\begin{ex} Considere $x^x$, com $x>0$.
Escrevendo o $x$ (de baixo) como $x=e^{\ln x}$, temos $x^x=(e^{\ln
x})^x=e^{x\ln x}$, logo
$$(x^x)'=(e^{x\ln x})'=(x\ln x)'e^{x\ln x}=(\ln x+1)x^x\,.$$
\end{ex}

\begin{exo}
Derive as seguintes funções (supondo sempre que $x>0$).
\begin{multicols}{4}
\begin{enumerate}
\item\label{itderivfelevg1} $x^{\sqrt{x}}$
\item\label{itderivfelevg3} $(\sen x)^x$
\item\label{itderivfelevg4} $x^{\sen x}$
\item\label{itderivfelevg2} $x^{x^x}$
\end{enumerate}
\end{multicols}
\vspace{0.01cm}
\begin{sol}
\eqref{itderivfelevg1} 
$(x^{\sqrt{x}})'=(e^{\sqrt{x}\ln x})'=(\frac{\ln x}{2}+1){
x^{\sqrt{x}-\frac12}}$.
\eqref{itderivfelevg3} $((\sen x)^x)'=(\ln \sen x+x\cot x)(\sen x)^x$.
\eqref{itderivfelevg4} $(x^{\sen x})'=(\cos x\ln x+\frac{\sen x}{x})x^{\sen x}$.
\eqref{itderivfelevg2} $(x^{x^x})'=\bigl((\ln x+1)\ln
x+\frac1x\bigr)x^xx^{x^x}$.
\end{sol}
\end{exo}

\subsection{Derivadas logarítmicas}
\index{derivada!logarítmica}
Vimos que derivar uma soma é mais simples do que derivar um produto: a derivada da
soma se calcula termo a termo, enquanto para derivar o produto, é necessário
usar a regra de Leibniz repetitivamente.
Ora, lembramos que \emph{o logaritmo transforma produtos em soma}, e que esse
fato pode ser usado para simplificar as contas que aparecem para derivar um
produto. \\

Considere uma função $f$ definida como o produto de $n$ funções, que suporemos
todas positivas e deriváveis: 
$$f(x)=h_1(x)h_2(x)\dots h_n(x)\equiv \prod_{k=1}^nh_k(x)\,.$$ 
Para calcular $f'(x)$, calculemos
primeiro
$$\ln f(x)=\ln h_1(x)+\ln h_2(x)+\dots+\ln h_n(x)\equiv \sum_{k=1}^n\ln
h_k(x)\,,$$ 
e derivamos ambos lados
com respeito a $x$. Do lado esquerdo, usando a regra da cadeia, $(\ln
f(x))'=\frac{f'(x)}{f(x)}$. Derivando termo a termo do lado direito, obtemos
\begin{align*}
\frac{f'(x)}{f(x)}&=(\ln h_1(x)+\ln h_2(x)+\dots+\ln h_n(x))'\\
&=(\ln h_1(x))'+(\ln h_2(x))'+\dots+(\ln h_n(x))'\\
&=\frac{h_1'(x)}{h_1(x)}+\frac{h_2'(x)}{h_2(x)}+\dots+\frac{h_n'(x)}{h_n(x)}\,.
\end{align*}
Logo, obtemos uma fórmula
$$
f'(x)=f(x)\Bigl(
\frac{h_1'(x)}{h_1(x)}+\frac{h_2'(x)}{h_2(x)}+\dots+\frac{h_n'(x)}{h_n(x)}
\Bigr)
$$

\begin{exo}
Derive, usando o método sugerido acima:
\begin{multicols}{3}
\begin{enumerate}
\item\label{itderitruclog1} $\frac{(x+1)(x+2)(x+3)}{(x+4)(x+5)(x+6)}$
\item\label{itderitruclog2} $\frac{x\sen^3x}{\sqrt{1+\cos^2x}}$
\item\label{itderitruclog3} $\prod_{k=1}^n(1+x^k)$
\end{enumerate}
\end{multicols}
\vspace{0.01cm}
\begin{sol} As derivadas são dadas por:
\eqref{itderitruclog1}
$\frac{(x+1)(x+2)(x+3)}{(x+4)(x+5)(x+6)}
(\frac{1}{x+1}+\frac{1}{x+2}+\frac{1}{x+3}-\frac{1}{x+4}-\frac{1}{x+5}-\frac{1}{
x+6})$
\eqref{itderitruclog2}
$\frac{x\sen^3x}{\sqrt{1+\cos^2x}}\bigl(\frac{1}{x}+3\cot x+\frac{\sen 
x\cos x}{{1+\cos^2x}}\bigr)$
\eqref{itderitruclog3}
$\bigl(\prod_{k=1}^n(1+x^k)\bigr)\sum_{k=1}^n\frac{kx^{k-1}}{1+x^k}$
\end{sol}
\end{exo}


\subsection{Derivar uma função inversa}\label{Sec:DerivInversa}
Sabemos que $(\sen x)'=\cos x$ e $(a^x)'=(\ln a)a^x$, mas como derivar as suas
respectivas funções inversas, isto é, $(\arcsen x)'$ e $(\log_ax)'$?\\
 
Vimos que o inverso de uma função $f$, quando é bem definido, satisfaz às
relações:
$$\forall x,\quad (f(f^{-1}(x))=x\,.$$
Logo, derivando em ambos lados com respeito a $x$, e usando a regra da cadeia
do lado esquerdo,
$$
f'(f^{-1}(x))\cdot (f^{-1})'(x)=1
$$
Logo,
$$\boxed{(f^{-1})'(x)=\frac{1}{f'(f^{-1}(x))}\,.}$$

\begin{ex}
Calculemos a derivada do $\arcsen x$, que é por definição a inversa da
função $f(x)=\sen x$, e bem definida para $x\in [-1,1]$. Como $f'(x)=\cos x$, a
fórmula acima dá
$$
(\arcsen x)'=\frac{1}{f'(f^{-1}(x))}=\frac{1}{\cos(\arcsen
x)}\,.$$
Usando a identidade provada no Exemplo
\ref{Ex:identidadesenoinverso}: $\cos(\arcsen
x)=\sqrt{1-x^2}$, obtemos
\eq{\boxed{(\arcsen x)'=\frac{1}{\sqrt{1-x^2}}\,.}}
Observe que, como pode ser visto no gráfico da Seção
\ref{Sec:Functriginversas}, as retas tangentes ao gráfico de $\arcsen x$ são
verticais nos pontos $x=\pm 1$, o que se traduz pelo fato de $(\arcsen x)'$ não
existir nesses pontos.
\end{ex}

\begin{exo}
Mostre que 
\eq{\boxed{(\log_ax)'=\frac{1}{(\ln a)x}\,,\quad\quad(\arcos
x)'=\frac{-1}{\sqrt{1-x^2}}\,,\quad\quad (\arctan
x)'=\frac{1}{1+x^2}\,.}}
\vspace{0.01cm}
\end{exo}

\begin{exo}
 Calcule as derivadas das funções abaixo.
\begin{multicols}{2}
\begin{enumerate}
\item\label{itderivfuncinv1} $\log_a(1-x^2)$
\item\label{itderivfuncinv2} $\arcsen(1-x^2)$
\item\label{itderivfuncinv3} $\arctan (\tan x)$, $-\pisobredois<x<\pisobredois$
\item\label{itderivfuncinv4} $\arcsen(\cos x)$, $0<x<\pisobredois$
\item\label{itderivfuncinv5} $\cos(\arcsen x)$, $-1<x<1$
\end{enumerate}
\end{multicols}
\vspace{0.01cm}
\begin{sol}
\eqref{itderivfuncinv1} $\frac{-2x}{(\ln a)(1-x^2)}$
\eqref{itderivfuncinv2} $\frac{-2x}{\sqrt{1-(1-x^2)^2}}$
\eqref{itderivfuncinv3} $1$
\eqref{itderivfuncinv4} $-1$
\eqref{itderivfuncinv5} $\frac{-x}{\sqrt{1-x^2}}$
\end{sol}
\end{exo}

\begin{exo}
Seja $f(x)=\arcos(\frac{1-x^2}{1+x^2})$. Mostre que a reta 
de equação $y=-\frac{3}{2}(x+\frac{1}{\sqrt{3}})+\frac{\pi}{3}$
é tangente ao gráfico de $f$ em algum ponto $P$.
\end{exo}

\section{O Teorema de Rolle}
\index{Teorema!de Rolle}
A seguinte afirmação geométrica é intuitiva:
se $A$ e $B$ são dois pontos de mesma altura (isto é: com a mesma segunda
coordenada) no gráfico de
uma função diferenciável $f$, então existe pelo menos um ponto $C$ no gráfico
de $f$, entre $A$ e $B$, tal que a reta tangente ao gráfico 
em $C$ seja horizontal.
Em outras palavras: 
\begin{teo}\label{Teo:Rolle}
Seja $f$ uma função contínua em $[a,b]$ e derivável em $(a,b)$. Se
$f(a)=f(b)$, então
existe $c\in (a,b)$ tal que $$f'(c)=0\,.$$
\end{teo}

\begin{ex}
Considere $f(x)=\sen x$, e $a=0$, $b=\pi$. Então $f(a)=f(b)$. Nesse caso, o
ponto $c$ cuja existência é garantida pelo teorema é
$c=\pisobredois$:
\begin{center}
\begin{bmlimage}\begin{tikzpicture}
%\newcommand{\funcao}[1]{(#1)^2}
%\newcommand{\dfuncao}[2]{ (\funcao{#1+#2})/{#2}-(\funcao{#1})/{#2}}
\draw[ ->, thin] (-0.3,0)--(3.5,0);
\draw[ ->, thin] (0,0)--(0,1.2);
\draw (0,0) node[below]{$\scriptstyle{0}$};
\draw (0,0) node[above left]{$A$};
\draw[dotted] (1.57,0)--(1.57,1);
\draw (1.57,0) node[below]{$\scriptstyle{\tfrac{\pi}{2}}$};
\draw (1.57,1) node[above]{$C$};
\draw (3.1415,0) node[below]{$\scriptstyle{{\pi}}$};
\draw (3.1415,0) node[above right]{$B$};
\draw[color=gray, domain=-0.2:3.4] plot (\x,{sin(\x r)});
\draw[thick, domain=0:3.1415] plot (\x,{sin(\x r)});
\draw[thick] ({1.57-0.4},1)--({1.57+0.4},1);
\fill (1.57,1) circle (0.40mm);
\fill (3.1415,0) circle (0.40mm);
\fill (3.1415,0) circle (0.40mm);
\end{tikzpicture}\end{bmlimage}
\end{center}
De fato, $f'(x)=\cos x$, logo $f'(\pisobredois)=0$.
\end{ex}

\begin{exo}
Em cada um dos casos a seguir, mostre que a afirmação do Teorema de Rolle é
verificada, achando explicitamente o ponto $c$.
\begin{multicols}{2}
\begin{enumerate}
\item\label{itRolleA1} $f(x)=x^2+x$, $a=-2$, $b=1$.
\item\label{itRolleA2} $f(x)=\cos x$, $a=-\frac{3\pi}{2}$, $b=\frac{3\pi}{2}$
\item\label{itRolleA3} $f(x)=x^4+x$, $a=-1$, $b=0$.
\end{enumerate}
\end{multicols}
\vspace{0.01cm}
\begin{sol}
(O gráfico da função pode ser usado para interpretar o resultado.)
\eqref{itRolleA1} Temos $f(-2)=f(1)$, e como $f'(x)=2x+1$, vemos que a derivada
se anula em $c=-\frac{1}{2}\in (-2,1)$.
\eqref{itRolleA2} Aqui são três pontos possíveis: $c=-\pi$, $c=0$ e $c=+\pi$.
\eqref{itRolleA3} Temos $f(-1)=f(0)$ e $f'(x)=4x^3+1$, cuja raiz é 
$-(\frac14)^{1/3}\in (-1,0)$.
\end{sol}
\end{exo}

Como consequência do Teorema de Rolle,
\index{Teorema!do valor intermediário para derivada}
\begin{cor}\label{Corol:ValorIntermDeriv}
  Seja $f$ uma função contínua em $[a,b]$, derivável em $(a,b)$. Então
existe $c\in (a,b)$ tal que
$$\frac{f(b)-f(a)}{b-a}=f'(c)\,.$$
\end{cor}
\begin{proof}
 Defina $\tilde{f}(x)\pardef f(x)-\frac{f(b)-f(a)}{b-a}(x-a)$. 
Então $\tilde{f}$ é diferenciável, e 
como $\tilde{f}(a)=\tilde{f}(b)=f(a)$, pelo Teorema de Rolle existe um
$c\in [a,b]$ tal que $\tilde{f}'(c)=0$.
Mas como $\tilde{f}'(x)=f'(x)-\frac{f(b)-f(a)}{b-a}$, temos 
$f'(c)-\frac{f(b)-f(a)}{b-a}=0$.
\end{proof}

\begin{wrapfigure}{r}{5cm}
\vspace{-30pt}
\begin{center}
\parbox{5cm}{
\begin{bmlimage}\begin{tikzpicture}
\pgfmathsetmacro{\al}{2.5};
\pgfmathsetmacro{\bl}{2};
\newcommand{\funcao}[1]{(\bl-(#1-\al)^2)}
\newcommand{\dfuncao}[2]{ (\funcao{#1+#2})/(#2)-(\funcao{#1})/(#2)}
\pgfmathsetmacro{\a}{1.2};
\pgfmathsetmacro{\b}{3.4};
\pgfmathsetmacro{\c}{\al-(\funcao{\b}-\funcao{\a})/(2*(\b-\a))};
\draw[ ->] (0,0)--(4,0); 
\draw[ ->] (0,0)--(0,2); 
\draw[thick, domain=\a:\b] plot (\x,{\funcao{\x}}); 
\coordinate (A) at (\a,{\funcao{\a}});
\coordinate (B) at (\b,{\funcao{\b}});
\coordinate (C) at (\c,{\funcao{\c}});
\fill (A) circle (0.4mm);
\fill (B) circle (0.4mm);
\draw[dotted] (\a,0) node[below]{$\scriptstyle{a}$}--(A);
\draw[dotted] (\c,0) node[below]{$\scriptstyle{c}$}--(C);
\draw[dotted] (\b,0) node[below]{$\scriptstyle{b}$}--(B);
\pgfmathsetmacro{\m}{\dfuncao{\c}{0.01}};
\draw[thick,  domain={\c-0.5}:{\c+0.5}] plot
(\x,{\funcao{\c}+\m*(\x-\c)});
\draw (A) node[left]{$A$};
\draw (B) node[right]{$B$};
\draw (C) node[above]{$C$};
\fill (C) circle (0.4mm);
\draw[dashed] (A)--(B);
\end{tikzpicture}\end{bmlimage}
}
\end{center}
\end{wrapfigure}
Geometricamente, o Corolário \ref{Corol:ValorIntermDeriv}
representa um \emph{Teorema do valor intermediário} para a derivada: se
$A\pardef (a,f(a))$, $B\pardef (b,f(b))$,
o corolário afirma  \emph{que existe um ponto $C$ no gráfico de $f$, entre $A$
e $B$, em que a inclinação da reta tangente em $C$ ($f'(c)$) 
é igual à inclinação do segmento $AB$ ($\frac{f(b)-f(a)}{b-a}$)}. 

\vspace{1mm}
\begin{ex}
Considere por exemplo $f(x)=x^2$ no intervalo $[0,2]$.
%\begin{wrapfigure}{r}{5cm}
%\vspace{-30pt}
\begin{center}
%\parbox{5cm}{
%\begin{minipage}[l]{5cm}
\begin{bmlimage}\begin{tikzpicture}[scale=0.8]
\pgfmathsetmacro{\al}{2.5};
\pgfmathsetmacro{\bl}{2};
\newcommand{\funcao}[1]{(#1)^2}
\newcommand{\dfuncao}[2]{ (\funcao{#1+#2})/(#2)-(\funcao{#1})/(#2)}
\pgfmathsetmacro{\a}{0};
\pgfmathsetmacro{\b}{2};
\pgfmathsetmacro{\c}{(\funcao{\b}-\funcao{\a})/(2*(\b-\a))};
\draw[ ->] (-0.2,0)--(2.5,0); 
\draw[ ->] (0,-0.2)--(0,4); 
\draw[thick, domain=\a:\b] plot (\x,{\funcao{\x}}); 
\coordinate (A) at (\a,{\funcao{\a}});
\coordinate (B) at (\b,{\funcao{\b}});
\coordinate (C) at (\c,{\funcao{\c}});
\draw (A) node[above left]{$A$};
\draw (B) node[right]{$B$};
\draw (C) node[right]{$C$};
\fill (A) circle (0.4mm);
\fill (B) circle (0.4mm);
%\draw[dotted] (\a,0) node[below]{$\scriptstyle{a}$}--(A);
\draw[dotted] (\c,0) node[below]{$\scriptstyle{c}$}--(C);
\draw[dotted] (\b,0) node[below]{$\scriptstyle{2}$}--(B);
\pgfmathsetmacro{\m}{\dfuncao{\c}{0.01}};
\draw[thick,  domain={\c-0.35}:{\c+0.35}] plot
(\x,{\funcao{\c}+\m*(\x-\c)});
%\draw (A) node[left]{$A$};
%\draw (B) node[right]{$B$};
%\draw (C) node[above]{$C$};
\fill (C) circle (0.4mm);
\draw[dashed] (A)--(B);
\end{tikzpicture}\end{bmlimage}
%}
%\end{minipage}
\end{center}
%\end{wrapfigure}
A construção geométrica de $C$ é clara: traçamos a reta paralela a $AB$,
tangente à parábola.
Neste caso a posição do ponto $C=(c,f(c))$ pode ser {calculada} explicitamente: 
como $f'(x)=2x$, e como $c$ satisfaz $f'(c)=\frac{2^2-0^2}{2-0}=2$, temos
$2c=2$, isto é: $c=1$.
\end{ex}

\begin{exo}
Considere $f(x)=\sen x$, com $a=-\pisobredois$, $b=\pisobredois$. Ache
graficamente o ponto $C$ e em seguida, calcule-o usando uma calculadora.
\begin{sol}
Vemos que existem dois pontos $C$ em que a inclinação é igual à inclinação do
segmento $AB$:
\begin{center}
\begin{bmlimage}\begin{tikzpicture}
\newcommand{\funcao}[1]{sin(#1 r)}
\newcommand{\dfuncao}[2]{ (\funcao{#1+#2})/{#2}-(\funcao{#1})/{#2}}
\pgfmathsetmacro{\a}{-1.57};
\pgfmathsetmacro{\b}{1.57};
\pgfmathsetmacro{\c}{0.8};
\pgfmathsetmacro{\cc}{-\c};
\draw[ ->] (-2,0)--(2,0); 
\draw[ ->] (0,-1)--(0,1); 
% \draw[color=gray, domain=\a-1.3:\b+1.3] plot (\x,{\funcao{\x}}); 
\draw[thick, domain=\a:\b] plot (\x,{\funcao{\x}}); 
\coordinate (A) at (\a,{\funcao{\a}});
\coordinate (B) at (\b,{\funcao{\b}});
\coordinate (C) at (\c,{\funcao{\c}});
\coordinate (Cc) at (\cc,{\funcao{\cc}});
\fill (A) circle (0.4mm);
\fill (B) circle (0.4mm);
\pgfmathsetmacro{\m}{2/3.1415};
\draw[thick,  domain={\c-0.5}:{\c+0.5}] plot
(\x,{\funcao{\c}+\m*(\x-\c)});
\draw[thick,  domain={\cc-0.5}:{\cc+0.5}] plot
(\x,{\funcao{\cc}+\m*(\x-\cc)});
\draw (A) node[left]{$A$};
\draw (B) node[right]{$B$};
\draw (C) node[above]{$C$};
\fill (C) circle (0.4mm);
\draw (Cc) node[below]{$C'$};
\fill (Cc) circle (0.4mm);
\draw[dashed] (A)--(B);
\end{tikzpicture}\end{bmlimage}
\end{center}
O ponto $c\in [-\pisobredois,\pisobredois]$ é tal que
$f'(c)=\frac{f(b)-f(a)}{b-a}=\frac{\sen
(\pisobredois)-\sen(0)}{\pisobredois-0}=\frac{2}{\pi}$. Como $f'(x)=\cos x$, $c$
é solução de $\cos c=\frac{2}{\pi}$. Com a calculadora obtemos duas
soluções: $c=\pm \arcos(\frac{2}{\pi})\simeq \pm 0.69$.
\end{sol}
\end{exo}

\begin{exo}
Considere a função $f$ definida por $f(x)=\frac{x}{2}$ se $x\leq 2$, $f(x)=x-1$
se $x>2$, e $A=(0,f(0))$, $B=(3,f(3))$. Existe um
ponto $C$ no gráfico de $f$, entre $A$ e $B$, tal que a reta tangente ao
gráfico em $C$ seja paralela ao segmento $AB$?
Explique.
\begin{sol}
Como $f$ não é derivável no ponto $2\in [0,3]$, o teorema não se aplica. Não
existe ponto $C$ com as desejadas propriedades:
\begin{center}
\begin{bmlimage}\begin{tikzpicture}[scale=0.7]
\draw[->] (-0.5,0)--(3.5,0);
\draw[->] (0,-0.2)--(0,2.3);
\coordinate (A) at (0,0);
\coordinate (B) at (3,2);
\draw (-0.3,-0.15)--(2,1)--(3.3,2.3);
\draw[thick] (A)--(2,1)--(B);
\fill (A) circle (0.45mm);
\fill (B) circle (0.45mm);
\draw (A)node[above left]{$A$};
\draw (B)node[below right]{$B$};
\draw[dotted] (2,1)--(2,0) node[below]{$\scriptstyle{2}$};
\end{tikzpicture}\end{bmlimage}
\end{center}

\end{sol}
\end{exo}

\begin{exo}\label{exo_DERIV_senoLipschitz}
Mostre que para todo par de pontos $x_1,x_2$, 
vale a seguinte desigualdade:
\begin{equation}\label{eq_DERIV_sinussLipshh} 
|\sen x_2-\sen x_1|\leq |x_2-x_1|\,.
\end{equation}
Use esse fato para mostrar que $\sen x$ é uma função contínua.
Faça a mesma coisa com $\cos x$.
\begin{sol}
Sejam $x_1<x_2$. Pelo Corolário \ref{Corol:ValorIntermDeriv}, existe $c\in
(x_2,x_2)$ tal que
\[
\frac{\sen x_2-\sen x_1}{x_2-x_1}=\cos(c)\,.
\]
Como $|\cos (c)|\leq 1$, isso dá \eqref{eq_DERIV_sinussLipshh}.
Por ser derivável, já sabemos que $\sen x$ é contínua, mas \eqref{eq_DERIV_sinussLipshh} 
permite ver continuidade de uma maneira mais concreta. De fato, 
seja $a$ um ponto qualquer da reta. Para mostrar que $\sen x$ é contínua em
$a$, precisamos escolher um $\epsilon>0$ qualquer, e mostrar que se $x$ for
suficientemente perto de $a$, $|x-a|\leq \delta$ (para um certo $\delta$) então 
\[ 
|\sen x-\sen a|\leq \epsilon\,.
\]
Mas, usando \eqref{eq_DERIV_sinussLipshh}, vemos que a condição acima
vale se $\delta\equiv \epsilon$.
\end{sol}
\end{exo}

\section{Derivada e Variação}
\index{derivada!e variação}
Voltemos agora ao significado geométrico da derivada, e do seu uso no estudo de
funções.
Sabemos que para um ponto $x$ do domínio de uma função $f$,
a derivada $f'(x)$ (se existir) dá o valor da inclinação da reta tangente
ao gráfico de $f$ no ponto $(x,f(x))$.\\

A observação importante para ser feita aqui é que os valores
de $f'$ fornecem uma informação importante sobre a \emph{variação} de $f$, isto
é, sobre os intervalos em que ela cresce ou decresce (veja Seção \ref{sec_Funcoes_variacao}).
\index{variação}

\begin{ex}
Considere $f(x)=x^2$. 
\begin{center}
\begin{bmlimage}\begin{tikzpicture}
\newcommand{\funcao}[1]{(#1)^2}
\draw[ ->, thin] (-1.3,0)--(1.3,0);
\draw[ ->, thin] (0,-0.2)--(0,1.8);
\draw[thick, domain=-1.2:1.2] plot (\x,{\funcao{\x}});
\end{tikzpicture}\end{bmlimage}
\end{center}
Vemos que $f$ \emph{decresce} no intervalo
$(-\infty,0]$, e \emph{cresce} no intervalo
$[0,+\infty)$. Esses fatos se refletem nos valores da
inclinação da reta tangente: de fato, quando a função decresce, a
\emph{inclinação da sua reta tangente é negativa}, $f'(x)<0$, e 
quando a função cresce, a
\emph{inclinação da sua reta tangente é positiva}, $f'(x)>0$:
\begin{center}
\begin{bmlimage}\begin{tikzpicture}[scale=1.3]
\newcommand{\funcao}[1]{(#1)^2}
\newcommand{\dfuncao}[2]{ (\funcao{#1+#2})/(#2)-(\funcao{#1})/(#2)}
\draw[ ->, thin] (-1.3,0)--(1.3,0);
\draw[ ->, thin] (0,-0.2)--(0,1.8);
\draw[color=gray, domain=-1.2:1.2] plot (\x,{\funcao{\x}});
\pgfmathsetmacro{\e}{0.2};
\foreach \a in {-1,-0.7,-0.3, 1,0.7,0.3} {
\pgfmathsetmacro{\r}{\dfuncao{\a}{0.01}};
\pgfmathsetmacro{\i}{\e/(sqrt(1+\r*\r))};
%\pgfmathsetmacro{\i}{0.1};
\draw[thick,  domain={\a-\i}:{\a+\i}] plot
(\x,{(\dfuncao{\a}{0.01})*(\x-\a)+\funcao{\a}});
\fill (\a,{\funcao{\a}}) circle (0.35mm);
}
%\draw[dotted] (\e,0)--(\e,{\funcao{\e}});
\draw [decorate, decoration=brace] (-0.05,-0.1)--(-1.3,-0.1) node[midway,
below]{$\scriptstyle{f'(x)<0}$};
\draw [decorate, decoration=brace] (1.3,-0.1)--(0.05,-0.1) node[midway,
below]{$\scriptstyle{f'(x)>0}$};
%\draw (-1,{\funcao{-1}}) node[below left]{$\scriptstyle{f'(x)<0}$};
\end{tikzpicture}\end{bmlimage}
\end{center}
Como $f'(x)=2x$, 
montemos uma \emph{tabela de variação}, relacionando o sinal de $f'(x)$ com a
variação de $f$:
\begin{center}
\begin{bmlimage}\begin{tikzpicture}[scale=0.8]
\tkzTabInit[nocadre, espcl=2,  color, colorV=lightgray!5, colorL=gray!15,
colorC=gray!15]
{$x$ /.6, $f'(x)$ /.6, Variaç. de $f$ /1.2}%
{,$0$, }%
\tkzTabLine{,-,z,+,}
\tkzTabVar{+/,-/,+/}
%\tkzTabLine{,\searrow,\text{mín.},h,\text{mín.},\nearrow,}
\end{tikzpicture}\end{bmlimage}
\end{center}
em que ``$\searrow$'' significa que $f$ decresce e ``$\nearrow$'' que ela
cresce no intervalo.
Vemos também que em $x=0$, como a derivada muda de negativa para positiva, a
função atinge o seu valor mínimo, e nesse ponto $f'(0)=0$.
\end{ex}

No exemplo anterior, começamos com uma função conhecida ($x^2$), e observamos
que a sua variação é diretamente ligada \emph{ao sinal da
sua derivada}.
Nesse capítulo faremos o contrário: a partir de uma função dada $f$,
estudaremos o sinal da sua derivada, deduzindo a variação de $f$ de maneira
\emph{analítica}. Junto com outras propriedades básicas de $f$,
como o seu sinal e as suas assíntotas, isto permitirá esboçar o gráfico de $f$
com bastante precisão.

Vejamos agora, de maneira precisa, como a 
variação de uma função diferenciável pode ser obtida estudando o sinal da sua
derivada:

\begin{pro}\label{Prop:variacaosinalflinha}
Seja $f$ uma função derivável em $I$.
\index{derivada!e variação}
\begin{itemize}
\item Se $f'(z)\geq  0$ para todo $z\in
I$, então $f$ é crescente em $I$.
\item Se $f'(z)> 0$ para todo $z\in
I$, então $f$ é estritamente crescente em $I$.
\item Se $f'(z)\leq  0$ para todo $z\in
I$, então $f$ é decrescente em $I$.
\item Se $f'(z)< 0$ para todo $z\in
I$, então $f$ é estritamente decrescente em $I$.
\end{itemize}
\end{pro}

\begin{proof}
Provaremos somente a primeira afirmação (as outras se provam da mesma maneira).
Suponha que $f'(z)\geq 0$ para todo $z\in I$. Sejam $x,x'$ dois pontos
quaisquer em $I$, tais que
$x<x'$. Pelo Corolário \ref{Corol:ValorIntermDeriv}, existe $c\in [x,x']$ tal
\index{Teorema!do valor intermediário para derivada}
que
$$\frac{f(x')-f(x)}{x'-x}=f'(c)\,.$$
Como $f'(c)\geq 0$ por hipótese, temos $f(x')-f(x)=f'(c)(x'-x)\geq 0$, isto é,
$f(x')\geq f(x)$. Isso implica que $f$ é crescente em $I$.
\end{proof}

\begin{ex}
Estudemos a variação de $f(x)=\frac{x^3}{3}-x$, usando a proposição acima.
A derivada de $f$ é dada por $f'(x)=x^2-1$, seu sinal é fácil de se estudar, e 
permite determinar a variação de $f$:
\begin{center}
\begin{bmlimage}\begin{tikzpicture}[scale=0.8]
\tkzTabInit[nocadre, espcl=2,  color, colorV=lightgray!5, colorL=gray!15,
colorC=gray!15]
{$x$ /.6, $f'(x)$ /.6, Variaç. de $f$ /1.2}%
{,$-1$, $+1$,}%
\tkzTabLine{,+,z,-,z,+,}
\tkzTabVar{-/,+/,-/,+/}
\end{tikzpicture}\end{bmlimage}
\end{center}
Isto é: $f$
cresce em $(-\infty,-1]$ até o ponto de coordenadas $(-1,f(-1))=(-1,\tfrac23)$,
depois decresce em $[-1,+1]$ até o ponto de coordenadas
$(1,f(1))=(1,-\tfrac23)$, e depois cresce de novo em $[+1,\infty)$:
\begin{center}
\begin{bmlimage}\begin{tikzpicture}[scale=1.2]
\draw[ ->, thin] (-2.2,0)--(2.2,0);
\draw[ ->, thin] (0,-1)--(0,1);
\draw[thick, domain=-2:2, samples=50] plot (\x,{\x^3/3-\x});
\fill (-1.73,0) circle (0.40mm);
\fill (1.73,0) circle (0.40mm);
\fill (0,0) circle (0.40mm);
\fill (-1,0.66) circle (0.40mm);
\fill (1,-0.66) circle (0.40mm);
\draw (-1,0.66) node[above]{$\scriptstyle{(-1,\tfrac{2}{3})}$};
\draw (1,-0.66) node[below]{$\scriptstyle{(+1,-\tfrac{2}{3})}$};
\end{tikzpicture}\end{bmlimage}
\end{center}
Observe que em geral, o estudo da derivada não dá informações sobre os zeros da função. 
No entanto, neste caso,
os zeros de $f$ podem ser calculados:
$\frac{x^3}{3}-x=x(\frac{x^2}{3}-1)=0$. Isto é: $S=\{-\sqrt{3},0,\sqrt{3}\}$.
Logo, o sinal de $f$ (que não tem nada a ver com o sinal de $f'$) obtém-se facilmente:
\begin{center}
\begin{bmlimage}\begin{tikzpicture}[scale=0.8]
\tkzTabInit[nocadre, espcl=2,  color, colorV=lightgray!5, colorL=gray!15,
colorC=gray!15]
{$x$ /.6, $f(x)$ /.7}%
{,$-\sqrt{3}$, $0$, $+\sqrt{3}$,}%
\tkzTabLine{,-,z,+,z,-,z,+}
\end{tikzpicture}\end{bmlimage}
\end{center}

\end{ex}

\begin{ex}
Considere as potências $f(x)=x^p$, com $p\in \bZ$ (lembre os esboços da Seção
\ref{Sec:GraficosPotencias}). Temos que $(x^p)'=px^{p-1}$ se $p>0$,
$(\frac{1}{x^q})'=-qx^{-q-1}$ se $p=-q<0$.
\begin{itemize}
\item Se $p>0$ é par, então $p-1$ é ímpar, e $(x^p)'<0$ se $x<0$,
$(x^p)'>0$ se $x>0$. Logo, $x^p$ é decrescente em $(-\infty, 0]$,
crescente em $[0,\infty)$. (Por exemplo: $x^2$.)
\item Se $p>0$ é ímpar, então $p-1$ é par, e $(x^p)'\geq 0$ para todo
$x$. Logo, $x^p$ é crescente em todo $\bR$. (Por exemplo: $x^3$.)
\item Se $p=-q<0$ é par, então $-q-1$ é ímpar, e $(\frac{1}{x^q})'>0$
se $x<0$, $(\frac{1}{x^q})'<0$ se $x>0$. 
Logo, $\frac{1}{x^q}$ é crescente em $(-\infty, 0)$, e decrescente em
$(0,\infty)$. (Por exemplo: $\frac{1}{x^2}$.)
\item Se $p=-q<0$ é ímpar, então $-q-1$ é par, e $(\frac{1}{x^q})'<0$
para todo $x\neq 0$.
Logo, $\frac{1}{x^q}$ é decrescente em $(-\infty, 0)$, e decrescente também em
$(0,\infty)$. (Por exemplo: $\frac{1}{x}$ ou $\frac{1}{x^3}$.)
\end{itemize}
\end{ex}


\begin{ex}
Considere a função exponencial na base $a>0$, $a^x$ (lembre os esboços da Seção
\ref{Sec:Exponencial}). Como $(a^x)'=(\ln a)a^x$, temos que
\begin{itemize}
 \item se $a>1$, então $\ln a>0$, e $(a^x)'>0$ para todo $x$. Logo, $a^x$ é
sempre crescente.
 \item se $0<a<1$, então $\ln a<0$, e $(a^x)'<0$ para todo $x$. Logo, $a^x$ é
sempre decrescente.
\end{itemize}
Por outro lado, a função logaritmo na base $a>0$, $\log_ax$, é tal que
$(\log_ax)'=\frac{1}{x\ln a}$. 
\begin{itemize}
 \item Se $a>1$, então $\log_ax$ é crescente em $(0,\infty)$, e
 \item se $0<a<1$, então $\log_ax$ é decrescente em $(0,\infty)$.
\end{itemize}
\end{ex}



\begin{exo}\label{Ex:variacoesbasicas}
Estude a variação de $f$, usando a sua derivada, quando for possível.
Em seguida, junto com outras informações (p.ex. zeros, sinal de $f$), 
monte o gráfico de $f$.
\begin{multicols}{3}
\begin{enumerate}
%\item\label{itestudfunceleme1} $f(x)=\frac{x^3}{3}-x$
\item\label{itestudfunceleme2} $f(x)=\frac{x^4}{4}-\frac{x^2}{2}$
\item\label{itestudfunceleme21} $f(x)=\scriptstyle{2x^3-3x^2-12x+1}$
\item\label{itestudfunceleme22} $f(x)=|x+1|$
\item\label{itestudfunceleme3} $f(x)=||x|-1|$
\item\label{itestudfunceleme4} $f(x)=\sen x$
\item\label{itestudfunceleme5} $f(x)=\sqrt{x^2-1}$ 
\item\label{itestudfunceleme6} $f(x)=\frac{x+1}{x+2}$
\item\label{itestudfunceleme61} $f(x)=\frac{x-1}{1-2x}$
\item\label{itestudfunceleme7} $f(x)=e^{-\frac{x^2}{2}}$
\item\label{itestudfunceleme9} $f(x)=\ln(x^2)$
\item\label{itestudfunceleme10} $f(x)=\tan x$
\end{enumerate}
\end{multicols}
\vspace{0.01cm}
\begin{sol}

\eqref{itestudfunceleme2}: Como $f'(x)=x^3-x=x(x^2-1)$,
$f(x)$ é crescente em $[-1,0]\cup [1,\infty)$,
decrescente em $(-\infty,-1]\cup[0,1]$:
\begin{center}
\begin{bmlimage}\begin{tikzpicture}[scale=1.2]
\draw[ ->, thin] (-2.2,0)--(2.2,0);
\draw[ ->, thin] (0,-0.6)--(0,1);
\draw[thick, domain=-1.8:1.8, samples=50] plot (\x,{(\x)^4/4-(\x)^2/2});
 \fill (-1.414,0) circle (0.40mm);
 \fill (1.414,0) circle (0.40mm);
 \fill (0,0) circle (0.40mm);
 \fill (-1,-0.25) circle (0.40mm);
 \fill (1,-0.25) circle (0.40mm);
 \draw (-1,-0.25) node[below]{$\scriptstyle{(-1,-\tfrac{1}{4})}$};
 \draw (1,-0.25) node[below]{$\scriptstyle{(+1,-\tfrac{1}{4})}$};
\end{tikzpicture}\end{bmlimage}
\end{center}
\eqref{itestudfunceleme21}: $f(x)=2x^3-3x^2-12x+1$ é 
crescente em $(-\infty,-1]\cup[2,\infty)$, decrescente em $[-1,2]$:
\begin{center}
\begin{bmlimage}\begin{tikzpicture}[scale=0.8]
\newcommand{\funcao}[1]{(2*(#1)^3-3*(#1)^2-12*(#1)+1)/10}
\draw[ ->, thin] (-2.2,0)--(4.2,0);
\draw[ ->, thin] (0,-1)--(0,1);
\draw[thick, domain=-2.5:3.5, samples=50] plot (\x,{\funcao{\x}});
\fill (-1,{0.8}) circle (0.5mm);
\draw (-1,0.8) node[above]{$\scriptstyle{(-1,8)}$};
\draw (2,-1.9) node[below]{$\scriptstyle{(2,-19)}$};
\fill (2,{-1.9}) circle (0.5mm);
\end{tikzpicture}\end{bmlimage}
\end{center}
Observe que nesse caso, a identificação dos pontos em que o gráfico corta o
eixo $x$ é mais difícil (precisa resolver uma equação do terceiro grau).
\eqref{itestudfunceleme22}: $f$ decresce em $(-\infty,-1]$, cresce em
$[-1,\infty)$. Observe que $f$ não é derivável em $x=-1$.
\eqref{itestudfunceleme3}: Já encontramos o gráfico dessa função no Exercício
\ref{Ex:graficosbasicos}. Observe que 
$f(x)=||x|-1|$ não é derivável em $x=-1,0,+1$, então é melhor estudar a variação
sem a derivada: $f$ é decrescente em $(-\infty,-1]$ e em $[0,1]$,
crescente em $[-1,0]$ e em $[1,\infty)$.
\eqref{itestudfunceleme4} Como $(\sen x)'=\cos x$, vemos que o seno é crescente
em cada intervalo em que o cosseno é positivo, e decrescente em cada intervalo
em que o cosseno é negativo. Por exemplo, no intervalo $[-\pisobredois,
\pisobredois]$, $\cos x>0$, logo $\sen x$ é crescente:
\begin{center}
\begin{bmlimage}\begin{tikzpicture}[scale=0.7]
\draw[thin,  ->] (-6.2,0)--(6.2,0);
\draw[thin,  ->] (0,-1.2)--(0,1.3);
\draw[color=gray, domain=-6:6, samples=50] plot (\x,{cos(\x r)});
\draw[thick, domain=-6:6, samples=50] plot (\x,{sin(\x r)});
\draw[dotted] (-1.57,-1.1)
node[below]{$\scriptstyle{-\tfrac{\pi}{2}}$}--(-1.57,1.1);
\draw[dotted]
(1.57,-1.1)node[below]{$\scriptstyle{\tfrac{\pi}{2}}$}--(1.57,1.1);
\end{tikzpicture}\end{bmlimage}
\end{center} 
\eqref{itestudfunceleme5}:
$f(x)=\sqrt{x^2-1}$ tem domínio $(-\infty,-1]\cup[1,\infty)$, é sempre
não-negativa, e $f(-1)=f(1)=0$. Temos $f'(x)=\frac{x}{\sqrt{x^2-1}}$. Logo,
a variação de $f$ é dada por:
\begin{center}
\begin{bmlimage}\begin{tikzpicture}[scale=0.8]
\tkzTabInit[nocadre, espcl=2,  color, colorV=lightgray!5, colorL=gray!15,
colorC=gray!15]
{$x$ /.6, $f'(x)$ /.9, Variaç. de $f$ /1.5}%
{,$-1$, $+1$,}%
%\tkzTabLine{+,z,h,z,+}
\tkzTabLine{,-,t,h,t,+,}
\tkzTabVar{+/,-H/,-/,+/,}
%\tkzTabLine{,\searrow,\text{mín.},h,\text{mín.},\nearrow,}
\end{tikzpicture}\end{bmlimage}
\end{center}
Assim, o gráfico é do tipo:
\begin{center}
\begin{bmlimage}\begin{tikzpicture}
\newcommand{\func}[1]{sqrt((#1)^2-1)}
\draw[ ->] (-2.5,0)--(2.5,0);
\draw[ ->] (0,-0.2)--(0,1.5);
\draw[thick, domain=-2:-1] plot (\x,{\func{\x}});
\draw[thick, domain=1:2] plot (\x,{\func{\x}});
\foreach \k in {-1,+1} {
\draw (\k,0) node[below]{$\k$};
}
\end{tikzpicture}\end{bmlimage}
\end{center}
Observe que $\lim_{x\to -1^-}f'(x)=-\infty$, $\lim_{x\to +1^+}f'(x)=+\infty$
\eqref{itestudfunceleme5}:
Considere $f(x)=\frac{x+1}{x+2}$. Como
$\lim_{x\to \pm\infty}f(x)=1$, $y=1$ é assíntota horizontal, e como $\lim_{x\to
-2^-}f(x)=+\infty$, $\lim_{x\to -2^+}f(x)=-\infty$, $x=-2$ é assíntota vertical.
Como $f'(x)=\frac{1}{(x+2)^2}>0$ para todo $x\neq 2$, $f$ é crescente em
$(-\infty,-2)$ e em $(-2,\infty)$. Isso permite montar o gráfico:
\begin{center}
\begin{bmlimage}\begin{tikzpicture}[scale=0.7]
\draw[ ->] (-5,0)--(4,0);
\draw[dashed] (-2,-1)node[left]{$\scriptstyle{x=-2}$}--(-2,3);
\draw[dashed] (-5,1)--(4,1) node[above]{$\scriptstyle{y=1}$};
\draw[ ->] (0,-1)--(0,2.5);
\pgfmathsetmacro{\e}{0.5};
\draw[thick, domain=-5:{-2-\e}, samples=50] plot (\x,{(\x+1)/(\x+2)});
\draw[thick, domain={-2+\e}:4, samples=50] plot (\x,{(\x+1)/(\x+2)});
\end{tikzpicture}\end{bmlimage}
\end{center}
\eqref{itestudfunceleme61}: Um estudo parecido dá
\begin{center}
\begin{bmlimage}\begin{tikzpicture}[scale=0.7]
\draw[ ->] (-4,0)--(4,0);
\draw[dashed] (0.5,-2)node[right]{$\scriptstyle{x=\tfrac{1}{2}}$}--(0.5,1.5);
\draw[dashed] (-4,-0.5)--(4,-0.5) node[below]{$\scriptstyle{y=\tfrac{1}{2}}$};
\draw[ ->] (0,-2)--(0,1.5);
\pgfmathsetmacro{\e}{0.16};
\draw[thick, domain=-4:{0.5-\e}, samples=50] plot (\x,{(\x-1)/(1-2*\x)});
\draw[thick, domain={0.5+\e}:4, samples=50] plot (\x,{(\x-1)/(1-2*\x)});
\end{tikzpicture}\end{bmlimage}
\end{center}
\eqref{itestudfunceleme7}: Como $f'(x)=-xe^{-\frac{x^2}{2}}$, 
$f$ é crescente em $(-\infty,0]$, decrescente em $[0,\infty)$.
Como $f(x)\to 0$ quando $x\to \pm \infty$, temos:
\begin{center}
\begin{bmlimage}\begin{tikzpicture}[scale=0.7]
\draw[ ->] (-4,0)--(4,0);
\draw[ ->] (0,-0.2)--(0,1.3);
\draw[thick, domain=-4:4, samples=50] plot (\x,{exp(-\x*\x*0.5)});
\end{tikzpicture}\end{bmlimage}
\end{center}
\eqref{itestudfunceleme9}: Observe que $\ln(x^2)$ tem domínio
$D=\bR\setminus\{0\}$, e $(\ln(x^2))'=\frac{2}{x}$. Logo, $\ln(x^2)$ é
decrescente em $(-\infty,0)$, crescente em $(0,\infty)$:
\begin{center}
\begin{bmlimage}\begin{tikzpicture}[scale=0.5]
\draw[ ->] (-4,0)--(4,0);
\draw[ ->] (0,-2.5)--(0,2);
\draw[thick, domain=0.3:4, samples=50] plot (\x,{2*ln(\x)});
\draw[thick, domain=0.3:4, samples=50] plot (-\x,{2*ln(\x)});
\end{tikzpicture}\end{bmlimage}
\end{center}
\eqref{itestudfunceleme10} 
Lembre que o domínio da tangente é formado pela união dos intervalos da forma
$I_k=]-\pisobredois+k\pi,\pisobredois+k\pi[$. 
Como $(\tan x)'=1+\tan^2x>0$ para todo $x\in I_k$, $\tan x$ é crescente em cada
intervalo do seu domínio (veja o esboço na Seção \ref{Sec:GraficosTrigo}).
\end{sol}
\end{exo}

\section{Velocidade, aceleração, taxa de
variação}\label{sec:taxavariacao}
\index{taxa de variação}
Sabemos que o \emph{sinal} da derivada (quando ela existe)
permite caracterizar o {crescimento de uma função}.
Mais especificamente, a derivada deve ser entendida como
\emph{taxa de variação}.
O exemplo mais importante do significado da derivada como taxa
de variação é em
mecânica, estudando o movimento de uma partícula.\\

Considere uma partícula que evolui na reta, durante um intervalo de tempo
$[t_1,t_2]$. 
Suponha que a sua posição no tempo $t_1$ seja $x(t_1)$,
que no tempo $t_2$ a sua posição seja $x(t_2)$, e que para
$t\in [t_1,t_2]$, a posição seja dada por uma função $x(t)$.
\begin{center}
\begin{bmlimage}\begin{tikzpicture}
\pgfmathsetmacro{\a}{0};
\pgfmathsetmacro{\b}{5};
\pgfmathsetmacro{\c}{2};
\pgfmathsetmacro{\r}{1};
\pgfmathsetmacro{\h}{0.3};
\draw[thick] ({\a-1},0)--(\b+1,0);
\draw (\a,0)node{$\shortmid$} node[below]{$x(t_1)$};
\draw (\b,0)node{$\shortmid$} node[below]{$x(t_2)$};
\draw (\c,0)node{$\shortmid$} node[below]{$x(t)$};
\draw[color=gray!20, dotted] (\a,\h)--(\c,\h);
\fill[color=gray!35] (\a,\h) circle (\r mm);
\fill[color=gray!35] (\b,\h) circle (\r mm);
\fill (\c,\h) circle (\r mm);
\draw[->] (\c,\h)--(\c+0.5,\h);
\end{tikzpicture}\end{bmlimage}
\end{center}
A função
$t\mapsto x(t)$, para $t\geq 0$, representa a \grasA{trajetória} da partícula.

\begin{center}
\begin{bmlimage}\begin{tikzpicture}
\newcommand{\funcao}[1]{(#1)^2/8+0.3}
\newcommand{\dfuncao}[2]{ (\funcao{#1+#2})/{#2}-(\funcao{#1})/{#2}}
\pgfmathsetmacro{\a}{1.5};
\coordinate (P) at (\a,{\funcao{\a}});
\pgfmathsetmacro{\l}{3.5};
\coordinate (Q) at (\l,{\funcao{\l}});
\draw[->] (0,0)--({\l+1},0)node[right]{$t$};
\draw[->] (0,-0.7)--(0,2)node[left]{$x(t)$};
\draw[thick, domain=\a:\l] plot (\x,{\funcao{\x}});
\draw[dashed] (P)--(Q);
\draw (P) node[left]{$x(t_1)$};
\fill (P) circle (0.50mm);
\draw (Q) node[right]{$x(t_2)$};
\fill (Q) circle (0.50mm);
\draw[dotted] (\a,0)node[below]{$t_1$}--(\a,{\funcao{\a}});
\draw[dotted] (\l,0)node[below]{$t_2$}--(\l,{\funcao{\l}});
\end{tikzpicture}\end{bmlimage}
\end{center}

Uma informação útil pode ser extraida da trajetória, 
olhando somente para o deslocamento entre o ponto inicial e o ponto final:
definimos a \grasA{velocidade média ao longo de $[t_1,t_2]$}, 
\index{velocidade !média}
$$
\overline{v}=\frac{x(t_2)-x(t_1)}{t_2-t_1}\,.
$$ 
A interpretação de $\overline{v}$ é a seguinte: se uma segunda partícula sair de
$x(t_1)$ no tempo $t_1$, se movendo a velocidade \emph{constante}
$\overline{v}$, então ela chegará em $x(t_1)$ no tempo $t_2$, junto com a
primeira partícula. A trajetória dessa segunda partícula de velocidade
constante $\overline{v}$ é representada pela linha pontilhada do desenho
acima.\\

Mas a primeira partícula não anda necessariamente com uma velocidade
constante. Podemos então perguntar: como calcular a sua \emph{velocidade
instantânea} \index{velocidade!instantânea}
num determinado instante $t_1<t<t_2$? 
Para isso, é necessário olhar as posições em dois instantes próximos. Se
a partícula se
encontra na posição $x(t)$ no tempo $t$, então logo depois, no instante
$t+\Delta t>t$, ela se encontrará na posição $x(t+\Delta t)$. Logo, a sua
velocidade média no intervalo $[t,t+\Delta t]$ é dada por $\frac{x(t+\Delta
t)-x(t)}{\Delta t}$. Calcular a \emph{velocidade instantânea} significa
calcular a velocidade média em intervalos de tempo $[t,t+\Delta t]$
infinitesimais:
$$v(t)=\lim_{\Delta t\to 0}\frac{x(t+\Delta
t)-x(t)}{\Delta t}\equiv x'(t)\,, $$
isto é, a derivada de $x(t)$ com respeito a $t$.\\

Vemos assim como a derivada aparece no estudo da cinemática: se $x(t)$
(em metros) é a posição da partícula no tempo $t$ (em
segundos), então a sua velocidade instantânea neste instante é $v(t)=x'(t)$
metros/segundo. 

\begin{obs}
Existe uma relação interessante entre 
velocidade instantânea e média. 
Como consequência do Teorema de Rolle
(e o seu Corolário \ref{Corol:ValorIntermDeriv}), se $x(t)$ for contínua
e derivável num intervalo $[t_1,t_2]$, então deve existir um instante
$t_*\in (t_1,t_2)$ tal que
\[\overline{v}=
\frac{x(t_2)-x(t_1)}{t_2-t_1}=x'(t_*)=v(t_*)\,.
\]
Isso implica que ao longo da sua trajetória entre $t_1$ e $t_2$, existe
pelo menos um instante $t_1<t_*<t_2$ em que a velocidade instantânea é igual à
velocidade média.
\end{obs}

\begin{ex}
Considere uma partícula cuja trajetória é dada por 
\begin{equation}\label{eq:trajhomog}
x(t)=v_0t+x_0\,,\quad t\geq 0
\end{equation}
em que $x_0$ e $v_0$ são constantes.
Como $x(0)=x_0$, $x_0$ é a posição inicial da partícula.
A velocidade instantânea é dada por 
\[
x'(t)=v_0\,,
\]
o que significa que a partícula se move com uma velocidade constante
$v_0$ ao longo da sua trajetória. Diz-se que apartícula segue um 
\emph{movimento retilíneo uniforme}\index{movimento!retilíneo uniforme}.

\begin{center}
\begin{bmlimage}\begin{tikzpicture}
\newcommand{\funcao}[1]{(#1)^2/8+0.3}
\newcommand{\dfuncao}[2]{ (\funcao{#1+#2})/{#2}-(\funcao{#1})/{#2}}
\pgfmathsetmacro{\a}{0};
\coordinate (P) at (\a,{\funcao{\a}});
\pgfmathsetmacro{\l}{3.5};
\coordinate (Q) at (\l,{\funcao{\l}});
\draw[->] (0,0)--({\l+1},0)node[right]{$t$};
\draw[->] (0,-0.7)--(0,2)node[left]{$x(t)$};
%\draw[color=gray, domain=\a:\l] plot (\x,{\funcao{\x}});
\draw[thick] (P)--(Q);
\draw (P) node[left]{$x_0$};
\fill (P) circle (0.50mm);
%\draw (Q) node[right]{$x(t_2)$};
%\fill (Q) circle (0.50mm);
%\draw[dotted] (\a,0)node[below]{$t_1$}--(\a,{\funcao{\a}});
%\draw[dotted] (\l,0)node[below]{$t_2$}--(\l,{\funcao{\l}});
\end{tikzpicture}\end{bmlimage}
\end{center}
Observe que nesse caso, a velocidade média ao longo de um intervalo é 
igual à velocidade instantânea: $\overline{v}=v_0$. 
\end{ex}

É natural considerar também a \emph{taxa de variação instantânea de
velocidade}, chamada\index{aceleração} \grasA{aceleração}:
$$a(t)=\lim_{\Delta t\to 0}\frac{v(t+\Delta
t)-v(t)}{\Delta t}\equiv v'(t)\,.$$
Por $a(t)$ ser a derivada da derivada de $x(t)$, é a \grasA{derivada segunda}
de $x$ com respeito a $t$, denotada: $a(t)=x''(t)$.\\

No exemplo anterior, em que uma partícula se movia com velocidade
constante $v_0$, a aceleração é igual a zero:
\[
x''(t)=(v_0t+x_0)''=(v_0)'=0\,.
\]

\begin{ex}
Uma partícula que sai da origem no tempo $t=0$ com uma velocidade inicial
$v_0>0$ e evolui sob o efeito de uma força constante $-F<0$ (tende a
freiar a partícula) tem uma trajetória dada por
\[x(t)=-\frac{F}{2m}t^2+v_0t+x_0\,,\quad t\geq 0\,,\] 
onde $m$ é a massa da partícula. Então a velocidade não é mais
constante, e decresce com $t$:
\[v(t)=x'(t)=-\frac{F}{m}t+v_0\,.\] 
A aceleração, por sua vez,
é constante: 
\[a(t)=v'(t)=-\frac{F}{m}\,.\] 
%Veja também o Exercício \ref{Exo:TrajetPartic}.
\end{ex}

\begin{exo}
Considere uma partícula cuja trajetória é dada por:
\begin{center}
\begin{bmlimage}\begin{tikzpicture}
\draw[->] (0,0)--(10,0) node[right]{$t$};
\draw[->] (0,-1)--(0,2.5) node[left]{$x(t)$};
\pgfmathsetmacro{\r}{1};
\pgfmathsetmacro{\s}{1.5};
\pgfmathsetmacro{\t}{4};
\pgfmathsetmacro{\h}{1.5};
\pgfmathsetmacro{\k}{-0.9};
\pgfmathsetmacro{\u}{6};
\pgfmathsetmacro{\v}{7};
\pgfmathsetmacro{\w}{8};
\draw[thick] (0,0)--(\r,0)--(\s,\h)--(\t,\h)--(\u,\k)--(\v,\k);
\draw[thick, domain=\v:\w] plot (\x,{2*(\x-\v)^2+\k});
\draw (\r,0) node[below]{$t_1$};
\draw[dotted] (\s,0) node[below]{$t_2$}--(\s,\h);
\draw[dotted] (\t,0) node[below]{$t_3$}--(\t,\h);
\draw[dotted] (\u,0) node[above]{$t_4$}--(\u,\k);
\draw[dotted] (\v,0) node[above]{$t_5$}--(\v,\k);
\draw[dotted] (\w,0) node[below]{$t_6$}--(\w,{2*(\w-\v)^2+\k});
\fill (\w,{2*(\w-\v)^2+\k}) circle (0.45mm);
\draw[dotted] (0,\h) node[left]{$d_1$}--(\s,\h);
\draw[dotted] (0,\k) node[left]{$d_2$}--(\u,\k);
\end{tikzpicture}\end{bmlimage}
\end{center}
Descreva qualitativamente a evolução da partícula em cada um dos intervalos $[0,t_1]$,
$[t_2,t_3]$, etc., em termos de velocidade instantânea e aceleração.
\begin{sol}  Em $t=0$, a partícula está na origem, onde ela fica até o instante
$t_1$. Durante $[t_1,t_2]$, ela anda em direção ao ponto $x=d_1$, com
velocidade constante $v=\frac{d_1}{t_2-t_1}$ e aceleração $a=0$. No tempo $t_2$
ela chega em $d_1$
e fica lá até o tempo $t_3$. No tempo $t_3$ ela começa a andar em direção ao
ponto $x=d_2$ (isto é, ela \emph{recua}), com velocidade constante
$v=\frac{d_2-d_1}{t_4-t_3}<0$. Quando chegar em $d_1$ no tempo $t_4$, para, fica
lá até $t_5$. No tempo $t_5$, começa a acelerar com uma aceleração $a>0$, até
o tempo $t_6$.
\end{sol}
\end{exo}

\begin{exo}
Considere uma partícula se movendo ao longo da trajetória
$x(t)=\frac{t^2}{2}-t$ (medida em metros), $t\geq 0$.
Calcule a velocidade instântânea nos instantes $t_0=0$, $t_1=1$,
$t_2=2$, $t_3=10$. O que acontece com a velocidade instantânea $v(t)$ quando
$t\to \infty$? Descreva o que seria visto por um observador imóvel
posicionado em $x=0$, olhando para a partícula, em particular nos instantes
$t_0,\dots,t_3$.
Calcule a aceleração $a(t)$.
\begin{sol}
Como $v(t)=t-1$, temos $v(0)=-1<0$, $v(1)=0$, $v(2)=1>0$, $v(10)=9$.
Quando $t\to \infty$, $v(t)\to\infty$.
Observando a partícula, significa que no tempo $t=0$ ela está em
$x(0)=0$, recuando com uma velocidade de $-1$ metros por segundo. No instante
$t=1$, ela está com velocidade nula em $x(1)=-\frac12$. No instante $t=2$ ela
está de volta em $x(2)=0$, mas dessa vez com uma velocidade de $+1$ metro por
segundo.
A aceleração é \emph{constante}: $a(t)=v'(t)=+1$.
\end{sol}
\end{exo}

\begin{exo}
O movimento oscilatório \index{movimento oscilatório}
genérico é descrito por uma trajetória do tipo
$$x(t)=A\sen (\omega t)\,,$$ em que $A$ é a amplitude máxima e $\omega$ uma
velocidade angular.
Estude $x(t)$, $v(t)$ e $a(t)$. Em particular, estude os instantes em que
$v(t)$ e $a(t)$ são nulos ou atingem os seus valores extremos, e onde que a
partícula se encontra nesses instantes.
\begin{sol}
Temos $v(t)=x'(t)=A\omega \cos(\omega t)$, e $a(t)=v'(t)=-A\omega^2\sen (\omega
t)\equiv -\omega^2 x(t)$. 
\begin{center}
\begin{bmlimage}\begin{tikzpicture}
\pgfmathsetmacro{\o}{1};
\pgfmathsetmacro{\A}{1};
\pgfmathsetmacro{\l}{12.7};
\pgfmathsetmacro{\omeg}{1};
\draw[ ->] (0,0)--(\l,0);
\draw[ ->] (0,-\A-0.2)--(0,\A+0.3);
\draw[thick, domain=0:\l-1.8, samples=80] plot (\x,{\A*sin(\omeg*\x r)})
node[right]{$x(t)$};
\draw[dashed, domain=0:\l-0.5, samples=80] plot (\x,{\A*\omeg*cos(\omeg*\x r)})
node[right]{$v(t)$};
\draw[dotted, domain=0:\l-2, samples=80] plot (\x,{-\A*\omeg^2*sin(\omeg*\x
r)})
node[right]{$a(t)$};
\foreach \k in {1,2,3} {
\draw ({\k*3.1414/\omeg},0) node{$\shortmid$} node[above]{$\frac{\k
\pi}{\omega}$};
}
\end{tikzpicture}\end{bmlimage}
\end{center}
Observe que $v(t)$ é máxima quando $x(t)=0$, e é mínima quando $x(t)=\pm A$.
Por sua vez, $a(t)$ é nula quando $x(t)=0$ e máxima quando $x(t)=\pm A$.
\end{sol}
\end{exo}

Na prática, a derivada deve sempre ser interpretada como
taxa de variação.
Considere alguma quantidade $N(t)$, 
por exemplo o número de indivíduos numa população, que depende
de um parâmetro $t\geq 0$ que interpretaremos aqui como o
tempo.
A \emph{taxa de variação instantânea de $N(t)$} é
definida medindo de quanto que $N(t)$ cresce entre dois instantes
consecutivos, arbitrariamente próximos: 
$$\text{Taxa de variação de $N$ no 
instante }t=\lim_{\Delta t\to 0}\frac{N(t+\Delta
t)-N(t)}{\Delta t}\equiv N'(t)\,.
$$

\begin{exo}
Calcula-se que, daqui a $t$ meses, 
a população de uma certa comunidade será de
$P(t) = t^2 + 20t + 8000$ habitantes.
\begin{enumerate}
\item Qual é a taxa de variação da população da comunidade hoje?
\item Qual será a taxa de variação da população 
desta comunidade daqui a 15 meses ?
\item Qual será a variação real da população durante o $16^o$ 
mês?
\end{enumerate}
\begin{sol}
A taxa de variação no mês $t$ é dada por $P'(t)=2t+20$. Logo, hoje,
$P'(0)=+20$ hab./mês, o que significa que a população hoje cresce a medida de
$20$ habitantes por mês. Daqui a $15$ meses, $P'(15)=+50$ hab./mês. A
variação real da população durante o $16$-ésimo mês será $P(16)-P(15)=+51$
habitantes.
\end{sol}
\end{exo}

% \begin{ex}
% Considere \emph{uma população de
% bactérias que cresce com uma taxa de $100'000$ indivíduos/dia}. Isto é, se 
% $N_n$ representa o tamanho da população total no dia $n$, 
% então de um dia $n$ para o próximo $n+1$, $N_n$ passa para 
% $N_{n+1}=N_{n}+100'000$. Em termos da diferença: $N_{n+1}-N_n=100'000$.
% \end{ex}
\subsection{Taxas relacionadas}
\index{taxas relacionadas}
Em vários problemas, uma quantidade $X$ depende de uma quantidade $Y$:
$X=f(Y)$. 
Ora, se $Y$ por sua vez depende de um parâmetro por exemplo o tempo $t$,
então $X$ depende de $t$ também: $X(t)=f(Y(t))$. A taxa de variação de $X$
com respeito a $t$ pode ser obtida usando a regra da
cadeia: 
\[
X'(t)=f'(Y(t))Y'(t)\,.
\]
Essa expressão mostra como as taxas de variação de $X(t)$ e $Y(t)$, 
isto é $X'(t)$ e $Y')(t)$, são relacionadas.

\begin{ex}
Considere um quadrado de comprimento linear $L$, medido em
metros. Outras quantidades associadas ao 
quadrado podem ser expressas em função de $L$.
Por exemplo, o comprimento da sua diagonal, o seu perímetro (ambos em metros), e
a sua área (em metros quadrados):
$$D=\sqrt{2}L\,,\quad P=4L\,,\quad A=L^2\,.$$
Suponha agora que $L$ depende do tempo: $L=L(t)$ ($t$ é medido em segundos).
Então $D$, $P$ e $A$ também dependem do tempo
$$D(t)=\sqrt{2}L(t)\,,\quad P(t)=4L(t)\,,\quad A(t)=L(t)^2\,,$$ 
e como a taxa de variação de $L(t)$ é $L'(t)$ metros/segundo,
as taxas de variação
de $D$, $P$ e $A$ são obtidas derivando com respeito a $t$:
$$D'(t)=\sqrt{2} L'(t)\,,\quad P'(t)=4L'(t)\,,\quad
A'(t)=2L(t)L'(t)\,.$$
(Para $A'(t)$ usamos a regra da cadeia.)
Suponhamos, por exemplo, que \emph{o quadrado se expande de modo tal que o seu
lado cresça a razão constante de $6$ $m/s$}, isto é: $L'(t)=6$.
Logo, 
$$D'(t)=6\sqrt{2}\,,\quad P'(t)=24\,,\quad
A'(t)=12L(t)\,.$$
Isto é, a diagonal e o perímetro crescem com uma taxa constante, mas a taxa
de variação da área depende do tamanho do quadrado: quanto maior o quadrado,
maior a taxa $A'(t)$.
Por exemplo, no instante $t_ 1$ em que $L(t_1)=1$, $A'(t_1)=12$ $m^2/s$, e no
instante $t_2$ em que $L(t_2)=10$, $A'(t_2)=120$ $m^2/s$.
\end{ex}

\begin{exo}
Os lados de um cubo crescem a uma taxa de $0.5$ metros por segundo.
Determine a taxa de variação do volume do cubo no instante em que os lados
medem 1) $10$ metro 2) $20$ metros.
\begin{sol}
Como $V=L^3$, $V'=3L^2L'=\frac32 L^2$.
Logo, quando $L=10$, $V'=150$ $m^3/s$, e quando 
$L=20$, $V'=600$ $m^3/s$.
\end{sol}
\end{exo}

\begin{exo} (Segunda prova, 27 de maio de 2011)
Um balão esférico se enche de ar a uma taxa de $2$ metros
cúbicos por segundo.  Calcule
a taxa com a qual o raio do balão cresce no instante em que o seu volume 
atingiu $\frac{4\pi}{3}$ metros cúbicos.
\begin{sol}
O volume do balão no tempo $t$ é dado por $V(t)=\tfrac43 \pi R(t)^3$. 
Logo, $R(t)=(\frac{3}{4\pi}V(t))^{1/3}$, e pela regra da cadeia, 
$R'(t)=\tfrac13(\frac{3}{4\pi}V(t))^{-2/3}\frac{3}{4\pi}V'(t)$.
No instante $t_*$ que interessa, $V(t_*)=\frac{4\pi}{3}m^3$, e como
$V'(t)=2m^3/s$ para todo $t$, obtemos
$$
R'(t_*)=\tfrac13(\frac{3}{4\pi}\frac{4\pi}{3})^{-2/3}\frac{3}{4\pi}2\,m/s=\frac{
1}{2\pi}m/s\,.
$$
\end{sol}
\end{exo}

\begin{exo}
Uma vassoura de $2$ metros está apoiada numa parede. Seja $I$
seu ponto de contato com o chão, $S$ seu ponto de contato com
a parede. A vassoura começa a
escorregar, $I$ se afastando da parede a
uma velocidade de $0.8\, m/s$. 1) Com qual velocidade $S$ se
aproxima do chão no instante em que $I$ está a $1\,m$ da
parede? 2) O que acontece com a
velocidade de $S$ quando a distância de $I$ à parede se aproxima de $2$?
\begin{sol}
Seja $x$ a distância de $I$ até a parede, e $y$ a distância de $S$ até o chão:
$x^2+y^2=4$. Quando a vassoura começa a escorregar, $x$ e $y$ ambos se
tornam funções do tempo: $x=x(t)$ com $x'(t)=0.8\,m/s$, e $y=y(t)$. Derivando
implicitamente com respeito a $t$,
$2xx'+2yy'=0$. Portanto, 
$y'=-\frac{xx'}{y}=-0.8\frac{x}{y}=-\frac{0.8x}{\sqrt{4-x^2}}$.
1) Quando $x=1\,m$, $y'=-0.46\,m/s$ (da onde vém esse sinal ``-''?)
2) Quando $x\to 2^-$, $y'\searrow -\infty$.
Obs: Quando $I$ estiver a $2-7.11\cdot 10^{-22}\,m$ da parede,
$S$ ultrapassa a velocidade da luz.
\end{sol}
\end{exo}

\begin{exo}
Um laser em rotação ($0.5$ rad/s.) está a $10$ metros de uma parede
reta. Seja $P$ a posição da marca do laser na parede, $A$ o ponto da parede
mais perto do laser.
Calcule a velocidade do ponto $P$ no instante em que $P$ está 1) em $A$ 2) a
$10$ metros de $A$, 3) a $100$ metros de $A$.
\begin{sol}
Definamos $\theta$ e $x$ da seguinte maneira:
\begin{center}
\begin{bmlimage}\begin{tikzpicture}
\draw (-5,0)--(5,0);
\pgfmathsetmacro{\teta}{60};
\pgfmathsetmacro{\h}{2};
\pgfmathsetmacro{\p}{-\h*tan(\teta)};
\fill (\p,0) circle (0.50mm);
\draw (\p,0) node[above]{$P$};
\draw[thick, ->] (\p,0)--(\p-0.4,0);
\draw[dotted] (0,0)--(0,\h) node[right]{$L$};
\draw[dashed] (\p,0)--(0,\h);
\draw[->] (0,\h-0.8) arc (270:270-\teta:0.8); 
\draw (-0.5,1.05) node{$\theta$};
\draw[decorate, decoration=brace] (0,-0.2)--(\p,-0.2) node[midway, below]{$x$};
\pgfmathsetmacro{\e}{0.2};
\draw (0,0) node[above right]{$A$};
\pgfmathsetmacro{\f}{\e*sin(\teta)};
\pgfmathsetmacro{\g}{\e*cos(\teta)};
\draw[line width=4pt] (-\f,\h-\g)--(\f,\h+\g);
\end{tikzpicture}\end{bmlimage}
\end{center}
Temos $\tan \theta=\frac{x}{10}$ e como $\theta'=0.5$ rad/s, temos
$x'=10(1+\tan^2\theta)\theta'=5(1+\tan^2\theta)$.
1) Se $P=A$, então $\tan \theta=0$, logo $x'=5$ m/s. 2) Se $x=10\,m$, então
$\tan \theta=1$ e $x'=10\,m/s$.
3) Se  $x=100\,m$, então $\tan \theta=10$ e $x'=505\,m/s$ (mais rápido que a
velocidade do som, que fica em torno de $343\, m/s$).
\end{sol}
\end{exo}



\begin{exo}
Um balão cheio de hidrogênio é soltado, e sobe verticalmente a 
uma velocidade de $5m/s$. Um observador está a $50m$ do ponto de onde 
o balão foi largado. calcule a taxa de variação do ângulo sob o qual o
observador vê o balão subir, no instante em que este se encontra a 1) $30$
metros de
altura, 2) $1000$ metros de altura.
\begin{sol}
Seja $H$ a altura do balão e $\theta$ o ângulo sob o qual o observador vê o
balão. Temos $H'=5$, e $\tan \theta=\frac{H}{50}$. Como ambos $H$ e
$\theta$ dependem do tempo, ao derivar com respeito a $t$ dá 
$(1+\tan^2\theta)\theta'=\frac{H'}{50}=\frac{1}{10}$, isto é:
$\theta'=\frac{1}{10(1+\tan^2\theta)}$.
1) No instante em que o balão estiver a $30$ metros do chão, $\tan
\theta=\frac{30}{50}=\tfrac35$, assim $\theta'=\frac{5}{68}\simeq 0.0735$
rad/s. 
2) No instante em que o balão estiver a $1000$ metros do chão, $\tan
\theta=\frac{1000}{50}=20$, assim $\theta'=\frac{1}{4010}\simeq 0.0025$ rad/s.
\end{sol}
\end{exo}

\begin{exo}
A pressão $P$ de um gás ideal de temperatura fixa $T$ contido num container de
volume $V$ satisfaz à equação $PV=nkT$, em que $n$ e $k$ são constantes (que
dependem do gás). Suponha que, mantendo $T$ fixo, o gás tenha um
volume inicial de $V_1$, e que ele comece a diminuir com uma
taxa de $0.01$ $m^3/s$. Calcule a taxa de variação da pressão no instante em que
o volume vale $V_0<V_1$.
\begin{sol}
Como $P=\frac{nkT}{V}$, $P'=-\frac{nkT}{V^2}V'$. Logo,
no instante em que $V=V_0$,
$P'=-\frac{3nkT}{V_0^2}$.
\end{sol}
\end{exo}


\section{Linearização}
\index{linearização}

A derivada fornece um jeito eficiente de aproximar funções.
De fato, ao olhar \emph{localmente} o gráfico de uma função $f$ derivável 
em torno de um
ponto $P=(a,f(a))$, vemos que este é quase indistinguível da sua reta
tangente:

\begin{center}
\begin{bmlimage}\begin{tikzpicture}
\newcommand{\funcao}[1]{(#1)^2/4+1}
\newcommand{\dfuncao}[2]{ (\funcao{#1+#2})/(#2)-(\funcao{#1})/(#2)}
\pgfmathsetmacro{\a}{1.5};
\begin{scope}
\draw[ ->] (\a-1,0)--(\a+1,0);
\draw[thick, domain=\a-1:\a+1] plot (\x,{\funcao{\x}});
\coordinate (P) at (\a,{\funcao{\a}});
\draw[ thick, domain={\a-0.5}:{\a+0.5}] plot
(\x,{(\dfuncao{\a}{0.03})*(\x-\a)+\funcao{\a}});
%node[right]{$y=f(a)+f'(a)(x-a)$};
\draw (P) node[above]{$P$};
\fill (P) circle (0.50mm);
\draw (P) circle (5mm);
\draw[dotted] (\a,0)node[below]{$a$}--(\a,{\funcao{\a}});
\draw (\a+1,{\funcao{\a}}) node[right]{$\Rightarrow$};

\end{scope}

\begin{scope}[xshift=2.2cm, yshift=-1.5cm, scale=2]
\clip (\a,{\funcao{\a}}) circle (5mm);
\pgfmathsetmacro{\a}{1.5};
\draw[ ->] (\a-1,0)--(\a+1,0);
\draw[thick, domain=\a-1:\a+1] plot (\x,{\funcao{\x}});
\coordinate (P) at (\a,{\funcao{\a}});
\draw[ thick, domain={\a-0.5}:{\a+0.5}] plot
(\x,{(\dfuncao{\a}{0.03})*(\x-\a)+\funcao{\a}});
\draw (P) node[above]{$P$};
\fill (P) circle (0.35mm);
\draw (P) circle (4.9mm);
\end{scope}
\end{tikzpicture}\end{bmlimage}
\end{center}

Tornemos essa observação mais quantitativa.
A reta tangente tem inclinação dada pela derivada de $f$ em $a$:
$$f'(a)=\lim_{x\to a}\frac{f(x)-f(a)}{x-a}\,.$$
A existência do limite acima significa que quando $x$ fica
suficientemente perto de $a$, então o quociente 
$\frac{f(x)-f(a)}{x-a}$ pode ser aproximado pelo número 
$f'(a)$, o que pode ser escrito informalmente
\[\frac{f(x)-f(a)}{x-a}\simeq f'(a)\,.\]
Rerranjando obtemos
\eq{\label{eq:linearizacaodef}
f(x)\simeq \underbrace{f(a)+f'(a)(x-a)}_{\text{reta tangente em $P$}}\,.}
\index{reta!tangente}
Em função da variável $x$, o lado direito dessa expressão
representa a reta tangente ao gráfico de $f$ no ponto $(a,f(a))$. 
Assim, \eqref{eq:linearizacaodef} dá uma aproximação de $f(x)$ para $x$ numa 
vizinhança de $a$; a reta $y=f(a)+f'(a)(x-a)$ é chamada \grasA{linearização de
$f$ em torno $a$}.

\begin{ex} Já vimos que a linearização de $f(x)=x^2$ em torno de $x=-1$ é dada
por $f(x)\simeq -2x-1$. 
\end{ex}

\begin{ex} Para seno e cosseno, temos (lembre do Exercício
\ref{Exo:retastangentesseno}):
\begin{itemize}
 \item 
Em torno de $a=0$: $\sen x\simeq x $,  $\cos x\simeq 1$.
\item Em torno de $a=\pisobredois$:
$\sen x\simeq 1$, $\cos x\simeq -(x-\pisobredois)$.
\item Em torno de $a=\pi$: $\sen x\simeq -(x-\pi)$, $\cos x \simeq -1$.
\end{itemize}
\end{ex}

\begin{exo}
Calcule a linearização de $f$ em torno de $a$.
\begin{multicols}{2}
\begin{enumerate}
\item\label{itexolinearizac1} $f(x)=e^x$, $a=0,-1$.
\item\label{itexolinearizac2} $f(x)=\ln(1+x)$, $a=0$.
\item\label{itexolinearizac3} $f(x)=\frac{x}{x-1}$, $a=0$.
\item\label{itexolinearizac4} $f(x)=e^{-\frac{x^2}{2}}$, $a=0$.
\item\label{itexolinearizac5} $f(x)=\sen x$, $a=0,\pisobredois, \pi$.
\item\label{itexolinearizac6} $f(x)=\sqrt{1+x}$, $a=0$.
\end{enumerate}
\end{multicols}
\vspace{0.01cm}
\begin{sol}
\eqref{itexolinearizac1} $f(x)\simeq x+1$, $f(x)\simeq e^{-1}x+2e{^-1}$
\eqref{itexolinearizac2} $f(x)\simeq x$,
\eqref{itexolinearizac3} $f(x)\simeq -x$,
\eqref{itexolinearizac4} $f(x)\simeq 1$,
\eqref{itexolinearizac5} $f(x)\simeq x$, $f(x)\simeq 1$, $f(x)\simeq -x+\pi$
\eqref{itexolinearizac6} $f(x)\simeq 1+\frac{x}{2}$.
\end{sol}
\end{exo}

Linearização é usada em muitas situações práticas, com o intuito de
\emph{simplificar} a complexidade de uma função perto de um ponto. Ela pode
também ser usada como um simples método de cálculo, como no seguinte exemplo.

\begin{ex}
Como calcular $\sqrt{9.12}$, \emph{sem calculadora}?
Observe que $\sqrt{9}=3$, então o número procurado deve ser perto de $3$. Se
$f(x)=\sqrt{x}$, temos $f(9)=3$, e queremos $f(9.12)$. Como $9.12$ é próximo de
$9$, façamos uma linearização de $f$ em o de $9$: como
$f'(x)=\frac{1}{2\sqrt{x}}$, temos para $x\simeq 9$:
$$f(x)\simeq f(9)+f'(9)(x-9)=3+\tfrac16(x-9)\,.$$
Logo, $f(9.12)\simeq 3.02$. 
Esse número é uma aproximação boa do verdadeiro valor, que pode ser obtido com
uma calculadora: $\sqrt{9.12}=3.019933...$
\end{ex}

\begin{exo}
Dê um valor aproximado de $\sqrt{3.99}$, 
$\ln(1.0123)$, $\sqrt{101}$.
\begin{sol}
Como $\sqrt{4+x}\simeq 2+\frac{x}{4}$, temos $\sqrt{3.99}=\sqrt{4-0.01}\simeq
2+\frac{-0.01}{4}=1.9975$ (HP: $\sqrt{3.99}=1.997498...$).
Como $\ln (1+x)\simeq x$, temos $\ln(1.0123)=\ln(1+0.123)\simeq 0.123$ (HP:
$\ln(1.123)=0.1160...$).
Como $\sqrt{101}=10\sqrt{1+\frac{1}{100}}$ e que $\sqrt{1+x}\simeq
1+\frac{x}{2}$, temos $\sqrt{101}\simeq 10\cdot(1+\frac{1/100}{2})=10.05$ (HP:
$\sqrt{101}=10.04987...$).
\end{sol}
\end{exo}

\begin{obs}
Em Cálculo II serão estudadas aproximações de uma função $f$ em torno
de um ponto $a$, que vão além da aproximação linear. Por exemplo, uma
aproximação de $f$ de ordem dois é da forma:
$$f(x)\simeq f(a)+f'(a)(x-a)+\tfrac12 f''(a)(x-a)^2\,,$$
onde $f''(a)$ é a \emph{segunda derivada} de $f$ em $a$.
\end{obs}

\section{Derivação implícita}
\index{derivada!implícita}

A maioria das funções encontradas até agora eram dadas 
 \emph{explicitamente}, o que significa que os seus valores
$f(x)$ eram calculáveis facilmente. Por exemplo, se $$f(x)\pardef
x^2-x\,,$$ então $f(x)$ pode ser calculado para qualquer valor de
$x$: $f(0)=0^2-0=0$, $f(2)=2^2-2=2$, etc. 
Além disso, $f(x)$ pode ser derivada aplicando simplesmente as
regras de derivação: 
$$f'(x)=(x^2-x)'=(x^2)'-(x)'=2x-1\,.$$

Mas às vezes, uma função pode ser definida de maneira
\emph{implícita}. Vejamos exemplos.

\begin{ex}
Fixe um $x$ e considere o número $y$ solução da seguinte equação:
\begin{equation}\label{eq:eximpliss}
x=y^3+1\,.
\end{equation}
Por exemplo, se $x=1$, então $y=0$. Se $x=9$ então $y=2$.
A cada $x$ escolhido corresponde um único $y$ que
satisfaça a relação acima. Os pares $(x,y)$ definem uma curva $\gamma$ no plano.
Essa curva é definida pela relação \eqref{eq:eximpliss}.

Quando $x$ varia, o $y$ correspondente varia também, logo $y$ é
\emph{função} de $x$: $y=f(x)$. Na verdade, 
$f$ pode ser obtida isolando $y$ em \eqref{eq:eximpliss}:
\begin{equation}\label{eq:exexpliss}
y=\sqrt[3]{x-1}\,,
\end{equation}
o que significa que $f(x)=\sqrt[3]{x-1}$.
A relação \eqref{eq:exexpliss} dá a relação \emph{explícita} entre $x$ e
$y$, enquanto em \eqref{eq:eximpliss} a relação era só \emph{implícita}.
Com a relação explícita em mão, pode-se estudar mais propriedades da
curva $\gamma$, usando por exemplo a derivada de $f$.
\end{ex}

\begin{ex}
Considere agora a seguinte relação implícita
%Considere a função $f$ definida
%da seguinte maneira: para um $x\in \bR$, $y=f(x)$ é 
%definido como a solução da equação 
\begin{equation}\label{eq:isolarydificil}
\sen y=y+x\,.
\end{equation}
Não o faremos aqui, mas pode ser provado que a cada $x\in \bR$
corresponde um único $y=f(x)$ que resolve a última equação. 
Ora, apesar disso permitir {definir} a função $f$
\emph{implicitamente}, 
os seus \emph{valores} são difíceis de se
calcular explicitamente.
Por exemplo, é fácil ver que $f(0)=0$, $f(\pm \pi)=\mp\pi$,
etc., mas outros valores, como $f(1)$ ou $f(7)$ não podem ser escritos de
maneira elementar. 
A dificuldade de conhecer os valores exatos de $f(x)$ é devida
ao problema de isolar $y$ em \eqref{eq:isolarydificil}.
\end{ex}

Se os valores de uma função já são complicados de se calcular,
parece mais difícil ainda estudar a sua derivada.
No entanto, veremos agora que em certos casos, informações úteis
podem ser extraidas sobre a derivada de uma função, mesmo esta
sendo definida de maneira implícita.

\begin{ex}
Considere o círculo $\gamma$ de raio $5$ centrado na origem. 
\index{círculo}
Suponha, como no Exercício \ref{Exo:tangenteaucercle}, que se queira 
calcular a
inclinação da reta tangente a $\gamma$ no ponto $P=(3,-4)$. 
Na sua forma implícita, a equação de $\gamma$  é dada por
$$x^2+y^2=25\,.$$ 
Para calcular a inclinação da reta tangente, 
é preciso ter uma {função} que
represente o círculo na vizinhança de $P$, e em seguida calcular
a sua derivada neste ponto. 
Neste caso, ao invés de \eqref{eq:isolarydificil}, é 
possível \emph{isolar} $y$ na equação do círculo. 
Lembrando que $P=(3,-4)$ pertence à metade \emph{inferior} do círculo, 
obtemos $y=f(x)=-\sqrt{25-x^2}$.
Logo, como a função é dada explicitamente, ela pode ser derivada, e 
a inclinação procurada é dada por 
$$f'(3)=\frac{x}{\sqrt{25-x^2}}\Bigr|_{x=3}=\tfrac{3}{4}\,.$$
Essa inclinação foi obtida \emph{explicitamente}, pois foi calculada a partir
de uma expressão explícita para $f$.

Vamos apresentar agora um jeito de fazer que \emph{não passa pela 
determinação
precisa da função $f$}. De fato, \emph{suponha} que a função que descreve o
círculo na vizinhança de $P$ seja bem definida: $y=y(x)$ (ou $y=f(x)$). 
Já que o gráfico
de $f$ passa por $P$, temos $y(3)=-4$. Mas também, 
como a função $y(x)$ representa o círculo numa vizinhança de $3$, ela 
satisfaz 
$$x^2+y(x)^2=25\,.$$
(Estamos assumindo que a última expressão \emph{define} $y(x)$, mas não a
calculamos expliciamente.) Derivamos ambos lados dessa expressão
com respeito a $x$: como 
$(x^2)'=2x$, $(y(x)^2)'=2y(x)y'(x)$ (regra da cadeia) e
$(25)'=0$, obtemos
\eq{\label{eq:derivimplic100}2x+2y(x)y'(x)=0\,.}
Isolando $y'(x)$ obtemos 
\eq{\label{eq:bidddul}y'(x)=-\frac{x}{y(x)}\,.}
Assim, não conhecemos $y(x)$ explicitamente, somente 
\emph{implicitamente}, mas já temos uma informação a respeito da sua 
derivada.
Como o nosso objetivo é calcular a inclinação da reta tangente em $P$,
precisamos calcular 
$y'(3)$. Como $y(3)=-4$, a fórmula \eqref{eq:bidddul} dá:
$$y'(3)
=-\frac{x}{y(x)}\Big|_{x=3}=
-\frac{3}{-4}=\frac34\,.$$ 
\end{ex}

Em \eqref{eq:derivimplic100} derivamos \emph{implicitamente} com respeito a
$x$. Isto é, calculamos formalmente a derivada de $y(x)$ supondo que ela 
existe. Vejamos um outro exemplo.

\begin{ex}
Considere a curva $\gamma$ do plano definida pelo conjunto dos pontos $(x,y)$
que satisfazem à condição
\eq{\label{eq:chachakshtil}x^3+y^3=4\,.}
Observe que o ponto $P=(1,\sqrt[3]{3})$ pertence a essa curva. Qual é a equação
da reta tangente à curva em $P$?
\begin{center}
\begin{bmlimage}\begin{tikzpicture}
\draw[ ->] (-2,0)--(2,0);
\draw[ ->] (0,-0.3)--(0,2);
\draw[thick, domain=1.5874:-2.2, samples=100] plot
(\x,{exp(0.3333*ln(4-\x*\x*\x))}) node[left]{$\gamma$};
% \draw[thick, domain=2.3:1.5874, samples=50] plot
% (\x,{-exp(0.3333*ln(1*abs(4-\x*\x*\x)))}) ;
\draw[ domain=0.6:1.4, samples=10] plot
(\x,{1.4422-0.48*(\x-1)});
\draw[dotted] (1,1.4422)--(1,0)node[below]{$\scriptstyle{1}$};
\fill (1,1.4422) node[above]{$P$} circle (0.45mm);
\end{tikzpicture}\end{bmlimage}
\end{center}

Supondo que a curva pode ser descrita por uma função $y(x)$ na vizinança de
$P$ e derivando \eqref{eq:chachakshtil} com respeito a $x$,
$$3x^2+3y^2y'=0\,,\quad \text{ isto é:, }\quad
y'=-\frac{x^2}{y^2}\,.$$
Logo, a inclinação da reta tangente em $P$ vale
$-\frac{(1)^2}{(\sqrt[3]{3})^2}=-\tfrac{1}{\sqrt[3]{9}}$, e
a sua equação é
$y=-\tfrac{1}{\sqrt[3]{9}}x+\sqrt[3]{3}+\frac{1}{\sqrt[3]{9}}$.
\end{ex}

Lembre que quando 
calculamos $(f^{-1})'(x)$, na Seção \ref{Sec:DerivInversa}, derivamos ambos
lados da expressão $f(f^{-1}(x))=x$, que contém \emph{implicitamente} a função
$f^{-1}(x)$. Nesta seção voltaremos a usar esse método.
\begin{exo}
Calcule $y'$ quando $y$ é definido implicitamente pela equação dada.
\begin{multicols}{2}
\begin{enumerate}
\item\label{itderivimplicit1} $y=\sen (3x+y)$
\item\label{itderivimplicit2} $y=x^2y^3+x^3y^2$
\item\label{itderivimplicit3} $x=\sqrt{x^2+y^2}$
\item\label{itderivimplicit4} $\frac{x-y^3}{y+x^2}=x+2$
\item\label{itderivimplicit5} $x\sen x+y\cos y=0$
\item\label{itderivimplicit6} $x\cos y=\sen (x+y)$
\end{enumerate}
\end{multicols}
\vspace{0.01cm}
\begin{sol}
\eqref{itderivimplicit1} $y'=\frac{3\cos(3x+y)}{1-\cos(3x+y)}$.
\eqref{itderivimplicit2} $y'=\frac{2xy^3+3x^2y^2}{1-3x^2y^2-2x^3y}$
\eqref{itderivimplicit3} Atenção: o único par $(x,y)$ solução da 
equação $x=\sqrt{x^2+y^2}$ é $(0,0)$! Logo, não há jeito de escrever $y$ como 
\emph{função} de $x$, assim não faz sentido derivar com respeito a $x$. 
\eqref{itderivimplicit4} $y'=\frac{1-3x^2-4x-y}{3y^2+x+2}$
\eqref{itderivimplicit5} $y'=\frac{-\sen x-x\cos x}{\cos y-y\sen y}$
\eqref{itderivimplicit6} $y'=\frac{\cos y-\cos(x+y)}{x\sen y+\cos(x+y)}$ 
\end{sol}
\end{exo}

\begin{exo}
Calcule a equação da reta tangente à curva no ponto dado.
\begin{multicols}{1}
\begin{enumerate}
\item\label{itderivimplicitB1} $x^2+(y-x)^3=9$, $P=(1,3)$.
\item\label{itderivimplicitB2} $x^2y+y^4=4+2x$, $P=(-1,1)$.
\item\label{itderivimplicitB3} $\sqrt{xy}\cos (\pi xy)+1=0$, $P=(1,1)$.
\end{enumerate}
\end{multicols}
\vspace{0.01cm}
\begin{sol}
\eqref{itderivimplicitB1} Com $y'=1-\frac{2x}{3(y-x)^2}$,  $y=\frac56
x+\frac{13}{6}$.
\eqref{itderivimplicitB2} Com $y'=\frac{2-2xy}{x^2+4y^3}$, $y=\frac45
x+\frac95$.
\eqref{itderivimplicitB3} $y=-x+2$.
Obs: curvas definidas implicitamente por equações do tipo acima podem ser
representadas usando qualquer programa simples de esboço de funções, por exemplo
\verb|kmplot|.
\end{sol}
\end{exo}


% Mais merde ca fait chier\footnote{Chier=faire caca.}.
% 
% \begin{minipage}{10cm}
% \begin{center}
% Salut Gordim\footnotemark, ça va?
% \end{center}
%  \end{minipage}
% \footnotetext{Ela se chama Gordim na verdade.}
% Et puis ca continue a me faire chier\footnote{Chier=aller aux toilettes.}


\section{Convexidade, concavidade}\label{Sec:Segundaderivada}

Vimos na Seção~\ref{sec:taxavariacao} que a segunda derivada de uma
função aparece naturalmente ao estudar a aceleração (taxa de variação
instantânea da
velocidade) de uma partícula. Nesta seção veremos qual é a
\emph{interpretação geométrica} da segunda derivada.  Comecemos com
uma definição.

\begin{defin} Seja $I\subset \bR$ um intervalo, $f:I\to \bR$ uma função.
\begin{enumerate}
\index{função!convexa}
\item $f$ é \grasA{convexa} em $I$ se para todo $x,y\in I$, $x\leq
y$,
\eq{\label{eq:defconvexidade}
f\bigl(\frac{x+y}{2}\bigr)\leq\frac{f(x)+f(y)}{2}\,.}
\item 
\index{função!côncava}
$f$ é \grasA{côncava} em $I$ se $-f$ é convexa em $I$, isto é,
se para todo $x,y\in I$, $x\leq y$,
\eq{\label{eq:defconcavidade}f\bigl(\frac{x+y}{2}\bigr)\geq
\frac{f(x)+f(y)}{2}\,.}
\end{enumerate}
\end{defin}

\begin{obs}\label{obs:convconc}
Observe que $f$ é concava se e somente se $-f$ é convexa.
\end{obs}

\grasA{Estudar a convexidade~\footnote{A 
terminologia a respeito da convexidade pode variar,
dependendo dos livros. Às vezes, uma função \emph{côncava} é chamada de
``convexa para baixo'', e uma função \emph{convexa} é chamada de ``côncava
para cima''...
} de uma função $f$} será entendido como
\emph{determinar os intervalos em que $f$ é convexa/côncava}.

\begin{ex}\label{ex:xoisconvexa}
A função $f(x)=x^2$ é convexa em $\bR$, isto é:
$(\frac{x+y}{2})^2\leq \frac{x^2+y^2}{2}$.  
De fato, desenvolvendo o quadrado 
$(\frac{x+y}{2})^2=\frac{x^2+2xy+y^2}{4}$, assim a desigualdade
pode ser reescrita 
$0\leq \frac{x^2-2xy+y^2}{4}$, que é equivalente a $0\leq  \frac{(x-y)^2}{4}$. 
Mas essa desigualdade é sempre satisfeita, já que $(x-y)^2\geq 0$ para qualquer
par $x,y$.
\end{ex}

\begin{exo}
Usando as definições acima, mostre que 
\begin{enumerate}
% \item $f(x)=x^3$ é convexa em $\bR_+$, côncava
% em $\bR_-$,
\item\label{itfonctionsconvexes1} $g(x)=\sqrt{x}$ é côncava em $\bR_+$,
\item\label{itfonctionsconvexes2} $h(x)=\frac{1}{x}$ é convexa em $\bR_+$,
côncava em $\bR_-$.
\end{enumerate}
\begin{sol}
\eqref{itfonctionsconvexes1} Queremos verificar que $\sqrt{\frac{x+y}{2}}\geq
\frac{\sqrt{x}+\sqrt{y}}{2}$ para todo $x,y\geq 2$.
Elevando ambos lados ao quadrado (essa operação é permitida, já que ambos
lados são positivos), $\frac{x+y}{2}\geq
(\frac{\sqrt{x}+\sqrt{y}}{2})^2
=\frac{x+2\sqrt{x}\sqrt{y}+y}{4}$, e rearranjando os termos obtemos $0\leq
\frac{(\sqrt{x}-\sqrt{y})^2}{4}$, que é sempre verdadeira.
\eqref{itfonctionsconvexes2}
Se $x,y>0$, $\frac{1}{\frac{x+y}{2}}\leq \frac{\frac{1}{x}+\frac{1}{y}}{2}$ é
equivalente a $4xy\leq (x+y)^2$, que por sua vez é equivalente a $0\leq
(x-y)^2$, que é sempre verdadeira. Logo, $\frac1x$ é convexa em $(0,\infty)$.
Como $\frac1x$ é ímpar, a concavidade em $(-\infty,0)$ segue imediatamente.
\end{sol}
\end{exo}

Geometricamente, \eqref{eq:defconvexidade} pode ser interpretado da seguinte
maneira:
$f$ é \grasA{convexa} se o gráfico de $f$ entre dois pontos quaisquer
$A=(x,f(x))$, $B=(y,f(y))$, fica \emph{abaixo} do segmento $AB$:
\begin{center}
\begin{bmlimage}\begin{tikzpicture}
\newcommand{\funcao}[1]{((#1)^2/3+1)}
\draw[ ->] (-1.6,0)--(2.5,0);
%\draw[ ->] (0,-0.5)--(0,2.5);
\draw[thick, domain=-1:1.9] plot (\x,{\funcao{\x}});
\pgfmathsetmacro{\a}{-0.5};
\coordinate (A) at (\a,{\funcao{\a}});
\pgfmathsetmacro{\b}{1.4};
\coordinate (B) at (\b,{\funcao{\b}});
\pgfmathsetmacro{\c}{(\a+\b)/2};
\coordinate (C) at (\c,{\funcao{\c}});
\coordinate (Cc) at (\c,{(\funcao{\a}+\funcao{\b})/2});
%\draw[dashed, domain=-1.6:\l+0.4] plot (\x,{(\l-1)*\x+\l});
\draw[ thick] (A)--(B);
\fill (A) circle (0.50mm);
\fill (B) circle (0.50mm);
\fill (Cc) circle (0.50mm);
\fill (C) circle (0.50mm);
\draw[dotted] (\a,0)node[below]{$x$}--(A);
\draw (A) node[above]{$A$};
\draw[dotted] (\b,0)node[below]{$y$}--(B);
\draw (B) node[above]{$B$};
%\draw[ thick] (A)--(B);
\draw[dotted] (\c,0)node[below]{$\scriptstyle{\frac{x+y}{2}}$}--(Cc);
%\draw (Cc) node[above]{$C$};
\coordinate (U) at (2,1.4);
\coordinate (Cmc) at ($(Cc)!0.1!(U)$);
\coordinate (V) at (2,0.5);
\coordinate (Cm) at ($(C)!0.1!(V)$);
%\coordinate (Cm) at ($(Cc)!(current)!(\x,10)$);
\draw[<-] (Cmc)--(U) node[right]{$\scriptstyle{\frac{f(x)+f(y)}{2}}$};
\draw[<-] (Cm)--(V) node[right]{$\scriptstyle{f(\frac{x+y}{2})}$};
\end{tikzpicture}\end{bmlimage}
\end{center}

Por exemplo, 
\begin{center}
\begin{bmlimage}\begin{tikzpicture}
\begin{scope}
\newcommand{\funcao}[1]{(#1)^2}
 \draw[ ->] (-2,0)--(2,0);
\draw[ ->] (0,-0.1)--(0,2);
\draw[thick, domain=-1.3:1.3] plot (\x,{\funcao{\x}}) node[right]{$x^2$};
\pgfmathsetmacro{\a}{-0.6};
\pgfmathsetmacro{\b}{1};
\coordinate (A) at (\a,{\funcao{\a}});
\coordinate (B) at (\b,{\funcao{\b}});
\draw[ thick] (A)--(B);
\fill (A) circle (0.40mm); 
\fill (B) circle (0.40mm); 
\end{scope}

\begin{scope}[xshift=5cm]
\newcommand{\funcao}[1]{exp(#1)}
 \draw[ ->] (-2,0)--(2,0);
\draw[ ->] (0,-0.1)--(0,2);
\draw[thick, domain=-1.3:1] plot (\x,{\funcao{\x}}) node[right]{$e^x$};
\pgfmathsetmacro{\a}{-0.7};
\pgfmathsetmacro{\b}{0.6};
\coordinate (A) at (\a,{\funcao{\a}});
\coordinate (B) at (\b,{\funcao{\b}});
\draw[ thick] (A)--(B);
\fill (A) circle (0.40mm); 
\fill (B) circle (0.40mm); 
\end{scope}

\begin{scope}[xshift=10cm]
\newcommand{\funcao}[1]{abs(#1)}
 \draw[ ->] (-2,0)--(2,0);
\draw[ ->] (0,-0.1)--(0,2);
\draw[thick, domain=-1.3:1.3, samples=17] plot (\x,{\funcao{\x}})
node[right]{$|x|$};
\pgfmathsetmacro{\a}{-0.5};
\pgfmathsetmacro{\b}{0.3};
\coordinate (A) at (\a,{\funcao{\a}});
\coordinate (B) at (\b,{\funcao{\b}});
\draw[ thick] (A)--(B);
\fill (A) circle (0.40mm); 
\fill (B) circle (0.40mm); 
\end{scope}
\end{tikzpicture}\end{bmlimage}
\captionof{figure}{Exemplos de funções convexas.}\label{Fig:exemplosconvexas}
\end{center}

Por outro lado, $f$ é \grasA{côncava} se o gráfico de $f$ entre dois pontos
quaisquer $A$ e $B$ fica \emph{acima} do segmento $AB$. Por exemplo,

\begin{center}
\begin{bmlimage}\begin{tikzpicture}

\begin{scope}[yshift=-0.5cm, scale=1.2]
\newcommand{\funcao}[1]{-1*(#1)*ln(#1)}
 \draw[ ->] (-0.2,0)--(2,0);
\draw[ ->] (0,-0.1)--(0,1);
\draw[thick, domain=0.01:1.5] plot (\x,{\funcao{\x}}) node[right]{$-x\ln x$};
\pgfmathsetmacro{\a}{0.2};
\pgfmathsetmacro{\b}{0.8};
\coordinate (A) at (\a,{\funcao{\a}});
\coordinate (B) at (\b,{\funcao{\b}});
\draw[color=gray, thick] (A)--(B);
\fill (A) circle (0.40mm); 
\fill (B) circle (0.40mm); 
\end{scope}

\begin{scope}[xshift=4cm]
\newcommand{\funcao}[1]{ln(#1)}
 \draw[ ->] (0,-1.5)--(0,1.3);
\draw[ ->] (-0.1,0)--(2,0);
\draw[thick, domain=-1.3:1] plot ({exp(\x)},\x) node[right]{$\ln x$};
\pgfmathsetmacro{\a}{0.5};
\pgfmathsetmacro{\b}{1.8};
\coordinate (A) at (\a,{\funcao{\a}});
\coordinate (B) at (\b,{\funcao{\b}});
\draw[color=gray, thick] (A)--(B);
\fill (A) circle (0.40mm); 
\fill (B) circle (0.40mm); 
\end{scope}

\begin{scope}[xshift=11cm, yshift=-1cm]
\newcommand{\funcao}[1]{1-(#1)^2}
 \draw[ ->] (-2,0)--(1.2,0);
\draw[ ->] (0,-0.2)--(0,1.5);
\draw[thick, domain=0:1.5] plot ({\funcao{\x}},\x) node[left]{$\sqrt{1-x}$};
\pgfmathsetmacro{\a}{0.5};
\pgfmathsetmacro{\b}{1.3};
\coordinate (A) at ({\funcao{\a}},\a);
\coordinate (B) at ({\funcao{\b}},\b);
\draw[color=gray, thick] (A)--(B);
\fill (A) circle (0.40mm); 
\fill (B) circle (0.40mm);
\end{scope}
\end{tikzpicture}\end{bmlimage}
\captionof{figure}{Exemplos de funções côncavas.}\label{Fig:exemplosconcavas}
\end{center}

Façamos agora uma observação importante a respeito do comportamento da
derivada em relação a convexidade.
Primeiro, vemos na Figura \ref{Fig:exemplosconvexas} que para qualquer uma das
funções, se $x<y$ são dois pontos que pertencem a um intervalo 
em que a derivada existe, então $f'(x)\leq
f'(y)$. Isto é, \emph{a derivada de cada uma das funções convexas da 
Figura \ref{Fig:exemplosconvexas} é crescente.} 
Do mesmo jeito, vemos que
\emph{a derivada de cada uma das funções côncavas da 
Figura \ref{Fig:exemplosconcavas} é decrescente}.
Como a variação de 
$f'$ é determinada a partir do estudo do sinal da derivada
de $f'$ (quando ela existe), isto é, $(f')'$,
vemos que a concavidade/convexidade de $f$ pode ser obtida a partir do estudo
do sinal da \grasA{segunda derivada de $f$}, $f''\pardef (f')'$:

\begin{teo}\label{Teo:Sinalfseconde}
Seja $f$ tal que $f'(x)$ e $f''(x)$ ambas existam em todo ponto $x\in
I$ ($I$ um intervalo).
\begin{enumerate}
 \item Se $f''(x)\geq 0$ para todo $x\in I$, então $f$ é convexa em $I$.
 \item Se $f''(x)\leq 0$ para todo $x\in I$, então $f$ é côncava em $I$.
\end{enumerate}
\end{teo}
\begin{proof}
Provemos a primeira afirmação (pela Observação \ref{obs:convconc}, a
segunda segue por uma simples mudança de sinal).
Para mostrar que $f$ é convexa, é preciso mostrar que
\eq{\label{eq:equivconvvvv} f(z)\leq 
\frac{f(x)+f(y)}{2}\,,} 
em que 
$x<y$ são dois pontos quaisquer de $I$, e $z\pardef \frac{x+y}{2}$ é o
ponto médio entre $x$ e $y$.
\begin{center}
\begin{bmlimage}\begin{tikzpicture}
\newcommand{\funcao}[1]{(#1)^2/3+1}
\newcommand{\dfuncao}[2]{ (\funcao{#1+#2})/(#2)-(\funcao{#1})/(#2)}
%ESPERANDO:%%%%%%%%%%%
% \draw[thick,  domain={\a-0.4}:{\l+0.4}] plot
% (\x,{(\dfuncao{\a}{0.03})*(\x-\a)+\funcao{\a}});
% \draw[dashed, domain={\a-0.4}:{\l+0.4}] plot
% (\x,{(\dfuncao{\a}{(\l-\a)})*(\x-\a)+\funcao{\a}}) node[above]{$r$};
%%%%%%%%%%%%%%%%%%%%%%%
\draw[ ->] (-1.6,0)--(2.5,0);
%\draw[ ->] (0,-0.5)--(0,2.5);
\draw[thick, domain=-1:1.9] plot (\x,{\funcao{\x}});
\pgfmathsetmacro{\a}{-0.5};
\coordinate (A) at (\a,{\funcao{\a}});
\pgfmathsetmacro{\b}{1.4};
\coordinate (B) at (\b,{\funcao{\b}});
\pgfmathsetmacro{\c}{(\a+\b)/2};
\coordinate (C) at (\c,{\funcao{\c}});
\coordinate (Cc) at (\c,{(\funcao{\a}+\funcao{\b})/2});
%\draw[dashed, domain=-1.6:\l+0.4] plot (\x,{(\l-1)*\x+\l});
%\draw[ thick] (A)--(B);
\fill (A) circle (0.50mm);
\fill (B) circle (0.50mm);
%\fill (Cc) circle (0.50mm);
\fill (C) circle (0.50mm);
\draw[dotted] (\a,0)node[below]{$\scriptstyle{x}$}--(A);
\draw (A) node[above]{$A$};
\draw[dotted] (\b,0)node[below]{$\scriptstyle{y}$}--(B);
\draw (B) node[above]{$B$};
%\draw[ thick] (A)--(B);
\draw[dotted] (\c,0)node[below]{$\scriptstyle{z}$}--(C);
\end{tikzpicture}\end{bmlimage}
\end{center}
\index{Teorema!do valor intermediário para derivada}
Aplicaremos três vezes o Teorema do valor intermediário para a derivada 
(Corolário \ref{Corol:ValorIntermDeriv}):
1) Para $f$ no intervalo $[x,z]$: existe $c_1\in [x,z]$ tal que 
$$f(z)-f(x)=f'(c_1)(z-x)\,.$$
2) Para $f$ no intervalo $[z,y]$: existe $c_2\in [z,y]$ tal que 
$$f(y)-f(z)=f'(c_2)(y-z)\equiv f'(c_2)(z-x)\,.$$
Subtraindo as duas expressões acima, obtemos
$2f(z)-(f(x)+f(y))=-(f'(c_2)-f'(c_1))(z-x)$.
3) Para $f'$ no intervalo $[c_1,c_2]$: existe $\alpha\in [c_1,c_2]$ tal que
$$f'(c_2)-f'(c_1)=f''(\alpha)(c_2-c_1)\,.$$
Como $f''(\alpha)\geq 0$ por hipótese, temos $f'(c_2)-f'(c_1)\geq 0$, o que
implica $2f(z)-(f(x)+f(y))\leq 0$, e prova \eqref{eq:equivconvvvv}.
\end{proof}

\begin{ex}
Considere $f(x)=x^2$. Como $f'(x)=(x^2)'=2x$, e como $f''(x)=(2x)'=2>0$ para
todo $x$, o Teorema \ref{Teo:Sinalfseconde} garante que $f$ é convexa em $\bR$,
como já tinha sido provado no Exemplo \ref{ex:xoisconvexa}.

Por outro lado, se $g(x)=x^3$, então $g''(x)=6x$:
\begin{center}
\begin{bmlimage}\begin{tikzpicture}[scale=0.8]
\tkzTabInit[nocadre, espcl=2,  color, colorV=lightgray!5, colorL=gray!15,
colorC=gray!15]
{$x$ /.6, $g''(x)$ /.6, Conv. /1.2}%
{,$0$, }%
\tkzTabLine{,-,z,+,}
\tkzTabLine{,\frown,z,\smile,}
\end{tikzpicture}\end{bmlimage}
\end{center}
Logo, (confere no gráfico visto
no Capítulo \ref{Cap:Funcoes}) $x^3$ é côncava em $]-\infty,0]$, 
convexa em $[0,\infty)$. O ponto $x=0$, em que a função passa de côncava para
convexa, é chamado de \grasA{ponto de inflexão}.
\end{ex}

\begin{ex}
Considere $f(x)=\ln x$ para $x>0$. Como $f'(x)=\frac{1}{x}$,
$f''(x)=-\frac{1}{x^2}$, temos $f''(x)<0$ para todo $x$. Isto é, \emph{$\ln x$
é uma função côncava}, como já foi observado na Figura
\eqref{Fig:exemplosconcavas}.
\end{ex}

\begin{exo}
Estude a convexidade das funções a seguir. Quando for possível, monte o gráfico.
\begin{multicols}{4}
\begin{enumerate}
\item\label{itexconvexidadeA1} $\frac{x^3}{3}-x$
\item\label{itexconvexidadeA2} $-x^3+5x^2-6x$
\item\label{itexconvexidadeA3} $\scriptstyle{3x^4-10x^3-12x^2+10x}$
\item\label{itexconvexidadeA4} $\frac{1}{x}$
\item\label{itexconvexidadeA5} $xe^x$
\item\label{itexconvexidadeA6} $\frac{x^2+9}{(x-3)^2}$
\item\label{itexconvexidadeA7} $xe^{-3x}$
%\item\label{itexconvexidadeA8} $4\sqrt{x+1}-\frac{1}{\sqrt{2}}x^2-1$
%\item\label{itexconvexidadeA9} $e^{-x}\cos x$, $x\in [0,2\pi]$
\item\label{itexconvexidadeA10} $|x|-x$ 
\item\label{itexconvexidadeA11} $\arctan x$
\item\label{itexconvexidadeA12} $e^{-\frac{x^2}{2}}$ 
\item\label{itexconvexidadeA13} $\frac{1}{x^2+1}$ 
\item\label{itexconvexidadeA14} $x+\frac{1}{x}$
\end{enumerate}
\end{multicols}
\vspace{0.01cm}
\begin{sol}
\eqref{itexconvexidadeA1}
$\frac{x^3}{3}-x$ é côncava em $(-\infty,0]$, convexa em $[0,\infty)$.
O gráfico se encontra na solução do Exercício \ref{Ex:variacoesbasicas}.
\eqref{itexconvexidadeA2} $-x^3+5x^2-6x$ é convexa em $(-\infty,\tfrac53]$,
côncava em $[\frac53,\infty)$:
\begin{center}
\begin{bmlimage}\begin{tikzpicture}[scale=0.5]
\draw[ ->] (-1,0)--(3.6,0);
\draw[ ->] (0,-2)--(0,1.7);
\newcommand{\funcao}[1]{-1*(#1)^3+5*(#1)^2-6*(#1)}
\draw[thick, domain=-0.1:3.4] plot (\x,{\funcao{\x}});
\pgfmathsetmacro{\a}{5/3};
\coordinate (I) at (\a,{\funcao{\a}});
\draw[dotted] (\a,0)node[above]{$\scriptstyle{\tfrac53}$}--(I);
\fill (I) circle (0.70mm);
\end{tikzpicture}\end{bmlimage}
\end{center}
\eqref{itexconvexidadeA3} Se $f(x)=3x^4-10x^3-12x^2+10x+9$, então 
$f''(x)=12(3x^2-5x-2)$.
Logo, $f(x)$ é convexa em $(-\infty,-\tfrac13]$ e em $[2,\infty)$,
côncava em $[-\tfrac13,2]$.
\begin{center}
\begin{bmlimage}\begin{tikzpicture}[scale=0.7]
\draw[ ->] (-1,0)--(3.6,0);
\draw[ ->] (0,-1)--(0,1.7);
\newcommand{\funcao}[1]{(3*(#1)^4-10*(#1)^3-12*(#1)^2+10*(#1)+9)/100}
\draw[thick, domain=-2:4.5, samples=50] plot (\x,{\funcao{\x}});
\pgfmathsetmacro{\a}{-0.333};
\coordinate (I) at (\a,{\funcao{\a}});
\draw[dotted] (\a,0)node[below]{$\scriptstyle{-\tfrac13}$}--(I);
\fill (I) circle (0.70mm);
\pgfmathsetmacro{\b}{2};
\coordinate (J) at (\b,{\funcao{\b}});
\draw[dotted] (\b,0)node[above]{$\scriptstyle{2}$}--(J);
\fill (J) circle (0.70mm);
\end{tikzpicture}\end{bmlimage}
\end{center}
\eqref{itexconvexidadeA4} Como $(\frac{1}{x})''=\frac{2}{x^3}$, $\frac{1}{x}$ é
côncava em $(-\infty,0)$, convexa em $(0,\infty)$ (confere no gráfico do
Capítulo \ref{Cap:Funcoes}).
\eqref{itexconvexidadeA5}: Como $f''(x)=(x+2)e^x$, $f$ é côncava em
$(-\infty, -2]$, convexa em $[-2,\infty)$:
\begin{center}
\begin{bmlimage}\begin{tikzpicture}[scale=0.7]
\draw[ ->] (-4,0)--(2,0);
\draw[ ->] (0,-0.6)--(0,1.7);
\newcommand{\funcao}[1]{(#1)*exp(#1)}
\draw[thick, domain=-4:0.8, samples=50] plot (\x,{\funcao{\x}});
 \pgfmathsetmacro{\a}{-2};
 \coordinate (I) at (\a,{\funcao{\a}});
 \draw[dotted] (\a,0)node[above]{$\scriptstyle{-2}$}--(I);
 \fill (I) circle (0.70mm);
\end{tikzpicture}\end{bmlimage}
\end{center}
\eqref{itexconvexidadeA6}: 
$f(x)=\frac{x^2+9}{(x-3)^2}$ é bem definida em $D=(-\infty,3)\cup
(3,+\infty)$. Como $f''(x)=\frac{12(x+6)}{(x-3)^4}$, $f(x)$ é côncava em
$(-\infty, -6]$, convexa em $(-6,3)$ e $(3,+\infty)$:
\begin{center}
\begin{bmlimage}\begin{tikzpicture}[scale=0.2]
 \newcommand{\funcao}[1]{( (#1)^2+ 9 )/( ( (#1) - 3)^2 )}
\draw[ ->] (-15,0)--(12,0);
\draw[ ->] (0,-0.6)--(0,12);
\draw[dashed] (3,0)node[below]{$\scriptstyle{3}$}--(3,12);
\draw[dashed] (-15,1)node[left]{$\scriptstyle{y=1}$}--(12,1);
\draw[thick, domain=-15:1.9, samples=50] plot (\x,{\funcao{\x}});
\draw[thick, domain=4.4:12, samples=50] plot (\x,{\funcao{\x}});
\pgfmathsetmacro{\a}{-6};
\coordinate (I) at (\a,{\funcao{\a}});
\draw[dotted] (\a,0)node[below]{$\scriptstyle{-6}$}--(I);
\fill (I) circle (3mm);
\end{tikzpicture}\end{bmlimage}
\end{center}
\eqref{itexconvexidadeA7} Com $f(x)=xe^{-3x}$ temos $f''(x)=(9x-6)e^{-3x}$.
Logo, $f$ é côncava em $(-\infty,\tfrac23]$, convexa em $[\tfrac23,\infty)$:
\begin{center}
\begin{bmlimage}\begin{tikzpicture}[scale=0.7]
 \newcommand{\funcao}[1]{ 5*(#1)*exp(-3*(#1))}
\draw[ ->] (-2,0)--(3,0);
\draw[ ->] (0,-0.6)--(0,2);
% \draw[dashed] (3,0)node[below]{$\scriptstyle{3}$}--(3,12);
% \draw[dashed] (-15,1)node[left]{$\scriptstyle{y=1}$}--(12,1);
\draw[thick, domain=-0.1:2, samples=50] plot (\x,{\funcao{\x}});
%\draw[thick, domain=4.4:12, samples=50] plot (\x,{\funcao{\x}});
\pgfmathsetmacro{\a}{0.66666};
\coordinate (I) at (\a,{\funcao{\a}});
\draw[dotted] (\a,0)node[below]{$\scriptstyle{\tfrac23}$}--(I);
\fill (I) circle (1mm);
\end{tikzpicture}\end{bmlimage}
\end{center}
\eqref{itexconvexidadeA10} $f(x)=|x|-x$ é $=0$ se $x\geq 0$, e $=-2x$ se
$x\leq 0$. Logo, $f$ é convexa. Obs: como $|x|$ não é derivável em $x=0$, a
convexidade não pode ser obtida com o Teorema \ref{Teo:Sinalfseconde}.
\eqref{itexconvexidadeA11} Se $f(x)=\arctan x$, então $f'(x)=\frac{1}{x^2+1}$,
e $f''(x)=\frac{-2x}{(x^2+1)^2}$. Logo, $\arctan x$ é convexa em $]-\infty,0]$,
côncava em $[0,\infty)$ (confere no gráfico da Seção
\ref{Sec:Functriginversas}).
\eqref{itexconvexidadeA12} $f(x)=e^{-\frac{x^2}{2}}$ tem
$f''(x)=(x^2-1)e^{-\frac{x^2}{2}}$. Logo, $f$ é convexa em $]-\infty,1]$ e
$[1,\infty)$, e côncava em $[-1,1]$ (veja o gráfico do Exercício
\ref{Ex:variacoesbasicas}).
\eqref{itexconvexidadeA13} $f(x)=\frac{1}{x^2+1}$ é convexa em
$(-\infty,-\frac{1}{\sqrt{3}}]$ e $[\frac{1}{\sqrt{3}},\infty)$, côncava em
$[-\frac{1}{\sqrt{3}},\frac{1}{\sqrt{3}}]$.
\begin{center}
\begin{bmlimage}\begin{tikzpicture}[scale=0.7]
 \newcommand{\funcao}[1]{ 1/( (#1)^2 +1)}
\draw[ ->] (-3,0)--(3,0);
\draw[ ->] (0,-0.6)--(0,1.3);
% \draw[dashed] (3,0)node[below]{$\scriptstyle{3}$}--(3,12);
% \draw[dashed] (-15,1)node[left]{$\scriptstyle{y=1}$}--(12,1);
\draw[thick, domain=-2.5:2.5, samples=50] plot (\x,{\funcao{\x}});
%\draw[thick, domain=4.4:12, samples=50] plot (\x,{\funcao{\x}});
\pgfmathsetmacro{\a}{0.5777};
\coordinate (I) at (\a,{\funcao{\a}});
\draw[dotted] (\a,0)node[below]{$\scriptstyle{\tfrac{1}{\sqrt{3}}}$}--(I);
\fill (I) circle (0.5mm);
\coordinate (J) at (-\a,{\funcao{-\a}});
\draw[dotted] (-\a,0)node[below]{$\scriptstyle{-\tfrac{1}{\sqrt{3}}}$}--(J);
\fill (J) circle (0.5mm);
\end{tikzpicture}\end{bmlimage}
\end{center}
\end{sol}
\end{exo}

\section{A Regra de Bernoulli-l'Hôpital}
\index{Regra de Bernoulli-l'Hôpital}
%Guillaume François Antoine, marquis de L'Hôpital (1661 - 1704)
%
Vejamos agora como a derivada fornece uma ferramenta útil para
calcular alguns limites 
de formas indeterminados do tipo
``$\frac00$'', ``$\frac{\pm \infty}{\pm \infty}$'', ``$1^{\infty}$'', tais como
$$
\lim_{x\to 0}\frac{e^x-1-x}{x^2}\,,\quad
\lim_{x\to 0}\frac{\tan x-x}{x^3}\,,\quad
%\lim_{x\to \infty}\frac{\ln x}{x}\,,\quad
%\lim_{x\to \infty}\frac{x^5}{e^{2x}}\,,\quad 
\lim_{x\to \infty}\Bigl(\frac{x+1}{x-1}\Bigr)^x\,.
$$
Os métodos apresentados até agora não permitem calcular esses
limites.
Nesta seção veremos como derivadas são úteis para estudar limites da forma
$\lim_{x\to a}\frac{g(x)}{h(x)}$, quando $\lim_{x\to a}g(x)=0$, $\lim_{x\to
a}h(x)=0$, ou quando $\lim_{x\to a}g(x)=\pm \infty$, $\lim_{x\to
a}h(x)=\pm \infty$.
A idéia principal é que \emph{limites indeterminados da forma $\tfrac00$ (ou
$\frac{\pm \infty}{\pm \infty}$)
podem, em geral, ser estudados via uma razão de duas derivadas}. 
Os métodos que aproveitam dessa idéia,
descritos abaixo, costumam ser chamados de \emph{Regra de
Bernoulli-l'Hôpital}~\footnote{Johann Bernoulli, Basileia (Suiça) 1667-1748. 
Guillaume François Antoine, marquis de
L'Hôpital (1661 - 1704).} (denotado por B.-H. abaixo).
Comecemos com um exemplo elementar.

\begin{ex}
Considere o limite
$$\lim_{x\to 0}\frac{e^x-1}{\sen x}\,.$$
Já que $\lim_{x\to 0}e^x-1=e^0-1=0$ e $\lim_{x\to 0}\sen x=\sen 0=0$, esse
limite é indeterminado da forma $\tfrac00$. Mas observe que,
dividindo o numerador e o denomindor por $x$, 
$$\lim_{x\to 0}\frac{e^x-1}{\sen x}=\lim_{x\to
0}\frac{\frac{e^x-1}{x}}{\frac{\sen x}{x}}=
\lim_{x\to 0}\frac{\frac{e^x-e^0}{x}}{\frac{\sen x-\sen 0}{x}}\,.
$$
Dessa forma, aparecem dois quocientes bem comportados quando $x\to 0$. O
numerador, $\frac{e^x-e^0}{x}$, tende à derivada da função $e^x$ em $x=0$, isto
é, $1$. O denominador, $\frac{\sen x-\sen 0}{x}$ tende à derivada da função
$\sen x$ em $x=0$, isto é: $1$, diferente de zero. Logo,
$$\lim_{x\to 0}\frac{e^x-1}{\sen x}=\frac{\lim_{x\to
0}\frac{e^x-e^0}{x}}{\lim_{x\to 0}\frac{\sen x-\sen 0}{x}}\equiv
\frac{(e^x)'|_{x=0}}{(\sen x)'|_{x=0}}=\frac{1}{1}=1\,.
$$
\end{ex}

A idéia do exemplo anterior pode ser generalizada:

\begin{teo}[Regra de Bernoulli-l'Hôpital, Primeira versão]\label{Teo:BH1}
Sejam $f$, $g$ duas funções deriváveis no ponto $a$, que se
anulam em $a$, $f(a)=g(a)=0$, e
tais que $\frac{f'(a)}{g'(a)}$ existe. Então 
\eq{\lim_{x\to a}\frac{f(x)}{g(x)}=\frac{f'(a)}{g'(a)}\,.}
\end{teo}
\begin{proof}
 Como antes, 
$$
\lim_{x\to a}\frac{f(x)}{g(x)}=
\lim_{x\to a}\frac{f(x)-f(a)}{g(x)-g(a)}=
\lim_{x\to a}\frac{\frac{f(x)-f(a)}{x-a}}{\frac{g(x)-g(a)}{x-a}}=
\frac{f'(a)}{g'(a)}\,.
$$
\end{proof}


\begin{exo}
Calcule os limites:
$$\lim_{s\to 0}\frac{\log(1+s)}{e^{2s}-1}\,,\quad
\lim_{t\to \pi}\frac{\cos t+1}{\pi-t}\,,\quad 
\lim_{\alpha\to
0}\frac{1-\cos(\alpha)}{\sen(\alpha+\frac{\pi}{2})}\,,\quad
\lim_{x\to 0}\frac{\sen x}{x^2+3x}
\,.
$$
\begin{sol}
Nos dois primeiros e último exemplos, as hipóteses do Teorema \ref{Teo:BH1} são
verificadas, dando
 $$\lim_{s\to 0}\frac{\log(1+s)}{e^{2s}-1}=
\frac{(\log(1+s))'|_{s=0}}{(e^{2s})'|_{s=0}}
=\frac{\frac{1}{1+s}|_{s=0}}{2e^{2s}|_{s=0}}=\frac{1}{2}$$
$$
\lim_{t\to \pi}\frac{\cos t+1}{\pi-t}=-(\cos t)'|_{t=\pi}=\sen t|_{t=0}=0\,.$$
$$
\lim_{x\to 0}\frac{\sen x}{x^2+3x}=\frac{(\sen
x)'|_{x=0}}{(x^2+3x)'|_{x=0}}
=\frac{\cos 0}{2\cdot 2+3}=\frac{1}{3}\,.
$$
No terceiro, o teorema não se aplica: apesar das funções $1-\cos(\alpha)$ e 
$\sen(\alpha+\frac{\pi}{2})$ serem deriváveis em $\alpha=0$, temos
$\sen (0+\pi/2)=1\neq 0$. Logo o limite se calcula sem a regra de B.H.:
$\lim_{\alpha\to 0}\frac{1-\cos(\alpha)}{\sen (\alpha+\pi/2)}=\tfrac01=0$.
\end{sol}
\end{exo}

O resultado acima pode ser generalizado a situações em que
$\frac{f'(a)}{g'(a)}$ não existe, ou em que $f$ e $g$ nem são definidas em $a$:

\begin{teo}[Regra de Bernoulli-l'Hôpital, Segunda versão]\label{Teo:BH2}\mbox{}
\begin{enumerate}
 \item \grasA{Limites $x\to a^+$:}
Sejam $f$, $g$ duas funções deriváveis em $(a,b)$, 
com $g(x)\neq 0$, $g'(x)\neq 0$ para todo $x\in (a,b)$. Suponha que $f$ e
$g$ são 
tais
que $\lim_{x\to a^+}f(x)=\pm \alpha$ e $\lim_{x\to a^+}g(x)=\pm \alpha$, com
$\alpha\in \{0,\infty\}$. Se $\lim_{x\to
a^+}\frac{f'(x)}{g'(x)}$ existir, ou se for $\pm \infty$, então
\eq{\lim_{x\to a^+}\frac{f(x)}{g(x)}=\lim_{x\to a^+}\frac{f'(x)}{g'(x)}\,.}
(Uma afirmação equivalente pode ser formulada para $x\to b^-$.)
\item \grasA{Limites $x\to \infty$:}
Sejam $f$, $g$ duas funções deriváveis em todo $x$ suficientemente grande, e
tais
que $\lim_{x\to \infty}f(x)=\pm \alpha$, $\lim_{x\to \infty}g(x)=\pm \alpha$,
com $\alpha\in \{0,\infty\}$. Se
$\lim_{x\to
\infty}\frac{f'(x)}{g'(x)}$ existir ou se for $\pm \infty$, então
\eq{\lim_{x\to \infty}\frac{f(x)}{g(x)}=\lim_{x\to
\infty}\frac{f'(x)}{g'(x)}\,.}
(Uma afirmação equivalente pode ser formulada para limites $x\to -\infty$.)
\end{enumerate}
\end{teo}
\begin{proof} Provemos somente o primeiro item. Fixe $z\in (a,b)$.
Podemos definir $f(a)\pardef 0$, $g(a)\pardef 0$, de modo tal que a função
$F(x)\pardef
(f(z)-f(a))g(x)-(g(z)-g(a))f(x)$ seja contínua em $[a,z]$ e derivável em
$(a,z)$.
Como $F(z)=F(a)$, o Teorema de Rolle \ref{Teo:Rolle} garante a existência de um
$c_z\in (a,z)$ tal que $F'(c_z)=0$, isto é,
$(f(z)-f(a))g'(c_z)-(g(z)-g(a))f'(c_z)=0$, que pode ser escrito
$$
\frac{f(z)-f(a)}{g(z)-g(a)}=\frac{f'(c_z)}{g'(c_z)}\,.
$$
Observe que se $z\to a^+$, então $c_z\to a^+$. Logo, com a mudança de variável
$y\pardef c_z$,
$$
\lim_{z\to a^+}\frac{f(z)}{g(z)}
=\lim_{z\to a^+}\frac{f(z)-f(a)}{g(z)-g(a)}=
\lim_{z\to a^+}\frac{f'(c_z)}{g'(c_z)}\equiv 
\lim_{y\to a^+}\frac{f'(y)}{g'(y)}\,,
$$
o que prova a afirmação.
\end{proof}

\begin{ex}
Considere $\lim_{x\to 1}\frac{x^2-1}{x-1}$. No Capítulo
\ref{Cap:Limites},
calculamos esse limite da seguinte maneira: 
$$\lim_{x\to
1}\frac{x^2-1}{x-1}=\lim_{x\to 1}\frac{(x-1)(x+1)}{x-1}=\lim_{x\to
1}(x+1)=2\,.$$ 
Vejamos agora como esse mesmo limite pode ser calculado também usando a Regra de
Bernoulli-l'Hôpital. Como o limite é da forma 
$\lim_{x\to 1}\frac{f(x)}{g(x)}$, com $f(x)=x^2-1$ e $g(x)=x-1$ ambas deriváveis
em $(1,2)$, que $g$ e $g'$ não se anulam nesse intervalo, e como $\lim_{x\to
1^+}\frac{f'(x)}{g'(x)}=
\lim_{x\to 1^+}\frac{2x}{1}=2$, o Teorema \ref{Teo:BH2} implica
$\lim_{x\to
1^+}\frac{x^2-1}{x-1}=2$. Do mesmo jeito, $\lim_{x\to
1^-}\frac{x^2-1}{x-1}=2$, o que implica $\lim_{x\to
1}\frac{x^2-1}{x-1}=2$.
\end{ex}

\begin{obs}
A Regra de Bernoulli-l'Hôpital (que será às vezes abreviada "regra de B.H.") 
fornece uma ferramenta poderosa para calcular alguns limites, mas é
importante sempre verificar se as hipóteses do teorema são 
satisfeitas, e \emph{não querer a usar para calcular qualquer limite}!
Também, ela pode às vezes se aplicar mas não ser de nenhuma utilidade (ver o
Exercício \ref{Exo:BHbasic}).
\end{obs}


% \begin{ex}
% Considere $\lim_{x\to 0}\frac{\ln (1+x)}{x}$. Como $f(x)=\ln(1+x)$ e $g(x)=x$
% são ambas deriváveis na vizinhança de 
% \end{ex}

Às vezes, é preciso usar a regra de B.H. mais de uma vez para calcular um
limite:
\begin{ex}
Considere o limite $\lim_{x\to 0}\frac{1-\cos x}{x^2}$, já
encontrado no Exercício \ref{Exo:variantessinxsurx}.
Como $1-\cos x$ e $x^2$ ambas tendem a zero e são deriváveis na vizinhança de
zero, as hipóteses do Teorema \eqref{Teo:BH2} são satisfeitas:
$$\lim_{x\to 0^+}\frac{1-\cos x}{x^2}
=\lim_{x\to 0^+}\frac{(1-\cos x)'}{(x^2)'}=
\lim_{x\to 0^+}\frac{\sen x}{2x}.
$$
Já sabemos 
que $\lim_{x\to 0}\frac{\sen x}{x}=1$. Mesmo assim, 
sendo também da forma $\frac00$, esse limite pode ser calculado aplicando a
regra de B.-H. uma segunda vez: $\lim_{x\to 0^+}\frac{\sen x}{x}=\lim_{x\to
0^+}\frac{\cos
x}{1}=1$. Logo, $\lim_{x\to 0^+}\frac{1-\cos x}{x^2}=\frac12$.
Como a função é par, o limite lateral $x\to 0^-$ é igual ao limite
lateral $x\to 0^+$, logo $\lim_{x\to 0}\frac{1-\cos x}{x^2}=\frac12$.
\end{ex}

Vejamos agora um exemplo de limite $x\to \infty$:
\begin{ex}\label{Ex:logsurx}
Considere $\lim_{x\to \infty}\frac{\ln x}{x}$ (já calculado na
Seção~\ref{sec_Lim_parenteseAldo}, usando a fórmula do binômio
de Newton). 
Observe que $\frac{\ln x}{x}\equiv \frac{f(x)}{g(x)}$ 
é um quociente de duas funções deriváveis para todo $x>0$, 
e que $\lim_{x\to \infty}f(x)=\infty$, $\lim_{x\to \infty}g(x)=\infty$.
Além disso, $\lim_{x\to \infty}\frac{f'(x)}{g'(x)}=\lim_{x\to
\infty}\frac{1/x}{1}=0$, o que implica, pelo segundo item do 
Teorema \ref{Teo:BH2},
\eq{\lim_{x\to \infty}\frac{\ln x}{x}=0\,.}
\end{ex}

Vejamos em seguida um exemplo em que é necessário tomar um limite lateral:

\begin{ex}\label{Ex:xlogxemzero}
Considere $\lim_{x\to 0^+}x\ln x$. Aqui, consideremos $f(x)=\ln x$ e
$g(x)=\tfrac1x$, ambas deriváveis no intervalo $(0,1)$. Além disso, $g(x)\neq
0$ e $g'(x)\neq 0$ para todo  $x\in (0,1)$.
O limite pode ser escrito na forma de um quociente, escrevendo 
$x\ln x=\frac{\ln x}{1/x}$. Logo,
$$\lim_{x\to 0^+}x\ln x=\lim_{x\to 0^+}\frac{\ln x}{1/x}=
\lim_{x\to 0^+}\frac{1/x}{-1/x^2}=-\lim_{x\to 0^+}x=0\,,
$$
onde B.H. foi usada na segunda igualdade.

Um outro jeito de calcular o limite acima é de fazer uma mudança de variável:
se $y\pardef 1/x$, então $x\to 0^+$ implica $y\to
+\infty$. Logo, 
$$
\lim_{x\to 0^+}x\ln x=\lim_{y\to \infty}\tfrac{1}{y}\ln \tfrac{1}{y}
=-\lim_{y\to \infty}\tfrac{\ln y}{y}\,,
$$
e já vimos no último exemplo que esse limite vale $0$.
\end{ex}


\begin{exo}\label{Exo:BHbasic}
Calcule os limites abaixo. Se quiser usar a Regra de Bernoulli-l'Hôpital,
verifique primeiro que as hipóteses sejam satisfeitas.
\begin{multicols}{3}
\begin{enumerate}
\item\label{itexBH1} $\lim_{x\to 0^+}\frac{x}{3}$
\item\label{itexBH2} $\lim_{x\to 2}\frac{x^2-x-2}{3x^2-5x-2}$
\item\label{itexBH2b} $\lim_{x\to 1^+}\frac{x^2-2x+2}{x^2+x-2}$
\item\label{itexBH3} $\lim_{x\to 0}\frac{(\sen x)^2}{x^2}$
\item\label{itexBH4} $\lim_{x\to 0}\frac{\ln\frac{1}{1+x}}{\sen x}$
\item\label{itexBH14} $\lim_{x\to 0}\frac{1+\sen x-\cos x}{\tan x}$ %D76
\item\label{itexBH7} $\lim_{x\to 0}\frac{x-\sen x}{1-\cos x}$
\item\label{itexBH5} $\lim_{x\to 0^+}\frac{x-\sen x}{x\sen x}$
\item\label{itexBH10} $\lim_{x\to 0}\frac{\sen x-x}{x^3}$
\item\label{itexBH9} $\lim_{x\to 0}\frac{\tan x -x}{x^3}$
\item\label{itexBH922} $\lim_{x\to 0}\frac{\ln(1+\sen x)}{x}$
\item\label{itexBH11} $\lim_{x\to 0}\frac{x\sen x}{1+\cos(x-\pi)}$ %D76
\item\label{itexBH12} $\lim_{x\to 0^+}\frac{\sqrt{x}}{\ln x}$
\item\label{itexBH12aa} $\lim_{x\to 0^+}x(\ln x)^2$
\item\label{itexBH12a} $\lim_{x\to \infty}\frac{(\ln x)^2}{x}$
\item\label{itexBH12ab} $\lim_{x\to \infty}\frac{x}{e^{x}}$
\item\label{itexBH12b} $\lim_{x\to 0^+}\frac{e^{\ln x}}{x}$
\item\label{itexBH12c} $\lim_{x\to \infty}\frac{\sqrt{x+1}}{\sqrt{x-1}}$
\item\label{itexBH12d} $\lim_{x\to
\infty}\frac{x^{100}-x^{99}}{20x-3x^{100}}$
\item\label{itexBH13} $\lim_{x\to 0}\frac{\ln (1+x)-\ln(1-x)}{\sen x}$ %D76

\item\label{itexBH15} $\lim_{x\to 0}\frac{\sen^2x}{1-x^2}$ 
\item\label{itexBH16} $\lim_{x\to \infty}\frac{x+\sen x}{x}$ 
\item\label{itexBH6} $\lim_{x\to 0^+}\frac{x^2-\sen^2 x}{x^2\sen^2 x}$
\item\label{itexBH17} $\lim_{x\to 0^+}\frac{x^2 \sen \frac{1}{x}}{x}$ 
%%ZWAHLEN p.105
\item\label{itexBH18} $\lim_{x\to 0}\frac{e^{\tan x}-e^x}{x^3}$ %D77
\item\label{itexBHww8} $\lim_{x\to
0^+}\bigl(\frac{1}{x}-\frac{1}{e^x-1}\bigr)$ %D77
\item\label{itexBH20} $\lim_{x\to 0^+}\frac{\arctan(\frac1x)-\pisobredois}{x}$ 
\end{enumerate}
\end{multicols}
\vspace{0.01cm}
\begin{sol}
\eqref{itexBH1} $0$ (B.H. não se aplica)
\eqref{itexBH2} $\tfrac37$
\eqref{itexBH2b} $+\infty$ (B.H. não se aplica)
\eqref{itexBH3} $\lim_{x\to 0}\frac{(\sen x)^2}{x^2}=(\lim_{x\to 0}\frac{\sen
x}{x})^2=1^2=1$ (não precisa de B.H.)
\eqref{itexBH4} Usando B.H.,  $\lim_{x\to 0}\frac{\ln\frac{1}{1+x}}{\sen x}
=-\lim_{x\to 0}\frac{\ln(1+x)}{\sen x}=-\lim_{x\to 0}\frac{\frac{1}{x+1}}{\cos
x}=-1$.
\eqref{itexBH14} $1$
\eqref{itexBH7} $0$
\eqref{itexBH5} $0$
\eqref{itexBH10} $-\frac{1}{6}$
\eqref{itexBH9} $\tfrac13$ 
\eqref{itexBH922} $1$
\eqref{itexBH11} $2$
\eqref{itexBH12} $0$ (B.H. não se aplica)
\eqref{itexBH12aa} $0$ 
\eqref{itexBH12a} $0$ (aplicando duas vezes B.H.)
\eqref{itexBH12ab} $0$
\eqref{itexBH12b} Como $e^{\ln x}=x$, o limite é $1$ (B.H. se aplica mas não
serve para nada!)
\eqref{itexBH12c} Esse limite se calcula como no Capítulo \ref{Cap:Limites}:
$\lim_{x\to \infty}\frac{\sqrt{x+1}}{\sqrt{x-1}}=
\lim_{x\to \infty}\frac{\sqrt{x}\sqrt{1+\frac1x}}{\sqrt{x}\sqrt{1-\frac1x}}=
1$. 
\eqref{itexBH12d} $-1/3$ (sem B.H.!)
\eqref{itexBH13} $2$
\eqref{itexBH15} $0$ (B.H. não se aplica)
\eqref{itexBH16} $\lim_{x\to \infty}\frac{x+\sen x}{x}=\lim_{x\to
\infty}(1+\frac{\sen x}{x})=1+0=1$ (Obs: Aqui B.H. não se aplica, porqué
$\lim_{x\to \infty}\frac{(x+\sen x)'}{(x)'}=\lim_{x\to
\infty}(1+\cos x)$, que  não existe.)
\eqref{itexBH6} $\tfrac13$
\eqref{itexBH17} $\lim_{x\to 0^+}\frac{x^2 \sen \frac{1}{x}}{x}=
\lim_{x\to 0^+} x\sen \frac{1}{x}=0$, com um ``sanduíche''. Aqui B.H. não se
aplica, porqué o limite $\lim_{x\to 0^+}(x^2 \sen \frac{1}{x})'$ não existe.
\eqref{itexBH18} $\frac13$. \eqref{itexBH20} (Segunda prova, Segundo semestre de
2011) Como $\lim_{y \to
\infty}\arctan y=\frac{\pi}{2}$, o limite 
é da forma $\frac00$. As funções são deriváveis em $x>0$, logo pela
regra de B.H.,
$$
\lim_{x\to 0^+}\frac{\arctan(\frac1x)-\tfrac{\pi}{2}}{x}=
\lim_{x\to 0^+}\frac{\frac{1}{1+(\frac{1}{x})^2}(-\frac{1}{x^2})}{1}=
\lim_{x\to 0^+}\frac{-1}{1+x^2}=-1\,.
$$
\eqref{itexBHww8} $1/2$.
\end{sol}
\end{exo}


Vários outros tipos de limites, por exemplo indeterminações
``$1^\infty$'', podem ser calculados usando o Teorema
\ref{Teo:BH2}. Aqui usaremos \emph{exponenciação}.
\begin{ex}
Para calcular $\lim_{x\to \infty}(\frac{x}{x-a})^x$, que é da forma 
c``$1^\infty$'', comecemos exponenciando
$$\Bigl(\frac{x}{x-a}\Bigr)^x=\exp\Bigl(x\ln \frac{x}{x-a}\Bigr)\,.$$ 
Como
$x\mapsto e^x$ é contínua, $\lim_{x\to \infty}(\frac{x}{x-a})^x=\exp(
\lim_{x\to \infty}x\ln \frac{x}{x-a})$ (lembre da Seção \ref{Sec:FuncConteLim}).
Ora, o limite $\lim_{x\to \infty}x\ln \frac{x}{x-a}$ pode ser escrito na forma
de um quociente:
$$
\lim_{x\to \infty}x\ln\frac{x}{x-a}=\lim_{x\to
\infty}\frac{\ln\frac{x}{x-a}}{\frac{1}{x}}
=\lim_{x\to\infty}\frac{\frac{1}{x}-\frac{1}{x-a}}{-\frac{1}{x^2}}=
\lim_{x\to \infty}\frac{ax^2}{x(x-a)}=a\,.
$$
A segunda igualdade é justificada pela regra de B.-H. (as funções
são {deriváveis} em todo $x$ suficientemente grande), 
a última por uma conta fácil de limite, colocando $x^2$ em evidência.
Portanto, 
$$
\lim_{x\to \infty}\Bigl(\frac{x}{x-a}\Bigr)^x=\exp\Big(
\lim_{x\to \infty}x\ln \frac{x}{x-a}\Big)=e^a\,.
$$
% Observe que o limite original podia também ser calculado escrevendo
% $$\lim_{x\to
% \infty}\Big(\frac{x}{x-a}\Big)^x=\lim_{x\to
% \infty}\frac{1}{\big(1+\frac{-a}{x}\big)^x}\to
% \frac{1}{e^{-a}}
% =\frac{1}{\lim_{x\to
% \infty}\big(1+\frac{-a}{x}\big)^x}\to
% \frac{1}{e^{-a}}
% =e^a\,.$$
\end{ex}

\begin{ex} Considere $\lim_{x\to 0}(\cos x)^{1/x^2}=\exp(\lim_{x\to
0}\frac{\ln(\cos x)}{x^2})$.
Como $\ln(\cos x)$ e $x^2$ são ambas deriváveis na vizinhança de zero, e como
$$
\lim_{x\to 0}\frac{(\ln(\cos x))'}{(x^2)'}=
\lim_{x\to 0}\frac{-\tan x}{2x}=
-\tfrac12\lim_{x\to 0}\frac{\sen x}{x}\frac{1}{\cos x}=-\tfrac12\,,
$$
temos 
$$\lim_{x\to 0}(\cos x)^{1/x^2}
=e^{-\tfrac12}=\frac{1}{\sqrt{e}}\,.
$$
\end{ex}

\begin{exo} Calcule:
\begin{multicols}{3}
\begin{enumerate}
\item\label{itexBHB1} $\lim_{x\to 0^+}(\sqrt{1+x})^{\tfrac1x}$
\item\label{itexBHB2} $\lim_{x\to 0^+}x^x$
\item\label{itexBHB3} $\lim_{x\to 0}(1+\sen (2x))^{\tfrac1x}$
\item\label{itexBHB5} $\lim_{x\to 0}(\sen x)^{\sen x}$
\item\label{itexBHB58} $\lim_{x\to \infty}(e^x+ x^2)^{\tfrac1x}$
\item\label{itexBHB6} $\lim_{x\to \infty}(\ln x)^{\tfrac1x}$
\item\label{itexBHB7} $\lim_{x\to 0^+}(1+x)^{\ln x}$
\item\label{itexBHB75} $\lim_{x\to \infty}(xe^{\tfrac1x}-x)$
\item\label{itexBHB8} {\footnotesize{$\lim_{x\to \infty}{(\pisobredois-\arctan
x)^{\tfrac{1}{\ln x}}}$}}
\item\label{itexBHB2bis} $\lim_{x\to 0^+}x^{x^x}$
\item\label{itexBHB4} $\lim_{x\to 0^+}\frac{(1+x)^{\tfrac1x}-e}{x}$
\end{enumerate}
\end{multicols}
\vspace{0.01cm}
\begin{sol} 
\eqref{itexBHB1} $\sqrt{e}$
\eqref{itexBHB2} $\lim_{x\to 0^+}x^x=\exp(\lim_{x\to 0^+}x\ln x)=e^0=1$.
\eqref{itexBHB3} $e^2$
\eqref{itexBHB5} $1$
\eqref{itexBHB58} $e$
\eqref{itexBHB6} $1$
\eqref{itexBHB7} $1$
\eqref{itexBHB75} $1$
\eqref{itexBHB8} $e^{-1}$
\eqref{itexBHB2bis} $0$
\eqref{itexBHB4} $-e/2$
\end{sol}
\end{exo}

\begin{exo} (Segunda prova, 27 de maio de 2011)
 Calcule os limites 
$$\lim_{z\to \infty}\Bigl(\frac{z+9}{z-9}\Bigr)^z\,,\quad\quad
\lim_{x\to \infty}x^{\ln x}e^{-x}\,,\quad
\lim_{x\to \infty}\frac{\sqrt{2x+1}}{\sqrt{x-1000}}\,.
$$
\begin{sol} Para o primeiro,
\begin{align*}
 \lim_{z\to \infty}\bigl(\frac{z+9}{z-9}\bigr)^z&=\exp \Bigl(\lim_{z\to \infty}
z \ln \frac{z+9}{z-9}\Bigr)\\
&=\exp \Bigl(\lim_{z\to \infty} \frac{\ln (z+9)-\ln
(z-9)}{\frac{1}{z}}\Bigr)\text{ e as hipót. de BH satisfeitas, logo}\\
&=\exp \Bigl(\lim_{z\to \infty}
\frac{\frac{1}{z+9}-\frac{1}{z-9}}{\frac{-1}{z^2}}\Bigr)\\
&=\exp \Bigl(\lim_{z\to \infty} \frac{18 z^2}{z^2-81}\Bigr)\\
&=e^{18}\,. 
\end{align*}
Para o segundo,
\begin{align*}
 \lim_{x\to \infty}x^{\ln x}e^{-x}&=\exp \Bigl(\lim_{x\to \infty} \big((\ln
x)^2-x \big)\Bigr)
=\exp \Bigl(\lim_{x\to \infty} x\big(\frac{(\ln x)^2}{x}-1
\big)\Bigr)
\end{align*}
Usando BH duas vezes, verifica-se que $\lim_{x\to \infty}\frac{(\ln
x)^2}{x}=0$, 
o que implica $\lim_{x\to \infty} x(\frac{(\ln
x)^2}{x}-1)=-\infty$.
Logo, $\lim_{x\to \infty}x^{\ln x}e^{-x}=0$.
O último limite se calcula sem usar B.H.:
$$\lim_{x\to \infty}\frac{\sqrt{2x+1}}{\sqrt{x-1000}}=\sqrt{2}\lim_{x\to
\infty}\frac{\sqrt{1+\frac{1}{2x}}}{\sqrt{1-\frac{1000}{x}}}=\sqrt{2}\frac{1}{1}
=\sqrt{2}\,.$$
\end{sol}
\end{exo}

%%%%INSERIR ASSINTOTAS OBLIQUAS:





% !TeX spellcheck = pt_BR
% !TEX encoding = UTF-8 Unicode

\chapter{Extremos e problemas de otimização}\label{Cap:Extremos}
  
\ifdefined\updateans
% Only need to run once in a lifetime, when the file ans.tex needs to be updated.
\Writetofile{ans}{\protect\section*{Capítulo \ref{Cap:Extremos}}}
\fi

\index{máximos e mínimos}

%\section{Valores extremos}\label{Sec:MineMax}
Neste capítulo resolveremos vários problemas concretos de \emph{otimização}. 
Basicamente, se tratará de definir uma função associada a uma situação
concreta, e de encontrar os maiores e menores valores tomados por
ela.
Primeiro, definiremos o que significa ``maior/menor valor'', no
sentido global e local. Em seguida veremos como a derivada aparece na
procura desses valores. Nos problemas de otimização estudados depois,
mostraremos como a Lei de Snell, bem conhecida em ótica, pode ser
obtida a partir de um problema de otimização.

\section{Extremos globais}

\begin{defin}
Considere uma função $f:D\to \bR$.
\begin{enumerate}
\item Um ponto $x_*\in D$ é chamado de \grasA{máximo global
de $f$} se $f(x)\leq f(x_*)$ para todo $x\in D$.
\index{máximo!global}
Diremos então que $f$ \grasA{atinge o seu valor máximo em $x_*$}.
\index{mínimo!global}
\item Um ponto $x_*\in D$ é chamado de \grasA{mínimo global
de $f$} se $f(x)\geq f(x_*)$ para todo $x\in D$.
Diremos então que $f$ \grasA{atinge o seu valor mínimo em $x_*$}.
\end{enumerate}
\end{defin}
Um \grasA{problema de otimização} 
consiste em achar um extremo (isto é, um mínimo ou
um máximo) global de uma função dada. 

\begin{ex}\label{Ex:extremxdois}
A função $f(x)=x^2$, em $D=[-1,2]$,
atinge o seu mínimo global em $x=0$ e 
o seu máximo global em $x=2$.
Observe que ao considerar a mesma função $f(x)=x^2$ com um domínio diferente,
os extremos globais mudam. Por exemplo, com $D=[\tfrac12,\frac32]$, 
$f$ atinge o seu mínimo global em $x=\tfrac12$, e o seu máximo global em
$x=\frac32$.
\begin{center}
\begin{bmlimage}\begin{tikzpicture}[scale=0.7]
\newcommand{\funcao}[1]{(#1)^2}

\begin{scope}
\draw[ ->, thin] (-1.3,0)--(2.3,0);
\draw[ ->, thin] (0,-0.2)--(0,4.3);
\draw[thick, domain=-1:2] plot (\x,{\funcao{\x}});
\fill (-1,1) circle (0.4mm);
\fill (2,4) circle (0.4mm);
\draw(0,3) node[left]{${D=[-1,2]}$};
\draw[dotted] (-1,0)node[below]{$\scriptstyle{-1}$}--(-1,1);
\draw[dotted] (2,0)node[below]{$\scriptstyle{2}$}--(2,4);
\coordinate (m) at (0,0);
\coordinate (M) at (2,4);
\fill (m) circle (0.5mm);
\draw (m) node[below]{\footnotesize{mín.}};
\fill (M) circle (0.5mm);
\draw (M) node[above]{\footnotesize{máx.}};
\end{scope}

\begin{scope}[xshift=8cm]
\draw[ ->, thin] (-1.3,0)--(2.3,0);
\draw[ ->, thin] (0,-0.2)--(0,4.3);
\draw[color=gray, domain=-1:2] plot (\x,{\funcao{\x}});
\draw[thick, domain=0.5:1.5] plot (\x,{\funcao{\x}});
%\fill (-0.5,0.25) circle (0.4mm);
\fill (1.5,2.25) circle (0.4mm);
\draw(0,3) node[left]{${D=[\tfrac12,\tfrac32]}$};
\draw[dotted] (0.5,0)node[below]{$\scriptstyle{\tfrac12}$}--(0.5,0.25);
\draw[dotted] (1.5,0)node[below]{$\scriptstyle{\tfrac32}$}--(1.5,2.25);
\coordinate (m) at (0.5,0.25);
\coordinate (M) at (1.5,2.25);
\fill (m) circle (0.5mm);
\draw (m) node[left]{\footnotesize{mín.}};
\fill (M) circle (0.5mm);
\draw (M) node[above]{\footnotesize{máx.}};
\end{scope}
\end{tikzpicture}\end{bmlimage}
\end{center}
\end{ex}

\begin{ex}
Considere $f(x)=\frac{x^3}{3}-x$ em $[-\sqrt{3},\sqrt{3}]$. 
Pelo gráfico do Exercício \ref{Ex:variacoesbasicas}, vemos que $f$ atinge o seu
máximo global em $x=-1$ e o seu mínimo global em $x=+1$.
\end{ex}

Uma função pode \emph{não possuir} mínimos e/ou máximos, por várias razões.

\begin{ex}\label{Ex:naoexistenciaextremos1}
$f(x)=e^{-\tfrac{x^2}{2}}$ (lembre do Exercício
\ref{Ex:variacoesbasicas}) em $\bR$  possui um máximo global em $x=0$:
\begin{center}
\begin{bmlimage}\begin{tikzpicture}[scale=0.7]
\draw[ ->] (-4,0)--(4,0);
\draw[ ->] (0,-0.2)--(0,1.3);
\draw[thick, domain=-4:4, samples=50] plot (\x,{exp(-\x*\x*0.5)});
\end{tikzpicture}\end{bmlimage}
\end{center}
Mas $f$ não possui ponto de mínimo global.
De fato, a função é sempre positiva e tende a zero quando $x\to\pm
\infty$. Logo, escolhendo pontos $x$ sempre mais longe da origem,
consegue-se alcançar valores sempre menores, não nulos: não 
pode existir um ponto $x_*$ em que a função toma um
valor menor ou igual a todos os outros pontos.
\end{ex}

\begin{ex}\label{Ex:naoexistenciaextremos2}
A função $f(x)=\frac{1}{1-x}$ em $D=[0,1)$ possui um mínimo global em
$x=0$:
\begin{center}
\begin{bmlimage}\begin{tikzpicture}
\newcommand{\funcao}[1]{ 1/(1-#1)}
\draw[ ->, thin] (-0.3,0)--(1.3,0);
\draw[ ->, thin] (0,-0.2)--(0,2.3);
%\draw[color=, domain=0:0.6] plot (\x,{\funcao{\x}});
\draw[thick, domain=0:0.6] plot (\x,{\funcao{\x}});
\fill (0,1) circle (0.4mm);
%\fill (2,4) circle (0.4mm);
%\draw(0,3) node[left]{${D=[-1,2]}$};
\draw[dashed] (1,-0.2)--(1,2.5)node[right]{$x=1$};
%\draw[dotted] (-1,0)node[below]{$\scriptstyle{-1}$}--(-1,1);
%\draw[dotted] (2,0)node[below]{$\scriptstyle{2}$}--(2,4);
\coordinate (m) at (0,1);
% \coordinate (M) at (2,4);
% \fill (m) circle (0.5mm);
 \draw (m) node[left]{\footnotesize{mín.}};
% \fill (M) circle (0.5mm);
% \draw (M) node[above]{\footnotesize{máx.}};
\end{tikzpicture}\end{bmlimage}
\end{center}
Mas, como $x=1$ é assíntota vertical, $f$ não possui máximo
global: ao se aproximar de $1$ pela esquerda, a função toma valores
arbitrariamente grandes.
\end{ex}

\begin{ex}\label{Ex:funcaonaocontnaotemminmax}
Uma função pode também ser limitada e não possuir extremos globais:
\begin{center}
\begin{bmlimage}\begin{tikzpicture}
\draw(-2,0) node[left]{$\displaystyle{
f(x)\pardef 
\begin{cases}
x &\text{ se } 0\leq x<1\,,\\
0 &\text{ se } x=1\,,\\
x-2  & \text{ se }1<x\leq 2\,.
\end{cases}
}$};
\draw[ ->, thin] (-0.3,0)--(2.3,0);
\draw[ ->, thin] (0,-1.2)--(0,1.5);
\fill (0,0) circle (0.4mm);
\fill (1,0) circle (0.4mm);
\fill (2,0) circle (0.4mm);
\draw[thick, ->] (0,0)--(1,1);
\draw[thick, <-] (1,-1)--(2,0);
\draw[dotted] (1,-1)--(1,1);
\end{tikzpicture}\end{bmlimage}
\end{center}
\end{ex}

Os três últimos exemplos mostram que a não-existência de extremos globais
para uma função definida num intervalo pode ser oriundo 1) do intervalo não ser
limitado (como no Exemplo \ref{Ex:naoexistenciaextremos1}) ou não fechado
(como no Exemplo \ref{Ex:naoexistenciaextremos2}), 2) da função não ser
contínua (como no Exemplo \ref{Ex:funcaonaocontnaotemminmax}).
O seguinte resultado garante que se a função é contínua e o intervalo fechado,
então sempre existem extremos globais.

\begin{teo}\label{Teo:funcaocontcompactpossuiextr}
Sejam  $a<b$, e $f$ uma função contínua em $[a,b]$. Então $f$ possui
(pelo menos) um mínimo e (pelo menos) um máximo global em $[a,b]$.
\end{teo} 

\begin{exo}
Para cada função $f:D\to \bR$ a seguir, verifique se as hipóteses do Teorema 
\ref{Teo:funcaocontcompactpossuiextr} são satisfeitas. Em seguida,
procure os pontos de mínimo/máximo global (se tiver).
\begin{multicols}{2}
\begin{enumerate}
\item\label{itminmaxbasico1} $f(x)=3$, $D=\bR$.
\item\label{itminmaxbasico101} $f(x)=\ln x$, $D=[1,\infty)$
\item\label{itminmaxbasico10} $f(x)=e^{-x}$ em $\bR_+$
\item\label{itminmaxbasico2} $f(x)=|x-2|$, $D=(0,4)$
\item\label{itminmaxbasico3} $f(x)=|x-2|$, $D=[0,4]$
\item\label{itminmaxbasico4} $f(x)=|x^2-1|+|x|-1$, $D=[-\tfrac32,\tfrac32]$
\item\label{itminmaxbasico5} $f(x)=\frac{x^3}{3}-x$, $D=[-2,2]$
\item\label{itminmaxbasico6} $f(x)=\frac{x^3}{3}-x$, $D=[-1,1]$
\item\label{itminmaxbasico7} $\displaystyle{f(x)=
\begin{cases}
x &\text{ se } x\in [0,2)\,,\\
(x-3)^2  & \text{ se }x\in [2,4]\,.
\end{cases}
}$
\item\label{itminmaxbasico8} $\displaystyle{f(x)=
\begin{cases}
x &\text{ se } x\in [0,2)\,,\\
(x-3)^2+1  & \text{ se }x\in [2,4]\,.
\end{cases}
}$
\item\label{itminmaxbasico9} $f(x)=x^{\frac23}$ em $\bR$
\item\label{itminmaxbasico11} $f(x)=\sen x$ em $\bR$
\end{enumerate}
\end{multicols}
\vspace{0.01cm}
\begin{sol}
\eqref{itminmaxbasico1} As hipóteses do teorema não são satisfeitas, pois o
domínio não é um intervalo finito e fechado. Mesmo assim, qualquer $x\in \bR$ é
ponto de máximo e mínimo global ao mesmo tempo.
\eqref{itminmaxbasico101} As hipóteses não são satisfeitas: o intervalo
não é limitado. Tém um
mínimo global em $x=1$, não tem máximo global.
\eqref{itminmaxbasico10} Hipóteses não satisfeitas (domínio não limitado).
Máximo global em $x=0$, não tem mínimo global.
\eqref{itminmaxbasico2} Hipóteses não satisfeitas (o intervalo não é fechado).
Tém mínimo global em $x=2$, não tem máximo global.
\eqref{itminmaxbasico3} Hipóteses satisfeitas: mínimo global em $x=2$, máximos
globais em $x=0$ e $x=2$.
\eqref{itminmaxbasico4} Hipóteses satisfeitas: 
mínímos globais em $1,-1$ e $0$, máximos globais em 
$-\tfrac32$ e $\tfrac32$.
\begin{center}
\begin{bmlimage}\begin{tikzpicture}
\pgfmathsetmacro{\a}{1.5};
\draw [thick, domain=-\a:\a, samples=150] plot (\x,{abs(1-\x^2)+abs(\x)-1});
\pgfmathsetmacro{\x}{0.6};
\draw [ ->] (-\a-0.2,0)--(\a+0.2,0);
\draw [ ->] (0,-0.2)--(0,2);
\foreach \k in {-1.5,1.5}{
\draw (\k,0) node{$\shortmid$};
\draw[dotted] (\k,0)--(\k,{abs(1-\k^2)+abs(\k)-1});
\fill (\k,{abs(1-\k^2)+abs(\k)-1}) circle (0.45mm);
}
\draw (-1.5,0) node[below]{$-\tfrac32$};
\draw (1.5,0) node[below]{$\tfrac32$};
\end{tikzpicture}\end{bmlimage}
\end{center}
\eqref{itminmaxbasico5} Hipóteses satisfeitas: mínimos globais em $x=-2$ e
$+1$, máximos globais em $x=-1$ e $+2$.
\eqref{itminmaxbasico6} Hipóteses satisfeitas: mínimo global em $x=+1$, máximo
global em $x=-1$.
\eqref{itminmaxbasico7} Hipóteses não satisfeitas ($f$ não é contínua). Não tem
máximo global, tem mínimos globais em $x=0$ e $+3$.
\eqref{itminmaxbasico8} Hipóteses satisfeitas: mínimo global em $x=0$, máximos
locais em $x=2$ e $4$.
\eqref{itminmaxbasico9} Hipóteses não satisfeitas ($f$ é contínua, mas o
domínio não é limitado). Tém mínimo global em $x=0$, não possui máximo global.
\eqref{itminmaxbasico11} Hipóteses não satisfeitas (intervalo não
limitado). No entanto, tem infinitos mínimos
globais, em todos os pontos da forma $x=-\pisobredois+k2\pi$, e infinitos
máximos globais, em todos os pontos da forma $x=\pisobredois+k2\pi$.
\end{sol}
\end{exo}

\section{Extremos locais}

\begin{defin}
Considere uma função real $f$.
\begin{enumerate}
\item Um ponto $x_*\in D$ é chamado de \grasA{máximo local
de $f$} se existir um intervalo aberto $I\ni x_*$ tal que
$f(x)\leq f(x_*)$ para todo $x\in I$.
\index{máximo!local}
\item Um ponto $x_*\in D$ é chamado de \grasA{mínimo local
de $f$} se existir um intervalo aberto $I\ni x_*$ tal que
$f(x)\geq f(x_*)$ para todo $x\in I$.
\index{mínimo!local}
\end{enumerate}
\end{defin}
\begin{center}
\begin{bmlimage}\begin{tikzpicture}[scale=0.7]
\newcommand{\funcao}[1]{ -0.3* ( (#1)^4- 4*(#1)^2 + 3*(#1) ) + 1.7}
\draw[->] (-3,0)--(2.4,0);
\pgfmathsetmacro{\a}{-1.56};
\pgfmathsetmacro{\b}{1.17};
\fill[color=gray!15]
(\b-0.3,0)--plot[domain=\b-0.3:\b+0.3](\x,{\funcao{\x}})--(\b+0.3,0)--cycle;
\draw[thick, domain=-2.3:1.8, samples=50] plot (\x,{\funcao{\x}});
\draw[dotted] (\a,0)node[below]{$x_1$}--(\a,{\funcao{\a}});
\fill (\a,{\funcao{\a}}) circle (0.4mm);
\draw (\a,{\funcao{\a}}) node[above]{\footnotesize{global}};
\draw[dotted] (\b,0)node[below]{$x_2$}--(\b,{\funcao{\b}});
\fill (\b,{\funcao{\b}}) circle (0.4mm);
\draw (\b,{\funcao{\b}}) node[above]{\footnotesize{local}};
\draw[->] (0.3,0.8)node[left]{$I$}--(\b-0.2,0.1);
\end{tikzpicture}\end{bmlimage}
\captionof{figure}{Uma função com um máximo global
em $x_1$ e um máximo local em $x_2$.}\label{Fig:maxgloballocal}
\end{center}

Observe que um ponto de máximo (resp. mínimo) global, quando pertencente
ao interior do domínio, é local ao mesmo tempo. 
Vejamos agora como que extremos locais podem ser encontrados usando derivada.

\begin{teo}\label{Teo:maxminlocderivadazero}
Seja $f$ uma função com um máximo (resp. mínimo) local em $x_*$. 
Se $f$ é derivável em $x$, então $f'(x_*)=0$.
\end{teo}
\begin{proof}
Seja $x_*$ um máximo local (se for mínimo local, a prova é parecida).
Isto é, $f(x)\leq f(x_*)$ para todo $x$ suficientemente perto de $x_*$. 
Como $f'(x_*)$ existe por hipótese, podemos escrever 
$f'(x_*)=\lim_{x\to x_*^+}\frac{f(x)-f(x_*)}{x-x_*}$. Mas aqui $x-x_*>0$, e 
como $x_*$ é máximo local, $f(x)-f(x_*)\leq 0$. Portanto, 
$f'(x_*)\leq 0$. 
Por outro lado, podemos escrever 
$f'(x_*)=\lim_{x\to x_*^-}\frac{f(x)-f(x_*)}{x-x_*}$. Aqui, $x-x_*<0$, e 
$f(x)-f(x_*)\leq 0$, logo $f'(x_*)\geq 0$. Consequentemente, $f'(x_*)=0$.
\end{proof}

O resultado acima permite achar \emph{candidatos} a pontos de mínimo/máximo
local. Vejamos alguns exemplos.

\begin{ex}
Considere $f(x)=1-x^2$, que é obviamente derivável.
Logo, sabemos pelo Teorema \ref{Teo:maxminlocderivadazero} que qualquer extremo
local deve anular a derivada. Como $f'(x)=-2x$, e como $f'(x)=0$ se e somente se
$x=0$, o ponto $x=0$ é candidato a ser um extremo local. Para determinar se de
fato é, estudemos o sinal de $f'(x)$, e observemos que $f'(x)>0$ 
se $x<0$, $f'(x)<0$ se $x>0$. Logo, $f$ cresce antes de $0$, decresce depois:
$x=0$ é um ponto de máximo local:
\begin{center}
\begin{bmlimage}\begin{tikzpicture}[scale=0.8]
\tkzTabInit[nocadre,espcl=2,  color, colorV=lightgray!5, colorL=gray!15,
colorC=gray!15]
{$x$ /.6,  $f'(x)$ /.9, Var. $f$ /1.5}%
{,$0$, }%
\tkzTabLine{,+,z,-,}
\tkzTabVar{-/,+/\text{máx.},-/$0$,}
\begin{scope}[xshift=10cm, yshift=-2.2cm]
 \draw[ ->] (-1.5,0)--(1.5,0);
 \draw[ ->] (0,-0.3)--(0,2);
\draw[thick, domain=-1.1:1.1] plot (\x,{1-(\x)^2});
\fill (0,1) circle (0.4mm);
\draw (0,1) node[above]{máx.};
\end{scope}
\end{tikzpicture}\end{bmlimage}
\end{center}
Observe que podia também calcular $f''(x)=-2$, que é sempre $<0$, o que implica
que $f$ é côncava, logo $x=0$ só pode ser um máximo local.
A \emph{posição} do máximo local no gráfico de $f$ é $(0,f(0))=(0,1)$.
\end{ex}

\begin{obs}
No exemplo anterior, localizamos um ponto onde a primeira derivada é nula, e
em seguida usamos o \grasA{teste da segunda derivada}: 
estudamos o sinal da segunda derivada neste mesmo ponto para determinar se ele
é um mínimo ou um máximo local.
\end{obs}

\begin{ex}
Considere $f(x)=x^3$, derivável também.
Como $f'(x)=3x^2$, $x=0$ é candidato a ser ponto de mínimo ou máximo local.
Ora, vemos que $f'(x)\geq 0$ para todo $x$, logo \emph{$f'$ não muda de sinal em
$x=0$}. Portanto esse ponto não é nem mínimo, nem máximo.
\end{ex}

\begin{ex}
A função $f(x)=|x|$ possui um mínimo local (que também é global) em $x=0$.
Observe que esse fato não segue do Teorema
\ref{Teo:maxminlocderivadazero}, já que $f$ não é derivável em zero.
\end{ex}

\begin{ex}
Considere
$f(x)=\frac{x^4}{4}-\frac{x^2}{2}$, que é derivável em todo $x$.
Como $f'(x)=x^3-x=x(x^2-1)$, as soluções de $f'(x)=0$ são $x=-1$, $x=0$,
$x=+1$. A tabela de variação já foi montada no Exercício
\ref{Ex:variacoesbasicas}. Logo,
$x=-1$ e $x=+1$ são pontos de mínimo local (posições:
$(-1,f(-1))=(-1,-\frac12)$ e $(+1,f(+1))=(+1,-\frac12)$), e $x=0$ é máximo
local (posição: $(0,0)$).
\end{ex}

\begin{exo}
Para cada função abaixo (todas são deriváveis), determine os extremos locais
(se tiver). %Quando puder, monte um gráfico, por exemplo usando 
\begin{multicols}{3}
\begin{enumerate}
\item\label{itextremoslocais1} $2x^3+3x^2-12x+5$
\item\label{itextremoslocais2} $2x^3+x$
\item\label{itextremoslocais3} $\frac{x^4}{4}+\frac{x^3}{3}$
\item\label{itextremoslocais30} $\frac{x^2+1}{x^2+x+1}$
\item\label{itextremoslocais31} $e^{-\frac{x^2}{2}}$
\item\label{itextremoslocais32} $xe^{-x}$
\item\label{itextremoslocais5} $\frac{x}{1+x^2}$
\item\label{itextremoslocais6} $x^x$, $x>0$
\item\label{itextremoslocais4} $x(\ln x)^2$, $x>0$
\end{enumerate}
\end{multicols}
\vspace{0.01cm}
\begin{sol}
\eqref{itextremoslocais1} Máximo local no ponto $(-2,25)$, um mínimo local (e
global) em $(1,-2)$.
\eqref{itextremoslocais2} Sem mín./máx.
\eqref{itextremoslocais3} Mínimo local (e global) em
$(-1,-\frac{1}{12})$ (Atenção: a derivada é nula em $x=0$, mas não é nem
máximo nem mínimo pois a derivada não muda de sinal).
\eqref{itextremoslocais30} $f'(x)=-\frac{1-x^2}{x^2+x+1}$, tem um mínimo local
(em global) em $(1,f(1))$, um máximo local (e global) em $(-1,f(-1))$.
\eqref{itextremoslocais31} Máximo local (e global) em $(0,1)$.
\eqref{itextremoslocais32} Máximo local em $(1,e^{-1})$.
\eqref{itextremoslocais5} Mínimo local em $(-1,-\frac12)$, máximo local em
$(1,\frac12)$.
\eqref{itextremoslocais6} Mínimo local em $(e^{-1},e^{-1/e})$.
\eqref{itextremoslocais4} Máximo local em $(e^{-2}, 4e^{-2})$, mínimo local em
$(1,0)$.
\end{sol}
\end{exo}

\begin{exo}
Determine os valores dos parâmetros $a$ e $b$ para que $f(x)=x^3+ax^2+b$ tenha
um extremo local posicionado em $(-2,1)$.
\begin{sol}
$a=-b=3$.
\end{sol}
\end{exo}

\begin{exo} A energia de interação entre dois átomos (ou moléculas) a
distância $r>0$ é modelizado pelo \grasA{potencial de 
Lennard-Jones}~\footnote{Sir John Edward Lennard-Jones (27 de outubro de 1894 –
1 de novembro de 1954).}:
$$
V(r)=4\epsilon\Big\{
\big(\frac{\sigma}{r}\big)^{12}
-\big(\frac{\sigma}{r}\big)^6
\Big\}\,,
$$
onde $\epsilon$ e $\sigma$ são duas constantes positivas.
\begin{enumerate}
 \item\label{itLJ1} Determine a distância $r_0$ tal que o potencial seja zero.
\item\label{itLJ2} Determine a distância $r_*$ tal que a interação seja mínima.
Existe máximo global? Determine a variação e esboce $V$.
\index{Potencial!de Lennard-Jones}
\end{enumerate}
\begin{sol}
\eqref{itLJ1} $r_0=\sigma$, \eqref{itLJ2} $r_*=\sqrt[6]{2}\sigma$. 
Como $\lim_{r\to 0^+}V(r)=+\infty$, $V$ não possui máximo global.
$V$ decresce em $(0,r_*]$, cresce em $[r_*,\infty)$:
\begin{center}
\begin{bmlimage}\begin{tikzpicture}
\pgfmathsetmacro{\e}{1};
\pgfmathsetmacro{\s}{1};
\newcommand{\funcao}[1]{4*\e*(\s/(#1))^(12)-4*\e*(\s/(#1))^(6)}
\draw[->] (0,0)--(4.4,0)node[right]{$r$};
\draw[->] (0,-1)--(0,2)node[left]{$V(r)$};
\pgfmathsetmacro{\rzer}{\s};
\pgfmathsetmacro{\ret}{\s*1.122};
\draw[thick, domain=0.95:3.8, samples=50] plot (\x,{\funcao{\x}});
\draw[dotted] (\ret,0)node[above right]{$r_*$}--(\ret,{\funcao{\ret}});
\end{tikzpicture}\end{bmlimage}
\end{center}
Obs: O potencial de Lennard-Jones $V(r)$ descreve a energia de interação entre
dois átomos neutros a distância $r$. 
Quando $0<r<r_0$ essa energia é positiva (os átomos se repelem), e quando
$r_0<r<\infty$ essa energia é negativa (os átomos se atraem).
Vemos que quando $r\to \infty$, a energia tende a zero e que ela tende a
$+\infty$ quando $r\to 0^+$: a distâncias longas, os átomos não interagem, e a
distâncias curtas a energia diverge (caroço duro).
A posição mais estável é quando a distância entre os dois átomos é
$r=r_*$.
\end{sol}
\end{exo}

\section{A procura de extremos em intervalos
fechados}\label{sec:extremintfechados}

Daremos agora o método geral para determinar os extremos globais de uma função
$f:[a,b]\to \bR$. Suporemos que $f$ é \emph{contínua}; assim o Teorema
\ref{Teo:funcaocontcompactpossuiextr} garante que os extremos existem. \\

Vimos que extremos \emph{locais} são ligados, quando $f$ é derivável, aos
pontos onde a derivada de $f$ é nula. Chamaremos tais pontos de
\emph{pontos críticos}.

\begin{defin} Seja $f:D\to \bR$.
Um ponto $a\in D$ é chamado de \grasA{ponto crítico} de $f$ se a
derivada de $f$ não existe em $a$, ou se ela existe e é nula: $f'(a)=0$.
\end{defin}

Por exemplo, $a=0$ é ponto crítico de $f(x)=x^2$, porqué $f'(0)=0$. Por outro
lado, $a=0$ é ponto crítico da função $f(x)=|x|$, porqué $f$ não é
derivável em zero.\\

Às vezes, os extremos são ligados a pontos críticos mas vimos que eles podem
também se encontrar na \emph{fronteira} do intervalo considerado (como nos
Exemplos \ref{Ex:naoexistenciaextremos1} e \ref{Ex:extremxdois}). Logo, o
procedimento para achar os valores extremos de $f$ é o seguinte:\\

\emph{
 Seja $f$ uma função contínua no intervalo fechado e limitado
$[a,b]$. Os extremos globais de $f$ são determinados da seguintes maneira:
\begin{itemize}
 \item Procure os pontos críticos $x_1,x_2,\dots,x_n$ 
de $f$ contidos em $(a,b)$ (isto é, em $[a,b]$ mas diferentes de $a$ e de $b$). 
\item Olhe $f$ na fronteira do intervalo, calcule $f(a)$, $f(b)$.
\item Considere a lista $\{f(a), f(b),
f(x_1),\dots,f(x_n)\}$. O maior valor dessa lista dá o máximo global; o menor
dá o mínimo global.
\end{itemize}
}

\begin{ex}
Procuremos os extremos globais da função $f(x)=2x^3-3x^2-12x$ no intervalo
$[-3,3]$. Como esse intervalo é fechado e que $f$ é contínua, podemos aplicar o
método descrito acima.
Os pontos críticos são solução de $f'(x)=0$, isto é, solução de $6(x^2+x-2)=0$.
Assim, $f$ possui dois pontos críticos, $x_1=-1$ e $x_2=+2$, e ambos pertencem
a $(-3,3)$. Observe também que $f(x_1)=f(-1)=+7$, e $f(x_2)=f(2)=-20$.
Agora, na fronteira do intervalo temos $f(-3)=-45$, $f(+3)=-9$. Assim, olhando
para os valores $\{f(-3), f(+3), f(-1), f(+2)\}$, vemos que o maior é
$f(-1)=+7$ (máximo global), e o menor é $f(-3)=-45$ (mínimo global).
(Essa função já foi considerada no Exercício \ref{Ex:variacoesbasicas}.)
\end{ex}

\begin{ex}
Procuremos os extremos globais da função $f(x)=x^{2/3}$ no intervalo
$[-1,2]$.
Se $x\neq 0$, então $f'(x)$ existe e é dada por $f'(x)=\tfrac23x^{-1/3}$. 
Em $x=0$, $f$ não é derivável (lembre do Exemplo \ref{Ex:derivracine}). 
Logo, o único ponto crítico de $f$ em $(-1,2)$ é $x=0$. 
Na fronteira, $f(-1)=1$, $f(2)=\sqrt[3]{4}$. Comparando os valores 
$\{f(-1),f(2),f(0)\}$, vemos que o máximo global é atingido em $x=2$ e o mínimo
local em $x=0$:
\begin{center}
\begin{bmlimage}\begin{tikzpicture}
\draw[ ->] (-1.5,0)--(2.5,0);
\draw[ ->] (0,-0.2)--(0,1.5);
\draw[thick, domain=0.0001:2] plot (\x,{exp(0.6666*ln(\x))});
\draw[thick, domain=0.0001:1] plot (-\x,{exp(0.6666*ln(\x))});
\draw[dotted] (-1,0)node[below]{$-1$}--(-1,1);
\draw[dotted] (2,0)node[below]{$2$}--(2,1.587);
\draw (0,0) node[below]{mín.};
\draw (2,1.587) node[above]{máx.};
\fill (0,0) circle (0.40mm);
\fill (2,1.587) circle (0.40mm);
\end{tikzpicture}\end{bmlimage}
\end{center}
\end{ex}

Os exercícios relativos a essa seção serão incluidos na resolução dos
problemas de otimização.

\section{Problemas de otimização}
\index{otimização}
\begin{ex} 
\emph{Dentre os retângulos contidos debaixo da parábola $y=1-x^2$, com o
lado inferior no eixo $x$, qual é que tem maior área?}
Considere a família dos retângulos inscritos debaixo da parábola:
\begin{center}
\begin{bmlimage}\begin{tikzpicture}[scale=1.2]
\newcommand{\funcao}[1]{1-(#1)^2}
\pgfmathsetmacro{\a}{1.1};
\draw [domain=-\a:\a] plot (\x,{\funcao{\x}});
\pgfmathsetmacro{\x}{0.6};
\foreach \x in {0.3, 0.65, 0.9} {
%\draw ({-\x},0)--({-\x},{\funcao{\x}})--({\x},{\funcao{\x}})--({\x},0)--cycle;
\fill[color=gray!20] ({-\x},0) rectangle ({\x},{\funcao{\x}});
}
\foreach \x in {0.3, 0.65, 0.9} {
\draw ({-\x},0)--({-\x},{\funcao{\x}})--({\x},{\funcao{\x}})--({\x},0)--cycle;
%\fill[color=gray!20] ({-\x},0) rectangle ({\x},{\funcao{\x}});
}
\pgfmathsetmacro{\x}{0.65};
\draw[thick] ({-\x},0)--({-\x},{\funcao{\x}})--({\x},{\funcao{\x}})--({\x},
0)--cycle;
\draw[decorate, decoration=brace] (\x,0)--(0,0) node[midway, below]{$x$};
\draw [ ->] (-1.3,0)--(1.3,0);
\draw [ ->] (0,-0.4)--(0,1.2);
\end{tikzpicture}\end{bmlimage}
\end{center}
Fixemos um retângulo e chamemos de
$x$ a metade do comprimento do lado horizontal.
Como os cantos superiores estão
no gráfico de $y=1-x^2$, a altura do retângulo é
igual a $1-x^2$. Portanto, a área em função de $x$ é 
dada pela função $$A(x)=2x(1-x^2)\,.$$
Observe que $A$ tem domínio $[0,1]$ (o menor lado horizontal possível é $0$, o
maior é
$2$). 
Para achar os valores extremos de $A$, procuremos os seus pontos críticos em
$(0,1)$, soluções de $A'(x)=0$.
Como $A'(x)=2-6x^2$, o único ponto crítico é
$x_*=\frac{1}{\sqrt{3}}$. O estudo do
sinal mostra que $x_*$ 
é um ponto de máximo local de $A$. Como $A(0)=0$ e
$A(1)=0$, o máximo global é atingido em $x_*$ mesmo. Logo o retângulo de
maior área tem largura $2x_*\simeq 1.154$ e altura
$1-x_*^2=\frac{2}{3}=0.666\dots$.
\end{ex}

O método usado neste último exemplo pode ser usado na resolução de
outros problemas: 
\begin{enumerate}
\item Escolher uma variável que descreve a situação e os
objetos envolvidos no problema. Determinar os valores possíveis dessa
variável.
\item Montar uma função dessa variável, que represente a quantidade a
ser maximizada (ou minimizada).
\item Resolver o problema de otimização correspondente, usando as
ferramentas descritas na Seção~\ref{sec:extremintfechados}.
\end{enumerate}

\begin{exo}
Qual é o retângulo de maior área que pode ser inscrito 
\begin{enumerate}
 \item\label{itexoretanginscrito1} em um círculo de raio $R$?
\item\label{itexoretanginscrito2} no triângulo determinado pelas três retas
$y=x$, $y=-2x+12$ e $y=0$?
\end{enumerate}

\begin{sol} \eqref{itexoretanginscrito1}
A função área é dada por $A(x)=4x\sqrt{R^2-x^2}$, $x\in [0,R]$. O leitor pode
verificar que o seu máximo global em $[0,R]$ é atingido em
$x_*=\frac{R}{\sqrt{2}}$. Logo, o retângulo de maior área inscrito no círculo
tem largura $2x_*=\sqrt{2}R$, e altura $2\sqrt{R^2-x_*^2}=\sqrt{2}R$. Logo, é
um quadrado!
\eqref{itexoretanginscrito2} Usaremos a variável $h\in [0,4]$ definida da
seguinte maneira
\begin{center}
\begin{bmlimage}\begin{tikzpicture}[yscale=0.3]
\newcommand{\funcao}[1]{-2*(#1)+12}
\draw[ ->] (-0.2,0)--(6.5,0);
\draw[ ->] (0,-0.2)--(0,13);
\draw (-0.3,{\funcao{-0.3}})node[left]{$y=-2x+12$}--(7,{\funcao{7}});
\draw (-0.3,-0.3)--(5,5)node[right]{$y=x$};
\fill (4,4) circle (0.40mm);
\pgfmathsetmacro{\h}{2.2};
\draw[decorate, decoration=brace] (0,0)--(0,\h) node[midway, left]{$h$};
\draw[dotted] (0,\h)--(\h,\h);
\fill[color=gray!15] (\h,0) rectangle ({-\h/2+6},\h);
\draw[thick] (\h,0) rectangle ({-\h/2+6},\h);
\draw (\h,0) node[below]{$x_1$};
\draw ({-\h/2+6},0) node[below]{$x_2$};
\draw (4,4) node{$\bullet$};
% \fill (\h,\h) circle (0.50mm);
% \fill ({-\h/2+6},\h) circle (0.50mm);
\draw (4,4) node[above]{$(4,4)$};
\end{tikzpicture}\end{bmlimage}
\end{center}
A área do retângulo é dada por $A(h)=h(x_2-x_2)$. Ora, $x_1=h$ e
$x_2=6-\frac{h}{2}$. Logo, $x_2-x_1=6-\frac{3h}{2}$. Portanto,
queremos maximizar $A(h)=h(6-\frac{3h}{2})$ em
$h\in [0,4]$.
É fácil ver que o de máximo é atingido em $h_*=2$. Logo o maior retângulo tem
altura $h_*=2$, e largura $6-\frac{3h_*}{2}=3$.
\end{sol}
\end{exo}

\begin{exo}(Segunda prova, Segundo semestre de 2011)
Considere a família de todos os triângulos isósceles cujos dois lados iguais
tem tamanho igual a $1$:
\begin{center}
\begin{bmlimage}\begin{tikzpicture}
\draw[thick] (0,0)--(1,1)node[midway, above left]{$1$}--(2,0)node[midway,
above right]{$1$}--cycle;
\draw (0.8,0.8) arc (225:315:0.3);
\draw (1,0.5) node{$\theta$};
\end{tikzpicture}\end{bmlimage}
\end{center}
Qual desses triângulos tem maior área? 
\begin{sol}
A altura do triângulo de abertura
$\theta\in [0,\pi]$ é $\cos \frac{\theta}{2}$, a sua base é $2\sen
\frac{\theta}{2}$, logo a sua área é dada por
$$A(\theta)=\cos(\frac{\theta}{2})\sen (\frac{\theta}{2})=\frac12 \sen
\theta\,.\pt{3}$$
Queremos maximizar $A(\theta)$ quando $\theta\in [0,\pi]$.
Ora, $A(0)=A(\pi)=0$, e como $A'(\theta)=\frac12\cos \theta$,
$A'(\theta)=0$ se e somente se $\cos
\theta=0$, isto é, se e somente se $\theta=\frac{\pi}{2}$ $pt{1}$. Ora, como
$A'(\theta)>0$ se $\theta<\frac{\pi}{2}$, 
$A'(\theta)<0$ se $\theta>\frac{\pi}{2}$, $\frac{\pi}{2}$ é um máximo de $A$
$\pt{2}$. 
Logo, {o triângulo que tem maior área é aquele cuja abertura vale
$\frac{\pi}{2}$ $\pt{2}$.} Obs: pode também expressar a área em função do lado
horizontal $x$, $A(x)=\tfrac12 x\sqrt{1-(\tfrac{x}{2})^2}$.
Obs: Pode também introduzir a variável $h$, definida como
\begin{center}
\begin{bmlimage}\begin{tikzpicture}
\draw[thick] (0,0)--(1,0)node[midway,
below]{$\scriptstyle{1}$}--(1.5,0.866)node[midway,
below]{$\scriptstyle{1}$}--cycle;
\draw[dotted] (1.5,0)--(1.5,0.866) node[midway, right]{$h$};
\fill (1.5,0) circle (0.40mm);
%\draw (1,0.5) node{$\theta$};
\end{tikzpicture}\end{bmlimage}
\end{center}
e fica claro que o triângulo de maior área é aquele que tem maior altura $h$,
isto é, $h=1$ (aqui nem precisa calcular uma derivada...), o que acontece quando
a abertura vale $\frac{\pi}{2}$.
\end{sol}
\end{exo}

\begin{exo}
Dentre todos os retângulos de perímetro fixo igual a $L$, qual é o de maior
área?
\begin{sol}
Seja $x$ o tamanho do lado horizontal do retângulo, e $y$ o seu lado vertical.
A área vale $A=xy$.
Como o perímetro é fixo e vale $2x+2y=L$, podemos expressar $y$ em função de
$x$, $y=\frac{L}{2}-x$, e expressar tudo em termos de $x$:
$A(x)=x(\frac{L}{2}-x)$. Maximizar essa função em $x\in [0,L/2]$ mostra que $A$
é máxima quando $x=x_*=\frac{L}{4}$. Como $y_*=\frac{L}{2}-x_*=\frac{L}{4}$, o
retângulo com maior área é um quadrado!
\end{sol}
\end{exo}


\begin{exo}
Uma corda de tamanho $L$ é cortada em dois pedaços. Com o primeiro pedaço,
faz-se um quadrado, e com o segundo, um círculo. Como que a corda deve ser
cortada para que a área total (quadrado $+$
círculo) seja máxima? mínima?
\begin{sol}
Suponha que a corda seja cortada em dois pedaços. Com o primeiro pedaço, de
tamanho $x\in [0,L]$, façamos um quadrado: cada um dos seus lados tem lado
$\frac{x}{4}$, e a sua área vale $(\frac{x}{4})^2$. Com o outro pedaço façamos
um círculo, de perímetro $L-x$, logo o seu raio é $\frac{L-x}{2\pi}$, e a sua
área $\pi(\frac{L-x}{2\pi})^2$. Portanto, queremos maximizar a função 
$$
A(x)\pardef \frac{x^2}{16}+\frac{(L-x)^2}{4\pi}\,,\quad \text{ com }x\in
[0,L]\,.
$$
Na fronteira, $A(0)=\frac{L^2}{4\pi}$ (a corda inteira usada para fazer um
círculo), $A(L)=\frac{L^2}{16}$ (a corda inteira para fazer um quadrado).
Procuremos os pontos críticos de $A$: é fácil ver que $A'(x)=0$ se e somente
$x=x_*=\frac{L}{1+\frac{\pi }{4}}\in (0,L)$.
Como $A(x_*)=\frac{L^2}{4(4+\pi)}$, temos que $A(x_*)<A(L)<A(0)$. Logo, 
a área total mínima é obtida fazendo um quadrado com o primeiro pedaço de
tamanho $x_*\simeq 0.56 L$, e  um círculo com o outro pedaço ($L-x_*\simeq
0.43 L$). A área total máxima é obtida usando a corda toda para fazer um
círculo.
\end{sol}
\end{exo}

\begin{ex}
\emph{Qual é o ponto $Q_*$ da reta $y={2x}$ que está mais próximo do ponto
$P=(1,0)$?}
\begin{center}
\begin{bmlimage}\begin{tikzpicture}[scale=1.5]
\draw[->] (-0.5,0)--(3,0);
\draw[->] (0,-0.5)--(0,1.5);
\draw[thick] (-0.5,-0.25)--(2,1) node[right]{$y=2x$};
\fill (1,0) circle (0.5mm);
\fill (1.5,0.75) circle (0.5mm);
\draw (1,0) node[below]{$P$};
\draw[dotted] (1,0)--(1.5,0.75) node[above]{$Q$};
\end{tikzpicture}\end{bmlimage}
\end{center}
Se $Q=(x,y)$ é um ponto qualquer do plano, então
\[ d(Q,P)=\sqrt{(x-1)^2+y^2}\,.
\]
Mas se $Q$ pertence à reta, então $y=2x$ e podemos escrever a
distância em função da variável $x$ só: $d(Q,P)=f(x)$, onde 
\[ 
f(x)=\sqrt{(x-1)^2+(2x)^2}=
\sqrt{5x^2-2x+1}\,
\]
é a função que queremos \emph{minimizar}.
Como $Q$ pode se mover na reta toda, $f$ tem $\bR$ como domínio.
Como $f$ é derivável e $f'(x)=0$
se e somente se $x=\frac15$, e como $d$ é convexa ($d''(z)>0$ para todo $z$), o
ponto de abcissa $x=\frac15$ é um ponto de mínimo global de $d$. Logo, o ponto
procurado é $Q_*=(\frac{1}{5},\frac{2}{5})$.
Observe que a inclinação do segmento $Q_*P$ é igual a $-\tfrac12$: ele
é perpendicular à reta, como era de se esperar (sabemos desde o curso
de geometria elementar que o caminho mais curto
entre um ponto $P$ e uma reta é o segmento perpendicular à reta passando
por $P$).
\end{ex}

\begin{exo}
Qual é o ponto $Q_*$ da parábola $y=x^2$ cuja distância a $P=(10,2)$ é
mínima?
\begin{sol}
$Q_*=(2,4)$
\end{sol}
\end{exo}

\begin{exo}
Considere os pontos $A=(1,3)$, $B=(8,4)$. Determine o ponto $C$ do eixo $x$,
tal que o perímetro do triângulo $ABC$ seja mínimo.
\begin{sol}
Seja $C=(x,0)$, com $1\leq x\leq 8$. É preciso minimizar
$f(x)=\sqrt{(x-1)^2+3^2}+\sqrt{(x-8)^2+4^2}$
para $x\in [1,8]$.
Os pontos críticos de $f$ são soluções de $7x^2+112x-560=0$ (em $[1,8]$), isto é,
$x=4$. Como $f''(4)>0$, $x=4$ é um mínimo de $f$ (pode verificar calculando os
valores $f(1)$, $f(8)$).
Logo, $C=(4,0)$ é tal que o perímetro de $ABC$ seja mínimo.
\end{sol}
\end{exo}

\begin{exo}
Seja $r_\alpha$ a reta tangente ao gráfico da função $f(x)=3-x^2$, no ponto
$(\alpha,f(\alpha))$, $\alpha\neq 0$.  
Seja $\cT_\alpha$ o triângulo determinado pela origem e pelos pontos em
que $r_\alpha$ corta os eixos de coordenada. Determine o(s) valores de $\alpha$
para os quais a área de $\cT_\alpha$ é mínima.
\begin{sol}
$\alpha=\pm 1$.
\end{sol}
\end{exo}

\begin{exo}
 Considere um ponto $P=(a,b)$ fixo no primeiro quadrante. 
Para um ponto $Q$ no eixo $x$ positivo, considere a área do triângulo
determinado pelos eixos de coordenadas e pela reta que passa por $P$ e
$Q$. Ache a posição do ponto $Q$ que minimize a área do triângulo, e dê o valor
dessa área.
\begin{sol}
Considere a variável $x$ definida da seguinte maneira:
\begin{center}
\begin{bmlimage}\begin{tikzpicture}
\draw[ ->] (-0.2,0)--(4,0);
\draw[ ->] (0,-0.2)--(0,2);
\pgfmathsetmacro{\a}{1.5};
\pgfmathsetmacro{\b}{0.5};
\coordinate (P) at (\a,\b);
\pgfmathsetmacro{\d}{1.4};
\coordinate (Q) at (\a+\d,0);
\coordinate (Qp) at (0,{\b*(\d+\a)/\d});
\fill (P) circle (0.4mm);
\draw (P) node[above right]{$P=(a,b)$};
\fill (Q) circle (0.4mm);
\draw (Q) node[below]{$Q$};
\draw[dashed] (Q)--(Qp);
\draw[decorate, decoration=brace] (0,0)--(Qp) node[midway, left]{$h$};
\draw[decorate, decoration=brace] (\a,0)--(P) node[midway, left]{$b$};
\draw[decorate, decoration=brace] (\a,0)--(0,0) node[midway, below]{$a$};
\draw[decorate, decoration=brace] (Q)--(\a,0) node[midway, below]{$x$};
\end{tikzpicture}\end{bmlimage}
\end{center}
Assim temos que a área do triângulo em função de $x$, $A(x)$, é dada por 
$A(x)=\half (a+x)\cdot h$. Mas, como $\frac{h}{a+x}=\frac{b}{x}$, temos
$h=\frac{b(x+a)}{x}$, que dá
$A(x)=\frac{b}{2}\frac{(x+a)^2}{x}$.
Procuremos o mínimo de $A(x)$ para $x\in (0,\infty)$.
Como $A$ é derivável em todo $x>0$, $A'(x)=\frac{b}{2}\frac{(x-a)(x+a)}{x^2}$,
vemos que $A$ possui dois pontos críticos, em $-a$ e $+a$, e $A'(x)>0$ se
$x<-a$, $A'(x)<0$ se $-a<x<a$, e $A'(x)>0$ se $x>a$. Desconsideremos o $-a$ pois
queremos um ponto em $(0,\infty)$. Assim, o mínimo de $A$ é atingido em $x=a$,
e nesse ponto $A(a)=2ab$:
\begin{center}
\begin{bmlimage}\begin{tikzpicture}
\pgfmathsetmacro{\a}{1.5};
\pgfmathsetmacro{\b}{0.5};
\draw[ ->] (0,0)--(4,0)node[right]{$x$};
\draw[ ->] (0,-0.2)--(0,2.2) node[left]{$A(x)$};
\draw[thick, domain=0.4:4] plot (\x,{(\b*(\x^2+2*\a*\x+\a^2))/(2*\x)});
\draw[dotted] (\a,0)node[below]{$a$}--(\a,{2*\a*\b})--(0,{2*\a*\b})
node[left]{$2ab$};
\fill (\a,{2*\a*\b}) circle (0.45mm);
\end{tikzpicture}\end{bmlimage}
\end{center}
\end{sol}
\end{exo}

\begin{exo}
Qual é o triângulo isósceles de maior área que pode ser inscrito
dentro de um disco de raio $R$?
\begin{sol} Representamos o triângulo da seguinte maneira:
\begin{center}
\begin{bmlimage}\begin{tikzpicture}
\pgfmathsetmacro{\r}{1};
\draw (0,0) circle(\r cm);
\draw[->] (-\r-0.2,0)--(\r+0.2,0);
\draw[->] (0,-\r-0.2)--(0,\r+0.2);
\coordinate (A) at (0,\r);
\coordinate (B) at (0.5*\r,-0.866*\r);
\coordinate (T) at (0,-0.866*\r);
\coordinate (C) at  (-0.5*\r,-0.866*\r);
\fill[color=gray!30, opacity=0.8] (A)--(B)--(C)--cycle;
\draw[thick] (A)--(B)--(C)--cycle;
\draw[decorate, decoration={brace, raise=1pt}] (B)--(T)
node[midway, below]{$x$};
\end{tikzpicture}\end{bmlimage}
\end{center}
Parametrizando o triângulo usando a variável $x$ acima (pode
também usar um ângulo),
obtemos a área como sendo a função
$A(x)=x(R+\sqrt{R^2-x^2})$, com $x\in [0,R]$. 
Observe que não é necessário considerar os triângulos cuja
base fica acima do eixo $x$. (Por qué?)
Deixamos o leitor verificar que o máximo da função $A(x)$ é
atingido no ponto $x_*=\tfrac{\sqrt{3}}{2}R$, e que esse $x_*$
corresponde ao triângulo equilátero.
\end{sol}
\end{exo}

\begin{exo}
Sejam $x_1,\dots,x_n$ os resultados de medidas repetidas feitas a respeito de
uma grandeza. Procure o número $x$ que minimize 
$$\sigma(x)=\sum_{i=1}^n(x-x_i)^2\,.$$
\begin{sol}
O único ponto crítico de $\sigma(x)$ é $x_*=\frac{x_1+\dots+ x_n}{n}$ (isto é,
a média aritmética). Como $\sigma''(x)=2n>0$, $x_*$ é mínimo global.
\end{sol}
\end{exo}


\begin{exo}\label{Exo:TelaoBIS}
Uma formiga entra no cinema, e vê que 
o telão tem $5$ metros de altura e está afixado na parede, $3$ metros acima do
chão.
A qual distância da parede a formiga deve ficar para que o ângulo sob o qual
ela vê o telão seja máximo? (Vide: Exercício \ref{Exo:Telao}.)
\begin{sol}
Seja $F$ a formiga, $S$ (respectivamente $I$) a extremidade superior
(respectivamente inferior) do telão, $\theta$ o ângulo $SFI$, e $x$ a distância
de $F$ à parede:
\begin{center}
\begin{bmlimage}\begin{tikzpicture}[yscale=0.3]
\draw (-1,0)--(0,0)--(0,10);
\coordinate (F) at (-7,0);
\coordinate (S) at (0,8);
\coordinate (I) at (0,3);
\draw[thick] (S)--(F)--(I);
\draw (F) node{$\bullet$} node[below]{$F$};
\draw (S) node[right]{$S$};
\draw (I) node[right]{$I$};
\draw (0,0) node[right]{$O$};
\draw[decorate, decoration=brace] (I)--(0,0) node[midway, right]{$3$};
\draw[decorate, decoration=brace] (S)--(I) node[midway, right]{$5$};
\draw[decorate, decoration=brace] (0,0)--(F) node[midway, below]{$x$};
\end{tikzpicture}\end{bmlimage}
\end{center}
Se $x$ é a distância de $F$ à parede, precisamos expressar $\theta$ em função
de $x$. Para começar, $\theta=\alpha-\beta$, em que $\alpha$ é o ângulo $SFO$,
e $\beta$ o ângulo $IFO$. Mas $\tan \alpha =\frac{8}{x}$ e $\tan
\beta=\frac{3}{x}$. Logo, precisamos achar o máximo da função 
$$
\theta(x)=\arctan\tfrac{8}{x}-\arctan \tfrac{3}{x}\,,\quad \text{ com }x>0\,.
$$
Observe que $\lim_{x\to 0^+}\theta(x)=0$ (indo infinitamente perto da
parede, a formiga vê o telão sob um ângulo nulo) e $\lim_{x\to
\infty}\theta(x)=0$ (indo infinitamente longe da parede, a
formiga também vê o telão sob um ângulo nulo), é claro que deve existir (pelo
menos) um $0<x_*<\infty$ que maximize $\theta(x)$. Como $\theta$ é derivável,
procuremos os seus pontos críticos: 
$$
\theta'(x)=\frac{1}{1+(\tfrac8x)^2}(\frac{-8}{x^2})
-\frac{1}{1+(\tfrac3x)^2}(\frac{-3}{x^2})=(\cdots)=\frac{120-5x^2}{
(x^2+8^2)(x^2+3^2)}\,.
$$
Logo o único ponto crítico de $\theta$ no intervalo $(0,\infty)$ é
$x_*=\sqrt{24}$. Vemos também que $\theta'(x)>0$ se $x<x_*$ e 
$\theta'(x)<0$ se $x>x_*$, logo $x_*$ é o ponto onde $\theta$ atinge o seu
valor máximo. 
Logo, para ver o telão sob um ângulo máximo, a formiga precisa ficar a uma
distância de $\sqrt{24}\simeq 4.9$ metros da parede.
\end{sol}
\end{exo}

Consideremos alguns exemplos de problemas de otimização em três dimensões:

\begin{ex}
\emph{Qual é, dentre os cilíndros inscritos numa esfera de raio $R$, o de volume
máximo?}
Um cílindro cuja base tem raio $r$, e cuja altura é $h$ tem volume 
$V=\pi r^2h$. Quando o cilíndro é inscrito na esfera de raio $R$ centrada na
origem, $r$ e $h$ dependem um do outro:
\begin{center}
\begin{bmlimage}\begin{tikzpicture}
\def\R{1.5}
\def\alf{1}
\draw (0,0) circle (\R cm);
\coordinate (A) at ({\R*cos(\alf r)},{\R*sin(\alf r)});
\coordinate (W) at ({\R*cos(\alf r)},0);
\coordinate (B) at ({-\R*cos(\alf r)},{-\R*sin(\alf r)});
\coordinate (U) at (0,{-\R*sin(\alf r)});
\coordinate (V) at ({\R*cos(\alf r)},{-\R*sin(\alf r)});
\fill[color=gray!10] (B) rectangle (A);
\draw[thin] (-\R,0)--(\R,0);
\draw[thin] (0,-\R)--(0,\R);
\draw[thick] (B) rectangle (A);
\draw[dotted, ->] (0,0)--(A) node[midway, above, sloped]{$R$};
\draw[decorate, decoration={brace, raise=2pt}] (V)--(U) node[midway,
below]{$r$};
\draw[decorate, decoration={brace, raise=2pt}] (A)--(V) node[midway,
right]{$h$};
\draw (0,0) circle (\R cm);
\draw ({-\R-2},0) node[left]{$r^2+(\tfrac{h}{2})^2=R^2$};
\end{tikzpicture}\end{bmlimage}
\end{center}
Assim, $V$ pode ser escrito como função de uma variável só.
Em função de $r$,
$$V(r)=2\pi r^2\sqrt{R^2-r^2}\,,\quad r\in [0,R]\,,$$
ou em função de $h$:
$$
V(h)=\pi h(R^2-\tfrac{h^2}{4})\,,\quad h\in [0,2R]\,.
$$
Para achar o cílindro de volume máximo, procuremos o máximo global de qualquer
uma dessas funções no seu domínio. Consideremos por exemplo $V(r)$. Como $V$ é
derivável em $(0,R)$, temos 
$$
V'(r)=2\pi\Big\{
2r\sqrt{R^2-r^2}+r^2\frac{-r}{\sqrt{R^2-r^2}}
\Big\}=2\pi r\frac{2R^2-3r^2}{\sqrt{R^2-r^2}}\,.
$$
Portanto, $V'(r)=0$ se e somente se $r=0$ ou $2R^2-3r^2=0$. Logo, o único ponto
crítico
de $V$ em $(0,R)$ é $r_*=\sqrt{2/3}R$ ($\simeq 0.82 R$). Estudando o sinal de
$V'$ obtemos a variação de $V$:
\begin{center}
\begin{bmlimage}\begin{tikzpicture}[scale=0.8]
\tkzTabInit[nocadre, espcl=2,  color, colorV=lightgray!5, colorL=gray!15,
colorC=gray!15]
{$r$ /.6, $V'(r)$ /.6, Variaç. de $V$ /1.2}%
{,$\scriptstyle{\sqrt{2/3}}R$, }%
\tkzTabLine{,+,z,-,}
\tkzTabVar{-/,+/\text{máx.},-/}
%\tkzTabLine{,\searrow,\text{mín.},h,\text{mín.},\nearrow,}
\end{tikzpicture}\end{bmlimage}
\end{center}
Na fronteira do intervalo $[0,R]$, $V(0)=0$ e $V(R)=0$. Logo, $V$ atinge o
seu máximo global em $r_*$. Portanto, o cilíndro com volume máximo que pode
ser inscrito numa esfera de raio $R$ tem base com raio $r_*$, e altura
$h_*=2\sqrt{R^2-r_*^2}=\frac{2}{\sqrt{3}}R$ ($\simeq 1.15 R$).
\end{ex}

\begin{exo}
Qual é, dentre os cilíndros inscritos em um cone de altura $H$ e base circular
de raio $R$, o de volume máximo?
\begin{sol}
Seja $R$ o raio da base do cone, $H$ a sua altura, $r$ o raio da base do
cilíndro e $h$ a sua altura.
Para o cilíndro ser inscrito, $\frac{h}{H}=\frac{R-r}{R}$ (para entender essa
relação, faça um desenho de um corte vertical).
Logo, expressando o volume do cilíndro em função de $r$, $V(r)=\frac{\pi
H}{R}r^2(R-r)$. É fácil ver que essa função possui um máximo local em $[0,R]$
atingido em $r_*=\frac{2}{3}R$. A altura do cilíndro correspondente é
$h_*=\frac{H}{3}$.
(Obs: pode também expressar $V$ em função de $h$: $V(h)=\pi
R^2h(1-\frac{h}{H})^2$.)
\end{sol}
\end{exo}

\begin{exo} (Segunda prova, 27 de maio de 2011)
Considere um cone de base circular, inscrito numa esfera de raio $R$.
Expresse o volume $V$ do cone em função da sua altura $h$. 
Dê o domínio de $V(h)$ e ache os seus pontos de mínimo e máximo globais. 
Dê as dimensões exatas do cone que tem volume máximo.
\begin{sol}
Seja $r$ o raio da base do cone, $h$ a sua altura.
O volume do cone é dado por $V=\tfrac13 \times \pi r^2\times h$. Como $h$ e
$r$ são ligados pela relação $(h-R)^2+r^2=R^2$, podemos expressar $V$ somente
em termos de $h$:
$$V(h)=\tfrac{\pi}{3}h(R^2-(h-R)^2)=\tfrac{\pi}{3}(2Rh^2-h^3)\,,$$
onde $h\in [0,2R]$.
Os valores na fronteira são $V(0)=0$, $V(2R)=0$.
Procurando os pontos críticos dentro do intervalo: $V'(h)=0$ se e somente se
$4Rh-3h^2=0$. Como $h=0$ não está \emph{dentro} do intervalo, somente
consideramos o ponto crítico $h_*=\tfrac{4}{3}R$. (Como $V''(h_*)<0$, é máximo
local.) Comparando $V(h_*)$ com os valores na fronteira, vemos que $h_*$ é
máximo global de $V$ em $[0,2R]$, e que tem dois mínimos globais, em $h=0$ e
$h=2R$. 
{O maior cone, portanto, tem altura $\tfrac{4}{3}R$, e raio
$\sqrt{R^2-(\tfrac{4}{3}R-R)^2}=\frac{\sqrt{8}}{3}R$.}
\end{sol}
\end{exo}

\begin{exo}\label{exo_coneemtornoesfera}
De todos os cones que contêm uma esfera de raio $R$, qual tem o menor
volume?
\end{exo}

\begin{exo}
Uma caixa retangular  é feita retirando quatro quadrados dos cantos de uma
folha de papelão de dimensões $2m\times1m$, e dobrando os quatro
lados:
\begin{center}
\newcommand{\bzzz}{0.3}
\begin{bmlimage}\begin{tikzpicture}[scale=2]
\draw[dotted] (0,0)--(2,0)--(2,1)--(0,1)--cycle;
\draw[very thick] (0,\bzzz)--(\bzzz,\bzzz)--(\bzzz,0)--({2-\bzzz},0)--({2-\bzzz},\bzzz)--(2,\bzzz)--(2,{1-\bzzz})--({2-\bzzz},{1-\bzzz})--({2-\bzzz},1)--(\bzzz,1)--(\bzzz,{1-\bzzz})--(0,{1-\bzzz})--(0,\bzzz);
\draw[thick, ->] (1,0)--(1,{\bzzz/2});
\draw[thick, ->] (1,1)--(1,{1-\bzzz/2});
\draw[thick, ->] (0,0.5)--({\bzzz/2},0.5);
\draw[thick, ->] (2,0.5)--({2-\bzzz/2},0.5);
\end{tikzpicture}\end{bmlimage}
\end{center}
Qual deve ser o tamanho dos quadrados retirados para maximizar o
volume da caixa obtida?
\begin{sol}
Cada quadrado retirado deve ter os seus lados iguas a 
$\tfrac12(1-\frac{1}{\sqrt{3}})$.
\end{sol}
\end{exo}

\section{A Lei de Snell}
\index{Lei de Snell}
Considere uma partícula que evolui na interface entre dois ambientes,
$1$ e $2$ (veja a figura abaixo).
Suponhamos que num ambiente dado, a partícula anda sempre em linha reta e que a
partícula evolui no ambiente $1$ com uma 
velocidade constante $v_1$ e no ambiente $2$ com uma velocidade constante
$v_{2}$.
Suponhamos também que a partícula queira viajar de um ponto $A$ no
ambiente $1$ para um ponto $B$ no ambiente $2$; qual estratégia a
partícula deve adotar
para \emph{minimizar o seu tempo de viagem entre $A$ e $B$}? 
%\begin{wrapfigure}{r}{7cm}
%\vspace{-5mm}
\begin{center}
\begin{bmlimage}\begin{tikzpicture}
\pgfmathsetmacro{\a}{0};
\coordinate (A) at (\a,-3);
\pgfmathsetmacro{\b}{4};
\coordinate (B) at (\b,+2);
\pgfmathsetmacro{\c}{1};
\coordinate (C) at (\c,0);

\fill[color=gray!15] (\a-1,-3.5) rectangle (\b+1,0);
\draw (\a-1,0)--(\b+1,0);
\draw[thick] (A)--(C) node[midway, above left]{$v_1$}--(B)node[midway,
above left]{$v_2$};
\fill (A) circle (0.4mm);
\fill (B) circle (0.4mm);

\draw (\a-0.8,0.2) node{$2$};
\draw (\a-0.8,-0.2) node{$1$};

\draw[thick, ->] (A)--($(A)!0.5!(C)$);
\draw[thick, ->] (C)--($(C)!0.5!(B)$);

\draw (A) node[left]{$A$};
\draw (B) node[right]{$B$};
\draw (C) node[above left]{$C$};
\end{tikzpicture}\end{bmlimage}
\end{center}
%\vspace{-5mm}
%\end{wrapfigure}
É claro que se $v_1=v_2$, a partícula não precisa se preocupar com  a
interface, e pode andar em linha reta de $A$ até $B$. Mas se porventura
$v_1<v_2$, a partícula precisa escolher um ponto $C$ na
interface entre $1$ e $2$, mais perto de $A$ do que de $B$, andar em linha reta
de $A$ até $C$, para depois andar em linha reta de $C$ até $B$.
O problema é de saber como escolher $C$, de maneira tal que o tempo total de
viagem seja mínimo. Modelemos a situação da seguinte maneira:
%\begin{wrapfigure}{r}{7cm}
%\vspace{-5mm}
\begin{center}
\begin{bmlimage}\begin{tikzpicture}
\pgfmathsetmacro{\a}{0};
\coordinate (A) at (\a,-3);
\pgfmathsetmacro{\b}{4};
\coordinate (B) at (\b,+2);
\pgfmathsetmacro{\c}{1};
\coordinate (C) at (\c,0);
%\fill[color=gray!15] (\a-1,-3.5) rectangle (\b+1,0);
\draw[->] (\a,0)--(\b+1,0);
\draw[->] (\a,-3)--(\a,2.3);
\draw[thick] (A) node[left]{$A$}--(C) node[midway, right]{$d_1$}--(B)
node[right]{$B$} node[midway, below]{$d_2$};
\fill (A) circle (0.4mm);
\fill (B) circle (0.4mm);
\draw[decorate, decoration=brace] (C)node[above] {$C$}
--(\a,0) node[midway, below]{$x$} ;
\draw[dotted] (\b,0)--(B)node[midway, right]{$h_2$};
\draw ($(\a,0)!0.5!(A)$) node[left]{$h_1$};
\draw[dotted, <->] (0,2)--(B) node[midway, above]{$L$};
\end{tikzpicture}\end{bmlimage}
\end{center}
%\vspace{-5mm}
%\end{wrapfigure}
A nossa variável será $x$, a distância entre $C$ e a projeção de $A$ na
horizontal.
Quando $x$ é fixo, a distância de $A$ até $C$ é dada por
$d_1=\sqrt{h_1^2+x^2}$, e a distância de $C$ até $B$ é dada por
$d_2=\sqrt{h_2^2+(L-x)^2}$.
Indo de $A$ até $C$, a partícula percorre a distância $d_1$ em um
tempo $t_1=\frac{d_1}{v_1}$, e indo de $C$ até $B$, percorre a distância $d_2$
em um
tempo $t_2=\frac{d_2}{v_2}$. Logo, o tempo total de viagem de $A$ até $B$ é
de $T=t_1+t_2$. Indicando explicitamente a dependência em $x$, 
$$
T(x)=\frac{\sqrt{h_1^2+x^2}}{v_1}+\frac{\sqrt{h_2^2+(L-x)^2}}{v_2}\,.
$$
Assim, o nosso objetivo é \emph{achar o mínimo global da função $T(x)$,
para $x\in [0,L]$.} Comecemos procurando os pontos críticos de $T$ em
$(0,L)$, isto é, os $x_*$ tais que $T'(x_*)=0$, isto é,

\eq{\label{eq:solSNell}\frac{x_*}{v_1\sqrt{h_1^2+x_*^2}}-\frac{L-x_*}{v_2\sqrt{
h_2^2+(L-x_*)^2}}
=0\,.}
Essa equação é do quarto grau em $x_*$. Pode ser mostrado que a sua solução
existe, é única, e dá o mínimo global de $T$ em $[0,L]$. 
Em vez de
buscar o valor exato do $x_*$, daremos uma interpretação geométrica da solução.
De fato, observe que em \eqref{eq:solSNell} aparecem dois quocientes 
que podem ser interpretados,
respectivamente, como os senos dos ângulos entre $AC$ e a vertical, e $BC$ e a
vertical:
$$\frac{x_*}{\sqrt{h_1^2+x_*^2}}\equiv
\sen \theta_1\,,\quad\frac{L-x_*}{\sqrt{h_2^2+(L-x_*)^2 }}\equiv \sen
\theta_2\,.$$
Portanto, vemos que o mínimo de $T$ é atingido uma vez que os ângulos
$\theta_1$ e $\theta_2$ são tais que
\begin{center}
\begin{bmlimage}\begin{tikzpicture}[scale=0.9]
\pgfmathsetmacro{\a}{0};
\coordinate (A) at (\a,-2);
\pgfmathsetmacro{\b}{3};
\coordinate (B) at (\b,+2);
\pgfmathsetmacro{\c}{1};
\coordinate (C) at (\c,0);
\fill[color=gray!15] (\a-0.5,-2) rectangle (\b,0);
\draw (\a-0.5,0)--(\b,0);
\draw[thick] (A)--(C)--(B);
\draw[dotted] (\c,-2)--(\c,2);
\draw[->] (\c,-1) arc (270:245:1);
\draw[ultra thin] (\c,-1) arc (270:255:1) node[below]{$\theta_1$};
\draw[->] (\c,1) arc (90:45:1);
\draw[ultra thin] (\c,1) arc (90:65:1) node[above]{$\theta_2$};
\draw (\b+4,0) node{$\boxed{\displaystyle{\frac{\sen \theta_1}{\sen
\theta_2}=\frac{v_1}{v_2}}}$};
 \draw[thick, ->] (A)--($(A)!0.3!(C)$);
 \draw[thick, ->] (C)--($(C)!0.7!(B)$);
\end{tikzpicture}\end{bmlimage}
\end{center}
Em ótica, quando um raio de luz passa de ambiente $1$ para um ambiente $2$,
observe-se um desvio ao atravessar a interface;
$\theta_1$ é chamado o \grasA{ângulo de incidência}, $\theta_2$ o
\index{ângulo! de refração}
\grasA{ângulo de refração}. O ângulo de refração depende das propriedades dos
ambientes $1$ e $2$ via $v_1$ e $v_2$, e a relação acima é chamada a
\grasA{Lei de Snell~\footnote{Willebrord Snellius van Royen, Leiden, 1580 -
1626.}}.\\

No exemplo acima não obtivemos um valor explícito para o $x_*$ que minimize o
tempo de viagem de $A$ até $B$, mas aprendemos alguma coisa a respeito dos
ângulos $\theta_1$ e $\theta_2$. Em alguns casos particulares, $x_*$ pode
ser calculado explicitamente: 
\begin{exo}
Um ponto $A$ flutuando a $h$ metros da praia precisa atingir um ponto $B$
situado na beirada da água, a $L$ metros do ponto da praia mais perto de $A$.
Supondo que $A$ se move na água com uma velocidade $v_1$ e na areia com uma
velocidade $v_2>v_1$, elabore uma estratégia para que $A$ atinja $B$ o mais
rápido possível. E se $v_1<v_2$?
\begin{sol}
Como no exemplo anterior, $T(x)=\frac{\sqrt{x^2+h^2}}{v_1}+\frac{L-x}{v_2}$. 
Procuremos o mínimo global de $T$ em $[0,L]$.
O ponto crítico $x_*$ é solução de
$\frac{x}{v_1\sqrt{x^2+h^2}}-\frac{1}{v_2}=0$. Isto é,
$x_*=\frac{h}{\sqrt{(v_2/v_1)^2-1}}$. 
Se $v_1\geq v_2$, $T$ não tem ponto critico no intervalo, e  $T$ atinge o seu
mínimo global em $x=L$ (a melhor estratégia é de nadar diretamente até $B$). Se 
$v_1<v_2$, e se $\frac{h}{\sqrt{(v_2/v_1)^2-1}}<L$, então $T$ tem um mínimo
global em $x_*$ (como $T''(x)=\frac{h^2}{v_1(x^2+h^2)}>0$
para todo $x$, $T$ é convexa, logo $x_*\in (0,L)$ é bem um ponto de
mínimo global).
Por outro lado, se $\frac{h}{\sqrt{(v_2/v_1)^2-1}}\geq L$, então $x_*$ não
pertence a $(0,L)$, e o mínimo global de $T$ é atingido em $x=L$.
\end{sol} 
\end{exo}

\begin{exo}
Uma partícula parte de um ponto $A$ para 
atingir o mais rápido possível 
um ponto $B$ situado do outro lado de uma piscina redonda:
\begin{center}
\begin{bmlimage}\begin{tikzpicture}
\fill[color=lightgray] (0,0) circle (1cm);
\fill (-1,0) circle (0.5mm);
\fill (1,0) circle (0.5mm);
\draw (-1,0) node[left]{$A$};
\draw (1,0) node[right]{$B$};
\end{tikzpicture}\end{bmlimage}
\end{center}
Se $A$ nada com uma velocidade de $2km/h$ e anda
com uma velocidade de $4km/h$, será que é melhor 1) dar a volta toda
andando, 2) usar o caminho mais direto, atravessando a piscina
nadando, 3) adotar uma outra estratégia?
\begin{sol}
Seja $O$ o centro da piscina. Uma estratégia que minimize o tempo de
viagem é de nadar em linha
reta de $A$ até um ponto $C$ na beirada tal que o ângulo $COB$ seja igual a
$\frac{\pi}{3}$ (ou $-\frac{\pi}{3}$). Depois, andar na beirada de $C$
até $B$.
\end{sol}
\end{exo}

\begin{exo}\label{exo:escada}
Considere a esquina do corredor em formato de L representado  na figura
abaixo (suponha-se que o corredor é infinitamente extenso nas
direções perpendiculares).
Qual é o tamanho $\ell$ da maior vara rígida  que pode passar por esse
corredor?
\begin{center}
\begin{bmlimage}\begin{tikzpicture}
\pgfmathsetmacro{\L}{1};
\pgfmathsetmacro{\M}{2};
\pgfmathsetmacro{\m}{2};
\pgfmathsetmacro{\l}{4};
\coordinate (A) at (0,{\M+\m});
\coordinate (B) at (\L,{\M+\m});
\coordinate (C) at (\L,{\M});
\coordinate (D) at ({\L+\l},\M);
\coordinate (E) at ({\L+\l},0);
\draw (A)--(0,0)--(E); 
\draw (B)--(C)--(D);
\coordinate (X) at (1.8,1.5);
\coordinate (Y) at (4.2,0.5);
\fill (X) circle (0.3mm);
\fill (Y) circle (0.3mm);
\draw[thick] (X)--(Y) node[midway, above]{$\ell$};
\draw[dotted, <->] (A)--(B) node[midway, above]{$L$};
\draw[dotted, <->] (D)--(E) node[midway, right]{$M$};
%\fill[color=gray!15] (0,0)--(A)--(B)--(C)--(D)--(E)--cycle; 
\end{tikzpicture}\end{bmlimage}
\end{center}
\begin{sol}
%%%%%%%%%
A maior vara corresponde ao menor segmento que passa por $C$ e
encosta nas paredes em dois pontos $P$ e $Q$ (ver imagem
abaixo).
\begin{center}
\begin{bmlimage}\begin{tikzpicture}
\pgfmathsetmacro{\L}{1};
\pgfmathsetmacro{\M}{2};
\pgfmathsetmacro{\m}{2};
\pgfmathsetmacro{\l}{4};
\coordinate (A) at (0,{\M+\m});
\coordinate (B) at (\L,{\M+\m});
\coordinate (C) at (\L,{\M});
\coordinate (D) at ({\L+\l},\M);
\coordinate (E) at ({\L+\l},0);
\draw (A)--(0,0)--(E); 
\draw (B)--(C)--(D);
\pgfmathsetmacro{\t}{30};
\coordinate (P) at (0,{\M+(\L*sin(\t)/cos(\t))});
\draw (P) node[left]{$P$};
\coordinate (Q) at ({\L+(\M*cos(\t)/sin(\t))},0);
\draw (Q) node[below]{$Q$};
\draw[thick] (P)--(C);
\draw[thick] (Q)--(C);
\draw (C) node[below left]{$C$};
\draw (D) node[right]{$D$};
%\coordinate (X) at (1.8,1.5);
%\coordinate (Y) at (4.2,0.5);
%\fill (X) circle (0.3mm);
%\fill (Y) circle (0.3mm);
%\draw[thick] (X)--(Y) node[midway, above]{$\ell$};
%\draw[dotted, <->] (A)--(B) node[midway, above]{$L$};
%\draw[dotted, <->] (D)--(E) node[midway, right]{$M$};
%\fill[color=gray!15] (0,0)--(A)--(B)--(C)--(D)--(E)--cycle; 
\fill (P) circle (0.4mm);
\fill (Q) circle (0.4mm);
\fill (C) circle (0.4mm);
\end{tikzpicture}\end{bmlimage}
\end{center}
Seja $\theta$ o ângulo $QCD$. Quando $\theta$ é fixo, a
distância de $P$ a $Q$ vale 
$$
f(\theta)=\frac{L}{\cos \theta}+\frac{M}{\sen \theta}\,.
$$
Precisamos minimizar $f$ no intervalo $(0,\pisobredois)$.
(Observe que $\lim_{\theta\to 0^+}f(\theta)=+\infty$,
$\lim_{\theta\to {\pisobredois}^-}f(\theta)=+\infty$.)
Resolvendo $f'(\theta)=0$, vemos que o único ponto crítico
$\theta_*$ satisfaz $\tan^3\theta_*=M/L$. É fácil verificar
que $f$ é convexa, logo $\theta_*$ é um ponto de mínimo global
de $f$.
Assim, o tamanho da maior vara possível é igual a
$$
f(\theta_*)=\cdots=L\bigl(1+(M/L)^{2/3}\bigr)^{3/2}\,.
$$
Observe que quando $L=M$, a maior vara tem tamanho
$2\sqrt{2}L$, e quando $M\to 0^+$, a maior vara tende a ter
tamanho igual a $L$.
\end{sol}
\end{exo}



% !TeX spellcheck = pt_BR
% !TEX encoding = UTF-8 Unicode

\chapter{Estudos de funções}\label{Cap:Estudos}

\ifdefined\updateans
% Only need to run once in a lifetime, when the file ans.tex needs to be updated.
\Writetofile{ans}{\protect\section*{Capítulo \ref{Cap:Estudos}}}
\fi

Neste capítulo juntaremos as técnicas desenvolvidas anteriormente
para estudar \emph{funções}.
Já estudamos algumas funções em bastante detalhes no último capítulo, 
ao resolver problemas de otimização.\\

Antes de estudar casos particulares, faremos mais dois comentários
sobre o comportamento
de uma função quando $x\to \pm \infty$.


\section{Sobre o crescimento das funções no $\infty$}
\index{crescimento no $\infty$}

É importante se lembrar, ao estudar funções, de quais são os
comportamentos das funções fundamentais  
(polinômios, exponenciais e logaritmos) que tendem ao infinito
quando $x\to \infty$.\\
%Nesta pequena seção usaremos 
%a regra de Bernoulli-l'Hôpital para estabelecer uma
%\emph{hierarquia} entre essas funções no infinito.\\

Para começar, 
já vimos na Seção~\ref{sec_Lim_parenteseAldo}
(ou no item~\eqref{itexBH12ab} do Exercício~\ref{Exo:BHbasic})
que 
$$\lim_{x\to \infty}\frac{x}{e^{x}}=0\,.$$
Pode também ser mostrado que para qualquer $p>0$,
\eq{\label{eq:expcrescemaisrapidoquepolin}
\lim_{x\to \infty}\frac{x^p}{e^x}=0\,.}
Podemos resumir esse fato da seguinte maneira: seja
$P(x)$ um polinômio cujo coeficiente de grau maior é positivo.
Então $P(x)\nearrow \infty$ e $e^x\nearrow \infty$, mas 
$$\boxed{P(x)\ll e^x\,,\quad\text{ quando
}x\to \infty\,.}$$
O símbolo ``$\ll$'' é usado para significar: ``é muito menor que''.
Em palavras: \emph{no infinito, o crescimento exponencial é
muito mais rápido que qualquer crescimento polinomial}.\\

Vimos também que
$$
\lim_{x\to \infty}\frac{\ln x}{x}=0\,,\quad\quad 
\lim_{x\to \infty}\frac{(\ln x)^2}{x}=0\,,
$$
e pode ser mostrado (veja exercício abaixo) que para qualquer
$p>0$ e qualquer $q>0$,
\eq{\label{eq:polincrescemaisrapidoquelog}\lim_{x\to \infty}\frac{(\ln
x)^p}{x^q}=0\,.}
Como $x^q$ pode também ser trocado
por qualquer polinômio $P(x)$ (supondo que o coeficiente do
seu termo de grau maior é positivo), esse fato costuma ser
resumido da seguinte
maneira:
$$\boxed{(\ln x)^p\ll P(x)\,,\quad\text{ quando }x\to \infty\,.}$$
Isto é: $(\ln x)^p\nearrow \infty$, e 
$P(x)\nearrow \infty$ quando $x\to \infty$,
mas \emph{o crescimento polinomial é muito mais rápido que
qualquer crescimento logaritmico.}

\begin{exo}
Mostre que para qualquer $p>0$, e $q>0$, $\lim_{x\to \infty}\frac{(\ln
x)^p}{x^q}=0$.
\begin{sol}
(Já vimos no Exemplo \ref{Ex:logsurx} que a afirmação vale para $p=1$, $q=1$.)
Observe que 
$\frac{(\ln
x)^p}{x^q}=(\frac{(\ln
x)^{p/q}}{x})^q$. Logo, basta provar a afirmação para $q=1$ e $p>0$ qualquer:
$\lim_{x\to \infty}\frac{(\ln
x)^p}{x}=0$.
Mostremos por indução que se a afirmação vale para $p>0$ 
($\lim_{x\to \infty}\frac{(\ln
x)^{p}}{x}=0$), então ela vale para $p+1$. De fato, pela regra de B.H., 
$$
\lim_{x\to \infty}\frac{(\ln x)^{p+1}}{x}=\lim_{x\to \infty}\frac{(p+1)(\ln
x)^{p}\tfrac{1}{x}}{1}=
(p+1)\lim_{x\to \infty}\frac{(\ln
x)^{p}}{x}=0\,.
$$
Então, a afirmação estará provada para qualquer $p>0$ se ela for provada para
$0<p\leq 1$. Mas para tais $p$, $(\ln x)^p\leq \ln x$ para todo $x>1$, logo, 
$$
\lim_{x\to \infty}\frac{(\ln x)^p}{x}\leq \lim_{x\to \infty}\frac{\ln x}{x}=0\,,
$$ 
pelo Exemplo \ref{Ex:logsurx}.
\end{sol}
\end{exo}

Assim, \emph{quando $x\to \infty$}, a hierarquia entre logaritmo, polinômio e
exponencial é
\eq{\boxed{(\ln x)^p\ll P(x)\ll e^x\,.}}
\begin{exo}
Mostre que para qualquer $p>0$, $\lim_{x\to \infty}\frac{x^p}{e^{x}}=0$.
\end{exo}

\begin{exo}
Estude os seguintes limites
\begin{multicols}{2}
\begin{enumerate}
\item\label{itexBHlast1} $\lim_{x\to \infty}\frac{x^{1000}+e^{-x}}{x^{100}+e^x}$
\item\label{itexBHlast2} $\lim_{x\to \infty}\frac{e^{(\ln x)^2}}{2^{x}}$
\item\label{itexBHlast3} $\lim_{x\to \infty}(x^3-(\ln x)^5-\frac{e^x}{x^7})$
\item\label{itexBHlast4} $\lim_{x\to \infty}x^{\ln x}e^{-x/2}$
\item\label{itexBHlast5_a} $\lim_{x\to \infty}\frac{x}{e^{(\ln x)^2}}$
\item\label{itexBHlast5_b} $\lim_{x\to \infty}\frac{\sqrt{x}}{e^{\sqrt{\ln x}}}$
\item\label{itexBHlast6} $\lim_{x\to
\infty}\frac{\ln(\ln(\ln(x)))}{\ln(\ln(x))}$
\item\label{itexBHlast24}
$\lim_{x\to\infty}\{e^{\sqrt{(\ln x)^2+1}}-x\}$
\end{enumerate}
\end{multicols}
\vspace{0.01cm}
\begin{sol}
\eqref{itexBHlast1} $0$
\eqref{itexBHlast2} $0$
\eqref{itexBHlast3} $-\infty$
\eqref{itexBHlast4} $0$
\eqref{itexBHlast5_a} $0$ 
\eqref{itexBHlast5_b} $\infty$ 
\eqref{itexBHlast6} $0$
\eqref{itexBHlast24} $\infty$
\end{sol}
\end{exo}


\section{Assíntotas oblíquas}\label{Sec:Obliquas}
\index{assíntota! oblíqua}
A noção de \emph{assíntota} permitiu obter informações a respeito do comportamento
qualitativo de uma função \emph{longe da origem}, em direções paralelas aos eixos de
coordenadas: ou horizontal, ou vertical.\\

Veremos nesta seção que existem funções cujo gráfico, longe da origem, 
se aproxima de uma reta que não é nem vertical, nem horizontal, mas \emph{oblíqua},
isto é de inclinação finita  e não nula. Comecemos com um exemplo.

\begin{ex}\label{ex:assolbbb}
Considere a função $f(x)=\frac{x^3+1}{2 x^2}$. 
É claro que esta função possui a reta $x=0$ como assíntota vertical, já que 
$$
\lim_{x\to 0^+}\frac{x^3+1}{2x^2}=+\infty\,,\quad \lim_{x\to
0^-}\frac{x^3+1}{2x^2}=+\infty\,.
$$
Por outro lado, $f$ não possui assíntotas horizontais, já que 
$$
\lim_{x\to +\infty}\frac{x^3+1}{2x^2}=+\infty\,,\quad \lim_{x\to
-\infty}\frac{x^3+1}{2x^2}=-\infty\,.
$$
\begin{center}
\begin{bmlimage}\begin{tikzpicture}[scale=0.8]
\newcommand{\funcao}[1]{ (((#1)^3+1)/(2*(#1)^2)) }
\pgfmathsetmacro{\a}{4};
\pgfmathsetmacro{\b}{3};
\draw[ ->] ({-\a+1},0)--({\a-1},0) node[right]{$x$};
\draw[ ->] (0,{-\b+1})--(0,\b);
\draw[thick, domain=-\a:-0.4, samples=50] plot (\x,{\funcao{\x}});
\draw[thick, domain=0.4:\a, samples=50] plot (\x,{\funcao{\x}});
\end{tikzpicture}\end{bmlimage}
\end{center}
Apesar de não possuir assíntota horizontal, vemos que longe da origem, 
o gráfico parece se aproximar de uma reta de inclinação
positiva. Como determinar essa reta?\\

Para começar, demos uma idéia do que está acontecendo. 
Observe primeiro que
$\frac{x^3+1}{2x^2}=
\frac{x}{2}+\frac{1}{2x^2}$. Logo, quando $x$ for grande, a
contribuição 
do termo $\frac{1}{2x^2}$ é desprezível em relação a
$\frac{x}{2}$, e $f(x)$ é aproximada por
$$
f(x)\simeq \frac{x}{2}\,.$$
Ora, a função $x\mapsto \frac{x}{2}$ é uma reta de inclinação $\frac{1}{2}$. De fato,
esboçando o gráfico de $f$ junto com a reta $y=\frac{x}{2}$:

\begin{center}
\begin{bmlimage}\begin{tikzpicture}[scale=0.8]
\newcommand{\funcao}[1]{ (((#1)^3+1)/(2*(#1)^2)) }
\pgfmathsetmacro{\a}{4};
\pgfmathsetmacro{\b}{3};
\draw[ ->] ({-\a+1},0)--({\a-1},0) node[right]{$x$};
\draw[ ->] (0,{-\b+1})--(0,\b);
\draw[thick, domain=-\a:-0.4, samples=50] plot (\x,{\funcao{\x}});
\draw[thick, domain=0.4:\a, samples=50] plot (\x,{\funcao{\x}});
\draw[dashed, domain=-\a:\a] plot (\x,{\x/2}) node[below right]{$y=\frac{x}{2}$};
\end{tikzpicture}\end{bmlimage}
\end{center}
Podemos agora verificar que de fato, 
quando $x\to \infty$, \emph{a distância entre o gráfico
de $f$ e a reta $y=\frac{x}{2}$ tende a zero}:
\begin{equation}\label{eq:biduleolbik}
\lim_{x\to \infty}\bigl|f(x)-\tfrac{x}{2}\bigr|=
\lim_{x\to \infty}\bigl|(\tfrac{x}{2}+\tfrac{1}{2x^2})-\tfrac{x}{2}\bigr|=
\lim_{x\to \infty}\tfrac{1}{2x^2}=0\,.
\end{equation}
Portanto, a reta $y=\frac{x}{2}$ é chamada de \emph{assíntota oblíqua} da função $f$.
\end{ex}

O exemplo anterior leva naturalmente à seguinte definição:

\begin{defin}
A reta de equação $y=mx+h$ é chamada de \grasA{assíntota oblíqua para $f$} 
se pelo menos um
dos limites abaixo existe e é nulo:
$$\lim_{x\to +\infty}\bigl|f(x)-(mx+h)\bigr|\,,\quad\lim_{x\to
-\infty}\bigl|f(x)-(mx+h)\bigr|\,.$$
(Obs: quando $m=0$, essa definição coincide com a de assíntota
horizontal.)
\end{defin}

Como saber se uma função possui uma assíntota oblíqua? E se ela tiver uma, como
identificar os coeficientes $m$ e $h$?\\

Para começar, observe que $h$ pode ser obtido a partir de $m$, já que 
$$
\lim_{x\to \pm\infty}\bigl \{f(x)-(mx+h)\bigr\}=
\lim_{x\to \pm\infty}\bigl \{(f(x)-mx)-h\bigr\}
$$
é zero se e somente se 
\begin{equation}\label{eq:hparaoblik}
h=\lim_{x\to \pm\infty}\{f(x)-mx\}\,.
\end{equation}

Para identificar $m$, podemos escrever
$$\lim_{x\to \pm\infty}\bigl \{f(x)-(mx+h)\bigr\}
=\lim_{x\to \pm\infty}x\cdot \bigl\{\tfrac{f(x)}{x}-(m+\tfrac{h}{x})\bigr\}\,,
$$
e observar que para este último limite 
existir e ser igual a zero quando $x\to \pm \infty$, é necessário 
que 
$\lim_{x\to \pm \infty} \bigl\{\tfrac{f(x)}{x}-(m+\tfrac{h}{x})\bigr\}=0$. 
Como $\frac{h}{x}\to 0$, isso implica que
\begin{equation}\label{eq:mparaoblik}
m=\lim_{x\to \infty}\frac{f(x)}{x}\,.
\end{equation}

Assim, vemos que se $f$ possui uma 
assíntota oblíqua, esta tem uma inclinação
dada por \eqref{eq:mparaoblik}, e uma abcissa na origem dada por
\eqref{eq:hparaoblik}.
Por outro lado, é claro que se os dois limites em
\eqref{eq:mparaoblik} e 
\eqref{eq:hparaoblik} existirem e forem \emph{ambos finitos}, então $f$
possui uma assíntota oblíqua dada por $y=mx+h$. 
É claro que os limites $x\to +\infty$ precisam ser calculados
\emph{separadamente}, pois uma função pode possuir assíntotas
oblíquas diferentes em $+\infty$ e $-\infty$.\\

Voltando para o Exemplo \ref{ex:assolbbb}, temos
$$
m=\lim_{x\to \pm \infty} \frac{f(x)}{x}=\lim_{x\to\pm\infty}
\frac{\frac{x^3+1}{2x^2}}{x}=
\lim_{x\to\pm\infty}\frac{x^3+1}{2x^3}
=\lim_{x\to\pm\infty}\bigl\{
\tfrac12+\tfrac{1}{2x^3}
\bigr\}=\tfrac12\,,
$$
e, como já visto anteriormente,
$$h=\lim_{x\to\pm \infty}
\{f(x)-\tfrac{1}{2}x\}=\lim_{x\to\pm \infty}
\tfrac{1}{2x^3}=0\,.
$$
Logo, $y=\tfrac12 x+0$ é assíntota oblíqua.
Vejamos como usar   o critério acima em outros exemplos.


\begin{ex}
Considere $f(x)=\sqrt{x^2+2x}$.
Primeiro, tentaremos procurar uma inclinação. Pela presença
da raiz quadrada, cuidamos de distinguir os
limites $x\to-\infty$ e $x\to-\infty$:
$$
\lim_{x\to +\infty}\frac{f(x)}{x}=\lim_{x\to+\infty}
\frac{\sqrt{x^2+2x}}{x}=\lim_{x\to +\infty}
\frac{x\sqrt{1+\tfrac{2}{x}}}{x}=+1
$$
Em seguida calculemos 
\begin{align*}
\lim_{x\to\infty}\{f(x)-(+1)x\}=\lim_{x\to+\infty}\{\sqrt{x^2+2x}-x\}
&=\lim_{x\to+\infty}\frac{2x}{\sqrt{x^2+2x}+x}\\
&=\lim_{x\to+\infty}\frac{2}{\sqrt{1+\frac{2}{x}}+1}=1\,.
\end{align*}
Assim, $f$ possui a assíntota oblíqua $y=x+1$ em
$+\infty$. Refazendo contas parecidas para $x\to-\infty$,
obtemos
$$
\lim_{x\to-\infty}\frac{f(x)}{x}=-1\,,\text{ e }\quad
\lim_{x\to-\infty}\{f(x)-(-1)x\}=-1\,,
$$
logo $f$ possui a assíntota oblíqua $y=-x-1$ em
$-\infty$. De fato (observe que $f$ tem domínio
$D=(-\infty,-2]\cup[0,+\infty)$),
\begin{center}
\begin{bmlimage}\begin{tikzpicture}[scale=1]
\newcommand{\funcao}[1]{ (sqrt((#1)^2+2*(#1))) }
\pgfmathsetmacro{\a}{3};
\pgfmathsetmacro{\b}{4};
\draw[ ->] ({-\a-1},0)--({\a+1},0) node[right]{$x$};
\draw[ ->] (0,{-0.2})--(0,\b) node[above]{$\sqrt{x^2+2x}$};
\draw[thick, domain={-\a-2}:-2, samples=30] plot (\x,{\funcao{\x}});
\draw[thick, domain=0:\a,  samples=30] plot (\x,{\funcao{\x}});
\draw[dashed] (-1,0)--({-\a-2},{\a+1})
node[right]{{$y=-x-1$}};
\draw[dashed] (-1,0)--({\a},{\a+1})
node[left]{{$y=x+1$}};
\end{tikzpicture}\end{bmlimage}
\end{center}
\end{ex}

\begin{ex}
Considere $f(x)=x+\sqrt{x}$, definida somente se $x>0$.
Então 
$$\lim_{x\to\infty}\frac{f(x)}{x}=
\lim_{x\to\infty}\bigl\{1+\frac{\sqrt{x}}{x}\bigr\}=1\,.
$$
Mas, como
$$
\lim_{x\to\infty}\{f(x)-x\}=\lim_{x\to\infty}\sqrt{x}=\infty\,,
$$
vemos que $f$ \emph{não} possui assíntota oblíqua (apesar de
$\lim_{x\to\infty}\frac{f(x)}{x}$ existir e ser finita).
\end{ex}

\begin{exo}
Determine quais das funções abaixo possuem assíntotas
 (se tiver, calcule-as).
\begin{multicols}{3}
\begin{enumerate}
\item\label{itasobl1} $4x-5$
\item\label{itasobl11} $x^2$
\item\label{itasobl6} $\frac{x^2-1}{x+2}$
\item\label{itasobl12} $\ln( x^6+1)$
\item\label{itasobl2} $\ln(1+e^x)$
\item\label{itasobl3} $\sqrt{x^2-\ln x}$
\item\label{itasobl4} $\ln(\cosh x)$
\item\label{itasobl5} $e^{\sqrt{(\ln x)^2+1}}$
\end{enumerate}
\end{multicols}
\vspace{0.01cm}
\begin{sol}
\eqref{itasobl1} A função é a sua própria assíntota oblíqua.
\eqref{itasobl11} Não possui ass.
\eqref{itasobl6} $y=-2$ (vertical), $y=x-2$ em $\pm\infty$. 
\eqref{itasobl12} Não possui ass.
\eqref{itasobl2} $y=0$ em $-\infty$, $y=x$ em $+\infty$.
\eqref{itasobl3} $y=x$ em $+\infty$.
\eqref{itasobl4} $y=x-\ln 2$ em $+\infty$, $y=-x-\ln 2$ em
$-\infty$.
\eqref{itasobl5} Não possui assíntotas: apesar de
$m=\lim_{x\to \infty}\frac{e^{\sqrt{\ln^2x+1}}}{x}$
existir e valer $1$,
$\lim_{x\to\infty}\{e^{\sqrt{\ln^2x+1}}-x\}=\infty$. 
\end{sol}
\end{exo}

\begin{exo}
Se uma função possui uma assíntota oblíqua $y=mx+h$ em
$+\infty$, é
verdade que $\lim_{x\to\infty}f'(x)=m$?
\begin{sol}
Em geral, náo. 
Por exemplo, $f(x)=x+\tfrac{1}{x}\sen (x^2)$ possui $y=x$ como assíntota
oblíqua em $+\infty$, 
mas $f'(x)=1-\frac{\sen x^2}{x^2}+2\cos (x^2)$ 
não possui limite quando $x\to\infty$.
Na verdade, uma função pode possuir uma assíntota (oblíqua ou
outra)
sem sequer ser derivável.
\end{sol}
\end{exo}

\section{Estudos de funções}\label{Sec:Estudos}
\index{estudos de funções}
Podemos agora juntar as técnicas conhecidas para estabelecer um roteiro
para o estudo completo de uma função $f$:
\begin{itemize}
 \item Para começar, encontrar o \emph{domínio} de $f$. O domínio precisa ser
especificado para evitar divisões por zero e raizes (ou logaritmos) de números
negativos. A função poderá depois ser estudada na vizinança de alguns dos
pontos que não pertencem ao domínio, caso
sejam associados a assíntotas verticais.
\item Se for possível (e não sempre é), 
estudar os \emph{zeros} e o \emph{sinal} de $f$.
\item Determinar se $f$ possui algumas \emph{simetrias}, via o estudo da
\emph{paridade}: $f$ é par se $f(-x)=f(x)$, ímpar se $f(-x)=-f(x)$. 
\item Estudar o comportamento assíntotico de $f$, isto é, $f(x)$ quando $x\to
\pm \infty$ (se o domínio
o permite). Se um dos limites $\lim_{x\to \pm\infty}f(x)$ existir (esses limites
podem precisar da regra de Bernoulli-l'Hôpital), então a
função possui uma \emph{assíntota horizontal}. 
Lembre que pode ter assíntotas
horizontais diferentes em $+\infty$ e $-\infty$.
Se um dos limites $\lim_{x\to\infty}f(x)$ for infinito, poderá
procurar saber se existem \emph{assíntotas oblíquas}, como
descrito na Seção \ref{Sec:Obliquas}.
\item Procurar pontos na vizinhança dos quais $f(x)$ toma valores
arbitrariamente grandes, isto é:
\emph{assíntotas verticais}. Calculando os
limites laterais $\lim_{x\to a^+}f(x)$ e $\lim_{x\to a^-}f(x)$ nos pontos $a$
perto dos quais $f$ não é limitada. Isto acontece em geral perto de uma divizão
por zero, ou quando a variável de um logaritmo tende a zero.
\item Estudar a primeira derivada de $f$ (se existir). Em particular,
procurar os \emph{pontos críticos de $f$}. Deduzir a
\emph{variação} de $f$ via o estudo do sinal de $f'$. Determinar os pontos de
mínimo e máximo,  locais ou globais.
\item Estudar $f''$ e a convexidade/concavidade de $f$, via o sinal de $f''$.
O sinal de $f''$ nos pontos críticos (se tiver) permite
determinar quais são mínimos/máximos locais.
Os \emph{pontos de inflexão} são aqueles onde $f$ passa de convexa para
côncava, ou o contrário.
\item Juntando essas informações, montar o \emph{gráfico} de $f$.
Por exemplo, se $f$ é par, o gráfico é simétrico com respeito ao eixo $y$.
Para montar um gráfico completo, 
pode ser necessário calcular mais alguns limites, por exemplo para observar o
comportamento da derivada perto de alguns pontos particulares.
\end{itemize}

\begin{ex} Comecemos com $f(x)=\frac{x+1}{1-x}$, cujo domínio é $D=\bR\setminus
\{1\}$. A função se anula no ponto $x=-1$, e o seu sinal é dado por:
\begin{center}
\begin{bmlimage}\begin{tikzpicture}
\tkzTabInit[lgt=3, nocadre, espcl=2, colorC=red, colorV=gray]
{Valores de $x$: /.6,  $x+1$ /.6, $1-x$ /.6, $f(x)$ /.8}%
{,$-1$, $1$,}
\tkzTabLine{,-,z,+, ,+,}
\tkzTabLine{,+, ,+,z,-,}
\tkzTabLine{,-,z,+,d,-,}
\end{tikzpicture}\end{bmlimage}
\end{center}
(A dupla barra em $x=1$ é para indicar que $f$ não é definida em $x=1$.)
A funçao não é nem par, nem ímpar.
Como
$$
\lim_{x\to \pm\infty}\frac{x+1}{1-x}=\lim_{x\to
\pm\infty}\frac{1+\frac{1}{x}}{\frac{1}{x}-1}=
\frac{1}{-1}=-1\,,
$$
$f$ possui a reta $y=-1$ como assíntota horizontal.
Por outro lado, como
$$
\lim_{x\to 1^+}\frac{x+1}{1-x}=-\infty\,,\quad 
\lim_{x\to 1^-}\frac{x+1}{1-x}=+\infty\,,\quad 
$$
$f$ possui a reta $x=1$ como assíntota vertical.
A derivada existe em todo $x\neq 1$, e vale
$$
f'(x)=\frac{(x+1)'(1-x)-(x+1)(1-x)'}{(1-x)^2}=
\frac{1-x+(x+1)}{(1-x)^2}=
\frac{2}{(1-x)^2}\,.
$$
O sinal de $f'$ dá logo a tabela de variação de $f$:
\begin{center}
\begin{bmlimage}\begin{tikzpicture}[scale=0.8]
\tkzTabInit[nocadre, espcl=2,  color, colorV=lightgray!5, colorL=gray!15,
colorC=gray!15]
{$x$ /.6, $f'(x)$ /.6, Variaç. de $f$ /1.2}%
{,$1$, }%
\tkzTabLine{,+,d,+,}
\tkzTabVar{-/,+D-/$\scriptscriptstyle{+\infty}$/$\scriptscriptstyle{-\infty}$,+/
}
%\tkzTabLine{,\searrow,\text{mín.},h,\text{mín.},\nearrow,}
\end{tikzpicture}\end{bmlimage}
\end{center}
(Indicamos o fato de $x=1$ ser uma assíntota vertical.)
Assim, $f$ não possui pontos críticos, e é crescente nos intervalos
 $(-\infty,1)$ e $(1,\infty)$.
A segunda derivada se calcula facilmente (para $x\neq 0$):
$$f''(x)=2((1-x)^{-2})'=2(-2)(1-x)^{-3}(-1)=\frac{4}{(1-x)^3}\,.$$
Esta muda de sinal em $x=1$, e permite descrever a convexidade de $f$:
\begin{center}
\begin{bmlimage}\begin{tikzpicture}[scale=0.8]
\tkzTabInit[nocadre, espcl=2,  color, colorV=lightgray!5, colorL=gray!15,
colorC=gray!15]
{$x$ /.6, $f''(x)$ /.6, Conv. de $f$ /1.2}%
{,$1$, }%
\tkzTabLine{,+,d,-,}
\tkzTabLine{,\smile,d,\frown,}
\end{tikzpicture}\end{bmlimage}
\end{center}
Isto é, $f$ é convexa em $(-\infty,1)$, côncava em $(1,\infty)$. Assim, o
gráfico é da forma
\begin{center}
\begin{bmlimage}\begin{tikzpicture}[yscale=0.6]
\draw[ ->] (-4,0)--(5,0);
\draw[ ->] (0,-4)--(0,+2.5);
\draw[dashed] (-4,-1)node[below right]{$y=-1$}--(5,-1);
\draw[dashed] (1,-4)--(1,3)node[right]{$x=1$};
\draw[thick, domain=-4:0.5, samples=50] plot (\x,{(\x+1)/(1-\x)});
\draw[thick, domain=1.5:5, samples=50] plot (\x,{(\x+1)/(1-\x)});
\draw (-1,0) node{$\shortmid$} node[above]{$-1$};
\draw (1,0) node{$\shortmid$} node[above right]{$1$};
\end{tikzpicture}\end{bmlimage}
\end{center}
\end{ex}

\begin{ex}
Estudemos agora a função $f(x)=\frac{x^2-1}{x^2+1}$.
%\eqref{itexoEstudA2}
O seu domínio é $D=\bR$, e o seu sinal: $f(x)$ é $\geq 0$ se $|x|\geq 1$,
$<0$ caso contrário.
Como $f(-x)=\frac{(-x)^2-1}{(-x)^2+1}=\frac{x^2-1}{x^2+1}=f(x)$, $f$ é par.
Como 
$$
\lim_{x\to \pm \infty}\frac{x^2-1}{x^2+1}=\lim_{x\to \pm
\infty}\frac{1-\tfrac{1}{x^2}}{1+\tfrac{1}{x^2}}=1\,,
$$
a reta $y=1$ é assíntota horizontal. Não tem assíntotas verticais (o
denominador não se anula em nenhum ponto).
A primeira derivada é dada por $f'(x)=\frac{4x}{(x^2+1)^2}$. Logo,
\begin{center}
\begin{bmlimage}\begin{tikzpicture}
\tkzTabInit[nocadre,espcl=2,  color, colorV=lightgray!5, colorL=gray!15,
colorC=gray!15]
%{$x$ /.6,  $f'(x)$ /.9, Variação de $f$ /1.5}%
{$x$ /.5,  $f'(x)$ /.5, Var. de $f$ /1}{,$0$,}
%\tkzTabLine{,+,z,+,,+,}
\tkzTabLine{,-,z,+,}
\tkzTabVar{+/,-/\text{min.},+/,}
%\tkzTabLine{,\searrow,\text{mín.},h,\text{mín.},\nearrow,}
\end{tikzpicture}\end{bmlimage}
\end{center}
O mínimo local (que é global também) tem coordenada $(0,f(0))=(0,-1)$. A
segunda derivada é dada por
$f''(x)=\frac{4(1-3x^2)}{(x^2+1)^3}$, logo:
\begin{center}
\begin{bmlimage}\begin{tikzpicture}
\tkzTabInit[nocadre,espcl=2,  color, colorV=lightgray!5, colorL=gray!15,
colorC=gray!15]
%{$x$ /.5,  $f''(x)$ /.7, Conc. de $f$ /1.3}
{$x$ /.5,  $f''(x)$ /.5, Conc. de $f$ /1}
{,$-1/\sqrt{3}$, $-1/\sqrt{3}$,}
\tkzTabLine{,-,z,+,z,-,}
\tkzTabLine{,{\frown},,\smile,,\frown,}
\end{tikzpicture}\end{bmlimage}
\end{center}
Os pontos de inflexão estão em
$(\tfrac{-1}{\sqrt{3}},f(\tfrac{-1}{\sqrt{3}}))=(\tfrac{-1}{\sqrt{3}},-\tfrac{1}
{2})$,
e
$(\tfrac{+1}{\sqrt{3}},f(\tfrac{+1}{\sqrt{3}}))=(\tfrac{+1}{\sqrt{3}},-\tfrac{1
}{2})$. Finalmente,
\begin{center}
\begin{bmlimage}\begin{tikzpicture}[scale=1.3]
\draw [thick, domain=-4:4, samples=100] plot (\x,{((\x)^2-1)/((\x)^2+1)});
\draw [ ->] (-4,0)--(4,0) node[right] {$x$};
\draw [ ->] (0,-0.1)--(0,1.5) node[left] {$f(x)$};
\draw [dotted] (-4,1)--(4,1) node[above left] {$y=1$};
%\fill (1,0) circle (0.35mm);
%\fill (-1,0) circle (0.35mm);
\fill (0,-1) circle (0.35mm);
\draw (0,-1) node[below] {$(0,-1)$};
\fill (-0.577,-0.5) circle (0.35mm);
\draw (-0.577,-0.6) node[left]{$(\tfrac{-1}{\sqrt{3}},-\half)$};
\fill (+0.577,-0.5) circle (0.35mm);
\draw (+0.577,-0.6) node[right]{$(\tfrac{+1}{\sqrt{3}},-\half)$};
\end{tikzpicture}\end{bmlimage}
\end{center}
\end{ex}

\begin{exo}\label{Exo:DoisEstudosLegais} Faça um estudo completo das seguintes
funções. 
% \begin{multicols}{1}
\begin{enumerate}
\item\label{itexoEstudA1} $\bigl(\frac{x-1}{x}\bigr)^2$ (Segunda prova,
primeiro semestre 2011)
%\item \label{itexoEstudA2} $\frac{x^2-1}{x^2+1}$
\item \label{itexoEstudA3} $x(\ln x)^2$ (Segunda prova, primeiro semestre 2010)
\end{enumerate}
% \end{multicols}
\vspace{0.01cm}
\begin{sol}
%%%%%%%%%%%%%%%%%%%%%%%%%%%%%%%%5
\eqref{itexoEstudA1}:
O domínio de $\bigl(\frac{x-1}{x}\bigr)^2$ é $D=\bR\setminus \{0\}$, o sinal é
sempre não-negativo, tem um zero
em $x=1$. $f$ não é par, nem ímpar.
Os limites relevantes são $\lim_{x\to 0^{\pm}}f(x)=+\infty$, logo $x=0$ é
assíntota vertical, e
$$\lim_{x\to \pm\infty}\bigl(\frac{x-1}{x}\bigr)^2=\Bigl(\lim_{x\to \pm
\infty}\frac{x-1}{x}\Bigr)^2==\Bigl(\lim_{x\to \pm
\infty}\bigl(1-\frac{1}{x}\bigr)\Bigr)^2=1^2=1\,.$$
Logo, $y=1$ é assíntota horizontal. 
$f$ é derivável em $D$, e $f'(x)=\frac{2(x-1)}{x^3}$.
\begin{center}
\begin{bmlimage}\begin{tikzpicture}
\tkzTabInit[nocadre,espcl=2,  color, colorV=lightgray!5, colorL=gray!15,
colorC=gray!15]
{$x$ /.6,  $f'(x)$ /.6, Var. de $f$ /1.3}%
{,$0$, $1$,}%
%\tkzTabLine{,+,z,+,,+,}
\tkzTabLine{,+,d,-,z,+,}
\tkzTabVar{-/,+D+/$+\infty$/$+\infty$,-/mín,+/,}
%\tkzTabLine{,\searrow,\text{mín.},h,\text{mín.},\nearrow,}
\end{tikzpicture}\end{bmlimage}
\end{center}
$f$ possui um mínimo global em $(1,0)$.
A segunda derivada é dada por $f''(x)=\frac{2(3-2x)}{x^4}$. Ela se anula em
$x=\tfrac32$, e muda de sinal neste ponto:
\begin{center}
\begin{bmlimage}\begin{tikzpicture}
\tkzTabInit[nocadre,espcl=2,  color, colorV=lightgray!5, colorL=gray!15,
colorC=gray!15]
{$x$ /.6,  $f''(x)$ /.7, Conv. de $f$ /1.2}%
{,$0$, $\tfrac32$,}%
\tkzTabLine{,+,d,+,z,-,}%
\tkzTabLine{,\smile,d,\smile,z,\frown,}%
\end{tikzpicture}\end{bmlimage}
\end{center}
Logo, $f$ é convexa em $(-\infty,0)$ e $(0,\frac32)$,  côncava em
$(\frac32,\infty)$, e possui um ponto de inflexão em 
$(\tfrac{3}{2},f(\tfrac{3}{2}))=(\tfrac{3}{2},\tfrac19)$.
\begin{center}
\begin{bmlimage}\begin{tikzpicture}
\draw [thick, domain=-4:-1.2, samples=100] plot
(\x,{((\x)-1)^2/((\x)^2)});
\draw [thick, domain=0.4:4, samples=100] plot (\x,{(\x-1)^2/((\x)^2)});
\draw [ ->] (-4,0)--(4,0) node[right] {$x$};
\draw [ ->] (0,-0.1)--(0,3) node[left] {$f(x)$};
\draw [dotted] (-4,1)--(4,1) node[above] {$y=1$};
\draw [dotted] (0,0)--(0,3.5) node[right] {$x=0$};
\fill (1,0) circle (0.35mm);
\draw (1,0) node[below] {$(1,0)$};
\fill (1.5,0.1111) circle (0.35mm);
\draw [ <-] (1.52,0.0911)--(2,-0.3) node[right]
{$(\tfrac{3}{2},\tfrac{1}{9})$};
\end{tikzpicture}\end{bmlimage}
\end{center}
\eqref{itexoEstudA3}:
O domínio de  $f(x)=x(\ln x)^2$ é
$D=(0,+\infty)$, e o seu sinal é: $f(x)\geq 0$ para todo $x\in D$.
A função não é { par} nem { ímpar}.
Como $\lim_{x\to \infty}f(x)=+\infty$, não tem assintota horizontal.
Para ver se tem assíntota vertical em $x=0$, calculemos 
$\lim_{x\to 0^+}f(x)=\lim_{x\to 0^+}\frac{(\ln x)^2}{1/x}$. Como ambas funções
$(\ln x)^2$ e $1/x$ são deriváveis em $(0,1)$ e tendem a $+\infty$ quando $x\to
0^+$, apliquemos a regra de B.H.:
$$
\lim_{x\to 0^+}\frac{(\ln x)^2}{1/x}=
\lim_{x\to 0^+}\frac{2(\ln x)1/x}{-1/x^2}=
-2\lim_{x\to 0^+}x\ln x\,.
$$
Usando a regra de B.H. de novo, pode ser mostrado que esse segundo limite é
zero (ver Exemplo \ref{Ex:xlogxemzero}). Logo, $\lim_{x\to 0^+}f(x)=0$: não
tem assíntota vertical em $x=0$.
A derivada é dada por $f'(x)=\ln x(\ln x+2)$.
\begin{center}
\begin{bmlimage}\begin{tikzpicture}[scale=0.8]
\tkzTabInit[nocadre, espcl=2,  color, colorV=lightgray!5, colorL=gray!15,
colorC=gray!15]
{$x$ /.6, $f'(x)$ /.6, Variaç. de $f$ /1.2}%
{,$e^{-2}$, $1$, }%
\tkzTabLine{,+,z,-,z,+}
\tkzTabVar{-/,+/{máx.},-/{mín.},+/}
%\tkzTabLine{,\searrow,\text{mín.},h,\text{mín.},\nearrow,}
\end{tikzpicture}\end{bmlimage}
\end{center}
O máximo local está em
$(e^{-2},f(e^{-2}))=(e^{-2},4e^{- 2})$, e o
mínimo global em $(1,f(1))=(1,0)$.
A {segunda derivada} de $f$ é dada por
$f''(x)=\frac{2(\ln x+1)}{x}$.
\begin{center}
\begin{bmlimage}\begin{tikzpicture}[scale=0.8]
\tkzTabInit[nocadre, espcl=2,  color, colorV=lightgray!5, colorL=gray!15,
colorC=gray!15]
{$x$ /.6, $f''(x)$ /.6, Conv. de $f$ /1.2}%
{,$e^{-1}$, }%
\tkzTabLine{,-,z,+,}
\tkzTabLine{,\frown,,\smile,}
\end{tikzpicture}\end{bmlimage}
\end{center}
Logo, $f$ é côncava em $(0,e^{-1})$, possui um ponto de inflexão em
$(e^{-1},f(e^{-1}))=(e^{-1},e^{-1})$, e é convexa em $(e^{-1},+\infty)$.
\begin{center}
\begin{bmlimage}\begin{tikzpicture}[scale=1.3]
\draw [thick, domain=0.001:2.5, samples=100] plot (\x,{(\x)*(ln(\x))^2});
 \draw [ ->] (0,0)--(2.5,0) node[right] {$x$};
 \draw [ ->] (0,-0.1)--(0,2);
% \draw [dotted] (-4,1)--(4,1) node[above left] {Assíntota horiz.: $y=1$};
 \fill (1,0) circle (0.35mm);
 \draw (1,0) node[below] {$\scriptscriptstyle{(1,0)}$};
 \fill (0.367,0.367) circle (0.35mm);
 \draw[<-] (0.39,0.39)--(0.9,0.5) node[above]
{$\scriptscriptstyle{(e^{-1},e^{-1})}$};
 \fill (0.1353,0.541) circle (0.35mm);
 \draw[<-] (0.14,0.58)--(0.9,1.5) node[above]
{$\scriptscriptstyle{(e^{-2},4e^{-2})}$};
\end{tikzpicture}\end{bmlimage}
\end{center}
Podemos também notar que $\lim_{x\to 0^+}f'(x)=+\infty$.
\end{sol}
\end{exo}

\begin{exo}(Segunda prova, segundo semestre de 2011)
Para $f(x)\pardef\frac{x^2-4}{x^2-16}$, estude:
o sinal, os zeros, as assíntotas
(se tiver), a variação, e a posição dos pontos de mín./máx. (se tiver). A
partir dessas informações, monte o gráfico de $f$. Em seguida,
complete a sua análise com a determinação dos intervalos em que $f$ é
convexa/côncava.
\begin{sol}
$D=\bR\backslash \{\pm 4\}$. Os zeros de $f(x)\pardef\frac{x^2-4}{x^2-16}$ são
$x=-2$, $x=+2$, e o seu sinal:
\begin{center}
\begin{bmlimage}\begin{tikzpicture}
\tkzTabInit[lgt=3, nocadre, espcl=2]
{ /.6,  $x^2-4$ /.6, $x^2-16$ /.6, $f(x)$ /.8}%
{,$-4$, $-2$, $2$, $4$,}%
%\tkzTabLine{,+,z,+,,+,}
\tkzTabLine{,+,,+,z,-,z,+,,+,}
\tkzTabLine{,+,z,-,,-,,-,z,+,}
\tkzTabLine{,+,d,-,z,+,z,-,d,+,}
%\tkzTabLine{,+,z,-,z,+,}
%\tkzTabVar{-/,+/\text{a.v.},-/$0$,+/,}
%\tkzTabLine{,\searrow,\text{mín.},h,\text{mín.},\nearrow,}
\end{tikzpicture}\end{bmlimage}
\end{center}
Como 
$$
\lim_{x\to \pm\infty}f(x)=\lim_{x\to
\pm \infty}\frac{1-\frac{4}{x^2}}{1-\frac{16}{x^2}}
=1\,,$$
a reta $y=1$ é assíntota horizontal.
Como 
$$
\lim_{x\to -4^\pm}f(x)=\mp \infty\,,\quad \lim_{x\to +4^\pm}f(x)=\pm \infty\,,$$
as retas $x=-4$ e $x=+4$ são assíntotas verticais. 
A primeira derivada se calcula facilmente: $f'(x)=\frac{-24 x }{(x^2-16)^2}$,
logo a variação de $f$ é dada por:
\begin{center}
\begin{bmlimage}\begin{tikzpicture}[scale=0.8]
\tkzTabInit[nocadre, espcl=2,  color, colorV=lightgray!5, colorL=gray!15,
colorC=gray!15]
{$x$ /.6, $f'(x)$ /.6, Variaç. de $f$ /1.2}%
{,$-4$,$0$,$4$, }%
\tkzTabLine{,+,d,+,z,-,d,-}
\tkzTabVar{-/,+D-/{},+/{máx.},-D+/{},-/}
%\tkzTabLine{,\searrow,\text{mín.},h,\text{mín.},\nearrow,}
\end{tikzpicture}\end{bmlimage}
\end{center}
A posição do máximo local é: $(0,f(0))=(0,\tfrac14)$.
O gráfico:
\begin{center}
\begin{bmlimage}\begin{tikzpicture}[scale=0.7]
\pgfmathsetmacro{\a}{10}
\pgfmathsetmacro{\b}{4}
\newcommand{\funcao}[1]{ ( (#1)^2-4 )/( (#1)^2-16) }
\draw[->] (-\a,0)--(\a,0);
\draw[->] (0,-\b)--(0,\b);
\draw[thick, domain=-\a:-4.5, samples=50] plot (\x,{\funcao{\x}});
\draw[thick, domain=-3.5:3.5, samples=50] plot (\x,{\funcao{\x}});
\draw[thick, domain=4.5:\a, samples=50] plot (\x,{\funcao{\x}});
\draw[dashed] (-\a,1)node[below]{$y=1$}--(\a,1);
\draw[dashed] (-4,-\b)node[left]{$x=-4$}--(-4,\b);
\draw[dashed] (4,-\b)node[right]{$x=+4$}--(4,\b);
\draw[<-] (0.1,0.3)--(2,2)node[above]{máx.: $(0,\frac14)$};
\draw (-2,0) node{$\shortmid$} node[above]{$-2$};]
\draw (2,0) node{$\shortmid$} node[above]{$+2$};]
\end{tikzpicture}\end{bmlimage}
\end{center}
A segunda derivada: $f''(x)=24\frac{16+3x^2}{(x^2-16)^3}$, e a convexidade é
dada por
\begin{center}
\begin{bmlimage}\begin{tikzpicture}[scale=0.8]
\tkzTabInit[nocadre, espcl=2,  color, colorV=lightgray!5, colorL=gray!15,
colorC=gray!15]
{$x$ /.6, $f''(x)$ /.6, Conv. $f$ /1.2}%
{,$-4$,$4$, }%
\tkzTabLine{,+,d,-,d,+}
\tkzTabLine{,\smile,d, \frown,d,\smile,}
%\tkzTabLine{,\searrow,\text{mín.},h,\text{mín.},\nearrow,}
\end{tikzpicture}\end{bmlimage}
\end{center}
\end{sol}
\end{exo}


\begin{exo}\label{Exo:EstudosBasicos}
Faça um estudo completo das funções abaixo:
\begin{multicols}{3}
\begin{enumerate}
%\item\label{itEstBas9a} $\frac{x^2}{(x+1)^2}$
\item\label{itEstBas1} $x+\frac{1}{x}$
\item\label{itEstBas6} $x+\frac{1}{x^2}$
\item\label{itEstBas9} $\frac{1}{x^2+1}$
\item\label{itEstBas2} $\frac{x}{x^2-1}$
\item\label{itEstBas3} $xe^{-x^2}$
%\item\label{itEstBas5} $x^4-x^2$
\item\label{itEstBas7} $\senh x$
\item\label{itEstBas8} $\cosh x$
\item\label{itEstBas8t} $\tanh x$
\item\label{itEstBas13} $\frac{x^3-1}{x^3+1}$, 
\item\label{itEstBas14} $\tfrac12\sen (2x)-\sen(x)$, 
\item\label{itEstBas15} $\frac{x}{\sqrt{x^2+1}}$
\item\label{itEstFuncB30} $\frac{\sqrt{x^2-1}}{x-2}$
%\item\label{itEstBas4} $\frac{(x+1)^3}{(x-1)^4}$
\end{enumerate}
\end{multicols}
\vspace{0.01cm}
\begin{sol}
%%%%%%%%%%%%%%%%%%
% \eqref{itEstBas9a} 
%  { Domínio}: $D=\R\backslash \{-1\}$. { Sinal}:
% $f(x)\geq 0$ para todo $x\in D$, e $f(x)=0$ se e somente se $x=0$.
% Assíntotas:
% como $\lim_{x\to -1^-}f(x)=\lim_{x\to -1^+}f(x)=+\infty$, { a reta $x=-1$ é
% assíntota vertical} 
% (é a única). Como 
% $$\lim_{x\to \pm\infty}\frac{x^2}{(x+1)^2}=\lim_{x\to
% \pm\infty}\frac{1}{(1+\frac{1}{x})^2}=1\,,$$ 
%  { a reta $y=1$ é assíntota
% horizontal} (a direita e a esquerda). Como
% $f'(x)=\frac{2x}{(x+1)^3}$, 
% \begin{center}
% \begin{bmlimage}\begin{tikzpicture}[scale=0.8]
% \tkzTabInit[nocadre, espcl=2,  color, colorV=lightgray!5, colorL=gray!15,
% colorC=gray!15]
% {$x$ /.6, $f'(x)$ /.6, Variaç. de $f$ /1.2}%
% {,$-1$, $0$, }%
% \tkzTabLine{,+,d,-,z,+}
% \tkzTabVar{-/,+D+/,-/mín.,+/}
% %\tkzTabLine{,\searrow,\text{mín.},h,\text{mín.},\nearrow,}
% \end{tikzpicture}\end{bmlimage}
% \end{center}
% $f$ possui um mínimo local no ponto $(0,f(0))=(0,0)$.
% Como $f''(x)=\frac{2(1-2x)}{(x+1)^4}$, temos:
% \begin{center}
% \begin{bmlimage}\begin{tikzpicture}[scale=0.8]
% \tkzTabInit[nocadre, espcl=2,  color, colorV=lightgray!5, colorL=gray!15,
% colorC=gray!15]
% {$x$ /.6, $f''(x)$ /.6, Conv. de $f$ /1.2}%
% {,$-1$, $\tfrac12$, }%
% \tkzTabLine{,+,d,+,z,-}
% \tkzTabLine{,\smile,d,\smile,z,\frown}
% %\tkzTabVar{-/,+D+/${+\infty}$/${+\infty}$,-/mín.,+/}
% %\tkzTabLine{,\searrow,\text{mín.},h,\text{mín.},\nearrow,}
% \end{tikzpicture}\end{bmlimage}
% \end{center}
% Logo, $f$ é convexa nos intervalos $]-\infty,-1[$ e $]-1,\tfrac12[$, possui um
% ponto de inflexão em $(\tfrac12,f(\tfrac12))=(\tfrac12,\tfrac19)$, e é côncava
% em $(\tfrac12,+\infty)$. Gráfico:
% \begin{center}
% \begin{bmlimage}\begin{tikzpicture}
% \draw[ ->] (-4,0)--(4,0);
% \draw[dashed] (-5,1)node[below]{$y=1$}--(4,1);
% \draw[ ->] (0,-0.5)--(0,4);
% \draw[dashed] (-1,-0.5)node[below]{$x=-1$}--(-1,4);
% \draw[thick, domain=-5:-1.9, samples=50] plot (\x,{\x^2/(\x+1)^2});
% \draw[thick, domain=-0.65:4, samples=100] plot (\x,{\x^2/(\x+1)^2});
% \end{tikzpicture}\end{bmlimage}
% \end{center}
% Observe que esse gráfico é o gráfico da função $(\frac{x}{x-1})$
OBS: Para as demais funções, colocamos somente um \emph{resumo} das
soluções, na forma de um gráfico no qual o leitor pode verificar os resultados
do seu estudo.

\eqref{itEstBas1} Ass. vert.: $x=0$. Ass. oblíqua: $y=x$.
\begin{center}
\begin{bmlimage}\begin{tikzpicture}[yscale=0.7]
\draw [thick, domain=-3:-0.3, samples=100] plot (\x,{(\x)+1/(\x)});
\draw [thick, domain=0.3:3, samples=100] plot (\x,{(\x)+1/(\x)});
 \draw [ ->] (-3,0)--(3,0) node[right] {$x$};
 \draw [ ->] (0,-3)--(0,3) node[left]{$x+\tfrac{1}{x}$};
 \fill (1,2) circle (0.45mm);
  \draw (1,2) node[below] {$\scriptscriptstyle{(1,2)}$};
  \fill (-1,-2) circle (0.45mm);
  \draw (-1,-2) node[above] {$\scriptscriptstyle{(-1,-2)}$};
\end{tikzpicture}\end{bmlimage}
\end{center}


\eqref{itEstBas6} 
Ass. vert.: $x=0$. Ass. obl.: $y=x$.
\begin{center}
\begin{bmlimage}\begin{tikzpicture}[yscale=0.7]
\draw [thick, domain=-3:-0.5, samples=100] plot
(\x,{\x+1/((\x)^2)});
\draw [thick, domain=0.6:3, samples=100] plot
(\x,{\x+1/((\x)^2)})node[right]{$x+\tfrac{1}{x^2}$};
 \draw [ ->] (-3,0)--(3,0) node[right] {$x$};
 \draw [ ->] (0,-3)--(0,3);
 \fill (1.256,1.88) circle (0.45mm);
  \draw (1.256,1.88) node[below]
{$\scriptscriptstyle{(2^{1/3},2^{1/3}+2^{-2/3})}$};
\draw (-1,0) node{$\shortmid$} node[above left]{$-1$};
\end{tikzpicture}\end{bmlimage}
\end{center}

\eqref{itEstBas9} 
\begin{center}
\begin{bmlimage}\begin{tikzpicture}
\draw [thick, domain=-3:3, samples=100] plot
(\x,{1/((\x)^2+1)});
\draw [ ->] (-3,0)--(3,0);
\draw [ ->] (0,-0.5)--(0,1.5)node[right]{$\tfrac{1}{x^2+1}$};
\fill (0.577,0.75) circle (0.45mm);
\draw[<-] (0.6,0.8)--(1.3,1)
node[right]{inflex: $(\tfrac{1}{\sqrt{3}},\tfrac34)$};
\fill (-0.577,0.75) circle (0.45mm);
\draw[<-] (-0.6,0.8)--(-1.3,1)
node[left]{inflex: $(-\tfrac{1}{\sqrt{3}},\tfrac34)$};
\draw (-5,2) node[left]{$\displaystyle{f'(x)=\frac{-2x}{(x^2+1)^2}}$};
\draw (-5,0.5) node[left]{$\displaystyle{f''(x)=\frac{2(3x^2-1}{(x^2+1)^3}}$};
\end{tikzpicture}\end{bmlimage}
\end{center}



\eqref{itEstBas2} 
\begin{center}
\begin{bmlimage}\begin{tikzpicture}[yscale=0.7]
\draw [ ->] (-3,0)--(3,0) node[right] {$x$};
\draw [ ->] (0,-3)--(0,3) node[left]{$\frac{x}{x^2-1}$};
\draw [thick, domain=-3:-1.2, samples=100] plot (\x,{\x/((\x)^2-1)});
\draw[dashed] (-1,-2.5)--(-1,2.5) node[below left]{$\scriptscriptstyle{x=-1}$};
\draw [thick, domain=-0.8:0.8, samples=100] plot (\x,{\x/((\x)^2-1)});
\draw [thick, domain=1.2:3, samples=100] plot (\x,{\x/((\x)^2-1)});
\draw[dashed] (1,-2.5)node[right]{$\scriptscriptstyle{x=1}$}--(1,2.5) ;
\draw[<-] (0.3,-0.1)--(1.5,-1)
node[right]{pt. inflex.: $\scriptstyle{(0,0)}$};
\draw (4,2) node[right]{$\displaystyle{f'(x)=\frac{-(1+x^2)}{(x^2-1)^2}}$};
\draw (4,0.5)
node[right]{$\displaystyle{f''(x)=\frac{-2x(3x^2+1)}{(x^2-1)^3}}$};
\end{tikzpicture}\end{bmlimage}
\end{center}

\eqref{itEstBas3}
\begin{center}
\begin{bmlimage}\begin{tikzpicture}
\draw [ ->] (-3,0)--(3,0) node[right] {$x$};
\draw [ ->] (0,-1)--(0,1) node[above]{$xe^{-x^2}$};
\draw [thick, domain=-2.5:2.5, samples=100] plot (\x,{\x*exp(-(\x)^2)});
\fill (0.707,0.428) circle (0.45mm);
  \draw[<-] (0.71,0.44)-- (0.9,1) node[right]
{$\scriptstyle{(\tfrac{1}{\sqrt{2}},\tfrac{1}{\sqrt{2}}e^{-\tfrac12})}$};
\fill (-0.707,-0.428) circle (0.45mm);
  \draw[<-] (-0.71,-0.5)-- (-0.9,-1) node[left]
{$\scriptstyle{(-\tfrac{1}{\sqrt{2}},-\tfrac{1}{\sqrt{2}}e^{-\tfrac12})}$};
\draw[<-] (0.1,-0.1)--(0.5,-1.3) node[right]{pt. inflex. $\scriptstyle{(0,0)}$};
\fill (1.225,0.273) circle (0.40mm);
\fill (-1.225,-0.273) circle (0.40mm);
\draw[<-] (1.225,0.24)--(1.5,-0.6)
node[right]{pt. inflex.: $\scriptstyle{(\sqrt{3/2},f(\sqrt{3/2}))}$};
\draw[<-] (-1.225,-0.24)--(-1.5,0.6)
node[left]{pt. inflex.: $\scriptstyle{(-\sqrt{3/2},f(\sqrt{3/2}))}$};
\draw (4,1.2) node[right]{$\displaystyle{f'(x)=(1-2x^2)e^{-x^2}}$};
\draw (4,0.5)
node[right]{$\displaystyle{f''(x)=-2x(3-2x^2)e^{-x^2}}$};
\end{tikzpicture}\end{bmlimage}
\end{center}

\eqref{itEstBas7}, \eqref{itEstBas8},
\eqref{itEstBas8t}:
\begin{center}
\begin{bmlimage}\begin{tikzpicture}[scale=0.5]
\draw [ ->] (0,-0.1)--(0,3);
\pgfmathsetmacro{\a}{2};
\draw [ ->] (-\a,0)--(\a,0);
\draw [thick, domain=-\a:\a, samples=100] plot (\x,{(exp(\x)+exp(-\x))/2})
node[right]{$\cosh x$};

\begin{scope}[xshift=9cm, yshift=1cm]
\draw [ ->] (0,-2)--(0,2);
\pgfmathsetmacro{\a}{1.6};
\draw [ ->] (-\a,0)--(\a,0);
\draw [thick, domain=-\a:\a, samples=100] plot (\x,{(exp(\x)-exp(-\x))/2})
node[right]{$\senh x$};
\end{scope}

\begin{scope}[xshift=18cm, yshift=1cm]
\draw [ ->] (0,-1.5)--(0,1.5);
\pgfmathsetmacro{\a}{3};
\draw [ ->] (-\a,0)--(\a,0);
\draw [thick, domain=-\a:\a, samples=100] plot
(\x,{(exp(\x)-exp(-\x))/(exp(\x)+exp(-\x))})
node[below right]{$\tanh x$};
\draw[dashed] (0,1)--(\a,1) node[above]{$x=+1$};
\draw[dashed] (0,-1)--(-\a,-1) node[below]{$x=-1$};
\end{scope}

\end{tikzpicture}\end{bmlimage}
\end{center}

\eqref{itEstBas13}
\begin{center}
\begin{bmlimage}\begin{tikzpicture}[yscale=0.7]
\draw [ ->] (-3,0)--(3,0);
\draw [ ->] (0,-3)--(0,3) node[right]{$\frac{x^3-1}{x^3+1}$};
\draw [thick, domain=-3:-1.2, samples=100] plot (\x,{((\x)^3-1)/((\x)^3+1)});
\draw [thick, domain=-0.8:3, samples=100] plot (\x,{((\x)^3-1)/((\x)^3+1)});
\draw[dashed] (-1,-3)node[left]{$\scriptscriptstyle{x=-1}$}--(-1,3) ;
\draw[dashed] (-3,1) node[below]{$\scriptscriptstyle{x=1}$}--(3,1) ;
\fill (0,-1) circle (0.45mm);
\fill (0.793, -0.3333) circle (0.45mm);
\draw[<-] (0.1,-1.1)--(1,-3)node[right]{Pt. de inflexão e crítico: $(0,-1)$};
\draw[<-] (0.8, -0.4)--(1.2,-1) node[right]{Pt. de inflexão:
$(2^{-1/3},-1/3)$};
\draw (1,0) node{$\shortmid$} node[above]{$1$};
\draw (4,2) node[right]{$\displaystyle{f'(x)=\frac{6x^2}{(x^3+1)^2}}$};
\draw (4,0.5)
node[right]{$\displaystyle{f''(x)=\frac{12x(1-2x^3)}{(x^3+1)^3}}$};
\end{tikzpicture}\end{bmlimage}
\end{center}

\eqref{itEstBas14}:
\begin{center}
\begin{bmlimage}\begin{tikzpicture}
\draw [ ->] (-0.2,0)--(2*pi+0.5,0);
\draw [ ->] (0,-1.5)--(0,1.5) node[right]{$\scriptstyle{\tfrac12\sen
(2x)-\sen(x)}$};
\draw [color=gray!20, domain=-1:2*pi+1, samples=100] plot (\x,{0.5*sin(2*\x
r)-sin(\x r)});
\draw [thick, domain=0:2*pi, samples=100] plot (\x,{0.5*sin(2*\x r)-sin(\x r)});
\foreach \k in {0, 0.666, 1.333, 2} {
\draw ({\k*pi},0) node{$\shortmid$};
}
\draw (0.666*pi,0) node[above]{$\tfrac{2\pi}{3}$};
\draw (1.333*pi,0) node[below]{$\tfrac{4\pi}{3}$};
\fill (1.318,-0.726) circle (0.40mm); 
\fill (4.965,0.726) circle (0.40mm); 
\fill (0,0) circle (0.40mm); 
\fill (pi,0) circle (0.40mm); 
\fill (2*pi,0) circle (0.40mm); 
% \draw[dashed] (-1,-3)node[left]{$\scriptscriptstyle{x=-1}$}--(-1,3) ;
% \draw[dashed] (-3,1) node[below]{$\scriptscriptstyle{x=1}$}--(3,1) ;
% \fill (0,-1) circle (0.45mm);
% \fill (0.793, -0.3333) circle (0.45mm);
% \draw[<-] (0.1,-1.1)--(1,-3)node[right]{Pt. de inflexão e crítico: $(0,-1)$};
% \draw[<-] (0.8, -0.4)--(1.2,-1) node[right]{Pt. de inflexão:
% $(2^{1/3},f(2^{1/2}))$};
% \draw (1,0) node{$\shortmid$} node[above]{$1$};
\end{tikzpicture}\end{bmlimage}
\end{center}

\eqref{itEstBas15}: 
\begin{center}
\begin{bmlimage}\begin{tikzpicture}
\draw [ ->] (-5,0)--(5,0);
\draw [ ->] (0,-1.3)--(0,1.3) node[left]{$\frac{x}{\sqrt{x^2+1}}$};
\draw [thick, domain=-5:5, samples=50] plot (\x,{\x/(sqrt((\x)^2+1))});
 \draw[dashed] (0,1)--(5,1)node[above]{$\scriptscriptstyle{y=1}$};
 \draw[dashed] (-5,-1) node[below]{$\scriptscriptstyle{y=-1}$}--(0,-1) ;
 \draw[<-] (0.2, -0.2)--(0.5,-0.7) node[right]{Pt. de inflexão: $(0,0)$};
\draw (6,0.5) node[right]{$\displaystyle{f'(x)=\frac{1}{(x^2+1)^{3/2}}}$};
\draw (6,-0.5)
node[right]{$\displaystyle{f''(x)=\frac{-3x}{(x^2+1)^{5/2}}}$};
\end{tikzpicture}\end{bmlimage}
\end{center}


\end{sol}
\end{exo}


\begin{exo}
Faça um estudo completo das seguintes funções.
\begin{multicols}{3}
\begin{enumerate}
\item\label{itEstFuncB1} $\ln |2-5x|$
\item\label{itEstFuncB3} $\ln(\ln x)$
\item\label{itEstFuncB7} $e^{-x}(x^2-2x)$.
\item\label{itEstFuncB70} $\sqrt[x]{x}$.
\item\label{itEstFuncB4} $\frac{\ln x}{\sqrt{x}}$
\item\label{itEstFuncB5} $\frac{\ln x-2}{(\ln x)^2}$
\item\label{itEstFuncB2} $\ln(e^{2x}-e^x+3)$
\item\label{itEstFuncB29} $(e^{|x|}-2)^3$
\item\label{itEstFuncB33} $\frac{e^x}{e^x-x}$
\item\label{itEstFuncB33a} $\arcos(\frac{1-x^2}{1+x^2})$
\item\label{itEstFuncB36} $\sqrt[5]{x^4(x-1)}$
%\item\label{itEstFuncB6} $\arcsen(2x^2-1)$
%\item   $\frac{\sqrt{x^2-1}}{x-2}$
%\item\label{itEstFuncB8} $\frac{\sqrt{x^2-1}}{x-2}$ 
%\item\label{itEstFuncB9} $(\ln x)^2+\ln x$.
\end{enumerate}
\end{multicols}
\vspace{0.01cm}

\begin{sol}
\eqref{itEstFuncB1}
\begin{center}
\begin{bmlimage}\begin{tikzpicture}[yscale=0.8]
\draw [ ->] (-4,0)--(4,0);
\draw [ ->] (0,-1.3)--(0,2.3) node[above]{$\ln |2-5x|$};
\draw [thick, domain=-4:0.3, samples=100] plot (\x,{ln(abs(2-5*(\x)))});
\draw[dashed] (0.4,-1.5)node[right]{$\scriptscriptstyle{x=\frac25}$}--(0.4,1.5);
\draw [thick, domain=0.5:4, samples=100] plot (\x,{ln(abs(2-5*\x))});
%  \draw[dashed] (-5,-1) node[below]{$\scriptscriptstyle{y=-1}$}--(0,-1) ;
% \fill (0,-1) circle (0.45mm);
% \fill (0.793, -0.3333) circle (0.45mm);
% \draw[<-] (0.1,-1.1)--(1,-3)node[right]{Pt. de inflexão e crítico: $(0,-1)$};
% \draw[<-] (0.2, -0.2)--(0.5,-0.7) node[right]{Pt. de inflexão: $(0,0)$};
% \draw (1,0) node{$\shortmid$} node[above]{$1$};
\end{tikzpicture}\end{bmlimage}
\end{center}


\eqref{itEstFuncB3}
\begin{center}
\begin{bmlimage}\begin{tikzpicture}[yscale=0.7]
\draw [ ->] (0,0)--(5,0);
\draw [ ->] (0,-1.3)--(0,2.3) node[left]{$\ln(\ln x)$};
\draw [thick, domain=1.2:5, samples=100] plot (\x,{ln(ln(\x))});
% \draw[dashed]
%(0.4,-1.5)node[right]{$\scriptscriptstyle{x=\frac25}$}--(0.4,1.5);
% \draw [thick, domain=0.5:4, samples=100] plot (\x,{ln(abs(2-5*\x))});
 \draw[dashed] (1,-2) node[left]{$\scriptscriptstyle{x=1}$}--(1,2) ;
% \fill (-0.693,1.012) circle (0.45mm);
% \fill (-1.365,1.033) circle (0.45mm);
% \fill (2.46,4.86) circle (0.45mm);
% \fill (0.793, -0.3333) circle (0.45mm);
% \draw[<-] (0.1,-1.1)--(1,-3)node[right]{Pt. de inflexão e crítico: $(0,-1)$};
% \draw[<-] (0.2, -0.2)--(0.5,-0.7) node[right]{Pt. de inflexão: $(0,0)$};
% \draw (1,0) node{$\shortmid$} node[above]{$1$};
\end{tikzpicture}\end{bmlimage}
\end{center}

\eqref{itEstFuncB7}
\begin{center}
\begin{bmlimage}\begin{tikzpicture}
\newcommand{\funcao}[1]{2.5*exp( -1*(#1) )*( (#1)^2 - 2*(#1))}
\draw [ ->] (-1,0)--(6.5,0);
\draw [ ->] (0,-1)--(0,2.3) node[right]{$e^{-x}(x^2-2x)$};
\draw [thick, domain=-0.35:5.5, samples=100] plot (\x,{\funcao{\x}});
\fill ({2-sqrt(2)},{\funcao{2-sqrt(2)}}) circle (0.40mm);
\draw ({2-sqrt(2)},{\funcao{2-sqrt(2)}})
node[below]{$\scriptstyle{(2-\sqrt{2},f(2-\sqrt{2}))}$};
\draw ({2+sqrt(2)},{\funcao{2+sqrt(2)}})
node[above]{$\scriptstyle{(2+\sqrt{2},f(2+\sqrt{2}))}$};
\fill ({2+sqrt(2)},{\funcao{2+sqrt(2)}}) circle (0.40mm);
\fill ({(6+sqrt(10))/2},{\funcao{(6+sqrt(10))/2}}) circle (0.40mm);
\draw[<-] ({(6+sqrt(10))/2+0.1},{\funcao{(6+sqrt(10))/2}+0.1})--
({(6+sqrt(10))/2+0.5},{\funcao{(6+sqrt(10))/2}+0.3})
node[right]{$\scriptstyle{(3+\sqrt{10}/2,f(3+\sqrt{10}/2))}$};
\fill ({(6-sqrt(10))/2},{\funcao{(6-sqrt(10))/2}}) circle (0.40mm);
\draw[<-] ({(6-sqrt(10))/2-0.1},{\funcao{(6-sqrt(10))/2}+0.1})--
({(6-sqrt(10))/2-0.5},{\funcao{(6-sqrt(10))/2}+2})
node[right]{$\scriptstyle{(3-\sqrt{10}/2,f(3-\sqrt{10}/2))}$};
\draw[<-] (5,-0.2)--(4.5,-0.5)node[below]{ass. horiz.: $y=0$};
\draw (6,2)
node[right]{$\displaystyle{f'(x)=-(x^2-4x+2)e^{-x}}$};
\draw (6,1.5)
node[right]{$\displaystyle{f''(x)=(x^2-6x+6)e^{-x}}$};
\end{tikzpicture}\end{bmlimage}
\end{center}




\eqref{itEstFuncB70}


\begin{center}
\begin{bmlimage}\begin{tikzpicture}[yscale=0.7]
\draw [ ->] (0,0)--(2.5,0);
\draw [ ->] (0,-0.3)--(0,1.5) node[left]{$\sqrt[x]{x}$};
\draw [thick, domain=0.2:6, samples=100, <-] plot 
(\x,{exp(ln(\x)/\x)});
\fill (2.718,1.444) circle (0.45mm) node[above]{máx. glob.:
$(e,\sqrt[e]{e})$};
\coordinate (A) at (0.539,0.318);
\coordinate (B) at (5.04,1.37);
\fill (A) circle (0.45mm);
\fill (B) circle (0.45mm);
\draw[<-] (A)--(1.2,-0.3) node[right]{pt. infl.:
$(x_1,f(x_1))$};
\draw[<-] (B)--(5.2,1.9) node[right]{pt. infl.:
$(x_2,f(x_2))$};
%\draw[<-] (-0.67,0.9)--(0.2,0.4)node[right]{mín. global: $(\ln \tfrac12,f(\ln
%\tfrac12))$};
%\fill (-1.365,1.033) circle (0.45mm);
%\draw[<-] (-1.4,0.9)--(-1.6,0.5)node[left]{pt. infl.};
%\fill (2.46,4.86) circle (0.45mm);
%\draw[<-] (2.55,4.7)--(3,4)node[right]{pt. infl.};
\draw[dashed] (0,1)--(6,1);
\draw (2,1) node[below right]{Ass. Horiz.: $y=1$};
%\draw (5,2.5)
%node[right]{$\displaystyle{f'(x)=\frac{e^x(2e^x-1)}{e^{2x}-e^x+3}}$};
%\draw (5,0.5)
%node[right]{$\displaystyle{f''(x)=\frac{e^x(12e^x-e^{2x}
%-3)}{(e^{2x}-e^x+3)^2}}$};
\end{tikzpicture}\end{bmlimage}
\end{center}
Os pontos de inflexão são soluções da equação $(1-\ln
x)^2-3x+2x\ln x=0$. Pode ser mostrado que esses satisfazem
$x_1\simeq 0.58$, $x_1\simeq 4.37$.  

\eqref{itEstFuncB4}
\begin{center}
\begin{bmlimage}\begin{tikzpicture}[yscale=0.7]
\newcommand{\funcao}[1]{ln( 5*(#1) )/sqrt( 5*(#1) ) }
\draw [ ->] (0,0)--(8,0);
\draw [ ->] (0,-1.3)--(0,2.3) node[left]{$\frac{\ln x}{\sqrt{x}}$};
\draw [thick, domain=0.1:8, samples=100] plot (\x,{\funcao{\x}});
% \draw[dashed] (1,-2) node[left]{$\scriptscriptstyle{x=1}$}--(1,2) ;
\fill ({2.718^2/5},{\funcao{2.718^2/5}}) circle (0.40mm);
\draw ({2.718^2/5},{\funcao{2.718^2/5}}) node[above]{$(e^2,2/e)$};
\fill ({2.718^(8/3)/5},{\funcao{2.718^(8/3)/5}}) circle (0.40mm);
\draw[<-] ({2.718^(8/3)/5+0.1},{\funcao{2.718^(8/3)/5}+0.2})--
({2.718^(8/3)/5+0.3},{\funcao{2.718^(8/3)/5}+1.3})
node[above]{pt. infl.: $(e^{8/3},f(e^{8/3}))$};
\draw (6,2.8)
node[right]{$\displaystyle{f'(x)=\frac{2-\ln x}{2x^{3/2}}}$};
\draw (6,1.5)
node[right]{$\displaystyle{f''(x)=-\frac{\sqrt{x}}{2}\frac{4-\tfrac32 \ln
x}{|x|^3}}$};
\draw[<-] (5,-0.2)--(4,-0.6) node[below]{ass. horiz.: $y=0$};

\end{tikzpicture}\end{bmlimage}
\end{center}

\eqref{itEstFuncB5} 
%$\frac{\ln x-2}{(\ln x)^2}$
\begin{center}
\begin{bmlimage}\begin{tikzpicture}[yscale=0.5]
\newcommand{\funcao}[1]{( ln( (#1) ) -2)/ ( (ln( (#1) ))^2 ) }
\draw [ ->] (0,0)--(12,0);
\draw [ ->] (0,-1.3)--(0,2.3) node[left]{$\frac{\ln x-2}{(\ln x)^2}$};
\draw[dashed] (1,1)node[above]{$x=1$}--(1,-5);
\draw [thick, domain=0.1:0.5, samples=100] plot (\x,{\funcao{\x}});
\draw [thick, domain=1.6:11, samples=100] plot (\x,{\funcao{\x}});
% \draw[dashed] (1,-2) node[left]{$\scriptscriptstyle{x=1}$}--(1,2) ;
% \fill ({2.718^2/5},{\funcao{2.718^2/5}}) circle (0.40mm);
% \draw ({2.718^2/5},{\funcao{2.718^2/5}}) node[above]{$(e^2,2/e)$};
% \fill ({2.718^(8/3)/5},{\funcao{2.718^(8/3)/5}}) circle (0.40mm);
% \draw[<-] ({2.718^(8/3)/5+0.1},{\funcao{2.718^(8/3)/5}+0.2})--
% ({2.718^(8/3)/5+0.3},{\funcao{2.718^(8/3)/5}+1.3})
% node[above]{pt. infl.: $(e^{8/3},f(e^{8/3}))$};
% \draw (6,2.8)
% node[right]{$\displaystyle{f'(x)=\frac{2-\ln x}{2x^{3/2}}}$};
% \draw (6,1.5)
% node[right]{$\displaystyle{f''(x)=-\frac{\sqrt{x}}{2}\frac{4-\tfrac32 \ln
% x}{|x|^3}}$};
\draw[<-] (6,0.2)--(5,0.6) node[above]{ass. horiz.: $y=0$};
\draw (7,-1.5) node[right]{máx. global em $(e^4,f(e^4))$};
\draw (7,-3) node[right]{pt. infl. em
$(e^{1+\sqrt{13}},f(e^{1+\sqrt{13}})$};
\draw (5,-4.5) node[right]{$f'(x)=\frac{4-\ln x}{x(\ln x)^3}$, 
$f''(x)=\frac{(\ln x)^2-2\ln x-12}{x^2(\ln x)^4}$};
\end{tikzpicture}\end{bmlimage}
\end{center}


\eqref{itEstFuncB2}
Ass. horiz.: $y=\ln 3$. Ass. obl.: $y=2x$.
\begin{center}
\begin{bmlimage}\begin{tikzpicture}[yscale=0.7]
\draw [ ->] (-4,0)--(2.5,0);
\draw [ ->] (0,-1.3)--(0,2.3) node[left]{$\ln(e^{2x}-e^x+3)$};
\draw [thick, domain=-4:2.6, samples=100] plot (\x,{ln(exp(2*\x)-exp(\x)+3)});
\fill (-0.693,1.012) circle (0.45mm);
\draw[<-] (-0.67,0.9)--(0.2,0.4)node[right]{mín. global: $(\ln \tfrac12,f(\ln
\tfrac12))$};
\fill (-1.365,1.033) circle (0.45mm);
\draw[<-] (-1.4,0.9)--(-1.6,0.5)node[left]{pt. infl.};
\fill (2.46,4.86) circle (0.45mm);
\draw[<-] (2.55,4.7)--(3,4)node[right]{pt. infl.};
\draw[dashed] (-4,{ln(3)})node[above]{$y=\ln 3$}--(-1,{ln(3)});
\draw (5,2.5)
node[right]{$\displaystyle{f'(x)=\frac{e^x(2e^x-1)}{e^{2x}-e^x+3}}$};
\draw (5,0.5)
node[right]{$\displaystyle{f''(x)=\frac{e^x(12e^x-e^{2x}
-3)}{(e^{2x}-e^x+3)^2}}$};
\end{tikzpicture}\end{bmlimage}
\end{center}

\eqref{itEstFuncB29} Observe que $(e^{|x|}-2)^3$ é par, e não
é derivável em $x=0$. 
\begin{center}
\begin{bmlimage}\begin{tikzpicture}[yscale=0.7]
\draw [ ->] (-2,0)--(2,0);
\draw [ ->] (0,-1.3)--(0,2.3) node[above]{$(e^{|x|}-2)^3$};
\draw [thick, domain=0:1.2, samples=50] plot
(\x,{(exp(\x)-2)^3});
\draw [thick, domain=0:1.2, samples=50] plot
(-\x,{(exp(\x)-2)^3});
%\fill (-0.693,1.012) circle (0.45mm);
\draw[<-] (0.1,-1.05)--(1,-1.3) node[right]{mín. global: $(0,-1)$};
%\fill (-1.365,1.033) circle (0.45mm);
\draw[<-] (0.683,-0.1)--({0.693+0.5},-0.5) 
node[right]{pt. infl.: $(\ln 2,0)$};
\draw[<-] (-0.683,-0.1)--({-0.693-0.5},-0.5) 
node[left]{pt. infl.: $(-\ln 2,0)$};
\fill (0.693,0) circle (0.45mm);
\fill (-0.693,0) circle (0.45mm);
%\draw[<-] (2.55,4.7)--(3,4)node[right]{pt. infl.};
%\draw[dashed] (-4,{ln(3)})node[above]{$y=\ln 3$}--(-1,{ln(3)});
%\draw (5,2.5)
%node[right]{$\displaystyle{f'(x)=\frac{e^x(2e^x-1)}{e^{2x}-e^x+3}}$};
%\draw (5,0.5)
%node[right]{$\displaystyle{f''(x)=\frac{e^x(12e^x-e^{2x}
%-3)}{(e^{2x}-e^x+3)^2}}$};
\end{tikzpicture}\end{bmlimage}
\end{center}

\eqref{itEstFuncB33}
\begin{center}
\begin{bmlimage}\begin{tikzpicture}[yscale=0.7]
\draw [ ->] (-3,0)--(3,0);
\draw [ ->] (0,-0.3)--(0,1.4) node[above]{$\frac{e^x}{e^x-x}$};
\draw [thick, domain=-3:3, samples=50] plot
(\x,{exp(\x)/(exp(\x)-\x)});
\draw[dashed] (0,1)--(3,1);
%\draw [thick, domain=0:1.2, samples=50] plot
%(-\x,{(exp(\x)-2)^3});
%\fill (-0.693,1.012) circle (0.45mm);
%\draw[<-] (0.1,-1.05)--(1,-1.3) node[right]{mín. global: $(0,-1)$};
%\fill (-1.365,1.033) circle (0.45mm);
%\draw[<-] (0.683,-0.1)--({0.693+0.5},-0.5) 
%node[right]{pt. infl.: $(\ln 2,0)$};
%\draw[<-] (-0.683,-0.1)--({-0.693-0.5},-0.5) 
%node[left]{pt. infl.: $(-\ln 2,0)$};
%\fill (0.693,0) circle (0.45mm);
%\fill (-0.693,0) circle (0.45mm);
%\draw[<-] (2.55,4.7)--(3,4)node[right]{pt. infl.};
%\draw[dashed] (-4,{ln(3)})node[above]{$y=\ln 3$}--(-1,{ln(3)});
%\draw (5,2.5)
%node[right]{$\displaystyle{f'(x)=\frac{e^x(2e^x-1)}{e^{2x}-e^x+3}}$};
%\draw (5,0.5)
%node[right]{$\displaystyle{f''(x)=\frac{e^x(12e^x-e^{2x}
%-3)}{(e^{2x}-e^x+3)^2}}$};
\end{tikzpicture}\end{bmlimage}
\end{center}

\eqref{itEstFuncB33a}
\begin{center}
\begin{bmlimage}\begin{tikzpicture}[yscale=1]
\draw [ ->] (-4,0)--(4,0);
\draw [ ->] (0,-0.3)--(0,3.3) node[above right]{$\displaystyle{
\arcos(\frac{1-x^2}{1+x^2})}$};
\draw [thick, domain=-3.7:3.7, samples=51] plot
(\x,{3.1415/180*acos((1-\x*\x)/(1+\x*\x))});
\draw[dashed] (-4,3.14)--(4,3.14) node[right]{$y=\pi$};
\draw (5,1.7) node{Obs: a função não é derivável em $x=0$!};
\end{tikzpicture}\end{bmlimage}
\end{center}

\eqref{itEstFuncB36}

\begin{center}
\begin{bmlimage}\begin{tikzpicture}[yscale=0.9]
%\newcommand{\funcao}[1]{(abs(#1))^(0.8)*(abs((#1)-1))^(0.2)*(-1)};
\draw [ ->] (-3,0)--(3,0);
\draw [ ->] (0,-0.3)--(0,1.4) node[above]{
$\sqrt[5]{x^4(x-1)}$};
\pgfmathsetmacro{\e}{0.002};
\coordinate (A) at (0.8,-0.606);
\fill (A) circle (0.45mm);
\draw[<-] (0.8,-0.73)--(1.3,-1) node[right]{mín. loc.: 
$(\tfrac45,f(\tfrac45))$};
\draw [thick, domain=-2:-\e, samples=50] plot
%%PROBLEMA:
(\x,{-exp(0.8*ln(abs(\x))+0.2*ln(abs(\x-1)))});
\draw [thick, domain=\e:{1-\e}, samples=50] plot
%(\x,{(abs(\x))^(0.8)*(abs(\x-1))^(0.2)*(-1)});
(\x,{-exp(0.8*ln(abs(\x))+0.2*ln(abs(\x-1)))});
\draw [thick, domain={1+\e}:3, samples=50] plot
(\x,{exp(0.8*ln(abs(\x))+0.2*ln(abs(\x-1)))});
\draw[<-] (-0.1,0.1)--(-1.5,1) node[left]{máx. loc.: $(0,0)$};
\draw[thick]
({1-\e},{-exp(0.8*ln(abs(1-\e))+0.2*ln(abs(1-\e-1)))})--
({1+\e},{exp(0.8*ln(abs(1+\e))+0.2*ln(abs(1+\e-1)))});
\draw[dashed] (-2,-2.2)--(3,2.8) node[right]{Ass. obl.:
$y=x-\tfrac15$.};
\end{tikzpicture}\end{bmlimage}
\end{center}
Obs: $f'(x)=f(x)\varphi(x)$, onde
$\varphi(x)=\tfrac15(\tfrac{4}{x}+\tfrac{1}{x-1})$. 
A função não é derivável nem em $x=0$, nem em $x=1$
(apesar de ser contínua nesses pontos).
$f''(x)=(\varphi(x)^2+\varphi'(x))f(x)=-\tfrac{4}{25}
\frac{f(x)}{x^2(x-1)^2}$, logo, $f$ é convexa em
$(-\infty,0)$ e $(0,1)$, côncava em $(1,\infty)$.
Essa função possui uma assíntota \emph{oblíqua}:
$y=x-\tfrac15$.
\end{sol}
\end{exo}

%\begin{itemize}
% \item Antes de tudo, encontrar o \emph{domínio} de $f$. O domínio precisa ser
%especificado para evitar indeterminações, divisões por zero, e raizes
%ou logaritmos de números negativos.
%Em particular, alguns pontos excluidos do domínio serão estudados depois, caso
%sejam associados a assíntotas verticais.
%\item Se for possível, estudar os \emph{zeros} e o \emph{sinal} de $f$.
%\item Determinar se $f$ possui algumas \emph{simetrias}, via o estudo da
%\emph{paridade} de $f$. Exemplos de funções pares são $x^p$ ($p$ par), $\cos
%x$, $\cosh x$, etc.
%\item Estudar o comportamento de $f$ para valores de $x$ grandes (se o domínio
%o permite). Isto é, procurar \emph{assíntotas horizontais}. Os limites
%envolvidos podem precisar da regra de Bernoulli-l'Hopital.
% \item Estudar o comportamento de $f$ perto dos valores de $x$ onde $f(x)$ toma
%valores grandes. Isto é, procurar \emph{assíntotas verticais} calculando os
%limites laterais $\lim_{x\to a^+}f(x)$ e $\lim_{x\to a^+}f(x)$ nos pontos $a$
%perto dos quais $f$ não é limitada.
%\item Estudar a derivabilidade de $f$ e da sua derivada. Em particular,
%procurar os \emph{pontos críticos de $f$}. Deduzir a sua \emph{variação} (via o
%estudo do sinal de $f'$).
%\item Estudar a convexidade de $f$, via o sinal da segunda derivada.
%Às vezes, o estudo do sinal de $f''$ nos pontos críticos (se tiver) pode
%permitir determinar se é mínimo ou máximo local.
%\item Juntando todas essas informações, montar o gráfico de $f$.
%Por exemplo, se $f$ é par, o gráfico é simétrico com respeito ao eixo $y$.
%Na hora de montar o gráfico, pode ser necessário calcular mais alguns limites.
%\end{itemize}
%
%
%
%
%O QUE NAO CAI NA PROVA: linearizacao.
%
%\begin{ex} Comecemos com $f(x)=\frac{x+1}{1-x}$, cujo domínio é $D=\bR\setminus
%\{1\}$. A função se anula no ponto $x=-1$, e o seu sinal é dado por:
%\begin{center}
%\begin{bmlimage}\begin{tikzpicture}
%\tkzTabInit[lgt=3, nocadre, espcl=2, colorC=red, colorV=blue]
%{Valores de $x$: /.6,  $x+1$ /.6, $1-x$ /.6, $f(x)$ /.8}%
%{,$-1$, $1$,}
%\tkzTabLine{,-,z,+, ,+,}
%\tkzTabLine{,+, ,+,z,-,}
%\tkzTabLine{,-,z,+,d,-,}
%\end{tikzpicture}\end{bmlimage}
%\end{center}
%(A dupla barra em $x=1$ é para indicar que $f$ não é definida em $x=1$.)
%A funçao não é nem par, nem ímpar.
%Como
%$$
%\lim_{x\to \pm\infty}\frac{x+1}{1-x}=\lim_{x\to
%\pm\infty}\frac{1+\frac{1}{x}}{\frac{1}{x}-1}=
%\frac{1}{-1}=-1\,,
%$$
%$f$ possui a reta $y=-1$ como assíntota horizontal.
%Por outro lado, como
%$$
%\lim_{x\to 1^+}\frac{x+1}{1-x}=-\infty\,,\quad 
%\lim_{x\to 1^-}\frac{x+1}{1-x}=+\infty\,,\quad 
%$$
%$f$ possui a reta $x=1$ como assíntota vertical.
%A derivada existe em todo $x\neq 1$, e vale
%$$
%f'(x)=\frac{(x+1)'(1-x)-(x+1)(1-x)'}{(1-x)^2}=
%\frac{1-x+(x+1)}{(1-x)^2}=
%\frac{2}{(1-x)^2}\,.
%$$
%O sinal de $f'$ dá logo a tabela de variação de $f$:
%\begin{center}
%\begin{bmlimage}\begin{tikzpicture}[scale=0.8]
%\tkzTabInit[nocadre, espcl=2,  color, colorV=lightgray!5, colorL=blue!15,
%colorC=blue!15]
%{$x$ /.6, $f'(x)$ /.6, Variaç. de $f$ /1.2}%
%{,$1$, }%
%\tkzTabLine{,+,d,+,}
%\tkzTabVar{-/,+D-/$\scriptscriptstyle{+\infty}$/$\scriptscriptstyle{-\infty}$,+/
%}
%%\tkzTabLine{,\searrow,\text{mín.},h,\text{mín.},\nearrow,}
%\end{tikzpicture}\end{bmlimage}
%\end{center}
%(Indicamos o fato de $x=1$ ser uma assíntota vertical.)
%Assim, $f$ não possui pontos críticos, e é crescente nos intervalos
% $]-\infty,1[$ e $]1.\infty[$.
%A segunda derivada se calcula facilmente (para $x\neq 0$):
%$$f''(x)=2((1-x)^{-2})'=2(-2)(1-x)^{-3}(-1)=\frac{4}{(1-x)^3}\,.$$
%Esta muda de sinal em $x=1$, e permite descrever a convexidade de $f$:
%\begin{center}
%\begin{bmlimage}\begin{tikzpicture}[scale=0.8]
%\tkzTabInit[nocadre, espcl=2,  color, colorV=lightgray!5, colorL=blue!15,
%colorC=blue!15]
%{$x$ /.6, $f''(x)$ /.6, Conv. de $f$ /1.2}%
%{,$1$, }%
%\tkzTabLine{,+,d,-,}
%\tkzTabLine{,\smile,d,\frown,}
%\end{tikzpicture}\end{bmlimage}
%\end{center}
%Isto é, $f$ é convexa em $]-\infty,1[$, côncava em $]1,\infty[$. Assim, o
%gráfico é da forma
%\begin{center}
%\begin{bmlimage}\begin{tikzpicture}[yscale=0.6]
%\draw[>=latex, ->] (-4,0)--(5,0);
%\draw[>=latex, ->] (0,-4)--(0,+2.5);
%\draw[dashed] (-4,-1)node[below right]{$y=-1$}--(5,-1);
%\draw[dashed] (1,-4)--(1,3)node[right]{$x=1$};
%\draw[thick, domain=-4:0.5, samples=50] plot (\x,{(\x+1)/(1-\x)});
%\draw[thick, domain=1.5:5, samples=50] plot (\x,{(\x+1)/(1-\x)});
%\draw (-1,0) node{$\shortmid$} node[above]{$-1$};
%\draw (1,0) node{$\shortmid$} node[above right]{$1$};
%\end{tikzpicture}\end{bmlimage}
%\end{center}
%\end{ex}
%
%\begin{exo} Faça um estudo completo das seguintes funções. 
%% \begin{multicols}{1}
%\begin{enumerate}
%\item\label{itexoEstudA1} $\bigl(\frac{x-1}{x}\bigr)^2$ (Segunda prova,
%primeiro semestre 2011)
%\item \label{itexoEstudA2} $\frac{x^2-1}{x^2+1}$
%\item \label{itexoEstudA3} $x(\ln x)^2$ (Segunda prova, 2010)
%\end{enumerate}
%% \end{multicols}
%\begin{sol}
%%%%%%%%%%%%%%%%%%%%%%%%%%%%%%%%%5
%\eqref{itexoEstudA1}
%O domínio de $\bigl(\frac{x-1}{x}\bigr)^2$ é $D=\bR\setminus \{0\}$, o sinal é
%sempre não-negativo, tem um zero
%em $x=1$. $f$ não é par, nem ímpar.
%Os limites relevantes são $\lim_{x\to 0^{\pm}}f(x)=+\infty$, logo $x=0$ é
%assíntota vertical, e
%$$\lim_{x\to \pm\infty}\bigl(\frac{x-1}{x}\bigr)^2=\Bigl(\lim_{x\to \pm
%\infty}\frac{x-1}{x}\Bigr)^2==\Bigl(\lim_{x\to \pm
%\infty}\bigl(1-\frac{1}{x}\bigr)\Bigr)^2=1^2=1\,.$$
%Logo, $y=1$ é assíntota horizontal. 
%$f$ é derivável em $D$, e $f'(x)=\frac{2(x-1)}{x^3}$.
%\begin{center}
%\begin{bmlimage}\begin{tikzpicture}
%\tkzTabInit[nocadre,espcl=2,  color, colorV=lightgray!5, colorL=blue!15,
%colorC=blue!15]
%{$x$ /.6,  $f'(x)$ /.6, Var. de $f$ /1.3}%
%{,$0$, $1$,}%
%%\tkzTabLine{,+,z,+,,+,}
%\tkzTabLine{,+,d,-,z,+,}
%\tkzTabVar{-/,+D+/$+\infty$/$+\infty$,-/mín,+/,}
%%\tkzTabLine{,\searrow,\text{mín.},h,\text{mín.},\nearrow,}
%\end{tikzpicture}\end{bmlimage}
%\end{center}
%$f$ possui um mínimo global em $(1,0)$.
%A segunda derivada é dada por $f''(x)=\frac{2(3-2x)}{x^4}$. Ela se anula em
%$x=\tfrac32$, e muda de sinal neste ponto:
%\begin{center}
%\begin{bmlimage}\begin{tikzpicture}
%\tkzTabInit[nocadre,espcl=2,  color, colorV=lightgray!5, colorL=blue!15,
%colorC=blue!15]
%{$x$ /.6,  $f''(x)$ /.7, Conv. de $f$ /1.2}%
%{,$0$, $\tfrac32$,}%
%\tkzTabLine{,+,d,+,z,-,}%
%\tkzTabLine{,\smile,d,\smile,z,\frown,}%
%\end{tikzpicture}\end{bmlimage}
%\end{center}
%Logo, $f$ é convexa em $]-\infty,0[$ e $]0,\frac32[$,  côncava em
%$]\frac32,\infty[$, e possui um ponto de inflexão em 
%$(\tfrac{3}{2},f(\tfrac{3}{2}))=(\tfrac{3}{2},\tfrac19)$.
%\begin{center}
%\begin{bmlimage}\begin{tikzpicture}
%\draw [thick, domain=-4:-1.2, samples=100] plot (\x,{(\x-1)^2/\x^2});
%\draw [thick, domain=0.4:4, samples=100] plot (\x,{(\x-1)^2/\x^2});
%\draw [>=latex, ->] (-4,0)--(4,0) node[right] {$x$};
%\draw [>=latex, ->] (0,-0.1)--(0,3) node[left] {$f(x)$};
%\draw [dotted] (-4,1)--(4,1) node[above] {$y=1$};
%\draw [dotted] (0,0)--(0,3.5) node[right] {$x=0$};
%\fill (1,0) circle (0.35mm);
%\draw (1,0) node[below] {$(1,0)$};
%\fill (1.5,0.1111) circle (0.35mm);
%\draw [>=latex, <-] (1.52,0.0911)--(2,-0.3) node[right]
%{$(\tfrac{3}{2},\tfrac{1}{9})$};
%\end{tikzpicture}\end{bmlimage}
%\end{center}
%\eqref{itexoEstudA2}
%Domínio: $D=\bR$.  Sinal: $f(x)$ é $\geq 0$ se $|x|\geq 1$, $<0$ caso contrário.
%Como $f(-x)=\frac{(-x)^2-1}{(-x)^2+1}=\frac{x^2-1}{x^2+1}=f(x)$, $f$ é par.
%Como 
%$$
%\lim_{x\to \pm \infty}\frac{x^2-1}{x^2+1}=\lim_{x\to \pm
%\infty}\frac{1-\tfrac{1}{x^2}}{1+\tfrac{1}{x^2}}=1\,,
%$$
%a reta $y=1$ é assíntota horizontal. Não tém assíntotas verticais.
%A derivada é dada por $f'(x)=\frac{4x}{(x^2+1)^2}$. Logo,
%\begin{center}
%\begin{bmlimage}\begin{tikzpicture}
%\tkzTabInit[nocadre,espcl=2,  color, colorV=lightgray!5, colorL=blue!15,
%colorC=blue!15]
%%{$x$ /.6,  $f'(x)$ /.9, Variação de $f$ /1.5}%
%{$x$ /.5,  $f'(x)$ /.5, Var. de $f$ /1}{,$0$,}
%%\tkzTabLine{,+,z,+,,+,}
%\tkzTabLine{,-,z,+,}
%\tkzTabVar{+/,-/\text{min.},+/,}
%%\tkzTabLine{,\searrow,\text{mín.},h,\text{mín.},\nearrow,}
%\end{tikzpicture}\end{bmlimage}
%\end{center}
%O mínimo tém coordenadas $(0,f(0))=(0,-1)$. A segunda derivada é dada por
%$f''(x)=\frac{4(1-3x^2)}{x^2+1}$, logo:
%\begin{center}
%\begin{bmlimage}\begin{tikzpicture}
%\tkzTabInit[nocadre,espcl=2,  color, colorV=lightgray!5, colorL=blue!15,
%colorC=blue!15]
%%{$x$ /.5,  $f''(x)$ /.7, Conc. de $f$ /1.3}
%{$x$ /.5,  $f''(x)$ /.5, Conc. de $f$ /1}
%{,$-1/\sqrt{3}$, $-1/\sqrt{3}$,}
%\tkzTabLine{,-,z,+,z,-,}
%\tkzTabLine{,{\frown},,\smile,,\frown,}
%\end{tikzpicture}\end{bmlimage}
%\end{center}
%Pontos de inflexão:
%$(\tfrac{-1}{\sqrt{3}},f(\tfrac{-1}{\sqrt{3}}))=(\tfrac{-1}{\sqrt{3}},-\tfrac{1}
%{2})$,
%$(\tfrac{+1}{\sqrt{3}},f(\tfrac{+1}{\sqrt{3}}))=(\tfrac{+1}{\sqrt{3}},-\tfrac{1}
%{2})$.
%\begin{center}
%\begin{bmlimage}\begin{tikzpicture}[scale=1.3]
%\draw [thick, domain=-4:4, samples=100] plot (\x,{(\x^2-1)/(\x^2+1)});
%\draw [>=latex, ->] (-4,0)--(4,0) node[right] {$x$};
%\draw [>=latex, ->] (0,-0.1)--(0,1.5) node[left] {$f(x)$};
%\draw [dotted] (-4,1)--(4,1) node[above left] {$y=1$};
%%\fill (1,0) circle (0.35mm);
%%\fill (-1,0) circle (0.35mm);
%\fill (0,-1) circle (0.35mm);
%\draw (0,-1) node[below] {$(0,-1)$};
%\fill (-0.577,-0.5) circle (0.35mm);
%\draw (-0.577,-0.6) node[left]{$(\tfrac{-1}{\sqrt{3}},-\half)$};
%\fill (+0.577,-0.5) circle (0.35mm);
%\draw (+0.577,-0.6) node[right]{$(\tfrac{+1}{\sqrt{3}},-\half)$};
%\end{tikzpicture}\end{bmlimage}
%\end{center}
%\eqref{itexoEstudA3}
%O domínio de  $f(x)=x(\ln x)^2$ é
%$D=(0,+\infty)$, e o seu sinal é: $f(x)\geq 0$ para todo $x\in D$.
%A função não é { par} nem { ímpar}.
%Como $\lim_{x\to \infty}f(x)=+\infty$, não tém assintota horizontal.
%Para ver se tém assíntota vertical em $x=0$, calculemos 
%$\lim_{x\to 0^+}f(x)=\lim_{x\to 0^+}\frac{(\ln x)^2}{1/x}$. Como ambas funções
%$(\ln x)^2$ e $1/x$ são deriváveis em $]0,1[$ e tendem a $+\infty$ quando $x\to
%0^+$, apliquemos a regra de B.H.:
%$$
%\lim_{x\to 0^+}\frac{(\ln x)^2}{1/x}=
%\lim_{x\to 0^+}\frac{2(\ln x)1/x}{-1/x^2}=
%-2\lim_{x\to 0^+}x\ln x\,.
%$$
%Usando a regra de B.H. de novo, pode ser mostrado que esse segundo limite é
%zero (ver Exemplo \ref{Ex:xlogxemzero}). Logo, $\lim_{x\to 0^+}f(x)=0$: não
%tém assíntota vertical em $x=0$.
%A derivada é dada por $f'(x)=\ln x(\ln x+2)$.
%\begin{center}
%\begin{bmlimage}\begin{tikzpicture}[scale=0.8]
%\tkzTabInit[nocadre, espcl=2,  color, colorV=lightgray!5, colorL=blue!15,
%colorC=blue!15]
%{$x$ /.6, $f'(x)$ /.6, Variaç. de $f$ /1.2}%
%{,$e^{-2}$, $1$, }%
%\tkzTabLine{,+,z,-,z,+}
%\tkzTabVar{-/,+/{máx.},-/{mín.},+/}
%%\tkzTabLine{,\searrow,\text{mín.},h,\text{mín.},\nearrow,}
%\end{tikzpicture}\end{bmlimage}
%\end{center}
%O máximo local está em
%$(e^{-2},f(e^{-2}))=(e^{-2},4e^{- 2})$, e o
%mínimo global em $(1,f(1))=(1,0)$.
%A {segunda derivada} de $f$ é dada por
%$f''(x)=\frac{2(\ln x+1)}{x}$.
%\begin{center}
%\begin{bmlimage}\begin{tikzpicture}[scale=0.8]
%\tkzTabInit[nocadre, espcl=2,  color, colorV=lightgray!5, colorL=blue!15,
%colorC=blue!15]
%{$x$ /.6, $f''(x)$ /.6, Conv. de $f$ /1.2}%
%{,$e^{-1}$, }%
%\tkzTabLine{,-,z,+,}
%\tkzTabLine{,\frown,,\smile,}
%\end{tikzpicture}\end{bmlimage}
%\end{center}
%Logo, $f$ é côncava em $(0,e^{-1})$, possui um ponto de inflexão em
%$(e^{-1},f(e^{-1}))=(e^{-1},e^{-1})$, e é convexa em $(e^{-1},+\infty)$.
%\begin{center}
%\begin{bmlimage}\begin{tikzpicture}[scale=1.3]
%\draw [thick, domain=0.001:2.5, samples=100] plot (\x,{\x*(ln(\x))^2});
% \draw [>=latex, ->] (0,0)--(2.5,0) node[right] {$x$};
% \draw [>=latex, ->] (0,-0.1)--(0,2);
%% \draw [dotted] (-4,1)--(4,1) node[above left] {Assíntota horiz.: $y=1$};
% \fill (1,0) circle (0.35mm);
% \draw (1,0) node[below] {$\scriptscriptstyle{(1,0)}$};
% \fill (0.367,0.367) circle (0.35mm);
% \draw[<-] (0.39,0.39)--(0.9,0.5) node[above]
%{$\scriptscriptstyle{(e^{-1},e^{-1})}$};
% \fill (0.1353,0.541) circle (0.35mm);
% \draw[<-] (0.14,0.58)--(0.9,1.5) node[above]
%{$\scriptscriptstyle{(e^{-2},4e^{-2})}$};
%\end{tikzpicture}\end{bmlimage}
%\end{center}
%Podemos também notar que $\lim_{x\to 0^+}f'(x)=+\infty$.
%\end{sol}
%\end{exo}
%
%
%\begin{exo}\label{Exo:EstudosBasicos}
%Faça um estudo completo das funções abaixo:
%\begin{multicols}{3}
%\begin{enumerate}
%\item\label{itEstBas1} $x+\frac{1}{x}$
%\item\label{itEstBas6} $x+\frac{1}{x^2}$
%\item\label{itEstBas9} $\frac{1}{x^2+1}$
%\item\label{itEstBas2} $\frac{x}{x^2-1}$
%\item\label{itEstBas3} $xe^{-x^2}$
%%\item\label{itEstBas5} $x^4-x^2$
%\item\label{itEstBas7} $\senh x$
%\item\label{itEstBas8} $\cosh x$
%\item\label{itEstBas8t} $\tanh x$
%\item\label{itEstBas13} $(x^3-1)/(x^3+1)$, 
%\item\label{itEstBas14} $\tfrac12\sen (2x)-\sen(x)$, 
%\item\label{itEstBas15} $x/\sqrt{x^2+1}$
%\item\label{itEstBas4} $\frac{(x+1)^3}{(x-1)^4}$
%\end{enumerate}
%\end{multicols}
%\begin{sol}
%OBS: Colocamos aqui somente um \emph{resumo} das soluções, na forma de um
%gráfico. Os detalhes são deixados para o leitor.
%
%\eqref{itEstBas1} 
%\begin{center}
%\begin{bmlimage}\begin{tikzpicture}[yscale=0.7]
%\draw [thick, domain=-3:-0.3, samples=100] plot (\x,{\x+1/\x});
%\draw [thick, domain=0.3:3, samples=100] plot (\x,{\x+1/\x});
% \draw [>=latex, ->] (-3,0)--(3,0) node[right] {$x$};
% \draw [>=latex, ->] (0,-3)--(0,3) node[left]{$x+\tfrac{1}{x}$};
% \fill (1,2) circle (0.45mm);
%  \draw (1,2) node[below] {$\scriptscriptstyle{(1,2)}$};
%  \fill (-1,-2) circle (0.45mm);
%  \draw (-1,-2) node[above] {$\scriptscriptstyle{(-1,-2)}$};
%\end{tikzpicture}\end{bmlimage}
%\end{center}
%
%
%\eqref{itEstBas6} 
%\begin{center}
%\begin{bmlimage}\begin{tikzpicture}[yscale=0.7]
%\draw [thick, domain=-3:-0.5, samples=100] plot
%(\x,{\x+1/(\x*\x)});
%\draw [thick, domain=0.6:3, samples=100] plot
%(\x,{\x+1/(\x*\x)})node[right]{$x+\tfrac{1}{x^2}$};
% \draw [>=latex, ->] (-3,0)--(3,0) node[right] {$x$};
% \draw [>=latex, ->] (0,-3)--(0,3);
% \fill (1.256,1.88) circle (0.45mm);
%  \draw (1.256,1.88) node[below]
%{$\scriptscriptstyle{(2^{1/3},2^{1/3}+2^{-2/3})}$};
%\draw (-1,0) node{$\shortmid$} node[above left]{$-1$};
%\end{tikzpicture}\end{bmlimage}
%\end{center}
%
%\eqref{itEstBas9} 
%\begin{center}
%\begin{bmlimage}\begin{tikzpicture}
%\draw [thick, domain=-3:3, samples=100] plot
%(\x,{1/(\x*\x+1)});
%\draw [>=latex, ->] (-3,0)--(3,0);
%\draw [>=latex, ->] (0,-0.5)--(0,1.5)node[right]{$\tfrac{1}{x^2+1}$};
%\fill (0.577,0.75) circle (0.45mm);
%\draw[<-] (0.6,0.8)--(1.3,1)
%node[right]{inflex: $(\tfrac{1}{\sqrt{3}},\tfrac34)$};
%\fill (-0.577,0.75) circle (0.45mm);
%\draw[<-] (-0.6,0.8)--(-1.3,1)
%node[left]{inflex: $(-\tfrac{1}{\sqrt{3}},\tfrac34)$};
%% {$\scriptscriptstyle{(2^{1/3},2^{1/3}+2^{-2/3})}$};
%% \draw (-1,0) node{$\shortmid$} node[above left]{$-1$};
%\end{tikzpicture}\end{bmlimage}
%\end{center}
%
%\eqref{itEstBas2} 
%\begin{center}
%\begin{bmlimage}\begin{tikzpicture}[yscale=0.7]
%\draw [>=latex, ->] (-3,0)--(3,0) node[right] {$x$};
%\draw [>=latex, ->] (0,-3)--(0,3) node[left]{$\frac{x}{x^2-1}$};
%\draw [thick, domain=-3:-1.2, samples=100] plot (\x,{\x/(\x^2-1)});
%\draw[dashed] (-1,-2.5)--(-1,2.5) node[below left]{$\scriptscriptstyle{x=-1}$};
%\draw [thick, domain=-0.8:0.8, samples=100] plot (\x,{\x/(\x^2-1)});
%\draw [thick, domain=1.2:3, samples=100] plot (\x,{\x/(\x^2-1)});
%\draw[dashed] (1,-2.5)node[right]{$\scriptscriptstyle{x=1}$}--(1,2.5) ;
%\end{tikzpicture}\end{bmlimage}
%\end{center}
%
%\eqref{itEstBas3}
%\begin{center}
%\begin{bmlimage}\begin{tikzpicture}
%\draw [>=latex, ->] (-3,0)--(3,0) node[right] {$x$};
%\draw [>=latex, ->] (0,-1)--(0,1) node[above]{$xe^{-x^2}$};
%\draw [thick, domain=-2.5:2.5, samples=100] plot (\x,{\x*exp(-\x*\x)});
%\fill (0.707,0.428) circle (0.45mm);
%  \draw[<-] (0.71,0.44)-- (0.9,1) node[right]
%{$\scriptstyle{(\tfrac{1}{\sqrt{2}},\tfrac{1}{\sqrt{2}}e^{-\tfrac12})}$};
%\fill (-0.707,-0.428) circle (0.45mm);
%  \draw[<-] (-0.71,-0.5)-- (-0.9,-1) node[left]
%{$\scriptstyle{(-\tfrac{1}{\sqrt{2}},-\tfrac{1}{\sqrt{2}}e^{-\tfrac12})}$};
%\draw[<-] (0.1,-0.1)--(0.5,-1.3) node[right]{pt. inflex. $\scriptstyle{(0,0)}$};
%\fill (1.225,0.273) circle (0.40mm);
%\fill (-1.225,-0.273) circle (0.40mm);
%\draw[<-] (1.225,0.24)--(1.5,-0.6)
%node[right]{pt. inflex.: $\scriptstyle{(\sqrt{3/2},f(\sqrt{3/2}))}$};
%\draw[<-] (-1.225,-0.24)--(-1.5,0.6)
%node[left]{pt. inflex.: $\scriptstyle{(-\sqrt{3/2},f(\sqrt{3/2}))}$};
%\end{tikzpicture}\end{bmlimage}
%\end{center}
%
%\eqref{itEstBas7}, \eqref{itEstBas8},
%\eqref{itEstBas8t}:
%\begin{center}
%\begin{bmlimage}\begin{tikzpicture}[scale=0.5]
%\draw [>=latex, ->] (0,-0.1)--(0,3);
%\pgfmathsetmacro{\a}{2};
%\draw [>=latex, ->] (-\a,0)--(\a,0);
%\draw [thick, domain=-\a:\a, samples=100] plot (\x,{(exp(\x)+exp(-\x))/2})
%node[right]{$\cosh x$};
%
%\begin{scope}[xshift=9cm, yshift=1cm]
%\draw [>=latex, ->] (0,-2)--(0,2);
%\pgfmathsetmacro{\a}{1.6};
%\draw [>=latex, ->] (-\a,0)--(\a,0);
%\draw [thick, domain=-\a:\a, samples=100] plot (\x,{(exp(\x)-exp(-\x))/2})
%node[right]{$\senh x$};
%\end{scope}
%
%\begin{scope}[xshift=18cm, yshift=1cm]
%\draw [>=latex, ->] (0,-1.5)--(0,1.5);
%\pgfmathsetmacro{\a}{3};
%\draw [>=latex, ->] (-\a,0)--(\a,0);
%\draw [thick, domain=-\a:\a, samples=100] plot
%(\x,{(exp(\x)-exp(-\x))/(exp(\x)+exp(-\x))})
%node[below right]{$\tanh x$};
%\draw[dashed] (0,1)--(\a,1) node[above]{$x=+1$};
%\draw[dashed] (0,-1)--(-\a,-1) node[below]{$x=-1$};
%\end{scope}
%
%\end{tikzpicture}\end{bmlimage}
%\end{center}
%
%\eqref{itEstBas13}
%\begin{center}
%\begin{bmlimage}\begin{tikzpicture}[yscale=0.7]
%\draw [>=latex, ->] (-3,0)--(3,0);
%\draw [>=latex, ->] (0,-3)--(0,3) node[right]{$\frac{x^3-1}{x^3+1}$};
%\draw [thick, domain=-3:-1.2, samples=100] plot (\x,{(\x^3-1)/(\x^3+1)});
%\draw [thick, domain=-0.8:3, samples=100] plot (\x,{(\x^3-1)/(\x^3+1)});
%\draw[dashed] (-1,-3)node[left]{$\scriptscriptstyle{x=-1}$}--(-1,3) ;
%\draw[dashed] (-3,1) node[below]{$\scriptscriptstyle{x=1}$}--(3,1) ;
%\fill (0,-1) circle (0.45mm);
%\fill (0.793, -0.3333) circle (0.45mm);
%\draw[<-] (0.1,-1.1)--(1,-3)node[right]{Pt. de inflexão e crítico: $(0,-1)$};
%\draw[<-] (0.8, -0.4)--(1.2,-1) node[right]{Pt. de inflexão:
%$(2^{1/3},f(2^{1/2}))$};
%\draw (1,0) node{$\shortmid$} node[above]{$1$};
%\end{tikzpicture}\end{bmlimage}
%\end{center}
%
%\eqref{itEstBas14}:
%\begin{center}
%\begin{bmlimage}\begin{tikzpicture}[yscale=0.7]
%\draw [>=latex, ->] (-3,0)--(3,0);
%\draw [>=latex, ->] (0,-3)--(0,3) node[right]{$\tfrac12\sen (2x)-\sen(x)$};
%\draw [thick, domain=0:2*pi, samples=100] plot (\x,{sin(2*\x r)-sin(\x r)});
%% \draw [thick, domain=-0.8:3, samples=100] plot (\x,{(\x^3-1)/(\x^3+1)});
%% \draw[dashed] (-1,-3)node[left]{$\scriptscriptstyle{x=-1}$}--(-1,3) ;
%% \draw[dashed] (-3,1) node[below]{$\scriptscriptstyle{x=1}$}--(3,1) ;
%% \fill (0,-1) circle (0.45mm);
%% \fill (0.793, -0.3333) circle (0.45mm);
%% \draw[<-] (0.1,-1.1)--(1,-3)node[right]{Pt. de inflexão e crítico: $(0,-1)$};
%% \draw[<-] (0.8, -0.4)--(1.2,-1) node[right]{Pt. de inflexão:
%% $(2^{1/3},f(2^{1/2}))$};
%% \draw (1,0) node{$\shortmid$} node[above]{$1$};
%\end{tikzpicture}\end{bmlimage}
%\end{center}
%
%
%\end{sol}
%\end{exo}
%
%
%\begin{exo}
%Faça um estudo completo das seguintes funções.
%\begin{multicols}{3}
%\begin{enumerate}
%\item $\ln |2-5x|$
%\item $\ln(e^{2x}-e^x+3)$
%\item $\ln(\ln x)$
%\item $\frac{\ln x}{\sqrt{x}}$
%\item $\frac{\ln x-2}{(\ln x)^2}$
%\item $\arcsen(2x^2-1)$
%\item (Gilcione, legal) $e^{-x}(x^2+2x)$.
%\item $\frac{\sqrt{x^2-1}}{x-2}$ 
%\item $(\ln x)^2+\ln x$.
%\end{enumerate}
%\end{multicols}
%(Pris dans \verb|exan_ln.pdf|, dans le dossier Coisasinternet,
%TRuc)
%\end{exo}
%
%\begin{exo} 
% Faça o esboço de uma função que tenha as 
%propriedades \ref{KKK1}-\ref{KKK4} abaixo: 
%\begin{enumerate}
%\item\label{KKK1} $f(0)=1$, $f'(0)=\half$,
%\item $x=-2$ e $x=2$ são assíntotas verticais,
%\item $y=1$ é assíntota horizontal,
%\item\label{KKK4} $f$ decresce no intervalo $[-3,-2[$, e cresce no intervalo
%$]2,3]$,
%\end{enumerate}
%\end{exo}
%
%
%\begin{exo}
%Mostre que $a^b=b^a$ n'a que deux solutions (Tikz pour l'impatient, page 61)
%\end{exo}


% !TeX spellcheck = pt_BR
% !TEX encoding = UTF-8 Unicode

\chapter{Integral}\label{CAP:Integral}

\ifdefined\updateans
% Only need to run once in a lifetime, when the file ans.tex needs to be updated.
\Writetofile{ans}{\protect\section*{Capítulo \ref{CAP:Integral}}}
\fi

O problema original e fundamental do \emph{cálculo integral} era
de \emph{calcular comprimentos, áreas, e volumes} de objetos geométricos no
plano ou no espaço, em particular de objetos 
mais gerais do que aqueles
considerados em geometria elementar que são retângulos,
triângulos, círculos (no plano), ou paralelepípedos, cones,
esferas (no espaço).\\

O maior avanço no cálculo integral 
veio com os trabalhos de Newton e Leibniz
no fim do século XVI, em que a noção de derivada tem papel fundamental.
Os métodos desenvolvidos por Newton e Leibniz
tornaram a integral uma ferramenta com inúmeras aplicações, bem além da 
geometria, em todas as áreas da ciência e da engenharia.
\\

Nesse capítulo introduziremos a noção de \emph{integral} para uma função $f$ de
uma variável real~\footnote{Integrais \emph{múltiplas} serão estudadas em
Cálculo III.} $x$, a partir da Seção \ref{Sec:IntRiemann}.
O \emph{Teorema Fundamental do Cálculo} 
(Teoremas \ref{Teo:TFC} e \ref{Teo:TFC2}) será provado na Seção
\ref{Sec:TeoremaFundamental}.

\section{Introdução}

\emph{Como calcular, em geral, a área de uma região limitada do plano?}
Para sermos um pouco mais específicos, faremos a mesma pergunta para áreas
delimitadas pelo gráfico de uma função. \emph{Dada uma função positiva
$f:[a,b]\to \bR$,
como calcular a área debaixo do seu gráfico, isto é, a área da região $R$,
delimitada pelo gráfico de $f$, pelo eixo $x$, e pelas retas $x=a$, $x=b$?}
\index{gráfico! área debaixo de um}
\begin{center}
\begin{bmlimage}\begin{tikzpicture}[scale=1]
%\clip (0.5,0.5) rectangle (1.5,1.5);
\newcommand{\funcao}[1]{(0.05*(#1)^3-0.2*(#1)^2+2)}
\fill[areagrafico] (1,0)--plot[domain=1:3](\x,{\funcao{\x}})--(3,0)--cycle;
\draw [dotted, domain=0.5:3.5] plot (\x,{\funcao{\x}});
\draw [thick, domain=1:3] plot (\x,{\funcao{\x}});
\draw (2,0.8) node {$R$};
\draw (1,-0.22) node{$a$};
\draw (3,-0.2) node{$b$};
\draw (4,1.8) node {$f(x)$};
\draw [>=latex, ->] (0.5,0)--(3.5,0) node[right] {$x$};
\draw [dashed] (3,-0.05)--(3,1.52);
\draw [dashed] (1,-0.05)--(1,{\funcao{1}});
\end{tikzpicture}\end{bmlimage}
\end{center}


Para as funções elementares a seguir, a resposta pode ser dada sem muito
esforço.
Por exemplo, se $f$ é constante, $f(x)=h>0$, $R$ é um retângulo, logo

\begin{center}
\begin{bmlimage}\begin{tikzpicture}[scale=0.9]
%\clip (0.5,0.5) rectangle (1.5,1.5);
\fill[areagrafico] (1,0)--plot[domain=1:3](\x,{1.5})--(3,0)--cycle;
\draw [domain=0.5:3.5] plot (\x,{1.5});
\draw (2,0.8) node {$R$};
%\draw (1,0) node {$\shortmid$};
\draw (1,-0.22) node{$a$};
%\draw (3,0) node {$\shortmid$};
\draw (3,-0.2) node{$b$};
\draw (3.8,1.5) node {$h$};
\draw [>=latex, ->] (0.5,0)--(3.5,0) node[right] {$x$};
\draw [dashed] (1,-0.05)--(1,1.5);
\draw [dashed] (3,-0.05)--(3,1.5);
\draw [>=latex, ->] (0.7,-0.2)--(0.7,1.9) ; 
%\draw[very thin, gray] (0,0) grid[step=1] (5,2);
\draw (5,0.8) node[right] {$\Rightarrow \,\text{área}(R)=\text{base}\times 
\text{altura}=(b-a)h$};
\end{tikzpicture}\end{bmlimage}
\end{center}

Por outro lado, se o gráfico de $f$ for uma reta, por exemplo $f(x)=mx$ 
com $m>0$, e se $0<a<b$, então $R$ é um trapézio, e a sua área pode ser
 escrita como a diferença das áreas de dois triângulos (lembre o
Exercício \ref{Exo:primeiraarea}):

\begin{center}
\begin{bmlimage}\begin{tikzpicture}[scale=0.9]
%\clip (0.5,0.5) rectangle (1.5,1.5);
\fill[areagrafico] (1,0)--plot[domain=1:3](\x,0.6*\x)--(3,0)--cycle;
\draw [dotted, domain=-0.1:3.2] plot (\x,{0.6*\x});
\draw [thick, domain=1:3] plot (\x,{0.6*\x});
\draw (2,0.5) node {$R$};
%\draw (1,0) node {$\shortmid$};
\draw (1,-0.22) node{$a$};
%\draw (3,0) node {$\shortmid$};
\draw (3,-0.2) node{$b$};
%\draw (3.8,1.5) node {$h$};
\draw [>=latex, ->] (0,0)--(3.5,0) node[right] {$x$};
\draw [dashed] (1,-0.05)--(1,0.6);
\draw [dashed] (3,-0.05)--(3,1.8);
\draw [>=latex, ->] (0,0)--(0,2) ; 
%\draw[thick, gray] (0,0) grid[step=1] (5,2); \draw[very thin, gray] (0,0) 
%grid[step=0.2] (5,2);
\draw[dotted]  (1,0.6)--(0,0.6) node[left]{$ma$};
\draw[dotted]  (3,1.8)--(0,1.8) node[left]{$mb$};
\draw (4.2,0.8) node[right] {$\Rightarrow \,\text{área}(R)=\half b\times
mb-\half
 a\times ma=\half m(b^2-a^2)$};
\end{tikzpicture}\end{bmlimage}
\end{center}

O nosso último exemplo ``simples'' 
será $f(x)=\sqrt{1-x^2}$, com $a=0$, $b=1$. Neste caso 
reconhecemos a região $R$ como 
a sendo o quarto do disco de raio $1$ centrado na origem, contido no primeiro
quadrante:
\index{disco}
\begin{center}
\begin{bmlimage}\begin{tikzpicture}[scale=0.8]
%\clip (0.5,0.5) rectangle (1.5,1.5);
%\fill[color=gray!30] (1,0)--plot[domain=1:3](\x,0.6*\x)--(3,0)--cycle;
%\draw [domain=-0.1:3.2] plot (\x,{0.6*\x});
%\draw (1,0) node {$\shortmid$};
\draw (0,-0.2) node{$0$};
%\draw (3,0) node {$\shortmid$};
\draw (2,-0.22) node{$1$};
%\draw (0,-1)--(0,0)--(-1,0) ;
\fill[areagrafico] (0,0)--(2,0) arc (0:90:2)--cycle;
\draw[dotted] (2,0) arc (0:360:2);
\draw (2,0) arc (0:90:2);
\draw (0.8,0.8) node {$R$};
%\draw (-0.7,-0.7) node {$\pi/2$};
%\draw (0,0) node {$\bullet$};
%\draw (3.8,1.5) node {$h$};
\draw [>=latex, ->] (0,0)--(2.5,0) node[right] {$x$};
%\draw [dashed] (1,-0.05)--(1,0.6);
%\draw [dashed] (3,-0.05)--(3,1.8);
\draw [>=latex, ->] (0,0)--(0,2.2) ; 
%\draw[thick, gray] (0,0) grid[step=1] (5,2); \draw[very thin, gray] (0,0) grid[step=0.2] (5,2);
%\draw[dotted]  (1,0.6)--(0,0.6) node[left]{$ma$};
%\draw[dotted]  (3,1.8)--(0,1.8) node[left]{$mb$};
\draw (3.2,1) node[right] {$\Rightarrow \,\text{área}(R)=\tfrac14 \times\pi
1^2=\tfrac{\pi}{4}$};
\end{tikzpicture}\end{bmlimage}
\end{center}

Consideremos agora $f(x)=1-x^2$, 
também com $a=0$, $b=1$:

\begin{center}
\begin{bmlimage}\begin{tikzpicture}[scale=1.5]
\fill[areagrafico] (0,1)--plot[domain=0:1](\x,{1-(\x)^2})--(0,0)--cycle;
\draw [dotted, domain=-0.5:1.1] plot (\x,{1-(\x)^2});
\draw [domain=0:1] plot (\x,{1-(\x)^2});
\draw (0,0) node[below]{$0$};
\draw (0,1) node[left]{$1$};
\draw (1,0) node[below]{$1$};
\draw (0.4,0.4) node {$R$};
\draw [>=latex, ->] (-0.5,0)--(1.3,0) node[right] {$x$};
\draw [>=latex, ->] (0,0)--(0,1.2); 
%\draw[thick, gray] (0,0) grid[step=1] (2,2); \draw[very thin, gray] (0,0) grid[step=0.2] (2,2);
%\draw[dotted]  (1,0.6)--(0,0.6) node[left]{$ma$};
%\draw[dotted]  (3,1.8)--(0,1.8) node[left]{$mb$};
\draw (1.7,0.5) node[right] {$\Rightarrow \,R=\,?$};
\end{tikzpicture}\end{bmlimage}
\end{center}

Apesar da função $f(x)=1-x^2$ ser elementar, não vemos um jeito simples
de decompor $R$ em um número finito de regiões simples do tipo
retângulo, triângulo, ou disco.\\

No entanto, o que pode ser feito 
é \emph{aproximar $R$ por regiões mais
simples}, a começar com retângulos~\footnote{Já encontramos esse tipo de
construção, mas com triângulos, no Exercício
\ref{Exo:DecomporCircemTriang}.}.
Começemos aproximando $R$ de maneira grosseira, usando uma região $R_2$
formada por dois retângulos, da seguinte maneira:
\index{aproximação! por retângulos}
\begin{center}
\begin{bmlimage}\begin{tikzpicture}[scale=1.8]
\pgfmathsetmacro{\numretangulos}{2}
\foreach \k in {1,...,\numretangulos} {
\pgfmathsetmacro{\cantinho}{\k/\numretangulos}
\pgfmathsetmacro{\altura}{1-((\k-1)/\numretangulos)^2}
\fill[corretangulos] (\cantinho-1/\numretangulos,0) rectangle (\cantinho,\altura);
\draw (\cantinho-1/\numretangulos,0) rectangle (\cantinho,\altura);
\fill (\cantinho-1/\numretangulos,\altura) circle (0.15mm);
}
\draw [dotted, domain=0:1] plot (\x,{1-(\x)^2});
\draw (0,0) node[below]{$0$};
\draw (1,0) node[below]{$1$};
\draw (0.5,0) node[below]{$\half$};
\draw[dotted]  (0.5,0.75)--(0,0.75) node[left]{$1-(\half)^2=\tfrac34$};
\draw [>=latex, ->] (0,0)--(1.3,0) node[right] {$x$};
\draw [>=latex, ->] (0,0)--(0,1.2); 
\draw (1.7,0.5) node[right] {$\Rightarrow \,
\text{área}(R_2)=\bigl\{\half\times
1\big\}+\bigl\{\half \times \tfrac34\big\}=\tfrac78$};
\end{tikzpicture}\end{bmlimage}
\end{center}

A área de $R_2$ é a soma das áreas dos dois retângulos de bases
iguais $\tfrac12$ mas de alturas diferentes: 
o canto esquerdo superior do primeiro retângulo está em $(0,1)$, e o do segundo
foi escolhido \emph{no gráfico de $1-x^2$}, no ponto $(\tfrac12,\tfrac34)$.
Logo, $\text{área}(R_2)=\tfrac78$.
É claro que $\text{área}R_2$ somente dá uma \emph{estimativa}:
$\text{área}(R)<\text{área}R_2$.\\

Tentaremos agora melhorar 
essa aproximação: fixemos um inteiro $n\in \bN$, e
aproximemos
$R$ pela região $R_n$ formada pela união de $n$ retângulos de larguras iguais a
$1/n$, mas com alturas 
escolhidas tais que o canto superior esquerdo esteja sempre \emph{na curva}
$1-x^2$. Por exemplo, 
se $n=5$, ${15}$ e ${25}$,

\begin{center}
\begin{bmlimage}\begin{tikzpicture}[scale=1.8]

\newcommand{\parabole}[1]{
\pgfmathsetmacro{\numretangulos}{#1}
\foreach \k in {1,...,\numretangulos}
{\pgfmathsetmacro{\cantinho}{\k/\numretangulos}
\pgfmathsetmacro{\altura}{1-((\k-1)/\numretangulos)^2}
\fill[corretangulos] (\cantinho-1/\numretangulos,0) rectangle
(\cantinho,\altura);
\draw (\cantinho-1/\numretangulos,0) rectangle (\cantinho,\altura);
\fill (\cantinho-1/\numretangulos,\altura) circle (0.15mm);
}
\draw [dotted, domain=0:1] plot (\x,{1-(\x)^2});
%\draw (0,1) node[left]{$1$};
%\draw (0,0) node[below]{$0$};
%\draw (1,0) node[below]{$1$};
\draw [>=latex, ->] (0,0)--(1.1,0);
\draw [>=latex, ->] (0,0)--(0,1.2);
}

\begin{scope}
 \parabole{5}
\end{scope}

\begin{scope}[xshift=2cm]
 \parabole{15}
\end{scope}

\begin{scope}[xshift=4cm]
 \parabole{25}
\end{scope}

\end{tikzpicture}\end{bmlimage}
\end{center}

Vemos que quanto maior o número de retângulos $n$, melhor a aproximação da
verdadeira área de $R$.
Logo, tentaremos calcular $\text{área}(R)$ via um \emph{limite}:
$$\text{área}(R)=\lim_{n\to \infty}\text{área}(R_n)\,.$$
Olhemos os retângulos de mais perto. Por exemplo, para calcular
$\text{área}(R_5)$, calculemos a soma das áreas de $5$ retângulos:
\begin{align*}
\text{área}(R_5)&=\tfrac15\big(1-(\tfrac{0}{5})^2)
+\tfrac15\big(1-(\tfrac{1}{5})^2)
+\tfrac15\big(1-(\tfrac{2}{5})^2)
+\tfrac15\big(1-(\tfrac{3}{5})^2)
+\tfrac15\big(1-(\tfrac{4}{5})^2)\\
&=1-\tfrac{1^2+2^2+3^2+4^2}{5^3}(=0.76)\,.
\end{align*}

Para um $n$ qualquer,
\begin{align}
\text{área}(R_n)&=\tfrac1n\big(1-(\tfrac{0}{n})^2)
+\tfrac1n\big(1-(\tfrac{1}{n})^2)+\dots
+\tfrac1n\big(1-(\tfrac{n-2}{n})^2)
+\tfrac1n\big(1-(\tfrac{n-1}{n})^2)\nonumber\\
&=1-\tfrac{1^2+2^2+\dots+(n-2)^2+(n-1)^2}{n^3}\,.\label{eq:somaquadrados}
\end{align}

Pode ser mostrado (ver Exercício \ref{exo:provaporinducao}) que para todo
$k\geq 1$,
\eq{\label{eq:inducnnn}1^2+2^2+\dots+k^2=\frac{k(k+1)(2k+1)}{6}\,.}
Usando essa expressão em \eqref{eq:somaquadrados} com $k=n-1$, obtemos
\begin{align*}
\text{área}(R)=\lim_{n\to \infty}\text{área}(R_n)&=1-\lim_{n\to \infty}
\frac{(n-1)((n-1)+1)(2(n-1)+1)}{6n^3}\\
&=1-\lim_{n\to \infty}
\frac{n(n-1)(2n-1)}{6n^3}\\
&=1-\tfrac{1}{3}\\
&=\tfrac{2}{3}\,.
\end{align*}

\begin{obs}
É interessante observar que no limite $n\to\infty$, o número de retângulos que
aproxima $R$ tende ao infinito,
mas que a área de cada um tende a zero. Assim podemos dizer, informalmente, que
depois do processo de limite, a área exata de $R$
é obtida ``somando infinitos retângulos de largura zero''.
\end{obs}


\begin{exo}\label{exo:provaporinducao}
Mostre por indução que para todo $n\geq 1$, 
$$1+2+3+\dots+n=\frac{n(n+1)}{2}\,,\quad
1^2+2^2+\dots+n^2=\frac{n(n+1)(2n+1)}{6}\,.$$
\end{exo}

\begin{exo}
Considere a aproximação da área $R$ tratada acima, usando retângulos cujo canto
superior \emph{direito} sempre fica na curva $y=1-x^2$, e mostre que quando 
$n\to\infty$, o limite é o mesmo: $\tfrac23$.
\end{exo}


O método usado para calcular a área debaixo de $1-x^2$ 
funcionou graças à fórmula \eqref{eq:inducnnn}, que
permitiu transformar a soma dos $k$ primeiros quadrados em um
polinômio de grau $3$ em $k$. Essa fórmula foi particularmente bem adaptada à
função $1-x^2$, mas não será útil em outras situações. 
Na verdade, são poucos casos em que a conta pode ser feita ne maneira
explícita.

\begin{ex}
Considere $f(x)=\cos(x)$ entre $a=0$ e $b=\pi/2$.
\begin{center}
\begin{bmlimage}\begin{tikzpicture}[scale=1.5]
\fill[areagrafico] (0,1)--plot[domain=0:1.57](\x,{cos(\x r)})--(0,0)--cycle;
\draw [dotted, domain=0:1.57] plot (\x,{cos(\x r)});
\draw [domain=-0.1:1.59] plot (\x,{cos(\x r)});
\draw (0,0) node[below]{$0$};
\draw (1.57,0) node[below]{$\tfrac{\pi}{2}$};
\draw (0.6,0.4) node {$R$};
\draw [>=latex, ->] (-0.1,0)--(1.7,0) node[right] {$x$};
\draw [>=latex, ->] (0,0)--(0,1.2); 
\end{tikzpicture}\end{bmlimage}
\end{center}
Neste caso, uma aproximação da área $R$ debaixo do gráfico por 
retângulos de largura $\tfrac1n$ dá:
\begin{align}
\text{área}(R_n)&=\tfrac1n\cos(\tfrac{1}{n})
+\tfrac1n\cos(\tfrac{2}{n})+\dots
+\tfrac1n\cos(\tfrac{\frac{n\pi}{2}}{n})\,.
\end{align}
Para calcular o limite $n\to\infty$ desta soma, o leitor interessado 
pode começar verificando por indução~\footnote{Fonte: Folhetim de
Educação Matemática, Feira de Santana, Ano 18, Número 166, junho de
2012.} que para todo $a>0$ e todo inteiro $k$,
\[
\tfrac12+\cos(a)+\cos(2a)+\cos(3a)+\dots+\cos(ka)=\frac{\sen(\frac{2k+1}{2}a)}{2\sen(\frac{a}{2})}\,.
\]
Usando esta fórmula com $a$ e $n$ bem escolhidos, pode 
mostrar que  $\lim_{n\to \infty}\text{área}(R_n)=1$. Portanto,
$\text{área}(R)=1$.
\end{ex}

\begin{exo}
Considere $f(x)=e^x$ entre $a=0$ e $b=1$.
Monte $\text{área}(R_n)$ usando retângulos de largura $\tfrac1n$.
Usando
\[
1+r+r^2+\dots+r^n=\frac{1-r^n}{1-r}\,,
\]
calcule $\lim_{n\to\infty}\text{área}(R_n)$.
\begin{sol}
A soma associada dá, usando a fórmula sugerida,
\[
\text{área}(R_n)=\frac{e^0}{n}+\frac{e^{1/n}}{n}
+\frac{e^{2/n}}{n}+\dots+\frac{e^{(n-1)/n}}{n}
=\frac{e-1}{\frac{e^{1/n}-1}{1/n}}\,.
\]
Mas $\lim_{n\to\infty}\frac{e^{1/n}-1}{1/n}=\lim_{t\to
0^+}\frac{e^t-1}{t}=1$. Logo,
$\text{área}(R)=e-1$.
\end{sol}
\end{exo}

O que foi feito nesses últimos exemplos foi calcular uma área por um
procedimento chamado \emph{integração}. 
Mais tarde, desenvolveremos um método que permite calcular
integrais usando um método completamente diferente. Mas
antes disso precisamos definir o que significa \emph{integrar} de
maneira mais geral.

\section{A integral de Riemann}\label{Sec:IntRiemann}

De modo geral, a área da região $R$ delimitada pelo gráfico de uma função
$f:[a,b]\to \bR$ pode ser definida via um processo de limite,
como visto acima no caso de $f(x)=1-x^2$.\\

Primeiro, 
escolhemos um inteiro $n$, e escolhemos pontos distintos em $(a,b)$:
$x_0\equiv a<x_1<x_2<\dots<x_{n-1}<x_n\equiv b$. 
Esses pontos formam uma \grasA{partição} de $[a,b]$.
Em seguida, escolhemos um
ponto $x_j^*$ em cada intervalo $[x_{j-1},x_{j}]$, e definimos a \grasA{soma
de \index{Riemann (Georg Friedrich)} 
Riemann~\footnote{Georg Friedrich Bernhard Riemann, 1826 – 1866.}} $I_n$ por:
\begin{center}
 \begin{bmlimage}\begin{tikzpicture}[scale=3]
\newcommand{\funcao}[1]{( 0.5+ ((#1)^2-(#1)^3)/2 )}
\pgfmathsetmacro{\numintervalos}{15}
\pgfmathsetmacro{\a}{-0.6}
\pgfmathsetmacro{\b}{1.2}
\pgfmathsetmacro{\incr}{(\b-\a)/\numintervalos}
\foreach \k in {1,...,\numintervalos}
{\pgfmathsetmacro{\cantinho}{\a+(\k-1)*\incr}
\pgfmathsetmacro{\pontoaleat}{\cantinho+(rnd*\incr)}
\pgfmathsetmacro{\altura}{\funcao{\pontoaleat}}
\fill[corretangulos] (\cantinho,0) rectangle (\cantinho+\incr,\altura);
\draw (\cantinho,0) rectangle (\cantinho+\incr,\altura);
\draw[dotted] (\pontoaleat,0)--(\pontoaleat,\altura);
\fill (\pontoaleat,\altura) circle (0.15mm);
}
\draw (\a,0) node{$\shortmid$} node[below] {$a$};
\draw (\b,0) node{$\shortmid$} node[below] {$b$};
\draw (\a-0.2,0)--(\b+0.2,0) ;
\draw [thick, domain=\a-0.1:\b+0.1] plot (\x,{\funcao{\x}}) node[right]{$f$};
\draw (-1,0.3) node[left]{$\displaystyle{I_n\pardef
\sum_{j=1}^nf(x_j^*)\Delta x_{j}}\,,$};
 \end{tikzpicture}\end{bmlimage}
\end{center}
$I_n$ aproxima a área debaixo do gráfico pela soma das áreas dos retângulos, em
que o $j$-ésimo retângulo tem como base $\Delta x_j\pardef x_{j}-x_{j-1}$, e
como altura \emph{o valor da função no ponto $x_j^*$}: $f(x_j^*)$. (Na imagem
acima os pontos $x_i$ foram escolhidos equidistantes,
$\Delta x_{j}=\tfrac{b-a}{n}$.)\\

A \emph{integral} de $f$ é obtida considerando $I_n$ para uma sequência de
partições em que o tamanho dos intervalos $\Delta x_j$
tendem a zero:
\begin{defin}\index{função! integrável}
A função $f:[a,b]\to \bR$ é \grasA{integrável} se o limite $\lim_{n\to
\infty}I_n$ existir, qualquer que seja a 
sequência de partições em que $\max_j\Delta x_j\to 0$, e
qualquer que seja a escolha de $x_j^*\in [x_{j-1},x_{j}]$. Quando $f$ é
integrável, o limite $\lim_{n\to \infty}I_n$
é chamado de \grasA{integral (de Riemann) de $f$}, ou \grasA{integral definida
de $f$}, e denotado\index{integral! de Riemann}
\eq{\label{eq:DefinIntRiem}\lim_{n\to \infty}I_n\equiv \int_a^bf(x)dx\,.}
Os números $a$ e $b$ são chamados de \grasA{limites de integração}.
\end{defin}\index{limite! de integração}

Inventada por Newton, 
a notação  ``$\int_a^bf(x)dx$'' lembra que a integral é definida a partir de
uma
\emph{soma} (o ``$\int$'' é parecido com um ``s'') de retângulos
contidos entre $a$ e $b$, de áreas $f(x^*_j)\Delta x_j$ (o
``$f(x)dx$''). 

\begin{obs}
É importante lembrar que $\int_a^bf(x)dx$ \emph{é um número, não uma função}:
a variável ``$x$'' que aparece em $\int_a^bf(x)dx$ é usada somente
para indicar que $f$ está sendo integrada, com a sua variável
varrendo o intervalo $[a,b]$. 
Logo, seria equivalente escrever essa
integral $\int_a^bf(t)dt$, $\int_a^bf(z)dz$, etc., ou simplesmente $\int_a^bf\,
dx$. Por isso, a variável $x$ que aparece em \eqref{eq:DefinIntRiem} é
\index{variável! muda}
chamada de \emph{muda}.
\end{obs}

\begin{obs} A definição de integrabilidade faz sentido mesmo se $f$ não é
positiva.
Neste caso, o termo $f(x_j^*)\Delta x_{j}$ da soma de Riemann não pode ser mais
interpretado
como a área do $j$-ésimo retângulo, e $\int_a^bf\,dx$
não possui necessariamente uma interpretação geométrica.
O Exercício \ref{Exo:IntegraleNegative} abaixo esclarece esse ponto.
\end{obs}

Enunciemos algumas propriedades básicas da integral, que podem ser provadas a
partir da definição.
\index{integral! propriedades da}
\begin{pro}\label{Prop:ProprIntegral} Seja $f:[a,b]\to \bR$ integrável.
\begin{enumerate}
 \item\label{itProprIntegr1} Se $\lambda\in \bR$ é uma
 constante, então $\lambda f$ é
integrável, e $\int_a^b\lambda f\, dx=\lambda\int_a^bf\,dx$.
\item\label{itProprIntegr2} Se $g:[a,b]\to \bR$ também é integrável, então 
$f+g$ é integrável e
$\int_a^b(f+g)dx=\int_a^bf\,dx+\int_a^bg\,dx$.
\item\label{itProprIntegr3} Se $a<c<b$, então
$\int_a^cf\,dx+\int_c^bf\,dx=\int_a^bf\,dx$.
\end{enumerate}
\end{pro}

Observe que se $f$ é uma constante, $f(x)=c$, então qualquer soma de Riemann
pode ser calculada via um retângulo só, e
\eq{\label{eq:integrconstante} \int_a^bf(x)\,dx=c(b-a)\,.}
Mais tarde precisaremos da seguinte propriedade:

\begin{pro} Se $f$ e $g:[a,b]\to \bR$ são integráveis,
e se $f\leq g$, então 
 \eq{\label{eq:comparacaointegrais}\int_a^bf\,dx\leq \int_a^bg\,dx\,.}
Em particular, se $f$ é limitada,
$M_-\leq f(x)\leq M_+$ para todo $x\in [a,b]$, então 
\eq{\label{ineq_estim_int}M_-(b-a)\leq \int_a^bf\,dx\leq M_+(b-a)\,.}
\end{pro}
Para funções positivas, a interpretação de \eqref{eq:comparacaointegrais} em
termos de áreas é imediata: se o gráfico de $f$ está sempre abaixo do gráfico
de $g$, então a área debaixo de $f$ é menor do que a área abaixo de $g$.

\begin{exo}
Justifique as seguintes afirmações:
\begin{enumerate}
\item Se $f$ é par, $\int_{-a}^af(x)\,dx=2\int_0^af(x)\,dx$.
\item Se $f$ é ímpar, $\int_{-a}^af(x)\,dx=0$.
\end{enumerate}
\end{exo}

Em geral, verificar se uma função é integrável pode ser difícil. O seguinte
resultado garante que
as maioria das funções consideradas no restante do
curso \emph{são} integráveis.

\begin{teo}
 Se $f:[a,b] \to \bR$ é contínua, então ela é integrável.
\end{teo}

Por exemplo, $f(x)=1-x^2$ é contínua, logo integrável, e vimos na
introdução que
$$\int_0^1(1-x^2)dx=\tfrac23\,.$$
Sabendo que uma função contínua é integrável, queremos um jeito de 
\emph{calcular} a sua integral.
Mas como já foi dito, o procedimento de limite descrito acima (calcular a soma
de Riemann, tomar o limite $n\to \infty$, etc.) é díficil de
se implementar, mesmo se $f$ é simples.

\section{O Teorema Fundamental do Cálculo}\label{Sec:TeoremaFundamental}

Suponha que se queira calcular a integral de uma função
contínua $f:[a,b]\to \bR$: 
\begin{center}
\begin{bmlimage}\begin{tikzpicture}[scale=0.9]
\newcommand{\funcao}[1]{( 2- (0.2*( ( (#1) -1.4))^2))}
\fill[areagrafico]
(0.5,0)--plot[domain=0.5:3.5](\x,{\funcao{\x}})--(3.5,0)--cycle;
\draw [dotted, domain=0:4] plot (\x,{\funcao{\x}});
\draw [thick, domain=0.5:3.5] plot (\x,{\funcao{\x}});
\draw[dotted] (0.5,-0.1)--(0.5,{\funcao{0.5}});
 \draw (0.5,0) node[below]{$a$};
\draw[dotted] (3.5,-0.1)--(3.5,{\funcao{3.5}});
 \draw (3.5,0) node[below]{$b$};
\draw[>=latex, ->] (0,0)--(4.3,0);
 \draw (0,1.7) node[left]{$f(x)$};
\draw (6,1) node[right]{$\displaystyle{I= \int_a^bf(t)dt\,.}$};
\end{tikzpicture}\end{bmlimage}
\end{center}
Podemos supor sem perda de generalidade que $f\geq
0$, o que deve ajudar a entender geometricamente alguns dos raciocínios a
seguir. Para calcular $I$ passaremos pelo estudo de uma função auxiliar,
chamada de \grasA{função área}, definida da seguinte maneira:
\index{função área}
\begin{center}
\begin{bmlimage}\begin{tikzpicture}[scale=0.9]
\newcommand{\funcao}[1]{( 2- (0.2*( ( (#1) -1.4))^2))}
%\fill[areagrafico]
%(0.5,0)--plot[domain=0.5:3.5](\x,{\funcao{\x}})--(3.5,0)--cycle;
\fill[areafuncaoarea]
(0.5,0)--plot[domain=0.5:2.5](\x,{\funcao{\x}})--(2.5,
0)--cycle;
\draw [dotted, domain=0:4] plot (\x,{\funcao{\x}});
\draw [thick, domain=0.5:3.5] plot (\x,{\funcao{\x}});
\draw[dotted] (0.5,-0.1)--(0.5,{\funcao{0.5}});
 \draw (0.5,0) node[below]{$a$};
\draw[dotted] (3.5,-0.1)--(3.5,{\funcao{3.5}});
 \draw (3.5,0) node[below]{$b$};
\draw (2.5,-0.1)--(2.5,{\funcao{2.5}});
 \draw (2.5,0) node[below]{$x$};
\draw[>=latex, ->] (0,0)--(4.3,0);
 \draw (0,1.7) node[left]{$f(x)$};
 \draw (1.5,1) node{$I(x)$};
\draw (6,1) node[right]{$\displaystyle{I(x)\pardef \int_a^xf(t)dt\,.}$};
\end{tikzpicture}\end{bmlimage}
\end{center}
Isto é, $I(x)$ representa a área debaixo do gráfico de $f$,
entre as retas verticais em $a$ (fixa) e em $x$ (móvel).
Como $f$ é positiva, $x\mapsto I(x)$ é crescente.
Além disso, $I(a)=0$, e a integral original procurada é
$I(b)\equiv I$.

\begin{ex}
Se $f(x)=mx$, a função área pode ser calculada explicitamente:
\begin{center}
\begin{bmlimage}\begin{tikzpicture}[scale=1]
\pgfmathsetmacro{\m}{0.6};
\fill[areafuncaoarea] (1,0)--plot[domain=1:2.5](\x,\m*\x)--(2.5,0)--cycle;
\draw [dashed, domain=-0.1:3.2] plot (\x,{\m*\x});
\draw [thick, domain=1:2.5] plot (\x,{\m*\x});
\draw (1.8,0.5) node {$I(x)$};
%\draw (1,0) node {$\shortmid$};
\draw (1,-0.22) node{$a$};
%\draw (3,0) node {$\shortmid$};
\draw (3,-0.2) node{$b$};
\draw (2.5,\m*2.5)--(2.5,0) node[below]{$x$};
\draw [>=latex, ->] (-0.5,0)--(3.5,0);
\draw [>=latex, ->] (0,-0.5,0)--(0,2);
\draw [dotted] (1,-0.05)--(1,0.6);
\draw [dotted] (3,-0.05)--(3,1.8);
\draw (4.2,0.8) node[right] {$I(x)=\half m(x^2-a^2)$};
\end{tikzpicture}\end{bmlimage}
\end{center}
Podemos observar que 
$$I'(x)=\bigl(\tfrac12 m(x^2-a^2)\bigr)'=mx\equiv f(x)\,!$$
\end{ex}

\begin{exo} Calcule as funções área associadas às 
funções $f:[0,1]\to \bR$ abaixo.
\begin{multicols}{3}
\begin{enumerate}
\item\label{itExFuncArea1} $\displaystyle{f(x)=
\begin{cases}
 0&\text{ se }x\leq \frac12\,,\\
 1&\text{ se }x> \frac12\,.
\end{cases}
}$
%$f(x)=0$ se $x\leq \frac12$, $1$ se $x>\frac12$
\item\label{itExFuncArea2} $f(x)=-x+1$
\item\label{itExFuncArea3} $f(x)=2x-1$
\end{enumerate}
\end{multicols}
\vspace{0.01cm}
\begin{sol}
\eqref{itExFuncArea1} $I(x)=0$ se $x\leq \frac12$, $I(x)=(x-\frac12)$ se
$x>\frac12$
\eqref{itExFuncArea2} $I(x)=-\frac{x^2}{2}+x$
\eqref{itExFuncArea3} $I(x)=x^2-x$.
\end{sol}
\end{exo}

A relação entre $I$ e $f$ é surpreendentemente simples:
\index{Teorema Fundamental do Cálculo|textbf}
\index{função!contínua}
\begin{teo}[Teorema Fundamental do Cálculo]\label{Teo:TFC}
 Seja $f:[a,b]\to \bR$ contínua. Então a função área $I:[a,b]\to \bR$,
definida por $I(x)\pardef \int_a^xf(t)dt$
é derivável em todo $x\in (a,b)$, e a sua derivada é igual a $f$:
\eq{I'(x)=f(x)\,.}
\end{teo}
O seguinte desenho deve ajudar a entender a prova:
\begin{center}
\begin{bmlimage}\begin{tikzpicture}[scale=2]
\newcommand{\funcao}[1]{ 2- (0.2*( ( (#1) -1.4))^2)}
\pgfmathsetmacro{\e}{1.5};
\pgfmathsetmacro{\c}{2.5};
\pgfmathsetmacro{\a}{2.9};
\fill[areafuncaoarea]
(\e-0.2,0)--plot[domain=\e-0.2:\a](\x,{\funcao{\x}})--(\a,
0)--cycle;
\draw[dotted] (\c,0) rectangle (\a,{\funcao{\c}});
\draw [thick, domain=\e-0.4:\a+0.2] plot (\x,{\funcao{\x}});
\draw[dashed] (\c,0)--(\c,{\funcao{\c}});
\draw[decorate, decoration=brace] (\c,0)--(\c,{\funcao{\c}})
node[midway, left]{$f(x)$};
\draw[dashed] (\a,0)--(\a,{\funcao{\a}});
\draw (\c,0) node[below]{$\scriptstyle{x}$};
\draw (\a,0) node[below]{$\scriptstyle{x+h}$};
\draw[decorate, decoration=brace]
(\c,{\funcao{\c}})--(\a,{\funcao{\c}})node[above, midway]{$h$};
\draw[>=latex, ->] ({\e-0.4},0)--({\a+0.5},0);
\draw (4,1) node[right]{$\displaystyle{\Rightarrow\, I(x+h)\simeq I(x)+f(x)\cdot
h}$};
\draw (4,0.3) node[right]{$\displaystyle{\Rightarrow\,
\frac{I(x+h)-I(x)}{h}\simeq f(x)}$};
\end{tikzpicture}\end{bmlimage}
\end{center}
De fato, entre $x$ e $x+h$, a função área $I$ cresce de uma
quantidade que pode ser aproximada, quando $h>0$ é pequeno, 
pela área do retângulo pontilhado, cuja base é $h$ e altura $f(x)$.
Isso sugere 
\eq{\label{eq:letrucmuch}\lim_{h\to 0^+}\frac{I(x+h)-I(x)}{h}=f(x)\,.}
\begin{proof}
Seja $x\in (a,b)$. Provemos \eqref{eq:letrucmuch} 
(o limite $h\to 0^-$ se trata da mesma maneira).
Pela propriedade \eqref{itProprIntegr3} da Proposição \ref{Prop:ProprIntegral},
$$I(x+h)=\int_a^{x+h}f(t)\,dt=\int_a^xf(t)\,dt+\int_{x}^{x+h}f(t)\,dt=
I(x)+\int_{x}^{x+h}f(t)\,dt\,.$$
Observe também que por \eqref{eq:integrconstante}, $f(x)$ pode
ser escrito como a diferença
$f(x)=\frac{1}{h}f(x)\int_x^{x+h}\,dt=\frac{1}{h}\int_x^{x+h}f(x)\,dt$.
Logo, \eqref{eq:letrucmuch} é equivalente a mostrar que
\eq{\label{eq:maaachin}
\frac{I(x+h)-I(x)}{h}-f(x)=\tfrac{1}{h}\int_x^{x+h}(f(t)-f(x))dt}
tende a zero quando $h\to 0$.
Como $f$ é contínua em $x$, sabemos que para todo $\epsilon>0$,
$-\epsilon\leq f(t)-f(x)\leq +\epsilon$, desde que $t$ seja suficientemente
perto de $x$.
Logo, para $h>0$ suficientemente pequeno, a integral em
\eqref{eq:maaachin} pode ser limitada por 
$$
-\epsilon =\tfrac1h\int_x^{x+h}(-\epsilon)\,dt
\leq \tfrac{1}{h}\int_x^{x+h}(f(t)-f(x))dt\leq 
\tfrac1h\int_x^{x+h}(+\epsilon)\,dt=+\epsilon\,.
$$
(Usamos \eqref{ineq_estim_int}.)
Isso mostra que \eqref{eq:maaachin} fica arbitrariamente pequeno  quando $h\to
0^+$, o que prova \eqref{eq:letrucmuch}.
\end{proof}

Assim, provamos que integral e derivada são duas noções intimamente ligadas, já
que a função área é \emph{uma função derivável cuja
derivada é igual a $f$}. 

\begin{defin}
Seja $f$ uma função. Se $F$ é uma função derivável tal que $$F'(x)=f(x)$$ 
para todo $x$, então $F$ é chamada \grasA{primitiva de} $f$.
\index{primitiva|textbf}
\end{defin}

\begin{ex} Se $f(x)=x$, então
$F(x)=\frac{x^2}{2}$ é primitiva de $f$, já que
$$F'(x)=\bigl(\frac{x^2}{2}\bigr)'=\tfrac12(x^2)'=\tfrac12 2x=x\,.$$
Observe que como $(\frac{x^2}{2}+1)'=x$,
$G(x)=\frac{x^2}{2}+1$ é \emph{também} primitiva de $f$.
\end{ex}

\begin{ex}
Se $f(x)=\cos x$, então $F(x)=\sen x$ é primitiva de $f$. Observe que
$G(x)=\sen x+14$ e $H(x)=\sen x-7$ também são primitivas de $f$.
\end{ex}

Os dois exemplos acima mostram que \emph{uma função admite infinitas
primitivas}, e que aparentemente duas primitivas de uma mesma função somente
diferem por uma constante:

\begin{lem}
 Se $F$ e $G$ são duas primitivas de uma mesma função $f$, então existe uma
constante $C$ tal que $F(x)-G(x)=C$ para todo $x$.
\end{lem}
\begin{proof}
Defina $m(x)\pardef F(x)-G(x)$. Como $F'(x)=f(x)$ e $G'(x)=f(x)$, temos 
$m'(x)=0$ para todo
$x$. Considere dois pontos $x_1<x_2$ quaisquer. Aplicando o Corólário
\eqref{Corol:ValorIntermDeriv} a $m$ no intervalo $[x_1,x_2]$: existe $c\in
[x_1,x_2]$ tal que $\frac{m(x_2)-m(x_1)}{x_2-x_1}=m'(c)$. Como $m'(c)=0$, temos
$m(x_2)=m(x_1)$. Como isso pode ser feito para qualquer ponto $x_2<x_1$, temos
que $m$ toma o mesmo valor em qualquer ponto, o que implica que é uma função
constante.
\end{proof}

Em geral, escreveremos uma primitiva genérica de $f(x)$ como
$$F(x)=\text{primitiva}+C\,,$$ 
para indicar que é sempre possível adicionar uma constante $C$ arbitrária.

\begin{exo}\label{Exo:PrimitivasBasicas}
Ache as primitivas das funções abaixo.
\begin{multicols}{4}
\begin{enumerate}
\item\label{itExoPrimitTriv0} $-2$
\item\label{itExoPrimitTriv1} $x$
\item\label{itExoPrimitTriv2} $x^2$
\item\label{itExoPrimitTriv3} $x^n$ ($n\neq -1$)
\item\label{itExoPrimitTriv35} $\sqrt{1+x}$
\item\label{itExoPrimitTriv5} $\cos x$
\item\label{itExoPrimitTriv6} $\sen x$
\item\label{itExoPrimitTriv7} $\cos (2x)$
\item\label{itExoPrimitTriv9} $e^x$
\item\label{itExoPrimitTriv95} $1-e^{-x}$
\item\label{itExoPrimitTriv10} $e^{2x}$
\item\label{itExoPrimitTriv105} $3xe^{-x^2}$
\item\label{itExoPrimitTriv8} $\frac{1}{\sqrt{x}}$
\item\label{itExoPrimitTriv4} $\frac1x$, $x>0$
\item\label{itExoPrimitTriv11} $\frac{1}{1+x^2}$
\item\label{itExoPrimitTriv12} $\frac{1}{\sqrt{1-x^2}}$
\end{enumerate}
\end{multicols}
\vspace{0.01cm}
\begin{sol}
\eqref{itExoPrimitTriv0} $-2x+C$
\eqref{itExoPrimitTriv1} $\frac{x^2}{2}+C$
\eqref{itExoPrimitTriv2} $\frac{x^3}{3}+C$
\eqref{itExoPrimitTriv3} $\frac{x^{n+1}}{n+1}+C$
\eqref{itExoPrimitTriv35} $\tfrac{2}{3}(1+x)^{3/2}+C$
\eqref{itExoPrimitTriv5} $\sen x+C$
\eqref{itExoPrimitTriv6} $-\cos x+C$
\eqref{itExoPrimitTriv7} $\frac{1}{2}\sen (2x)+C$
\eqref{itExoPrimitTriv9} $e^x+C$
\eqref{itExoPrimitTriv95} $x+e^{-x}+C$
\eqref{itExoPrimitTriv10} $\tfrac12 e^{2x}+C$
\eqref{itExoPrimitTriv105} $-\tfrac32e^{-x^2}+C$
\eqref{itExoPrimitTriv8} $2\sqrt{x}+C$
\eqref{itExoPrimitTriv4} $\ln x+C$
\eqref{itExoPrimitTriv11} $\arctan x+C$
\eqref{itExoPrimitTriv12} Com $-1<x<1$, $\arcsen x+C$
\end{sol}
\end{exo}

\begin{exo}
Mostre que $(2x^2-2x+1)e^{2x}$ é primitiva da função $4x^2e^{2x}$.
\end{exo}


Mais tarde olharemos de mais perto o problema de calcular primitivas.
Voltemos agora ao nosso problema:
\index{Teorema Fundamental do Cálculo}
\begin{teo}[Teorema Fundamental do Cálculo]\label{Teo:TFC2}
Seja $f:[a,b]\to \bR$ uma função contínua, e $F$ uma primitiva de $f$.
Então
\eq{\label{eq:TFUNDAM}\int_a^bf(t)\,dt=F(b)-F(a)\equiv F(x)\big|_{a}^b\,.}
\end{teo}

\begin{proof} Lembre que $\int_a^bf(t)\,dt=I(b)$, onde $I(x)$ é a função
área. Ora, sabemos pelo Teorema \ref{Teo:TFC} que $I(x)$ é primitiva
de $f$. Assim, $I(x)=F(x)+C$, onde $F(x)$ é uma primitiva qualquer de $f$, e
onde se trata de achar o valor de $C$.
Mas $I(a)=0$ implica $F(a)+C=0$, logo $C=-F(a)$, e $I(x)=F(x)-F(a)$. Em
particular, $I(b)=F(b)-F(a)$.
\end{proof}

\begin{ex}
Considere $I=\int_0^1x^2dx$, que representa
a área debaixo do gráfico da parábola $y=f(x)=x^2$, entre $x=0$ e $x=1$. 
Como $F(x)=\frac{x^3}{3}$ é primitiva de $f$, temos 
$$\int_0^1x^2\,dx=\frac{x^3}{3}\Big|_{0}^1=\frac{1^3}{3}-\frac{0^3}{3}=\frac{1}{
3} \,.$$
Podemos também calcular a integral da introdução, dessa vez usando o Teorema
Fundamental:
$$\int_0^1(1-x^2)\,dx=\int_0^11\,dx-\int_0^1x^2\,dx=1-\tfrac13=\tfrac23\,.$$
\end{ex}


\begin{exo}\label{Exo:IntegraleNegative}
Mostre que $\int_0^2(x-1)\,dx=0$. Como interpretar esse resultado
geometricamente?
\begin{sol}
Como $\tfrac{x^2}{2}-x$ é primitiva de $f(x)=x-1$, temos
$\int_0^2(x-1)\,dx=(\tfrac{x^2}{2}-x)|_0^2=0$.  
Esse resultado pode ser interpretando decompondo a integral em duas partes: 
$\int_0^2f(x)\,dx=\int_0^1f(x)\,dx+\int_1^2f(x)\,dx$.
Esboçando o gráfico de $f(x)$ entre $0$ e $2$,
\begin{center}
\begin{bmlimage}\begin{tikzpicture}
\fill[areagrafico] (1,0)--(2,1)--(2,0)--cycle;
\fill[areafuncaoarea] (1,0)--(0,-1)--(0,0)--cycle;
\draw (1.65,0.25) node{$+$};
\draw (0.35,-0.3) node{$-$};
\draw (1,0) node{$\shortmid$} node[above]{$1$};
\draw[dashed] (2,0)node[below]{$2$}--(2,1);
\draw[dashed] (0,0)--(0,-1);
\draw[>=latex, ->] (-0.3,0)--(2.4,0);
\draw[>=latex, ->] (0,-1.2)--(0,1.3);
\draw[thick] (0,-1)--(2,1);
\end{tikzpicture}\end{bmlimage}
\end{center}
Vemos que a primeira parte 
$\int_0^1f(x)\,dx=-\tfrac12$ é a contribuição do intervalo em
que $f$ é \emph{negativa}, e é exatamente
compensada pela contribuição da parte \emph{positiva}
$\int_1^2f(x)\,dx=+\tfrac12$.
\end{sol} 
\end{exo}

\begin{exo}\label{exo_TFCnaoseaplica}
A seguinte conta está certa? Justifique.
\[
\int_{-1}^2\frac{1}{x^2}\,dx=\bigl(-\frac{1}{x}\bigr)\Big|_{-1}^2=-\tfrac32\,.
\]
\begin{sol}
Não, a conta não está certa. É porqué a função $\frac{1}{x^2}$ não é
contínua (nem definida) em $0$, ora $0$ pertence ao intervalo de
integração. Logo, o Teorema Fundamental não se aplica.
No entanto, será possível dar um sentido a
$\int_{-1}^2\frac{1}{x^2}\,dx$, usando \emph{integrais impróprias}.
\end{sol}
\end{exo}

O Teorema Fundamental mostra que se uma primitiva de $f$ é
conhecida, então a integral de $f$ em qualquer intervalo $[c,d]$ pode ser
obtida, calculando simplesmente $F(d)-F(c)$. 
Isto é, o problema de calcular integral é reduzido ao de achar uma primitiva de
$f$. 
Ora, \emph{calcular uma primitiva} é uma operação mais complexa do que calcular
uma derivada. De fato, calcular
uma derivada significa simplesmente aplicar mecanicamente as regras de
derivação descritas no Capítulo \ref{Cap:Derivacao}, enquanto uma certa
ingeniosidade pode ser necessária para achar uma primitiva, mesmo de uma
função simples como $\sqrt{1+x^2}$ ou $\ln x$.\\

Portanto, estudaremos \emph{técnicas} para calcular primitivas, ao longo
do capítulo. Por enquanto, vejamos primeiro como usar integrais para calcular
áreas mais gerais do plano.

\section{Áreas de regiões do plano}
\index{área! de região do plano}


Sejam $f$ e $g$ duas funções definidas no mesmo intervalo $[a,b]$, tais que
$g(x)\leq f(x)$ para todo $x\in [a,b]$. Como calcular a área da região $R$
contida entre os gráficos das duas funções, delimitada lateralmente pelas
retas verticais $x=a$ e $x=b$?

\begin{center}
\begin{bmlimage}\begin{tikzpicture}[scale=0.8]
\newcommand{\funcaof}[1]{ 4+sin(((#1)-1) r) }
\newcommand{\funcaog}[1]{ 1+0.2*(#1) }
\draw[>=latex, ->] (0,0)--(5,0);
\pgfmathsetmacro{\a}{1};
\pgfmathsetmacro{\b}{4};
\coordinate (A) at (\a,{\funcaof{\a}});
\coordinate (B) at (\b,{\funcaof{\b}});
\coordinate (C) at (\a,{\funcaog{\a}});
\coordinate (D) at (\b,{\funcaog{\b}});
\coordinate (P) at ({\a+0.4*(\a+\b)/2},{\funcaof{\a+0.4*(\a+\b)/2}});
\draw (P) node[above]{$f$}; 
\coordinate (Q) at ({\a+0.7*(\a+\b)/2},{\funcaog{\a+0.7*(\a+\b)/2}});
\draw (Q) node[below]{$g$}; 
\draw[thick, domain=\a:\b] plot (\x,{\funcaof{\x}});
\draw[thick, domain=\a:\b] plot (\x,{\funcaog{\x}});
\fill[areagrafico, opacity=0.8]
(C)--(A)--plot[domain=\a:\b] (\x,{\funcaof{\x}})--(B)--
(D)--plot[domain=\b:\a] (\x,{\funcaog{\x}})--cycle;
\draw[dashed] (A)--(C); \draw[dashed] (B)--(D);
\draw[>=latex, ->] (0,0)--(5,0);
\draw (\a,0) node{$\shortmid$} node[below]{$a$};
\draw (\b,0) node{$\shortmid$} node[below]{$b$};
\draw[dotted] (\a,0)--(A);
\draw[dotted] (\b,0)--(B);
\end{tikzpicture}\end{bmlimage}
\end{center}

Por uma translação vertical, sempre podemos supor que 
$0\leq g\leq f$. Logo, a área de
$R$ pode ser obtida calculando primeiro a área debaixo do gráfico de $f$, que
vale $\int_a^bf\,dx$, da qual se subtrai a área debaixo do gráfico de $g$,
que vale $\int_a^bg\,dx$. 
\eq{\label{eq:areaentregraficos}\text{área}(R)=\int_a^bf\,dx-\int_a^b
g\,dx\equiv \int_a^b(f-g)\,dx\,.}

\begin{ex}
Considere a região finita $R$ delimitada pela parábola $y=2-x^2$ e pela
reta $y=-x$:
\begin{center}
\begin{bmlimage}\begin{tikzpicture}[scale=0.8]
\newcommand{\funcaof}[1]{ 2-(#1)^2 }
\newcommand{\funcaog}[1]{ -1*(#1) }
\draw[>=latex, ->] (-2,0)--(3,0);
\draw[>=latex, ->] (0,-2)--(0,2.5);
\pgfmathsetmacro{\a}{-1};
\pgfmathsetmacro{\b}{2};
\coordinate (A) at (\a,{\funcaof{\a}});
\coordinate (B) at (\b,{\funcaof{\b}});
\coordinate (C) at (\a,{\funcaog{\a}});
\coordinate (D) at (\b,{\funcaog{\b}});
\coordinate (P) at ({\a+0.7*(\b-\a)},{\funcaof{\a+0.7*(\b-\a)}});
\draw (P) node[above right]{$y=2-x^2$}; 
\coordinate (Q) at ({\a+0.7*(\b-\a)},{\funcaog{\a+0.7*(\b-\a)}});
%\draw (Q) node[below]{$g$};
\fill[areagrafico, opacity=0.8]
(C)--(A)--plot[domain=\a:\b] (\x,{\funcaof{\x}})--(B)--
(D)--plot[domain=\b:\a] (\x,{\funcaog{\x}})--cycle;
\draw[thick, domain=-1.5:2.2] plot (\x,{\funcaof{\x}});
\draw[thick, domain=2.3:-1.5] plot (\x,{\funcaog{\x}}) node[left]{$y=-x$};
% \draw (C) node[left]{$C$};
% \draw (B) node[right]{$B$};
% \draw (D) node[right]{$D$};
\draw[dashed] (A)--(C); \draw[dashed] (B)--(D);
\draw[>=latex, ->] (0,0)--(5,0);
\draw (\a,0) node{$\shortmid$} node[below]{$-1$};
\draw (\b,0) node{$\shortmid$} node[above]{$2$};
\draw[dotted] (\a,0)--(A);
\draw[dotted] (\b,0)--(B);
 \draw (0.5,0.5) node{$R$};
\end{tikzpicture}\end{bmlimage}
\end{center}
Pode ser verificado que os pontos de interseção entre as duas
curvas são $x=-1$ e $x=2$. Observe também que no intervalo $[-1,2]$, a parábola
está sempre \emph{acima} da reta.
Logo, por \eqref{eq:areaentregraficos}, a área de
$R$ é dada pela integral 
$$
\int_{-1}^2\bigl((2-x^2)-(-x)\bigr)\,dx=
\int_{-1}^2\bigl(-x^2+x+2\bigr)\,dx=\Bigl(
-\frac{x^3}{3}+\frac{x^2}{2}+2x
\Bigr)\Big|_{-1}^2=\tfrac92\,.
$$
\end{ex}

\begin{exo}
Esboce e calcule a área da região delimitada pelas curvas abaixo.
\begin{multicols}{2}
\begin{enumerate}
\item\label{itareaRbas1}  $y=-2$, $x=2$, $x=4$, $y=\half x-1$.
\item\label{itareaRbas2} $y=-2$, $x=2$, $x=4$, $y=\half (x-2)^2$.
\item\label{itareaRbas3} $y=x^2$, $y=-(x+1)^2+1$.
\item \label{itareaRbas4} $y=0$, $x=1$, $x=e$, $y=\tfrac1x$.
\item \label{itareaRbas5} $y=-2$,  $y=4+x-x^2$.
\end{enumerate}
\end{multicols}
\vspace{0.01cm}
\begin{sol}
\eqref{itareaRbas1} $5$, 
\eqref{itareaRbas2} $\frac{16}{3}$,
\eqref{itareaRbas3} $\frac{1}{3}$,
\eqref{itareaRbas4} $1$.
\eqref{itareaRbas5} $\tfrac{125}{6}$.
\end{sol}
\end{exo}



\begin{ex} Considere a área da região finita delimitada pelas curvas $x=1-y^2$
e $x=5-5y^2$. 
\begin{center}
\begin{bmlimage}\begin{tikzpicture}
\newcommand{\funcaof}[1]{ 1-(#1)^2 }
\newcommand{\funcaog}[1]{ 5-5*(#1)^2 }
\draw[>=latex, ->] (-1,0)--(6,0);
\draw[>=latex, ->] (0,-1.5)--(0,1.5);
\pgfmathsetmacro{\a}{-1};
\pgfmathsetmacro{\b}{1};
\coordinate (A) at ({\funcaof{\a}},\a);
\coordinate (B) at ({\funcaof{\b}},\b);
\coordinate (C) at ({\funcaog{\a}},\a);
\coordinate (D) at ({\funcaog{\b}},\b);
\coordinate (P) at ({\funcaof{\a+0.7*(\b-\a)}},{\a+0.7*(\b-\a)});
\coordinate (Q) at ({\funcaog{\a+0.7*(\b-\a)}},{\a+0.7*(\b-\a)});
%\draw (Q) node[below]{$g$};
\fill[areagrafico, opacity=0.8]
(C)--(A)--plot[domain=\a:\b] ({\funcaof{\x}},\x)--(B)--
(D)--plot[domain=\b:\a] ({\funcaog{\x}},\x)--cycle;
\draw[thick, domain=-1.3:1.3] plot ({\funcaof{\x}},\x);
\draw[thick, domain=-1.1:1.1] plot ({\funcaog{\x}},\x);
\draw[dashed] (A)--(C); \draw[dashed] (B)--(D);
\draw[>=latex, ->] (0,0)--(5,0);
\draw (0,\a) node{-} node[above left]{$-1$};
\draw (0, \b) node{-} node[below left]{$1$};
\draw[dotted] (0,\a)--(A);
\draw[dotted] (0,\b)--(B);
\draw (P) node[right]{$\scriptstyle{x=1-y^2}$}; 
\draw (Q) node[above right]{$\scriptstyle{x=5-5y^2}$};
% \draw (0.5,0.5) node{$R$};
\end{tikzpicture}\end{bmlimage}
\end{center}
%\begin{sol}
Neste caso, é mais natural expressar a área procurada como um
integral \emph{com respeito a $y$}. Como função de $y$, as curvas são parábolas:
$x=f(y)$ com $f(y)=5-5y^2$ e $x=g(y)$ com $f(y)=1-y^2$, e o gráfico de $f(y)$
está sempre acima do gráfico de $g(y)$. Logo, 
a área procurada é dada por 
 $\int_a^b[f(y)-g(y)]dy$, que vale 
$$\int_{-1}^{1}\big\{(5-5y^2)-(1-y^2)\big\}dy=\int_{-1}^{1}\big\{4-4y^2\big\}
dy=\big\{4y-\tfrac43y^3\big\}\Big|_{-1}^1=\tfrac{16}{3}\,.$$
%\end{sol}
\end{ex}


\begin{exo}(3a prova, primeiro semestre de 2011)
Calcule a área da região finita delimitada pelo gráfico da função $y=\ln x$ e
pelas retas $y=-1$, $y=2$, $x=0$.
\begin{sol}\mbox{}
\begin{center}
\begin{bmlimage}\begin{tikzpicture}[scale=0.6]
\fill[areagrafico]
(0,-1)--(0.368,-1)--plot[domain=0.368:7.38](\x,{ln(\x)})--(0,2)--cycle;
\draw [thick, domain=0.3:8, samples=80] plot (\x,{ln(\x)}) node[right] {$\ln
x$};
\draw [>=latex, ->] (-0.5,0)--(4,0) node[right] {$x$};
\draw [>=latex, ->] (0,-1.2)--(0,2.5);
\draw [dotted] (0,-1)--(0.368,-1);
\draw [dotted] (0,2)--(7.38,2);
\draw (0,-1) node[left]{$-1$};
\draw (0,2) node[left]{$2$};
\draw (8,0) node[right]{$A=\int_{-1}^2e^ydy=e^2-e^{-1}\,.$};
\end{tikzpicture}\end{bmlimage}
\end{center}
Observe que expressando a área com uma integral com respeito a $x$, 
$$A=\int_0^{e^{-1}}(2-(-1))dx+\int_{e^{-1}}^{e^2}(2-\ln x)
dx\,.$$
Essa integral requer a primitiva de $\ln x$, o que não
sabemos (ainda) fazer.
\end{sol}
\end{exo}

%\newpage
\begin{exo} Fixe $\alpha>0$.
Considere $f_\alpha(x)\pardef \alpha^{-2}e^{-\alpha}(\alpha^2-x^2)$. Esboce
$x\mapsto f_\alpha(x)$ para diferentes valores de $\alpha$ (em particular para
$\alpha$
pequeno e grande). Determine o valor de $\alpha$ que maximize a área delimitada
pelo gráfico de $f_\alpha$ e pelo eixo $x$.
\begin{sol}
Consideremos $f_\alpha$ para diferentes valores de $\alpha$:
\begin{center}
\begin{bmlimage}\begin{tikzpicture}[scale=1.3]
\newcommand{\funcao}[2]{ ( exp(-1*(#1))/((#1)^2) )*( (#1)^2 - (#2)^2)}

\foreach \a in {0.3, 0.6,1,2} {
\fill[areagrafico, opacity=0.8] (-\a,0)--
plot[domain=-\a:\a] (\x,{\funcao{\a}{\x}})--(\a,0)--cycle;
}

\foreach \a in {0.3, 0.6,1,2} {
\draw[thick, domain=-\a:\a, samples=50] plot (\x,{\funcao{\a}{\x}});
}

\draw[>=latex,->] (-2.3,0)--(2.3,0);
\draw[>=latex,->] (0,-0.1)--(0,1.3);
\end{tikzpicture}\end{bmlimage}
\end{center}
A área debaixo do gráfico de $f_\alpha$ é dada pela integral 
$$
I_\alpha=\int_{-\alpha}^\alpha f_\alpha(x)\,dx=\frac{e^{-\alpha}}{\alpha^2}
\int_{-\alpha}^\alpha(\alpha^2-x^2)\,dx=(\cdots)=\tfrac43 \alpha
e^{-\alpha}\,.$$
Um simples estudo de $\alpha\mapsto I_\alpha$ mostra que o seu máximo é
atingido em $\alpha=1$.
\end{sol}
\end{exo}

\begin{exo}
Se $a>0$, calcule $I_n=\int_0^ax^{1/n}dx$. Calcule $\lim_{n\to \infty}I_n$, e dê
a interpretação geométrica da solução. (Dica: lembre dos esboços das funções
$x\mapsto x^{1/p}$, no Capítulo \ref{CAP:Funcoes}.)
\begin{sol}
Como $I_n=\frac{n}{n+1}a^{\frac{n+1}{n}}$, temos $\lim_{n\to \infty}I_n=a$.
Quando $n\to \infty$, o gráfico de $x\mapsto x^{1/n}$ em $\bR_+$ tende
ao gráfico da função constante $f(x)\equiv 1$. Ora, $\int_0^a f(x)\,dx=a$!
\end{sol}
\end{exo}

\section{Primitivas}
\index{primitiva}
O Teorema Fundamental mostra a importância de saber calcular
primitivas. Por isso, será útil desenvolver \emph{técnicas de
integração}.
Mas antes de apresentarmos essas técnicas, 
faremos alguns comentários sobre as notações usadas para denotar
primitivas.\\

Para uma dada função $f$, queremos \emph{achar uma primitiva} $F$, isto é 
uma função cuja derivada $F'$ é igual a $f$. Essa operação, \emph{inversa da
derivada}~\footnote{Às vezes, essa operação é naturalmente chamada de
\emph{antiderivada}.}, será chamada de \grasA{integrar $f$}.
Por isso, é útil introduzir uma notação que mostra que
$F$ é o resultado de uma transformação aplicada a $f$:
$$F(x)=\int f(x)dx+C\,,$$
em que $C$ é uma constante arbitrária.
Ao invés da integral definida $\int_a^bf(x)\,dx$, a integral \emph{indefinida}
$\int f(x)\,dx$ \emph{é uma função de $x$}, que por definição satisfaz 
$$\Bigl(\int f(x)\,dx\Bigr)'=f(x)\,.$$
Como a operação ``integrar com respeito a $x$'' é a operação
inversa da derivada, temos 
\eq{\label{eq:integrderivegalf}\int f'(x)\,dx=f(x)+C\,.}
Além disso, as seguintes propriedades são satisfeitas ($\lambda\in \bR$ é uma
constante):
$$\int\lambda f(x)\, dx=\lambda\int f(x)\,dx\,,\quad
\int(f(x)+g(x))dx=\int f(x)\,dx+\int g(x)\,dx\,.$$

As seguintes primitivas fundamentais foram calculadas no Exercício
\ref{Exo:PrimitivasBasicas}:
\begin{multicols}{2}
\begin{enumerate}
\item $\int k\,dx=kx+C$
\item $\int x\,dx=\frac{x^2}{2}+C$
\item\label{itPrimitFund3} $\int x^p\,dx=\frac{x^{p+1}}{p+1}+C$ ($p\neq -1$)
\item $\int \cos x\,dx=\sen x+C$
\item $\int \sen x\,dx=-\cos x+C$
\item $\int e^x\,dx=e^x+C$
\item $\int \frac{dx}{1+x^2}=\arctan x+C$
\item $\int \frac{dx}{\sqrt{1-x^2}}=\arcsen x+C$
\end{enumerate}
\end{multicols}
\vspace{0.01cm}

O caso $p=-1$ em \eqref{itPrimitFund3} corresponde a $\int \frac1x\,dx$, que
obviamente é definida somente para $x\neq 0$. Ora, se $x>0$, temos $(\ln
(x))'=\tfrac1x$, e se $x<0$, temos $(\ln (-x))'=\tfrac{-1}{-x}=\tfrac1x$. Logo,
%\framebox[1.1\width]
$$\boxed{\int \frac1x \,dx=\ln|x|+C\,\quad (x\neq 0).}$$

\begin{exo}
Calcule as primitivas das seguintes funções.
\begin{multicols}{3}
\begin{enumerate}
\item\label{itprimitsubst1} $(1-x)(1+x)^2$
\item\label{itprimitsubst2} $\frac{1}{x^3}-\cos(2x)$
\item\label{itprimitsubst3} $\frac{x+5x^7}{x^9}$
%\item\label{itprimitsubst4} $x\sen(x^2)$
%\item\label{itprimitsubst5} $\cos^2(t)$
\item\label{itprimitsubst6} $2+2\tan^2(x)$
%\item\label{itprimitsubst7} $\frac{x}{1+x^2}$
%\item\label{itprimitsubst8} $\tan x$
\end{enumerate}
\end{multicols}
\vspace{0.01cm}
\begin{sol}
\eqref{itprimitsubst1}
$-\frac{x^4}{4}-\frac{x^3}{3}+\frac{x^2}{2}+x+C$, 
\eqref{itprimitsubst2} $\frac{-1}{2x^2}-\frac{\sen (2x)}{2}+C$,
\eqref{itprimitsubst3} $-\frac{1}{7x^7}-\frac{5}{x}+C$, 
%\eqref{itprimitsubst4} $-\frac{1}{2}\cos(x^2)+C$,
%\eqref{itprimitsubst5} $\frac{x}{2}+\frac{\sen x \cos x}{2}+C$,
\eqref{itprimitsubst6} $2\tan x+C$.
%\eqref{itprimitsubst7} $\tfrac12\ln(1+x^2)+C$
%\eqref{itprimitsubst8} $-\ln(\cos x)+C$
\end{sol}
\end{exo}

Vamos agora apresentar os métodos clássicos usados para 
calcular primitivas.
O leitor interessado em \emph{usar} a integral de Riemann para
resolver problemas concretos pode pular para o
Capítulo~\ref{CAP:Applicacoes}, e voltar depois para as Seçoes 
\ref{Sec:MetodoSubstit} até \ref{Sec:MetodoSubstitTrig} abaixo para
exercitar a sua habilidade a calcular primitivas.

\subsection{Integração por Substituição}\label{Sec:MetodoSubstit}
\index{integração!por substituição}
\begin{ex}
Suponha que se queira calcular $$\int x\cos (x^2)\,dx\,.$$
Apesar da função $x\cos (x^2)$ não ser a derivada de uma função elementar,
ela possui uma estrutura particular: o ``$x$'' que multiplica o cosseno é um
polinômio cujo grau é um a menos do que o polinômio ``$x^2$'' contido dentro do
cosseno. Ora, sabemos que a derivada diminui o grau de um polinômio. No nosso
caso: $(x^2)'=2x$. 
Logo, ao multiplicar e dividir a primitiva por $2$, podemos escrever
$$\int x\cos (x^2)\,dx=\tfrac12\int (2x)\cos (x^2)\,dx
=\tfrac12\int (x^2)'\cos (x^2)\,dx\,.$$
Agora, reconhecemos em $(x^2)'\cos (x^2)$ uma derivada. De fato, pela regra da
cadeia, $(\sen (x^2))'=\cos(x^2)\cdot (x^2)'$. Logo, usando
\eqref{eq:integrderivegalf},
$$
\int (x^2)'\cos (x^2)\,dx=\int (\sen (x^2))'\,dx=
\sen(x^2)+C\,.
$$
Portanto,
$$\int x\cos (x^2)\,dx=\tfrac12\sen(x^2)+C\,.$$
Do mesmo jeito, 
$$\int x^2 \cos (x^3)\,dx=\tfrac13\int 3x^2 \cos (x^3)\,dx=
\tfrac13\int (x^3)' \cos (x^3)\,dx= \tfrac13\sen (x^3)+C\,.$$
\end{ex}

A ideia apresentada nesse último exemplo consiste em conseguir 
escrever a função integrada na forma da derivada 
de uma função composta; é a base do método de integração chamado
\emph{integração por substituição}.
Lembremos a regra da cadeia:
\index{regra!da cadeia}
$$\bigl(f(g(x))\bigr)'=f'(g(x))g'(x)\,.$$
Integrando ambos lados dessa identidade com respeito a $x$ e usando de
novo \eqref{eq:integrderivegalf} obtemos
$f(g(x))=\int f'(g(x))g'(x)\,dx+\text{constante}$, que é equivalente à
\grasA{fórmula de integração por substituição}:
\eq{\boxed{\int f'(g(x))g'(x)\,dx=f(g(x))+C\,.}}
Existem vários jeitos de escrever a mesma fórmula. Por exemplo, se $H$ é
primitiva de $h$,
\eq{\int h(g(x))g'(x)\,dx=H(g(x))+C\,.}
Senão, a função $g(x)$ pode ser considerada como
\emph{uma nova váriavel}: $u\pardef g(x)$. 
Derivando com respeito a $x$,
$\frac{du}{dx}=g'(x)$, que pode ser simbolicamente escrita como $du=g'(x)dx$.
Assim, a primitiva inicial pode ser escrita
somente em termos da variável $u$, \emph{substituindo $g(x)$ por $u$}:
\eq{\label{eq:substitporu}\int h(g(x))g'(x)\,dx=\int h(u)\,du\,.}
Em seguida, se trata de calcular uma primitiva de $h$, e no final voltar para a
variável $x$. O objetivo é sempre tornar $\int h(u)\,du$ o mais próximo possível
de uma primitiva elementar como as descritas no início da seção.

\begin{ex}
Considere $\int \frac{\cos x}{\sen^2x}\,dx$. Aqui queremos usar o fato do
$\cos x$ ser a derivada da função $\sen x$. Façamos então a {substituição}
$u=\sen x$, que implica $du=(\sen x)'dx=\cos x\,dx$, o que implica
$$\int \frac{\cos x}{\sen^2x}\,dx=\int \frac{1}{u^2}\,du\equiv \int
h(u)\,du\,.$$ 
Mas $h(u)=\tfrac{1}{u^2}$, é a derivada (com respeito a $u$!) de 
$H(u)=-\frac{1}{u}$. Logo, 
$$\int \frac{\cos x}{\sen^2x}\,dx=\int h(u)\,du=
H(u)+C=-\frac{1}{\sen x}+C\,.$$
\end{ex}

\begin{ex}
Para calcular $\int \frac{x}{1+x}\,dx$, definemos $u\pardef 1+x$. Logo, $du=dx$
e $x=u-1$. Assim,
\begin{align*}
\int\frac{x}{1+x}\,dx=\frac{u-1}{u}\,du=\int\bigl\{1-\tfrac{1}{u}\bigr\}\,du
&=\int du-\int\tfrac{1}{u}\,du\\
&=u-\ln u+C=1+x-\ln(1+x)+C\,.
\end{align*}
\end{ex}

\begin{ex}
Calculemos agora $\int\frac{x+1}{\sqrt{1-x^2}}dx$.
Para começar, separemos a primitiva em dois termos:
$$
\int\frac{x+1}{\sqrt{1-x^2}}\,dx=\int\frac{x}{\sqrt{1-x^2}}\,dx+\int\frac{1}{
\sqrt {1-x^2}}\,dx\,.
$$
Para o primeiro termo, vemos que com $u=g(x)\pardef 1-x^2$, cuja derivada é
$g'(x)=-2x$, temos $du=-2x\,dx$, e
$$
\int\frac{x}{\sqrt{1-x^2}}\,dx=
%-\int\frac{-2x}{2\sqrt{1-x^2}}\,dx=
-\int \frac{1}{2\sqrt{u}}du=-\sqrt{u}+C=-\sqrt{1-x^2}+C\,.$$
No segundo termo reconhecemos a derivada da função arcseno. Logo, somando,
\eq{\label{eq:rienderien}\int\frac{x+1}{\sqrt{1-x^2}}dx
=-\sqrt{1-x^2}+{\mathrm{arcsen}}\,x+C\,.}
\end{ex}

\begin{obs}
Lembra que um cálculo de primitiva pode sempre
ser \emph{verificado}, derivando o resultado obtido! Por exemplo, não
perca a oportunidade de verificar que derivando o lado direito de
\eqref{eq:rienderien}, obtém-se $\frac{x+1}{\sqrt{1-x^2}}$!
\end{obs}

Às vezes, é preciso transformar a função integrada antes de fazer
uma substituição útil, como visto nos três próximos exemplos.

\begin{ex}
Para calcular $\int \frac{1}{9+x^2}\,dx$
podemos colocar $9$ em evidência no denominador, e em seguida fazer a
substituição $u=\tfrac{x}{3}$:
\begin{align*}\int \frac{1}{9+x^2}\,dx=
\tfrac19\int \frac{1}{1+(\tfrac{x}{3})^2}\,dx&=\tfrac19
\int\frac{3}{1+u^2}\,dx\\
&=\tfrac13\int \frac{1}{1+u^2}\,du =\tfrac13\arctan u+C
=\tfrac13\arctan(\tfrac{x}{3})+C\,.
\end{align*}
\end{ex}

\begin{ex} Para calcular $\int\frac{1}{x^2+2x+2}\,dx$ comecemos 
completando o quadrado: $x^2+2x+2=\{(x+1)^2-1\}+2=1+(x+1)^2$. Logo,
usando $u\pardef x+1$,
\begin{align*}
\int\frac{1}{x^2+2x+2}\,dx&=\int\frac{1}{1+(x+1)^2}\,dx\\
&=\int\frac{1}{1+u^2}\,du=\arctan u+C=
\arctan(x+1)+C\,.
\end{align*}
\end{ex}

\begin{ex}\label{Ex:sencarre}
Considere $\int \sen^2x\,dx$. Lembrando a identidade trigonométrica
$\sen^2x=\frac{1-\cos(2x)}{2}$, 
$$
\int\sen^2x\,dx=\tfrac12 \int \,dx-\tfrac12 \int \cos(2x)\,dx=
\tfrac{x}{2}-\tfrac12 \int \cos(2x)\,dx\,.
$$ 
Agora com $u=2x$ obtemos $\int \cos(2x)\,dx=\tfrac{1}{2}\int
\cos(u)\,du=\tfrac12 \sen u+\text{constante}$. Logo,
$$
\int \sen^2x\,dx=\tfrac{x}{2}-\tfrac14\sen(2x)+C\,.
$$
\end{ex}




\begin{exo}
Calcule as primitivas das seguintes funções.
\begin{multicols}{3}
\begin{enumerate}
\item\label{itprimitsubst40} $(x+1)^7$
\item\label{itprimitsubst400} $\frac{1}{(2x+1)^2}$
\item\label{itprimitsubst401} $\frac{1}{(1-4x)^3}$
\item\label{itprimitsubst4} $x\sen(x^2)$
\item \label{itprimitsubst4000} $\sen x \cos x$
\item\label{itprimitsubst45} $\tfrac{1}{\sqrt{x}}\cos (\sqrt{x})$
\item\label{itprimitsubst5} $\cos^2(t)$
\item\label{itprimitsubst7} $\frac{x}{1+x^2}$
\item\label{itprimitsubst71} $\cos x\sqrt{1+\sen x}$
\item\label{itprimitsubst8} $\tan x$
\item\label{itprimitsubst9} $\frac{3x+5}{1+x^2}$
\item\label{itprimitsubst10} $\frac{1}{x^2+2x+3}$
\item\label{itprimitsubst12} $e^x\tan(e^x)$
%\item\label{itprimitsubst121} $\frac{x}{1+x}$
\item\label{itprimitsubst13} $\frac{y}{(1+y)^3}$
\item\label{itprimitsubst14} $x\sqrt{1+x^2}$
\item\label{itprimitsubst15} $\frac{x}{(1+x^2)^2}$
\item\label{itprimitsubst11} $\frac{\cos^3t}{\sen^4t}$
\item\label{itprimitsubst16} $\sen^3x\cos^3x$
\end{enumerate}
\end{multicols}
\vspace{0.01cm}
\begin{sol}
\eqref{itprimitsubst40} $\frac{1}{8}(x+1)^8+C$ (Obs: aqui, basta fazer a
substituição $u=x+1$. Pode também fazer sem, mas implica desenvolver um
polinômio de grau $7$!)
\eqref{itprimitsubst400} $\frac{-1}{2(2x+1)}+C$
\eqref{itprimitsubst401} $\frac{1}{8(1-4x)^2}+C$
\eqref{itprimitsubst4} $-\frac{1}{2}\cos(x^2)+C$,
\eqref{itprimitsubst4000} $\frac{1}{2}\sen^2(x)+C$, ou $-\frac{1}{2}\cos^2(x)+C$
\eqref{itprimitsubst45} $2\sen(\sqrt{x})+C$,
\eqref{itprimitsubst5} $\frac{x}{2}+\tfrac14\sen (2x)+C$,
\eqref{itprimitsubst7} $\tfrac12\ln(1+x^2)+C$,
\eqref{itprimitsubst71} $\frac{2}{3}(1+\sen x)^{\frac{3}{2}}+C$
\eqref{itprimitsubst8} $\int \tan x\,dx=\int\frac{\sen x}{\cos
x}\,dx=-\int\frac{(\cos x)'}{\cos
x}\,dx -\ln|\cos x|+C$.
\eqref{itprimitsubst9} $\tfrac32 \ln(1+x^2)+5\arctan x+C$
\eqref{itprimitsubst10} $\frac{1}{\sqrt{2}}\arctan(\frac{x+1}{\sqrt{2}})+C$
\eqref{itprimitsubst12} 
Com a substituição $u:=e^x$, $du=e^x dx$, 
$\int e^x\tan(e^x)dx=\int \tan u du=-\ln|\cos u|+C=-\ln|\cos(e^x)|+C$.
\eqref{itprimitsubst13} $\frac{1}{2(1+y)^2}-\frac{1}{1+y}+C$
\eqref{itprimitsubst14} $\frac{1}{3}(1+x^2)^{\frac{3}{2}}+C$
\eqref{itprimitsubst15} $\frac{-1}{2(1+x^2)}+C$
\eqref{itprimitsubst11} $-\frac{1}{3\sen^3t}+\frac{1}{\sen t}+C$ (a ideia aqui
é escrever $\frac{\cos^3t}{\sen^4t}=\frac{\cos^2t}{\sen^4t}\cos
t=\frac{1-\sen^2t}{\sen^4t}\cos t$)
\eqref{itprimitsubst16} $\frac{(\sen x)^4}{4}-\frac{(\sen x)^6}{6}$
\end{sol}
\end{exo}

A fórmula \eqref{eq:substitporu} mostra que a primitiva (ou integral 
indefinida) de uma função 
da forma $h(g(x))g'(x)$ se reduz a achar uma primitiva de $h$.
Aquela fórmula pode também ser usada para integrais definidas: se
$h(g(x))g'(x)$ é integrada com $x$ percorrendo o intervalo $[a,b]$, então
$u=g(x)$ percorre o intervalo $[g(a),g(b)]$, logo
\eq{\label{eq:substitporudefin}
\int_a^bh(g(x))g'(x)\,dx=\int_{g(a)}^{g(b)}h(u)\,du\,.}


\begin{exo}
%(Gilcione, 13/11/2009) 
Calcule as primitivas
\begin{multicols}{3}
\begin{enumerate}
\item\label{itttit1} $\int \frac{2x^3dx}{\sqrt{1-x^2}}\,dx$
\item\label{itttit2} $\int \frac{dx}{\sqrt{x-x^2}}$
\item\label{itttit3} $\int \frac{\ln x}{x}\,dx$
\item\label{itttit4} $\int e^{e^x}e^x\,dx$
\item\label{itttit5} $\int \frac{\sqrt{x}}{1+\sqrt{x}}\,dx$
\item\label{itttit6} $\int \tan^2x\,dx$
\end{enumerate}
\end{multicols}
\vspace{0.01cm}

\begin{sol}
\eqref{itttit1}
Com $u=1-x^2$, $du=-2x\,dx$, temos
\begin{align*}
\int \frac{2x^3dx}{\sqrt{1-x^2}}\,dx=-\int
\frac{x^2}{\sqrt{1-x^2}}(-2x)\,dx
&=-\int \frac{1-u}{\sqrt{u}}\,du\\
&=-2\sqrt{u}+\tfrac23 u^{3/2}+C\\
&=-2\sqrt{1-x^2}+\tfrac23 (1-x^2)^{3/2}+C\,.
\end{align*}
\eqref{itttit2}
Completando o quadrado, e fazendo a substituição $u=2x-1$,
\begin{align*}
\int \frac{dx}{\sqrt{x-x^2}}=\int 
\frac{dx}{\sqrt{\tfrac14-(x-\tfrac12)^2}}&=
\int \frac{2 dx}{\sqrt{1-(2x-1)^2}}\\
&=\int \frac{du}{\sqrt{1-u^2}}=\arcsen u+C=\arcsen (2x-1)+C\,.
\end{align*}
\eqref{itttit3} Com $u=\ln t$, $\int \frac{\ln x}{x}\,dx=\int
u\,du=\tfrac{u^2}{2}+C=\tfrac12(\ln x)^2+C$
\eqref{itttit4} Com $u=e^x$, $\int e^{e^x}e^x\,dx=\int e^u\,du=e^u+C=e^{e^x}+C$.
\eqref{itttit5} $\int \frac{\sqrt{x}}{1+\sqrt{x}}\,dx=x-2\sqrt{x}+2\ln
(1+\sqrt{x})+C$.
\eqref{itttit6} $\int \tan^2x\,dx=\int(1+\tan^2x-1)\,dx=\tan x-x+C$.
\end{sol}
\end{exo}

\subsection{Integração por Partes}\label{Sec:MetodoPartes}
\index{integração!por partes}
Vimos que o método de integração por substituição decorreu da regra da cadeia.
Vejamos agora qual método pode ser obtido a partir da regra de derivação de um
produto.

\begin{ex}
Suponha que se queira calcular a primitiva
$$\int x\cos x\,dx\,.$$
Aqui não vemos (e na verdade: não há) uma substituição que seja útil para
transformar essa primitiva.
O que pode ser útil é escrever $x\cos x=x(\sen x)'$, e de interpretar
$x(\sen x)'$ como o segundo termo da derivada
$$(x\sen x)'=(x)'\sen x+x(\sen x)'=\sen x+x(\sen x)'\,.$$
Assim,
\begin{align*}
\int x\cos x\,dx=\int\bigl\{(x\sen x)'-\sen x\bigr\}\,dx&=
x\sen x-\int\sen x\,dx\\
&=x\sen x+\cos x+C
\end{align*}
\end{ex}
A ideia usada no último exemplo pode ser generalizada da seguinte maneira. Pela
regra de Leibniz,
$$(f(x)g(x))'=f'(x)g(x)+f(x)g'(x)\,.$$
Integrando com respeito a $x$ em ambos lados,
$$
f(x)g(x)=\int f'(x)g(x)\,dx+\int f(x)g'(x)\,dx\,.
$$
Essa última expressão pode ser reescrita como
\eq{\label{eq:integrporpartes}
\boxed{\int f'(x)g(x)\,dx=f(x)g(x)-\int f(x)g'(x)\,dx\,,}
}
(ou a mesma trocando os papéis de $f$ e $g$)
% correcao de "papeis":  Hugo Reis <hugofreitasreis@gmail.com>
chamada \grasA{fórmula de integração por partes}. Ela possui uma forma definida
também:
\eq{\label{eq:integrporpartesdefin}
\boxed{\int_a^b f'(x)g(x)\,dx=f(x)g(x)\big|_a^b-\int_a^b
f(x)g'(x)\,dx\,.}}

A fórmula \eqref{eq:integrporpartes} acima será usada com o intuito de
transformar a integral $\int f'(x)g(x)\,dx$ numa integral (mais simples,
espera-se) $\int f(x)g'(x)\,dx$.

\begin{ex}
Considere $\int x\ln x\,dx$. Aqui definamos $f$ e $g$ da seguinte maneira:
$f'(x)=x$, $g(x)=\ln x$. Assim, $f(x)=\tfrac{x^2}{2}$, $g'(x)=(\ln
x)'=\tfrac{1}{x}$. Usando \eqref{eq:integrporpartes},
\begin{align*}
\int x\ln x\,dx&\equiv \int f'(x)g(x)\,dx\\
&=f(x)g(x)-\int f(x)g'(x)\,dx\\
&\equiv (\tfrac{x^2}{2})(\ln x)-\int (\tfrac{x^2}{2})(\tfrac1x)\,dx=
\tfrac{x^2}{2}\ln x-\tfrac12\int x\,dx=\tfrac{x^2}{2}\ln x-\tfrac{x^2}{4}+C\,
\end{align*}
\end{ex}



\begin{exo} Calcule as primitivas das funções abaixo. 
(Obs: às vezes, pode precisar integrar por partes duas vezes.)
\begin{multicols}{3}
\begin{enumerate}
\item\label{itintpartes1} $x\sen x$
\item\label{itintpartes2} $x\cos(5x)$
\item\label{itintpartes3} $x^2\cos x$
\item\label{itintpartes4} $xe^x$
\item\label{itintpartes5} $x^2e^{-3x}$
\item\label{itintpartes6} $x^3\cos (x^2)$
%\item\label{itintpartes7} $x\ln x$
\end{enumerate}
\end{multicols}
\vspace{0.01cm}
\begin{sol}
\eqref{itintpartes1} $\sen x-x\cos x+C$,
\eqref{itintpartes2} $\frac{1}{5}x\sen(5x)+\frac{1}{25}\cos(5x)+C$
\eqref{itintpartes3} Integrando duas vezes por partes:
$$
\int x^2\cos x\,dx=x^2\sen x-\int (2x)\sen x\,dx=x^2\sen x-2\Bigl\{
x(-\cos x)-\int (-\cos x)\,dx\,.
\Bigr\}$$
Portanto $\int x^2\cos x\,dx=x^2\sen x-2(\sen x-x\cos x)+C$.
\eqref{itintpartes4} $(x-1)e^x+C$
\eqref{itintpartes5} $-\tfrac13 e^{-3x}(x^2-\tfrac23 x-\tfrac29)+C$
\eqref{itintpartes6} 
\begin{align*}
\int x^3\cos (x^2)\,dx=\int x^2 (x\cos(x^2))\,dx&=x^2(\tfrac12
\sen(x^2))-\int(2x)(\tfrac12 \sen (x^2))\,dx\,\\
&=\tfrac12 x^2 \sen(x^2)+\tfrac12 \cos (x^2)+C\,.
\end{align*}
%\eqref{itintpartes7} $x^2(\ln x-\tfrac12)+C$ 
\end{sol}
\end{exo}

Às vezes, escrevendo ``$1$'' como $1=(x)'$, integração por partes pode ser usada mesmo quando não tem duas partes:
\begin{ex}
Considere $\int \ln x\, dx$. Escrevendo $\ln x=1\cdot \ln x=(x)'\ln x$, 
$$
\int \ln x\,dx=\int (x)'\ln x\,dx=x\ln x-\int x(\ln x)'\,dx=x\ln x-\int x\cdot
\tfrac1x\,dx=x\ln x-x+C\,.
$$
\end{ex}

\begin{exo} Calcule 
\begin{multicols}{2}
\begin{enumerate}
\item\label{ititnpartmntriv1} $\int \arctan x \,dx$
\item\label{ititnpartmntriv2} $\int (\ln x)^2\,dx$
\item\label{ititnpartmntriv3} $\int \arcsen x\,dx$
\item\label{ititnpartmntriv4} $\int x\arctan x\,dx$
\end{enumerate}
\end{multicols}
\vspace{0.01cm}
\begin{sol} \eqref{ititnpartmntriv1}
$\int \arctan x dx=x\arctan
x-\int\frac{x}{1+x^2}\,dx=x\arctan x-\tfrac12 \ln (1+x^2)+C$.
\eqref{ititnpartmntriv2} $x(\ln x)^2-2x(\ln x-1)+C$
\eqref{ititnpartmntriv3} $x\arcsen x+\sqrt{1-x^2}+C$
\eqref{ititnpartmntriv4} $\int x\arctan x\,dx=\frac12(x^2\arctan x-x+\arctan x)+C$
\end{sol}
\end{exo}



Consideremos agora um mecanismo particular que pode aparecer quando se aplica
integração por partes:
\begin{ex}
Considere $\int\sen(x)\cos (3x)\,dx$. Integrando duas vezes por partes:
\begin{align*}
\int \sen(x)\cos (3x)dx&=(-\cos x) \cos 3x-\int (-\cos x)(-3\sen 3x)dx\\
&=-\cos x\cos 3x-3\int \cos x\sen 3x\,dx\\
&=-\cos x\cos 3x-3\Big\{
\sen x\sen 3x-\int \sen x(3\cos 3x)\,dx
\Big\}\\
&=-\cos x\cos 3x-3
\sen x\sen 3x+9\int \sen x\cos 3x\,dx\,.
\end{align*}
Assim, a primitiva procurada $I(x)=\int\sen(x)\cos (3x)\,dx$
é solução da equação
$$
I(x)=-\cos x\cos 3x-3\sen x\sen 3x+9 I(x)\,.
$$
Isolando $I(x)$ obtemos
$I(x)=\frac{1}{8}\big\{\cos x\cos 3x+3\sen x\sen 3x \big\}$. Isto é,
$$\int \sen(x)\cos (3x)\,dx=\frac{1}{8}\big\{
\cos x\cos 3x+3\sen x\sen 3x \big\}+C\,.$$
\end{ex}

\begin{exo}\label{Exo:porpartesmach}
Calcule 
\begin{multicols}{3}
\begin{enumerate}
\item\label{itintpartestruc1} $\int e^{-x}\sen x\,dx$
\item\label{itintpartestruc2} $\int e^{-st}\cos t\,dt$
\item\label{itintpartestruc3} $\int \sen (\ln x)\,dx$
\end{enumerate}
\end{multicols}
\vspace{0.01cm}
\begin{sol}
\eqref{itintpartestruc1} $-\frac{e^{-x}}{2}(\sen x+\cos x)+C$
\eqref{itintpartestruc2} $\frac{e^{-st}}{1+s^2}(\sen t- s\cos t)+C$
\eqref{itintpartestruc3} $\frac{x}{2}(\sen (\ln x)-\cos (ln x))+C$
\end{sol}
\end{exo}

Integração por partes pode ser combinada com substituição:

\begin{ex}
Considere $\int x\ln(1+x)\,dx$. Integrando primeiro por partes,
$$
\int x\ln(1+x)\,dx=\tfrac{x^2}{2}\ln(1+x)-\tfrac{1}{2}\int \frac{x^2}{1+x}\,dx\,.
$$
Essa segunda pode ser calculada substituindo $1+x$ por $u$:
\begin{align*}
\int\frac{x^2}{1+x}\,dx=\int\frac{(u-1)^2}{u}\,du&=
\int\{u-2+\tfrac1u\}\,du\\
&=\tfrac{u^2}{2}-2u+\ln|u|+C\\
&=\tfrac12(1+x)^2-2x+\ln|1+x|+C'\,.
\end{align*}
Logo,
$$
\int x\ln (1+x)\,dx=\tfrac{x^2}{2}\ln(1+x)-\tfrac14(1+x)^2
+x-\tfrac12\ln |x|+C'\,.
$$
\end{ex}

\begin{exo}\label{Exo:Bonnardpartessubsit}
Calcule $\int_0^3e^{\sqrt{x+1}}\,dx$, $\int x(\ln x)^2\,dx$.
\begin{sol}
Chamando $u=\sqrt{x+1}$, temos 
$$
\int_0^3e^{\sqrt{x+1}}\,dx=\int_1^2
2ue^u\,du=2\bigl\{ue^u-e^u\bigr\}\big|_1^2=2e^2\,.
$$
Chamando $u=\ln x$, temos $e^u\,du=dx$, e
$$
\int x(\ln x)^2\,dx=\int
u^2e^{2u}\,du=\tfrac{u^2}{2}e^{2u}-\tfrac{u}{2}e^{2u}+\tfrac14 e^{2u}+C\,.
$$
Logo, $\int x(\ln x)^2\,dx=\tfrac12 x^2(\ln x)^2-\tfrac12 x^2\ln
x+\tfrac14x^2+C$. 
\end{sol}
\end{exo}



\subsection{Integração de funções racionais}\label{Sec:FracParciais}
\index{integração!de funções racionais}
Nesta seção estudaremos métodos para calcular primitivas da forma 
$$
\int \frac{dx}{1-x^2}\,,\quad \int \frac{dx}{(1-x)(x+1)^2}\,,
\quad \int\frac{x^2}{x^2+1}\,dx\,,\quad
\int\frac{x^4}{x^3+1}\,dx\,.
$$
Essas primitivas são todas da forma
\eq{\label{eq:primitPQ}
\int\frac{ P(x)}{Q(x)}\,dx\,,}
em que $P(x)$ e $Q(x)$ são polinômios em $x$. Lembramos que um \emph{polinômio em $x$} 
é uma soma finita de potências inteiras e não negativas de $x$: $a_0+a_1x+a_2x^2+\dots+ a_nx^n$, em que
os $a_i$ são constantes. Por exemplo, $x^3-x+1$ é um polinômio, mas
$x^{2/3}+\sqrt{x}$ não é. Lembramos que o \emph{grau} de um polinômio $a_0+a_1x+a_2x^2+\dots+ a_nx^n$ é
o maior índice $i$ tal que $a_i\neq 0$.\\

Existe uma teoria geral que descreve os
métodos que permitem calcular primitivas da forma \eqref{eq:primitPQ}. 
Aqui ilustraremos somente as
ideias principais em casos simples.\\

A primeira etapa tem como objetivo simplificar a expressão para ser
integrada:
\begin{itemize}
 \item \emph{Se o grau de $P$ for maior ou igual ao grau de $Q$, divide
$P$ por $Q$.}
\end{itemize}


\begin{ex}
Considere $\int \frac{x^2}{x^2+1}\,dx$. Aqui, $P(x)=x^2$ é de grau
$2$, que é igual ao grau de $Q(x)=x^2+1$. Logo, como a divisão de $P(x)$ por
$Q(x)$ dá $1$ com um resto de $-1$, temos
$\frac{x^2}{x^2+1}=1-\frac{1}{x^2+1}$. Logo,
$$
\int \frac{x^2}{x^2+1}\,dx=\int\Big\{
1-\frac{1}{x^2+1}\Big\}\,dx=
x-\arctan x+C\,.
$$
(Observe que em vez de fazer uma divisão, podia ter observado que 
$\frac{x^2}{x^2+1}=\frac{x^2+1-1}{x^2+1}=\frac{x^2+1}{x^2+1}-\frac{1}{x^2+1}
=1-\frac{1}{x^2+1}$.)
\end{ex}

\begin{ex}
Considere $\int \frac{x^3}{x^2+1}\,dx$. Aqui, $P(x)=x^3$ é de grau
$3$, que é maior do que o grau de $Q(x)=x^2+1$. Logo, como a divisão de $P(x)$
por
$Q(x)$ dá $x$ com um resto de $-x$, temos
$\frac{x^3}{x^2+1}=x-\frac{x}{x^2+1}$. Logo,
\begin{align*}
\int \frac{x^3}{x^2+1}\,dx=\int\Big\{
x-\frac{x}{x^2+1}\Big\}\,dx&=
\tfrac{x^2}{2}-\tfrac12\int\frac{2x}{x^2+1}\,dx\\
&=\tfrac{x^2}{2}- \tfrac{1}{2}\ln (x^2+1)+C\,.
\end{align*}
\end{ex}

Em geral, quando $\mathrm{grau}(P)\geq \mathrm{grau}(Q)$, 
a divisão de $P$ por $Q$ dá
$$\frac{P(x)}{Q(x)}=\text{polinômio em $x$ }+\frac{\widetilde{P}(x)}{Q(x)}\,,
$$
em 
que $\mathrm{grau}(\widetilde{P})<\mathrm{grau}(Q)$. A primitiva do primeiro 
polimômio é imediata, e o próximo passo é de estudar a primitiva da razão
$\frac{\widetilde{P}(x)}{Q(x)}$.\\

Portanto, é preciso agora desenvolver técnicas para calcular primitivas de
frações de polinômios, em que o grau do numerador é \emph{estritamente menor}
que o grau do denominador. Já sabemos tratar casos do tipo:
$$\int\frac{dx}{x^3}=-\frac{1}{2x^2}+C\,,\quad \int\frac{dx}{x^2+1}=\arctan
x+C\,,\quad \int\frac{x}{x^2+1}dx=\tfrac12 \ln (x^2+1)+C\,.$$

O objetivo será de sempre \emph{decompor} a fração
$\frac{\widetilde{P}(x)}{Q(x)}$ numa soma de frações elementares desse tipo.
O método geral, descrito abaixo em exemplos simples, pode
ser resumido da seguinte maneira:
\begin{itemize}
\item \emph{Fatore completamente o polinômio $Q$, o escrevendo como um produto
de fatores de grau $2$, possivelmente repetidos.} Em seguida,
\item \emph{Procure uma decomposição de $\frac{\widetilde{P}(x)}{Q(x)}$ em
frações parciais.}
\end{itemize}

\begin{ex}\label{Ex:unsurxdeuxmun}
Considere $\int \frac{dx}{x^2-1}$. Aqui, $x^2-1$ tem discriminante $\Delta>0$,
logo ele pode ser \emph{fatorado}: $x^2-1=(x-1)(x+1)$. Procuremos agora um
jeito de escrever a função integrada na forma de uma
soma de frações elementares:
\eq{\label{eq:decompelem}\frac{1}{x^2-1}=\frac{1}{(x-1)(x+1)}=\frac{A}{x-1}
+\frac{B}{x+1}\,.}
Observe que \emph{se} tiver um jeito de achar duas
constantes (isto é: \emph{números que não dependem de $x$}) $A$ e $B$ tais
que a expressão acima seja verificada para todo $x$, então a primitiva será
fácil de se calcular:
$$
\int\frac{dx}{x^2-1}=A\int\frac{dx}{x-1}+B\int\frac{dx}{x+1}=A\ln
|x-1|+B\ln|x+1|+C\,.
$$
Verifiquemos então que as constantes $A$ e $B$ existem. Colocando no mesmo
denominador no lado direito de \eqref{eq:decompelem} e igualando os
numeradores, vemos que $A$ e $B$ devem ser escolhidos tais que
\eq{\label{eq:machiiii}1=A(x+1)+B(x-1)\,.} 
Rearranjando os coeficientes, 
\eq{(A+B)x+A-B-1=0\,.}
Para essa expressão valer para todo $x$, é necessário ter 
$$
A+B=0\,,\quad A-B-1=0\,.
$$
Essas expressões representam um \emph{sistema} de duas equações nas
incógnitas $A$ e $B$, cuja solução pode ser calculada facilmente:
$A=\tfrac12$, $B=-\tfrac12$.
Verifiquemos que os valores calculados para $A$ e $B$ são
corretos:
$$
\frac{\tfrac12}{x-1}
+\frac{-\tfrac12}{x+1}=\frac{\tfrac12 (x+1)-\tfrac12(x-1)}{(x-1)(x+1)}
\equiv \frac{1}{(x-1)(x+1)}\,.
$$
Portanto, 
$$\int\frac{dx}{x^2-1}=\tfrac12\ln
|x-1|-\tfrac12\ln|x+1|+C=\tfrac12\ln \Bigl|\frac{x-1}{x+1}\Bigr|+C\,.$$
\end{ex}

\begin{obs}
Às vezes, os 
valores de $A$ e $B$ podem ser achados de um outro jeito. Por exemplo,
tomando o limite $x\to -1$ em \eqref{eq:machiiii} obtemos
$$1=-2B\,,$$
isto é $B=-\frac12$. Tomando agora $x\to +1$ em \eqref{eq:machiiii} obtemos 
$$1=2A\,,$$ isto é $A=\tfrac12$. 
\end{obs}

A decomposição \eqref{eq:decompelem} é chamada de \grasA{decomposição em
frações parciais}. \index{decomposição em frações parciais}
Esta decomposição pode ser feita a cada vez que o
denominador se encontra na forma de um produto de fatores irredutíveis de
grau $2$. A decomposição deve às vezes ser adaptada. 

\begin{ex}\label{Ex:decomppp}
Considere $\int\frac{dx}{x(x^2+1)}$. 
Vendo o que foi feito acima, uma 
decomposição natural seria de decompor a fração da seguinte maneira:
\eq{\label{eq:decomposp22}
\frac{1}{x(x^2+1)}=\frac{A}{x}+\frac{B}{x^2+1}\,.}
Infelizmente, pode ser verificado (veja o Exercício \ref{Exo:naopode} abaixo) que \emph{não existem} constantes $A$
e $B$ tais que a relação acima valha para todo $x$. O problema é que o denominador da fração original
contém $x^2+1$, que é \emph{irredutível} (isto é: possui um discriminante
negativo), de grau $2$. Assim, procuremos uma decomposição da forma
\eq{\label{eq:decomposp}
\frac{1}{x(x^2+1)}=\frac{A}{x}+\frac{Bx+C}{x^2+1}\,.}
Igualando os numeradores,
$1=A(x^2+1)+(Bx+C)x$, o que equivale a dizer que o polinômio $(A+B)x^2+Cx+A-1=0$
é nulo para todo $x$. Isto é: todos os seus coeficientes são nulos:
$$
A+B=0\,,\quad C=0 \,,\quad A-1=0\,.$$
Assim vemos que $A=1$, $B=-1$, $C=0$.
Verificando:
$$
\frac{1}{x}+\frac{-x}{x^2+1}=\frac{1(x^2+1)+(-x)x}{x(x^2+1)}
\equiv \frac{1}{x(x^2+1)}\,.
$$
Logo,
$$\int\frac{dx}{x(x^2+1)}=\int\frac{dx}{x}-\int \frac{x}{x^2+1}\,dx
=\ln |x|-\tfrac12\ln(x^2+1)+c\,.
$$
\end{ex}


\begin{exo}\label{Exo:naopode}
No Exemplo \ref{Ex:decomppp}, verifique que \emph{não tem}
decomposição da forma 
$\frac{1}{x(x^2+1)}=\frac{A}{x}+\frac{B}{x^2+1}$.
\begin{sol}
Para ter $\frac{1}{x(x^2+1)}=\frac{A}{x}+\frac{B}{x^2+1}$, isto é
$1=A(x^2+1)+Bx$, $A$ e $B$
devem satisfazer às três condições $A=0$, $B=0$, $A=1$, que obviamente é
impossível.
\end{sol}

\end{exo}

\begin{obs}
O esquema de decomposição usado em 
\eqref{eq:decomposp} pode ser generalizado:
$$
\frac{1}{(\alpha_1x^2+\beta_1)(\alpha_2x^2+\beta_2)\cdots
(\alpha_nx^2+\beta_n)}=\frac{A_1x+C_1}{\alpha_1x^2+\beta_1}+
\frac{A_2x+C_2}{\alpha_2 x^2+\beta_2}+\dots
+\frac{A_nx+C_n}{\alpha_n x^2+\beta_n}\,.
$$
Na expressão acima, todos os $\alpha_k>0$ e $\beta_k>0$.
\end{obs}

\begin{ex}\label{Ex:decompppp}
Considere $\int\frac{dx}{x(x+1)^2}$. Aqui o denominador contém o polinômio
irredutível $x+1$ elevado à potência $2$. Assim procuremos uma decomposição da
forma 
\eq{\label{eq:decompospp}
\frac{1}{x(x+1)^2}=\frac{A}{x}+\frac{B}{x+1}+\frac{C}{(x+1)^2}\,.
}
Igualando os numeradores, $1=A(x+1)^2+Bx(x+1)+Cx$, isto é
$(A+B)x^2+(2A+B+C)x+A-1=0$. Para isso valer para todo $x$, é preciso que sejam
satisfeitas as seguintes relações:
$$
A+B=0\,,\quad 2A+B+C=0\,,\quad A-1=0
$$
Assim vemos que $A=1$, $B=-1$, $C=-1$. Deixemos o leitor verificar a
decomposição.
Logo,
\begin{align*}
 \int\frac{dx}{x(x+1)^2}&=\int\Big\{\frac1x-\frac{1}{x+1}-\frac{1}{(x+1)^2}\}\,
dx\\
&=\ln|x|-\ln|x+1|+\frac{1}{x+1}+c\,.
\end{align*}
\end{ex}
\begin{obs}
A decomposição \eqref{eq:decompospp} pode ser usada a cada vez que aparece uma
potência de um fator irredutível. Por exemplo,
$$
\frac{1}{x(x+1)^4}=\frac{A}{x}+\frac{B}{x+1}+\frac{C}{(x+1)^2}+\frac{D}{(x+1)^3}
+\frac{E}{(x+1)^4}\,.
$$
\end{obs}

\begin{exo}
No Exemplo \ref{Ex:decompppp}, verifique que \emph{não tem} decomposição da
forma 
$\frac{1}{x(x+1)^2}=\frac{A}{x}+\frac{B}{(x+1)^2}$.
\begin{sol}
Para ter $\frac{1}{x(x+1)^2}=\frac{A}{x}+\frac{B}{(x+1)^2}$, isto é
$1=A(x+1)^2+Bx$, $A$ e $B$ precisariam satisfazer às três condições $A=0$,
$2A+B=0$, $A=1$, que obviamente é impossível.
\end{sol}
\end{exo}

Os métodos acima podem ser combinados:

\begin{ex}
Para $\int \frac{dx}{x^2(x^2+4)}$, procuremos uma decomposição da
forma
$$
\frac{1}{x^2(x^2+4)}=\frac{A}{x}+\frac{B}{x^2}+\frac{Cx+D}{x^2+4}\,.
$$
Igualando os numeradores e expressando os coeficientes do polinômio em função
de $A,B,C,D$ obtemos o seguinte sistema:
$$
A+C=0\,,\quad B+D=0\,,\quad 4A=0\,,\quad 4B=1\,.
$$
A solução é obtida facilmente: $A=0$, $B=\frac14$, $C=0$, $D=-\frac14$.
Logo,
$$\int \frac{dx}{x^2(x^2+4)}=
\tfrac14\int\frac{dx}{x^2}-\tfrac14\int\frac{dx}{x^2+4}=-\tfrac{1}{4x}
-\tfrac18\arctan(\tfrac{x}{2})+c\,.
$$
\end{ex}

\begin{exo}\label{Exo:PrimitivasDecomposicao}
Calcule as primitivas.
\begin{multicols}{4}
\begin{enumerate}
\item\label{itfracparciais1} $\int \frac{dx}{2x^2+1}$
\item\label{itfracparciais2} $\int\frac{x^5}{x^2+1}\,dx$
\item\label{itfracparciais3} $\int\frac{dx}{(x+2)^2}$ 
\item\label{itfracparciais30} $\int\frac{1}{x^2+x}\,dx$
\item\label{itfracparciais31} $\int\frac{1}{x^3+x}\,dx$
\item\label{itfracparciais4} $\int\frac{dx}{x^2+2x-3}$
\item\label{itfracparciais5} $\int\frac{dx}{x^2+2x+3}$
\item\label{itfracparciais50} $\int\frac{dx}{x(x-2)^2}$ 
\item\label{itfracparciais51} $\int\frac{dx}{x^2(x+1)}$
\item\label{itfracparciais7} $\int\frac{1}{t^4+t^3}dt$
\item\label{itfracparciais52} $\int\frac{dx}{x(x+1)^3}$
\item\label{itfracparciais9} $\int \frac{x^2+1}{x^3+x}\,dx$
\item\label{itfracparciais10} $\int \frac{x^3}{x^4-1}\,dx$
\item\label{itfracparciais104} $\int \frac{x\ln x}{(x^2+1)^2}\,dx$
\item\label{itfracparciais6} $\int\frac{dx}{x^3+1}$
\end{enumerate}
\end{multicols}
\vspace{0.01cm}
\begin{sol}
\eqref{itfracparciais1} $\tfrac{1}{\sqrt{2}}\arctan(\sqrt{2}x)+C$
\eqref{itfracparciais2} Como $\frac{x^5}{x^2+1}=x^3-x+\frac{x}{x^2+1}$, temos
$\int\frac{x^5}{x^2+1}\,dx=\tfrac{x^4}{4}-\tfrac{x^2}{2}+\tfrac12\ln (x^2+1)+C$.
\eqref{itfracparciais3} $\frac{-1}{x+2}+C$

\eqref{itfracparciais30}
A decomposição em frações parciais é da forma
$\frac{1}{x(x+1)}=\frac{A}{x}+\frac{B}{x+1}$.
Colocando no mesmo denominador, $A$ e $B$
tem que satisfazer $1=(A+B)x+A$ para todo $x$. Logo, $A=1$ e $B=-1$. Isto é,
$\frac{1}{x^2+x}=\frac{1}{x}-\frac{1}{x+1}$. Logo,
\begin{align*}
\int \frac{1}{x^2+x}\,dx&=\int \frac{1}{x}\,dx-\int\frac{1}{x+1}\,dx\\
&=\ln |x|-\ln |x+1|+C\,,\quad\quad 
\end{align*}
\eqref{itfracparciais31} 
O integrante é da forma $\frac{P(x)}{Q(x)}$, em que o grau 
de $P$ é menor do que o de $Q$. Além disso, podemos fatorar $x^3+x=x(x^2+1)$. O
polimômio de ordem $2$ tem discriminante negativo. Logo, é irredutível,
e podemos tentar uma decomposição da forma
$$
\frac{1}{x(x^2+1)}=\frac{A}{x}+\frac{Bx+C}{x^2+1}\quad \forall x\,.
$$
Colocando no mesmo denominador, $A$ $B$ e $C$ 
tem que satisfazer $1=(A+B)x^2+Cx+A$ para todo $x$. Logo, $A=1$, $C=0$, e
$B=-A=-1$. Isto é,
\begin{align*}
\int \frac{1}{x^3+x}\,dx=\int \frac{1}{x}\,dx-\int\frac{x}{x^2+1}\,dx
&=\ln |x|-\int\frac{x}{x^2+1}\,dx\\
&=\ln |x|-\tfrac{1}{2}\ln (x^2+1)+C\,,\quad\quad 
\end{align*}
Nesta última integral, fizemos $u=x^2+1$, $du=2x\,dx$.
\eqref{itfracparciais4} Como $\Delta=16>0$, podemos procurar fatorar e fazer uma
separação em frações parciais, 
$$\int\frac{dx}{x^2+2x-3}=\int\frac{dx}{(x+3)(x-1)}=-\tfrac14\int\frac{dx}{x+3}
+\tfrac14\int\frac{dx}{x-1}=\tfrac14\ln \Bigl|\frac{x-1}{x+3}\Bigr|+C\,.
$$
\eqref{itfracparciais5} Como $\Delta=-8<0$, o denominador não se fatora.
Completando o quadrado,
$$
\int\frac{dx}{x^2+2x+3}=\int\frac{dx}{(x+1)^2+2}=\tfrac12\int\frac{dx}{(\frac{x+
1}{\sqrt{2}})^2+1}=\tfrac{1}{\sqrt{2}}\arctan\bigl(\frac{x+
1}{\sqrt{2}}\bigr)+C\,.
$$
\eqref{itfracparciais50} Como
$\frac{1}{x(x-2)^2}=\frac{1}{4x}-\frac{1}{4(x-2)}+\frac{1}{2(x-2)^2}$, temos
$$
\int\frac{dx}{x(x-2)^2}=\tfrac14\ln|x|-\tfrac14\ln|x-2|-\frac{1}{2(x-2)}+C\,.
$$
\eqref{itfracparciais51}
$\frac{1}{x^2(x+1)}=\frac{A}{x}+\frac{B}{x^2}+\frac{C}{x+1}$, com $A=-1$,
$B=1$, $C=1$. Logo,
$$
\int\frac{dx}{x^2(x+1)}=-\ln |x|-\tfrac1x+\ln|x+1|+C'\,.
$$ 

\eqref{itfracparciais7} 
Como $t^4+t^3=t^3(t+1)$, procuramos uma separação da forma 
$$
\frac{1}{t^4+t^3}=\frac{A}{t}+\frac{B}{t^2}+\frac{C}{t^3}+\frac{D}{t+1}\,\quad
\forall t.
$$
Colocando no mesmo denominador e juntando os termos vemos que $A,B,C,D$ têm que
satisfazer 
$$
1=(A+D)t^3+(A+B)t^2+(B+C)t+C\quad\forall t\,.
$$
Identificando os coeficientes obtemos $C=1$, $B=-C=-1$, $A=-B=+1$, e
$D=-A=-1$. Isso implica
\begin{align*}
\int \frac{1}{t^4+t^3}dt&=\int\frac{dt}{t}-\int \frac{dt}{t^2}+\int
\frac{dt}{t^3}-\int \frac{dt}{t+1}\\
&=\ln|t|+\frac{1}{t}-\frac{1}{2t^2}-\ln|t+1|+C\,.
\end{align*}
\eqref{itfracparciais52} 
\begin{align*}
\int\frac{dx}{x(x+1)^3}
&=\int
\frac{dx}{x}-\int\frac{dx}{x+1}-\int\frac{dx}{(x+1)^2}-\int\frac{dx}{(x+1)^3}\\
&=\ln|x|-\ln|x+1|+\frac{1}{x+1}+\frac{1}{2(x+1)^2}+C\,.
\end{align*}
\eqref{itfracparciais9} $\int\frac{x^2+1}{x^3+x}\,dx=\int \frac{dx}{x}=\ln|x|+C$
\eqref{itfracparciais10} Com
$u=x^4-1$, $\int\frac{x^3}{x^4-1}\,dx=\tfrac14\ln|x^4-1|+C$ (é bem mais simples do que começar uma
decomposição em frações parciais...)
\eqref{itfracparciais104} Começando com uma integração por partes, 
\[ 
\int \frac{x\ln x}{(x^2+1)^2}\,dx=\frac{-1}{2(x^2+1)}\ln x+\frac12\int
\frac{1}{(x^2+1)x}\,dx\,,
\]
e essa última integral se calcula como no Exemplo \ref{Ex:decomppp}.
\eqref{itfracparciais6} Primeiro, observe que $x^3+1$ possui $x=-1$ como raiz.
Logo, ele pode ser fatorado como $x^3+1=(x+1)(x^2-x+1)$. 
Como $x^2-x+1$ tem um discriminante negativo,
procuremos uma decomposição da forma
$$
\frac{1}{x^3+1}=\frac{A}{x+1}+\frac{Bx+C}{x^2-x+1}\,.
$$
É fácil ver que $A$, $B$ e $C$ satisfazem às três condições $A+B=0$,
$-A+B+C=0$, $A+C=1$. Logo, $A=\frac13$, $B=-\frac13$, $C=\frac23$. Escrevendo
\begin{align*}
 \int\frac{dx}{x^3+1}&=\tfrac{1}{3}\int\frac{dx}{x+1}-\tfrac13\int
\frac{x-2}{x^2-x+1}\,dx\\
&=\tfrac{1}{3}\ln|x+1|-\tfrac13\int
\frac{x-2}{x^2-x+1}\,dx\\
\end{align*}
Agora, 
\begin{align*}
\int \frac{x-2}{x^2-x+1}\,dx&=\tfrac12\int \frac{2x-1}{x^2-x+1}\,dx-\tfrac{3}{2}
\int\frac{dx}{x^2-x+1}\\
&=\tfrac12 \ln|x^2-x+1|-\tfrac{3}{2}
\int\frac{dx}{x^2-x+1}\\
&=\tfrac12 \ln|x^2-x+1|-\tfrac{4}{\sqrt{3}}\arctan\bigl(\tfrac{2}{\sqrt{3}}
(x-\tfrac12) \bigr)+C\,.
\end{align*}
Juntando,
$$
\int\frac{dx}{x^3+1}=\tfrac{1}{3}\ln|x+1|-\tfrac16\ln|x^2-x+1|+\tfrac{4}{3\sqrt{
3}}\arctan\bigl(\tfrac{2}{\sqrt{3}}(x-\tfrac12) \bigr)+C\,.
$$ 
\end{sol}
\end{exo}


\begin{exo}\label{exo:primunsurseno}
Calcule $\int \frac{1}{\cos x}\,dx$. (Dica: multiplique e divida por $\cos x$.)
\begin{sol}
Com a dica, e a substituição $u=\sen x$,
\begin{align*}
\int \frac{dx}{\cos x}=\int\frac{\cos x}{1-\sen^2
x}dx=\int\frac{du}{1-u^2}&=-\int\frac{du}{u^2-1}\\
&=-\tfrac{1}{2}\ln\Bigl|\frac{u-1}{u+1}\Bigr|+C\\
&=\tfrac{1}{2}\ln\Bigl|\frac{1+\sen x}{1-\sen x}\Bigr|+C
\end{align*}
Observe que essa última expressão pode ser transformada da seguinte maneira:
\begin{align*}
\tfrac{1}{2}\ln\Bigl|\frac{\sen x+1}{\sen x-1}\Bigr|=
\tfrac{1}{2}\ln\Bigl|\frac{(1+\sen x)^2}{\cos^2x}\Bigr|=
\ln\Bigl|\frac{1+\sen x}{\cos x}\Bigr|=
\ln\Bigl|\frac{1}{\cos x}+\tan x\Bigr|\,.
\end{align*}
\end{sol}
\end{exo}

\begin{exo} (3a Prova 2010, Turmas N) Calcule  $\int\frac{x}{x^2+4x+13}dx$.
\begin{sol}
Como $\Delta=4^2-4\cdot 13<0$, o polinômio $x^2+4x+13$ tem discriminante
negativo. Logo, completando o quadrado:\index{completar um
quadrado}
$x^2+4x+13=(x+2)^2-4+13=(x+2)^2+9$, e
\begin{align*}
\int \frac{x}{x^2+4x+13}dx=\int
\frac{x}{(x+2)^2+9}dx=\tfrac19\int\frac{x}{(\tfrac13(x+2))^2+1}dx
\end{align*}
Com $u=\frac{1}{3}(x+2)$, $x=3u-2$, $3du=dx$,
\begin{align*}
 \tfrac19\int\frac{x}{(\tfrac13(x+2))^2+1}dx&=\tfrac13\int\frac{3u-2}{u^2+1}du\\
&=\tfrac12\int\frac{2u}{u^2+1}du-\tfrac23\int\frac{du}{u^2+1}\\
&=\tfrac12\ln (u^2+1)-\tfrac23 \arctan(u)+C\\
&=\tfrac12\ln (x^2+4x+13)-\tfrac23\arctan(\frac{1}{3}(x+2))+C
\end{align*}
\end{sol}
\end{exo}

%\fi
%\newpage

\subsection{Integrar potências de funções trigonométricas}
\index{integração!de funções trigonométricas}
Nesta seção estudaremos primitivas de funções que envolvem 
funções trigonométricas. Essas
aparecem em geral após ter feito uma \emph{substituição
trigonométrica}, que é o nosso último método de integração, e que será
apresentado na próxima seção.

\subsubsection{Primitivas das funções $\sen^mx\cos^nx$}\label{Sec:primsincos}
Aqui estudaremos primitivas da forma
$$\int \sen^mx\cos^nx\,dx\,.$$
Consideremos primeiro integrais contendo somente potências de $\sen x$, ou de
$\cos x$. Além dos casos triviais $\int \sen x\,dx=-\cos x+C$ e $\int \cos
x\,dx=\sen x+C$ já encontramos, no Exemplo \ref{Ex:sencarre},
$$\int\sen^2x\,dx=\int \tfrac{1-\cos(2x)}{2}\,dx=
\tfrac{x}{2}-\tfrac14 \sen(2x)+C\,.$$ 
Consequentemente,
\eq{\label{eq:intcoscarre}\int\cos^2x\,dx=\int\{1-\sen^2x\}\,dx=x-\int
\sen^2x\,dx=
\tfrac{x}{2}+\tfrac14 \sen(2x)+C\,.}
Potências \emph{ímpares} podem ser tratadas da seguinte maneira:
$$
\int \cos^3x\,dx=\int (\cos x)^2 \cos x\,dx=\int (1-\sen^2x)\cos x\,dx\,.
$$
Chamando $u\pardef \sen x$, obtemos
$$
\int \cos^3x\,dx=\int (1-u^2)\,du=u-\tfrac13 u^3+C=\sen x-\tfrac13\sen^3x+C\,.
$$
A mesma ideia pode ser usada para integrar
$\int \sen^mx\cos^nx\,dx$ quando \emph{pelo menos um dos expoentes, $m$ ou $n$,
é ímpar}. Por exemplo, 
\begin{align*}
\int \sen^2x\cos^3x\,dx&= \int \sen^2x\cos^2x\cos x\,dx\\
&=\int \sen^2x(1-\sen^2x)\cos x\,dx=\int u^2(1-u^2)\,du\,,
\end{align*}
onde $u=\sen x$. Logo,
$$
\int \sen^2x\cos^3x\,dx=\tfrac13u^3-\tfrac15 u^5+C=
\tfrac13\sen^3x-\tfrac15 \sen^5x+C\,.
$$
Para tratar potências \emph{pares}, comecemos usando uma
integração por partes. Por exemplo,
\begin{align*}
\int \cos^4x\,dx=\int \cos x\cos^3x\,dx&=\sen x\cos^3x-\int \sen x
(-3\cos^2x\sen x)\,dx \\
&=\sen x\cos^3x+3\int \sen^2 x
\cos^2x\,dx \\
&=\sen x\cos^3x+3\int (1-\cos^2x)
\cos^2x\,dx \\
&=\sen x\cos^3x+3\int\cos^2x\,dx
-3\int \cos^4x\,dx \\
\end{align*}
Isolando $\int \cos^4x\,dx$ nessa última expressão e usando
\eqref{eq:intcoscarre},
\eq{\label{eq:intcosquatre}
\int \cos^4x\,dx=\tfrac14\sen x\cos^3x+\tfrac{3x}{8}+\tfrac{3}{16}
\sen(2x)+C\,.
}


\begin{exo} Calcule as primitivas.
\begin{multicols}{3}
\begin{enumerate}
\item\label{itPotTrig0} $\int \sen^3x\,dx$
\item\label{itPotTrig01} $\int \cos^5x\,dx$
\item\label{itPotTrig1} $\int (\cos x\sen x)^5\,dx$
\item\label{itPotTrig10} $\int \cos^{1000} x\sen x\,dx$
\item\label{itPotTrig2} $\int (\sen^2t\cos t) e^{\sen t}\,dt$
\item\label{itPotTrig3} $\int \sen^3x \sqrt{\cos x}\,dx$
\item\label{itPotTrig4} $\int \sen^2x\cos^2x\,dx$
\end{enumerate}
\end{multicols}
\vspace{0.01cm}
\begin{sol}
\eqref{itPotTrig0} $-\cos x+\tfrac13\cos^3x+C$
\eqref{itPotTrig01} Com $u=\sen x$, $\int \cos^5x\,dx=\int
(1-u^2)^2\,du=\cdots=\sen x-\tfrac23\sen^3x+\tfrac15\sen^5x+C$
\eqref{itPotTrig1} Escrevemos
$\int (\cos x\sen x)^5dx=\int
\sen^5x(1-\sen^2x)^2\cos xdx$. 
Com $u=\sen x$ dá 
\begin{align*}
 \int \sen^5x(1-\sen^2x)^2\cos xdx&=
\int u^5(1-u^2)^2du\\
&=\int (u^5-2u^7+u^9)du\\
&=\frac{u^6}{6}-2\frac{u^8}{8}+\frac{u^{10}}{10}+C\\
&=\frac{\sen^6x}{6}-\frac{\sen^8x}{4}+\frac{\sen^{10}x}{10}+C\,.
\end{align*}
\eqref{itPotTrig10} $-\frac{\cos^{1001}x}{1001}+C$
\eqref{itPotTrig2}
Com $u=\sen t$,
$\int (\sen^2t\cos t) e^{\sen t}dt=\int u^2e^udu$.
Integrando duas vezes por partes e voltando para a variável $t$,
\begin{align*}
 \int u^2e^udu&=u^2e^u-\int (2u)e^udu\\
&=u^2e^u-2\big\{ue^u-\int e^udu\big\}\\
&=u^2e^u-2\{ue^u-e^u\}+C\\
&=e^u(u^2-2u+2)+C\\
&=e^{\sen t}(\sen^2 t-2\sen t+2)+C\,.
\end{align*}
\eqref{itPotTrig3} Com $u=\cos x$, 
$\int \sen^3x \sqrt{\cos x}\,dx=-\int(1-u^2)\sqrt{u}\,du=-\int
(u^{1/2}-u^{5/2})\,du=-\tfrac23 u^{3/2}+\tfrac27 u^{7/2}+C=-\tfrac23
(\cos x)^{3/2}+\tfrac27 (\cos x)^{7/2}+C$.
\eqref{itPotTrig4} $\int
\sen^2x\cos^2x\,dx=\int(1-\cos^2x)\cos^2x\,dx=\int\cos^2x\,dx-\int\cos^4x\,dx$,
e essas duas primitivas já foram calculadas anteriormente. 
\end{sol}
\end{exo}


\subsubsection{Primitivas das funções $\tan^mx\sec^nx$}
Nesta seção estudaremos primitivas da forma
$$\int \tan^mx\sec^nx\,dx\,,$$
onde lembramos que a função \grasA{secante} é definida como 
$$
\sec x\pardef \frac{1}{\cos x}\,.$$
Como $1+\tan^2x=1+\frac{\sen^2x}{\cos^2x}=\frac{1}{\cos^2x}$,
a seguinte relação vale: 
$$1+\tan^2x=\sec^2x\,.$$
Lembramos que
$(\tan x)' =1+\tan^2x=\sec^2x$.
Então, para calcular por exemplo 
\eq{\label{eq:primtances}\int\tan x\sec^2x\,dx\,,}
podemos chamar $u=\tan x$, $du=\sec^2x \,dx$, e escrever
$$
\int\tan x\sec^2x\,dx=
\int u\,du=\tfrac{1}{2}u^2+C=\tfrac12 \tan^2x+C\,.
$$
Na verdade, é facil ver que a mesma substituição pode ser usada \emph{a cada
vez que a potência da secante é par}. Por exemplo,
\begin{align*}
\int\tan x\sec^4x\,dx=\int \tan x\sec^2 x(\sec^2x)\,dx&=
\int \tan x(1+\tan^2x)(\sec^2x)\,dx\\
&=\int u(1+u^2)\,du\\
&=\tfrac12 u^2+\tfrac14 u^4+C\\
&=\tfrac12 (\tan x)^2+\tfrac14(\tan x)^4+C\,.
\end{align*}

Por outro lado, a relação
$$(\sec
x)'=\frac{\sen x}{\cos^2x}\equiv \tan x\sec x\,$$
permite um outro tipo de substituição. 
Por exemplo, \eqref{eq:primtances} pode ser calculada também via a
mudança de variável $w=\sec x$, $dw=\tan x\sec x\,dx$:
$$
\int\tan x\sec^2x\,dx=\int \sec x(\tan x\sec x)\,dx=\int w\,dw=\tfrac12 w^2+C
=\tfrac12 \sec^2x+C\,.
$$
A mesma mudança de variável $w=\sec x$ se aplica \emph{a cada vez que a potência
da tangente é ímpar (e que a potência da secante é pelo menos $1$)}. Por
exemplo,
\begin{align*}
\int  \tan^3x\sec x\,dx&=\int \tan^2 x(\tan x\sec x)\,dx\\
&=\int (\sec^2x-1) (\tan x\sec x)\,dx\\
&=\int(w^2-1)\,dw\\
&=\tfrac13 w^3-w+C\\
&=\tfrac13 \sec^3x-\sec x+C\,.
\end{align*}
Os casos em que a potência da tangente é ímpar e que não tem secante
são tratados separadamente. 
Por exemplo, lembramos que 
$$
\int \tan x\,dx=\int\frac{\sen x}{\cos x}\,dx=-\ln|\cos x|+C\,.
$$
Ou, 
\begin{align*}
\int \tan^3 x\,dx&=\int \tan x(\tan^2 x)\,dx\\
&=\int\tan x(\sec^2x-1)\,dx=\int \tan x \sec^2x\,dx-\int \tan x\,dx\,,
\end{align*}
e essas duas primitivas já foram calculadas acima.
Finalmente, deixemos o leitor fazer o Exercício
\ref{exo:primunsurseno} para mostrar que
$$
\int \sec x\,dx=\ln\bigl|\sec x+\tan x\bigr|+C\,.
$$
\begin{exo}\label{Exo:PrimitTangSec}
Calcule as primitivas.
\begin{multicols}{3}
\begin{enumerate}
\item\label{itInttansec3} $\int \sec^2x\,dx$
\item\label{itInttansec1} $\int\tan^2x \,dx$
\item\label{itInttansec1a} $\int\tan^3x \,dx$
\item\label{itInttansec6} $\int \tan x\sec x\,dx$
\item\label{itInttansec2} $\int\tan^4 x\sec^4x\,dx$
\item\label{itInttansec4} $\int \cos^5x\tan^5x\,dx$
%\item\label{itInttansec5} $\int \sec^4x\tan^4x\,dx$
\item\label{itInttansec7} $\int \sec^5x\tan^3x\,dx$
\item\label{itInttansec8} $\int \sec^3x\,dx$
\end{enumerate}
\end{multicols}
\vspace{0.01cm}
\begin{sol}
\eqref{itInttansec3} $\int \sec^2x\,dx=\tan x+C$.
\eqref{itInttansec1} $\int\tan^2x \,dx=\int(\tan^2x+1-1)\,dx=\tan x-x+C$.
\eqref{itInttansec1a} $\int\tan^3x \,dx=\int\tan x(1+\tan^2x)\,dx-\int \tan x\,dx=\tfrac12\tan^2 x-\ln
|\cos x|+C$.
\eqref{itInttansec6} $\int \tan x\sec x\,dx=\sec x+C$.
\eqref{itInttansec2} $\int\tan^4 x\sec^4x\,dx=\int
\tan^4x(\tan^2x+1)\sec^2x\,dx=\int u^4(u^2+1)\,du=
\tfrac17u^7+\tfrac15u^5+C=\tfrac17\tan^7x+\tfrac15\tan^5x+C$.
\eqref{itInttansec4} $\int\cos^5x\tan^5x\,dx=\int
\sen^5x\,dx=\int(1-\cos^2x)^2\sen x\,dx=
-\int(1-u^2)^2\,du=-u+\tfrac23u^3-\tfrac15 u^5+C=
-\cos x+\tfrac23\cos^3x-\tfrac15 \cos^5x+C$.
\eqref{itInttansec7} $\int \sec^5x\tan^3x\,dx=\int \sec^4x(\sec^2x-1)(\tan x\sec
x)\,dx=\int w^4(w^2-1)\,dw=\tfrac17 w^7-\tfrac15 w^5+C=\tfrac17 \sec^7x-\tfrac15
\sec^5x+C$.
\eqref{itInttansec8} Por partes (lembra que
$(\sec\theta)'=\tan\theta\sec\theta$):
\begin{align*}
 \int\sec^2\theta\sec\theta\,d\theta
&=\tan \theta\sec\theta-\int\tan^2\theta\sec\theta\,d\theta\\
&=\tan \theta\sec\theta-\int(\sec^2\theta-1)\sec\theta\,d\theta\,.
\end{align*}
Logo,
$$
\int\sec^3\theta\,d\theta=
\tfrac12\tan\theta\sec\theta+\tfrac12\int\sec\theta\,d\theta\,.
$$
Já calculamos a primitiva de $\sec \theta$ no Exercício
\ref{exo:primunsurseno}: 
$\int\sec\theta\,d\theta=\ln\bigl|\sec\theta+\tan \theta\bigr|+C$. Logo,
$$
\int\sec^3\theta\,d\theta=
\tfrac12\tan\theta\sec\theta+\tfrac12\ln\bigl|\sec\theta+\tan \theta\bigr|+C\,.
$$
\end{sol}
\end{exo}

\subsection{Substituições trigonométricas}\label{Sec:MetodoSubstitTrig}
\index{substituição! trigonométrica}
Nesta seção final apresentaremos métodos para calcular primitivas de funções
particulares onde aparecem raizes de polinômio do segundo grau:
$$
\int \sqrt{1-x^2}\,dx\,,\quad
\int x^3{\sqrt{1-x^2}}\,dx\,,\quad
\int \frac{dx}{\sqrt{x^2+2x+2}}\,,\quad
\int x^3\sqrt{x^2-3}dx\,,\dots
$$

O nosso objetivo é fazer uma substituição \emph{que transforme o
polinômio que está dentro da raiz em um quadrado perfeito}. 
Essas substituições serão baseadas nas seguintes idenditades trigonométricas:

\eq{\label{eq:relatsinusss}1-\sen^2\theta=\cos^2\theta\,,}
\eq{\label{eq:relattangente} 1+\tan^2\theta=\sec^2\theta\,.}

Ilustraremos os métodos em três exemplos elementares, integrando
$\sqrt{1-x^2}$, $\sqrt{1+x^2}$ e $\sqrt{x^2-1}$. Em seguida aplicaremos as
mesmas ideias em casos mais gerais.

\subsubsection{A primitiva $\int \sqrt{1-x^2}\,dx$}

Observe primeiro que $\sqrt{1-x^2}$ é bem definido se $x\in [-1,1]$.
Para calcular $\int \sqrt{1-x^2}\,dx$
usaremos \eqref{eq:relatsinusss} para
transformar $1-x^2$ em um quadrado perfeito.
Portanto, consideremos a substituição 
$$x=\sen \theta\,,\quad dx=\cos \theta\,d\theta\,.$$
Como $x\in [-1,1]$, 
essa substituição é bem definida, e implica que $\theta$ pode ser
escolhido $\theta\in [-\pisobredois,\pisobredois]$:
\begin{center}
\begin{bmlimage}\begin{tikzpicture}[scale=1]
\pgfmathsetmacro{\a}{2.3};
\draw[->] (0,-\a) -- (0,\a);
\draw[->] (0,0)-- (\a,0);
\draw (0,0)--(1.1,1.665);
\draw [thick] (1.1,1.665)--(1.1,0) node[midway, above, sloped]{$x$};
%\draw [color=\coultang, thick] (2,3)--(2,0) node[midway, above, sloped]{$\tan \alpha$};
%\draw [color=\coulcoseno, thick] (1.1,0)--(0,0) node[midway, below]{$\cos \alpha$};
%\draw (1.1,1.665) node[above]{$B$};
\draw[dotted] (0,-2) arc (-90:90:2);
%\draw(0,0)--(2,3);
%\draw (0,0)--(2,0);
\draw (0.5,1.1) node{$1$};
\draw[ ->] (0.5,0) arc (0:1 r:0.5);
\draw (0.4,0.3) node[right]{$\theta$};
\fill (1.1,1.665) circle (0.45mm);
\draw (0,-2) node[left]{$-1$};
\draw (0,2) node[left]{$+1$};
\end{tikzpicture}\end{bmlimage}
\end{center}

Expressemos agora a primitiva somente em termos de $\theta$:
$$
\int \sqrt{1-x^2}\,dx=\int \sqrt{1-\sen^2\theta}\cos \theta\,d\theta=
\int \sqrt{\cos^2\theta}\cos \theta\,d\theta=\int \cos^2\theta\,d\theta\,.
$$
De fato, como $\theta\in [-\pisobredois,\pisobredois]$, 
$\cos \theta\geq 0$, o que
significa $\sqrt{\cos^2\theta}=\cos \theta$. Mas a primitiva de $\cos^2\theta$ é

$$
\int\cos^2\theta\,d\theta=\tfrac{1}{2}\theta+\tfrac{1}{4}\sen (2\theta)+C\,.
$$
Agora precisamos voltar para a variável $x$. Primeiro,
$x=\sen \theta$ implica $\theta=\arcsen x$. Por outro lado, $\sen
(2\theta)=2\sen \theta\cos\theta=2x\sqrt{1-x^2}$. Logo, 
$$
\boxed{\int \sqrt{1-x^2}\,dx=\tfrac12\arcsen
x+\tfrac12x\sqrt{1-x^2}+C\,.}
$$

\begin{exo}
Verifique esse último resultado, derivando com respeito a $x$.
\begin{sol}
De fato,
\begin{align*}
\bigl(\tfrac12\arcsen x+\tfrac12x\sqrt{1-x^2}\bigr)'&=
\tfrac12\frac{1}{\sqrt{1-x^2}}+\tfrac12
\sqrt{1-x^2}+\tfrac12 x\frac{-2x}{2\sqrt{1-x^2}} \\
&=\tfrac12\frac{1-x^2}{\sqrt{1-x^2}}+\tfrac12
\sqrt{1-x^2}\\
&=\tfrac12 \sqrt{1-x^2}+\tfrac12 \sqrt{1-x^2}=\sqrt{1-x^2}\,.
\end{align*}
\end{sol}
\end{exo}

O método descrito acima costuma ser eficiente
a cada vez que se quer integrar uma função 
que contém uma raiz da forma $\sqrt{a^2-b^2x^2}$, com $a,b>0$.
Para transformar o polinómio $a^2-b^2x^2$ em um quadrado perfeito, podemos
tentar as seguintes subsituições: 
$$x\pardef \tfrac{a}{b}\sen \theta\,,\text{  ou  }\quad x\pardef
\tfrac{a}{b}\cos \theta\,.$$
De fato, uma substituição desse tipo permite cancelar a raiz:
$$
\sqrt{a^2-b^2(\tfrac{a}{b}\sen \theta)^2}=
\sqrt{a^2-a^2\sen^2\theta}=a\sqrt{1-\sen^2\theta}=a\cos \theta\,.
$$
Depois de ter feito a substituição, aparece em geral uma primitiva 
de potências de funções trigonométricas, parecidas com aquelas encontradas na
Seção \ref{Sec:primsincos}.

\begin{ex} Neste exemplo verificaremos que a área de um disco de raio $R$ é
igual a $\pi R^2$.
\begin{center}
\begin{bmlimage}\begin{tikzpicture}
\pgfmathsetmacro{\R}{2};
\pgfmathsetmacro{\alf}{-40};
\pgfmathsetmacro{\bet}{50};
\filldraw[areagrafico] (0,0)--(\R,0) arc (0:90:\R)--cycle;
\draw[>=latex, ->] ({-\R-0.3},0)--({\R+0.3},0);
\draw[>=latex, ->] (0,{-\R-0.3})--(0,{\R+0.3});
\draw (\R,0) arc (0:360:\R);
\draw[thick] (\R,0) arc (0:90:\R);
\draw[->] (0,0)--({\R*cos(\alf)},{\R*sin(\alf)}) node[midway, above]{$R$};
\draw[<-]
({\R*cos(\bet)},{\R*sin(\bet)})--(\R,{\R})node[right]{$y=f(x)=\sqrt{R^2-x^
2}$};
\end{tikzpicture}\end{bmlimage}
\end{center}
A área do disco completo é dada pela integral
$$
A=4\int_{0}^R\sqrt{R^2-x^2}\,dx.
$$
Usemos a substituição trigonométrica $x=R\sen \theta$, $dx=R\cos
\theta\,d\theta$. Se $x=0$, então $\theta=0$, e se $x=R$
então $\theta=\pisobredois$. Logo,
\begin{align*}
 \int_{0}^R\sqrt{R^2-x^2}\,dx&=\int_0^{\pisobredois}\sqrt{R^2-(R\sen
\theta)^2}R\cos \theta\,d\theta\\
&=R^2\int_0^{\pisobredois} \cos^2\theta\,d\theta\\
&=R^2\bigl\{\tfrac12\theta+\tfrac14\sen(2\theta)\bigr\}_0^{\pisobredois}\\
&=R^2\frac{\pi}{4}\,.
\end{align*}
Logo, $A=4 R^2\frac{\pi}{4}=\pi R^2$.
\end{ex}

\begin{ex}
Calculemos a primitiva $\int x^3{\sqrt{4-x^2}}\,dx$. 
Usemos a substituição $x=2\sen \theta$, $dx=2\cos \theta\,d\theta$.
Como $x\in [-2,2]$, temos $\theta\in [-\pisobredois, \pisobredois]$.
$$
\int x^3{\sqrt{4-x^2}}\,dx=\int(2\sen \theta)^3\sqrt{4-(2\sen \theta)^2}2\cos
\theta\,d\theta=32\int\sen^3\theta\cos^2\theta\,d\theta\,.
$$
A última primitiva se calcula feito na seção anterior: com $u=\cos \theta$,
\begin{align*}
\int\sen^3\theta\cos^2\theta\,d\theta&=\int
(1-\cos^2\theta)\cos^2\theta\sen\theta\,d\theta\\
&=-\int (1-u^2)u^2\,du=-\tfrac13 u^3+\tfrac15u^5+C
=-\tfrac13 \cos^3\theta+\tfrac15\cos^5\theta+C\,.
\end{align*}
Para voltar para a variável $x$, observe que $x=2\sen \theta$ implica $\cos
\theta=\sqrt{1-\sen^2\theta}=\sqrt{1-(\tfrac{x}{2})^2}=\sqrt{1-\tfrac{x^2}{4}}$.
Logo,
$$
\int
x^3{\sqrt{4-x^2}}\,dx=-\tfrac{32}{3}\sqrt{1-\tfrac{x^2}{4}}^3+\tfrac{32}{5}\sqrt{1-\tfrac{
x^2}{4}}^5+C\,.
$$
\end{ex}

\begin{exo}
Calcule a área da região delimitada pela elipse cuja equação é
dada por
$$
\frac{x^2}{\alpha^2}+\frac{y^2}{\beta^2}=1\,,
$$
Em seguida, verifique que quando a elipse é um círculo, $\alpha=\beta=R$, a sua área é
$\pi R^2$.  
\begin{sol}
A área é dada por $A=4\int_0^{\alpha}\beta\sqrt{1-\frac{x^2}{\alpha^2}}\,dx$.
Com $x=\alpha\sen \theta$, 
$$A=4\beta\int_0^{\alpha}\sqrt{1-\frac{x^2}{\alpha^2}}\,dx=
4\alpha\beta \int_0^{\pisobredois}\cos^2\theta\,d\theta=\pi \alpha\beta\,.
$$
Quando $\alpha=\beta=R$, a elipse é um disco de raio $R$, de área $\pi 
R\cdot R=\pi R^2$.
\end{sol}
\end{exo}


\begin{ex}
Considere $\int \frac{dx}{x\sqrt{5-x^2}}$.
Com $x=\sqrt{5}\sen \theta$, obtemos
$$\int \frac{dx}{x\sqrt{5-x^2}}=
\int\frac{\sqrt{5}\cos \theta}{(\sqrt{5}\sen \theta)\sqrt{5-(\sqrt{5}\sen
\theta)^2}}\,d\theta
=\tfrac{1}{\sqrt{5}}\int\frac{d\theta}{\sen \theta}\,.
$$
Essa última primitiva pode ser tratada como no Exercício
\ref{exo:primunsurseno}:
$$
\int\frac{d\theta}{\sen \theta}=\tfrac12\ln \Bigl|\frac{1-\cos
\theta}{1+\cos\theta}\Bigr|+C=\tfrac{1}{2}\ln\Bigl|\frac{1-\sqrt{1-\frac{x^2}{5}
} } { 1+\sqrt{1-\frac{x^2}{5}
} } \Bigr|+C\,.
$$
Logo,
$$
\int
\frac{dx}{x\sqrt{5-x^2}}=\tfrac{1}{2\sqrt{5}}
\ln\Bigl|\frac{\sqrt{5}-\sqrt{5-{x^2}
} } { \sqrt{5}+\sqrt{5-{x^2}
} } \Bigr|+C\,.
$$
%\ref{Ex:unsurxdeuxmun}
\end{ex}

\begin{exo} Calcule as primitivas
\begin{multicols}{3}
\begin{enumerate}
\item\label{itPrimSubSinus1} $\int \frac{dx}{\sqrt{1-x^2}}$
%\item\label{itSubstitTrig0} $\int\frac{dx}{\sqrt{3-x^2}}$
%\item\label{itSubstitTrig00} $\int\frac{x}{\sqrt{3-x^2}}\,dx$
%\item\label{itSubstitTrig4} $\int x^3\sqrt{3-x^2}\,dx$.
\item\label{itSubstitTrig2} $\int\frac{x^7}{\sqrt{10-x^2}}\,dx$.
\item\label{itPrimSubSinus3} $\int \frac{x^2}{\sqrt{1-x^3}}\,dx$
\item\label{itPrimSubSinus2} $\int x\sqrt{1-x^2}\,dx$
\item\label{itSubstitTrig6} $\int \frac{x}{\sqrt{3-2x-x^2}}\,dx$.
\item\label{itSubstitTrig000} $\int x^2{\sqrt{9-x^2}}\,dx$
\end{enumerate}
\end{multicols}
\vspace{0.01cm}
\begin{sol}
\eqref{itPrimSubSinus1}
Sabemos que $\int \frac{dx}{\sqrt{1-x^2}}=\arcsen x+C$, mas isso pode ser
verificado de novo fazendo a substituição $x=\sen \theta$:
$\frac{dx}{\sqrt{1-x^2}}=\int\frac{1}{\sqrt{1-\sen^2\theta}}\cos\theta\,
d\theta\int d\theta=\theta+C=\arcsen x+C$.
%\eqref{itSubstitTrig00} $\int\frac{x}{\sqrt{3-x^2}}\,dx=...$
\eqref{itSubstitTrig2} Com $x=\sqrt{10}\sen t$ dá
\begin{align*}
\int\frac{x^7}{\sqrt{10-x^2}}dx=\int\frac{\sqrt{10}^7\sen^7t}{\sqrt{10}\cos
t}\sqrt{10}\cos tdt
&=\sqrt{10}^7\int \sen^7tdt
\end{align*}
Uma segunda substituição $u=\cos t$ dá
\begin{align*}
\int \sen^7tdt&=\int (1-\cos^2t)^3\sen tdt\\
\quad&=-\int(1-u^2)^3du\\
&=-\int(1-3u^2+3u^4-u^6)du \\
&=-\Big\{u-u^3+\frac{3}{5}u^5-\frac{1}{7}u^7\Big\}+C
\end{align*}
Para voltar para $x$, observe que $u=\cos
t=\sqrt{1-\sen^2t}=\sqrt{1-(x/\sqrt{10})^2}$.
Logo,
$$
\int\frac{x^7}{\sqrt{10-x^2}}dx=\sqrt{10}^7\Bigl\{-\sqrt{1-\frac{x^2}{10}}
+\sqrt{1-\frac{x^2}{10}}^3-\frac{3}{5}\sqrt{1-\frac{x^2}{10}}^5
+\frac{1}{7}\sqrt{1-\frac{x^2}{10}}^7\Bigr\}+C
$$
\eqref{itPrimSubSinus3} Observe que $\sqrt{1-x^3}$ \emph{não é da forma
$\sqrt{a^2-b^2x^2}$!} Mas com a substituição $u=1-x^3$, 
$\int \frac{x^2}{\sqrt{1-x^3}}\,dx=-\tfrac13\int
\frac{du}{\sqrt{u}}=-\tfrac23\sqrt{u}+C=-\tfrac23\sqrt{1-x^3}+C$.
\eqref{itPrimSubSinus2} Aqui uma simples substituição $u=1-x^2$ dá
$\int x\sqrt{1-x^2}\,dx=-\tfrac13(1-x^2)^{3/2}+C$. (Pode também fazer $x=\sen
\theta$, é um pouco mais longo.)
\eqref{itSubstitTrig6} Completando o quadrado, 
$3-2x-x^2=4-(x+1)^2$. Chamando $x+1=2\sen \theta$,
\begin{align*}
\int\frac{x}{\sqrt{3-2x-x^2}}\,dx=\int\frac{2\sen
\theta-1}{\sqrt{4-4\sen^2\theta}}2\cos \theta\,d\theta&=
2\int \sen \theta\,d\theta-\int \,d\theta\\
&=-2\cos\theta-\theta+C\,.
\end{align*}
Voltando para $x$, temos
$$
\int\frac{x}{\sqrt{3-2x-x^2}}\,dx=-2\sqrt{1-(\tfrac{x+1}{2})^2}
-\arcsen(\tfrac{x+1}{2})+C\,.$$
\eqref{itSubstitTrig000} Com $x=3\sen \theta$ obtemos 
$\int x^2{\sqrt{9-x^2}}\,dx=3^4\int \sen^2\theta\cos^2\theta\,d\theta$.
\end{sol}
\end{exo}


\subsubsection{A primitiva $\int \sqrt{1+x^2}\,dx$}

Para calcular $\int \sqrt{1+x^2}\,dx$
usaremos \eqref{eq:relattangente} para
transformar $1+x^2$ em um quadrado perfeito.
Portanto, consideremos a substituição 
$$x=\tan \theta\,,\quad dx=\sec^2 \theta\,d\theta\,.$$
Expressemos agora a primitiva somente em termos de $\theta$:
$$
\int \sqrt{1+x^2}\,dx=\int \sqrt{1+\tan^2\theta}\sec^2 \theta\,d\theta=
\int \sqrt{\sec^2\theta}\sec^2\theta\,d\theta=\int \sec^3\theta\,d\theta\,.
$$
Vimos no Exercício \ref{Exo:PrimitTangSec} que
$$
\int\sec^3\theta\,d\theta=
\tfrac12\tan\theta\sec\theta+\tfrac12\ln\bigl|\sec\theta+\tan
\theta\bigr|+C\,.
$$
Para voltar à variável $x$: $\sec\theta=x$,
$\tan\theta=\sqrt{1+\sec^2\theta}=\sqrt{1+x^2}$.
Logo,
$$\boxed{\int \sqrt{1+x^2}dx=\tfrac12
x\sqrt{1+x^2}+\tfrac12 \ln |x+\sqrt{1+x^2}|
+C\,.}$$

O método descrito acima se aplica a cada vez que se quer integrar uma função 
que contém uma raiz da forma $\sqrt{a^2+b^2x^2}$, com $a,b>0$.
Para transformar o polinómio $a^2+b^2x^2$ em um quadrado perfeito, podemos
tentar as seguintes subsituições: 
$$x\pardef \tfrac{a}{b}\tan \theta \,.$$
De fato, uma substituição desse tipo permite cancelar a raiz:
$$
\sqrt{a^2+b^2(\tfrac{a}{b}\tan \theta)^2}=
\sqrt{a^2+a^2\tan^2\theta}=a\sqrt{1+\tan^2\theta}=a\sec \theta\,.
$$

\begin{exo}
Calcule as primitivas
\begin{multicols}{3}
\begin{enumerate}
\item\label{itSubstitTrig3} $\int\frac{x^3}{\sqrt{4x^2+1}}\,dx$.
\item\label{itSubstitTrig31} $\int x^3\sqrt{x^2+1}\,dx$
\item\label{itSubstitTrig32} $\int x\sqrt{x^2+a^2}\,dx$
\item\label{itSubstitTrig33} $\int \frac{dx}{\sqrt{x^2+2x+2}}$ 
\item\label{itSubstitTrig10} $\int\frac{dx}{(x^2+1)^3}$
\item\label{itSubstitTrig100} $\int\frac{dx}{x^2\sqrt{x^2+4}}$
\end{enumerate}
\end{multicols}
\vspace{0.01cm}
\begin{sol}
\eqref{itSubstitTrig3}
fazendo $x=\tfrac12\tan \theta$ dá
\begin{align*}
 \int \frac{x^3}{\sqrt{4x^2+1}}dx&=\int \frac{(\tfrac12 \tan
\theta)^3}{\sqrt{\sec^2\theta}}
\half\sec^2\theta d\theta\\
&=\tfrac{1}{16} \int\tan^3\theta \sec\theta d\theta\\
&=\tfrac{1}{16} \int (\sec^2\theta-1)\sec\theta \tan\theta d\theta
\end{align*}
Com $w=\sec \theta$, obtemos 
$\int (\sec^2\theta-1)\sec\theta \tan\theta
d\theta=\frac{\sec^3\theta}{3}-{\sec\theta}+C$. 
Mas $\tan \theta=2x$ implica $\sec \theta=\sqrt{\tan^2\theta+1}=\sqrt{1+4x^2}$.
Logo,
$$
\int \frac{x^3}{\sqrt{4x^2+1}}dx=\frac{(1+4x^2)^{\frac{3}{2}}}{48}
-\frac{\sqrt{1+4x^2}}{16}+C\,.
$$
Observe que pode também rearranjar um pouco a função e fazer por partes:
\begin{align*}
 \int \frac{x^3}{\sqrt{4x^2+1}}dx&=
 \tfrac14\int x^2\frac{8x}{2\sqrt{4x^2+1}}dx\\
&=\tfrac14\Bigl\{x^2\sqrt{4x^2+1}-\int (2x)\sqrt{4x^2+1}dx\Bigr\}\\
&=\tfrac14\Bigl\{x^2\sqrt{4x^2+1}-\tfrac14\frac{(4x^2+1)^{3/2}}{3/2}\Bigr\}+C\,,
\end{align*}
dá na mesma!
\eqref{itSubstitTrig31} Com $x=\tan \theta$, temos
\begin{align*}
\int x^3\sqrt{x^2+1}\,dx&=\int \tan^3\theta\sec^3\theta\,d\theta\\
&=\int (\sec^2\theta-1)\sec^2\theta(\tan\theta\sec\theta)\,d\theta\\
(\text{via }w=\sec\theta)\,
&=\tfrac15\sec^5\theta-\tfrac13\sec^3\theta+C\\
&=\tfrac15(x^2+1)^{5/2}-\tfrac13(x^2+1)^{3/2}+C\,.
\end{align*}
\eqref{itSubstitTrig32} Aqui não precisa fazer substituição trigonométrica:
$u=x^2+a^2$ dá $\int
x\sqrt{x^2+a^2}\,dx=\tfrac12\int\sqrt{u}\,du=\tfrac13u^{3/2}+C=
\tfrac13(x^2+a^2)^{3/2}+C$.
\eqref{itSubstitTrig33} Como $x^2+2x+2=(x+1)^2+1$, a substituição
$x+1=\tan\theta$ dá 
$\int
\frac{dx}{\sqrt{x^2+2x+2}}=\int\frac{\sec^2\theta}{\sec\theta}\,
d\theta=\int\sec\theta\,d\theta=\ln|\sec\theta+\tan\theta|+C=\ln\bigl|x+1+\sqrt{
x^2+2x+2}\bigr|+C$.
\eqref{itSubstitTrig10} Apesar da função $\frac{1}{(x^2+1)^3}$ não possuir
raizes, façamos a substituição $x=\tan\theta$:
\begin{align*}
\int\frac{dx}{(x^2+1)^3}&=\int\frac{\sec^2\theta}{(\tan^2\theta+1)^3}\,
d\theta=\int\frac{d\theta}{\sec^4\theta}=\int\cos^4\theta\,d\theta\,.
\end{align*}
Essa última primitiva já foi calculada em \eqref{eq:intcosquatre}:
$\int \cos^4\theta\,d \theta=\tfrac14\sen
\theta\cos^3\theta+\tfrac{3\theta}{8}+\tfrac{3}{16}
\sen(2\theta)+C$. Ora, se $\tan\theta=x$, então
$\sen\theta=\frac{x}{\sqrt{1+x^2}}$ e $\cos\theta=\frac{1}{\sqrt{1+x^2}}$.
Logo,
$$
\int\frac{dx}{(x^2+1)^3}=\frac{x}{4(1+x^2)^2}+\frac38\Bigl\{\arctan
x+\frac{x}{1+x^2}\Bigr\}+C\,.
$$
\eqref{itSubstitTrig100} Com $x=2\tan \theta$, 
$\int\frac{dx}{x^2\sqrt{x^2+4}}=\tfrac14\int
\frac{\cos\theta}{\sen^2\theta}\,d\theta=-\frac{1}{4\sen\theta}+C$.
Agora observe que $2\tan \theta=x$ implica $\sen\theta=\frac{x}{\sqrt{x^2+4}}$.
Logo,
$\int\frac{dx}{x^2\sqrt{x^2+4}}=-\frac{\sqrt{x^2+4}}{4x}+C$.
\end{sol}
\end{exo}

\begin{exo}\label{exo:comprparabola}
Calcule o comprimento do arco da parábola $y=x^2$, contido 
entre as retas $x=-1$ e  $x=1$.
\begin{sol}
Já montamos a integral no Exemplo \ref{ex:comprparabdifiss}, e esta pode ser
calculada com os métodos dessa seção: 
$L=2\int_0^1\sqrt{1+4x^2}\,dx=\frac{\sqrt{5}}{4}+\frac12\ln(\frac12+\frac{\sqrt{5}}{2})$.
\end{sol}
\end{exo}

\subsubsection{A primitiva $\int \sqrt{x^2-1}\,dx$}
Finalmente, consideremos a primitiva $\int \sqrt{x^2-1}\,dx$. Para transformar
$x^2-1$ num quadrado perfeito, usaremos a relação \eqref{eq:relattangente}:
$\sec^2\theta-1=\tan^2\theta$. Assim, chamando $x=\sec\theta$, temos
$dx=\tan\theta\sec\theta\,d\theta$, portanto

$$\int
\sqrt{x^2-1}\,dx=\int\sqrt{\sec^2\theta-1}\tan\theta\sec\theta\,d\theta
=\int \tan^2\theta\sec\theta\,d\theta\,.$$
Integrando por partes,
\begin{align*}
\int
(\tan\theta\sec\theta)\tan\theta\,
d\theta&=\sec\theta\tan\theta-\int\sec^3\theta\, d\theta\\
&=\sec\theta\tan\theta-\Bigl\{
\tfrac12\tan\theta\sec\theta+\tfrac12\ln\bigl|\sec\theta+\tan
\theta\bigr|
\Bigr\}+C\\
&=\tfrac12\sec\theta\tan\theta-\tfrac12\ln\bigl|\sec\theta+\tan
\theta\bigr|+C\,.
\end{align*}
Como $\sec\theta=x$ implica
$\tan\theta=\sqrt{\sec^2\theta-1}=\sqrt{x^2-1}$, obtemos
$$\boxed{
\int\sqrt{x^2-1}\,dx=\tfrac12
x\sqrt{x^2-1}-\tfrac12 \ln\bigl|x+\sqrt{x^2-1}\bigr|+C\,.}
$$
O método apresentado acima sugere que para integrar uma função que contém um
polinômio do segundo grau da forma $\sqrt{a^2x^2-b^2}$, pode-se tentar fazer a
substituição $$x\pardef \frac{b}{a}\sec \theta\,.$$

\begin{ex}
Consideremos a primitiva $\int\frac{dx}{x^2\sqrt{x^2-9}}$, fazendo a
substituição $x=3\sec\theta$, $dx=3\tan\theta\sec\theta\,d\theta$:
$$
\int\frac{dx}{x^2\sqrt{x^2-9}}=\int\frac{3\tan\theta\sec\theta}{(3\sec\theta)^2
\sqrt{(3\sec\theta)^2-9}}\,d\theta
=\tfrac{1}{9}\int\frac{d\theta}{\sec\theta}=
\tfrac19\int\cos\theta\,d\theta
=\tfrac19\sen\theta+C\,.
$$
Para voltar à variável $x$, façamos uma interpretação geométrica da nossa
substituição. A relação $x=3\sec\theta$, isto é $\cos\theta =\frac{3}{x}$, se
concretiza no seguinte triângulo:
\begin{center}
\begin{bmlimage}\begin{tikzpicture}[scale=1.5]
\draw (0,0)--(2,1) node[midway, above, sloped]{$x$}--(2,0) 
node[midway, right]{$\sqrt{x^2-9}$} --(0,0) node[midway,
below]{$3$};
\draw (0.4,0) arc (0:26.56:0.4);
\draw (0.56,0.13) node{$\theta$};
\draw (4.8,0.5) node{$\Rightarrow\,\,\sen\theta=\frac{\sqrt{x^2-9}}{x}$};
\end{tikzpicture}\end{bmlimage}
\end{center}
Assim, 
$$
\int\frac{dx}{x^2\sqrt{x^2-9}}=\frac{\sqrt{x^2-9}}{9x}+C\,.
$$
\end{ex}

\begin{exo} Calcule as primitivas.
\begin{multicols}{3}
\begin{enumerate}
\item\label{itPrimSubSecante1} $\int x^3\sqrt{x^2-3}dx$
\item\label{itSubstitTrig8} $\int \frac{dx}{\sqrt{x^2-a^2}}\,dx$.
\item\label{itSubstitTrig1} $\int\frac{x^3}{\sqrt{x^2-1}}\,dx$
\end{enumerate}
\end{multicols}
\vspace{0.01cm}
\begin{sol}
\eqref{itPrimSubSecante1}
Seja $x=\sqrt{3}\sec \theta$. Então $dx=\sqrt{3}\sec \theta\tan \theta$, e
\begin{align*}
 \int x^3\sqrt{x^2-3}dx&=\int (\sqrt{3}\sec \theta)^3 \sqrt{3}  \tan \theta
\sqrt{3}\sec \theta\tan \theta d\theta\\
&=\sqrt{3}^5\int \{\sec^2\theta \tan^2 \theta\}\sec^2 \theta d\theta\\
(\text{ com }u=\tan \theta)&=\sqrt{3}^5\int (u^2+1)u^2du\\
&=\sqrt{3}^5(u^5/5+u^3/3)+C
\end{align*}
Mas como $\cos \theta=\sqrt{3}/x$, temos (fazer um desenho) $u=\tan
\theta=\sqrt{x^2-3}/\sqrt{3}$. Logo,
$$
\int x^3\sqrt{x^2-3}dx=\tfrac15\sqrt{x^2-3}^5+\sqrt{x^2-3}^3+C
$$
Um outro jeito de calcular essa primitiva é de começar com uma integração por
partes:
\begin{align*}
\int x^3\sqrt{x^2-3}dx=
 \tfrac12\int x^2\,
\big\{2x\sqrt{x^2-3}\big\}dx&=\tfrac12\Big\{x^2\frac{(x^2-3)^{3/2}}{3/2}-\int
2x\frac{(x^2-3)^{3/2}}{3/2}dx\Big\}\\
&=\tfrac12\Big\{x^2\frac{(x^2-3)^{3/2}}{3/2}-\tfrac23\int
2x{(x^2-3)^{3/2}}dx\Big\}\\
&=\tfrac12\Big\{x^2\frac{(x^2-3)^{3/2}}{3/2}-\tfrac23\frac{(x^2-3)^{5/2}}{5/2}
\Big\}+C\\
&=\tfrac13x^2{(x^2-3)^{3/2}}-\tfrac{2}{15}{(x^2-3)^{5/2}}+C\,\,)
\end{align*}
\eqref{itSubstitTrig8} Com $x=a\sec\theta$, 
$\int
\frac{dx}{\sqrt{x^2-a^2}}\,dx=\int\sec\theta\,
d\theta=\ln|\sec\theta+\tan\theta|+C$. Como $\cos\theta=\frac{a}{x}$,
\begin{center}
\begin{bmlimage}\begin{tikzpicture}[scale=1.5]
\draw (0,0)--(2,1) node[midway, above, sloped]{$x$}--(2,0) 
node[midway, right]{$\sqrt{x^2-a^2}$} --(0,0) node[midway,
below]{$a$};
\draw (0.4,0) arc (0:26.56:0.4);
\draw (0.56,0.13) node{$\theta$};
\draw (4.8,0.5) node{$\Rightarrow\,\,\tan\theta=\frac{\sqrt{x^2-a^2}}{a}$};
\end{tikzpicture}\end{bmlimage}
\end{center}
Logo,
$\int
\frac{dx}{\sqrt{x^2-a^2}}\,dx=\ln|\tfrac{x}{a}+\tfrac{\sqrt{x^2-a^2}}{a}|+C$.
\eqref{itSubstitTrig1}
Com $x=\sec \theta$, $dx=\sec \theta\tan \theta
d\theta$:
\begin{align*}
\int\frac{x^3}{\sqrt{x^2-1}}dx&=\int 
\frac{\sec^3\theta}{\tan \theta}\sec \theta\tan \theta d\theta\\
&=\int \sec^2\theta \sec^2\theta d\theta\\
&=\int (\tan^2\theta+1)\sec^2\theta d\theta\\
(u\pardef \tan \theta)\quad&=\int (u^2+1)du\\
&=\frac{u^3}{3}+u+C\\
&=\frac{\tan^3\theta}{3}+\tan\theta+C\,.
\end{align*}
Mas $\sec \theta=x$ implica $\tan\theta=\sqrt{x^2-1}$.
Logo,
$$
\int\frac{x^3}{\sqrt{x^2-1}}dx=
\frac{1}{3}(x^2-1)^{\tfrac32}+\sqrt{x^2-1}+C\,.
$$

\end{sol}
\end{exo}



%\fi




% !TeX spellcheck = pt_BR
% !TEX encoding = UTF-8 Unicode

\chapter{Aplicações}\label{CAP:Applicacoes}

\ifdefined\updateans
% Only need to run once in a lifetime, when the file ans.tex needs to be updated.
\Writetofile{ans}{\protect\section*{Capítulo \ref{CAP:Applicacoes}}}
\fi

\section{Comprimento de arco}
\index{comprimento de arco}
O procedimento usado na definição da integral de Riemann (cortar, somar, tomar
um limite) pode ser útil em outras situações.
As três próximas seções serão dedicadas ao uso de integrais para
calcular  quantidades geométricas associadas a funções.
Começaremos com o comprimento de arco.\\

Vimos acima que a integral de Riemann permite calcular a área debaixo 
do gráfico de uma função $f:[a,b]\to \bR$. 
Mostraremos agora como calcular o \emph{comprimento} 
do gráfico, via uma outra integral formada a partir da função.\\

Procederemos seguindo a mesma ideia, \emph{aproximando} o comprimento 
por uma soma. Escolhamos uma subdivisão do intervalo $[a,b]$ por intervalos $[x_i,x_{i+1}]$:

\begin{center}
\begin{bmlimage}\begin{tikzpicture}[scale=1.5]
\newcommand{\func}[1]{(#1)^2/4+1}
\pgfmathsetmacro{\a}{0};
\pgfmathsetmacro{\b}{2};
\pgfmathsetmacro{\n}{5};
\pgfmathsetmacro{\d}{(\b - \a )/\n};
\draw[>=latex, ->] (\a-0.2,0)--(\b+0.2,0);
\draw [thin, domain=\a:\b] plot (\x,{\func{\x}});

\foreach \j in {1,...,\n}{
\draw[color=blue] 
({\a+\j*\d},{\func{\a+\j*\d}})--({\a+(\j-1)*\d},{\func{\a+(\j-1)*\d}});
\fill ({\a+\j*\d},{\func{\a+\j*\d}}) circle (0.40mm);
%\pgfmathsetmacro{\x}{0};
}
\draw ({\a+2*\d},0) node[below]{\footnotesize{$x_{i}$}};
\draw[dotted] ({\a+2*\d},0)--({\a+2*\d},{\func{\a+2*\d}});

\draw ({\a+3*\d},0) node[below]{\footnotesize{$x_{i+1}$}};
\draw[dotted] ({\a+3*\d},0)--({\a+3*\d},{\func{\a+3*\d}});

\draw[dashed] (\a,0) node[below]{$a$}--(\a,{\func{\a}});
\draw[dashed] (\b,0) node[below]{$b$}--(\b,{\func{\b}});
\end{tikzpicture}\end{bmlimage}
\end{center}

Aproximaremos o comprimento do gráfico da função, 
em cada intervalo $[x_i,x_{i+1}]$, pelo comprimento do segmento que liga 
$(x_i,f(x_i))$ a $(x_{i+1},f(x_{i+1}))$, dado por 
\begin{align*}
 \sqrt{\Delta x_i^2+(f(x_{i+1})-f(x_i))^2}
&=\Delta x_i\sqrt{1+\Bigl(\frac{f(x_{i+1})-f(x_i)}{\Delta x_i}\Bigr)^2}\,,
\end{align*}
em que $\Delta x_i=x_{i+1}-x_i$. Quando $\Delta x_i\to 0$, o quociente 
$\frac{f(x_{i+1})-f(x_i)}{\Delta x_i}$ tende a $f'(x_i)$. Logo, o comprimento
do gráfico, $L$, é aproximado pela soma
$$\sum_{i=1}^n\sqrt{1+f'(x_i)^2}\Delta x_i\,,$$
que é uma soma de Riemann associada à função $\sqrt{1+f'(x)^2}$. Logo, 
tomando
um limite em que o número de intervalos cresce e o tamanho de cada intervalo
tende a zero,
obtemos uma expressão para $L$ via uma integral:
\eq{\label{eq:ComprimentoArco}\boxed{L=\int_a^b\sqrt{1+f'(x)^2}\,dx\,.}}

\begin{ex}
Calculemos o comprimento do gráfico da curva $y=\tfrac23 x^{3/2}$, entre $x=0$ e $x=1$.
Como $(\tfrac23 x^{3/2})'=\sqrt{x}$, 
$$
L=\int_0^1\sqrt{1+(\sqrt{x})^2}\,dx=\int_0^1\sqrt{1+x}\,dx=
\tfrac23(\sqrt{8}-1)\,.
$$
\end{ex}
Devido à raiz que apareceu na fórmula \eqref{eq:ComprimentoArco} (após o uso do
Teorema de Pitágoras), as integrais que aparecem para calcular comprimentos de
gráficos podem ser difíceis de calcular, isso mesmo quando a função $f$ é
simples:

\begin{ex}\label{ex:comprparabdifiss}
O comprimento da parábola $y=x^2$ entre $x=-1$ e $x=1$ é dado pela integral 
$$L=\int_{-1}^1\sqrt{1+4x^2}\,dx\,.$$
Vimos na Seção \ref{Sec:MetodoSubstitTrig} (ver o Exercício \ref{exo:comprparabola}) 
como calcular a primitiva de $\sqrt{1+4x^2}$ usando uma substituição
trigonométrica.
\end{ex}

\begin{exo}
Mostre, usando uma integral, que a circunferência de um disco de raio $R$ é $2\pi R$.
\begin{sol} Representando a metade superior do círculo de raio $R$ centrado na origem com a função $f(x)=\sqrt{R^2-x^2}$, podemos expressar o 
comprimento da circunferência como
$$
2\int_{-R}^R\sqrt{1+[(\sqrt{R^2-x^2})']^2}\,dx=2R\int_{-R}^R\frac{dx}{\sqrt{R^2-x^2}}=2R\int_{-1}^1\frac{du}{\sqrt{1-u^2}}=2\pi R\,.
$$
\end{sol}
\end{exo}

\begin{exo}
Calcule o comprimento da corda pendurada entre dois pontos $A$ e $B$, descrita pelo gráfico
 da função $f(x)=\cosh x$, entre $x=-1$ e $x=1$.  
\begin{sol}
Lembrando que $\cosh'(x)=\senh x$, que $\cosh^2 x-\senh^2x=1$, e que $\cosh x$ é par,
\begin{align*}
 L=\int_{-1}^1\sqrt{1+(\senh x)^2}\,dx=2\int_{0}^1\cosh x\,dx=2\senh (1)=e-e^{-1}\,.
\end{align*} 
\end{sol}
\end{exo}

\begin{exo}\label{Exo_finalmente}
Calcule o comprimento do gráfico da função exponencial $f(x)=e^x$, entre $x=0$
e $x=1$. (\emph{Dica}: $u=\sqrt{1+e^{2x}}$.)
\begin{sol}
O comprimento é dado por $L=\int_0^1\sqrt{1+e^{2x}}\,dx$.
Se $u=\sqrt{1+e^{2x}}$, então $dx=\frac{u}{u^2-1}du$, logo
$$L=\int_{\sqrt{2}}^{\sqrt{1+e^4}}\frac{u^2}{u^2-1}du
=\int_{\sqrt{2}}^{\sqrt{1+e^4}}1\,du+\int_{\sqrt{2}}^{\sqrt{1+e^4}}\frac{du}{
u^2-1}\,.
$$ 
Essa última integral pode ser calculada como no Exemplo \ref{Ex:unsurxdeuxmun}:
$\int\frac{du}{
u^2-1}=\tfrac{1}{2}\ln\Bigl|\frac{u-1}{u+1}\Bigr|+C$. Logo,
$$
L=\sqrt{1+e^4}-\sqrt{2}+\tfrac12\ln\Bigl[\frac{\sqrt{1+e^4}-1}{\sqrt{1+e^4}
+1}\cdot\frac{\sqrt{2}+1}{\sqrt{2}-1}\Bigr]\,.
$$
\end{sol}
\end{exo}

\section{Sólidos de revolução}\label{Sec_Solidos}
\index{sólidos de revolução}
Nesta seção usaremos a integral para calcular o volume de um 
tipo particular de região do
espaço, chamada de \emph{sólidos de revolução}. (Em Cálculo III, volumes de regiões mais
gerais serão calculados usando integral tripla.)\\

Considere uma função \emph{positiva} no intervalo $[a,b]$, $f:[a,b]\to
\bR_+$. Seja $R$ a região 
delimitada pelo gráfico de $f$, pelo eixo $x$ e pelas retas $x=a$,
$x=b$:
\begin{center}
\begin{bmlimage}\begin{tikzpicture}[scale=2]
\newcommand{\funcao}[1]{{1/((#1)^2+1)}}
\pgfmathsetmacro{\a}{0.2};
\pgfmathsetmacro{\b}{1.5};
\pgfmathsetmacro{\c}{(\a+\b)/2};
\draw[->] (0,-0.1)--(0,1.3);
\fill[areagrafico] (\a,0)--plot[domain=\a:\b](\x,\funcao{\x})
--(\b,0)--cycle;
\draw[->] (-0.1,0)--({\b+0.3},0) node[right]{$x$};
\draw[thick, domain=\a:\b] plot (\x,\funcao{\x});
\draw[dotted] (\a,0) node[below]{$a$} -- (\a,\funcao{\a});
\draw[dotted] (\b,0) node[below]{$b$} -- (\b,\funcao{\b});
\draw (\c,{(\funcao{\c})*0.5}) node{$R$};
\draw (\c,{(\funcao{\c})*1.2}) node[above]{$f(x)$};
\end{tikzpicture}\end{bmlimage}
\end{center}
Sabemos que a área de $R$ é dada pela integral de Riemann 
$$\text{área}(R)=\int_a^bf(x)\,dx\,.$$

Consideremos agora o
sólido $S$ obtido girando a região $R$ em torno do
eixo $x$, como na figura abaixo:

\begin{center}
\begin{bmlimage}\begin{tikzpicture}[scale=2]
\tdplotsetmaincoords{55}{115}

\newcommand{\funcao}[1]{(1/((#1)^2+1))}

\pgfmathsetmacro{\a}{0.2};
\pgfmathsetmacro{\b}{1.5};

\begin{scope}[tdplot_main_coords]
\pgfmathsetmacro{\nx}{20};
\pgfmathsetmacro{\Dx}{(\b-\a)/\nx};

\pgfmathsetmacro{\alphamin}{0};
\pgfmathsetmacro{\alphamax}{20};
\pgfmathsetmacro{\nalpha}{10};
\pgfmathsetmacro{\Dalpha}{(\alphamax-\alphamin)/\nalpha};


\draw[->] (0,0,0) -- (1.2,0,0);
\draw[->] (0,0,0) -- (0,{\b+0.3},0) node[anchor=north west]{$x$};
\draw[->] (0,0,0) -- (0,0,1.2);

%desenhar a flechinha de tras:
\draw[->, domain=0:330, variable=\alpha, samples=50] plot
({sin(\alpha)*(\funcao{\a})},\a,{cos(\alpha)*(\funcao{\a})}); 
%desenhar a linha pontilhada de tras:
\draw[thick, dotted] (0,\a,0)--(0,\a,{\funcao{\a}});

\fill[areagrafico]
(0,\a,0)--plot[domain=\a:\b](0,\x,{\funcao{\x}})--(0,\b,0)--cycle;
%desenhar o grafico de f em [a,b]
\draw[thick, domain=\a:\b, variable=\t] plot
(0,\t,{\funcao{\t}}); 
%desenhar a flechinha de frente:
\draw[->, domain=0:320, variable=\alpha] plot
({sin(\alpha)*\funcao{\b}},\b,{cos(\alpha)*\funcao{\b}}); 
%desenhar a linha pontilhada de frente:
\draw[thick,dotted] (0,\b,0)--(0,\b,{\funcao{\b}});
\end{scope}

\begin{scope}[xshift=2.5cm,tdplot_main_coords]
\pgfmathsetmacro{\nx}{15};
\pgfmathsetmacro{\Dx}{(\b-\a)/\nx};
\pgfmathsetmacro{\alphamin}{-40};
\pgfmathsetmacro{\alphamax}{150};
\pgfmathsetmacro{\nalpha}{35};
\pgfmathsetmacro{\Dalpha}{(\alphamax-\alphamin)/\nalpha};

\draw[->] (0,0,0) -- (1.2,0,0);
\draw[->] (0,0,0) -- (0,{\b+0.3},0) node[anchor=north west]{$x$};
\draw[->] (0,0,0) -- (0,0,1.2);
%desenhar o grande circulo de tras:
\draw[domain=0:360, variable=\alpha] plot
({sin(\alpha)*\funcao{\a}},\a,{cos(\alpha)*\funcao{\a}}); 
%desenhar a linha pontilhada de tras:
\draw[thick, dotted] (0,\a,0)--(0,\a,{\funcao{\a}});
%desenhar o grande circulo de frente:
\draw[domain=0:360, variable=\alpha] plot
({sin(\alpha)*\funcao{\b}},\b,{cos(\alpha)*\funcao{\b}}); 

\fill[areagrafico]
(0,\a,0)--plot[domain=\a:\b](0,\x,{\funcao{\x}})--(0,\b,0)--cycle;
\foreach \i in {0,...,\nalpha} {
\draw[thick, color=gray, domain=\a:\b, variable=\t] plot
({sin(\alphamin+\Dalpha*\i)*\funcao{\t}},\t,{cos(\alphamin+\Dalpha*\i)*\funcao{\t}}); 
}
\foreach \i in {0,...,\nx} {
\pgfmathsetmacro{\point}{\a+(\i*\Dx)};
\draw[thick, color=gray, domain=\alphamin:\alphamax, variable=\alpha] plot
({sin(\alpha)*\funcao{\point}},\point,{cos(\alpha)*\funcao{\point}}); 
 }
%desenhar o grafico de f em [a,b]
\draw[thick, thick, domain=\a:\b, variable=\t] plot
(0,\t,{\funcao{\t}}); 
%desenhar a linha pontilhada de frente:
\draw[thick,dotted] (0,\b,0)--(0,\b,{\funcao{\b}});
\end{scope}

\begin{scope}[xshift=5cm,tdplot_main_coords]
\pgfmathsetmacro{\nx}{20};
\pgfmathsetmacro{\Dx}{(\b-\a)/\nx};
\pgfmathsetmacro{\alphamin}{-40};
\pgfmathsetmacro{\alphamax}{150};
\pgfmathsetmacro{\nalpha}{40};
\pgfmathsetmacro{\Dalpha}{(\alphamax-\alphamin)/\nalpha};


\foreach \i in {0,...,\nalpha} {
\draw[thick, color=gray, domain=\a:\b, variable=\t] plot
({sin(\alphamin+\Dalpha*\i)*\funcao{\t}},\t,{cos(\alphamin+\Dalpha*\i)*\funcao{\t}}); 
}
\foreach \i in {0,...,\nx} {
\pgfmathsetmacro{\point}{\a+(\i*\Dx)};
\draw[thick, color=gray, domain=\alphamin:\alphamax, variable=\alpha] plot
({sin(\alpha)*\funcao{\point}},\point,{cos(\alpha)*\funcao{\point}}); 
 }

%Encher o disco de frente:
\filldraw[color=gray] plot[domain=0:360, variable=\alpha] 
({sin(\alpha)*\funcao{\b}},\b,{cos(\alpha)*\funcao{\b}}); 

\draw (0,\b,{\funcao{\b}+0.5}) node[left]{$S$};
\end{scope}
\end{tikzpicture}\end{bmlimage}
\end{center}

Sólidos que podem ser gerados dessa maneira, girando uma região em torno de
um eixo, são chamados de
\grasA{sólidos de revolução}. 
Veremos situações em que a região não precisa ser delimitada pelo 
gráfico de uma função, e que o eixo não precisa ser o eixo $x$.	

\begin{exo}
Quais dos seguintes 
corpos são sólidos de revolução? (Quando for o caso, dê a região e
o eixo)
\begin{enumerate}
\item\label{itexsolrev1} A esfera de raio $r$.
\item\label{itexsolrev2} O cilindro com base circular de raio $r$, e de altura $h$.
\item\label{itexsolrev3} O cubo de lado $L$.
\item\label{itexsolrev4} O cone de base circular de raio $r$ e de altura $h$.
\item\label{itexsolrev5} O toro de raios $0<r<R$.
\end{enumerate}
\begin{sol}
\eqref{itexsolrev1} A esfera pode ser obtida girando o semi-disco,
delimitado pelo gráfico da função
$f(x)=\sqrt{r^2-x^2}$, $x\in [-r,r]$, em torno do eixo $x$.
\eqref{itexsolrev2} O cilíndro pode ser obtido girando o gráfico da função
constante $f(x)=r$, no intervalo $[0,h]$.
\eqref{itexsolrev3} O cubo não é um sólido de revolução.
\eqref{itexsolrev4} O cone pode ser obtido girando o gráfico da função
$f(x)=\frac{r}{h}x$ (ou $f(x)=r-\frac{r}{h}x$), no intervalo $[0,h]$. 
\end{sol}
\end{exo}


Nesta seção desenvolveremos 
métodos para \emph{calcular o volume $V(S)$ de um sólido de
revolução $S$}. Antes de começar, consideremos um caso elementar, que será também usado para o caso geral.

\begin{ex}\label{Ex_vol_cilindro}
Suponha que $f$ é constante em $[a,b]$, isto é: $f(x)=r>0$ para todo $x\in [a,b]$:

%
\begin{center}
\begin{bmlimage}\begin{tikzpicture}[scale=2]
\tdplotsetmaincoords{55}{115}
\pgfmathsetmacro{\r}{0.6};
\newcommand{\funcao}[1]{(\r)}
\pgfmathsetmacro{\a}{0.2};
\pgfmathsetmacro{\b}{1.5};
\pgfmathsetmacro{\c}{(\a+\b)/2};
\draw[->] (0,-0.1)--(0,\r+0.3);
\fill[areagrafico]
(\a,0)--plot[domain=\a:\b](\x,{\funcao{\x}})--(\b,0)--cycle;
\draw[->] (-0.1,0)--({\b+0.3},0);
\draw[thick, domain=\a:\b] plot (\x,{\funcao{\x}});
\draw[dotted] (\a,0) node[below]{$a$} --
(\a,{\funcao{\a}});
\draw[dotted] (\b,0) node[below]{$b$} --
(\b,{\funcao{\b}});
%\draw (\c,{0.5*\funcao{\c}}) node{$R$};
\draw[dashed] (-0.1,\r)--(\b,\r);
\draw (-0.1,\r) node[left]{$r$};

\begin{scope}[xshift=3.5cm, yshift=0.5cm, tdplot_main_coords]
\pgfmathsetmacro{\nx}{20};
\pgfmathsetmacro{\Dx}{(\b-\a)/\nx};
\pgfmathsetmacro{\alphamin}{-40};
\pgfmathsetmacro{\alphamax}{150};
\pgfmathsetmacro{\nalpha}{40};
\pgfmathsetmacro{\Dalpha}{(\alphamax-\alphamin)/\nalpha};

\draw[->] (0,0,0) -- (1.2,0,0);
\draw[->] (0,0,0) -- (0,{\b+0.3},0) node[anchor=north west]{$x$};
\draw[->] (0,0,0) -- (0,0,1.2);
%desenhar o grande circulo de tras:
\draw[domain=0:360, variable=\alpha] plot
({sin(\alpha)*\funcao{\a}},{\a},{cos(\alpha)*\funcao{\a}}); 
%desenhar a linha pontilhada de tras:
\draw[thick, dotted] (0,\a,0)--(0,\a,{\funcao{\a}});
%desenhar o grande circulo de frente:
\draw[domain=0:360, variable=\alpha] plot
({sin(\alpha)*\funcao{\b}},{\b},{cos(\alpha)*\funcao{\b}}); 

\fill[areagrafico]
(0,\a,0)--plot[domain=\a:\b](0,\x,{\funcao{\x}})--(0,\b,0)--cycle;
\foreach \i in {0,...,\nalpha} {
\draw[thick, color=gray, domain=\a:\b, variable=\t] plot
({sin(\alphamin+\Dalpha*\i)*\funcao{\t}},\t,{cos(\alphamin+\Dalpha*\i)*\funcao{\t}}); 
}
\foreach \i in {0,...,\nx} {
\pgfmathsetmacro{\point}{\a+(\i*\Dx)};
\draw[thick, color=gray, domain=\alphamin:\alphamax, variable=\alpha] plot
({sin(\alpha)*\funcao{\point}},\point,{cos(\alpha)*\funcao{\point}}); 
 }
%desenhar o grafico de f em [a,b]
\draw[thick, thick, domain=\a:\b, variable=\t] plot
(0,\t,{\funcao{\t}}); 
%desenhar a linha pontilhada de frente:
\draw[thick,dotted] (0,\b,0)--(0,\b,{\funcao{\b}});
\draw[thick, ->] (0,\b,0)--({sin(120)*\funcao{\b}},\b,{cos(120)*\funcao{\b}})
node[midway, right]{$r$};
\end{scope}
\end{tikzpicture}\end{bmlimage}
\end{center}
%
Neste caso, o sólido gerado $S$ é um cilindro (deitado). 
\index{cilindro}
A sua base é circular de
raio $r$, e a sua altura é $b-a$. Pela fórmula bem conhecida do volume
de um cilíndro,
\begin{equation}
V(S)=\text{área da base }\times\text{ altura}=\pi r^2(b-a)\,.
\end{equation}
\end{ex}

Queremos agora calcular $V(S)$ 
para um sólido de revolução qualquer.\\

O procedimento será o mesmo que levou à propria definição
\index{integral de Riemann}da integral de Riemann:
\emph{aproximaremos $S$ por sólidos mais elementares}. Usaremos dois tipos
de sólidos elementares: cilíndros e cascas. 


\subsection{Aproximação por cilindros}
\index{aproximação!por cilindros}
Voltemos para 
o sólido de revolução da seção anterior. Um jeito de decompor o sólido $S$ 
é de aproximá-lo por uma união de fatias verticais, centradas no eixo
$x$:

\begin{center}
\begin{bmlimage}\begin{tikzpicture}[scale=2]
\tdplotsetmaincoords{55}{115}

\newcommand{\funcao}[1]{( 1/((#1)^2+1) )}
\pgfmathsetmacro{\a}{0.2};
\pgfmathsetmacro{\b}{1.5};
\pgfmathsetmacro{\c}{(\a+\b)/2};

%nombre de tranches:
\pgfmathsetmacro{\n}{10};

\draw[->] (0,-0.1)--(0,1.3);
\pgfmathsetmacro{\dx}{(\b-\a)/\n};

\foreach \i in {1,...,\n} {
\filldraw[corretangulos] ({\a+(\i-1)*\dx},0)
rectangle ({\a+\i*\dx},{\funcao{\a+\i*\dx}});
\draw ({\a+(\i-1)*\dx},0)
rectangle ({\a+\i*\dx},{\funcao{\a+\i*\dx}});
\fill ({\a+\i*\dx},{\funcao{\a+\i*\dx}}) circle (0.2mm);
}

\draw[->] (-0.1,0)--({\b+0.3},0);
\draw[thick, domain=\a:\b] plot (\x,{\funcao{\x}});
\draw[dotted] (\a,0) node[below]{$a$} --
(\a,{\funcao{\a}});
\draw[dotted] (\b,0) node[below]{$b$} --
(\b,{\funcao{\b}});

\begin{scope}[xshift=3cm, yshift=0.5cm, tdplot_main_coords]

\pgfmathsetmacro{\nx}{2};
\pgfmathsetmacro{\Dx}{\n/\nx};
\pgfmathsetmacro{\alphamin}{-40};
\pgfmathsetmacro{\alphamax}{150};
\pgfmathsetmacro{\nalpha}{40};
\pgfmathsetmacro{\Dalpha}{(\alphamax-\alphamin)/\nalpha};

%dessiner les axes x et z
\draw[->] (0,0,0) -- (1.2,0,0);
\draw[->] (0,0,0) -- (0,0,1.2);

%dessiner le traitille de derriere:
\draw[dotted] (0,\a,0)--(0,\a,{\funcao{\a}});


%dessiner les tranches:
\foreach \i in {1,...,\n} {
\pgfmathsetmacro{\ap}{(\a+(\i-1)*\dx)};
\pgfmathsetmacro{\bp}{(\a+(\i*\dx))};
\pgfmathsetmacro{\valf}{ \funcao{\bp} };
%remplir le premier cercle (derriere) de la tranche:
\filldraw[corretangulos, domain=0:360, variable=\alpha] plot
({sin(\alpha)*\valf},{\ap},{cos(\alpha)*\valf}); 
%prononcer un peu le bord:
\draw[color=gray, domain=0:360, variable=\alpha] plot
({sin(\alpha)*\valf},{\ap},{cos(\alpha)*\valf}); 
%dessiner les petits traits de la tranche:
\foreach \j in {0,...,\nalpha} {
\draw[color=gray, domain=\ap:\bp, variable=\t] plot
({sin(\alphamin+\Dalpha*\j)*\valf},\t,{cos(\alphamin+\Dalpha*\j)*\valf}); 
%remplir le disque (devant) de la tranche:
\filldraw[corretangulos, domain=0:360, variable=\alpha] plot
({sin(\alpha)*\valf},\bp,{cos(\alpha)*\valf}); 
%prononcer un peu le bord de ce disque
\draw[color=gray, domain=0:360, variable=\alpha] plot
({sin(\alpha)*\valf},\bp,{cos(\alpha)*\valf}); 
\fill (0,\bp,{\funcao{\bp}}) circle (0.2mm);
}
}

%dessiner le graphe de la fonction
\draw[thick, domain=\a:\b, variable=\t] plot
(0,\t,{\funcao{\t}});
%dessiner le petit traitille:
\draw[dotted] (0,\b,0)--(0,\b,{\funcao{\b}});

\draw[->] (0,\b,0) -- (0,{\b+0.3},0) node[right]{$x$};

\end{scope}
\end{tikzpicture}\end{bmlimage}
\end{center}
%
Cada fatia é obtida girando um retângulo cujo tamanho é determinado pela função $f$. Para
ser mais preciso, escolhemos pontos no intervalo $[a,b]$,
$x_0\equiv a<x_1<x_2<\dots<x_{n-1}<x_n\equiv b$, e a cada intervalo
$[x_{i-1},x_{i}]$ associamos o retângulo cuja base tem tamanho
$(x_i-x_{i-1})$ e cuja altura é de $f(x_i)$. Ao
girar em torno do eixo $x$, cada um desses retângulos gera uma
fatia cilíndrica $F_i$, como no Exemplo
\ref{Ex_vol_cilindro}:

\begin{center}
\begin{bmlimage}\begin{tikzpicture}[scale=2]
\tdplotsetmaincoords{55}{115}
\newcommand{\funcao}[1]{( 1/((#1)^2+1) )}
\pgfmathsetmacro{\a}{0.2};
\pgfmathsetmacro{\b}{1.5};
\pgfmathsetmacro{\c}{(\a+\b)/2};

%nombre de tranches:
\pgfmathsetmacro{\n}{10};

\draw[->] (0,-0.1)--(0,1.3);
%\fill[areagrafico] (\a,0)--plot[domain=\a:\b](\x,{\funcao{}})--(\b,0)--cycle;

\pgfmathsetmacro{\dx}{(\b-\a)/\n};

\foreach \i in {\n/2} {
%\draw ({\a + \i*\dx},0)--(1,1);
\filldraw[corretangulos] ({\a+(\i-1)*\dx},0)
rectangle ({\a+\i*\dx},{\funcao{\a+\i*\dx}});
\draw ({\a+(\i-1)*\dx},0)
rectangle ({\a+\i*\dx},{\funcao{\a+\i*\dx}});
\fill ({\a+\i*\dx},{\funcao{\a+\i*\dx}}) circle (0.2mm);
\draw[<-] ({\a+(\i-1)*\dx},-0.02)--
({\a+(\i-1)*\dx-0.2},-0.25) node[below]{$x_{i-1}$};
\draw[<-] ({\a+(\i)*\dx},-0.02)--({\a+(\i)*\dx+0.2},-0.25) node[below]{$x_{i}$}; 
}

\draw[->] (-0.1,0)--({\b+0.3},0);
\draw[thick, domain=\a:\b] plot (\x,{\funcao{\x}});
\draw[dotted] ({\a},0) node[below]{$a$} --
(\a,{\funcao{\a}});
\draw[dotted] (\b,0) node[below]{$b$} -- (\b,{\funcao{\b}});

\begin{scope}[xshift=3.5cm, yshift=0cm, tdplot_main_coords]

\pgfmathsetmacro{\nx}{2};
\pgfmathsetmacro{\Dx}{\n/\nx};
\pgfmathsetmacro{\alphamin}{-40};
\pgfmathsetmacro{\alphamax}{150};
\pgfmathsetmacro{\nalpha}{40};
\pgfmathsetmacro{\Dalpha}{(\alphamax-\alphamin)/\nalpha};

%dessiner les axes x et z
\draw[->] (0,0,0) -- (1.2,0,0);
\draw[->] (0,0,0) -- (0,0,1.2);

%dessiner le traitille de derriere:
\draw[dotted] (0,\a,0)--(0,\a,{\funcao{\a}});

\foreach \i in {\n/2} {
\pgfmathsetmacro{\ap}{(\a+(\i-1)*\dx)};
\draw (0,0,0)--(0,{\ap},0);
\pgfmathsetmacro{\bp}{\a+(\i*\dx)};
\pgfmathsetmacro{\valf}{\funcao{\bp}};
%remplir le premier cercle (derriere) de la tranche:

\filldraw[corretangulos, domain=0:360, variable=\alpha] plot
({sin(\alpha)*\valf},{\ap},{cos(\alpha)*\valf}); 

%prononcer un peu le bord:
\draw[color=gray, domain=0:360, variable=\alpha] plot
({sin(\alpha)*\valf},{\ap},{cos(\alpha)*\valf}); 

%dessiner les petits traits de la tranche:
\foreach \j in {0,...,\nalpha} {
\draw[color=gray, domain=\ap:\bp, variable=\t] plot
({sin(\alphamin+\Dalpha*\j)*\valf},{\t},{cos(\alphamin+\Dalpha*\j)*\valf}); 
%remplir le disque (devant) de la tranche:
\filldraw[corretangulos, domain=0:360, variable=\alpha] plot
({sin(\alpha)*\valf},{\bp},{cos(\alpha)*\valf}); 
%prononcer un peu le bord de ce disque
\draw[color=gray, domain=0:360, variable=\alpha] plot
({sin(\alpha)*\valf},{\bp},{cos(\alpha)*\valf}); 
\fill (0,\bp,{\funcao{\bp}}) circle (0.2mm);
}

%terminer le bout de l'axe des x:
\draw[thin] (0,{\bp},0)--(0,{\b},0);
\draw[thin] ({1.3*\valf},{\bp},0) node[left]{$F_i$};
}

%dessiner le graphe de la fonction
\draw[thick, domain=\a:\b, variable=\t] plot
(0,\t,{\funcao{\t}});
%dessiner le petit traitille:
\draw[dotted] (0,{\b},0)--(0,\b,{\funcao{\b}});
%terminer l'axe des x:
\draw[->] (0,{\b},0) -- (0,{\b+0.3},0) node[right]{$x$};

\end{scope}

\end{tikzpicture}\end{bmlimage}
\end{center}


Mas, como a fatia $F_i$ é um cilindro deitado de 
raio $f(x_i)$ e de altura $\Delta x_i=x_i-x_{i-1}$, o seu
volume é dado por $V(F_i)=\pi f(x_i)^2\Delta x_i$.
Logo, o volume do sólido $S$ pode ser aproximado 
pela soma dos volumes das fatias, que é 
uma soma de Riemann:
$$
\sum_{j=1}^nV(F_i)=\sum_{i=1}^n\pi f(x_i)^2\Delta x_i\,.
$$
Quando o número de retângulos
$n\to \infty$ e que todos os $\Delta x_i\to 0$, esta soma
converge (quando $f(x)^2$ pe contínua, por exemplo) 
para a uma integral de
Riemann que permite (em princípio) calcular o volume exato do sólido $S$:
\begin{equation}\label{eq_formula_volume}
\boxed{V(S)=\int_a^b\pi f(x)^2\,dx\,.}
\end{equation}

\begin{ex}\label{ex_giraseno}
Seja $R$ a região delimitada pela curva $y=\sen x$, pelo eixo $x$, e pelas duas retas
verticais $x=0$ e $x=\pi$.
Calculemos o volume do sólido $S$ obtido girando $R$ em torno do eixo $x$:

\begin{center}
\begin{bmlimage}\begin{tikzpicture}[scale=1.7]
\tdplotsetmaincoords{55}{115}

%\newcommand{\funcao}[1]{{(1-(#1)^2)}}
\newcommand{\funcao}[1]{(sin(#1 r))}

\pgfmathsetmacro{\a}{0};
\pgfmathsetmacro{\b}{pi};

\begin{scope}
\fill[areagrafico]
(\a,0)--plot[domain=\a:\b](\x,{\funcao{\x}})--(\b,0)--cycle;
\draw[thick] (\a,0) plot[domain=\a:\b](\x,{\funcao{\x}});
\draw[->] (\a-0.2,0)--(\b+0.2,0);
\draw[->] (0,-0.2)--(0,1.2);
\draw (pi,0) node[below]{$\pi$};
\end{scope}

\begin{scope}[xshift=5cm, yshift=0.7cm, tdplot_main_coords]
\pgfmathsetmacro{\nx}{20};
\pgfmathsetmacro{\Dx}{(\b-\a)/\nx};

\draw[->] (0,0,0) -- (1.2,0,0);
\draw[->] (0,0,0) -- (0,{\b+0.3},0) node[anchor=north west]{$x$};
\draw[->] (0,0,0) -- (0,0,1.2);
\fill[areagrafico]
(0,\a,0)--plot[domain=\a:\b](0,\x,{\funcao{\x}})--(0,\b,0)--cycle;
%desenhar a flechinha de frente:
\draw[->, domain=0:320, variable=\alpha] plot
({sin(\alpha)*\funcao{\b}},\b,{cos(\alpha)*\funcao{\b}}); 
%desenhar a linha pontilhada de frente:
\draw[thick,dotted] (0,\b,0)--(0,\b,{\funcao{\b}});
\draw (0,pi,0) node[above right]{$\pi$};

\pgfmathsetmacro{\alphamin}{-50};
\pgfmathsetmacro{\alphamax}{150};
\pgfmathsetmacro{\nalpha}{50};
\pgfmathsetmacro{\Dalpha}{(\alphamax-\alphamin)/\nalpha};

%Encher o grande disco de tras:
% \filldraw[color=gray!50] plot[domain=0:360, variable=\alpha] 
% ({sin(\alpha)*\funcao{\a}},\a,{cos(\alpha)*\funcao{\a}}); 

\foreach \i in {0,...,\nalpha} {
\draw[thick, color=gray, domain=\a:\b, variable=\t] plot
({sin(\alphamin+\Dalpha*\i)*\funcao{\t}},\t,{cos(\alphamin+\Dalpha*\i)*\funcao{\t}}); 
}
\foreach \i in {0,...,\nx} {
\pgfmathsetmacro{\point}{\a+(\i*\Dx)};
\draw[thick, color=gray, domain=\alphamin:\alphamax, variable=\alpha] plot
({sin(\alpha)*\funcao{\point}},\point,{cos(\alpha)*\funcao{\point}}); 
}

\draw[thick, domain=\a:\b, variable=\t] plot
(0,\t,{\funcao{\t}}); 
\end{scope}
\end{tikzpicture}\end{bmlimage}
\end{center}
%AQUI


Pela fórmula 
\eqref{eq_formula_volume}, o volume deste sólido é dado pela integral 
$$V=\int_{0}^\pi\pi(\sen x)^2\,dx=\pi \Bigl\{
\frac{x}{2}-\frac{\sen (2x)}{4}
\Bigr\}\Big|_{0}^\pi=\tfrac12 \pi^2\,.
$$

\end{ex}


O método permite calcular volumes clássicos da geometria.

\begin{ex}\label{ex_giradisco}
Seja $r>0$ fixo e $R$ a região delimitada pela semi-circunferência 
$y=\sqrt{r^2-x^2}$, entre $x=-r$ e $x=+r$, e pelo eixo $x$.
O sólido $S$ obtido girando $R$ em torno do eixo $x$ é uma esfera
\index{esfera} de raio $r$ centrada na origem:

\begin{center}
\begin{bmlimage}\begin{tikzpicture}[scale=1.7]
\tdplotsetmaincoords{55}{105}

%\newcommand{\funcao}[1]{{(1-(#1)^2)}}
\pgfmathsetmacro{\r}{1};
\newcommand{\funcao}[1]{(sqrt((\r)^2-(#1)^2))}
\pgfmathsetmacro{\a}{-\r};
\pgfmathsetmacro{\b}{\r};

\begin{scope}
\fill[areagrafico]
(\a,0)--plot[domain=\a:\b, samples=200](\x,{\funcao{\x}})--(\b,0)--cycle;
\draw[thick] (\a,0) plot[domain=\a:\b, samples=220](\x,{\funcao{\x}});
\draw[->] (\a-0.2,0)--(\b+0.2,0);
\draw[->] (0,-0.2)--(0,1.2);
\draw (\a,0) node[below]{$-r$};
\draw (\b,0) node[below]{$+r$};
\end{scope}

%\iffalse
\begin{scope}[xshift=4cm, yshift=0.2cm, tdplot_main_coords]
\pgfmathsetmacro{\nx}{40};
\pgfmathsetmacro{\Dx}{(\b-\a)/\nx};

\draw (0,0,0) -- (\r,0,0);
\draw[->] (0,0,0) -- (0,{\b+0.3},0) node[anchor=north west]{$x$};

% Dessiner la region R:
\fill[areagrafico]
(0,\a,0)--plot[domain=\a:\b](0,\x,{\funcao{\x}})--(0,\b,0)--cycle;

\pgfmathsetmacro{\alphamin}{-50};
\pgfmathsetmacro{\alphamax}{150};
\pgfmathsetmacro{\nalpha}{50};
\pgfmathsetmacro{\Dalpha}{(\alphamax-\alphamin)/\nalpha};

% Dessiner les cercles inclines:
\foreach \i in {0,...,\nalpha} {
\draw[thick, color=gray, domain=\a:\b, variable=\t] plot
({sin(\alphamin+\Dalpha*\i)*\funcao{\t}},\t,{cos(\alphamin+\Dalpha*\i)*\funcao{\t}}); 
}
%Dessiner les tranches verticales:
\foreach \i in {0,...,\nx} {
\pgfmathsetmacro{\point}{\a+(\i*\Dx)};
\draw[thick, color=gray, domain=\alphamin:\alphamax, variable=\alpha,
samples=220] plot
({sin(\alpha)*\funcao{\point}},\point,{cos(\alpha)*\funcao{\point}}); 
}

\draw[thick, domain=\a:\b, variable=\t, samples=220] plot
(0,\t,{\funcao{\t}}); 

% Le petit bout de l'axe z:
\draw[->] (0,0,\r) -- (0,0,\r+0.5);
% Le petit bout de l'axe x:
\draw[->] (\r,0,0) -- (\r+0.8,0,0);
\end{scope}
%\fi
\end{tikzpicture}\end{bmlimage}
\end{center}

Pela fórmula 
\eqref{eq_formula_volume}, o volume da esfera é dado pela integral 
\begin{align*}
V(\text)&=\int_{-r}^{+r}\pi\bigl(\sqrt{r^2-x^2}\bigr)^2\,dx\\
&=\pi\int_{-r}^{+r}(r^2-x^2)\,dx\\
&=\pi\Bigl\{r^2x-\tfrac{x^3}{3}\Bigr\}\Big|_{-r}^{+r}\\
&=\frac{4}{3}\pi r^3\,...
\end{align*}

\end{ex}


\begin{exo}
Um vaso \'e obtido rodando a curva $y=f(x)$ em torno do eixo
$x$, onde
$$
f(x)=
\begin{cases}
-x+3&\text{ se }0\leq x\leq 2,\\
x-1&\text{ se }2<x\leq 3\,.
\end{cases}
$$
Esboce o vaso obtido, em três dimensões, e calcule o seu volume.
\begin{sol} $11\pi$
 \end{sol}
\end{exo}

O importante, nesta seção, é de não tentar
\emph{decorar fórmulas}, e sim entender como montar uma nova
fórmula em cada situação. Vejamos como, no seguinte exemplo.

\begin{ex}\label{ex_gira_parabola}
Considere a região $R$ do primeiro quadrante, delimitada pelo gráfico
da função $f(x)=1-x^2$. Considere os sólidos $S_1$ e $S_2$, obtidos
rodando $R$ em torno, respectivamente, do eixo $x$ e $y$:

\begin{center}
\begin{bmlimage}\begin{tikzpicture}[scale=1.7]
\tdplotsetmaincoords{55}{115}

\newcommand{\funcao}[1]{(1-(#1)^2)}
%\newcommand{\funcao}[1]{{sin(#1 r)}}
\pgfmathsetmacro{\a}{0};
\pgfmathsetmacro{\b}{1};

\begin{scope}
\fill[areagrafico]
(\a,0)--plot[domain=\a:\b](\x,{\funcao{\x}})--(\b,0)--cycle;
\draw[thick] (\a,0) plot[domain=\a:\b](\x,{\funcao{\x}});
\draw[->] (\a-0.2,0)--(\b+0.2,0);
\draw[->] (0,-0.2)--(0,1.2);
\draw (\b,0) node[below]{$1$};
\end{scope}

\begin{scope}[xshift=3cm, yshift=0.6cm, tdplot_main_coords]
\pgfmathsetmacro{\nx}{20};
\pgfmathsetmacro{\Dx}{(\b-\a)/\nx};

%Encher o grande disco de tras:
 \filldraw[color=gray!50] plot[domain=0:360, variable=\alpha] 
 ({sin(\alpha)*\funcao{\a}},\a,{cos(\alpha)*\funcao{\a}}); 

\draw[->] (0,0,0) -- (1.2,0,0);
\draw[->] (0,0,0) -- (0,{\b+0.3},0) node[anchor=north west]{$x$};
\draw[->] (0,0,0) -- (0,0,1.2);

\fill[areagrafico]
(0,\a,0)--plot[domain=\a:\b](0,\x,{\funcao{\x}})--(0,\b,0)--cycle;

%desenhar a flechinha de frente:
\draw[->, domain=0:320, variable=\alpha] plot
({sin(\alpha)*\funcao{\b}},\b,{cos(\alpha)*\funcao{\b}}); 

%desenhar a linha pontilhada de frente:
\draw[thick,dotted] (0,\b,0)--(0,\b,{\funcao{\b}});
\draw (0,1,0) node[below right]{$1$};

\pgfmathsetmacro{\alphamin}{-50};
\pgfmathsetmacro{\alphamax}{150};
\pgfmathsetmacro{\nalpha}{50};
\pgfmathsetmacro{\Dalpha}{(\alphamax-\alphamin)/\nalpha};


\foreach \i in {0,...,\nalpha} {
\draw[thick, color=gray, domain=\a:\b, variable=\t] plot
({sin(\alphamin+\Dalpha*\i)*\funcao{\t}},\t,{cos(\alphamin+\Dalpha*\i)*\funcao{\t}}); 
}
\foreach \i in {0,...,\nx} {
\pgfmathsetmacro{\point}{\a+(\i*\Dx)};
\draw[thick, color=gray, domain=\alphamin:\alphamax, variable=\alpha] plot
({sin(\alpha)*\funcao{\point}},\point,{cos(\alpha)*\funcao{\point}}); 
}

\draw[thick, domain=\a:\b, variable=\t] plot
(0,\t,{\funcao{\t}}); 
\draw (0.9,0,1.2) node{$S_1$};
\end{scope}

\begin{scope}[xshift=6.3cm, yshift=0.5cm, tdplot_main_coords]
\pgfmathsetmacro{\nx}{20};
\pgfmathsetmacro{\Dx}{(\b-\a)/\nx};

%Encher o grande disco debaixo:
 \filldraw[color=gray!50] plot[domain=0:360, variable=\alpha] 
 ({sin(\alpha)*\funcao{\a}},{cos(\alpha)*\funcao{\a}},0); 

\draw[->] (0,0,0) -- (1.2,0,0);
\draw[->] (0,0,0) -- (0,{\b+0.3},0) node[anchor=north west]{$x$};
\draw[->] (0,0,0) -- (0,0,1.2);

\fill[areagrafico]
(0,\a,0)--plot[domain=\a:\b](0,\x,{\funcao{\x}})--(0,\b,0)--cycle;

\pgfmathsetmacro{\alphamin}{0};
\pgfmathsetmacro{\alphamax}{360};
\pgfmathsetmacro{\nalpha}{50};
\pgfmathsetmacro{\Dalpha}{(\alphamax-\alphamin)/\nalpha};

\foreach \i in {0,...,\nalpha} {
\draw[thick, color=gray, domain=\a:\b, variable=\t] plot
({sin(\alphamin+\Dalpha*\i)*\funcao{\t}},{cos(\alphamin+\Dalpha*\i)*\funcao{\t}},\t); 
}
\foreach \i in {0,...,\nx} {
\pgfmathsetmacro{\point}{\a+(\i*\Dx)};
\draw[thick, color=gray, domain=\alphamin:\alphamax, variable=\alpha] plot
({sin(\alpha)*\funcao{\point}},{cos(\alpha)*\funcao{\point}},\point); 
}

\draw[thick, domain=\a:\b, variable=\t] plot
(0,\t,{\funcao{\t}}); 
\draw (0,1.2,1.1) node{$S_2$};
\end{scope}
\end{tikzpicture}\end{bmlimage}
\end{center}

Calculemos, para começar, o volume do sólido $S_1$.
O raciocíno já descrito acima permite usar a fórmula:
$$
V(S_1)=\int_0^1 \pi
(1-x^2)^2\,dx=\pi\int_0^1\{1-2x^2+x^4\}\,dx=\tfrac{8\pi}{15}\,.
$$
Consideremos agora o sólido $S_2$.
Por ser um sólido de revolução em torno do eixo $y$, a aproximação mais
natural é de usar fatias horizontais, centradas no eixo $y$, 
como na figura a seguir:
\begin{center}
\begin{bmlimage}\begin{tikzpicture}[scale=2]
\tdplotsetmaincoords{55}{115}
\newcommand{\funcao}[1]{(sqrt(1-(#1)))}
\pgfmathsetmacro{\a}{0};
\pgfmathsetmacro{\b}{1};
\pgfmathsetmacro{\c}{(\a+\b)/2};

%nombre de tranches:
\pgfmathsetmacro{\n}{10};

%\fill[areagrafico]
%(\a,0)--plot[domain=\a:\b](\x,{\funcao{\x}})--(\b,0)--cycle;


\pgfmathsetmacro{\dx}{(\b-\a)/\n};


%%desenhar os retangulos no plano
\foreach \i in {1,...,\n} {
%\draw ({\a + \i*\dx},0)--(1,1);
\filldraw[corretangulos] (0,{\a+(\i-1)*\dx})
rectangle ({\funcao{\a+\i*\dx}},{\a+\i*\dx});
\draw (0,{\a+(\i-1)*\dx})
rectangle ({\funcao{\a+\i*\dx}},{\a+\i*\dx});
\fill ({\funcao{\a+\i*\dx}},{\a+\i*\dx}) circle (0.2mm);
}
\draw (0,1) node[left]{$1$};
\draw[->] (-0.1,0)--({\b+0.3},0);
\draw[thick, domain=\a:\b] plot ({\funcao{\x}},\x);
\draw[dotted] (\a,0) node[below]{$0$} --
(\a,{\funcao{\a}});
\draw[dotted] (\b,0) node[below]{$1$} --
(\b,{\funcao{\b}});

\begin{scope}[xshift=3cm, yshift=0.5cm, tdplot_main_coords]

\pgfmathsetmacro{\nx}{2};
\pgfmathsetmacro{\Dx}{\n/\nx};
\pgfmathsetmacro{\alphamin}{-40};
\pgfmathsetmacro{\alphamax}{150};
\pgfmathsetmacro{\nalpha}{40};
\pgfmathsetmacro{\Dalpha}{(\alphamax-\alphamin)/\nalpha};

%dessiner les axes x et z
\draw[->] (0,0,0) -- (1.2,0,0);
%\draw[->] (0,0,0) -- (0,0,1.2);

%dessiner le traitille de derriere:
\draw[dotted] (0,\a,0)--(0,\a,{\funcao{\a}});


%dessiner les tranches:
\foreach \i in {1,...,\n} {
\pgfmathsetmacro{\ap}{(\a+(\i-1)*\dx)};
\pgfmathsetmacro{\bp}{(\a+(\i*\dx))};
\pgfmathsetmacro{\valf}{\funcao{\bp}};
%remplir le premier cercle (derriere) de la tranche:
\filldraw[corretangulos, domain=0:360, variable=\alpha] plot
({sin(\alpha)*\valf},{cos(\alpha)*\valf},{\ap}); 
%prononcer un peu le bord:
\draw[color=gray, domain=0:360, variable=\alpha] plot
({sin(\alpha)*\valf},{cos(\alpha)*\valf},{\ap}); 
%dessiner les petits traits de la tranche:
\foreach \j in {0,...,\nalpha} {
\draw[color=gray, domain=\ap:\bp, variable=\t] plot
({sin(\alphamin+\Dalpha*\j)*\valf},{cos(\alphamin+\Dalpha*\j)*\valf},\t); 
%remplir le disque (devant) de la tranche:
\filldraw[corretangulos, domain=0:360, variable=\alpha] plot
({sin(\alpha)*\valf},{cos(\alpha)*\valf},\bp); 
%prononcer un peu le bord de ce disque
\draw[color=gray, domain=0:360, variable=\alpha] plot
({sin(\alpha)*\valf},{cos(\alpha)*\valf},\bp); 
\fill (0,{\funcao{\bp}},\bp) circle (0.2mm);
}
}
%dessiner le graphe de la fonction
\draw[thick, domain=\a:\b, variable=\t, samples=15] plot
(0,{\funcao{\t}},\t);
%dessiner le petit traitille:
\draw[dotted] (0,\b,0)--(0,\b,{\funcao{\b}});

\draw[->] (0,\b,0) -- (0,{\b+0.3},0) node[right]{$x$};

\draw[->] (0,1)--(0,1.3);
\end{scope}
\end{tikzpicture}\end{bmlimage}
\end{center}
%%
Neste caso, dividimos o intervalo $y\in [0,1]$ em
intervalos $[y_{i-1},y_{i}]$. Ao intervalo $[y_{i-1},y_i]$ 
associamos uma fatia horizontal
$F_i$ de altura $\Delta y_i=y_i-y_{i-1}$ de
de raio $\sqrt{1-y_i}$. De fato, já que $F_i$ está na altura $y_i$, o
seu raio é dado pelo \emph{inverso} da função $x\to 1-x^2$
(isto é
$y\mapsto \sqrt{1-y}$) no ponto $y_i$. Assim, $V(F_i)=\pi \sqrt{1-y_i}^2
\Delta y_i$, e o volume de $V(S_2)$ é aproximado pela soma das fatias:
$$
\sum_{i=1}^nV(F_i)=\sum_{i=1}^n\pi(1-y_i)\Delta y_i\,.
$$
Portanto, no limite $n\to \infty$, combinado com $\Delta y_i\to 0$,
obtemos:
$$
V(S_2)=\int_0^1\pi (1-y)\,dy=\tfrac{\pi}{2}\,.
$$
Na próxima seção mostraremos um outro jeito de calcular $V(S_2)$.
\end{ex}

\begin{exo}
Considere a região finita $R$ contida no primeiro quadrante, 
delimitada pelas curvas $y=x^2$, $y=x^4$.
Calcule o volume do sólido de revolução obtido girando $R$ em torno do
eixo $y$.
\begin{sol}
$\tfrac{\pi}{6}$.
\end{sol}
\end{exo}
(Haverá mais exercícios no fim da próxima seção.)

\subsection{Aproximação por cascas}
\index{aproximação!por cascas}
Os exemplos considerados na seção anterior partiam de uma decomposição
do sólido usando \emph{fatias cilíndricas}.
Vejamos agora um outro tipo de decomposição, usando \emph{cascas}.

\begin{ex}
Considere de novo a região $R$ do 
Exemplo \ref{ex_gira_parabola} (a área debaixo da parábola), e
o sólido $S_2$ gerado pela rotação de $R$ em torno do eixo $y$.
Lá, $V(S_2)$ foi calculado usando uma integral, que foi construida
a partir de uma soma de cilindros, obtidos
pela rotação de retangulos \emph{horizontais} em torno do
eixo $y$. Procuremos agora calcular o mesmo volume
$V(S_2)$, mas com uma integral obtida a partir de uma soma de
\emph{cascas}. Cascas são obtidas pela rotação de
retângulos \emph{verticais}, em torno do eixo $y$:

\begin{center}
\begin{bmlimage}\begin{tikzpicture}[scale=2.3]
\tdplotsetmaincoords{70}{110}
\newcommand{\funcao}[1]{{(1-(#1)^2)}}
\pgfmathsetmacro{\a}{0};
\pgfmathsetmacro{\b}{1};
%nombre de tranches:
\pgfmathsetmacro{\n}{6};
%la largeur d'un rectangle
\pgfmathsetmacro{\dx}{(\b-\a)/\n};


%la tranche que je vais faire tourner:
\pgfmathsetmacro{\num}{4}

%le xmin
\pgfmathsetmacro{\xmin}{\a+(\num-1)*\dx}
%le xmax
\pgfmathsetmacro{\xmax}{\a+\num*\dx}



%%%LE GRAPHE DE LA FONCTION ET LE RECTANGLE%%%%%
\draw[->] (0,-0.1)--(0,1.3);
%%desenhar um retangulo no plano
\filldraw[corretangulos] ({\a+(\num-1)*\dx},0)
rectangle ({\a+\num*\dx},\funcao{\a+\num*\dx});
\draw ({\a+(\num-1)*\dx},0)
rectangle ({\a+\num*\dx},\funcao{\a+\num*\dx});
\fill ({\a+\num*\dx},\funcao{\a+\num*\dx}) circle (0.2mm);
%desenhar a funcao
\draw[->] (-0.1,0)--({\b+0.3},0);
\draw[thick, domain=\a:\b] plot (\x, \funcao{\x});

\draw[<-, >=latex] (\xmin,0)--({\xmin-0.3},-0.3)
node[below]{$x_{i-1}$};
\draw[<-, >=latex] (\xmax,0)--({\xmax+0.3},-0.3)
node[below]{$x_{i}$};

\draw[decorate, decoration=brace] 
(\xmax+0.1,{\funcao{\xmax}})--(\xmax+0.1,0)node[midway,
right]{$f(x_i)$};

%3D%%% 
%dessin en trois dimension da casca
\begin{scope}[xshift=3.5cm, yshift=0cm, tdplot_main_coords]
\pgfmathsetmacro{\alphamin}{-40};
\pgfmathsetmacro{\alphamax}{150};
%dessiner les axes x et z
\draw (-\b+0.2,0,0) -- (\b-0.2,0,0);
\draw (0,-\b+0.2,0) -- (0,\b-0.2,0);
\draw[->] (0,0,0) -- (0,0,1.2);
%dessiner les deux cercles du bas:
\draw[domain=0:360, variable=\angle, samples=50, densely
dotted] 
plot ({\xmin*cos(\angle)},{\xmin*sin(\angle)},-0.005);
\draw[domain=0:360, variable=\angle, samples=50, densely
dotted] 
plot ({\xmax*cos(\angle)},{\xmax*sin(\angle)},-0.005);

%dessiner le rectangle en traitille:
\fill[corretangulos] (0,\xmin,0)--(0,\xmax,0)--
(0,\xmax,{\funcao{\xmax}})--(0,\xmin,{\funcao{\xmax}})--cycle;
\draw[thick] (0,\xmin,0)--(0,\xmax,0)--
(0,\xmax,{\funcao{\xmax}})--(0,\xmin,{\funcao{\xmax}})--cycle;
\pgfmathsetmacro{\tetaz}{50};
\fill[corretangulos]
({\xmin*cos(\tetaz)},{\xmin*sin(\tetaz)},0)--
({\xmax*cos(\tetaz)},{\xmax*sin(\tetaz)},0)--
({\xmax*cos(\tetaz)},{\xmax*sin(\tetaz)},{\funcao{\xmax}})--
({\xmin*cos(\tetaz)},{\xmin*sin(\tetaz)},{\funcao{\xmax}})--cycle;

\draw[dotted]
(0,0,0)--
({\xmax*cos(\tetaz)},{\xmax*sin(\tetaz)},0)--
({\xmax*cos(\tetaz)},{\xmax*sin(\tetaz)},{\funcao{\xmax}})--
(0,0,{\funcao{\xmax}})--cycle;
\draw[thick]
({\xmin*cos(\tetaz)},{\xmin*sin(\tetaz)},0)--
({\xmax*cos(\tetaz)},{\xmax*sin(\tetaz)},0)--
({\xmax*cos(\tetaz)},{\xmax*sin(\tetaz)},{\funcao{\xmax}})--
({\xmin*cos(\tetaz)},{\xmin*sin(\tetaz)},{\funcao{\xmax}})--cycle;

%dessiner les parois EXTERNE de DEVANT:
\pgfmathsetmacro{\valf}{ \funcao{\xmax} };
%ATENCAO: nangles eh o num de angulos em 180!!!
\pgfmathsetmacro{\nangles}{20};
\pgfmathsetmacro{\angleinit}{290};
\pgfmathsetmacro{\dalpha}{180/\nangles}

\foreach \i in {\nangles,...,1} {
\pgfmathsetmacro{\angleA}{\angleinit+(\i-1)*\dalpha};
\pgfmathsetmacro{\angleB}{\angleinit+\i*\dalpha};
\fill[corretangulos]({\xmax*cos(\angleA)},{\xmax*sin(\angleA)},0) --
 ({\xmax*cos(\angleB)},{\xmax*sin(\angleB)},0) --
 ({\xmax*cos(\angleB)},{\xmax*sin(\angleB)},{\valf}) --
 ({\xmax*cos(\angleA)},{\xmax*sin(\angleA)},{\valf}) -- cycle;
}


%dessiner le COUVERCLE
\pgfmathsetmacro{\xmin}{ \a + (\num-1)*\dx };
\pgfmathsetmacro{\xmax}{ \a + (\num*\dx) };
\pgfmathsetmacro{\valf}{\funcao{\xmax}};
%ATENCAO: nangles eh o num de angulos em 180!!!
\pgfmathsetmacro{\nangles}{20};
\pgfmathsetmacro{\angleinit}{0};
\pgfmathsetmacro{\dalpha}{360/\nangles}
\foreach \i in {\nangles,...,1} {
\pgfmathsetmacro{\angleA}{\angleinit+(\i-1)*\dalpha};
\pgfmathsetmacro{\angleB}{\angleinit+\i*\dalpha};
\fill[corretangulos]
({\xmin*cos(\angleA)},{\xmin*sin(\angleA)},{\valf}) --
({\xmax*cos(\angleA)},{\xmax*sin(\angleA)},{\valf}) --
({\xmax*cos(\angleB)},{\xmax*sin(\angleB)},{\valf}) --
({\xmin*cos(\angleB)},{\xmin*sin(\angleB)},{\valf}) 
-- cycle;
}


%dessiner les deux cercles du haut:
\draw[domain=0:360, variable=\angle, samples=50, densely dotted] 
plot ({\xmin*cos(\angle)},{\xmin*sin(\angle)},{\funcao{\xmax}});
\draw[domain=0:360, variable=\angle, samples=50, densely
dotted] 
plot ({\xmax*cos(\angle)},{\xmax*sin(\angle)},{\funcao{\xmax}});


%dessiner le graphe de la fonction
\draw[thick, domain=\a:\b, variable=\t, samples=15] plot
(0,\t, \funcao{\t});

\draw[->] (0,\xmax,0) -- (0,{\b+0.3},0) node[right]{$x$};

\fill (0,\xmax,{\funcao{\xmax}}) circle (0.2mm);
\draw (0,-0.6,0.85) node{$C_i$};

%Dessiner les fleches qui montrent que ca tourne:
\draw[domain=90:\tetaz, variable=\angle, samples=20, ->, thick] 
plot
({\xmax*cos(\angle)},{\xmax*sin(\angle)},{(\funcao{\xmax})*(0.5)});
\draw[domain=90:50, variable=\angle, samples=20, ->, thick] 
plot ({\xmax*cos(\angle)},{\xmax*sin(\angle)},{(\funcao{\xmax})});
\draw[domain=90:50, variable=\angle, samples=20, ->, thick] 
plot ({\xmax*cos(\angle)},{\xmax*sin(\angle)},0);
\end{scope}

\end{tikzpicture}\end{bmlimage}
\end{center}

O volume da casca $C_i$ pode ser calculado pela diferença dos
volumes de dois cilindros: o externo tem raio $x_i$, o
interno tem raio $x_{i-1}$, e ambos têm altura $f(x_i)$.
Logo, 
$$V(C_i)=\pi x_{i}^2\times f(x_i)-\pi x_{i-1}^2\times
f(x_i)=\pi(x_i^2-x_{i-1}^2)f(x_i)\,.$$
Fatorando,
$x_i^2-x_{i-1}^2=(x_i+x_{i-1})(x_i-x_{i-i})$. Quando 
$\Delta x_i=x_i-x_{i-1}$ for muito pequeno, isto é quando
$x_i$ e $x_{i-1}$ forem muito próximos, podemos
aproximar $x_i+x_{i+1}$ por $2x_i$. Logo, 
$$
V(C_i)\simeq 2\pi x_if(x_i)\Delta x_i\,.
$$
Obs: essa fórmula é facil de entender observando que a  
casca $C_i$ pode ser obtida torcendo um paralelepípedo
cuja base é o retângulo de base $(x_{i}-
x_{i-1})\times f(x_i)$ e de altura dada pela
circunferência do círculo de raio $x_i$, isto é $2\pi
x_i$. (Atenção: esse raciocíno é correto somente se a
base do retângulo é pequena em relação à sua distância ao
eixo de rotação!)\\

Portanto, o volume so sólido $S_2$ pode ser calculado via
a integral associada às somas de Riemann dos $V(C_i)$,
isto é:
$$
\boxed{
V(S_2)=\int_0^12\pi x f(x)\,dx\,.}
$$
Como era de se esperar, essa integral vale 
$$V(S_2)=\int_0^12\pi x(1-x^2)\,dx=\pisobredois\,.$$
\end{ex}


O último exemplo mostrou que o volume de um sólido pode ser calculado de
várias maneiras, usando cilindros ou cascas para o mesmo sólido
pode levar a integrar funções muito diferentes, e uma escolha
pode facilitar o cálculo da primitiva.

\begin{ex}\label{exemplo:solidos_girados}
Considere o triângulo $\cT$ determinado pelos pontos $A=(1,0)$,
$B=(1,1)$, $C=(2,0)$. 

\begin{center}
\begin{bmlimage}\begin{tikzpicture}[scale=2]
\coordinate (A) at (1,0);
\coordinate (B) at (1,1);
\coordinate (C) at (2,0);
\draw[->] (-0.2,0)--(3,0);
\draw[->] (0,-0.2)--(0,1);
\fill[areagrafico]  (A)--(B)--(C)--cycle;
\draw[thick] (A)--(B)--(C)--cycle;
\end{tikzpicture}\end{bmlimage}
\end{center}

Para começar, considere o cone $S_1$ obtido girando
$\cT$ em torno do eixo $x$:

\begin{center}
\begin{bmlimage}\begin{tikzpicture}[scale=2]
\tdplotsetmaincoords{55}{115}

\newcommand{\funcao}[1]{(2-(#1))}

\pgfmathsetmacro{\a}{1};
\pgfmathsetmacro{\b}{2};

\begin{scope}[tdplot_main_coords]
\pgfmathsetmacro{\nx}{20};
\pgfmathsetmacro{\Dx}{(\b-\a)/\nx};

\pgfmathsetmacro{\alphamin}{0};
\pgfmathsetmacro{\alphamax}{20};
\pgfmathsetmacro{\nalpha}{10};
\pgfmathsetmacro{\Dalpha}{(\alphamax-\alphamin)/\nalpha};


\draw[->] (0,0,0) -- (1.2,0,0);
\draw[->] (0,0,0) -- (0,{\b+0.3},0) node[anchor=north west]{$x$};
\draw[->] (0,0,0) -- (0,0,1.2);

%desenhar a flechinha de tras:
\draw[->, domain=0:330, variable=\alpha, samples=50] plot
({sin(\alpha)*(\funcao{\a})},\a,{cos(\alpha)*(\funcao{\a})}); 
%desenhar a linha pontilhada de tras:
\draw[thick, dotted] (0,\a,0)--(0,\a,{\funcao{\a}});

\fill[areagrafico]
(0,\a,0)--plot[domain=\a:\b](0,\x,{\funcao{\x}})--(0,\b,0)--cycle;
%desenhar o grafico de f em [a,b]
\draw[thick, domain=\a:\b, variable=\t] plot
(0,\t,{\funcao{\t}}); 
%desenhar a flechinha de frente:
\draw[->, domain=0:320, variable=\alpha] plot
({sin(\alpha)*\funcao{\b}},\b,{cos(\alpha)*\funcao{\b}}); 
%desenhar a linha pontilhada de frente:
\draw[thick,dotted] (0,\b,0)--(0,\b,{\funcao{\b}});
\end{scope}

\begin{scope}[xshift=3.5cm,tdplot_main_coords]
\pgfmathsetmacro{\nx}{15};
\pgfmathsetmacro{\Dx}{(\b-\a)/\nx};
\pgfmathsetmacro{\alphamin}{-50};
\pgfmathsetmacro{\alphamax}{155};
\pgfmathsetmacro{\nalpha}{35};
\pgfmathsetmacro{\Dalpha}{(\alphamax-\alphamin)/\nalpha};

\draw[->] (0,0,0) -- (1.2,0,0);
\draw[->] (0,0,0) -- (0,{\b+0.3},0) node[anchor=north west]{$x$};
\draw[->] (0,0,0) -- (0,0,1.2);
%desenhar o grande circulo de tras:
\draw[domain=0:360, variable=\alpha] plot
({sin(\alpha)*\funcao{\a}},\a,{cos(\alpha)*\funcao{\a}}); 
%desenhar a linha pontilhada de tras:
\draw[thick, dotted] (0,\a,0)--(0,\a,{\funcao{\a}});
%desenhar o grande circulo de frente:
\draw[domain=0:360, variable=\alpha] plot
({sin(\alpha)*\funcao{\b}},\b,{cos(\alpha)*\funcao{\b}}); 

\fill[areagrafico]
(0,\a,0)--plot[domain=\a:\b](0,\x,{\funcao{\x}})--(0,\b,0)--cycle;
\foreach \i in {0,...,\nalpha} {
\draw[thick, color=gray, domain=\a:\b, variable=\t] plot
({sin(\alphamin+\Dalpha*\i)*\funcao{\t}},\t,{cos(\alphamin+\Dalpha*\i)*\funcao{\t}}); 
}
\foreach \i in {0,...,\nx} {
\pgfmathsetmacro{\point}{\a+(\i*\Dx)};
\draw[thick, color=gray, domain=\alphamin:\alphamax, variable=\alpha] plot
({sin(\alpha)*\funcao{\point}},\point,{cos(\alpha)*\funcao{\point}}); 
 }
%desenhar o grafico de f em [a,b]
\draw[thick, thick, domain=\a:\b, variable=\t] plot
(0,\t,{\funcao{\t}}); 
%desenhar a linha pontilhada de frente:
\draw[thick,dotted] (0,\b,0)--(0,\b,{\funcao{\b}});
\draw (0,\b,{\funcao{\b}+0.8}) node[left]{$S_1$};
\end{scope}

\end{tikzpicture}\end{bmlimage}
\end{center}

Podemos calcular o volume de $S_1$ de duas maneiras. Primeiro,
girando retângulos verticais:

\begin{center}
\begin{bmlimage}\begin{tikzpicture}[scale=2]
\coordinate (A) at (1,0);
\coordinate (B) at (1,1);
\coordinate (C) at (2,0);
\draw[->] (0,0)--(2.5,0);
\draw[->] (0,-0.2)--(0,1);
%\fill[areagrafico]  (A)--(B)--(C)--cycle;
\draw (A)--(B)--(C)--cycle;

\pgfmathsetmacro{\s}{1.45};
\pgfmathsetmacro{\t}{1.55};
\fill[areagrafico] (\s,0)--(\s,{2-\t})--(\t,{2-\t})--(\t,0)--cycle;
\draw (\s,0)--(\s,{2-\t})--(\t,{2-\t})--(\t,0)--cycle;

\pgfmathsetmacro{\l}{0.3};
\draw[dotted] (\s,0)--(\s,{-\l});
\draw[dotted] (\t,0)--(\t,{-\l});
\draw[decorate, decoration=brace] 
(\t,{-\l})--(\s,{-\l})node[midway, below]{$dx$};

\draw (\t,0) node[below]{$x$};
\draw[thick] (1,0)node[below]{$1$}--(2,0)node[below]{$2$};

\pgfmathsetmacro{\g}{0.6};
\draw[decorate, decoration=brace] 
({\t+\g},{2-\t})--({\t+\g},{0})node[midway, right]{$f(x)$};
\draw[dotted] 
({\t+\g},{2-\t})--({\t},{2-\t});
\draw[dotted] 
({\t},{0})--({\t+\g},{0});

\end{tikzpicture}\end{bmlimage}
\end{center}

Seremos um pouco informais: 
o retângulo infinitesimal baseado em $x$ tem uma largura $dx$ e
uma altura $f(x)=2-x$ (que é a equação da reta que passa por $B$
e $C$). 
Ao girar em torno do eixo $x$, ele gera um
cilindro infinitesimal cuja base tem área igual a $\pi f(x)^2$, e altura 
$dx$.  Logo, o volume do cilindro é $\pi f(x)^2\times
dx=\pi(2-x)^2dx$, e o volume
de $S_1$ é obtido integrando todos os cilindros, quando $x$ varia de
$1$ até $2$:
\begin{equation}\label{eq_int_1}
V(S_1)=\int_1^2 \pi (2-x)^2\,dx\,.
\end{equation}

Mas é possível também calcular $V(S_1)$ girando retângulos
horizontais:

\begin{center}
\begin{bmlimage}\begin{tikzpicture}[scale=2]
\coordinate (A) at (1,0);
\coordinate (B) at (1,1);
\coordinate (C) at (2,0);
\draw[->] (0,0)--(2.5,0);
\draw[->] (0,-0.2)--(0,1.2);
%\fill[areagrafico]  (A)--(B)--(C)--cycle;
\draw (A)--(B)--(C)--cycle;

\pgfmathsetmacro{\s}{0.45};
\pgfmathsetmacro{\t}{0.55};
\fill[areagrafico] (1,\s)--({2-\t},\s)--({2-\t},\t)--(1,\t)--cycle;
\draw (1,\s)--({2-\t},\s)--({2-\t},\t)--(1,\t)--cycle;

\pgfmathsetmacro{\l}{0.3};
\draw[dotted] (1,\s)--({1-\l},\s);
\draw[dotted] (1,\t)--({1-\l},\t);
\draw[decorate, decoration=brace] ({1-\l},\s)--
({1-\l},\t) node[midway, left]{$dy$};

\draw (0,\t) node[left]{$y$};
\draw[dotted] (0,\t)--(1,\t);
\draw[dashed] (0,1)node[left]{$1$}--(1,1);
\draw (0,0)node[left]{$0$};
\draw[thick] (0,0)--(0,1);

\draw[dotted] (1,\s)--({1},{-\l});
\draw[dotted] ({2-\t},\s)--({2-\t},{-\l});
\draw[decorate, decoration=brace] ({2-\t},{-\l})-- 
(1,{-\l}) node[midway, below]{$h(y)$};

\end{tikzpicture}\end{bmlimage}
\end{center}

Um retângulo horizontal infinitesimal 
é definido pela sua posição com respeito
ao eixo $y$, pela sua altura, dada por $h(y)=(2-y)-1=1-y$ (aqui
calculamos a diferença entre a posição do seu ponto mais a
direita e do seu ponto mais a esquerda). 
Ao girar em torno do eixo $x$, esse retângulo gera uma casca
cujo raio é $y$, cuja altura é $h(y)$ e cuja espessura é $dy$,
logo, o seu volume é $2\pi y \times h(y)\times dy=2\pi y(1-y)dy$. Integrando
sobre todas as cascas, com $y$ variando entre $0$ e $1$:
\begin{equation}\label{eq_int_2}
V(S_1)=\int_0^12\pi y (1-y)\,dy\,.
\end{equation}


\begin{exo}\label{exo:verificar}
Verifique que os valores das integrais em \eqref{eq_int_1} e \eqref{eq_int_2}
são iguais.
\end{exo}

Consideremos agora o solído $S_2$ obtido girando $\cT$ em torno
da reta de equação $x=3$.

\begin{center}
\begin{bmlimage}\begin{tikzpicture}[scale=1.8]
\tdplotsetmaincoords{55}{115}

\newcommand{\funcao}[1]{(2-(#1))}

\pgfmathsetmacro{\a}{1};
\pgfmathsetmacro{\b}{2};

\begin{scope}[tdplot_main_coords]
\pgfmathsetmacro{\nx}{20};
\pgfmathsetmacro{\Dx}{(\b-\a)/\nx};

\pgfmathsetmacro{\alphamin}{0};
\pgfmathsetmacro{\alphamax}{20};
\pgfmathsetmacro{\nalpha}{10};
\pgfmathsetmacro{\Dalpha}{(\alphamax-\alphamin)/\nalpha};


\draw[->] (0,0,0) -- (1.2,0,0);
\draw[->] (0,0,0) -- (0,{5.5},0);
\draw[->] (0,0,0) -- (0,0,1.2);

%desenhar o eixo de rotacao:
\draw[dashed] (0,3,-0.4)--(0,3,1.6);

%desenhar a linha pontilhada de tras:
\draw[thick, dotted] (0,\a,0)--(0,\a,{\funcao{\a}});

%desenhar a flechinha de tras:
\draw[->, domain=180:-140, variable=\alpha, samples=50] plot
({2*sin(\alpha)},{3+2*cos(\alpha)},0); 

\draw (0,3,0.6) node[above right]{$x=3$};

\fill[areagrafico]
(0,\a,0)--plot[domain=\a:\b](0,\x,{\funcao{\x}})--(0,\b,0)--cycle;
%desenhar o grafico de f em [a,b]
\draw[thick, domain=\a:\b, variable=\t] plot
(0,\t,{\funcao{\t}}); 

\end{scope}
\end{tikzpicture}\end{bmlimage}
\end{center}

Comecemos girando retângulos verticais:
\begin{center}
\begin{bmlimage}\begin{tikzpicture}[scale=2]
\coordinate (A) at (1,0);
\coordinate (B) at (1,1);
\coordinate (C) at (2,0);
\draw[->] (0,0)--(4,0);
\draw[->] (0,-0.2)--(0,1);
%\fill[areagrafico]  (A)--(B)--(C)--cycle;
\draw (A)--(B)--(C)--cycle;

\pgfmathsetmacro{\s}{1.45};
\pgfmathsetmacro{\t}{1.55};
\fill[areagrafico] (\s,0)--(\s,{2-\t})--(\t,{2-\t})--(\t,0)--cycle;
\draw (\s,0)--(\s,{2-\t})--(\t,{2-\t})--(\t,0)--cycle;

\pgfmathsetmacro{\l}{0.3};
\draw[dotted] (\s,0)--(\s,{-\l});
\draw[dotted] (\t,0)--(\t,{-\l});
\draw[decorate, decoration=brace] 
(\t,{-\l})--(\s,{-\l})node[midway, below]{$dx$};

\draw (\t,0) node[below]{$x$};
\draw[thick] (1,0)node[below]{$1$}--(2,0)node[below]{$2$};

\pgfmathsetmacro{\g}{0.6};
\draw[decorate, decoration=brace] ({\g},{0})--
({\g},{2-\t})node[midway, left]{$f(x)$};
\draw[dotted] 
({\g},{2-\t})--({\t},{2-\t});
\draw[dotted] 
({\t},{0})--({\g},{0});

\draw[dashed] (3,-0.3)--(3,1.3);
\draw[decorate, decoration=brace] (\t,{2-\t})--
(3,{2-\t})node[midway, above]{$r(x)$};

\draw (3,0) node[below right]{$3$};


\end{tikzpicture}\end{bmlimage}
\end{center}

Ao girar o retângulo representado na figura em torno da reta $x=3$,
isto gera uma casca de raio
$r(x)=3-x$, de altura $f(x)=2-x$ e de espessura $dx$. Logo, o seu
volume é dado por $2\pi r(x)\times f(x)\times
dx=2\pi(3-x)(2-x)dx$. O volume de $S_2$ é obtido integrando com
respeito a $x$, entre $1$ e $2$:
\[ V(S_2)=\int_1^22\pi(3-x)(2-x)\,dx\,.  \]
Girando agora retângulos horizontais:


\begin{center}
\begin{bmlimage}\begin{tikzpicture}[scale=2]
\coordinate (A) at (1,0);
\coordinate (B) at (1,1);
\coordinate (C) at (2,0);
\draw[->] (0,0)--(4,0);
\draw[->] (0,-0.2)--(0,1.2);
%\fill[areagrafico]  (A)--(B)--(C)--cycle;
\draw (A)--(B)--(C)--cycle;


\draw[dashed] (3,-0.3)--(3,1.3);
\draw (3,0) node[above right]{$3$};

\pgfmathsetmacro{\s}{0.45};
\pgfmathsetmacro{\t}{0.55};
\fill[areagrafico] (1,\s)--({2-\t},\s)--({2-\t},\t)--(1,\t)--cycle;
\draw (1,\s)--({2-\t},\s)--({2-\t},\t)--(1,\t)--cycle;

\pgfmathsetmacro{\l}{0.3};
\draw[dotted] (1,\s)--({1-\l},\s);
\draw[dotted] (1,\t)--({1-\l},\t);
\draw[decorate, decoration=brace] ({1-\l},\s)--
({1-\l},\t) node[midway, left]{$dy$};

\draw (0,\t) node[left]{$y$};
\draw[dotted] (0,\t)--(1,\t);
\draw[dashed] (0,1)node[left]{$1$}--(1,1);
\draw (0,0)node[left]{$0$};
\draw[thick] (0,0)--(0,1);

\draw[dotted] (1,\s)--({1},{-1.5*\l});
\draw[dotted] ({2-\t},\s)--({2-\t},{-0.6*\l});
\draw[decorate, decoration=brace] ({3},{-1.5*\l})-- 
(1,{-1.5*\l}) node[midway, below]{$R(y)$};
\draw[decorate, decoration=brace] ({3},{-0.6*\l})-- 
({2-\t},{-0.6*\l}) node[midway, below]{$r(y)$};

\end{tikzpicture}\end{bmlimage}
\end{center}
Ao girar em torno da reta vertical $x=3$, o retângulo horizontal gera um
anel, de altura $dy$, de raio exterior $R(y)=2$, de raio
interior $r(y)=3-(2-y)=1+y$. O volume desse anel é dado por 
$\pi R(y)^2 \times dy-\pi r(y)^2\times dy$. Logo, o volume de
$S_2$ é dado pela integral
$$
V(S_2)=\int_0^1 (\pi 2^2-\pi (1+y)^2)\,dy\,.
$$

\end{ex}

\subsection{Exercícios}

\begin{exo}
Considere a região $R$ delimitada pelo gráfico da função $y=\sen x$,
pelo eixo $x$, e pelas duas retas $x=\pi/2$, $x=\pi$.
Calcule a área de $R$. Em seguida,
monte uma integral (não precisa calculá-la) cujo valor dê o volume so sólido
obtido girando $R$: 1) em torno do eixo $x$, 2) em torno da reta $x=\pi$.
\begin{sol}
A área é dada por 
$$\int_{\pi/2}^\pi\sen (x)dx=-\cos (x)|^{\pi}_{\pi/2}=-(-1)-0=1\,.$$
Girando em torno do eixo $x$:
$V_1=\int_{\pi/2}^{\pi}\pi(\sen x)^2\,dx$.
Ou, com as cascas: $V_1=\int_0^12\pi y (\pi/2-\arcsen y)\,dy$.
Em torno da reta $x=\pi$, usando as cascas:
$V_2=\int_{\pi/2}^\pi2\pi(\pi-x)\sen x\,dx$.
Sem usar as cascas:
$V_2=\pi(\tfrac{\pi}{2})^2\cdot 1-\int_0^1\pi (\arcsen y)^2\,.dy$.
\end{sol}
\end{exo}

\begin{exo}
Mostre que o volume de um cone de base circular de raio $R$ e de altura $H$ é
igual a $V=\frac{1}{3}\pi R^2H$.
\begin{sol}
O cone pode ser (tem vários jeitos, mas esse é o mais simples) 
obtido girando o gráfico da função $f(x)=\frac{R}{H}x$, $0\leq x\leq H$, em
torno do eixo $x$. Logo,
$$
V=\int_0^H\pi\big(\frac{R}{H}x\Big)^2dx=\pi\frac{R^2}{H^2}\int_0^Hx^2dx=
\pi\frac{R^2}{H^2}\frac{H^3}{3}=\frac{1}{3}\pi R^2H \,\,$$
Obs: pode também rodar o gráfico da função $f(x)=-\frac{H}{R}x+H$, $0\leq x\leq
R$, em torno do eixo $y$.
\end{sol}
\end{exo}

\begin{exo}(Prova 3, 2010, Turmas N)
Calcule o volume do sólido obtido girando a região $R=\{(x,y):1\leq x\leq
e\,,\,0\leq y\leq \sqrt{x}\ln x\}$ em torno da reta $y=0$.
\begin{sol}
O volume é dado por $V=\int_1^e\pi(\sqrt{x}\ln x)^2dx$. Integrando duas vezes
por partes, obtem-se
\begin{align*}
\int x(\ln x)^2dx&=\frac{x^2}{2}(\ln x)^2-\int \frac{x^2}{2}2(\ln
x)\frac{1}{x}dx\\
&=\frac{x^2}{2}(\ln x)^2-\int x\ln xdx\\
&=\frac{x^2}{2}(\ln x)^2-\big\{\frac{x^2}{2}\ln x-\int\frac{x^2}{2}\frac{1}{x}dx
\big\}\\
&=\frac{x^2}{2}(\ln x)^2-\frac{x^2}{2}\ln x+\frac{x^2}{4}+C
\end{align*}
Logo, $V=\pi\frac{e^2-1}{4}$.
\end{sol}
\end{exo}

\begin{exo}
Considere a região $R$ delimitada pela parábola $y=x^2$, 
pelo eixo $x$ e pela reta $x=1$,
contida no primeiro quadrante.
Para cada uma das retas abaixo, monte uma integral (sem calculá-la) que dê o volume do
sólido obtido girando $R$ em torno da reta $r$, usando a) cílindros, b) cascas.
\begin{multicols}{3}
\begin{enumerate}
\item\label{itexorotsol1}  $y=0$,
\item\label{itexorotsol2} $y=1$,
\item\label{itexorotsol3}   $y=-1$,
\item\label{itexorotsol4} $x=0$,
\item\label{itexorotsol5} $x=1$,
\item\label{itexorotsol6} $x=-1$.
\end{enumerate}
\end{multicols}
\vspace{0.01cm}
\begin{sol}
\eqref{itexorotsol1}
Cil.: $\int_0^1\pi (x^2)^2\,dx$, 
Casc.:
$\int_0^12\pi y(1-\sqrt{y})\,dy$.
\eqref{itexorotsol2}
Cil.: $\int_0^1\pi(1^2-(1-x^2)^2)\,dx$
Casc.: $\int_0^12\pi(1-y)(1-\sqrt{y})\,dy$,
\eqref{itexorotsol3}
Cil.: $\int_0^1\pi((1+x^2)^2-1^2)\,dx$
Casc.: $\int_0^1 2\pi(1+y)(1-\sqrt{y})\,dy$
\eqref{itexorotsol4}
Cil.: $\int_0^1\pi(1^2-\sqrt{y}^2)\,dy$
Casc.: $\int_0^12\pi x\cdot x^2\,dx$
\eqref{itexorotsol5}
Cil. $\int_0^1\pi(1-\sqrt{y})^2\,dy$
Casc.: $\int_0^12\pi(1-x)x^2\,dx$
\eqref{itexorotsol6}
Cil.: $\int_0^1\pi(2^2-(1+\sqrt{y})^2)\,dy$
Casc. $\int_0^12\pi(1+x)x^2\,dx$
\end{sol}
\end{exo}


\begin{exo}
Monte uma integral cujo valor seja igual ao volume do
sólido obtido girando a região $R$ (finita, delimitada pela curva $y=1-(x-2)^2$
e o eixo $x$) em torno da reta $y=2$.
\begin{sol}
Com o método dos cilíndros,
$$V=\int_1^3\pi 2^2dx-\int_1^3\pi\big(2-(1-(x-2)^2)\big)^2dx\,\,.$$
OU, usando o método das cascas, 
$$
V=\int_0^12\pi(2-y)2\sqrt{1-y}dy\,.
$$
OU, transladando o gráfico da função, e girando a nova região (finita,
delimitada pela nova curva $y=-1-x^2$ e o eixo $x$),
$$V=\int_{-1}^{+1}\pi 2^2dx-\int_{-1}^{+1}\pi(-1-x^2)^2dx\,.$$
\end{sol}
\end{exo}


\begin{exo}
Considere o sólido $S$ obtido girando o gráfico da função $f(x)=\cosh(x)$ em
torno da reta $y=0$, entre $x=-1$ e $x=+1$. Esboce $S$, e calcule o seu volume.
(Lembre que $\cosh(x)\pardef\frac{e^x+e^{-x}}{2}$.)
\begin{sol}
O volume é dado pela integral 
\begin{align*}
V=\int_{-1}^{+1}\pi \cosh^2xdx&=\pi\int_{-1}^{+1}\frac{e^{2x}+2+e^{-2x}}{4}dx\\
&=\frac{\pi}{4}\Big\{
\frac{e^{2x}}{2}+2x-\frac{e^{-2x}}{2}
\Big\}_{-1}^{+1}\\
&=\frac{\pi}{4}\big\{e^2+4-e^{-2}\big\}
\end{align*}
\end{sol}
\end{exo}

\begin{exo}
Considere a região $R$ delimitada pelo gráfico da função
$f(x)=\cos x$, pelas retas
$x=\frac{\pi}{2}$, $x=\pi$, e pelo eixo $x$.
Monte duas integrais, cujos valores dão o volume do
sólido de revolução obtido girando
$R$ em torno 1) da reta $x=\pi$, 2) da reta $y=-1$.
\begin{sol}
Em torno da reta $x=\pi$:
$$
V=\int_{\pi/2}^{\pi}2\pi(\pi-x)|\cos x|\,dx\,,\quad\text{ ou }
\quad V=\int_{-1}^0\pi(\pi-\arcos y)^2\,dy\,.
$$
Em torno da reta $y=-1$:
$$
V=\int_{\pi/2}^\pi \pi\cdot 1^2\,dx-\int_{\pi/2}^\pi\pi(\cos
x -(-1))^2\,dx\,,\quad\text{ ou }
\quad V=\int_{-1}^02\pi (y-(-1))(\pi-\arcos y)\,dy\,.
$$
\end{sol}
\end{exo}


\begin{exo}
Um toro é obtido girando um disco de raio $r$ em torno de
um eixo vertical, mantendo o centro do disco a distância
$R$ ($R>r$) do eixo.
Mostre que o volume desse toro é igual a $2\pi^2 r^2R$.
\end{exo}

\section{Áreas de superfícies de revolução}

Suponha que se queira calcular a \emph{área da superfície} do
sólido do início da Seção
\ref{Sec_Solidos} (sem os dois discos de frente
e de trás), denotada $A(S)$.
De novo, aproximaremos a área $A(S)$ por uma soma de áreas mais
simples.\\

Para decompor a área em áreas mais elementares, 
escolhamos uma divisão $a=x_0<x_1<\dots<x_n=b$,
e para cada intervalo $[x_{i-1},x_i]$, consideremos o anel $J_i$
obtido girando o segmento ligando $(x_{i-1},f(x_{i-1}))$ a 
$(x_i,f(x_i))$ em torno do eixo $x$:

\begin{center}
\begin{bmlimage}\begin{tikzpicture}[scale=2]
\tdplotsetmaincoords{55}{115}

\newcommand{\funcao}[1]{(1/((#1)^2+1))}

\pgfmathsetmacro{\a}{0.2};
\pgfmathsetmacro{\b}{1.5};

\pgfmathsetmacro{\nx}{7};
\pgfmathsetmacro{\len}{3}
\pgfmathsetmacro{\Dx}{(\b-\a)/\nx};

\pgfmathsetmacro{\point}{\a+(\len*\Dx)};
\pgfmathsetmacro{\pointb}{\a+((\len+1)*\Dx)};

\draw[->] (0,0) -- ({\b+0.3},0);
\draw[->] (0,0) -- (0,1.2);

\draw[dotted] (\a,0) -- (\a,{\funcao{\a}});
\draw[dotted] (\b,0) -- (\b,{\funcao{\b}});

\draw[densely dotted] (\point,0)--(\point,{\funcao{\point}});
\draw[densely dotted] (\pointb,0)--(\pointb,{\funcao{\pointb}});

\draw[color=gray!50, domain=\a:\b, variable=\t] plot
(\t,{\funcao{\t}}); 

\draw[thick] (\point,{\funcao{\point}})--
(\pointb,{\funcao{\pointb}});

\draw[<-,>=latex] (\point,0)--({\point-0.1},-0.2)
node[left]{$x_{i-1}$};


\draw[<-,>=latex] (\pointb,0)--({\pointb+0.1},-0.2)
node[right]{$x_{i}$};

\begin{scope}[xshift=4cm, yshift=0.5cm, tdplot_main_coords]

\pgfmathsetmacro{\alphamin}{0};
\pgfmathsetmacro{\alphamax}{360};
\pgfmathsetmacro{\nalpha}{40};
\pgfmathsetmacro{\Dalpha}{(\alphamax-\alphamin)/\nalpha};

\draw (0,\pointb,{\funcao{\pointb}+0.2}) node[above right]{$J_i$};

\draw[->] (0,0,0) -- (1.2,0,0);
\draw[->] (0,0,0) -- (0,{\b+0.3},0) node[anchor=north west]{$x$};
\draw[->] (0,0,0) -- (0,0,1.2);

\draw[thick, dotted] (0,\a,0)--(0,\a,{\funcao{\a}});

\draw[thick, color=gray, domain=\alphamin:\alphamax, variable=\alpha] plot
({sin(\alpha)*\funcao{\point}},\point,{cos(\alpha)*\funcao{\point}}); 
\draw[dotted] (0,\point,{\funcao{\point}})--(0,\point,0);
\draw[dotted] (0,\pointb,{\funcao{\pointb}})--(0,\pointb,0);
\foreach \j in {0,...,\nalpha} {
\draw ({sin(\alphamin+\Dalpha*\j)*\funcao{\point}},
\point,{cos(\alphamin+\Dalpha*\j)*\funcao{\point}})--
({sin(\alphamin+\Dalpha*\j)*\funcao{\pointb}},
\pointb,{cos(\alphamin+\Dalpha*\j)*\funcao{\pointb}})--
({sin(\alphamin+\Dalpha*(\j+1))*\funcao{\pointb}},
\pointb,{cos(\alphamin+\Dalpha*(\j+1))*\funcao{\pointb}});
}

\draw[thick, thick, domain=\a:\b, variable=\t] plot
(0,\t,{\funcao{\t}}); 
\draw[thick,dotted] (0,\b,0)--(0,\b,{\funcao{\b}});
\end{scope}

\end{tikzpicture}\end{bmlimage}
\end{center}

Pode ser verificado que o anel $J_i$ tem uma área dada por 
\begin{equation}\label{eq_area_anel}
A(J_i)=\pi\sqrt{(x_i-x_{i-1})^2+(f(x_i)-f(x_{i-1}))^2}(f(x_i)+f(x_{i-1}))\,.
\end{equation}

Quando $\Delta x_i=x_i-x_{i-1}$ for suficientemente pequeno, e se
$f$ for contínua, $f(x_i)+f(x_{i-1})$ pode ser aproximada por
$2f(x_i)$.
Logo, colocando $\Delta x_i$ em evidência dentro da raiz,
\begin{equation}
A(J_i)\simeq 2 \pi f(x_i)\sqrt{1+\Bigl(\frac{f(x_i)-f(x_{i-1})}{\Delta
x_i}\Bigr)^2)}\Delta x_i\,.
\end{equation}
Quando $\Delta x_i$ for pequeno, o quociente $(\frac{f(x_i)-f(x_{i-1})}{\Delta
x_i}$ pode ser aproximado por $f'(x_i)$. Logo, a 
área total pode ser aproximada pela soma de Riemann
$$
\sum_{i=1}^n A(J_i)\simeq \sum_{i=1}^n 
2\pi f(x_i)\sqrt{1+(f'(x_i))^2}\Delta x_i\,. 
$$
Quando $n\to \infty$ e todos os $\Delta x_i\to 0$, a soma de
Riemann acima converge para a integral
\begin{equation}
A(S)=\int_a^b2\pi f(x)\sqrt{1+(f'(x))^2}\,dx\,.
\end{equation}

\begin{ex}
Considere a superfície gerada pela rotação da curva
$y=\sqrt{x}$ em torno do eixo $x$, entre $x=0$ e $x=1$. A sua
área é dada pela integral 
\begin{align*}
A(S)&=\int_0^1
2\pi\sqrt{x}\sqrt{1+(\tfrac{1}{2\sqrt{x}})^2}\,dx=
\pi\int_0^1 \sqrt{1+4 x}\,dx=\tfrac{\pi}{6}(5^{3/2}-1)\,.
\end{align*}
\end{ex}

\begin{exo}
Prove \eqref{eq_area_anel}.
\begin{sol}
Se trata de mostrar que 
a área lateral de um cone truncado de raios $r\leq R$ 
e de altura $h$ é dada por 
$$
A=\pi (R+r)\sqrt{h^2+(R-r)^2}\,.
$$
De fato, fazendo o corte,
\begin{center}
\begin{bmlimage}\begin{tikzpicture}
\coordinate (A) at (0,0);
\coordinate (B) at (0,1);
\coordinate (C) at (0,3);
\coordinate (D) at (1.333,1);
\coordinate (E) at (2,0);
\draw (A)--(B) node[midway, left]{$h$}--
(C)--(E)--(A) node[midway, above]{$R$};
\draw (B)--(D) node[midway, above]{$r$};
\draw (C) node[left]{$C$};
\draw (D) node[right]{$D$};
\draw (E) node[right]{$E$};
\end{tikzpicture}\end{bmlimage}
\end{center}
Chamando a distância $CD$ de $l$, e a distância $CE$ de $L$, temos
$A=\pi R L-\pi rl$. Uma conta elementar mostra que
$l=\frac{r}{R-r}\sqrt{h^2+(R-r)^2}$, e que 
$L=\frac{R}{R-r}\sqrt{h^2+(R-r)^2}$.
Isso dá a fórmula desejada.
\end{sol}
\end{exo}

\begin{exo}
Mostre que a área da superfície de uma esfera de raio $R$ é igual a $4\pi R^2$.
\begin{sol}
Como a esfera é obtida girando o gráfico de
$f(x)=\sqrt{R^2-x^2}$, a sua área é dada por 
$$
A=2\pi\int_{-R}^R\sqrt{R^2-x^2}\sqrt{1+\bigl(\sqrt{R^2-x^2}'\bigr)^2}\,dx
=2\pi R\int_{-R}^R\,dx= 4\pi R^2\,.
$$
\end{sol}
\end{exo}

\section{Energia potencial}
(em construção)
\section{Resolvendo equações diferenciais}
(em construção)


% !TeX spellcheck = pt_BR
% !TEX encoding = UTF-8 Unicode

\chapter{Integrais impróprias}\label{CAP:Improprias}

\ifdefined\updateans
% Only need to run once in a lifetime, when the file ans.tex needs to be updated.
\Writetofile{ans}{\protect\section*{Capítulo \ref{CAP:Improprias}}}
\fi

A integral de Riemann foi definida naturalmente para uma função $f:[a,b]\to \bR$
contínua, como um limite de somas de retângulos. 
Nesta seção estudaremos integrais de funções em intervalos \emph{infinitos}, 
como $[0,\infty)$ ou a reta inteira, ou em intervalos do tipo $(a,b]$, em que a
função pode possuir alguma descontinuidade (uma assíntota vertical por exemplo)
em $a$. Tais integrais são chamadas de \emph{impróprias}, e são muito usadas,
em particular no estudo de \emph{séries} (Cálculo II e CVV) e na
resolução de \emph{equações diferenciais} (transformada de Laplace,
transformada de Fourier, etc).

\section{Em intervalos infinitos}
\index{integral imprópria! em intervalo infinito}
Consideremos para começar o problema de integrar uma função num
intervalo infinito, $f:[a,\infty)\to \bR$.
Vemos imediatamente que não tem como definir somas de Riemann num intervalo
infinito: qualquer subdivisão de $[a,\infty)$ contém um número
infinito de retângulos. O que pode ser feito é o seguinte:
escolheremos um número $L>a$
\emph{grande mas finito}, calcularemos a integral de Riemann de $f$ em $[a,L]$,
e \emph{em seguida} tomaremos o\index{limite} limite $L\to \infty$:

\begin{defin}
Seja $f:[a,\infty)\to \bR$ uma função contínua. Se o limite 
\eq{\int_a^\infty f(x)\,dx\pardef \lim_{L\to \infty}\int_a^L f(x)\,dx\,,}
existir e for finito, diremos que \grasA{a integral imprópria 
$\int_a^\infty f(x)\,dx$ converge}. Caso contrário, ela \grasA{diverge}.
Integrais impróprias para $f:(-\infty,b]\to \bR$ se definem da mesma maneira:
\eq{\int_{-\infty}^b f(x)\,dx\pardef \lim_{L\to \infty}\int_{-L}^b f(x)\,dx\,.}
\end{defin}


\begin{ex}
Considere $f(x)=e^{-x}$ em $[0,+\infty)$:
$$
\int_0^\infty e^{-x}\,dx=\lim_{L\to \infty}\int_0^L e^{-x}\,dx=
\lim_{L\to \infty} \bigl\{-e^{-x}\bigr\}\big|_0^{L}=
\lim_{L\to \infty} \bigl\{1-e^{-L}\bigr\}=1\,,
$$ 
que é finito. Logo, $\int_0^\infty e^{-x}\,dx$ converge e vale $1$.
Como $e^{-x}$ é uma função positiva no intervalo $[0,\infty)$ todo, o valor de
$\int_0^\infty e^{-x}\,dx$ pode
ser interpretado como
o valor da área delimitada pela parte do gráfico de $e^{-x}$ contida no
primeiro quadrante,
pelo eixo $x$ e pelo eixo $y$:
\begin{center}
\begin{bmlimage}\begin{tikzpicture}[xscale=1.5, yscale=2]
\fill[areagrafico] (0,0)--plot[domain=0:4](\x,{exp(-\x)})--(4,0)--cycle;
\draw[thick, domain=-0.2:4] plot (\x,{exp(-\x)});
\draw[>=latex, ->] (-0.2,0)--(5,0);
\draw[>=latex, ->] (0,-0.2)--(0,1.2) node[right]{$e^{-x}$};
\draw (0.55,0.2) node{área$=1$};
\end{tikzpicture}\end{bmlimage}
\end{center}
Observe que
apesar
dessa área não possuir
limitação espacial, ela é finita!
\end{ex}

\begin{ex}\label{Ex:unsurxdiverge}
Considere $f(x)=\frac{1}{x}$ em $[1,\infty)$:
$$
\int_1^\infty \frac{dx}{x}=\lim_{L\to \infty}\int_1^L \frac{dx}{x}=
\lim_{L\to \infty} \bigl\{\ln x\bigr\}\big|_1^{L}=
\lim_{L\to \infty} \ln L=\infty\,.
$$ 
\begin{center}
\begin{bmlimage}\begin{tikzpicture}[xscale=1.5, yscale=2]
\fill[areagrafico] (1,0)--plot[domain=1:6](\x,1/\x)--(6,0)--cycle;
\draw[thick, domain=0.8:6] plot (\x,{1/\x});
\draw[>=latex, ->] (-0.2,0)--(6.5,0);
\draw[>=latex, ->] (0,-0.2)--(0,1.2) node[right]{$\tfrac1x$};
\draw (1.8,0.25) node{área$=\infty$};
\draw[dotted] (1,0)--(1,1);
\end{tikzpicture}\end{bmlimage}
\end{center}
Neste caso, a interpretação de $\int_1^\infty \frac{dx}{x}=\infty$ é que a área
delimitada pelo gráfico de $f(x)=\tfrac1x$ é \emph{infinita}.
\index{gráfico! área debaixo de um}
\end{ex}

\begin{obs}
As duas funções consideradas acima, $e^{-x}$ e $\frac1x$, tendem a zero
no infinito. 
No entanto, a integral imprópria da primeira converge, enquanto a da segunda
diverge.
Assim, vemos que \emph{não basta uma função tender a zero no infinito
para a sua integral imprópria convergir}!
De fato, a convergência de uma integral imprópria depende de \emph{quão
rápido} a função tende a zero.
Nos exemplos acima, $e^{-x}$ tende a zero muito mais rápido~\footnote{Por
exemplo, usando a Regra de B.H., 
$\lim_{x\to \infty}\frac{e^{-x}}{\tfrac1x}=\lim_{x\to
\infty}\frac{x}{e^{x}}=0$.} que
$\tfrac1x$.
No caso, $e^{-x}$ tende a zero \emph{rápido o suficiente} para que a área
delimitada pelo seu gráfico seja \emph{finita}, e $\tfrac1x$ tende a zero
\emph{devagar o suficiente} para que a área
delimitada pelo seu gráfico seja \emph{infinita}.
\end{obs}

\begin{ex}
Considere a integral imprópria
$$\int_1^\infty\frac{1}{\sqrt{x}(x+1)}dx=\lim_{L\to\infty}
\int_1^L\frac{1}{\sqrt{x}(x+1)}dx\,.$$
Com $u=\sqrt{x}$ temos $dx=2u\,du$. Logo,
\begin{align*}
\int_1^L\frac{1}{\sqrt{x}(x+1)}dx=2
\int_1^{\sqrt{L}}\frac{1}{u^2+1}du=\bigl\{2\arctan
u\big\}_1^{\sqrt{L}}
\end{align*}
Tomando o limite $L\to \infty$,
\begin{align*}
\int_1^\infty\frac{1}{\sqrt{x}(x+1)}dx=2\lim_{L\to\infty}\big\{
\arctan(\sqrt{L})-\tfrac{\pi}{4}
\big\}=
2\bigl\{\tfrac{\pi}{2}-\tfrac{\pi}{4}\bigr\}=\tfrac{\pi}{2}\,,
\end{align*}
que é finito. Logo, a integral imprópria acima {converge}, e o seu valor é
$\tfrac{\pi}{2}$.
\end{ex}

A função integrada, numa integral imprópria, não precisa ser positiva:
\begin{ex}
Considere $\int_0^\infty e^{-x}\sen x\,dx$. 
Usando integração por partes (veja o Exercício \ref{Exo:porpartesmach}),
$$\int_0^\infty e^{-x}\sen x\,dx=\lim_{L\to\infty}\bigl\{-\tfrac12 e^{-x}(\sen
x+\cos x)\bigr\}\Big|_0^L=\tfrac12
\lim_{L\to\infty}\bigl\{1-e^{-L}(\sen
L+\cos L)\bigr\}=\tfrac12\,.
$$
Logo, a integral converge. Apesar do valor $\frac12$ ser $>0$, a
sua interpretação em termos de área não é possível neste caso, já que 
$x\mapsto e^{-x}\sen x$ é negativa em infinitos intervalos:
\begin{center}
\begin{bmlimage}\begin{tikzpicture}[xscale=1.5, yscale=2]
% \fill[areagrafico] (0,0)--plot[domain=0:4](\x,{exp(-\x)*sin(\x
% r)})--(4,0)--cycle;
\draw[thick, domain=-0.1:4, samples=70] plot (\x,{exp(-\x)*sin((3*\x)
r)});
\draw[>=latex, ->] (-0.2,0)--(5,0);
\draw[>=latex, ->] (0,-0.2)--(0,1) node[right]{$e^{-x}\sen x$};
%\draw (0.55,0.2) node{área$=1$};
\end{tikzpicture}\end{bmlimage}
\end{center}
\end{ex}

\begin{exo}
Estude a convergência das seguintes integrais impróprias.
\begin{multicols}{3}
\begin{enumerate}
\item\label{itImproprias0} $\int_3^\infty\frac{dx}{x-2}$
\item\label{itImproprias1} $\int_{2}^\infty x^2\,dx$
\item\label{itImproprias2} $\int_{1}^\infty \tfrac{dx}{x^7}$
\item\label{itImproprias3} $\int_0^\infty\cos x\,dx$
\item\label{itImproprias4} $\int_{0}^\infty \frac{dx}{x^2+1}$
\item\label{itImproprias5} $\int_{1}^\infty \frac{dx}{x^2+x}$
\item\label{itImproprias6} $\int_{-\infty}^0 e^{t}\sen(2t)dt$
\item\label{itImproprias65} $\int_{3}^\infty \frac{\ln x}{x}\,dx$
\item\label{itImproprias7} $\int_0^\infty\frac{x}{x^4+1}\,dx$
\end{enumerate}
\end{multicols}
\vspace{0.01cm}
\begin{sol}
\eqref{itImproprias0} Com $u=x-2$,
$\int_3^\infty\frac{dx}{x-2}=\lim_{L\to \infty}\int_3^L\frac{dx}{x-2}=
\lim_{L\to \infty}\int_1^{L-2}\frac{du}{u}=\lim_{L\to \infty}\ln(L-2)=\infty
$, diverge.
\eqref{itImproprias1} Diverge (é a área da região contida entre a parábola
$x^2$ e o eixo $x$!)
\eqref{itImproprias2} $\int_{1}^\infty
\tfrac{dx}{x^7}=\lim_{L\to\infty}\int_{1}^L \tfrac{dx}{x^7}=
\tfrac16\lim_{L\to\infty}\{1-\frac{1}{L^6}\}=\tfrac16$, logo converge.
\eqref{itImproprias3} Como $\int_0^L\cos x\,dx=\sen L$, e que $\sen L$ não
possui limite quando $L\to \infty$, a integral imprópria $\int_0^\infty\cos
x\,dx$ diverge.
\eqref{itImproprias4} $\int_{0}^\infty \frac{dx}{x^2+1}=\frac{\pi}{2}$, logo
converge.
\eqref{itImproprias5} Temos
$\frac{1}{x^2+x}=\frac{1}{x(x+1)}=\frac{1}{x}-\frac{1}{x+1}$, logo
$$\int_1^L\frac{dx}{x^2+x}=\{\ln x\}|_1^L-\{\ln(x+1)\}|_1^L
=\ln L-\ln(L+1)+\ln 2\,.
$$
Mas como $\lim_{L\to \infty}\{\ln L-\ln(L+1)\}=
\lim_{L\to \infty}\ln\frac{L}{L+1}=\ln 1=0$, temos
$\int_{1}^\infty \frac{dx}{x^2+x}=\ln 2<\infty$, logo converge.
\eqref{itImproprias6} converge.
\eqref{itImproprias65} Com $u=\ln x$, $\int\frac{\ln x}{x}\,dx=\int
u\,du=\frac{u^2}{2}+C$, logo $\int_{3}^\infty \frac{\ln x}{x}\,dx$ diverge.
\eqref{itImproprias7} converge (pode escrever $x^4=u^2$, onde $u=x^2$)
\end{sol}
\end{exo}

\begin{exo}
Se $f:[0,\infty)\to \bR$, a \grasA{transformada de Laplace} de $f(x)$ é a
função $L(s)$ definida pela integral imprópria
\eq{L(s)\pardef \int_0^\infty e^{-sx}f(x)\,dx\,,\quad s\geq 0\,.}
Calcule as transformadas de Laplace das seguintes funções $f(x)$:
\begin{multicols}{4}
\begin{enumerate}
\item\label{itTRansfLapl1} $k$ (constante)
\item\label{itTRansfLapl2} $x$
\item\label{itTRansfLapl4} $\sen x$
\item\label{itTRansfLapl5} $e^{-\alpha x}$
\end{enumerate}
\end{multicols}
\vspace{0.01cm}
\begin{sol}
\eqref{itTRansfLapl1} $L(s)=\frac{k}{s}$.
\eqref{itTRansfLapl2} $L(s)=\frac{1}{s^2}$.
\eqref{itTRansfLapl4} Integrando duas vezes por partes, é fácil
verificar que $L(s)$ satisfaz $L(s)=\frac{1}{s}(\frac{1}{s}-\frac{1}{s}L(s))$.
Logo, $L(s)=\frac{1}{1+s^2}$.
\eqref{itTRansfLapl5} $L(s)=\frac{1}{s+\alpha}$.
\end{sol}
\end{exo}

\begin{exo} Estude $f(x)\pardef \frac{x}{x^2+1}$. Em seguida, 
calcule a área da região contida no semi-espaço $x\geq 0$, delimitada pelo
gráfico de $f$ e pela sua assíntota horizontal.
\begin{sol}
A função tem domínio $\bR$, é ímpar e possui a assíntota horizontal 
$y=0$, a
direita e esquerda.
A sua derivada vale $f'(x)=\frac{1-x^2}{(x^2+1)^2}$. Logo, $f$ decresce em
$(-\infty,-1]$, possui um mínimo local em $(-1,\tfrac{1}{2})$, cresce em
$[-1,+1]$, possui um
um máximo local em $(+1,\tfrac{1}{2})$, e decresce em $[1,+\infty)$.
A derivada segunda vale $f''(x)=\frac{2x(x^2-3)}{(x^2+1)^3}$. Logo, $f$ possui
três pontos de inflexão: em $(-\sqrt{3},-\frac{\sqrt{3}}{4})$,
$(0,0)$ e $(\sqrt{3},\frac{\sqrt{3}}{4})$, e é côncava em
$(-\infty,-\frac{\sqrt{3}}{4}]$, convexa em $[-\tfrac{\sqrt{3}}{4},0]$, côncava
em $[0,\tfrac{\sqrt{3}}{4}]$, e convexa em
$[\tfrac{\sqrt{3}}{4},+\infty)$.
\begin{center}
\begin{bmlimage}\begin{tikzpicture}
\newcommand{\funcao}[1]{( 2*(#1)/( (#1)^2 + 1 ))}
\draw[>=latex, ->] (-6,0)--(6,0);
\draw[>=latex, ->] (0,-1.5)--(0,1.5);
\draw[thick, domain=-6:6, samples=70] plot (\x,{\funcao{\x}});
\fill[areagrafico] (0,0)--plot[domain=0:5.5,
samples=50](\x,{\funcao{\x}})--(5.5,0)--cycle;
\draw[thick, domain=-6:6, samples=70] plot (\x,{\funcao{\x}});
\end{tikzpicture}\end{bmlimage}
\end{center}
Vemos que a área procurada é dada pela integral imprópria
$$
\int_0^\infty\frac{x}{x^2+1}dx=\lim_{L\to\infty}\int_0^L\frac{x}{x^2+1}dx=\lim_{
L\to\infty} \ln (L^2+1)=+\infty\,.
$$
\end{sol}
\end{exo}

\begin{exo}
Estude a função $f(x)\pardef \frac{e^x}{1+e^x}$.
Em seguida, calcule a área da região contida no semi-plano $x\geq 0$ delimitada
pelo gráfico de $f$ e pela sua assíntota.
\begin{sol}
$f$ tem domínio $\bR$, e é sempre positiva. Já que 
$$
\lim_{x\to +\infty} \frac{e^x}{1+e^x}=\lim_{x\to+\infty}
\frac{1}{1+e^{-x}}=1\,,\quad 
\lim_{x\to-\infty} \frac{e^x}{1+e^x}=0\,,
$$
$f$ tem duas assíntotas horizontais: a reta $y=0$ a esquerda, e a reta $y=1$ a
direita.
Como $f'(x)=\frac{e^x}{(1+e^x)^2}$ é sempre positiva, $f$ é crescente em todo
$x$ (não possui mínimos ou máximos locais). Como
$f''(x)=\frac{e^x(1-e^x)}{(1+e^x)^2}$,
e que essa é positiva quando $x\leq 0$, negativa quando $x\geq 0$, temos que $f$
é convexa em $(-\infty,0]$, côncava em $[0,\infty)$, e possui um ponto de
inflexão em $(0,\tfrac12)$: 
\begin{center}
\begin{bmlimage}\begin{tikzpicture}
\newcommand{\funcao}[1]{(exp(#1))/(1+ exp(#1))}
\draw[>=latex, ->] (-6,0)--(6,0);
\draw[>=latex, ->] (0,-0.2)--(0,1.5);
\draw[dashed] (0,1)--(6,1);
\fill[areagrafico] (0,0.5)--plot[domain=0:5.5,
samples=50](\x,{\funcao{\x}})--(5.5,1)--(0,1)--cycle;
\draw[thick, domain=-6:6, samples=60] plot (\x,{\funcao{\x}});
\end{tikzpicture}\end{bmlimage}
\end{center}

A área procurada é dada pela integral imprópria
$$\int_0^\infty\Big\{1-\frac{e^x}{1+e^x}\Big\}dx=\int_0^\infty\frac{1}{1+e^x}
dx$$
Com $u=e^x+1$ dá $du=e^x\,dx=(u-1)\,dx$, e
$$
\int \frac{1}{1+e^x}dx=\int\frac{1}{u(u-1)}du\,.
$$
A decomposição desta última fração dá
$$
\int\frac{1}{u(u-1)}du=-\int\frac{du}{u}+\int \frac{du}{u-1}=-\ln|u|+\ln|u-1|+C
$$
Logo,
\begin{align*}
\int_0^\infty\frac{1}{1+e^x}dx=\lim_{L\to\infty}\int_0^L\frac{1}{1+e^x}dx=
&\lim_{L\to\infty}\Big\{
-\ln (e^x+1)+\ln e^x
\Big\}_0^L\\
&=\lim_{L\to\infty}\Big\{
-\ln (1+e^{-x})
\Big\}_0^L\\
&=\ln 2
\end{align*}
\end{sol}
\end{exo}

Intuitivamente, para uma função $f$ 
contínua possuir uma integral imprópria convergente no infinito, ela precisa
tender a zero.  Vejamos que precisa de mais do que isso, no seguinte
exercício:
\begin{exo}\label{Exo:tendpasverzerointfinie}
Dê um exemplo de uma função contínua positiva $f:[0,\infty)\to \bR_+$ que não
tende a zero no infinito, 
e cuja integral imprópria $\int_0^\infty f(x)\,dx$ converge.
\begin{sol}
Considere por exemplo a seguinte função $f$:
\begin{center}
\begin{bmlimage}\begin{tikzpicture}
\pgfmathsetmacro{\n}{7};
\draw[>=latex, ->] (-0.2,0)--({\n+0.5},0);
\draw[>=latex, ->] (0,-0.2)--(0,1.3);
\draw (0,1) node{$-$} node[left]{$1$};
\foreach \k in {1,...,\n} {
\draw (\k,0) node{$\shortmid$} node[below]{$\k$};
\coordinate (A) at (\k,1);
\coordinate (B) at ({\k-(1/(2^(\k)))},0);
\coordinate (C) at ({\k+(1/(2^(\k)))},0);
\fill[areagrafico] (B)--(A)--(C)--cycle; 
\draw[thick] (B)--(A)--(C);
}
\end{tikzpicture}\end{bmlimage}
\end{center}
Fora dos triângulos, $f$ vale zero.
O primeiro triângulo tem base de largura $1$, o segundo $\frac{1}{2}$, o
$k$-ésimo $\frac{1}{2^{k-1}}$, etc. Logo, a integral de $f$ é igual à soma
das áreas dos triângulos:
$$
\int_0^\infty f(x)\,dx=\tfrac12+\tfrac14+\tfrac18+\tfrac{1}{16}+\dots=1\,.
$$
Assim, a integral imprópria converge. Por outro lado, já que $f(k)=1$ para
todo inteiro positivo $k$, $f(x)$ não tende a zero quando $x\to \infty$.
\end{sol}
\end{exo}

\begin{exo}
Considere a \grasA{função Gamma}, definida da seguinte maneira:
\[ 
\forall z>0\,,\quad \Gamma(z)\pardef \int_0^\infty x^ze^{-x}\,dx\,.
\]
Verifique que $\Gamma(0)=1$, $\Gamma(1)=1$, $\Gamma(2)=2$, $\Gamma(3)=6$.
Mostre que para todo inteiro $n$,
\[ 
\Gamma(n)=n\cdot \Gamma(n-1)\,.
\]
Conclua que nos inteiros, $\Gamma(n)=n!$.
\end{exo}

\section{As integrais $\int_a^\infty\frac{dx}{x^p}$}

Consideremos as funções $f(x)=\frac{1}{x^p}$, onde $p$ é um número positivo.
Sabemos (lembre da Seção \ref{Subsec:graficpotnegativ}) que quanto maior $p$,
mais rápido $\tfrac{1}{x^p}$ tende a zero (lembre sa Seção
\ref{Sec:GraficosPotencias}):

\begin{center}
\begin{bmlimage}\begin{tikzpicture}[scale=1]
\pgfmathsetmacro{\a}{4}
\draw [thick, domain=0.55:\a, samples=60] plot (\x,{1/(\x)});
\draw [thick, dotted, domain=0.7:\a, samples=70] plot
(\x,{1/((\x)^2)});
\draw [thick, dashed, domain=0.8:\a, samples=70] plot
(\x,{1/((\x)^3)});

\draw [>=latex, ->] (-0.2,0)--(4,0) node[right]{$x$};
\draw [>=latex, ->] (0,-0.2)--(0,{2.7}) node[left]{$\tfrac{1}{x^p}$};
 \pgfmathsetmacro{\b}{5};
 \pgfmathsetmacro{\c}{2.2};
 \draw (\b,\c) node[left]{$p=1:$};
 \draw[thick] (\b,\c)--(\b+1,\c);
 \draw (\b,\c-0.5) node[left]{$p=2:$};
 \draw[thick, dotted] (\b,\c-0.5)--(\b+1,\c-0.5);
 \draw (\b,\c-1) node[left]{$p=3:$};
 \draw[thick, dashed] (\b,\c-1)--(\b+1,\c-1);
\end{tikzpicture}\end{bmlimage}
\end{center}

Logo, é razoável acreditar que para
valores de $p$ suficientemente grandes, a integral
imprópria $\int_a^\infty\frac{dx}{x^p}$ deve convergir.
O seguinte resultado determina exatamente os valores de $p$ para os quais a
integral converge ou diverge, e mostra que o valor $p=1$ é crítico:

\begin{teo}\label{Teo:ConvSerieHarmon} Seja $a>0$. Então
\eq{\int_a^\infty\frac{dx}{x^p}
\begin{cases}
\text{converge se $p>1$}\\
\text{diverge se $p\leq 1$.}
\end{cases}}
\end{teo}
\begin{proof}
O caso crítico $p=1$ já foi considerado no Exemplo \eqref{Ex:unsurxdiverge}:
para todo 
$a>0$,
$$\int_a^\infty\frac{dx}{x}=\lim_{L\to\infty}\int_a^L\frac{dx}{x}=\lim_{L\to
\infty}\bigl\{\ln L-\ln a\big\}=\infty\,.$$
Por um lado, quando $p\neq 1$, 
$$\int_a^L\frac{dx}{x^p}=\frac{x^{-p+1}}{-p+1}\Big|_a^L
=\frac{1}{1-p}\Bigl\{ \frac{1}{L^{p-1}}-\frac{1}{a^{p-1}}\Bigr\}
$$
Lembra que pelo Exercício \ref{utilparaimproprias} que se 
$p>1$ então $p-1>0$, logo $\lim_{L\to\infty}\frac{1}{L^{p-1}}=0$, e a
integral 
$$\int_a^\infty\frac{dx}{x^p}=\lim_{L\to\infty}\int_a^L\frac{dx}{x^p}
=\frac { 1 } { (p-1)a^{p-1}}<\infty\,,$$ logo converge.
Por outro lado se $p<1$, então $1-p>0$,
$\lim_{L\to\infty}\frac{1}{L^{p-1}}=\infty$ e
$$\int_a^\infty\frac{dx}{x^p}=\lim_{L\to\infty}\int_a^L\frac{dx}{x^p}=\infty\,,
$$
isto é diverge.
\end{proof}

% Assim,
% $\int_1^\infty \frac{dx}{x^{5}}$, $\int_1^\infty
% \frac{dx}{x^{\frac{5}{2}}}$, $\int_1^\infty \frac{dx}{x^{2}}$, e 
% $\int_1^\infty \frac{dx}{x^{1,000001}}$ são todas convergentes, enquanto
% $\int_1^\infty \frac{dx}{x^{\frac{1}{100}}}$, $\int_1^\infty
% \frac{dx}{x^{\frac{2}{5}}}$, $\int_1^\infty \frac{dx}{\sqrt{x}}$ e 
% $\int_1^\infty \frac{dx}{x^{0,99999}}$
% são todas divergentes.

\begin{exo}
Estude as seguintes integrais impróprias em função do parâmetro $\alpha$:
\begin{multicols}{3}
\begin{enumerate}
\item\label{itIntImpropp1} $\int_a^\infty\frac{dx}{\sqrt{x}^\alpha}$
\item\label{itIntImpropp2} $\int_1^\infty\frac{1}{x^{\alpha^2-3}}\,dx$
\item\label{itIntImpropp3} $\int_a^\infty\frac{dx}{(\ln x)^{2\alpha} x}$
\end{enumerate}
\end{multicols}
\vspace{0.01cm}
\begin{sol}
\eqref{itIntImpropp1} Como $\frac{1}{\sqrt{x}^\alpha}=\frac{1}{x^{p}}$ com
$p=\alpha/2$, a integral 
converge se e somente se $\alpha>2$.
\eqref{itIntImpropp2} Defina $p:=\alpha^2-3$.
Pelo Teorema \ref{Teo:ConvSerieHarmon}, sabemos que a integral converge se
$p>1$, diverge caso contrário. Logo, a integral converge se
$\alpha>2$ ou $\alpha<-2$, e ela diverge se $-2\leq \alpha\leq 2$.
\eqref{itIntImpropp3} Converge se e somente se $\alpha>1/2$ (pode fazer $u=\ln
x$).
\end{sol}
\end{exo}


\begin{exo} Fixe $q>0$ e considere o sólido de revolução obtido rodando a
curva $y=\frac{1}{x^q}$, $x\geq 1$, em
torno do eixo $x$. Determine para quais valores de $q$ esse sólido tem volume
finito.
\begin{sol}
O volume do sólido é dado pela integral imprópria 
$$
V=\pi\int_1^\infty\Bigl(\frac{1}{x^q}\Bigr)^2\,dx=\pi\int_1^\infty\frac{dx}{
x^{2q}}\,.
$$
Pelo Teorema \ref{Teo:ConvSerieHarmon}, essa integral converge se $2q>1$ (isto
é se $q>\tfrac12$), diverge caso contrário.
\end{sol}
\end{exo}

\section{O critério de comparação}
\index{critério! de comparação}
Em geral, nas aplicações, a primeira questão é de saber se uma integral
imprópria converge ou não. Em muitos casos, é mais importante saber se uma
integral converge do que conhecer o seu valor exato.\\

O nosso objetivo nesta seção será de mostrar como a convergência/divergência de
uma integral imprópria pode às vezes ser obtida por \emph{comparação} com uma
outra integral imprópria, mais fácil de estudar. Comecemos com um exemplo
elementar:

\begin{ex}
Pela definição, estudar a 
integral imprópria $\int_1^\infty\frac{dx}{x^3+1}$ significa estudar o limite 
$\lim_{L\to \infty}\int_1^L\frac{dx}{x^3+1}$. Ora, calcular a
primitiva de $\frac{1}{x^3+1}$ é possível, mas dá um certo trabalho, como
visto no Exercício \ref{Exo:PrimitivasDecomposicao}. Por outro lado, 
em termos do comportamento em $x$ para $x$ grande, a função $\frac{1}{x^3+1}$
não é muito diferente da função $\frac{1}{x^3}$. Na verdade, para
todo $x>0$, $x^3+1$ é sempre
\emph{maior} que $x^3$. Logo, $\frac{1}{x^3+1}$ é \emph{menor} que
$\frac{1}{x^3}$ no intervalo $[1,\infty)$, o que se traduz, em termos de
integral definida, por
$$\int_1^L\frac{dx}{x^3+1}\leq \int_1^L\frac{dx}{x^3}\,.$$
Tomando o limite $L\to \infty$ em ambos lados obtemos
\eq{\label{eq:intintint}\int_1^\infty\frac{dx}{x^3+1}\leq
\int_1^\infty\frac{dx}{x^3}\,.}
Logo, \emph{se a integral do lado direito de \eqref{eq:intintint} é finita, a do
lado esquerdo é finita também}. Ora, a do lado direito é da forma
$\int_1^\infty\frac{dx}{x^p}$ com $p=3>1$. Logo, pelo Teorema
\ref{Teo:ConvSerieHarmon}, ela converge, portanto \eqref{eq:intintint} implica
que $\int_1^\infty\frac{dx}{x^3+1}$ converge também. 

Assim, foi provado com custo
mínimo que  $\int_1^\infty\frac{dx}{x^3+1}$ converge, sem passar pela primitiva
de $\frac{1}{x^3+1}$.
O leitor interessado em calcular o valor exato de
$\int_1^\infty\frac{dx}{x^3+1}$, poderá usar a primitiva obtida no Exercício
\ref{Exo:PrimitivasDecomposicao}.
\end{ex}

Comparação pode ser usada também para mostrar que uma integral diverge:

\begin{ex}
Considere $\int_3^\infty\frac{\ln x}{x}\,dx$. Aqui, podemos lembrar da integral
$\int_3^\infty\frac{dx}{x}$, que diverge pelo Teorema \ref{Teo:ConvSerieHarmon}.
As duas integrais podem ser comparadas observando que $\ln x\geq 1$ para todo
$x\geq 3>e$, logo $\frac{\ln x}{x}\geq \frac{1}{x}$ para todo $x\in
[3,\infty)$. Logo, após ter tomado o limite $L\to \infty$,
$$
\int_3^\infty\frac{\ln x}{x}\,dx\geq \int_3^\infty\frac{dx}{x}\,.
$$
Logo, como a integral do lado diverge e vale $+\infty$, a do lado direito
também.  
\end{ex}

É importante ressaltar que o método usado acima funciona \emph{somente se as funções comparadas são
ambas não-negativas!}
O método de comparação pode ser resumido da seguinte maneira:

\begin{pro}
Sejam $f,g:[a,\infty)\to \bR$ contínuas, tais que $0\leq f(x)\leq g(x)$ para
todo $x\in
[a,\infty)$. Então 
$$\int_a^\infty f(x)\,dx\leq \int_a^\infty g(x)\,dx$$
Em particular, se $\int_a^\infty g(x)\,dx$ converge, então 
$\int_a^\infty f(x)\,dx$ converge também, e se $\int_a^\infty f(x)\,dx$
diverge, então $\int_a^\infty g(x)\,dx$ diverge também.
\end{pro}

\begin{obs}
O método de comparação é útil em certos casos, mas ele não diz 
\emph{qual} deve ser a função usada na comparação.
Em geral, a escolha da função depende da situação. Por exemplo, a presença
de $x^3$ no denominador levou naturalmente a comparar
$\frac{1}{x^3+1}$ com $\frac{1}{x^3}$, cuja integral imprópria é finita.
Portanto, para mostrar que uma integral imprópria $\int_a^\infty f(x)\,dx$
converge, é preciso procurar uma função $g$ tal que $0\leq f(x)\leq g(x)$ e cuja
integral imprópria é finita;
para mostrar que $\int_a^\infty f(x)\,dx$
diverge, é preciso procurar uma função $h$ tal que $f(x)\geq h(x)\geq 0$ e cuja
integral imprópria é infinita.
\end{obs}

\begin{exo}
Quando for possível, estude as seguintes integrais via uma comparação.
\begin{multicols}{3}
\begin{enumerate}
\item\label{itIntCompar1} $\int_1^\infty\frac{dx}{x^2+x}$
\item\label{itIntCompar2} $\int_1^\infty \frac{dx}{\sqrt{x}(x+1)}$
\item\label{itIntCompar3} $\int_0^\infty \frac{dx}{1+e^x}$
\item\label{itIntCompar4} $\int_2^\infty \frac{e^x}{e^x-1}\,dx$
\item\label{itIntCompar5} $\int_0^\infty\frac{dx}{2x^2+1}$ 
\item\label{itIntCompar6} $\int_3^\infty\frac{dx}{x^2-1}$ 
\item\label{itIntCompar22} 
$\int_1^\infty\frac{\sqrt{x^2+1}}{x^2}dx$
\item\label{itIntCompar7} $\int_1^\infty\frac{x^2-1}{x^4+1}dx$
\item\label{itIntCompar8} $\int_1^\infty\frac{x^2+1+\sen x}{x}dx$
\item\label{itIntCompar9} $\int_{e^2}^\infty e^{-(\ln x)^2}\,dx$
\end{enumerate}
\end{multicols}
\vspace{0.01cm}
\begin{sol}
\eqref{itIntCompar1} Como $x^2+x\geq x^2$ para
todo $x\in [1,\infty)$, temos também $\frac{1}{x^2+x}\leq \frac{1}{x^2}$ neste
intervalo, logo
$\int_1^\infty\frac{dx}{x^2+x}\leq
\int_1^\infty\frac{dx}{x^2}<\infty$, converge. 
\eqref{itIntCompar2} Como $x+1\geq x$ para todo $x\geq 1$,
$\int_1^\infty
\frac{dx}{\sqrt{x}(x+1)}\leq 
\int_1^\infty \frac{dx}{\sqrt{x}x}= \int_1^\infty \frac{dx}{x^{3/2}}<\infty$,
converge.
\eqref{itIntCompar3} $\int_0^\infty \frac{dx}{1+e^x}\leq \int_0^\infty
e^{-x}\,dx<\infty$, converge.
\eqref{itIntCompar4} $\int_1^\infty \frac{e^x}{e^x-1}\,dx\geq 
\int_1^\infty \frac{e^x}{e^x}\,dx=\int_1^\infty dx=\infty$, diverge.
\eqref{itIntCompar5} Como
$\int_0^\infty\frac{dx}{2x^2+1}=\int_0^1\frac{dx}{2x^2+1}
 +\int_1^\infty\frac{dx}{2x^2+1}$ e
$\int_1^\infty\frac{dx}{2x^2+1}\leq
\int_1^\infty\frac{dx}{2x^2}<\infty$, temos que $\int_0^\infty\frac{dx}{2x^2+1}$
converge.
\eqref{itIntCompar6} Escrevendo
$\tfrac{1}{x^2-1}=\tfrac{1}{x^2}\tfrac{x^2}{x^2-1}$, e observando que o máximo
da função $\tfrac{x^2}{x^2-1}$ no intervalo $[3,\infty)$ é $\tfrac98$, temos 
$\int_3^\infty\frac{dx}{x^2-1}\leq \tfrac98
\int_3^\infty\frac{dx}{x^2}<\infty$, logo a integral converge.
Um outro jeito de fazer é de observar que se $x\geq 3$, então $x^2-1\geq x^{3/2}$.
\eqref{itIntCompar22} 
Como $\sqrt{x^2+1}\geq \sqrt{x^2}=x$ em todo o intervalo de integração,
$\int_1^\infty\frac{\sqrt{x^2+1}}{x^2}dx\geq
\int_1^\infty\frac{x}{x^2}dx=\int_1^\infty\frac{1}{x}dx$.
Como aqui é uma integral do tipo $\int_1^\infty\frac{1}{x^p}dx$ com $p=1$, ela é
divergente. Logo, pelo critério de comparação,
$\int_1^\infty\frac{\sqrt{x^2+1}}{x^2}dx$ {diverge} também.\\
% (Obs: pode também calcular a primitiva da função, o que dá mais trabalho. Por
% exemplo, por partes,
% \begin{align*}
% \int\frac{1}{x^2}\sqrt{x^2+1}dx&=\frac{-1}{x}\sqrt{x^2+1}-\int(\frac{-1}{x}
% )\frac{2x}{2\sqrt{x^2+1}}dx\\
% &=\frac{-1}{x}\sqrt{x^2+1}+\int\frac{1}{\sqrt{x^2+1}}dx
% \end{align*}
% Usando $x=\sec\theta$ nesta última integral dá
% $\int\frac{1}{\sqrt{x^2+1}}dx=\int \sec\theta
% d\theta=(\cdots)=\ln|x+\sqrt{x^2+1}|+C$. Logo,
% $$\int\frac{1}{x^2}\sqrt{x^2+1}dx=\ln|x+\sqrt{x^2+1}|-\frac{\sqrt{x^2+1}}{x}+C\,
% .$$
% Calculando aquele limite, obtem-se  também que a integral diverge.)
\eqref{itIntCompar7} $\int_1^\infty\frac{x^2-1}{x^4+1}\,dx\leq
\int_1^\infty\frac{x^2}{x^4}\,dx=\int_1^\infty\frac{1}{x^2}\,dx<\infty$,
converge.
\eqref{itIntCompar8} Como $\sen x\geq -1$,
$\int_1^\infty\frac{x^2+1+\sen x}{x}\,dx\geq 
\int_1^\infty\frac{x^2}{x}\,dx
=\int_1^\infty x\,dx=\infty$, diverge.
\eqref{itIntCompar9}
Como $\ln x \geq 2$ para todo $x\geq e^2$, temos que
$\int_{e^2}^\infty e^{-(\ln x)^2}\,dx\leq 
\int_{e^2}^\infty e^{-2\ln x}\,dx=\int_{e^2}^\infty\frac{dx}{x^2}$, que converge.
\end{sol}
\end{exo}

Consideremos agora um resultado contraintuitivo, decorrente do manuseio
de integrais impróprias:

\begin{ex}
Considere o sólido de revolução obtido rodando o gráfico da função
$f(x)=\frac{1}{x^q}$ em torno do eixo $x$, para $x\geq 1$ (o sólido obtido é às
vezes chamado de ``vuvuzela''). O seu volume é dado por
$$
V=\int_1^\infty\pi f(x)^2\,dx=\pi\int_1^\infty\frac
{dx}{x^{2q}}\,,
$$
que é convergente se $p>\tfrac12$, divergente caso contrário. Por outro lado, 
como $f'(x)=\frac{-q}{x^{q+1}}$, a
área da sua superfície é dada por 
$$
A=\int_1^\infty2\pi f(x) \sqrt{1+f'(x)^2}\,dx
=2\pi\int_1^\infty\tfrac{1}{x^q}\sqrt{1+\tfrac{q^2}{x^{2(q+1)}}}\,dx
$$
Como $\sqrt{1+\tfrac{q^2}{x^{2(q+1)}}}\geq 1$, temos
$A\geq 2\pi \int_1^\infty\tfrac{dx}{x^{q}}$,
que é divergente se $q\leq 1$. Logo, é interessante observar que quando
$\tfrac12
<q\leq 1$, o sólido de revolução considerado possui um volume finito, mas uma
superfície infinita.
\end{ex}


\section{Integrais impróprias em $\bR$}

Integrais impróprias foram até agora definidas em intervalos
semi-infinitos, da forma $[a,\infty)$ ou $(-\infty,b]$.

\begin{defin}
Seja $f:\bR\to \bR$. Se existir um $a\in \bR$ tal que as integrais
impróprias 
$$
\int_{-\infty}^af(t)\,dt\,,\quad \int_a^\infty f(t)\,dt
$$
existem, então diz-se que a \grasA{integral imprópria
$\int_{-\infty}^\infty f(t)\,dt$ converge}, e o seu valor é
definido como 
$$\int_{-\infty}^\infty f(t)\,dt\pardef \int_{-\infty}^a f(t)\,dt+\int_{a}^\infty f(t)\,dt\,.$$
\end{defin}

\begin{exo}
Mostre que a função definida por 
$$
g(t)\pardef \frac{1}{\sqrt{2\pi t}}\int_{-\infty}^\infty
e^{-\frac{x^2}{2t}}\,dx\,,\quad t>0
$$
é bem definida. Isto é: a integral imprópria converge para
qualquer valor de $t>0$. Em seguida, mostre que $g$ é 
constante~\footnote{Pode ser mostrado (ver
Cálculo III) que essa constante é $1$.}.
\begin{sol} Observe que se $0\leq x<1$, então $e^{-x^2/2t}\leq 1$,
e se $x\geq 1$, então $x^2\geq x$, logo
$e^{-x^2/2t}\leq e^{-x/2t}$. Logo,
$$\int_{0}^\infty
e^{-\frac{x^2}{2t}}\,dx= \int_{0}^1
e^{-\frac{x^2}{2t}}\,dx+ \int_{1}^\infty
e^{-\frac{x^2}{2t}}\,dx \leq \int_0^1 \,dx+\int_1^\infty
e^{-x/2t}\,dx\,.$$
Como essa última integral converge (ela pode ser calculada
explicitamente), por comparação $\int_{0}^\infty
e^{-\frac{x^2}{2t}}\,dx$ converge também. Como $x\mapsto
e^{-x^2/2t}$ é par, isso implica que $f(t)$ é bem definida.
Com a mudança $y=x/\sqrt{t}$, temos 
$$
 \frac{1}{\sqrt{2\pi t}}\int_{0}^\infty
e^{-\frac{x^2}{2t}}\,dx= \frac{1}{\sqrt{2\pi}}\int_{0}^\infty
e^{-\frac{y^2}{2}}\,dy\,,
$$
que não depende de $t$. Assim, $f$ é constante.
\end{sol}
\end{exo}

%%%%%%%%%%%%%%%%%%%%%%%%%%%%%%%%%%
\section{Em intervalos finitos}
\index{integral imprópria! em intervalo finito}

Consideremos agora o problema de integrar uma função num
intervalo finito, por exemplo da forma $]a,b]$.
Aqui, suporemos que $f:]a,b]\to \bR$ é contínua, mas possui uma descontinuidade,
ou uma assíntota vertical em $a$.

A integral de $f$ em $]a,b]$ será definida de maneira parecida: 
escolheremos um $\epsilon>0$, calcularemos a integral
de Riemann de $f$ em $[a+\epsilon,b]$,
e \emph{em seguida} tomaremos o limite $\epsilon\to 0^+$:

\begin{defin}
Seja $f:]a,b]\to \bR$ uma função contínua. Se o limite 
\eq{\int_{a^+}^b f(x)\,dx\pardef \lim_{\epsilon\to 0^+}\int_{a+\epsilon}^b
f(x)\,dx\,}
existir e for finito, diremos que \grasA{a integral imprópria 
$\int_{a^+}^b f(x)\,dx$ converge}. Caso contrário, ela \grasA{diverge}.
Integrais impróprias para $f:[a,b)\to \bR$ se definem da mesma maneira:
\eq{\int_{a}^{b^-} f(x)\,dx\pardef \lim_{\epsilon\to 0^+}\int_{a}^{b-\epsilon}
f(x)\,dx\,.}
\end{defin}

\begin{ex}
A função $\frac{1}{\sqrt{x}}$ é contínua no intervalo $]0,1]$, mas possui uma
assíntota vertical em $x=0$. 
\begin{center}
\begin{bmlimage}\begin{tikzpicture}[scale=1.3]
\pgfmathsetmacro{\eps}{0.2};
\draw (\eps,0) node{$\shortmid$} node[below]{$\epsilon$};
\fill[areagrafico]
(\eps,0)--(1,0)--plot[domain=1:{1/sqrt(\eps)}]({1/(\x^2)},\x)--(\eps,{
1/sqrt(\eps)})--cycle;
\draw[->] (\eps,{0.5/sqrt(\eps)})--({\eps/2},{0.5/sqrt(\eps)});
\draw[>=latex, ->] (-0.2,0)--(1.5,0);
\draw[>=latex, ->] (0,-0.2)--(0,2.6) node[left]{$\tfrac{1}{\sqrt{x}}$};
%\draw (0.55,0.2) node{área$=1$};
\draw[dotted] (1,0)--(1,1);
\draw (1,0) node[below]{$1$};
\draw[thick, domain=\eps/2:1.3] plot (\x,{1/(sqrt(\x))});
\end{tikzpicture}\end{bmlimage}
\end{center}
Para definir a sua integral em $]0,1]$, usemos uma integral imprópria:
$$
\int_{0^+}^1\frac{dx}{\sqrt{x}}\pardef\lim_{\epsilon\to
0^+}\int_{\epsilon}^1\frac{dx}{\sqrt{x}}=
\lim_{\epsilon\to 0^+}\bigl\{2\sqrt{x}\bigr\}\Big|_\epsilon^1
=2\lim_{\epsilon\to 0^+}\{1-\sqrt{\epsilon}\}=2\,.
$$
Assim, apesar da função tender a $+\infty$ 
quando $x\to 0^+$, ela delimita uma área finita.
\end{ex}

\begin{ex}
Suponha que se queira calcular a área da região finita delimitada pelo
eixo $x$ e pelo gráfico da função $f(x)=x(\ln x)^2$ (essa função foi estudada no
Exercício \ref{Exo:DoisEstudosLegais}):
\begin{center}
\begin{bmlimage}\begin{tikzpicture}[xscale=2, yscale=1.5]
\fill[areagrafico] (0,0)--plot[domain=0.001:1]
(\x,{\x*(ln(\x))^2})--cycle;
\draw [thick, domain=0.001:1.7, samples=250] plot (\x,{\x*(ln(\x))^2})
node[right]{$x(\ln x)^2$};
 \draw [>=latex, ->] (0,0)--(1.8,0);
 \draw [>=latex, ->] (0,-0.1)--(0,0.8);
\draw (1,0) node{$\shortmid$} node[below]{$1$};
\end{tikzpicture}\end{bmlimage}
\end{center}
Como $f(x)$ não é definida em $x=0$, essa área precisa ser calculada via a
integral imprópria 
$$
\int_{0^+}^1x(\ln x)^2\,dx=\lim_{\epsilon\to 0^+}\int_\epsilon^1 x(\ln
x)^2\,dx\,.
$$
A primitiva de $x(\ln x)^2$ para $x>0$ já foi calculada no Exercício
\ref{Exo:Bonnardpartessubsit}: logo,
$$ \int_\epsilon^1 x(\ln
x)^2\,dx=\Bigl\{
\tfrac12 x^2(\ln x)^2-\tfrac12 x^2\ln
x+\tfrac14x^2\Bigr\}\Big|_\epsilon^1=
\tfrac14-\tfrac12 \epsilon^2(\ln \epsilon)^2+\tfrac12 \epsilon^2\ln
\epsilon-\tfrac14\epsilon^2
$$
Pode ser verificado, usando a Regra de B.H., que 
$\lim_{\epsilon\to 0^+}\epsilon^2(\ln \epsilon)^2=\lim_{\epsilon\to
0^+}\epsilon^2\ln \epsilon=0$, logo
$$
\int_{0^+}^1x(\ln x)^2\,dx=\tfrac14\,.
$$
\end{ex}


\begin{exo}
Estude as integrais impróprias abaixo. Se convergirem, dê os seus valores.
\begin{multicols}{3}
\begin{enumerate}
\item\label{itIntimpropr0} ${\int_{0}^{1^-}\frac{dx}{\sqrt{1-x}}}$
\item\label{itIntimpropr1} ${\int_{0^+}^1\frac{\ln(x)}{\sqrt{x}}dx}$
\item\label{itIntimpropr2}  $\int_{0^+}^\infty\frac{dt}{\sqrt{e^t-1}}$
\end{enumerate}
\end{multicols}
\vspace{0.01cm}
\begin{sol}
\eqref{itIntimpropr0} Por definição,
$\int_{0}^{1^-}\frac{dx}{\sqrt{1-x}}=\lim_{\epsilon\to
0^+}\int_0^{1-\epsilon}\frac{dx}{\sqrt{1-x}}=
\lim_{\epsilon\to 0^+}\{-2\sqrt{1-x}\}_0^{1-\epsilon}=2$. Logo, a integral
converge.
\eqref{itIntimpropr1} $\int_{0^+}^1\frac{\ln(x)}{\sqrt{x}}dx=\lim_{\epsilon\to
0^+}\int_\epsilon^1\frac{\ln(x)}{\sqrt{x}}dx$.
Integrando por partes, definindo $f'(x)\pardef \frac{1}{\sqrt{x}}$, $g(x)\pardef
\ln
(x)$, temos $f(x)=2\sqrt{x}$, $g'(x)=\frac{1}{x}$, e
\begin{align*}
\int \frac{\ln(x)}{\sqrt{x}}dx
=2\sqrt{x}\ln (x)-2\int \frac{\sqrt{x}}{x}dx
&=2\sqrt{x}\ln (x)-2\int \frac{1}{\sqrt{x}}dx\\
&=2\sqrt{x}\ln (x)-4\sqrt{x}+C\,.
\end{align*}
(Obs: pode também começar com $u=\sqrt{x}$, e acaba calculando $4\int
\ln(u)du$.)
Logo,
\begin{align*}
\int_{0^+}^1\frac{\ln(x)}{\sqrt{x}}dx&=\lim_{\epsilon\to 0^+}
\big\{
2\sqrt{x}\ln (x)-4\sqrt{x}+C
\big\}_\epsilon^1\\
&=\lim_{\epsilon\to 0^+}
-4-2\sqrt{\epsilon}\ln (\epsilon)+4\sqrt{\epsilon}=-4\,.\\
\end{align*}
Este último passo é justificado porqué $\lim_{\epsilon\to
0^+}\sqrt{\epsilon}=0$, e porqué uma simples aplicação da Regra de
Bernoulli-l'Hôpital dá $\lim_{\epsilon\to 0^+}\sqrt{\epsilon}\ln
(\epsilon)=-\lim_{y\to +\infty}\frac{\ln (y)}{\sqrt{y}}=0$.
Como o limite existe e é finito, a integral imprópria acima { converge e o
seu valor é $-4$}.

\eqref{itIntimpropr2}
Observe que a função $\frac{1}{\sqrt{e^t-1}}$ não é definida em $t=0$, logo é
necessário dividir a integral em duas integrais impróprias: 
\begin{align*}
\int_{0^+}^\infty\frac{1}{\sqrt{e^t-1}}dt&=\int_{0^+}^1\frac{1}{\sqrt{e^t-1}}
dt+\int_1^\infty\frac{1}{\sqrt{e^t-1}}dt\\
&=\lim_{\epsilon\to 0^+}\int_\epsilon^1\frac{1}{\sqrt{e^t-1}}dt+\lim_{L\to
\infty}\int_1^L\frac{1}{\sqrt{e^t-1}}dt\,.
\end{align*}
Para calcular a primitiva, seja $u=\sqrt{e^t-1}$,
$du=\frac{e^t}{2\sqrt{e^t-1}}dt$, i.é. $dt=\frac{2u}{u^2+1}du$, e 
\begin{align*}
\int\frac{1}{\sqrt{e^t-1}}dt=2\int\frac{du}{u^2+1} &=2\arctan (u)+C\\
&=2\arctan\sqrt{e^t-1}+C
\end{align*}
Logo,
$$\lim_{\epsilon\to
0^+}\int_\epsilon^1\frac{1}{\sqrt{e^t-1}}dt=2\lim_{\epsilon\to
0^+}\arctan\sqrt{e^t-1}\big|_\epsilon^1=2\arctan\sqrt{e-1}$$
$$\lim_{L\to \infty}\int_1^L\frac{1}{\sqrt{e^t-1}}dt=2\lim_{L\to
\infty}\arctan\sqrt{e^t-1}\big|_1^L=\pi-2\arctan\sqrt{e-1}$$
Como esses dois limites existem, 
$\int_0^\infty\frac{dt}{\sqrt{e^t-1}}$ {converge, e o seu valor é
$\pi$}.
\end{sol}
\end{exo}


%%%%%%%%%%%%%%%%%%%%%%%%%%%%%%%%%%%

%\newpage




\appendix



\chapter{Soluções dos exercícios}

\ifdefined\updateans
% Only need to run once in a lifetime, when the file ans.tex needs to be updated.
\Closesolutionfile{ans}
{\scriptsize{\Readsolutionfile{ans}}}
\else
\newenvironment{Solution}{\textbf}{}
{
\iflatexml\else\scriptsize\fi
\protect \section *{Capítulo \ref {Cap_Fundam}}
\begin{Solution}{1.1}
\eqref{it0} $S=\{0\}$
\eqref{it01} $S=\{\pm 1\}$
 \eqref{it02} Observe primeiro que $0$ não é solução (a divisão por zero no lado esquerdo
não é nem definida). Assim, multiplicando por $x$ e rearranjando obtemos $x^2+x-1=0$. Como
$\Delta=5>0$, obtemos duas soluções: $S=\{\tfrac{-1\pm \sqrt{5}}{2}\}$. (Obs: o número
$\tfrac{-1+\sqrt{5}}{2}=0.618033989...$ é às vezes chamado de \grasA{razão áurea}. Veja
$\verb|http://pt.wikipedia.org/wiki/Proporção_áurea|$)
 \eqref{it1} Para ter $(x+1)(x-7)=0$, é necessário que pelo menos um dos fatores, $(x+1)$
ou $(x-7)$, seja nulo. Isto é, basta ter $x=-1$ ou $x=7$. Assim, $S=\{-1,7\}$. Obs:
querendo aplicar a fórmula $x=\frac{-b\pm\sqrt{b^2-4ac}}{2a}$ de qualquer jeito, um aluno
com pressa pode querer expandir o produto $(x+1)(x-7)$ para ter $x^2-6x-7=0$, calcular
$\Delta=(-6)^2-4\cdot 1\cdot (-7)=64$, e obter
$S=\{\frac{-(-6)\pm\sqrt{64}}{2\cdot 1}\}=\{-1,7\}$.
 Mas além de mostrar uma falta de compreensão (pra que expandir uma expressão já
fatorada!?), isso implica aplicar uma fórmula e fazer \emph{contas}, o que cria várias
oportunidades de errar!)
\eqref{it2} $S=\bR$ (qualquer $x$ torna a equação verdadeira!)
 \eqref{it20} $S=\{0,1\}$
\eqref{it3} $S=\varnothing$
\eqref{it4} $S=\{-\tfrac13\}$
\eqref{it500} $S=\{\frac{-7\pm \sqrt{29}}{2}\}$.
\end{Solution}
\begin{Solution}{1.3}
 Resposta: não.
Sejam $a$ e $b$ os catetos do triângulo. Para ter uma área de $7$, é preciso ter
$\frac{ab}{2}=7$. Para ter um perímetro de $12$, é preciso ter $a+b+\sqrt{a^2+b^2}=12$
(o comprimento da hipotenusa foi calculada com o Teorema de Pitágoras).
Essa última expressão é equivalente a $12-a-b=\sqrt{a^2+b^2}$, isto é (tomando o quadrado
em ambos lados) $144-24(a+b)+2ab=0$. Como $b=\frac{14}{a}$, esta equação se reduz a uma
equação da única incógnita $a$: $6a^2-43a+84=0$. Como essa equação tem $\Delta=-167<0$,
não existe triângulo retângulo com aquelas propriedades.
\end{Solution}
\begin{Solution}{1.4}
$A=[-2,2]$, $B=[0,1)$, $C=(-\infty,0)$, $D=\varnothing$, $E=\bR$, $F=\{1\}$, $G=\{0\}$,
$H=\bR_+$.
\end{Solution}
\begin{Solution}{1.5}
 \eqref{itinequ1} $(-1,\infty)$
 \eqref{itinequ2} $(-\infty,\tfrac12]$
 \eqref{itinequ3} $(-\tfrac34,\infty)$
 \eqref{itinequ4} $(0,\infty)$
 \eqref{itinequ5} $(-\infty,-1]\cup [1,\infty)$
 \eqref{itinequ6} $\varnothing$
 \eqref{itinequ7} $\varnothing$
 \eqref{itinequ8} $\bR$
 \eqref{itinequ9} $(-\infty,0]\cup [1,\infty)$ Obs: aqui, um erro comum é de começar
dividindo ambos lados de $x\leq x^2$ por $x$, o que dá $1\leq x$. Isso dá somente uma
parte do conjunto das soluções, $[1,\infty)$, porque ao dividir por $x$, é preciso
considerar também os casos em que $x$ é negativo. Se $x$ é negativo, dividir por $x$ dá
$1\geq x$ (invertemos o sentido da desigualdade), o que fornece o outro pedaço das
soluções: $(-\infty,0]$.
 \eqref{itinequ10} $(-\infty,2)\cup (3,\infty)$
\eqref{itinequ10b} $(-\infty,-7]\cup \{0\}$
 \eqref{itinequ11} $(-1,+1)\cup (2,+\infty)$
\eqref{itinequ12} $[0,+\infty[$
\eqref{itinequ13} $S=(-\infty,-1]\cup (1,3]$. Cuidado: tem que excluir o valor
$x=1$ para evitar a divisão por zero\index{divisão por zero} e a inequação ser bem
definida.
\eqref{itt6} Primeiro observemos que os
valores $x=0$ e $x=-2$ são proibidos. Em seguida, colocando no mesmo denominador,
queremos resolver $\frac{2}{x(x+2)}\geq 0$. Isso é equivalente a resolver $x(x+2)\geq 0$,
cujo conjunto de soluções é dado por $(-\infty,-2]\cup [0,\infty)$. Logo,
$S=(-\infty,-2)\cup (0,\infty)$ (tiramos os dois valores proibidos).
\eqref{itt7} $S=(-\infty,0)\cup(2,\infty)$.
\eqref{itt7a} $S=(-\infty,-2]\cup [0,\tfrac43]\cup [3,+\infty)$.
\end{Solution}
\begin{Solution}{1.6}
Um só: $n=1$.
\end{Solution}
\begin{Solution}{1.7}
Resolvendo $0\leq 2x-3$ obtemos $S_1=[\tfrac32,\infty)$, e resolvendo $2x-3\leq
x+8$ obtemos $S_2=(-\infty,11]$. Logo, $S=S_1\cap S_2=[\tfrac32,11]$ é solução
das duas inequações no mesmo tempo. Mas esse intervalo contém os primos
$p=2,3,5,7,11$. Logo, a resposta é: $5$.
\end{Solution}
\begin{Solution}{1.8}
A expressão correta é a terceira, e vale para qualquer $x\in \bR$.
A primeira está certa quando $x\geq 0$, mas errada quando $x<0$ (por exemplo,
$\sqrt{(-3)^2}=\sqrt{9}=3(\neq -3)$). A segunda também está certa quando $x\geq
0$, mas $\sqrt{x}$ não é nem definido quando $x<0$.
\end{Solution}
\begin{Solution}{1.9}
\eqref{itt1} Observe que como um valor absoluto é sempre $\geq 0$,
qualquer $x$ é solução de $|x+27|\geq 0$. Logo, $S=\bR$. \eqref{itt2} Como no
item anterior, $|x-2|\geq 0$ para qualquer $x$. Logo, não tem nenhum $x$ tal que
$|x-2|<0$, o que implica $S=\varnothing$. \eqref{itt3} Para ter $|2x+3|>0$, a
única possibilidade é de excluir $|2x+3|=0$. Como isso acontece se e somente se
$2x+3=0$, isto é se e somente se $x=-\tfrac32$, temos $S=\bR\setminus
\{-\tfrac32\}=(-\infty,-\tfrac32)\cup(-\tfrac32,+\infty)$.
\eqref{itt4} Considere primeiro o caso em que $3-x\geq 0$ (isto é $x\leq 3$). A inequação
se torna $3<3-x$, isto é $x<0$. Logo, $S_1=(-\infty,0)$. No caso em que $3-x\leq 0$ (isto
é $x\geq 3$), a inequação se torna $3<-(3-x)$, isto é $x>6$. Assim, $S_2=(6,+\infty)$.
Finalmente, $S=S_1\cup S_2=(-\infty,0)\cup ]6,+\infty)$.
\eqref{itt4bis} $S=\varnothing$
\eqref{itt5} $S=[-\sqrt{2},\sqrt{2}]$. Observe que por
\eqref{eq:consequvalabsol}, $|x^2-1|\leq 1$ se e somente se
$-1\leq x^2-1\leq 1$. Assim, resolvendo separadamente as inequações $-1\leq x^2-1$ e
$x^2-1\leq 1$ leva ao mesmo conjunto de soluções.
\eqref{itt8} $S=(\tfrac43,2)\cup (2,4)$.
\end{Solution}
\begin{Solution}{1.10}
\eqref{itexsinal1} $<0$ se $x<-5$, $>0$ se $x>-5$, nula se $x=-5$.
\eqref{itexsinal2} $>0$ para todo $x\in \bR$.
\eqref{itexsinal21} $>0$ se $x\in\bR\setminus \{5\}$, nula se $x=5$.
\eqref{itexsinal3} $>0$ se $x\in (-\infty,-\sqrt{5})\cup (\sqrt{5},\infty)$, $<0$ se
$x\in (-\sqrt{5},\sqrt{5})$, nula se $x=\pm \sqrt{5}$
\eqref{itexsinal4} $>0$ se $x\in (-\infty,-8)\cup (2,6)$, $<0$ se $x\in (-8,2)\cup
(6,\infty)$, nula se $x\in \{-8,6\}$. Observe que a expressão \emph{não é definida em
$x=2$}.
\eqref{itexsinal5} $>0$ se $x\in (-1,1)\cup(1,\infty)$, $<0$ se $x<-1$, nula se $x\in
\{-1,1\}$.
\end{Solution}
\begin{Solution}{1.11}
\eqref{itplano1} $\{(x,y):y> 0\}$,
\eqref{itplano2} $\{(x,y):x< 0\}$,
\eqref{itplano3} $\{(x,y):|x|\leq \half, |y|\leq \half\}$,
\eqref{itplano4} $\{(x,y): x=2\}$,
\eqref{itplano5} $\{(x,y): y=-5\}$,
\eqref{itplano6} $\{(x,y): y=-5\}$,
\eqref{itplano7} $\{(x,y): 0\leq x\leq 2\}$,
\eqref{itplano8} $\{P=(x,y): d(P,(0,0))=1\}=\{(x,y):x^2+y^2=1\}$,
\eqref{itplano9} $\{P=(x,y): d(P,(1,-2))\leq 2\}=\{(x,y):(x-1)^2+(y+2)^2\leq 4\}$,
\end{Solution}
\begin{Solution}{1.12}
$R=(-\frac{391}{3},100)$, $T=(6,-\frac{9}{4})$.
\end{Solution}
\begin{Solution}{1.13}
\eqref{itreta1} $y=x$,
\eqref{itreta2} $y=1$,
\eqref{itreta3} $x=-3$,
\eqref{itreta4} $y=-\tfrac{5}{2}x+\tfrac12$,
\eqref{itreta5} $y=\tfrac{2}{3}x+5$.
\end{Solution}
\begin{Solution}{1.14}

\mbox{}
\begin{center}
\begin{bmlimage}\begin{tikzpicture}[scale=0.6]
\draw [ ->] (0,-3.3)--(0,2.4) node[left]{$y$};
\draw [ ->] (-5,0)--(5.4,0) node[right]{$x$};
\foreach \k in {-5,...,5}
{\draw ({\k},0) node{$\shortmid$};}
\foreach \k in {-3,...,2}
{\draw (0,{\k}) node{$-$};}
%%%%%%%%
\draw [very thick] (4,-2.5)--(4,2) node[right]{$r_1$};
%%%%%%%%%%%%
%\draw (4.2,0.2) node{$4$};
\draw [very thick] (4.8,-1.5)--(-4.5,-1.5) node[left]{$r_2$};
%\draw (0,-1.5) node[above left]{$-\tfrac{3}{2}$};
%%%%%%%%%%%%
%\draw (4.2,0.2) node{$4$};
\pgfmathsetmacro{\a}{-0.1}
\pgfmathsetmacro{\b}{2.3}
\draw [very thick] (\a,{(2*\a)-3})--(\b,{2*\b-3}) node[above]{$r_4$};
%%%
\pgfmathsetmacro{\c}{-3}
\pgfmathsetmacro{\d}{4.5}
\draw [very thick] (\d,{(-\d)/2})--(\c,{(-\c)/2}) node[left]{$r_3$};
\end{tikzpicture}\end{bmlimage}
\end{center}
\end{Solution}
\begin{Solution}{1.15}
\eqref{ittexreta1} $r':\,y=5x+10$.
\eqref{ittexreta2} $r':\,y=\tfrac{4}{3}x-9$
\end{Solution}
\begin{Solution}{1.16}
 Comecemos com um exemplo: considere a reta $r_1$ de inclinação $m_1=\tfrac13$
que passa pela origem. Qual é a equação da reta $r_2$, perpendicular a $r_1$, que passa
pela origem?
\begin{center}
 \begin{bmlimage}\begin{tikzpicture}
  \draw [ ->] (0,-0.1)--(0,3.2) node[right]{$y$};
\draw [ ->] (-1.5,0)--(3.5,0) node[right]{$x$};
\draw[thick] (-1,-0.33)--(3.3,1.1) node[right]{$r_1$};
\draw[dashed] (0.2,-0.6)--(-1.1,3.3) node[left]{$r_2$};
\fill (3,1) circle (0.55mm);
\draw (3,1) node[above left]{$P_1$};
\fill (-1,3) circle (0.55mm);
\draw (-1,3) node[above right]{$P_2$};
\foreach \k in {-1,...,3}
{\draw ({\k},0) node{$\shortmid$};}
\foreach \k in {0,...,3}
{\draw (0,{\k}) node{$-$};}
 \end{tikzpicture}\end{bmlimage}
\end{center}
 Observe que se $P_1=(3,1)\in r_1$, então o ponto $P_2=(-1,3)\in r_2$, já que o segmento
$OP_1$ precisa ser perpendicular a $OP_2$. Logo, a inclinação de $r_2$ pode ser obtida
usando o ponto $P_2$:
$m_2=\frac{0-3}{0-(-1)}=-3$,
 o que prova $m_2=-\frac{1}{m_1}$. Escolhendo qualquer outro ponto $P_1=(x,y)$ em $r_1$,
obteríamos um ponto $P_2=(-y,x)$, e $m_2$ seria calculada da mesma maneira.\\

 Para uma reta de inclinação $m_1$ qualquer, podemos escolher $P_1=(1,m_1)$ e
$P_2=(-m_1,1)$, assim $m_2=\frac{0-1}{0-(-m_1)}=-\frac{1}{m_1}$ é sempre verificada.
\end{Solution}
\begin{Solution}{1.17}
$r_2$ e $r_4$ são paralelas, e ambas são perpendiculares a $r_3$.
\end{Solution}
\begin{Solution}{1.18}
\eqref{itexcirc1} $C=(0,-1)$, $R=3$.
\eqref{itexcirc2} não é círculo: $-1$ não é um quadrado.
\eqref{itexcirc3} $C=(3,0)$, $R=3$.
 \eqref{itexcirc4} não é círculo: depois de ter completado o quadrado obtemos
$(x+\half)^2+(y+\half)^2=-\half$, que não é um quadrado.
 \eqref{itexcirc5} não é círculo: depois de ter completado o quadrado obtemos
$(x+1)^2+y^2=0$ (que poderia ser interpretado como um círculo de raio $R=0$ centrado em
$(-1,0)$).
\eqref{itexcirc6} não é círculo ($x^2-y^2=1$ é
uma \emph{hipérbole}).
\end{Solution}
\begin{Solution}{1.19}
 Durante uma hora e quinze minutos, o ponteiro dos segundos
dá $75$ voltas.
Como uma volta representa uma distância percorrida (pela ponta) de
$2\times \pi\times
20\simeq 125.66$ centímetros, a distância total é de $\simeq 9424.5$
centímetros, o que corresponde a $\simeq 94.25$ metros.
\end{Solution}
\begin{Solution}{1.21}
\mbox{}

\begin{center}
\begin{bmlimage}\begin{tikzpicture}[scale=2]
 \draw (0,0)--(0.5,0) node[midway, below]{$\tfrac12$} --(0.5,0.866)--(0,0) node[midway,
above left]{$1$};
\draw (0.6,0.35) node{$\frac{\sqrt{3}}{2}$};
\draw (0.225,0.15) node{$\tfrac{\pi}{3}$};
\draw (0.2,0) arc (0:60:0.2);
\draw (0.4,0.5) node{$\tfrac{\pi}{6}$};
\draw (0.4,0.7) arc (240:265:0.25);
\draw (1.5,0.7) node[right]{$\Rightarrow\,\sen \tfrac{\pi}{3}=\frac{\sqrt{3}}{{2}}\,, \quad
\cos \tfrac{\pi}{3}=\frac{1}{{2}}\,,\quad \tan \tfrac{\pi}{3}=\sqrt{3}$\,.};
\draw (1.5,0.2) node[right]{$\Rightarrow\,\sen \tfrac{\pi}{6}=\frac{1}{{2}}\,, \quad
 \cos \tfrac{\pi}{6}=\frac{\sqrt{3}}{{2}}\,,\quad \tan
\tfrac{\pi}{6}=\tfrac{1}{\sqrt{3}}$\,.};
\draw[dotted] (0.5,0)--(1,0)--(0.5,0.866)--cycle;
\end{tikzpicture}\end{bmlimage}
\end{center}
\end{Solution}
\begin{Solution}{1.22}
$H=\frac{d(\tan\beta-\tan\alpha)}{\tan \alpha\tan\beta}$.
\end{Solution}
\begin{Solution}{1.23}
Todas essas identidades seguem da observação do círculo trigonométrico. Por exemplo,
o desenho
\begin{center}
\begin{bmlimage}\begin{tikzpicture}[scale=3]
\pgfmathsetmacro{\a}{1};
\draw[dotted] (\a,0) arc (0:180:\a);
\draw[ ->, color=gray!70] (-1.1*\a,0) -- (1.1*\a,0);
\draw[ ->, color=gray!70] (0,0) -- (0,1.1*\a);

\pgfmathsetmacro{\alf}{35};

%DESSINER LES ANGLES:
\draw[ ->] ({0.4*\a},0) arc (0:\alf:{0.4*\a});
%\draw ({(\alf)/2}:{(\a)*(0.45)}) node{$\alpha$};
\draw ({\alf/2}:{\a*0.45}) node{$\alpha$};
\draw[ ->] ({0.3*\a},0) arc (0:180-\alf:{0.3*\a});
\draw ({(180-\alf)/2}:{0.35*\a}) node[above]{$\pi-\alpha$};

%DEFINIR LES POINTS:
\coordinate (B) at ({\a*cos(\alf)},{\a*sin(\alf)});
\draw (B) node[above right]{$B$};
\fill (B) circle (0.15 mm);
\draw (0,0)--(B);

\coordinate (C) at ({\a*cos(180-\alf)},{\a*sin(180-\alf)});
%\draw (C) node[above left]{$C$};
\fill (C) circle (0.15 mm);
\draw (0,0)--(C);

\draw[dotted] (C)--(B);

 \draw [color=\coulseno, thick] (B)--({\a*cos(\alf)},0) node[midway, above, sloped]{$\sen
\alpha$};
 \draw [color=\coulcoseno, thick] ({\a*cos(\alf)},0)--(0,0) node[midway, below]{$\cos
\alpha$};
 \draw [color=\coulseno, thick] (C)--({\a*cos(180-\alf)},0) node[midway, above,
sloped]{$\sen (\pi-\alpha)$};
 \draw [color=\coulcoseno, thick] ({\a*cos(180-\alf)},0)--(0,0) node[midway, below]{$\cos
(\pi-\alpha)$};

\end{tikzpicture}\end{bmlimage}
\end{center}
mostra que $\cos(\pi-\alpha)=-\cos\alpha$ e $\sen(\pi-\alpha)=\sen\alpha$.
 Como consequência,
$$\tan(\pi-\alpha)=\frac{\sen(\pi-\alpha)}{\cos(\pi-\alpha)}=-\tan \alpha\,.$$
Deixemos o leitor provar as identidades parecidas com $\pi+\alpha$.
Por outro lado, o desenho
\begin{center}
\begin{bmlimage}\begin{tikzpicture}[scale=3]
\pgfmathsetmacro{\a}{1};
\draw[dotted] (\a,0) arc (0:90:\a);
\draw[ ->, color=gray!70] (0,0) -- (1.1*\a,0);
\draw[ ->, color=gray!70] (0,0) -- (0,1.1*\a);

\pgfmathsetmacro{\alf}{35};

%DESSINER LES ANGLES:
\coordinate (P) at ({0.4*\a*cos(\alf)},{0.4*\a*sin(\alf)});
\draw[ <-] (P) arc (\alf:90:{0.4*\a});
\draw[ ->] ({0.4*\a},0) arc (0:\alf:{0.4*\a});
\draw ({\alf/2}:{0.45*\a}) node{$\alpha$};
\draw ({\alf+(90-\alf)/2}:{0.35*\a}) node[above right]{$\tfrac{\pi}{2}-\alpha$};

%DEFINIR LES POINTS:
\coordinate (B) at ({\a*cos(\alf)},{\a*sin(\alf)});
\coordinate (Bx) at ({\a*cos(\alf)},0);
\coordinate (By) at (0,{\a*sin(\alf)});

\draw (B) node[above right]{$B$};
\draw (0,0)--(B);
\draw [color=\coulseno, thick] (B)--(Bx) node[midway, above, sloped]{$\sen \alpha$};
\draw [color=\coulcoseno, thick] (Bx)--(0,0) node[midway, below]{$\cos \alpha$};
 \draw [color=\coulseno, thick] (By)--(B) node[midway, above,
sloped]{$\sen(\tfrac{\pi}{2}- \alpha)$};
 \draw [color=\coulcoseno, thick] (By)--(0,0) node[midway, below,
sloped]{$\cos(\tfrac{\pi}{2}- \alpha)$};

\fill (B) circle (0.15 mm);
\end{tikzpicture}\end{bmlimage}
\end{center}
 mostra que $\cos(\tfrac{\pi}{2}-\alpha)=\sen\alpha$ e
$\sen(\tfrac{\pi}{2}-\alpha)=\cos\alpha$.
Como consequência,
$$
 \tan
(\tfrac{\pi}{2}-\alpha)=\frac{\sen(\tfrac{\pi}{2}-\alpha)}{\cos(\tfrac{\pi}{2}-\alpha)}=
\frac{\cos\alpha}{\sen\alpha}\equiv \frac{1}{\tan \alpha}=\cot \alpha\,.
$$
\end{Solution}
\begin{Solution}{1.25}
 \eqref{eqsensomabis} segue de \eqref{eqsensoma} trocando $\beta$ por $-\beta$ e usando
\eqref{eqtrigo0}. Para provar
\eqref{eqcossoma}, basta usar  \eqref{eqsensomabis} da seguinte maneira:
\begin{align*}
 \cos(\alpha+\beta)&=\sen\bigl(\tfrac{\pi}{2}-(\alpha+\beta)\bigr)\\
&=\sen\bigl((\tfrac{\pi}{2}-\alpha)-\beta)\bigr)\\
&=\sen(\tfrac{\pi}{2}-\alpha)\cos\beta-\cos (\tfrac{\pi}{2}-\alpha)\sen \beta\\
&=\cos \alpha\cos\beta-\sen\alpha\sen\beta\,.
\end{align*}
Para \eqref{eqtansoma},
\begin{align*}
 \tan(\alpha+\beta)=\frac{\sen(\alpha+\beta)}{\cos(\alpha+\beta)}=
\frac{\sen\alpha\cos\beta+\cos \alpha\sen\beta}{
\cos\alpha\cos\beta-\sen \alpha\sen\beta}
=\frac{\tan \alpha+\tan\beta}{1-\tan\alpha\tan\beta}\,.
\end{align*}
 A última igualdade foi obtida dividindo o numerador e o denominador por
$\cos\alpha\cos\beta$.
\end{Solution}
\begin{Solution}{1.26}
As duas primeiras seguem das identidades anteriores, com $\beta=\alpha$.
A terceira obtem-se escrevendo:
$$
\sen\alpha=\sen(2\tfrac{\alpha}{2})=2\sen\tfrac{\alpha}{2}\cos\tfrac{\alpha}{2}=
2\tan\tfrac{\alpha}{2}\cos^2\tfrac{\alpha}{2}=\tan\tfrac{\alpha}{2}(\cos\alpha+1)\,.
$$
 Será que você consegue provar \eqref{eqidentandoisalpha} somente a partir do círculo
trigonométrico?

A última, \eqref{eqidentantreixx}, se obtem facilmente a partir de $\cos(\alpha\pm \beta)$. Observe que
a relação \eqref{eqidentantreixx} é a base da técnica chamada \emph{ring modulation} em música
eletrônica.
\end{Solution}
\begin{Solution}{1.27}
 A inclinação é dada por $\tan 60^o=\tan \frac{\pi}{3}=\sqrt{3}$ (Exercício
\ref{exo:calculsimple60}). Logo, a equação é $y=\sqrt{3}x-1-2\sqrt{3}$.
\end{Solution}
\begin{Solution}{1.28}
Observe que boa parte das equações desse exercício possuem \emph{infinitas} soluções!
As soluções obtêm-se essencialmente olhando para o círculo trigonométrico.
\eqref{itOinequ1} $S=\{\pisobredois\pm k\pi,\,k\in \bZ\}$.
\eqref{itOinequ10} $S=\{\pisobreseis\pm k2\pi\}\cup \{\tfrac{5\pi}{6}\pm k2\pi\}$
\eqref{itOinequ11} $S=\{\pisobrequatro\pm k\pi,\,k\in \bZ\}$.
\eqref{itOinequ2} $S=\{\pm k\pi\}\cup \{\pisobredois+2k\pi\}$.
\eqref{itOinequ3}  Observe que $z\pardef \sen x$ satisfaz $z^2+\tfrac{3}{2}z-1=0$, isto é
$z=\tfrac{1}{2}$ ou $-2$. Como o seno somente toma valores entre $-1$ e $1$, $\sen x=-2$
não possui soluções. Por outro lado, $\sen x=\half$ possui as soluções $\{\pisobreseis\pm
k2\pi\}\cup \{\tfrac{5\pi}{6}\pm k2\pi\}$, como visto em \eqref{itOinequ10}.
Portanto, $S=\{\pisobreseis\pm k2\pi\}\cup \{\tfrac{5\pi}{6}\pm k2\pi\}$.
\eqref{itOinequ4}  $S=[\tfrac{\pi}{6},\tfrac{5\pi}{6}]$ e as suas translações de $\pm
2k\pi$.
\eqref{itOinequ5}
$S=(\tfrac{\pi}{4},\tfrac{3\pi}{4})\cup(\tfrac{5\pi}{4},\tfrac{7\pi}{4})$ e as suas
translações de $\pm 2k\pi$.
 \eqref{itOinequ6} Rearranjando obtemos $\sen (2x)=-\tfrac12$, o que significa $2x\in
\{\frac{7\pi}{6}\pm 2k\pi\}\cup \{\frac{11\pi}{6}\pm 2k\pi\}$. Logo,
$S= \{\frac{7\pi}{12}\pm k\pi\}\cup \{\frac{11\pi}{12}\pm k\pi\}$
 \eqref{itOinequ7} $S=\{k\pi,k\in\bZ\}\cup \{\pisobretres+2k\pi,k\in\bZ\}\cup
\{\tfrac{5\pi}{3}+2k\pi,k\in\bZ\}$.
\end{Solution}
\protect \section *{Capítulo \ref {Cap:Funcoes}}
\begin{Solution}{2.1}
\eqref{itte1} $D=\bR\setminus\{-8,5\}$
\eqref{itte2} $D=\bR\setminus\{0\}$
\eqref{itte21} $D=\bR$
\eqref{itte3} $D=\bR$
\eqref{itte4} $D=\bR\setminus\{0,\tfrac12\}$
\eqref{itte45} $D=[1,\infty)$
\eqref{itte46} $D=(-\infty,-1]\cup [1,\infty)$
\eqref{itte5} $D=[1,\infty ) \setminus \{2\}$
\eqref{itdominio1} $D=\bR\setminus\{\pm 1\}$
\eqref{itdominio2} $D=(-1,+1)$
\eqref{itdominio3} $D=\{1\}$
 \eqref{itdominio4} $D=[0,1)$ (Atenção: é necessário que o numerador \emph{e} o
denominador sejam bem definidos.)
\eqref{itdominio41} $D=\bR\setminus \{\pisobredois+k\pi, k\in \bZ\}$
\eqref{itdominio5} $D=$união dos intervalos $[k2\pi,\pi +k2\pi]$, para $k\in \bZ$.
 \eqref{itdominio6} $D=\bR_+$. Observe que apesar da função ser identicamente nula, o seu
domínio não é a reta toda.
\eqref{itdominio7} $D=\{0\}$ (e não $D=\varnothing$!).
\end{Solution}
\begin{Solution}{2.2}
\eqref{itlimitacao1} $x^2$ é limitada inferiormente ($M_-=0$) mas não
superiormente: toma valores arbitrariamente grandes quando
$x$ toma valores grandes. \eqref{itlimitacao2} Não-limitada. De fato, $\tan x=\frac{\sen
x}{\cos x}$, e quando $x$ se aproxima por exemplo de $\pisobredois$, $\sen x$ se aproxima
de $1$ e $\cos x$ de $0$, o que dá uma divisão por zero. (Dê uma olhada no gráfico da
função tangente mais longe no capítulo.) \eqref{itlimitacao3} É limitada:
$\tfrac{1}{x^2+1}\geq 0\equiv M_-$, e como $x^2+1\geq
1$, temos
$\frac{1}{x^2+1}\leq \tfrac11=1\equiv M_+$. \eqref{itlimitacao4} Limitada
inferiormente ($M_-=0$), mas não superiormente: o domínio
dessa função é $(-\infty,1)$, e quando $x<1$ se aproxima de $1$, $\sqrt{1-x}$ se aproxima
de zero, o que implica que $\frac{1}{\sqrt{1-x}}$ toma valores arbitrariamente grandes.
\eqref{itlimitacao5} Observe que o denominador $x^3-x^2+x-1$ se anula em $x=1$. Logo, o
domínio da função é $\bR\setminus \{1\}$. Fatorando (ou fazendo a divisão),
$x^3-x^2+x-1=(x-1)(x^2+1)$. Portanto, quando $x\neq 1$,
$\frac{x-1}{x^3-x^2+x-1}=\frac{x-1}{(x-1)(x^2+1)}=\frac{1}{x^2+1}$. Como $\frac{1}{x^2+1}$
é limitada (item \eqref{itlimitacao3}), $\frac{x-1}{x^3-x^2+x-1}$ é limitada.
\eqref{itlimitacao6} Não-limitada. Apesar de $\sen x$ ser limitado por $-1$ e $+1$,
o ``$x$'' pode tomar valores arbitrariamente grandes.
\end{Solution}
\begin{Solution}{2.3}
\eqref{itgrafunc0} $f(x)=-1$, $D=\bR$
\eqref{itgrafunc1} $f(x)=-\sqrt{81-(x-5)^2}-4$, $D=[-4,14]$.
\eqref{itgrafunc2} $f(x)=\sqrt{25-x^2}$, $D=(-4,4)$
\eqref{itgrafunc3} $f(x)=-\sqrt{25-x^2}$, $D=[0,5]$
\end{Solution}
\begin{Solution}{2.4}
\mbox{}

\begin{bmlimage}\begin{tikzpicture}
\begin{scope}[xshift=-0.5cm]
\draw[thick] (-0.7,1)--(1,1);
\fill (1,1) circle (0.45mm);
\draw [thick, domain=1:1.5] plot (\x,{(\x)^2});
\draw [ ->] (-1.2,0)--(1.2,0);
\draw [ ->] (0,-0.4)--(0,{1.2});
\draw (0,1) node[right]{$1$};
\draw (-0.5,0.5) node{$f(x)$};
\end{scope}

\begin{scope}[xshift=1.8cm, yshift=1cm]
\draw [ ->] (-0.2,0)--(2.2,0);
\draw [ ->] (0,-1)--(0,0.5);
\draw (0,0.4) node[right]{$g(x)$};
\draw (1,0) node[above]{$1$};
\draw[thick] (0.2,-0.8)--(1,0)--(2,-1);
\end{scope}

\begin{scope}[xshift=5.8cm, yshift=0.5cm]
\draw [ ->] (-1.5,0)--(1.5,0);
\draw [ ->] (0,-1)--(0,1);
\draw (0,0.8) node[left]{$h(x)$};
\foreach \k in {-3,...,3}{
\pgfmathsetmacro{\a}{\k/3};
\fill (\a,\a) circle (0.45mm);
\draw[thick] (\a,\a)--({\a+0.333},\a);
\fill[intaberto] ({\a+0.333},\a) circle (0.45mm);
}
\end{scope}

\begin{scope}[xshift=9cm, yshift=0.5cm]
\draw [ ->] (-1.3,0)--(1.3,0);
\draw [ ->] (0,-1)--(0,1);
\draw (0,0.8) node[left]{$i(x)$};
\foreach \k in {-3,...,3}{
\pgfmathsetmacro{\a}{\k/3};
\fill (\a,0) circle (0.45mm);
\draw[thick] (\a,0)--({\a+0.333},0.333);
\fill[intaberto] ({\a+0.333},0.333) circle (0.45mm);
}
\end{scope}

\begin{scope}[xshift=12.5cm, yshift=0.1cm, scale=0.7]
\draw [ ->] (-1.3,0)--(1.3,0);
\draw [ ->] (0,-0.3)--(0,1.3) node[left]{$j(x)$};
%\draw [thick, domain=-1.5:1.5, samples=20] plot (\x,{abs(abs(\x)-1)});
\draw[thick] (-2.2,1.2)--(-1,0)--(0,1)--(1,0)--(2.2,1.2);
\end{scope}

\end{tikzpicture}\end{bmlimage}
\end{Solution}
\begin{Solution}{2.5}
A primeira curva é o gráfico da função $f(x)=-1$ se $x\leq 1$, $f(x)=2-x$ se $x>1$.
 A segunda não é um gráfico, pois os pontos $-\tfrac12 <x\leq 0$ têm duas saídas, o que
não é descrito por uma função (lembra que uma função é um mecanismo que a um entrada $x$
do domínio associa \emph{um (único)} número $f(x)$). No entanto, seria possível
interpretar aquela curva como a união dos gráficos de duas funções distintas: uma função
$f$ com domínio $(-\infty,0]$, e uma outra função $g$ com domínio $(-\tfrac12,\infty)$.
A terceira é o gráfico da função $f(x)=0$ se $x\in \bZ$, $f(x)=1$ caso contrário.
\end{Solution}
\begin{Solution}{2.6}
\eqref{itparidade1} É par: $f(-x)=\tfrac{(-x)}{(-x)^3-(-x)^5}=\tfrac{-x}{-(x^3-x^5)}=f(x)$.
\eqref{itparidade2} É par: $f(-x)=\sqrt{1-(-x)^2}=\sqrt{1-x^2}=f(x)$.
\eqref{itparidade3} É ímpar: $f(-x)=(-x)^2\sen (-x)=x^2(-\sen x)=-f(x)$.
\eqref{itparidade4} É par: $f(-x)=\sen (\cos(-x))=\sen(\cos x)=f(x)$.
\eqref{itparidade41} É ímpar: $f(-x)=\sen (\sen(-x))=\sen(-\sen x)=-\sen(\sen x)=-f(x)$.
\eqref{itparidade5} É par: $f(-x)=(\sen(-x))^2-\cos(-x)=(-\sen x)^2-\cos x=f(x)$.
 \eqref{itparidade6} Não é par nem ímpar, pois $f(\pisobrequatro)=\sqrt{2}$,
$f(-\pisobrequatro)=0$.
\eqref{itparidade7} Como $f(x)\equiv 0$, ela é par \emph{e} ímpar.
\end{Solution}
\begin{Solution}{2.7}
\eqref{itexodecr1} cresce na reta toda.
\eqref{itexodecr2} decrescce (estritamente) em $(-\infty,0]$, cresce (estritamente) em $[0,\infty)$.
\eqref{itexodecr3} decrescce (estritamente) em $(-\infty,0]$, cresce (estritamente) em $[0,\infty)$.
\eqref{itexodecr4}  cresce (estritamente) na reta toda.
\eqref{itexodecr5} decrescce (estritamente) em $(-\infty,0)$, decresce (estritamente) em $(0,\infty)$.
\eqref{itexodecr6} crescce (estritamente) em $(-\infty,0)$, decresce (estritamente) em $(0,\infty)$.
\eqref{itexodecr7} crescce (estritamente) em $(-\infty,\tfrac12]$, decresce (estritamente) em
$[\tfrac12,\infty$. (Será mais fácil resolver esse item depois de saber esboçar o gráfico de $x-x^2$,
veja o Exemplo \ref{exemplo_Funcoes_grafparabdesloc}.)
\eqref{itexodecr8} decrescce (estritamente) em $(-\infty,-1]$ e em $[0,1]$,
cresce (estritamente) em $[-1,0]$ e $[1,\infty)$.
\end{Solution}
\begin{Solution}{2.8}
Se a reta for vertical ($x=a$): $g(x)\pardef f(2a-x)$.
Se a reta for horizontal ($y=b$): $g(x)\pardef 2b-f(x)$.
\end{Solution}
\begin{Solution}{2.10}
\mbox{}
\begin{center}
 \begin{bmlimage}\begin{tikzpicture}[scale=0.5]
\draw [thick, domain=-8:8, samples=200] plot (\x,{1-abs(sin(\x r))});
\draw [ ->] (-9,0)--(9,0) node[right]{$x$};
\draw [ ->] (0,-0.3)--(0,{1.3}) node[left]{$f(x)$};
\pgfmathsetmacro{\Pi}{3.1415};
\draw ({-2*\Pi},0) node[below]{$\scriptstyle{ -2\pi}\,\,\,$};
\draw ({-2*\Pi},0) node{$\shortmid$};
\draw ({2*\Pi},0) node[below]{$\scriptstyle{ 2\pi}$};
\draw ({2*\Pi},0) node{$\shortmid$};
 \end{tikzpicture}\end{bmlimage}
\end{center}
Observe que o período de $f$ é $\pi$. Completando o quadrado\index{completar um quadrado},
$g(x)=-(x-\tfrac12)^2+\tfrac{5}{4}$:
\begin{center}
\begin{bmlimage}\begin{tikzpicture}
\draw [thick, domain=-0.8:1.8] plot (\x,{1.25-(\x-0.5)^2}) ;
\draw[dotted] (0,1.25)--(0.5,1.25)--(0.5,0);
\draw (0.5,1.25) node[above]{$\scriptstyle{(\tfrac12,\tfrac54)}$};
\fill (0.5,1.25) circle (0.45mm);
\draw [ ->] (-1,0)--(1.9,0);
\draw [ ->] (0,-0.4)--(0,1.7) node[left]{$g(x)$};
\end{tikzpicture}\end{bmlimage}
\end{center}
 Observe que a parábola corta o eixo $x$ nos pontos solução da equação $g(x)=0$, que são
$\frac{1\pm \sqrt{5}}{2}$.
 O gráfico da função $h$ já foi esboçado no Exercício \ref{Ex:graficosbasicos}. Mas aqui
vemos que ele pode ser obtido a partir do gráfico de $|x|$ por uma translação de $1$ para
baixo, composta por uma reflexão das partes negativas.
Como $i(x)$ é igual ao dobro de $\sen x$ e $j(x)$ à metade de $\sen x$, temos:
\begin{center}
 \begin{bmlimage}\begin{tikzpicture}[scale=0.5]
 \draw [color=gray!30, domain=-14:14, samples=100] plot (\x,{sin(\x r)})
node[color=gray!30, right]{$\sen x$};
\draw [thick, domain=-14:14, samples=100] plot (\x,{2*sin(\x r)}) node[right]{$i(x)$};
\draw [thick, domain=-14:14, samples=100] plot (\x,{0.5*sin(\x r)}) node[right]{$j(x)$};
\draw [ ->] (-15,0)--(15,0) node[right]{$x$};
\draw [ ->] (0,-1.3)--(0,{1.3});
\pgfmathsetmacro{\Pi}{3.1415};
 \end{tikzpicture}\end{bmlimage}
\end{center}
Completando o quadrado do numerador:
$k(x)=\frac{1-(x-1)^2}{(x-1)^2}=\frac{1}{(x-1)^2}-1$. Portanto, o gráfico pode ser obtido
a partir do gráfico de $\frac{1}{x^2}$:
\begin{center}
\begin{bmlimage}\begin{tikzpicture}[scale=0.7]
\pgfmathsetmacro{\a}{3.5}
\draw [thick, domain=-\a:-0.5, samples=100] plot (\x,{1/((\x)^2)});
\draw [thick, domain=0.5:\a, samples=100] plot (\x,{1/((\x)^2)});
\draw [ ->] (-1,-1)--(-1,2) node[left]{$y$};
\draw [ ->] (-3,1)--(2,1) node[right]{$x$};
\draw[dotted] (-3,0)--(4,0);
\draw[dotted] (0,-1)--(0,3);
\fill (0,0) circle (0.45mm);
\draw (0,0) node[below]{$(1,-1)$};
\end{tikzpicture}\end{bmlimage}
\end{center}
\end{Solution}
\begin{Solution}{2.11}
A trajetória é uma \emph{parábola}.
Resolvendo $y(x)=0$ para $x$, obtemos os pontos onde a parábola toca o chão: $x_1=0$
(ponto de partida), e
$x_2=\frac{2v_{\textsf{v}}v_{\textsf{h}}}{g}$ (distância na qual a partícula vai cair no
chão).
É claro que se o campo de gravitação é mais fraco (na lua por exemplo), $g$ é menor, logo
$x_2$ é maior: o objeto vai mais longe.
Por simetria sabemos que a abcissa do ponto mais alto da trajetória é
$x_*=\frac{x_2}{2}=\frac{v_{\textsf{v}}v_{\textsf{h}}}{g}$, e a sua abcissa é dada por
$y_*=y(x_*)=\tfrac12 \frac{v_{\textsf{v}}^2}{g}$. Observe que $y_*$ \emph{não depende de
$v_{\textsf{h}}$}.
O ponto $(x_*,y_*)$ pode também ser calculado a partir da trajetória $y(x)$, completando
o quadrado.
\end{Solution}
\begin{Solution}{2.12}
 \eqref{itineqgraf1} Se $f(x)=1-|x-1|$, $g(x)=|x|$,
\begin{center}
\begin{bmlimage}\begin{tikzpicture}[scale=0.7]
\draw [ ->] (-1.7,0)--(2,0) node[right]{$x$};
\draw [ ->] (0,-1)--(0,2) node[left]{$y$};
\draw[thick, dashed] (-1.5,1.5)--(0,0)--(1.5,1.5) node[right]{$g$};
\draw[thick] (-1,-1)--(1,1)--(3,-1) node[right]{$f$};
\pgfmathsetmacro{\x}{-1};
\pgfmathsetmacro{\y}{abs(\x-2)};
\draw (1,0) node{$\shortmid$};
\draw (1,0) node[below]{$\scriptstyle{1}$};
\end{tikzpicture}\end{bmlimage}
\end{center}
Logo, $S=[0,1]$. Para \eqref{itineqgraf2}, $S=\varnothing$.
\eqref{itineqgraf3} Se $f(x)=|x^2-1|$ (veja o gráfico do Exemplo
\ref{Ex:modulodografico}), vemos que $S=(-\sqrt{2},0)\cup(0,\sqrt{2})$.
\end{Solution}
\begin{Solution}{2.13}
Tinta: Como a esfera tem superfície igual a $4\pi r^2$, temos $T(r)=40\pi r^2$
(onde $r$ é medido em metros).
Concreto: Como o volume é dado por $V=\tfrac43\pi r^3$, o custo de concreto em
função do raio é $C(r)=40\pi r^3$. Como a superfície $s=4\pi r^2$ temos
$r=\sqrt{s/4\pi}$. Portanto,
$C(s)=40\pi(\tfrac{s}{4\pi})^{3/2}$.
\end{Solution}
\begin{Solution}{2.14}
 Por definição, $d(P,Q)=\sqrt{(a-1)^2+(b+2)^2}$.
Como $2a+b=2$, temos $d(a)=\sqrt{\tfrac54a^2-5a+10}$, e $d(b)=\sqrt{5b^2+5}$.
\end{Solution}
\begin{Solution}{2.15}
Perímetro: $P(n,r)=2nr\sen (\tfrac{\pi}{n})$.
Área: $A(n,r)=\tfrac12 nr^2\sen(\tfrac{2\pi}{n})$.
\end{Solution}
\begin{Solution}{2.16}
Suponha que o cone fique cheio de água, até uma altura de $h$ metros. Isso representa um
volume de
$V(h)=\tfrac13 (\pi h^2)\times h$ metros cúbicos. Logo, $h(V)=(\tfrac{3 V}{\pi})^{1/3}$.
Assim, a marca para $1m^3$ deve ficar na altura $h(1)\simeq 0.98$, para $2$ metros
cúbicos, $h(2)\simeq 1.24$, etc.
\begin{center}
\begin{bmlimage}\begin{tikzpicture}
 \draw (-3,3)--(0,0)--(3,3)--cycle;
 \fill[areagrafico] (-3,3)--(0,0)--(3,3)--cycle;
\draw (0,0)--(0,3);
\foreach \k in {1,...,5}{
\pgfmathsetmacro{\h}{(((3*\k)/3.141)^(0.333333))};
\draw (0,{\h}) node{$-$};
\draw[dotted] ({-\h},\h)--(\h,\h) node[right]{$\scriptscriptstyle{\k m^3}$};
}
\foreach \k in {6,...,28}{
\pgfmathsetmacro{\h}{(3*\k/3.1414)^(0.333333)};
\draw (0,{\h}) node{$-$};
\draw[dotted] ({-\h},{\h})--({\h},{\h});
}
\end{tikzpicture}\end{bmlimage}
\end{center}

\end{Solution}
\begin{Solution}{2.17}
Seja $x$ o tamanho do primeiro pedaço. Como os lados do quadrado medem
$\tfrac{x}{4}$, a área do quadrado é $\tfrac{x^2}{16}$. O círculo tem
circunferência igual a $L-x$, logo o seu raio vale $\tfrac{L-x}{2\pi}$, e a sua
área
$\pi(\tfrac{L-x}{2\pi})^2=\tfrac{(L-x)^2}{4\pi}$. Portanto a área total é dada por
$A(x)=\tfrac{x^2}{4}+\tfrac{(L-x)^2}{4\pi}$, e o seu domínio é $D=[0,L]$.
\end{Solution}
\begin{Solution}{2.18}
Seja $\alpha$ o ângulo entre $AB$ e $AC$.
Área: $A(\alpha)=\sen \tfrac{\alpha}{2}\cos\tfrac{\alpha}{2}=\tfrac12\sen \alpha$, com
$D=(0,\pi)$.
Logo, (olhe para a função $\sen \alpha$), a área é máxima para $\alpha=\tfrac{\pi}{2}$.
\end{Solution}
\begin{Solution}{2.19}
A área pode ser calculada via uma diferença de dois triângulos:
\begin{center}
\begin{bmlimage}\begin{tikzpicture}
\fill[areagrafico] (1,0)--(1,2)--plot[domain=1:2.5](\x,{\x+1})--(2.5,0)--cycle;
\draw[dotted] (1,0)--(1,2);
\draw (1,0) node{$\shortmid$};
\draw (1,0) node[below]{$1$};
\draw[dotted] (2.5,0)--(2.5,3.5);
\draw (2.5,0) node{$\shortmid$};
\draw (2.5,0) node[below]{$t$};
\draw (2.5,-0.1) node[below left]{$\leftarrow$};
\draw (2.5,-0.1) node[below right]{$\rightarrow$};
\draw [ ->] (0,-0.2)--(0,3);
\draw [ ->] (-0.2,0)--(3.5,0);
\draw[thick] (-0.5,0.5)--(3,4) node[right]{$r:\,y=x+1$};
\draw (5,2) node[right]{$A(t)=\tfrac{t^2}{2}+t-\tfrac32$};
\draw (1.7,1.3) node{$R_t$};
\end{tikzpicture}\end{bmlimage}
\end{center}
%\caption{Truc}
%\end{figure}
\end{Solution}
\begin{Solution}{2.21}
 Como $f(g(x))=\frac{1}{(x+1)^2}$, $g(f(x))=\frac{1}{x^2+1}$, temos
$(f\circ g)(0)=1$, $(g\circ f)(0)=1$, $(f\circ g)(1)=\frac14$, $(g\circ f)(1)=\frac12$.
Como $f(g(h(x)))=\frac{1}{(x+2)^2}$ e $h(f(g(x)))=\frac{1}{(x+1)^2}+1$,
 $f(g(h(-1)))=1$,
$h(f(g(3)))=\frac{17}{16}$.
\end{Solution}
\begin{Solution}{2.22}
\eqref{itexcompos1} $\sen (2x)=f(g(x))$, onde $g(x)=2x$, $f(x)=\sen x$.
\eqref{itexcompos2} $\frac{1}{\sen x}=f(g(x))$, onde $g(x)=\sen x$, $f(x)=\frac1x$.
\eqref{itexcompos3} $\sen(\frac{1}{x})=f(g(x))$, onde $f(x)=\sen x$, $g(x)=\frac1x$.
\eqref{itexcompos4} $\sqrt{\frac{1}{\tan (x)}}=f(g(h(x)))$, onde $f(x)=\sqrt{x}$,
$g(x)=\frac{1}{x}$, $h(x)=\tan x$.
\end{Solution}
\begin{Solution}{2.23}
$$
(g\circ f)(x)=
\begin{cases}
 2x+7&\text{ se }x\geq 0\,,\\
x^2&\text{ se }-\sqrt{3}<x<0\,,\\
2x^2+1&\text{ se }x\leq -\sqrt{3}\,.
\end{cases}
\quad\quad
(f\circ g)(x)=
\begin{cases}
 2x+4&\text{ se }x\geq 3\,,\\
x+3&\text{ se }0\leq x<3\,,\\
x^2&\text{ se }x<0\,.
\end{cases}
$$
\end{Solution}
\begin{Solution}{2.24}
\eqref{itconjimagem1} $\imagem(f)=\bR$,
\eqref{itconjimagem2} $\imagem(f)=[-1,3]$,
\eqref{itconjimagem21} Se $p>0$ então $D=\bR$ e $\imagem(f)=\bR$. Se
$p<0$ então $D=\bR\setminus\{0\}$ e $\imagem(f)=\bR\setminus\{0\}$
\eqref{itconjimagem22} $\imagem(f)=[0,\infty)$ se $p>0$, $\imagem(f)=(0,\infty)$ se $p<0$,
\eqref{itconjimagem3} $\imagem(f)=\bR\setminus \{0\}$,
\eqref{itconjimagem4} $\imagem(f)=(0,\infty)$,
\eqref{itconjimagem5a} $\imagem(f)=[1,\infty)$,
\eqref{itconjimagem5} $\imagem(f)=(-\infty,1]$,
\eqref{itconjimagem51} $\imagem(f)=[-1,\infty)$,
\eqref{itconjimagem6} $\imagem(f)=\bR$,
\eqref{itconjimagem7} $\imagem(f)=[-1,1]$,
\eqref{itconjimagem8} $\imagem(f)=(0,1]$,
\eqref{itconjimagem801} $\imagem(f)=[-\tfrac13,\tfrac13]$,
\eqref{itconjimagem81} $\imagem(f)=[-\tfrac{1}{\sqrt{2}},\tfrac{1}{\sqrt{2}}]$,
\eqref{itconjimagem9} $\imagem(f)=(0,1]$. De fato, $0<\frac{1}{1+x^2}\leq 1$. Melhor:
se $y\in (0,1]$ então $y=\frac{1}{1+x^2}$ possui uma única solução, dada por
$x=\sqrt{\frac{1-y}{y}}$.
\eqref{itconjimagem10} $\imagem(f)=(-\infty,-\tfrac12)\cup [1,\infty)$.

Para as funções do Exercício \ref{ExoEsbocosElementares}:
$\imagem(f)=(0,\infty)$,
$\imagem(g)=(-\infty,0]$,
$\imagem(h)=\bZ$,
$\imagem(i)=[0,1)$,
$\imagem(j)=[0,\infty)$.
\end{Solution}
\begin{Solution}{2.25}
Se trata de achar todos os $y\in \bR$ para os quais existe pelo menos um $x\in
\bR$ tal que $f(x)=y$. Isso corresponde a resolver a equação do segundo grau em
$x$: $yx^2-2x+25y=0$. Se $y=0$, então $x=0$. Se $y\neq 0$,
$x=\frac{1\pm\sqrt{1-25y^2}}{y}$, que tem solução se e somente se
$|y|\leq\tfrac15$.
Logo, $\imagem(f)=[-\tfrac15,\tfrac15]$. O ponto $y=0$ é o único que possui uma única preimagem, qualquer outro ponto de $\imagem(f)$ possui duas preimagens. Isso pode ser verificado no gráfico:
\begin{center}
 \begin{bmlimage}\begin{tikzpicture}[scale=0.5]
\draw[->] (-13,0)--(13,0);
\draw[->] (0,-3)--(0,3);
\draw[thick, domain=-12.5:12.5, samples=100] plot (\x,{(60*\x)/(9*(\x)^2+25)});
\draw[dotted] (0,2)node[left]{$\tfrac15$}--(5/3,2);
\draw[dotted] (0,-2)node[right]{$\tfrac15$}--(-5/3,-2);
\draw[dotted,->] (0,1)node[left]{$y$}--(0.5,1)--(0.5,0);
\draw (0,1) node{$-$};
\draw[dotted,->] (0,1)--(0.5,1)--(0.5,0);
\draw[dotted,->] (0,1)--(0.5,1)--(0.5,0);
\draw[dotted,->] (0,1)--(6,1)--(6,0);
 \end{tikzpicture}\end{bmlimage}
\end{center}
\end{Solution}
\begin{Solution}{2.26}
Observe que se $x\in (-1,0)$, então $f(x)\in (0,1)$. Por outro lado, se $y\in (0,1)$,
então existe um único $x\in (-1,0)$ tal que $f(x)=y$: $x=-\sqrt{1-y^2}$.
Logo, $f^{-1}:(0,1)\to(-1,0)$, $f^{-1}(x)=-\sqrt{1-x^2}$.
\begin{center}
\begin{bmlimage}\begin{tikzpicture}
 \draw[ ->] (-1.1,0)--(1.1,0) node[right]{$\scriptstyle{x}$};
 \draw[ ->] (0,-0.1)--(0,1.2) node[right]{$\scriptstyle{f(x)}$};
\draw[thick, <->] (0,1) arc (90:180:1);
\pgfmathsetmacro{\x}{-0.8};
\draw (\x,0) node[below]{$\scriptstyle{f^{-1}(y)}$};
\draw[dotted, <-] (\x,0)--(\x,{sqrt(1-(\x)^2)})--(0,{sqrt(1-(\x)^2)}) node[right]{$\scriptstyle{y}$};

\begin{scope}[xshift=5cm, yshift=1cm]
  \draw[ ->] (-1.1,0)--(1.3,0)node[right]{$\scriptstyle{x}$};
  \draw[ ->] (0,-1.2)--(0,0.4)node[left]{$\scriptstyle{f^{-1}(x)}$};
\draw[thick, <->] (0,-1) arc (270:360:1);
\pgfmathsetmacro{\x}{0.6};
\draw (\x,0) node[above]{$\scriptstyle{x}$};
\draw[dotted, ->] (\x,0)--(\x,{-sqrt(1-(\x)^2)})--(0,{-sqrt(1-(\x)^2)}) node[left]{$\scriptstyle{f^{-1}(x)}$};
\end{scope}
 \end{tikzpicture}\end{bmlimage}
\end{center}
\end{Solution}
\begin{Solution}{2.27}
O gráfico de $\frac{1}{x+1}$ é o de $\tfrac1x$ transladado de uma unidade para a esquerda.
O conjunto imagem é $(0,\infty)$. De fato, para todo $y\in (0,\infty)$, a equação $y=\frac{1}{x+1}$
possui uma solução dada por $x=\frac{1-y}{y}$. Logo, $f^{-1}:(0,\infty)\to (-1,\infty)$,
$f^{-1}(x)=\frac{1-x}{x}$.
\end{Solution}
\begin{Solution}{2.28}
Para verificar que $f^{-1}(-y)=-f^{-1}(y)$, usemos a definição: seja $x$ o único
$x$ tal que $f^{-1}(-y)=x$. Pela definição de função inversa ($(f\circ
f^{-1})(y)=y$), aplicando
$f$ temos $-y=f(x)$. Portanto, $y=-f(x)=f(-x)$ (pela imparidade de $f$).
Aplicando agora $f^{-1}$ obtemos $f^{-1}(y)=-x$, isto é, $x=-f^{-1}(y)$. Isso
mostra que
$f^{-1}(-y)=-f^{-1}(y)$.
\end{Solution}
\begin{Solution}{2.29}
Exemplos:
\eqref{itexobijecoes1} $f(x)=bx$
\eqref{itexobijecoes2} $f(x)=a+(b-a)x$
\eqref{itexobijecoes3} $f(x)=\tan \pisobredois{x}$, ou $f(x)=\tfrac{1}{(x-1)^2}-1$
\eqref{itexobijecoes4} $f(x)=\tan (\tfrac{2}{\pi}(x-\tfrac12))$
\end{Solution}
\begin{Solution}{2.32}
\ref{itexoresolvagraf1} $S=(\frac{1+\sqrt{5}}{2},+\infty)$
\ref{itexoresolvagraf2} $S=[0,1]$
\ref{itexoresolvagraf2} $S=\{-\tfrac52\}$
\end{Solution}
\begin{Solution}{2.33}
Por definição, $\sen y=\tfrac35$. Logo, $\cos y=+\sqrt{1-\sen^2 y}=\tfrac45$ (a raiz
positiva é escolhida, já que $y\in (0,\pisobredois)$). Portanto, $\tan y=\tfrac34$.
\end{Solution}
\begin{Solution}{2.34}
\eqref{itdomintriginv1} $[-1,1]$,
\eqref{itdomintriginv2} $[-\tfrac12,\tfrac12]$,
\eqref{itdomintriginv3} $(-1,1)$,
\eqref{itdomintriginv4} $(-\infty,-\tfrac{1}{\sqrt{2}}]\cup
[\tfrac{1}{\sqrt{2}},+\infty)$.
\end{Solution}
\begin{Solution}{2.35}
Seja $A$ a posição do topo da tela, $B$ a sua base, e $Q$ o ponto onde a parede toca o chão.
Seja $\alpha$ o ângulo $APQ$ e $\beta$ o ângulo $BPQ$.
Temos $\tan \alpha=\tfrac8x$, $\tan \beta=\tfrac3x$. Logo, em a):
$\theta(x)=\arctan\tfrac8x-\arctan\tfrac3x$. Em
b), $\theta(x)=\arctan\tfrac6x-\arctan\tfrac1x$.
\end{Solution}
\begin{Solution}{2.36}
\eqref{itexoinvtrig1} $x=\frac{1}{2}$
\eqref{itexoinvtrig2} $x=\sqrt{3}+1$
\eqref{itexoinvtrig3} $x=\tfrac16$
\eqref{itexoinvtrig4} $x=\tfrac{\sqrt{\pi}}{3}$
\end{Solution}
\begin{Solution}{2.37}
\eqref{itidenttriginv1} $\cos(2\arcos x)=2\cos^2(\arcos x)-1=2x^2-1$
\eqref{itidenttriginv2} $\cos(2\arcsin x)=1-2\sen^2(\arcsen x)=1-2x^2$
\eqref{itidenttriginv3} $\sen(2\arcos x)=2\sen (\arcos x)\cos (\arcos x)=2x\sqrt{1-x^2}$
\eqref{itidenttriginv4} $\cos(2\arctan x)=2\cos^2(\arctan x)-1=\tfrac{1-x^2}{1+x^2}$
\eqref{itidenttriginv5} $\sen (2\arctan x)=\frac{2x}{1+x^2}$
\eqref{itidenttriginv6} $\tan (2\arcsen x)=\frac{2x\sqrt{1-x^2}}{1-2x^2}$
\end{Solution}
\begin{Solution}{2.38}
Chamando $\alpha=\arcsen x$, $\beta=\arcos x$, temos $x=\sen \alpha$, $x=\cos \beta$:
\begin{center}
\begin{bmlimage}\begin{tikzpicture}[scale=2]
\pgfmathsetmacro{\a}{1};
\draw[dotted] (\a,0) arc (0:90:\a);
\draw[ ->, color=gray!70] (0,0) -- (1.1*\a,0);
\draw[ ->, color=gray!70] (0,0) -- (0,1.1*\a);
\pgfmathsetmacro{\alf}{35};
\coordinate (P) at ({0.4*\a*cos(\alf)},{0.4*\a*sin(\alf)});
\draw[<-] (P) arc (\alf:90:{0.4*\a});
\draw[->] ({0.4*\a},0) arc (0:\alf:{0.4*\a});
\draw ({\alf/2}:{0.45*\a}) node{$\alpha$};
\draw ({\alf+(90-\alf)/2}:{0.35*\a}) node[above right]{$\beta$};
\coordinate (B) at ({\a*cos(\alf)},{\a*sin(\alf)});
\coordinate (Bx) at ({\a*cos(\alf)},0);
\coordinate (By) at (0,{\a*sin(\alf)});
\draw (0,0)--(B);
\draw [thick] (B)--(Bx) node[midway, above, sloped]{$x$};
\draw [thick] (By)--(B);
\draw [thick] (By)--(0,0) node[midway, below, sloped]{$x$};
\end{tikzpicture}\end{bmlimage}
\end{center}
\end{Solution}
\protect \section *{Capítulo \ref {CAP:ExponLog}}
\begin{Solution}{3.1}
 Todos os gráficos podem ser obtidos por transformações de
$2^x$,\index{gráfico! transformação de}
$3^x$ e $(\frac32)^x$:
\begin{center}
 \begin{bmlimage}\begin{tikzpicture}[scale=0.7]
\draw[->] (-4.5,0)--(4.3,0) node[right]{$x$};
\draw[->] (0,-3)--(0,5);
\draw[domain=-2.1:4.1]  plot (\x,{1-exp(-\x*ln(2))}) node[right]{$1-2^{-x}$};
\draw[domain=-3.4:2.5]  plot (\x,{exp((\x-1)*ln(3))}) node[above]{$3^{x-1}$};
\draw[domain=-3.7:2.4]  plot (-\x,{exp(\x*ln(1.5)))})
node[above]{$(\tfrac{3}{2})^{-x}$};
\draw[domain=-3:3]  plot (\x,{(-1)*(exp(abs(\x)*ln(1.5)))})
node[right]{$-(\tfrac{3}{2})^{|x|}$};
\draw[dotted] (-3,1)--(4,1);
\draw (0,1) node[above right]{$1$};
 \end{tikzpicture}\end{bmlimage}
\end{center}
\end{Solution}
\begin{Solution}{3.2}
\eqref{itexoresolexp1} $S=\{0,2\}$.
\eqref{itexoresolexp2}  Tomando a raiz: $2^x=\pm 4$, mas como a função
exponencial somente toma valores positivos, $2^x=-4$ não possui soluções. Logo,
$S=\{2\}$.
\eqref{itexoresolexp3} Escrevendo a inequação como $2^{x+1}\leq 2^4$, vemos que
$S=\{x:x+1\leq 4\}=(-\infty,3]$.
\eqref{itexoresolexp6} $S=(-2,\infty)$.
\eqref{itexoresolexp4} $S=(-\infty,0)\cup (1,\infty)$.
\eqref{itexoresolexp5} $S=(\log_{20}\tfrac52,\infty)$.
\end{Solution}
\begin{Solution}{3.3}
Se $z=\log_a(x^y)$, então $z$ satisfaz $a^z=x^y$.
Por~\eqref{eq_ExpLog_debase},
podemos sempre escrever $x$ como $x=a^{\log_a x}$, o que permite
escrever $x^y=(a^{\log_ax})^y=a^{y\log_a x}$. Assim temos $a^z=a^{y\log_a x}$, o que implica
$z=y\log_ax$.
Se $z=\log_a\frac{x}{y}$, então
$$
a^z=\frac{x}{y}=\frac{a^{\log_ax}}{a^{\log_ay}}=a^{\log_ax-\log_ay},
$$
logo $z=\log_ax-\log_ay$.
\end{Solution}
\begin{Solution}{3.4}
$\log_4 16=2$,
$\log_\pi 1=0$,
$\log_2\frac{1}{16}=-4$,
$\log_{\tfrac12}8=-3$,
$7^{2\log_75}=25$.
\end{Solution}
\begin{Solution}{3.5}
Se $N(n)$ é o número de baratas depois de $n$ meses, temos $N(1)=3\cdot 2$,
$N(2)=3\cdot 2\cdot 2$, etc. Logo, $N(n)=3\cdot 2^n$. No fim de julho se
passaram $7$ meses, logo são $N(7)=3\cdot 2^7=384$ baratas. No fim do mês
seguinte são $384\times 2=768$ baratas.
Para saber quando a casa terá mais de um milhão de baratas, é preciso resolver
$N(n)>1000000$, isto é, $3\cdot 2^n>1000000$, que dá
$n>\log_2(1000000/3)=18,34...$,
isto é, no fim do $19$-ésimo mês, o que significa julho de $2012$...
\end{Solution}
\begin{Solution}{3.6}
\eqref{iteqdomlog2} $D=(-2,\infty)$
\eqref{iteqdomlog3} $D=(-\infty,2)$
\eqref{iteqdomlog1} Para $\log_6(1-x^2)$ ser definido, precisa $1-x^2>0$, que dá
$(-1,1)$. Por outro lado,
para evitar uma divisão por zero, precisa $\log_6(1-x^2)\neq 0$, isto é,
$1-x^2\neq 1$, isto é, $x\neq 0$. Logo, $D=(-1,0)\cup(0,1)$.
\eqref{iteqdomlog4} $D=(0,7]$
\eqref{iteqdomlog5} $D=(0,8)$
\eqref{iteqdomlog6} $D=(-\tfrac15,\infty)$
\eqref{iteqdomlog7} $D=\bR_+^*$
\end{Solution}
\begin{Solution}{3.7}
\eqref{itlogeq_1} $S=\{-3\}$,
\eqref{itlogeq_2} $S=\{997\}$,
\eqref{itlogeq_3} $S=\{0,1\}$,
\eqref{itlogeq_4} $S=\{\frac{\log_25}{1+\log_25}\}$,
\eqref{itlogeq_5} $S=\varnothing$,
\eqref{itlogeq_6} $S=\{-\tfrac{13}{8}\}$.
\eqref{itlogeq_7} $S=(-\infty,-1)$,
\eqref{itlogeq_8} $S=(-1,0)\cup(2,3)$,
\end{Solution}
\begin{Solution}{3.8}
As populações respectivas de bactérias depois de $n$ horas são:
$N_A(n)=123456\cdot 3^{\tfrac{n}{24}}$, $N_B(n)=20\cdot 2^n$.
Procuremos o $n_*$ tal que $N_A(n)=N_B(n)$, isto é (o logaritmo pode ser em
qualquer base):
$$n_*=\frac{\log_{10}123456-\log_{10}
20}{\log_{10}2-\tfrac{1}{24}\log_{10}3}=13.48...\,.$$
Isto é, depois de aproximadamente $13$ horas e meia, as duas colônias têm o
mesmo número de indivíduos.
Depois desse instante, as bactérias do tipo $B$ são sempre maiores em número.
De fato (verifique!), $N_A(n)<N_B(n)$ para todo $n>n_*$.
\end{Solution}
\begin{Solution}{3.9}
Se $y\in \bR_+^*$, procuremos uma solução de
$y=\frac{3^x+2}{3^{-x}}$. Essa equação se reduz a $(3^x)^2+2\cdot 3^x-y=0$, isto
é $3^x=-1\pm \sqrt{1+y}$. Como $y>0$, vemos que a solução positiva dá uma única
preimagem $x=\log_3(-1+\sqrt{1+y})\in \bR$. Logo $f$ é uma bijeção e
$f^{-1}:\bR_+^*\to \bR$ é dada por $f^{-1}(y)=\log_3(-1+\sqrt{1+y})$.
\end{Solution}
\begin{Solution}{3.10}
\eqref{itbanco1} Se $r=5\%$, $C_n=C_0\cdot 1,05^n$.
Logo, seu eu puser $1000$ hoje, daqui a $5$ anos terei
$C_5\simeq 1276$, e
para ter $2000$ daqui a $5$ anos, preciso por hoje $C_0\simeq 1814$.
Para por $1$ hoje e ter um milhão, preciso esperar
$n=\log_{1,05}(1000000/1)\simeq 283$ anos.
\eqref{itbanco2} Para ter um lucro de $600$ em $5$ anos, começando de $1000$,
preciso achar o $r$ tal que
$1000+600=1000(1+r/100)^5$. Isto é, $r=100\times
(10^{\frac{\log_{10}1,6}{5}}-1)\simeq 9,8\%$.
\end{Solution}
\begin{Solution}{3.11}
\eqref{itDobrafolha1}
Um pacote de $500$ folhas $A4$ para impressora tem uma espessura de
aproximadamente $5$ centímetros. Logo, uma folha tem uma espessura de
$E_0=5/500=0,01$ centrímetros. Como a espessura dobra a cada dobra, a espessura
depois de $n$ dobras é de $E_n=E_02^n$. Assim, $E_6=0,64$cm, $E_7=1.28$cm
\eqref{itDobrafolha1} a) Para ter $E_n=180$, são necessárias
$n=\log_{2}\frac{180}{0,01}\simeq 14$ dobras.
b) A distância média da terra à lua é de $D=384'403$km. Em centímetros:
$D=3,84403\times 10^{10}$cm. Assim, depois da $41$-ésima dobra, a distância
terra-lua já é ultrapassada.
Observe que depois desse tanto de dobras, o a largura do pacote de papel é
microscópica.
\end{Solution}
\begin{Solution}{3.12}
Se uma fonte é de $120$dB, a potência $P$ que ela produz se acha
isolando $P$ em $120=10\cdot \log_{10}(\tfrac{P}{P_{min}})$, o que dá
$P=10^{-2}W/m^2$.
Como duas fontes produzem o dobro da potência, isto é $2P$, o que representa
\[L=10\cdot
\log_{10}\Bigl(\frac{2P}{P_{min}}\Bigr)=120+\log_{10}2\simeq 120.3\text{dB}\]
\end{Solution}
\begin{Solution}{3.13}
Para ter $N_T=\tfrac{N_0}{2}$, significa que $e^{-\alpha T}=\tfrac12$. Isto é:
$T=\tfrac{\ln 2}{\lambda}$.
Depois de duas meia-vidas, $N_{2T}=N_0e^{-\lambda\tfrac{2 \ln
2}{\lambda}}=\frac{N_0}{4}$ ($>0$: logo, duas meia-vidas não são suficientes
para acabar com a substância!).
Para quatro, $N_{4T}=\frac{N_0}{16}$. Depois de $k$ meia-vidas,
$N_{kT}=\frac{N_0}{2^k}$:
depois de um número qualquer de meia-vidas, sempre sobre alguma coisa...
Para o uranio $235$, a meia-vida vale $T=\frac{\ln 2}{9.9\cdot 10^{-10}}$, isto
é aproximadamente: $700$ milhões de anos.
\end{Solution}
\begin{Solution}{3.14}
\eqref{iteqln0} $S=\{-e^2\}$
\eqref{iteqln1} $S=\{\pm 1\}$ Obs: aqui, se escrever $\ln(x^2)=2\ln x$, perde-se
a solução negativa! Lembre que $\ln (x^y)=y\ln x$ vale se $x$ é positivo! Então
aqui poderia escrever $\ln(x^2)=\ln (|x|^2)=2\ln |x|$.
\eqref{iteqln11} $S=\{e^{-\tfrac15}-1\}$
\eqref{iteqln2} $S=\varnothing$
\eqref{iteqln3} $S=...$
\eqref{iteqln4} $S=(-\infty,\tfrac34)$
\eqref{iteqln5} $S=(-\infty,-\tfrac13)\cup (-\tfrac18,\infty)$
\eqref{iteqln6} $S=(-\infty,-\tfrac23)\cup (\tfrac12,\infty)$
\eqref{iteqln7} $S=\{-5,-2,-1,2\}$
\eqref{iteqln8} $S=(0,e^{-1}]\cup [1,+\infty)$
\end{Solution}
\begin{Solution}{3.15}
\eqref{itparidadelog1} Nem par nem ímpar.
\eqref{itparidadelog2} Nem par nem ímpar (aqui, tem um problema de domínio: o
domínio do $\ln$ é $(0,\infty)$, então nem faz sentido verificar se
$\ln (-x)=\ln (x)$).
\eqref{itparidadelog3} Par: $e^{(-x)^2-(-x)^4}=e^{x^2-x^4}$.
\eqref{itparidadelog4} Par.
\eqref{itparidadelog5} Ímpar.
\eqref{itparidadelog6} Par (cuidado com o domínio: $\bR\setminus\{0\}$)
\eqref{itparidadelog7} Par.
\end{Solution}
\begin{Solution}{3.16}
Sabemos que o gráfico de $\frac{1}{(x-1)^2}$ é obtido transladando o de
$\frac{1}{x^2}$ de uma unidade para direita.
\begin{center}
\begin{bmlimage}\begin{tikzpicture}[scale=0.7]
\draw [ ->] (0,-1)--(0,2) node[left]{$y$};
\draw [ ->] (-3,0)--(4,0) node[right]{$x$};
\pgfmathsetmacro{\a}{3.5}
\pgfmathsetmacro{\e}{0.5}
\draw [thick, domain=-2:1-\e, samples=100] plot (\x,{1/((\x-1)^2)});
\draw [thick, domain=1+\e:4, samples=100] plot (\x,{1/((\x-1)^2)});
%\draw [thick, domain=0.5:\a, samples=100] plot (\x,{\x^{-2}});
%\draw[dotted] (-3,0)--(4,0);
\draw[dotted] (1,-1)--(1,3);
\fill (0,1) circle (0.55mm);
\fill (2,1) circle (0.55mm);
\end{tikzpicture}\end{bmlimage}
\end{center}
Ao tomar o logaritmo de $g(x)$, $f(x)=\ln(g(x))$, é bom ter o gráfico da função
$\ln x$ debaixo dos olhos.
Quando $x$ é grande (positivo ou negativo),
$g(x)$ é próximo de zero, logo $f(x)$ vai tomar valores grandes e negativos.
Quando $x$ cresce, $g(x)$ cresce até atingir o valor $1$ em $x=0$, logo $f(x)$
cresce até atingir o valor $0$ em $0$. Entre $x=0$ e $x=1$ ($x<1$), $g(x)$
diverge, logo $f(x)$ diverge também.
Entre $x=1$ ($x>1$) e $x=2$, $g(x)$ decresce até atingir o valor $1$ em $x=2$,
logo $f(x)$ decresce até atingir o valor $0$ em $x=2$.
Para $x>2$, $g(x)$ continua decrescendo, e toma valores que se aproximam de $0$,
logo $f(x)$ se toma valores negativos, e decresce para tomar valores
arbitrariamente grandes negativos.
\begin{center}
\begin{bmlimage}\begin{tikzpicture}[scale=0.7]
\draw [ ->] (0,-1)--(0,2) node[left]{$y$};
\draw [ ->] (-3,0)--(4,0) node[right]{$x$};
\pgfmathsetmacro{\a}{3.5}
\pgfmathsetmacro{\e}{0.5}
\draw [color=gray!60, domain=-2:1-\e, samples=100] plot (\x,{1/((\x-1)^2)});
\draw [color=gray!60, domain=1+\e:4, samples=100] plot (\x,{1/((\x-1)^2)});
\pgfmathsetmacro{\e}{0.2}
\draw [thick, domain=-1.8:1-\e, samples=100] plot (\x,{ln(1/((\x-1)^2))});
\draw [thick, domain=1+\e:3.8, samples=100] plot (\x,{ln(1/((\x-1)^2))});
\draw[dotted] (1,-1)--(1,3);
\fill (0,0) circle (0.55mm);
\fill (2,0) circle (0.55mm);
\end{tikzpicture}\end{bmlimage}
\end{center}
Observe que é também possível observar que $f(x)=-2\ln|x-1|$, e obter o seu
gráfico a partir do gráfico da função $\ln |x|$!

\end{Solution}
\begin{Solution}{3.17}
Lembramos que $y\in \bR$ pertence ao conjunto imagem de $f$ se e somente
se existe um $x$ (no domínio de $f$) tal que $f(x)=y$.
Ora $\frac{e^{x}}{e^{x}+1}=y$ implica $e^x=\frac{y}{1-y}$. Para ter uma
solução, é necessário ter $\frac{y}{1-y}>0$. É fácil ver que
$\frac{y}{1-y}>0$ se e somente se $y\in (0,1)$. Logo,
$\mathrm{Im}(f)=(0,1)$.
\end{Solution}
\begin{Solution}{3.18}
Por exemplo, $\senh (-x)=\frac{e^{(-x)}-e^{-(-x)}}{2}=\frac{e^{-x}-e^{x}}{2}
=-\frac{e^{x}-e^{-x}}{2}=-\senh (x).$
\end{Solution}
\protect \section *{Capítulo \ref {Cap:Limites}}
\begin{Solution}{4.1}
Em cada caso, fixemos uma tolerância $\epsilon>0$ e procuremos resolver uma
desigualdade elementar.
(1) Observe que $\frac{500}{x}>0$ para todo $x>0$. Seja $\epsilon>0$. Procuremos
quais são os $x>0$ grandes, positivos, para os quais
$0<\frac{500}{x}\leq \epsilon$.
Resolvendo a desigualdade achamos: $x\geq \frac{500}{\epsilon}$.
(2) Seja $\epsilon>0$. Procuremos resolver $0<\frac{9}{x^2}\leq \epsilon$, que dá $x\geq
\frac{3}{\sqrt{\epsilon}}$.
(3) Observe que $\frac{2}{3-x}<0$ quando $x$ for grande, positivo.
Fixemos $\epsilon>0$, e procuremos resolver
$-\epsilon\leq \frac{2}{3-x}<0$, e achamos $x\geq 3+\frac{2}{\epsilon}$.
\end{Solution}
\begin{Solution}{4.3}
%Primeiro, precisamos decidir qual deve ser o valor do limite. Podemos por exemplo
%observar os valores da função para alguns valores de $x$, grandes e
%positivos:
%\begin{center}
%\begin{tabular}{c|c|c|c|c}
%$x=$&10&100&1000&10'000\\
%\hline
%$\frac{2x-1}{3x+5}\simeq $&$0.5428$&$0.6524$&$0.6652$&$0.6665$
%\end{tabular}
%\end{center}
%Esses números parecem indicar que $\frac{2x-1}{3x+5}$ se
%aproxima de $0.6666\dots=\tfrac23$.
%Podemos também argumentar da seguinte maneira: na
%fração $\tfrac{2x-1}{3x+5}$, quando $x$ é grande,
%o numerador $2x-1$ e o denominador $3x+5$ são ambos grandes.
%No entanto, o ``$-1$'' no
%numerador se torna desprezível comparado com $2x$ (que é
%\emph{grande}!), logo $2x-1$ pode ser aproximado por $2x$. No denominador,
%o ``$5$'' é desprezível comparado com o ``$3x$'', logo
%$3x+5$ pode ser aproximado por $3x$. Portanto, para $x$ grande,
%$$
%\frac{2x-1}{3x+5}\quad\text{ pode ser aproximado por }
%\quad\frac{2x}{3x}=\frac23\,.
%$$
%Atenção: esse tipo de raciocínio ajuda a adivinhar qual deve ser o valor do
%limite (no caso
%$\tfrac23$) quando $x\to \infty$, mas não sempre funciona, e é ainda preciso
%\emph{mostrar} que o limite é $\tfrac23$ mesmo.

%Para tornar o argumento rigoroso, basta colocar $x$ em evidência no
%numerador e denominador, e \emph{simplificar por $x$}:
%$$
%\frac{2x-1}{3x+5}=\frac{x(2-\frac{1}{x})}{x(3+\frac{5}{x})}=
%\frac{2-\frac{1}{x}}{3+\frac{5}{x}}\,.
%$$
%Agora vemos que quando $x\to\infty$, o numerador dessa fração,
%$2-\frac{1}{x}$, tende a
%$2$ (pois já sabemos que $\frac{1}{x}$ tende a zero) e que o
%denominador, $3+\frac{5}{x}$, tende a $3$.
%Assim podemos escrever (as operações com limites serão justificadas mais
%tarde)
%$$\lim_{x\to\infty}f(x)=
%\lim_{x\to\infty}\frac{2-\frac{1}{x}}{3+\frac{5}{x}}
%=\frac{\lim_{x\to\infty}(2-\frac{1}{x})}{\lim_{x\to\infty}(3+\frac{5}{
%x})}
%=\frac{2-\lim_{x\to\infty}\frac{1}{x}}{3+5\lim_{x\to\infty}\frac{1}{x}
%}=\frac{2-0}{3+5\cdot 0}=\frac{2}{3}\,.
%$$
%
Vamos mostrar que
\begin{equation}\label{eq_ftendeadoistercos}
\lim_{x\to \infty}\frac{2x-1}{3x+5} =\frac23\,.
\end{equation}
Para isso fixemos uma tolerância $\epsilon>0$ (arbitrariamente pequena),
e verifiquemos que
\[
\Bigl|\frac{2x-1}{3x+5}-\frac23\Bigr|\leq \epsilon
\]
vale sempre que $x$ for tomado suficientemente grande.
Para começar, calculemos o valor absoluto da diferença:
\begin{equation}\label{eq_calculdifff}
\Bigl|\frac{2x-1}{3x+5}-\frac23\Bigr|=
\Bigl|\frac{-13}{3(3x+5)}\Bigr|=\frac{13}{3}
\frac{1}{3x+5}\,.
\end{equation}
Agora resolvemos a desigualdade (para $x$ grande, positivo)
\[ \frac{13}{3} \frac{1}{3x+5}
\leq \epsilon\,,
\]
e achamos a solução:
$x\geq 13\epsilon-15$. Assim, provamos
\eqref{eq_ftendeadoistercos}.
Deixamos o leitor tratar o limite
$x\to-\infty$.
Usando um computador, podemos verificar que de fato, os valores
de $\frac{2x-1}{3x+5}$, longe da origem, se aproximam
de $\tfrac23$:

\begin{center}
\begin{bmlimage}\begin{tikzpicture}
\pgfmathsetmacro{\a}{5.5}
\draw[dashed] (-\a,0.6666)--(\a,0.6666) node[above]{$y=\tfrac{2}{3}$};
\draw [thick, domain=-\a:-2.2, samples=100] plot
(\x,{(2*\x-1)/(3*\x+5)});
\draw [thick, domain=-0.8:\a, samples=100] plot
(\x,{(2*\x-1)/(3*\x+5)});
\draw [ ->] (-\a-0.3,0)--(\a+0.3,0) node[right]{$x$};
\draw [ ->] (0,-1)--(0,{3})
node[right]{$f(x)=\tfrac{2x-1}{3x+5}$};
\pgfmathsetmacro{\x}{4.5};
\draw[dotted] (\x,0) node[below]{$x$}
--(\x,{(2*\x-1)/(3*\x+5)})--(0,{(2*\x-1)/(3*\x+5)})
node[left]{$\scriptstyle{f(x)}$};
\fill (\x,{(2*\x-1)/(3*\x+5)}) circle (0.45mm);
\pgfmathsetmacro{\x}{-4.5};
\draw[dotted] (\x,0) node[below]{$x$}
--(\x,{(2*\x-1)/(3*\x+5)})--(0,{(2*\x-1)/(3*\x+5)})
node[right]{$\scriptstyle{f(x)}$};
\fill (\x,{(2*\x-1)/(3*\x+5)}) circle (0.45mm);
\end{tikzpicture}\end{bmlimage}
\end{center}
\end{Solution}
\begin{Solution}{4.4}
\eqref{itexoformallim2}
Vamos mostrar que o limite é $\tfrac15$.
Calculemos então
$\bigl|\frac{x^2-1}{5x^2}-\tfrac15\bigr|=\frac{1}{5x^2}$.
Seja $\epsilon>0$. Para ter $\tfrac{1}{5x^2}\leq \epsilon$, podemos tomar
$x\geq N$, onde $N=\frac{1}{\sqrt{5\epsilon}}$.
Logo, como isso pode ser feito com qualquer $\epsilon>0$, isso mostra que
$\lim_{x\to \pm\infty}\frac{x^2-1}{5x^2}=\tfrac15$.
\eqref{itexoformallim4}
Como a função é \emph{constante e igual a $1$} nos positivos, temos
$\lim_{x\to\infty}f(x)=1$. Observe aqui que para qualquer tolerância $\epsilon>0$,
podemos sempre tomar o mesmo $N$, por exemplo $N=0$. De fato, para todo $x\geq 0$,
$|f(x)-1|=0\leq \epsilon$, qualquer que seja a tolerância.
Esse exemplo mostra que uma função pode coincidir com a sua assíntota.
\eqref{itexoformallim3}
Como a função é a divisão de $1$ por um número grande, o limite deve ser zero.
De fato, seja $\epsilon>0$. Precisamos mostrar que
\[ \Bigl|\frac{1}{x^3+\sen^2 x}\Bigr|\leq \epsilon
\]
para todo $x$ suficientemente grande. Mas como não dá para resolver essa desigualdade
(isto é: isolar o $x$), podemos começar observando que
$\bigl|\frac{1}{x^3+\sen^2 x}\bigr|\leq	\frac{1}{x^3}$, e procurar
resolver $\frac{1}{x^3}\leq \epsilon$.
Vemos que se $x\geq N\pardef \epsilon^{-1/3}$, então essa desigualdade será
verificada, e $\bigl|\frac{1}{x^3+\sen^2 x}\bigr|\leq \epsilon$.
Isso mostra que $\lim_{x\to \infty}\frac{1}{x^3+\sen^2 x}=0$.
\end{Solution}
\begin{Solution}{4.5}
Seja $\epsilon>0$. Queremos mostrar que $|\frac{1}{f(x)}|\leq \epsilon$ para todo
$x$ suficientemente grande. Como $\lim_{x\to\infty}f(x)=\pm\infty$, sabemos que
se $A=\tfrac1\epsilon$, então existe $N$ tal que $f(x)\geq A$ para todo $x\geq
N$ (em particular, $f(x)>0$ para esses $x$). Mas isso implica também
$\frac{1}{f(x)}\leq \frac{1}{A}=\epsilon$, o que queríamos.
\end{Solution}
\begin{Solution}{4.7}
\eqref{itexliminfini2} Como $\lim_{x\to\pm\infty}\frac{1}{x^q}=0$ para
qualquer $q>0$, usando \eqref{eq:proprliminfty1} dá
$\lim_{x\to
\pm\infty}\{\frac{1}{x}+\frac{1}{x^2}+\frac{1}{x^3}\}=0$.
\eqref{itexliminfini3} $\lim_{x\to \pm\infty}\frac{x^2-1}{x^2}=1$
\eqref{itexliminfini6} $\lim_{x\to\pm\infty}\frac{1-x^2}{x^2-1}=-1$.
\eqref{itexliminfini7} Colocando $x^3$ em evidência e usando
\eqref{eq:proprliminfty3},
$$\lim_{x\to\pm\infty}\frac{2x^3+x^2+1}{x^3+x}=\lim_{x\to
\pm\infty}\frac{x^3(2+\frac{1}{x}+\frac{1}{x^3})}{x^3(1+\frac{1}{x^2})
} =\lim_{x\to
\pm\infty}\frac{2+\frac{1}{x}+\frac{1}{x^3}}{1+\frac{1}{x^2}}=\frac{2}
{1}=2\,.$$
\eqref{itexliminfini8} $\lim_{x\to
\pm\infty}\frac{2x^3-2}{x^4+x}=0$
\eqref{itexliminfini9} Colocando $x^4$ em
evidência no denominador, $x^2$ no numerador,
$\frac{1+x^4}{x^2+4}=x^2\frac{\frac{1}{x^4}+1}{1+\frac{4}{x^2}}$.
Como $x^2\to\infty$ e que a fração tende a $1$, temos
$\lim_{x\to\pm\infty}\frac{1+x^4}{x^2+4}=\infty$.
\eqref{itexliminfini10} ``$\lim_{x\to
-\infty}\frac{\sqrt{x+1}}{\sqrt{x}}$'' não é definido.
Por outro lado, colocando $\sqrt{x}$ em evidência,
$$\lim_{x\to+\infty}\frac{\sqrt{x+1}}{\sqrt{x}}=
\lim_{x\to +\infty}\frac{\sqrt{1+\frac{1}{{x}}}}{1}=1\,.
$$
\eqref{itexliminfini15}
Lembrando que $\sqrt{x^2}=|x|$ (Exercício
\ref{Exo:valorabscorreto}!), temos
$\frac{\sqrt{4x^2+1}}{x}=\frac{\sqrt{x^2(4+\frac{1}{x^2})}}{x}=\frac{
|x|}{x}\sqrt{4+\frac{1}{x^2}}$.
Como $\frac{|x|}{x}=+1$ se $x>0$, $=-1$ se $x<0$, temos $\lim_{x\to\pm\infty}\frac{|x|}{x}=\pm 1$. Como
$\lim_{x\to\pm\infty}\sqrt{4+\frac{1}{x^2}}=\sqrt{4}=2$, temos
$\lim_{x\to\pm\infty}\frac{\sqrt{4x^2+1}}{x}=\pm 2$.
\eqref{itexliminfini11}
Do mesmo jeito,
$\sqrt{x^2+3}=|x|\sqrt{1+\frac{3}{x^2}}$. Assim,
$$
\frac{3x+2}{\sqrt{x^2+3}-4}=\frac{x}{|x|}\frac{3+\frac{2}{x}}{\sqrt{1+
\frac{3}{x^2}}-\frac{4}{|x|}}
$$
Como $\lim_{x\to\pm\infty}\frac{x}{|x|}=\pm 1$, e que a razão tende a
$3$, temos
$$\lim_{x\to+\infty}\frac{3x+2}{\sqrt{x^2+3}-4}=+3\,,\quad
\lim_{x\to-\infty}\frac{3x+2}{\sqrt{x^2+3}-4}=-3\,.
$$
\eqref{itexliminfini12} O limite $x\to-\infty$
não é definido, e $\lim_{x\to+\infty}
\frac{\sqrt{x+\sqrt{x+\sqrt{x}}}}{\sqrt{x+1}}=1$.
\eqref{itexliminfini13} $\lim_{x\to\pm\infty}\frac{|x|}{x^2+1}=0$
\eqref{itexliminfini14} $\lim_{x\to\pm\infty}\sqrt{x^2+1}=+\infty$
\eqref{itexliminfini141} Como $\frac{1}{2^x}=2^{-x}$, temos $\lim_{x\to+\infty}\frac{1}{2^x}=0$,
$\lim_{x\to-\infty}\frac{1}{2^x}=+\infty$.
\eqref{itexliminfini16}
$\lim_{x\to+\infty}\frac{e^x+100}{e^{-x}-1}=-\infty$,
$\lim_{x\to-\infty}\frac{e^x+100}{e^{-x}-1}=0$.
\eqref{itexliminfini161}  Primeiro mostre (usando os mesmos métodos do que os que foram
usados nos outros itens) que $\lim_{x\to \pm \infty}(1+\frac{x+1}{x^2})=1$. Em seguida,
observe que
se $z$ se aproxima de $1$,  então $\ln(z)$ se aproxima de $\ln(1)=0$. Logo, $\lim_{x\to
\pm \infty}\ln(1+\frac{x+1}{x^2})=0$. Obs: dizer que ``se $z$ se aproxima de $1$, então
$\ln(z)$ se aproxima de $\ln(1)$'' presupõe que a função $\ln$ é \emph{contínua} em $1$.
Continuidade será estudada no fim do capítulo.
\eqref{itexliminfini162}  Escreve $(1+e^x)=e^x(1+e^{-x})$, logo
$\frac{\ln(1+e^x)}{x}=\frac{\ln
e^x}{x}+\frac{\ln(1+e^{-x})}{x}=1+\frac{\ln(1+e^{-x})}{x}$. Mas $\lim_{x\to
\infty}\frac{\ln(1+e^{-x})}{x}=0$, logo
$\lim_{x\to \infty}\frac{\ln(1+e^{x})}{x}=1$.
Por outro lado, $\ln(1+e^x)\to 0$  quando $x\to-\infty$, logo $\lim_{x\to
-\infty}\frac{\ln(1+e^{x})}{x}=0$.
\eqref{itexliminfini5} Como $\lim_{x\to\pm \infty}\frac{1}{x}=0$
temos, $\lim_{x\to\pm \infty}e^{\frac{1}{x}}=e^0=1$.
\eqref{itexliminfini17} $``\lim_{x\to \pm\infty}\sen^2x''$ não existe.
\eqref{itexliminfini19} $\lim_{x\to \pm\infty}\arctan
x=\pm\pisobredois$.
\eqref{itexliminfini20}
Por definição, $\senh x=\frac{e^x-e^{-x}}{2}$. Para estudar
$x\to\infty$, coloquemos $e^x$ em evidência:
$\frac{e^x-e^{-x}}{2}=e^x\frac{1-e^{-2x}}{2}$. Como $e^x\to\infty$ e
$1-e^{-2x}\to 1$ temos $\lim_{x\to \infty}\senh x=+\infty$. Como
$\senh x$ é ímpar, temos $\lim_{x\to -\infty}\senh x=-\infty$.
\eqref{itexliminfini21} $\lim_{x\to \pm\infty}\cosh x=+\infty$
\eqref{itexliminfini22} Para estudar, $x\to\infty$:
$\tanh x=\frac{e^x-e^{-x}}{e^x+e^{-x}}=\frac{e^x}{e^x}\frac{1-e^{-2x}}{1+e^{
-2x }}=\frac{1-e^{-2x}}{1+e^{
-2x }}$, logo $\lim_{x\to +\infty}\tanh x=+1$. Como $\tanh$ é ímpar,
$\lim_{x\to -\infty}\tanh x=-1$.
\end{Solution}
\begin{Solution}{4.8}
Pelo gráfico de $x\mapsto \tanh x$, vemos que $V(t)$ cresce e tende a
um valor limite, dado por
$$
V_{\rm lim}=\lim_{t\to\infty}V(t)=\sqrt{\frac{m
g}{k}}\lim_{t\to\infty}\tanh\Bigl(\sqrt{\frac{gk}{m}}t\Bigr)
$$
Vimos no Exercício \ref{Exo:limitesinfini} que
$\lim_{x\to\infty}\tanh x=1$. Portanto,
$$V_{\rm lim}=\sqrt{\frac{m
g}{k}}\,.$$
Observe que $V(t)<V_{\rm lim}$ para todo $t$, então o paraquedista
nunca atinge a velocidade limite, mesmo se ele cair um tempo infinito!
Com os valores propostos,
$V_{\rm lim}=\sqrt{80\cdot 9,81/0.1}\simeq 89m/s\simeq 318km/h$.
\end{Solution}
\begin{Solution}{4.9}
\eqref{itexliminfini1} $\lim_{x\to\infty}(7-x)=-\infty$,
$\lim_{x\to-\infty}(7-x)=+\infty$.
\eqref{itexliminfini4} ``$\lim_{x\to +\infty}\sqrt{1-x}$'' não é
definida, pois o domínio de $\sqrt{1-x}$ é $(-\infty,1]$.
$\lim_{x\to -\infty}\sqrt{1-x}=+\infty$.
\eqref{itexliminfini18} Como $\lim_{x\to\pm\infty}
x=\pm\infty$, e que $\cos x$ é limitado por $-1\leq \cos x\leq 1$,
temos $\lim_{x\to \pm\infty}x+\cos x=\pm\infty$.
\eqref{itexoinfinf1} $-\infty$.
\eqref{itexoinfinf2} $0$.
\eqref{itexoinfinf3} $+\infty$.
\eqref{itexoinfinf311} $-\infty$
\eqref{itexoinfinf31} $+\infty$
\eqref{itexoinfinf5} $\frac12$.
Esse ítem (e o próximo) mostram que argumentos informais do tipo
``$x^2+1\simeq x^2$
quando $x$ é grande'' não sempre são eficazes! De fato, aqui daria
$\sqrt{x^2+1}-\sqrt{x^2-x}\simeq \sqrt{x^2}-\sqrt{x^2}=0$...
\eqref{itexoinfinf51} $\frac32$.
\eqref{itexoinfinf4} Aqui não precisa multiplicar pelo conjugado: pode
simplesmente colocar $\sqrt{x}$ em evidência:
$\sqrt{2x}-\sqrt{x+1}=\sqrt{x}(\sqrt{2}-\sqrt{1+\frac1x})$. Como
$\sqrt{x}\to\infty$ e $\sqrt{2}-\sqrt{1+\frac1x}\to \sqrt{2}-1>0$,
temos $\sqrt{x}(\sqrt{2}-\sqrt{1+\frac1x})\to +\infty$.
\eqref{itexoinfinf6} $-\infty$ (Obs: pode observar que $e^x-e^{2x}=z-z^2$, em que $z=e^x$. Como $z\to \infty$
quando $x\to\infty$, temos $z-z^2\to \infty$, como no item \eqref{itexoinfinf1}.)
\eqref{itexoinfinf7} Como $\ln x-\ln(2x)=-\ln 2$, o limite é $-\ln 2$.
\eqref{itexoinfinf8} $\lim_{x\to \infty}\{\ln x-\ln(x+1)\}=
\lim_{x\to \infty}\ln (\frac{x}{x+1})=\ln 1=0$.
\end{Solution}
\begin{Solution}{4.10}
\eqref{itexosanduiche2} Para todo $x$, $-1\leq
\cos(x^2+3x)\leq +1$, logo
$0\leq \frac{1+\cos(x^2+3x)}{x^2}\leq \frac{2}{x^2}$.
Como $\frac{2}{x^2}$ tende a zero,
$\lim_{x\to\infty}\frac{1+\cos(x^2+3x)}{x^2}=0$.
\eqref{itexosanduiche1} Como $\frac{x+\sen x}{x-\cos
x}=\frac{1+\frac{\sen x}{x}}{1-\frac{\cos x}{x}}$, e como
$\lim_{x\to\infty}\frac{\sen x}{x}=0$, $\lim_{x\to\infty}\frac{\cos
x}{x}=0$ (mesmo método), temos que $\lim_{x\to\infty}\frac{x+\sen
x}{x-\cos x}=1$.
\eqref{itexosanduiche3}
Como $-e^{-x}\leq e^{-x}\sen x\leq e^{-x}$ e
$\lim_{x\to\infty}-e^{-x}=\lim_{x\to\infty}e^{-x}=0$, o limite
procurado vale $0$.
\eqref{itexosanduiche4} Como $0\leq x-\lfloor x\rfloor\leq 1$, temos
$\lim_{x\to\infty}\frac{x-\lfloor x\rfloor}{x}=0$.
\eqref{itexosanduiche5} Como
$-\frac{\pi/2}{\ln x}\leq \frac{\arctan(\sen x)}{\ln x}\leq
\frac{\pi/2}{\ln x}$, e $\lim_{x\to\infty}\frac{1}{\ln x}=0$, o
limite procurado é $0$.
\eqref{itexosanduiche6} Para todo $x$,
$-\frac{1}{x^2+4}\leq \frac{\sen x}{x^2+4}\leq
\frac{1}{x^2+4}$. Como
$\lim_{x\to \infty}\frac{1}{x^2+4}=0$, o limite procurado vale
$1$.
\end{Solution}
\begin{Solution}{4.11}
 A divisão dá $\frac{x^4-1}{x-1}=x^3+x^2+x+1$. Logo, como cada termo tende a $1$,
$\lim_{x\to 1^{\pm}}\frac{x^4-1}{x-1}=4$.
 No caso geral, $\frac{x^n-1}{x-1}=x^{n-1}+\dots+x+1$. Como são $n$ termos e que cada um
tende a $1$, temos $\lim_{x\to 1^{\pm}}\frac{x^n-1}{x-1}=n$.
\end{Solution}
\begin{Solution}{4.12}
$\lim_{x\to 0^+}f(x)=\lim_{x\to 0^+}\frac{x}{2}=0$,
$\lim_{x\to 0^-}f(x)=\lim_{x\to 0^-}\frac{x}{2}=0$.
$\lim_{x\to 2^+}f(x)=\lim_{x\to 2^+}5-x=3$.
$\lim_{x\to 2^-}f(x)=\lim_{x\to 2^-}\frac{x}{2}=1$, logo $f$ é descontínua em
$x=2$.
$\lim_{x\to 5^+}f(x)=\lim_{x\to 5^+}5-x=0$,
$\lim_{x\to 5^-}f(x)=\lim_{x\to 5^-}5-x=0$.
\end{Solution}
\begin{Solution}{4.13}
Escolha um ponto $a\in \bR$ qualquer.
Como os racionais diádicos \index{racionais diádicos}
são densos em $\bR$, existem infinitos diádicos $x_D>a$,
 arbitrariamente próximos de $a$, tais que $f(x_D)=1$. Mas existem também infinitos
irracionais $x_I>a$ arbitrariamente próximos de $a$ tais que $f(x_I)=0$. Portanto, $f(x)$
não pode tender a um valor quando $x\to a^+$. O mesmo raciocínio vale para $x\to a^-$.
Logo, a função $f$ não possui limites laterais em nenhum ponto da reta.
\end{Solution}
\begin{Solution}{4.14}
$\lim_{x\to \half^+}f(x)=\lim_{x\to \half^-}f(x)=0$,
$\lim_{x\to \frac{1}{3}^+}f(x)=\lim_{x\to \frac{1}{3}^-}f(x)=0$.
$\lim_{x\to 1^+}f(x)=1$, $\lim_{x\to 1^-}f(x)=0$. Para
$n\in \bZ$, $\lim_{x\to n^+}f(x)=n$, $\lim_{x\to n^-}f(x)=n-1$.
(Pode verificar essas afirmações também no seu esboço do
Exercício \ref{ExoEsbocosElementares}!)
\end{Solution}
\begin{Solution}{4.15}
\eqref{itlimbasic1} $0$
\eqref{itlimbasic2} $0$ (O limite é bem definido, no seguinte
sentido: como $\sqrt{x}$ é definida para $x>0$, o limite
somente pode ser do tipo $x\to 0^+$.)
\eqref{itlimbasic3} $1$
\eqref{itlimbasic4} $\frac45$
\eqref{itlimbasic5} $1$
\eqref{itlimbasic6} Sabemos que $\frac{|x-4|}{x-4}=+1$ se $x>4$, e
$=-1$ se $x<4$. Logo, $\lim_{x\to 4^+}\frac{|x-4|}{x-4}=+1$,
$\lim_{x\to 4^-}\frac{|x-4|}{x-4}=-1$, mas $\lim_{x\to
4}\frac{|x-4|}{x-4}$ não existe.
\eqref{itlimbasic61} $-1$
\eqref{itlimbasic7} $-\frac12$
\eqref{itexolimelem20} Como $\ln x$ muda de sinal em $1$, é preciso
que $x$ tenda a $1$ pela direita para $\sqrt{\ln x}$ ser bem definida,
e escrever esse limite como $\lim_{x\to 1^+}\sqrt{\ln x}=0$.
$\lim_{x\to 1^-}\sqrt{\ln x}$ não é definido.
\eqref{itexolimelem201} Não definido pois $\sqrt{x-2}$ não é definido perto de $x=-2$.
\end{Solution}
\begin{Solution}{4.16}
No primeiro caso, podemos comparar $0\leq f(x)\leq x^2$ para todo $x$.
Logo,
pelo Teorema \ref{Teo:Sanduichefinito},
$\lim_{x\to 0}f(x)$ existe e vale $0$.
No segundo caso,
$\lim_{x\to 0^-}g(x)=\lim_{x\to 0^-}\frac{1+x}{1+x^2}=1$, e
$\lim_{x\to 0^+}g(x)=\lim_{x\to 0^+}\sen(\frac{\pi}{2}+x)=\sen
\pisobredois=1$. Logo, $\lim_{x\to 0}g(x)$ existe e vale $1$.
\end{Solution}
\begin{Solution}{4.17}
\eqref{itlimzerozero1} $-4$.
\eqref{itlimzerozero2} $6$.
\eqref{itlimzerozero24} $-\tfrac12$.
\eqref{itlimzerozero4} $\frac{b}{2a}$.
\eqref{itlimzerozero44} $0$.
\eqref{itlimzerozero3} $\tfrac12$.
\end{Solution}
\begin{Solution}{4.18}
Observe que quando $x\to -2$, o denominador tende a $0$.
Para o limite existir, a única possibilidade é do numerador também
tender a zero quando $x\to -2$. Mas como $3x^2+ax+a+3$ tende a $15-a$
quando $x\to -2$, $a$ precisa satisfazer $15-a=0$, isto é: $a=15$.
Neste caso (e somente neste caso), o limite existe e vale
$$
\lim_{x\to -2}\frac{3x^2+15x+18}{x^2+x-2}
\lim_{x\to -2}\frac{(3x+9)(x+2)}{(x-1)(x+2)}=
\lim_{x\to -2}\frac{3x+9}{x-1}=-1\,.
$$
\end{Solution}
\begin{Solution}{4.19}
 \eqref{itexosinxx1}
Como $\frac{\tan x}{x}=\frac{\sen x}{x}\frac{1}{\cos x}$,
temos $\lim_{x\to 0}\tfrac{\tan x}{x}=1$.
\eqref{itexosinxx2}
Como $\frac{\sen x}{\tan x}=\cos x$, temos $\lim_{x\to 0}\tfrac{\sen
x}{\tan x}=1$.
\eqref{itexosinxx3} Como ${\sen 2x}\to 0$ e ${\cos x}\to 1$, temos $\lim_{x\to 0}\frac{\sen 2x}{\cos x}=\frac{0}{1}=0$ (não é um limite do tipo ``$\frac00$'').
\eqref{itexosinxx4}
Como $\frac{\sen 2x}{x\cos x}=2\frac{\sen x}{x}$,
temos $\lim_{x\to 0}\frac{\sen 2x}{x\cos x}=2$.
\eqref{itexosinxx5} Como
$$\frac{1-\cos x}{x^2}=\frac{1-\cos x}{x^2}\frac{1+\cos
x}{1+\cos x}=\frac{1-\cos^2x}{x^2}\cdot \frac{1}{1+\cos x}=\Bigl(\frac{\sen x}{x}\Bigr)^2\cdot \frac{1}{1+\cos x}\,,$$
temos
$\lim_{x\to 0}\tfrac{1-\cos x}{x^2}=(1)^2\cdot \frac12=\frac12$.
\eqref{itexosinxx6} $+\infty$
\eqref{itexosinxx7}  $\lim_{x\to 0^+}\tfrac{\sen (x^2)}{x}=\lim_{x\to
0^+}x\cdot\tfrac{\sen(x^2)}{x^2}=0\cdot 1=0$.
\end{Solution}
\begin{Solution}{4.20}
``$\lim_{x\to a^+}f(x)=+\infty$'' significa que $f(x)$ ultrapassa
qualquer valor dado (arbitrariamente grande), desde que $x>a$ esteja
suficientemente perto de $a$. Isto é: para todo $M>0$ (arbitrariamente
grande), existe um $\delta>0$ tal que se $a<x\leq a+\delta$, então
$f(x)\geq M$.
Por outro lado, $\lim_{x\to a^+}f(x)=-\infty$ significa que
para todo $M>0$ (arbitrariamente grande),
existe um $\delta>0$ tal que se $a<x\leq a+\delta$, então $f(x)\leq
-M$.
\end{Solution}
\begin{Solution}{4.21}
\eqref{itlimbasic9a} $5$
\eqref{itlimbasic9b} $1$
\eqref{itlimbasic9c} $\mp \infty$
\eqref{itlimbasic10} Observe que enquanto $x^2-4>0$, $\frac{x-2}{(
\sqrt{x^2-4})^2}=\frac{1}{x+2}$. Logo, $\lim_{x\to
2^+}\frac{x-2}{(\sqrt{x^2-4})^2}=\frac14$, e
\eqref{itlimbasic12} $\lim_{x\to -2^-}\frac{x-2}{(\sqrt{x^2-4})^2}=
-\infty$
\eqref{itlimbasic14} $-\infty$
\eqref{itlimbasic14b} Não é definido.
\eqref{itlimbasic15a}   $\lim_{t\to 0^+}\frac{1}{\sen t}=+\infty$,
$\lim_{t\to 0^-}\frac{1}{\sen t}=-\infty$
\eqref{itlimbasic15b}   $\lim_{t\to 0^\pm}\frac{t}{\sen t}=\lim_{t\to
0^\pm}\frac{1}{\frac{\sen t}{t}}=1$.
\eqref{itlimbasic15} Não existe, porqué quando $t\to 0^+$, $\sen
\frac1t$
oscila entre $+1$ e $-1$, enquanto $\frac1t$ tende a $+\infty$:
\begin{center}
\begin{bmlimage}\begin{tikzpicture}[scale=0.3]
\draw[->] (0,0)--(3,0) node[right]{$t$};
\draw[->] (0,-3)--(0,3) node[left]{$\frac{\sen \frac1t}{t}$};
\pgfmathsetmacro{\e}{0.2};
\draw[thick, domain=\e:{3}, samples=1000] plot (\x,{(sin(5/\x r))/\x});
\end{tikzpicture}\end{bmlimage}
\end{center}
\eqref{itlimbasic16} $\li{z}{0^+}9^{\frac{1}{z}}=+\infty$,
$\li{z}{0^-}9^{\frac{1}{z}}=0$.
\eqref{itlimbasic18b} $+\infty$
\eqref{itlimbasic19b} $-\infty$
\eqref{itlimbasic13} $1$ (veremos mais tarde como calcular esse
limite...)
\end{Solution}
\begin{Solution}{4.22}
A função $v\mapsto m_v$ tem domínio $[0,c)$, e a reta $v=c$ é
assíntota vertical:
\begin{center}
\begin{bmlimage}\begin{tikzpicture}
\draw[->] (0,0)--(4,0) node[right]{$v$};
\draw[->] (0,0)--(0,3) node[left]{$m_v$};
\pgfmathsetmacro{\c}{3.2};
\pgfmathsetmacro{\m}{1};
\fill (0,\m) circle (0.45mm);
\draw (0,\m) node[left]{$m_0$};
\draw[thick, domain=0:{\c-0.2}] plot (\x,{\m/(sqrt(1-(\x/\c)^2))});
\draw[dotted] (\c,0) node[below]{$c$}--(\c,3);
\draw (3.8,2.5) node[right]{$\displaystyle{\lim_{v\to c^-}m_v}=
+\infty$};
\end{tikzpicture}\end{bmlimage}
\end{center}
\end{Solution}
\begin{Solution}{4.23}
Observe que $\lim_{x\to \pm\infty}f(x)=+1$, logo $y=1$ é assíntota
horizontal.
Por outro lado, $\lim_{x\to 1^+}f(x)=+\infty$ e $\lim_{x\to 1^-}f(x)
=-\infty$. Portanto, $x=1$ é assíntota vertical.
Temos então: 1) o gráfico se aproxima da sua assintota horizontal em
$-\infty$, e ele tende a $-\infty$ quando $x\to 1^-$,
2) o gráfico se aproxima da sua assintota horizontal em $+\infty$, e
ele tende a $+\infty$ quando $x\to 1^+$.
Somente com essas informações, um esboço razoável pode ser montado:
\begin{center}
\begin{bmlimage}\begin{tikzpicture}[scale=0.5]
\draw[->] (-3,0)--(3,0) node[right]{$x$};
\draw[->] (0,-2)--(0,3) node[left]{$y$};
\pgfmathsetmacro{\e}{0.6};
%\pgfmathsetmacro{\m}{1};
%\fill (0,\m) circle (0.45mm);
%\draw (0,\m) node[left]{$m_0$};
\draw[thick, domain=-4:1-\e] plot (\x,{(\x+1)/(\x-1)});
\draw[thick, domain=1+\e:4] plot (\x,{(\x+1)/(\x-1)});
\draw[dotted] (1,-2) node[right]{$\scriptstyle{x=1}$}--(1,3);
\draw[dotted] (-4,1) node[above]{$\scriptstyle{y=1}$}--(4,1);
%\draw (3.8,2.5) node[right]{$\displaystyle{\lim_{v\to c^-}m_v}=+\infty$};
\end{tikzpicture}\end{bmlimage}
\end{center}
Observe que pode também escrever $\frac{x+1}{x-1}=\frac{2}{x-1}+1$,
logo o gráfico pode ser obtido a partir de transformações elementares
do gráfico de $\frac1x$...
\end{Solution}
\begin{Solution}{4.24}
\eqref{itexassint1} $D=\bR$, sem assíntotas.
\eqref{itexassint2} $D=\bR\setminus\{-1\}$. Horiz: $y=0$, Vertic: $x=-1$.
\eqref{itexassint3} $D=\bR\setminus\{3\}$. sem assíntotas.
\eqref{itexassint4} $D=\bR\setminus\{0\}$. Horiz: $y=2$, Vertic: $x=0$.
\eqref{itexassint5} $D=\bR\setminus\{-3\}$. Horiz: $y=-1$, Vertic: $x=-3$.
\eqref{itexassint6} $D=\bR\setminus\{0\}$. Horiz: $y=1$, Vertic: não tem.
\eqref{itexassint6b} $D=(-\infty,2)$. Horiz: não tem, Vertic: $x=2$.
\eqref{itexassint7} $D=\bR\setminus\{0\}$. Horiz: não tem, Vertic: $x=0$.
\eqref{itexassint8} $D=\bR\setminus\{0\}$. Horiz: $y=0$, Vertic: não tem.
\eqref{itexassint9} $D=\bR\setminus\{0\}$. Horiz: $y=0$, Vertic: $x=0$.
\eqref{itexassint10} $D=\bR$. Horiz: $y=1$, Vertic: não tem.
\eqref{itexassint11} Para garantir $1-x^2>0$, $D=(-1,1)$ Horiz: não
tem (já que o domínio é $(-1,1)$...), Vertic: $x=-1$ (porqué
$\lim_{x\to -1^+}\ln (1-x^2)=-\infty$), $x=+1$ (porqué $\lim_{x\to
+1^-}\ln (1-x^2)=-\infty$).
\eqref{itexassint12} $D=(-1,1)$. Horiz: não tem, Vertic: $x=-1$, $x=+1$.
\eqref{itexassint13} $D=\bR\setminus\{\pm 1, 3\}$. Horiz: $y=0$,
Vertic: $x=+1$, $x=-1$.
\eqref{itexassint14} $D=(-1,+1)\setminus\{ 0\}$. Horiz: não tem,
Vertic: $x=0$.
\eqref{itexassint17} $D=\bR\setminus\{0\}$. Horiz: $y=+1$, $y=-1$,
Vertic: $x=0$.
\eqref{itexassint15b} $D=(-1,1)$. Horiz: não tem, Vertic: $x=-1$, $x=+1$.
\eqref{itexassint18} $D=\bR\setminus\{0\}$. Horiz: $y=1$ (a direita), $y=0$ (a esquerda),
Vertic: $x=0$.
\end{Solution}
\begin{Solution}{4.26}
Por exemplo: $f(x)=\frac{1-x^2}{(x+1)(x-3)}$, ou $f(x)=\frac{1}{x+1}
+\frac{1}{x-3}-\frac{x^2}{x^2+1}$.
\end{Solution}
\begin{Solution}{4.27}
\eqref{itmudvarlim1} Com $z\pardef x-1$, $\lim_{x\to 1}\frac{\sen
(x-1)}{3x-3}=\lim_{z\to 0}\frac{\sen
z}{3z}=\frac13$.
\eqref{itmudvarlim11} $\frac35$ (Escreve $\frac{\sen (3x)}{\sen (5x)}=\frac{\sen (3x)}{3x}\frac{1}{\frac{\sen (5x)}{5x}}\frac{3x}{5x}$.)
\eqref{itmudvarlim2} Com $z\pardef x+1$, $\lim_{x\to
-1}\frac{\sen(x+1)}{1-x^2}=\lim_{z\to
0}\frac{\sen z}{z}\frac{1}{2-z}=\frac12$.
\eqref{itmudvarlim3} Com $h\pardef x-a$,
$\lim_{x\to a}\frac{x^n-a^n}{x-a}=\lim_{h\to
0}\frac{(a+h)^n-a^n}{h}=na^{n-1}$.
\eqref{itmudvarlim31} Chamando $t\pardef \sqrt{x}$,
$$\lim_{x\to 4}\frac{x-4}{x-\sqrt{x}-2}=\lim_{t\to 2}\frac{t^2-4}{t^2-t-2}=\lim_{t\to 2}\frac{(t-2)(t+2)}{(t-2)(t+1)}=\lim_{t\to 2}\frac{(t+2)}{(t+1)}=\tfrac43\,.$$
\eqref{itmudvarlim4} Com $z\pardef \frac{1}{x}$, temos (lembre o
item \eqref{itexliminfini22} do Exercício \ref{Exo:limitesinfini})
 $\lim_{x\to 0^+}\tanh \frac{1}{x}=\lim_{z\to +\infty}\tanh z=+1$,
$\lim_{x\to 0^-}\tanh \frac{1}{x}=\lim_{z\to -\infty}\tanh z=-1$.
\eqref{itmudvarlim5} Com a mesma mudança,
$\lim_{x\to 0^\pm}x\tanh \frac{1}{x}=\lim_{z\to \pm
\infty}\frac{1}{z}\tanh{z}=0\cdot (\pm 1)=0$.
\end{Solution}
\begin{Solution}{4.28}
Pela fórmula \eqref{eq:mudancabaselog} de mudança de base para o
logaritmo, $\log_a(1+h)=\frac{\ln(1+h)}{\ln a}$. Logo, por
\eqref{eq:derivlnenun},
$$\lim_{h\to 0}\frac{\log_a(1+h)}{h}=\frac{1}{\ln a}\lim_{h\to
0}\frac{\ln(1+h)}{h}=\frac{1}{\ln a}\,.$$
Por outro lado, chamando $z\pardef a^x$, $x\to 0$ implica $z\to 1$.
Mas $x=\log_az$, logo
$$
\lim_{x\to
0}\frac{a^x-1}{x}=\lim_{z\to 1}\frac{z-1}{\log_a
z}=\frac{1}{\lim_{z\to 1}\frac{\log_az}{z-1}}\,.
$$
Definindo $h\pardef z-1$ obtemos $\lim_{z\to
1}\frac{\log_az}{z-1}=\lim_{h\to 0}\frac{\log_a(1+h)}{h}=\frac{1}{\ln
a}$, o que prova a identidade desejada.
\end{Solution}
\begin{Solution}{4.29}
\eqref{it_exsuplAldo_1} $\infty$
\eqref{it_exsuplAldo_2} $\infty$
\eqref{it_exsuplAldo_3} $0$
\eqref{it_exsuplAldo_4} $\infty$
\eqref{it_exsuplAldo_5} $0$
\eqref{it_exsuplAldo_6} $1$
\end{Solution}
\begin{Solution}{4.30}
$\lim_{x\to 0^-}f(x)=\lim_{x\to 0^-}(2x+2)=2$,
$\lim_{x\to 0^+}f(x)=\lim_{x\to 0^+}(x^2-2)=-2$,
Já que esses dois limites laterais são diferentes, $\lim_{x\to 0}f(x)$ não
existe.
$\lim_{x\to 2^-}f(x)=\lim_{x\to 2^-}(x^2-2)=2$.
$\lim_{x\to 2^+}f(x)=\lim_{x\to 2^+}2=2$. Como
$\lim_{x\to 2^-}f(x)=\lim_{x\to 2^+}f(x)$, $\lim_{x\to 2}f(x)$ existe e vale
$2$.
\begin{center}
\begin{bmlimage}\begin{tikzpicture}[scale=0.7]
\draw[->] (-1.5,0)--(3,0)node[right]{$x$};
\draw[->] (0,-2)--(0,2.4);
\draw[->, thick] (-2,-2)--(0,2);
\fill (0,-2) circle (0.45mm);
\draw[thick, domain=0:2] plot (\x,{\x^2-2});
\fill (2,2) circle (0.45mm);
\draw[thick] (2,2)--(3,2);
\end{tikzpicture}\end{bmlimage}
\end{center}
\end{Solution}
\begin{Solution}{4.31}
O ponto $Q$ é da forma $Q=(\lambda,\lambda^2)$, e $Q\to O$
corresponde a $\lambda\to 0$.
Temos $M=(\frac{\lambda}{2},\frac{\lambda^2}{2})$.
É fácil ver que a equação da reta $r$ é
$y=-\frac{1}{\lambda}x+\frac{\lambda^2}{2}+\frac12$. Logo,
$R=(0,\frac{\lambda^2}{2}+\frac12)$. Quando $Q$ se aproxima da origem,
isto é, quando $\lambda$ se aproxima de $0$, $\lambda^2$ decresce,
o que significa que $R$ \emph{desce}. Quando $\lambda\to 0$, $R\to
(0,\frac12)$. (Pode parecer contra-intuitivo, já que o segmento $OQ$ tende a
ficar sempre mais horizontal, logo o segmento $MR$ fica mais vertical, à medida
que $Q\to O$.)
\end{Solution}
\begin{Solution}{4.32}
\mbox{}
\begin{center}
\begin{bmlimage}\begin{tikzpicture}
\pgfmathsetmacro{\r}{2};
\pgfmathsetmacro{\n}{10};
\pgfmathsetmacro{\incrang}{2*3.14152/\n};
\draw (0,0) circle (\r);
\foreach \k in {0,...,\n} {
\coordinate (Pk) at ({\r*cos(\k*\incrang r)},{\r*sin(\k*\incrang r)});
\coordinate (Pkm) at ({\r*cos((\k-1)*\incrang r)},{\r*sin((\k-1)*\incrang
r)});
\fill[color=gray!10] (0,0)--(Pk)--(Pkm)--cycle;
\draw (0,0)--(Pk)--(Pkm);
}
\coordinate (Pk) at ({\r*cos(\incrang r)},{\r*sin(\incrang r)});
\coordinate (Pkm) at ({\r*cos(0 r)},{\r*sin(0 r)});
\coordinate (M) at ($(Pk)!(0,0)!(Pkm)$);
\fill[color=gray!30] (0,0)--(Pk)--(Pkm)--cycle;
\draw[thick] (0,0)--(Pk)--(Pkm)--cycle;
\draw[thick] (0,0)--(M);
%\draw (M) node[above right]{$M$};
\end{tikzpicture}\end{bmlimage}
\end{center}
Como um setor tem abertura $\alpha_n=\frac{2\pi}{n}$,
a área de cada triângulo se calcula facilmente:
$$2\times \frac12
\times(r\cos\tfrac{\alpha_n}{2})\times
(r\sen\tfrac{\alpha_n}{2})=\frac{r^2}{2}\sen
{\alpha_n}=\frac{r^2}{2}\sen \tfrac{2\pi}{n}\,.$$
Logo, a área do polígono é dada por $A_n=n\times \frac{r^2}{2}\sen
\frac{2\pi}{n}$. No limite $n\to \infty$ obtemos
$$
\lim_{n\to \infty}A_n=r^2\lim_{n\to \infty}\frac{n}{2}\sen \frac{2\pi}{n}
=\pi r^2\lim_{n\to \infty}\frac{1}{\frac{2\pi}{n}}\sen \tfrac{2\pi}{n}
=\pi r^2\lim_{t\to 0^+}\frac{\sen t}{t}=\pi r^2\,.
$$
\end{Solution}
\begin{Solution}{4.33}
\eqref{itrevisaolimites1} $32$
\eqref{itrevisaolimites11} $\frac13$
\eqref{itrevisaolimites2} $2$
\eqref{itrevisaolimites3} $0$
\eqref{itrevisaolimites31} $1$
\eqref{itrevisaolimites4} $-1$
\eqref{itrevisaolimites5} Com a mudança $y=x+1$, $\frac12$
\eqref{itrevisaolimites6} $0$
\eqref{itrevisaolimites7} $-\infty$
\eqref{itrevisaolimites8} $0$
\eqref{itrevisaolimites9} $0$
\eqref{itrevisaolimites10} $\tfrac12$ (Pois é, esse limite é
um pouco mais difícil. Calcularemos ele no
Capítulo~\ref{Cap:Derivacao} usando a regra de
Bernoulli-l'Hôpital.)
\eqref{itrevisaolimites12} $\mp \pisobredois$
\eqref{itrevisaolimites13} Como $\sen$ é contínua em $\pisobredois$,
$\li{x}{+\infty}\sen(\frac{\pi}{2}+\frac{1}{1+x^2})=\sen(\frac{
\pi}{2}+\li{x}{+\infty}\frac{1}{1+x^2})=\sen \pisobredois =1$.
\eqref{itrevisaolimites14} $0$
\eqref{itrevisaolimites15} $-\frac{1}{10}$
\eqref{itrevisaolimites16} $\frac{\sqrt{3}}{2}$
\eqref{itrevisaolimites161} $\frac{2}{3}$
\eqref{itrevisaolimites17} $0$
\eqref{itrevisaolimites20} $1$
\eqref{itrevisaolimites21} $1$
\end{Solution}
\begin{Solution}{4.34}
Seja $\epsilon>0$ e $N$ grande o suficiente, tal que $|g(x)-\ell|\leq \epsilon$ e $|h(x)-\ell|\leq \epsilon$ para todo $x\geq N$.
Para esses $x$, podemos escrever $f(x)-\ell \leq h(x)-\ell\leq |h(x)-\ell|\leq \epsilon$, e $f(x)-\ell\geq g(x)-\ell\geq -|g(x)-\ell|\geq -\epsilon$. Logo, $|f(x)-\ell|\leq \epsilon$.
\end{Solution}
\begin{Solution}{4.35}
\eqref{itlimdificeis0} Como $\sqrt{1-\cos^2x}=\sqrt{\sen^2x}=|\sen x|$ e
$x\mapsto |x|$ é contínua,
$$\lim_{x\to 0}\frac{\sqrt{1-\cos x}}{|x|}=\lim_{x\to
0}\frac{1}{\sqrt{1+\cos x}}\frac{|\sen x|}{|x|}
=\Bigl(\lim_{x\to 0}\frac{1}{\sqrt{1+\cos x}}\Bigr)
\cdot \Bigl|
\lim_{x\to 0}\frac{\sen x}{x}\Bigr|=\frac{1}{\sqrt{2}}\,.
$$
\eqref{itlimdificeis1} Como $\sen (a+h)=\sen a\cos h+\sen h\cos a$, temos
$$
\lim_{h\to 0}\frac{\sen (a+h)-\sen a}{h}=\sen a \Bigl(\lim_{h\to 0}\frac{\cos
h-1}{x}\Bigr)+\cos a\Bigl(\lim_{h\to 0}\frac{\sen h}{h}\Bigr)=\cos a\,.
$$
\eqref{itlimdificeis2} Escrevendo
$$
\frac{x^3-\alpha^3}{\sen (\tfrac{\pi}{\alpha}x)}=
\frac{x^3-\alpha^3}{x-\alpha}\frac{1}{\frac{\sen
(\tfrac{\pi}{\alpha}x)}{x-\alpha}}\,.
$$
Já calculamos $\lim_{x\to \alpha}\frac{x^3-\alpha^3}{x-\alpha}= 3\alpha^2$, e
chamando $y\pardef \tfrac{\pi}{\alpha}x$ seguido por $y'\pardef y-\pi$,
$$\lim_{x\to \alpha}\frac{\sen(\tfrac{\pi}{\alpha}x)}{x-\alpha}
=\lim_{y\to \pi}\frac{\sen(y)}{\frac{\alpha}{\pi}(y-\pi)}
=\frac{\pi}{\alpha}\lim_{y'\to 0}\frac{\sen(y'+\pi)}{y'}
=-\frac{\pi}{\alpha}\lim_{y'\to 0}\frac{\sen(y')}{y'}=-\frac{\pi}{\alpha}\,.$$
Logo,
$$
\lim_{x\to \alpha}\frac{x^3-\alpha^3}{\sen
(\tfrac{\pi}{\alpha}x)}=(3\alpha^2)/ (-\frac{\pi}{\alpha})=-3\alpha^3/ \pi\,.
$$
\eqref{itlimdificeis3} Comecemos definindo $t$ tal que $\pi-3x=3t$, isto é:
$t\pardef \pisobretres-x$:
$$\lim_{x\to \pisobretres}\frac{1-2\cos
x}{\sen(\pi-3x)}=\lim_{t\to 0}\frac{1-2\cos (\pisobretres-t)}{\sen (3t)}\,.$$
Mas $\cos (\pisobretres-t)=\cos \pisobretres\cos t+\sen \pisobretres\sen
t=\frac12\cos t+\frac{\sqrt{3}}{2}\sen t$,
\begin{align*}
\lim_{t\to 0}\frac{1-2\cos (\pisobretres-t)}{\sen (3t)}&=
\lim_{t\to 0}\frac{1-\cos t}{\sen (3t)}-\sqrt{3}\lim_{t\to 0}\frac{\sen
(t)}{\sen (3t)}\\
&=\lim_{t\to 0}\frac{1-\cos t}{t}\frac{1}{
3\frac{\sen (3t)}{3t}}-
\sqrt{3}\lim_{t\to
0}\frac{\sen (t)}{t}\frac{1}{3\frac{\sen
(3t)}{3t}}=0-\sqrt{3}\frac{1}{3}=-\frac{1}{\sqrt{3}}\,.
\end{align*}
\eqref{itlimdificeis34}
Se $a\geq b$, é melhor escrever $a^x+b^x=a^x(1+(b/a)^x)$, logo
\[
\lim_{x\to\infty}\frac{1}{x}\ln(a^x+b^x)
=\ln a+
\lim_{x\to\infty}\frac{\ln(1+(b/a)^x)}{x}
=\ln a\,.
\]
O caso $a<b$ se trata da mesma maneira. Obtemos:
\[
\lim_{x\to\infty}\frac{1}{x}\ln(a^x+b^x)
=
\begin{cases}
\ln a&\text{ se }a\geq b\,,\\
\ln b&\text{ se }a< b\,.\\
\end{cases}
\]
\eqref{itlimdificeis37}
O caso $n=1$ é trivial: $(x_0+h)^1=x_0+h$. Quando $n=2$,
$(x_0+h)^2=x_0^2+2x_0h+h^2$, logo (veja o Exemplo
\ref{Ex:derivxissdoisemum})
$$
\lim_{h\to 0}\frac{(x_0+h)^2-x_0^2}{h}=
\lim_{h\to 0}(2x_0+h)=2x_0\,.
$$
Para $n=3,4,\dots$, usaremos a fórmula do binômio de
Newton:
$$(x_0+h)^n=x_0^n+\binom{n}{1}x_0^{n-1}h+\binom{n}{2}x_0^{n-2}
h^2+\dots+\binom{n}{k}x_0^{n-k} h^k+\dots+h^n\,,
$$
onde $\binom{n}{k}=\frac{n!}{(n-k)!k!}$. Portanto,
$$
\frac{(x_0+h)^n-x_0^n}{h}=\binom{n}{1}x_0^{n-1}+\binom{n}{2}x_0^{n-2}
h+\dots+\binom{n}{k}x_0^{n-k} h^{k-1}+\dots+h^{n-1}\,.
$$
Observe que cada termo dessa soma, a partir do segundo, contém
uma potência de $h$. Logo, quando $h\to 0$, só sobra
o primeiro termo: $\binom{n}{1}x_0^{n-1}=nx_0^{n-1}$. Logo,
\[
\lim_{h\to 0}\frac{(x_0+h)^n-x_0^n}{h}=nx_0^{n-1}\,.
\]
Esse limite será usado para \emph{derivar} polinômios, no próximo capítulo.
\end{Solution}
\protect \section *{Capítulo \ref {Cap:Continuidade}}
\begin{Solution}{5.1}
Em qualquer ponto $a\neq 0$, os limites laterais nem existem, então
$f$ é descontínua. Por outro lado vimos que $\lim_{x\to
0^+}f(x)=\lim_{x\to 0^- }f(x)=0$. Logo,
$\lim_{x\to 0}f(x)=f(0)$: $f$ é contínua em $0$.
\end{Solution}
\begin{Solution}{5.2}
$D=\bR$, $C=\bR_*$.
\end{Solution}
\begin{Solution}{5.3}
Considere um $a\neq 2$. $f$ sendo uma razão de polinómios, e como o denumerador não se
anula em $a$, a Proposição
\ref{Prop:continuidadechiante} implica que $f$ é contínua em $a$.
Na verdade, quando $x\neq 2$, $f(x)=\frac{x^2-3x+2}{x-2}=\frac{(x-1)(x-2)}{x-2}=x-1$.
Logo, $\lim_{x\to 2}f(x)=\lim_{x\to 2}(x-1)=1$. Como $1\neq f(2)=0$, $f$ é descontínua em
$2$.
Para tornar $f$ contínua na reta toda, é so redefiní-la em $x=2$, da seguinte maneira:
$$
\tilde{f}(x)\pardef
\begin{cases}
\frac{x^2-3x+2}{x-2}&\text{ se }x\neq 2\,,\\
1&\text{ se }x=2\,.
\end{cases}
$$
Agora, $\tilde{f}(x)=x-1$ para todo $x\in \bR$.
\end{Solution}
\begin{Solution}{5.4}
Como $\lim_{x\to 1}f(x)=1-a$ e que $f(1)=5+a$, é preciso ter $1-a=5+a$, o que implica
$a=-2$.
\end{Solution}
\begin{Solution}{5.5}
Por um lado, como $\tanh \tfrac1x$ é a composição de duas funções contínuas, ela é
contínua em todo $a\neq 0$.
Um raciocínio parecido implica que $g$ é contínua em todo $a\neq 0$.
Por outro lado,
vimos no item \eqref{itmudvarlim4} do Exercício \ref{Exo:mudvarlimites} que $\lim_{x\to
0^{\pm}}\tanh \frac{1}{x}=\pm 1$, o que implica que $f$ é descontínua em $a=0$.
Vimos no item \eqref{itmudvarlim5} do mesmo exercício que $\lim_{x\to
0^{\pm}}x\tanh \frac{1}{x}=0$, logo $\lim_{x\to 0}g(x)$ existe e vale $g(0)$.
Logo, $g$ é contínua em $a=0$.
\begin{center}
\begin{bmlimage}\begin{tikzpicture}
\pgfmathsetmacro{\a}{3};
\pgfmathsetmacro{\e}{0.12};
\draw[ ->] (-\a,0)--(\a,0)node[right]{$x$};
\draw[ ->] (0,-1.3)--(0,1.3)node[left]{$\tanh\frac1x$};
\draw[->, thick, domain=-\a:-\e] plot (\x,
{(exp(1/\x)-exp(-1/\x))/(exp(1/\x)+exp(-1/\x))});
\draw[<-, thick, domain=\e:\a] plot (\x, {(exp(1/\x)-exp(-1/\x))/(exp(1/\x)+exp(-1/\x))});
\fill (0,0) circle (0.50mm);

\begin{scope}[xshift=7cm]
\draw[ ->] (-\a,0)--(\a,0)node[right]{$x$};
\draw[ ->] (0,-1.3)--(0,1.3)node[left]{$x\tanh\frac1x$};
\draw[->, thick, domain=-\a:-\e] plot (\x,
{\x*(exp(1/\x)-exp(-1/\x))/(exp(1/\x)+exp(-1/\x))});
\draw[<-, thick, domain=\e:\a] plot (\x,
{\x*(exp(1/\x)-exp(-1/\x))/(exp(1/\x)+exp(-1/\x))});
\fill (0,0) circle (0.50mm);
\end{scope}
\end{tikzpicture}\end{bmlimage}
\end{center}
\end{Solution}
\begin{Solution}{5.6}
(Esboçar os gráficos de $f,g,h$ ajuda a compreensão do exercício).

Temos $f(-1)=1$, $f(2)=4$.
Como $f$ é contínua, o Teorema \eqref{Teo:ValInterm} se aplica:
se $1\leq h\leq 4$, o gráfico de $f$ corta a reta horizontal de
altura $y=h$ pelo menos uma vez. Na verdade, ele corta a reta
exatamente uma vez se $1<h\leq 4$, e duas vezes se $h=1$.

Temos $g(-1)=-1$, $g(1)=1$.
Como $g$ é descontínua em $x=0$, o teorema não se aplica. Por
exemplo, o gráfico de $g$ nunca corta a reta horizontal $y=\frac12$.

Temos $h(0)=-1$, $h(2)=1$. Apesar de $h$ não ser contínua, ela
satisfaz à propriedade do valor intermediário. De fato, o gráfico de
$h$ corta a reta $y=h_*$ duas vezes se $-1\leq h_*<1$, e uma vez se $h_*=1$.
\end{Solution}
\begin{Solution}{5.7}
$a=1$, $b=3$, $c=\pm 2$.
\end{Solution}
\begin{Solution}{5.8}
Seja $y\in \bR$ fixo, qualquer. Como $\lim_{x\to
+\infty}f(x)=+\infty$, existe $b>0$ grande o suficiente tal que
$f(b)>y$.
Como $\lim_{x\to -\infty}f(x)=-\infty$, existe $a<0$ grande o
suficiente tal que $f(a)<y$.
Pelo Teorema do Valor Intermediário, existe $c\in [a,b]$ tal que
$f(c)=y$. Isto implica que $y\in \imagem(f)$.
\end{Solution}
\begin{Solution}{5.9}
Considere $\lim_{x\to 0^-}f(x)$. Chamando $y\pardef -x$, $x\to 0^-$ corresponde
a $y\to 0^+$.
Logo, $$\lim_{x\to 0^-}f(x)=\lim_{y\to 0^+}f(-y)=-\lim_{y\to 0^+}f(y)\equiv-
\lim_{x\to 0^+}f(x)\,.$$
Portanto, para uma função ímpar ser contínua em $0$, é preciso ter
$\lim_{x\to 0^+}f(x)=f(0)=0$ (não pode ser $L>0$).
\end{Solution}
\protect \section *{Capítulo \ref {Cap:Derivacao}}
\begin{Solution}{6.1}
Se $P=(a,a^2)$, $Q=(\lambda,\lambda^2)$, a equação da reta $r^{PQ}$ é dada por
$y=(\lambda+a)x-a\lambda$. Quando $\lambda\to a$ obtemos
a equação da reta tangente à parábola em $P$: $y=2a x-a^2$.
Por exemplo, se $a=0$, a equação da reta tangente é $y=0$, se $a=2$, é
$y=4x-4$,
$a=-1$, é $y=-2x-1$ (o que foi calculado no Exemplo
\ref{Ex:primeiraretatangente}).
\end{Solution}
\begin{Solution}{6.2}
Como $x^2-x=(x-\frac12)^2-\frac14$, o gráfico obtém-se a partir do gráfico de
$x\mapsto x^2$ por duas translações.
Usando a definição de derivada, podemos calcular para todo $a$:
$$f'(a)=\lim_{x\to a}\frac{f(x)-f(a)}{x-a}
=\lim_{x\to a}\frac{(x^2-x)-(a^2-a)}{x-a}=
\lim_{x\to a}\Bigl\{\frac{x^2-a^2}{x-a}-1\Bigr\}=2a-1\,.$$
Aplicando essa fórmula para $a=0,\frac12,1$, obtemos $f'(0)=-1$,
$f'(\frac12)=0$, $f'(1)=+1$.
Esses valores correspondem às inclinações das retas
tangentes ao gráfico nos pontos $(0,f(0))=(0,0)$,
$(\frac12,f(\frac12))=(\frac12,-\frac14)$ e $(1,f(1))=(1,0)$:
\begin{center}
\begin{bmlimage}\begin{tikzpicture}[scale=1.7]
\newcommand{\funcao}[1]{((#1)^2-(#1))}
\newcommand{\dfuncao}[2]{ ((\funcao{#1+#2})/(#2)-(\funcao{#1})/(#2)) }
\draw[ ->] (0,-0.2)--(0,1)node[right]{$\scriptstyle{x^2-x}$};
\draw[ ->] (-0.5,0)--(1.5,0);
\draw[thick, domain=-0.5:1.5] plot (\x,{\funcao{\x}});
\foreach \a in {0,0.5,1} {
\draw[thick,  domain={\a-0.3}:{\a+0.3}] plot
(\x,{(\dfuncao{\a}{0.01})*(\x-\a)+\funcao{\a}});
\fill (\a,{\funcao{\a}}) circle (0.40mm);
}
\draw[dotted]
(0.5,{\funcao{0.5}})--(0.5,0)node[above]{$\scriptstyle{\tfrac12}$};

\draw (1,0) node[above]{$\scriptstyle{1}$};
\end{tikzpicture}\end{bmlimage}
\end{center}
\end{Solution}
\begin{Solution}{6.3}
\eqref{itderivelem11} $f'(1)=\half$,
\eqref{itderivelem1} $f'(0)=\half$ (a mesma do item anterior, pois o
gráfico de $\sqrt{1+x}$ é o de $\sqrt{x}$ transladado de $1$ para a esquerda!),
\eqref{itderivelem2} $f'(0)=1$,
\eqref{itderivelem3} $f'(-1)=-4$,
\eqref{itderivelem4} $f'(2)=-\frac{1}{4}$.
\end{Solution}
\begin{Solution}{6.4}
\eqref{iteqretang1} $y=3x+9$,
\eqref{iteqretang2} $y=\frac{1}{4}$,
\eqref{iteqretang3} $y=\half x+1$,
\eqref{iteqretang4} $y=-x-2$, $y=-x+2$
\eqref{iteqretang5} Observe que a função descreve a metade superior de um
circulo de raio $1$ centrado na origem. As retas tangentes são, em $(-1,0)$:
$x=-1$, em $(1,-1)$: não existe (o ponto nem pertence ao círculo!), em $(0,1)$:
$y=1$, e em $(1,0)$: $x=1$.
\eqref{iteqretang6} Mesmo sem saber ainda como calcular a derivada da
função seno: $y=x$, $y=1$.
\end{Solution}
\begin{Solution}{6.5}
Primeiro é preciso ter uma função para representar o círculo na vizinhança de
$P_1$: $f(x)\pardef \sqrt{25-x^2}$. A inclinação da tangente em $P_1$ é dada por
\begin{align*}
f'(3)=\lim_{x\to 3}\frac{f(x)-f(3)}{x-3}&=
\lim_{x\to 3}\frac{\sqrt{25-x^2}-\sqrt{16}}{x-3}\\
&=\lim_{x\to 3}\frac{(25-x^2)-{16}}{(x-3)(\sqrt{25-x^2}+\sqrt{16})}
=\lim_{x\to 3}\frac{-(3+x)}{\sqrt{25-x^2}+\sqrt{16}}=-\tfrac34\,.
\end{align*}
(Essa inclinação poderia ter sido obtido observando que a reta
procurada é perpendicular ao segmento $OP$, cuja inclinação é
$\frac43$...)
Portanto, a equação da reta tangente em $P_1$ é $y=-\frac34
x+\frac{25}{4}$.  No ponto $P_2$, é preciso tomar a função
$f(x)\pardef -\sqrt{25-x^2}$. Contas parecidas dão a equação
da tangente ao círculo em $P_2$: $y=\frac34 x-\frac{25}{4}$.
\begin{center}
\begin{bmlimage}\begin{tikzpicture}[scale=1]
\draw[ ->] (0,-1.2)--(0,1.2);
\draw[ ->] (-1.2,0)--(1.2,0);
\draw[dotted]
(0.6,0)node[below]{$\scriptstyle{3}$} -- (0.6,0.8) -- (0,0.8)node[left]
{$\scriptstyle{4}$};
\pgfmathsetmacro{\a}{0.6};
\draw[very thick, domain={\a-0.4}:{\a+0.4}] plot
(\x,{-0.75*(\x-\a)+0.8});
\draw[very thick,  domain={\a-0.4}:{\a+0.4}] plot
(\x,{+0.75*(\x-\a)-0.8});
\draw (0,0) circle (1cm);
\fill (0.6,0.8) circle (0.40mm);
\fill (0.6,-0.8) circle (0.40mm);
\draw (0.6,0.8) node[above right]{$P_1$};
\draw (0.6,-0.8) node[below right]{$P_2$};
\draw[very thick] (1,-0.4)--(1,0.4);
\fill (1,0) circle (0.40mm);
\draw (1,0) node[above right]{$P_3$};
\end{tikzpicture}\end{bmlimage}
\end{center}
A reta tangente ao círculo no ponto $P_3$ é vertical, e tem equação $x=5$.
Aqui podemos observar que a derivada de $f$ em $a=5$ \emph{não existe}, porqué
a inclinação de uma reta vertical não é definida (o que não impede achar a sua
equação...)!
\end{Solution}
\begin{Solution}{6.6}
Se $f(x)=\sqrt{x}$, temos que para todo $a>0$,
$f'(a)=\frac{1}{2\sqrt{a}}$.
Como a reta $8x-y- 1 = 0$ tem inclinação $8$, precisamos achar um $a$ tal que
$f'(a)=8$, isto é, tal que $\frac{1}{2\sqrt{a}}=8$: $a=\frac{1}{256}$.
Logo, o ponto procurado é $P=(a,f(a))=(\frac{1}{256},\frac{1}{16})$.
\end{Solution}
\begin{Solution}{6.7}
Para a reta $y=x-1$ (cuja inclinação é $1$) poder ser tangente ao gráfico de
$f$ em algum ponto $(a,f(a))$, esse $a$ deve satisfazer $f'(a)=1$. Ora, é fácil
ver que para um $a$ qualquer, $f'(a)=2a-2$. Logo, $a$ deve satisfazer $2a-2=1$,
isto é: $a=\frac32$. Ora, a reta e a função devem ambas passar pelo ponto
$(a,f(a))$, logo $f(a)=a-1$, isto é:
$(\frac32)^2-2\cdot\frac32+\beta=\frac32-1$. Isolando:
$\beta=\frac{5}{4}$.
\begin{center}
\begin{bmlimage}\begin{tikzpicture}
\newcommand{\funcao}[1]{(#1)^2-2*(#1)+1.25}
\newcommand{\dfuncao}[2]{ (\funcao{#1+#2})/(#2)-(\funcao{#1})/(#2)}
\draw[ ->] (0,-0.2)--(0,2.5) node[right]{$y$};
\draw[ ->] (-1,0)--(3,0) node[right]{$x$};
\draw[thick, domain=-0.5:2.5] plot (\x,{\funcao{\x}})
node[right]{$y=x^2-2x+\frac54$};
\pgfmathsetmacro{\a}{1.5};
\draw[thick,  domain={\a-1}:{\a+1}] plot
(\x,{(\dfuncao{\a}{0.01})*(\x-\a)+\funcao{\a}}) node[right]{$y=x-1$};
\fill (\a,{\funcao{\a}}) circle (0.40mm);
\end{tikzpicture}\end{bmlimage}
\end{center}
Esse problema pode ser resolvido sem usar derivada:
para a parábola $y=x^2-2x+\beta$ ter $y=x-1$ como reta tangente, a única
possibilidade é que as duas se intersectem em um ponto só, isto é, que a
equação $x^2-2x+\beta=x-1$ possua uma única solução. Rearranjando:
$x^2-3x+\beta+1=0$. Para essa equação ter uma única solução, é preciso que o
seu $\Delta=5-4\beta=0$. Isso implica $\beta=\frac{5}{4}$.
\end{Solution}
\begin{Solution}{6.8}
Seja $P=(a,\frac1a)$ um ponto qualquer do gráfico. Como
$f'(a)=-\frac{1}{a^2}$, a reta tangente ao gráfico em $P$ é
$y=f'(a)(x-a)+f(a)=-\frac{1}{a^2}(x-a)+\frac1a$. Para essa reta passar pelo
ponto $(0,3)$, temos $3=-\frac{1}{a^2}(0-a)+\frac1a$, o que
significa que $a=\frac{2}{3}$.
Logo, a reta tangente ao gráfico de $\frac1x$ no ponto $P=(\frac23,\frac32)$
passa pelo ponto $(0,3)$.
\end{Solution}
\begin{Solution}{6.9}
$P=(-1,2)$.
\end{Solution}
\begin{Solution}{6.10}
Por exemplo, $f(x)\pardef |x+1|/2-|x|+|x-1|$.
Mais explicitamente,
\begin{center}
\begin{bmlimage}\begin{tikzpicture}
\draw (-8,0.8) node{$\displaystyle{
f(x)=
\begin{cases}
\frac{1-x}{2}&\text{ se }x\leq -1\\
\frac{x+3}{2}&\text{ se }-1\leq x\leq 0\\
\frac{3-3x}{2}&\text{ se }0\leq x\leq 1\\
\frac{x-1}{2}&\text{ se }x\geq 1\,.
\end{cases}
}$};
\draw [thick, domain=-2:2, samples=200]plot
(\x,{abs(\x+1)/2-abs(\x)+abs(\x-1)});
\draw [->](-2,0)--(2,0) ;
\draw (2.2,0) node {$x$};
\draw [->](0,-0.5)--(0,2);
\draw (-0.5,1.8) node {$f(x)$};
\draw (-1,0) node {$\shortmid$};
\draw (-1,-0.4) node {$-1$};
\draw (1,0) node {$\shortmid$};
\draw (1,-0.4) node {$1$};
\end{tikzpicture}\end{bmlimage}
\end{center}
$f$ não é derivável em $x=1$, porqué
$\lim_{x\to 1^+}\frac{f(x)-f(1)}{x-1}=\lim_{x\to
1^+}\frac{\frac{x-1}{2}-0}{x-1}=\frac12$,
enquanto
$\lim_{x\to 1^-}\frac{f(x)-f(1)}{x-1}=\lim_{x\to
1^-}\frac{\frac{3-3x}{2}-0}{x-1}=-\frac32\neq \frac12$.
A não-derivabilidade nos pontos $-1$ e $0$ obtem-se da mesma maneira.
\end{Solution}
\begin{Solution}{6.11}
De fato, se $f$ é par,
\begin{align*}
f'(-x)=\lim_{h\to 0}\frac{f(-x+h)-f(-x)}{h}
&=\lim_{h\to 0}\frac{f(x-h)-f(x)}{h}\\
&=-\lim_{h'\to 0}\frac{f(x+h')-f(x)}{h'}=-f'(x)\,.
\end{align*}
\end{Solution}
\begin{Solution}{6.12}
$af'(a)-f(a)$
\end{Solution}
\begin{Solution}{6.13}
$(\sqrt{x})'=\lim_{h\to 0}\frac{\sqrt{x+h}-\sqrt{x}}{h}=
\lim_{h\to 0}\frac{1}{\sqrt{x+h}+\sqrt{x}}=\frac{1}{2\sqrt{x}}$.
O outro limite se calcula de maneira parecida:
$$(\frac{1}{\sqrt{x}})'=\lim_{h\to
0}\frac{\frac{1}{\sqrt{x+h}}-\frac{1}{\sqrt{x}}}{h}=
\lim_{h\to
0}\frac{\sqrt{x}-\sqrt{x+h}}{h\sqrt{x}\sqrt{x+h}}=\cdots=-\frac{1}{2\sqrt{x^3}}
\,.
$$
\end{Solution}
\begin{Solution}{6.14}
Como $(\sen)'(x)=\cos x$, a inclinação da reta tangente em $P_1$ é $\cos(0)=1$,
em $P_2$ é $\cos(\pisobredois)=0$, e em $P_3$ é $\cos(\pi)=-1$. Logo, as
equações das respectivas retas tangentes são $r_1$: $y=x$, $r_2$: $y=1$, $r_3$:
$y=-(x-\pi)$:
\begin{center}
\begin{bmlimage}\begin{tikzpicture}[scale=1]
\newcommand{\funcao}[1]{sin(#1 r)}
\newcommand{\dfuncao}[2]{ (\funcao{#1+#2})/{#2}-(\funcao{#1})/{#2} }
\draw[ ->] (0,-0.2)--(0,1)node[left]{$\scriptstyle{\sen x}$};
\draw[ ->] (-2,0)--(5,0);
\draw[thick, domain=-1.8:4.8] plot (\x,{\funcao{\x}});

\pgfmathsetmacro{\e}{0.75};
\pgfmathsetmacro{\a}{0};
\draw[thick,  domain={\a*1.5707-\e}:{\a*1.5707+\e}] plot
(\x,{\x});
\pgfmathsetmacro{\a}{1};
\draw[thick,  domain={\a*1.5707-\e}:{\a*1.5707+\e}] plot
(\x,{1});
\pgfmathsetmacro{\a}{2};
\draw[thick,  domain={\a*1.5707-\e}:{\a*1.5707+\e}] plot
(\x,{3.1415-\x});

\foreach \a in {0,1,2} {
\fill ({\a*1.5707},{sin(\a*1.5707 r)}) circle (0.50mm);
}
% \draw[dotted]
% (0.5,{\funcao{0.5}})--(0.5,0)node[above]{$\scriptstyle{\tfrac12}$};
%
% \draw (1,0) node[above]{$\scriptstyle{1}$};
\end{tikzpicture}\end{bmlimage}
\end{center}
\end{Solution}
\begin{Solution}{6.16}
Por exemplo, se $f(x)=g(x)=x$, temos $(f(x)g(x))'=(x\cdot x)'=(x^2)'=2x$,
e $f'(x)g'(x)=1\cdot 1=1$. Isto é, $(f(x)g(x))'\neq f'(x)g'(x)$.
\end{Solution}
\begin{Solution}{6.17}
Já sabemos que $(x)'=1$, e que $(x^2)'=2x$, o que prova a fórmula para $n=1$ e $n=2$.
Supondo que a fórmula foi provada para $n$, provaremos que ela vale para $n+1$
também.
De fato, usando a regra de Leibniz e a hipótese de indução,
\[
(x^{n+1})'=
(x\cdot x^n)'=1\cdot x^n+x\cdot nx^{n-1}=x^n+nx^n=(n+1)x^n\,.
\]
\end{Solution}
\begin{Solution}{6.18}
\eqref{itderivbas0} $-5$
\eqref{itderivbas1} $(x^3-x^7)'=(x^3)'-(x^7)'=3x^2-7x^6$.
\eqref{itderivbas111}
$(1+x+\frac{x^2}{2}+\frac{x^3}{3})'=(1)'+(x)'+(\frac{x^2}{2})'+(\frac{x^3}{3})'
=1+x+x^2$.
\eqref{itderivbas2}
$(\frac{1}{1-x})'=-\frac{1}{(1-x)^2}\cdot(1-x)'=\frac{1}{(1-x)^2}$
\eqref{itderivbas15} $\sen x+x\cos x$
\eqref{itderivbas151} Usando duas vezes a regra de Leibniz:
$((x^2+1)\sen x\cos x)'=2x\sen x\cos x+(x^2+1)(\cos^2x-\sen^2x)$
\eqref{itderivbas3} $\frac{x\cos x-\sen x}{x^2}$
\eqref{itderivbas4} $(\frac{x+1}{x^2-1})'=(\frac{1}{x-1})'=\frac{-1}{(x-1)^2}$.
\eqref{itderivbas1111} $(x+1)^5=f(g(x))$ com $f(x)=x^5$ e $g(x)=x+1$.
Logo, $((x+1)^5)'=f'(g(x))g'(x)=5(x+1)^4$. Obs: poderia também expandir
$(x+1)^5=x^5+\cdots$, derivar termo a termo, mas é muito mais longo, e a
resposta não é fatorada.
\eqref{itderivbas6} Como $(3+\frac{1}{x})^2=f(g(x))$ com
$f(x)=x^2$ e $g(x)=3+\frac{1}{x}$, e que $f'(x)=2x$,
$g'(x)=(3+\frac{1}{x})'=0-\frac{1}{x^2}$, temos
$((3+\frac{1}{x})^2)'=2(3+\frac{1}{x})\cdot(\frac{-1}{x^2})=-2\frac{3+\frac{1}
{x}}{x^2}$.
\eqref{itderivbas5} Como $\sqrt{1-x^2}=f(g(x))$, com $f(x)=\sqrt{x}$,
$g(x)=1-x^2$, e que $f'(x)=\frac{1}{2\sqrt{x}}$, $g'(x)=-2x$,
temos $(\sqrt{1-x^2})'=\frac{-x}{\sqrt{1-x^2}}$-
\eqref{itderivbas7} $3\sen^2x\cos x+7\cos^6x\sen x$
\eqref{itderivbas8} $\frac{\sen x}{(1-\cos x)^2}$
\eqref{itderivbas8meio} $\frac{2\sen (2x-1)}{(\cos(2x-1))^2}$
\eqref{itderivbas9}
$(\frac{1}{\sqrt{1+x^2}})'=((1+x^2)^{-\frac12})'=-\frac12(1+x^2)^{-\frac32}
\cdot (2x)=-\frac{x}{(1+x^2)^{\frac32}}=\frac{-x}{\sqrt{(1+x^2)^3}}$.
\eqref{itderivbas10}
$(\frac{(x^2-1)^2}{\sqrt{x^2-1}})'=((x^2-1)^{\frac32})'=\frac{3}{2}(x^2-1)^{
\frac12}
\cdot(2x)=3x\sqrt{x^2-1}$ Obs: vale a pena simplificar a fração antes de
usar a regra do quociente!
\eqref{itderivbas11} $\frac{9}{\sqrt{9+x^2}(x+\sqrt{9+x^2})^2}$
\eqref{itderivbas12} $\frac{1}{4\sqrt{x}\sqrt{1+\sqrt{x}}}$
\eqref{itderivbas16} $\frac{\cos x+x\sen x}{(\cos x)^2}$
\eqref{itderivbas17} Usando duas vezes a regra da cadeia:
$(\cos\sqrt{1+x^2})'=(-\sen \sqrt{1+x^2})(\sqrt{1+x^2})'=\frac{-x\sen
\sqrt{1+x^2}}{\sqrt{1+x^2}}$
\eqref{itderivbas18} $\cos(\sen x)\cdot\cos x$
\end{Solution}
\begin{Solution}{6.19}
\eqref{itderivexpon1} $(2e^{-x})'=2(e^{-x})'=2(e^{-x}\cdot(-x)')=-2e^{-x}$.
\eqref{itderivexpon2} $\frac{1}{x+1}$
\eqref{itderivexpon3} $(\ln (e^{3x}))'=(3x)'=3$
\eqref{itderivexpon61} $e^x(\sen x+\cos x)$
\eqref{itderivexpon4} $\cos x \cdot e^{\sen x}$
\eqref{itderivexpon5} $e^{e^x}\cdot e^x$
\eqref{itderivexpon6} $\frac{2e^{2x}}{1+e^{2x}}$
\eqref{itderivexpon7} $\ln x+x\frac{1}{x}=\ln x+1$
\eqref{itderivexpon8} $\frac{-e^{\frac1x}}{x^2}$
\eqref{itderivexpon12} $-\tan x$
\eqref{itderivexpon13} $\frac{-1}{\sen x}$
\end{Solution}
\begin{Solution}{6.20}
$(\senh
x)'=(\frac{e^x-e^{-x}}{2})'=\frac{e^x+e^{-x}}{2}\equiv \cosh x$.
Do mesmo jeito, $(\cosh x)'=\senh x$.
Para $\tanh$, basta usar a regra do quociente.
Observe as semelhanças entre as derivadas das funções trigonométricas
hiperbólicas e as funções trigonométricas.
\end{Solution}
\begin{Solution}{6.21}
\eqref{itlimbargeaotviaderiv1}
Sabemos que o limite $\lim_{x\to 1}\frac{x^{999}-1}{x-1}$ dá a inclinação da
reta tangente ao gráfico da função $f(x)=x^{999}$ no ponto $a=1$, isto é:
$\lim_{x\to 1}\frac{x^{999}-1}{x-1}=f'(1)$. Mas como
$f'(x)=999x^{998}$, temos $f'(1)=999$.
\eqref{itlimbargeaotviaderiv2}
Da mesma maneira, $\lim_{x\to \pi}\frac{\cos x+1}{x-\pi}=
\lim_{x\to \pi}\frac{\cos x-\cos(\pi)}{x-\pi}$ dá a inclinação da reta tangente
ao gráfico do $\cos$ no ponto $\pi$. Como $(\cos x)'=-\sen x$, o limite vale
$0$.
\eqref{itlimbargeaotviaderiv3} $2\pi \cos(\pi^2)$
\eqref{itlimbargeaotviaderiv4} $\frac12$
\eqref{itlimbargeaotviaderiv5} $\lambda$
\end{Solution}
\begin{Solution}{6.22}
Fora de $x=0$, $g$ é derivável e a sua derivada se calcula facilmente:
$g'(x)=(x^2\sen \frac1x)'=2x\sen \frac1x-\cos\frac1x$.
Do mesmo jeito $f$ é derivável fora de $x=0$.
Em $x=0$,
$$
g'(0)=\lim_{h\to 0}\frac{g(h)-g(0)}{h}=\lim_{h\to 0} h\sen \tfrac1h=0\,.
$$
(O último limite pode ser calculado como no Exemplo \ref{Ex:sanduicheseno},
escrevendo
$-h\leq h\sen \tfrac1h\leq +h$.)
Assim, $g$ é derivável também em $x=0$. No entanto, como
$$\lim_{h\to 0}\frac{f(h)-f(0)}{h}=\lim_{h\to 0} \sen \tfrac1h\,,$$
$f'(0)$ não existe: $f$ não é derivável em $x=0$.
\end{Solution}
\begin{Solution}{6.23}
\eqref{itderivfelevg1}
$(x^{\sqrt{x}})'=(e^{\sqrt{x}\ln x})'=(\frac{\ln x}{2}+1){
x^{\sqrt{x}-\frac12}}$.
\eqref{itderivfelevg3} $((\sen x)^x)'=(\ln \sen x+x\cot x)(\sen x)^x$.
\eqref{itderivfelevg4} $(x^{\sen x})'=(\cos x\ln x+\frac{\sen x}{x})x^{\sen x}$.
\eqref{itderivfelevg2} $(x^{x^x})'=\bigl((\ln x+1)\ln
x+\frac1x\bigr)x^xx^{x^x}$.
\end{Solution}
\begin{Solution}{6.24}
 As derivadas são dadas por:
\eqref{itderitruclog1}
$\frac{(x+1)(x+2)(x+3)}{(x+4)(x+5)(x+6)}
(\frac{1}{x+1}+\frac{1}{x+2}+\frac{1}{x+3}-\frac{1}{x+4}-\frac{1}{x+5}-\frac{1}{
x+6})$
\eqref{itderitruclog2}
$\frac{x\sen^3x}{\sqrt{1+\cos^2x}}\bigl(\frac{1}{x}+3\cot x+\frac{\sen
x\cos x}{{1+\cos^2x}}\bigr)$
\eqref{itderitruclog3}
$\bigl(\prod_{k=1}^n(1+x^k)\bigr)\sum_{k=1}^n\frac{kx^{k-1}}{1+x^k}$
\end{Solution}
\begin{Solution}{6.26}
\eqref{itderivfuncinv1} $\frac{-2x}{(\ln a)(1-x^2)}$
\eqref{itderivfuncinv2} $\frac{-2x}{\sqrt{1-(1-x^2)^2}}$
\eqref{itderivfuncinv3} $1$
\eqref{itderivfuncinv4} $-1$
\eqref{itderivfuncinv5} $\frac{-x}{\sqrt{1-x^2}}$
\end{Solution}
\begin{Solution}{6.28}
(O gráfico da função pode ser usado para interpretar o resultado.)
\eqref{itRolleA1} Temos $f(-2)=f(1)$, e como $f'(x)=2x+1$, vemos que a derivada
se anula em $c=-\frac{1}{2}\in (-2,1)$.
\eqref{itRolleA2} Aqui são três pontos possíveis: $c=-\pi$, $c=0$ e $c=+\pi$.
\eqref{itRolleA3} Temos $f(-1)=f(0)$ e $f'(x)=4x^3+1$, cuja raiz é
$-(\frac14)^{1/3}\in (-1,0)$.
\end{Solution}
\begin{Solution}{6.29}
Vemos que existem dois pontos $C$ em que a inclinação é igual à inclinação do
segmento $AB$:
\begin{center}
\begin{bmlimage}\begin{tikzpicture}
\newcommand{\funcao}[1]{sin(#1 r)}
\newcommand{\dfuncao}[2]{ (\funcao{#1+#2})/{#2}-(\funcao{#1})/{#2}}
\pgfmathsetmacro{\a}{-1.57};
\pgfmathsetmacro{\b}{1.57};
\pgfmathsetmacro{\c}{0.8};
\pgfmathsetmacro{\cc}{-\c};
\draw[ ->] (-2,0)--(2,0);
\draw[ ->] (0,-1)--(0,1);
% \draw[color=gray, domain=\a-1.3:\b+1.3] plot (\x,{\funcao{\x}});
\draw[thick, domain=\a:\b] plot (\x,{\funcao{\x}});
\coordinate (A) at (\a,{\funcao{\a}});
\coordinate (B) at (\b,{\funcao{\b}});
\coordinate (C) at (\c,{\funcao{\c}});
\coordinate (Cc) at (\cc,{\funcao{\cc}});
\fill (A) circle (0.4mm);
\fill (B) circle (0.4mm);
\pgfmathsetmacro{\m}{2/3.1415};
\draw[thick,  domain={\c-0.5}:{\c+0.5}] plot
(\x,{\funcao{\c}+\m*(\x-\c)});
\draw[thick,  domain={\cc-0.5}:{\cc+0.5}] plot
(\x,{\funcao{\cc}+\m*(\x-\cc)});
\draw (A) node[left]{$A$};
\draw (B) node[right]{$B$};
\draw (C) node[above]{$C$};
\fill (C) circle (0.4mm);
\draw (Cc) node[below]{$C'$};
\fill (Cc) circle (0.4mm);
\draw[dashed] (A)--(B);
\end{tikzpicture}\end{bmlimage}
\end{center}
O ponto $c\in [-\pisobredois,\pisobredois]$ é tal que
$f'(c)=\frac{f(b)-f(a)}{b-a}=\frac{\sen
(\pisobredois)-\sen(0)}{\pisobredois-0}=\frac{2}{\pi}$. Como $f'(x)=\cos x$, $c$
é solução de $\cos c=\frac{2}{\pi}$. Com a calculadora obtemos duas
soluções: $c=\pm \arcos(\frac{2}{\pi})\simeq \pm 0.69$.
\end{Solution}
\begin{Solution}{6.30}
Como $f$ não é derivável no ponto $2\in [0,3]$, o teorema não se aplica. Não
existe ponto $C$ com as desejadas propriedades:
\begin{center}
\begin{bmlimage}\begin{tikzpicture}[scale=0.7]
\draw[->] (-0.5,0)--(3.5,0);
\draw[->] (0,-0.2)--(0,2.3);
\coordinate (A) at (0,0);
\coordinate (B) at (3,2);
\draw (-0.3,-0.15)--(2,1)--(3.3,2.3);
\draw[thick] (A)--(2,1)--(B);
\fill (A) circle (0.45mm);
\fill (B) circle (0.45mm);
\draw (A)node[above left]{$A$};
\draw (B)node[below right]{$B$};
\draw[dotted] (2,1)--(2,0) node[below]{$\scriptstyle{2}$};
\end{tikzpicture}\end{bmlimage}
\end{center}

\end{Solution}
\begin{Solution}{6.31}
Sejam $x_1<x_2$. Pelo Corolário \ref{Corol:ValorIntermDeriv}, existe $c\in
(x_2,x_2)$ tal que
\[
\frac{\sen x_2-\sen x_1}{x_2-x_1}=\cos(c)\,.
\]
Como $|\cos (c)|\leq 1$, isso dá \eqref{eq_DERIV_sinussLipshh}.
Por ser derivável, já sabemos que $\sen x$ é contínua, mas \eqref{eq_DERIV_sinussLipshh}
permite ver continuidade de uma maneira mais concreta. De fato,
seja $a$ um ponto qualquer da reta. Para mostrar que $\sen x$ é contínua em
$a$, precisamos escolher um $\epsilon>0$ qualquer, e mostrar que se $x$ for
suficientemente perto de $a$, $|x-a|\leq \delta$ (para um certo $\delta$) então
\[
|\sen x-\sen a|\leq \epsilon\,.
\]
Mas, usando \eqref{eq_DERIV_sinussLipshh}, vemos que a condição acima
vale se $\delta\equiv \epsilon$.
\end{Solution}
\begin{Solution}{6.32}

\eqref{itestudfunceleme2}: Como $f'(x)=x^3-x=x(x^2-1)$,
$f(x)$ é crescente em $[-1,0]\cup [1,\infty)$,
decrescente em $(-\infty,-1]\cup[0,1]$:
\begin{center}
\begin{bmlimage}\begin{tikzpicture}[scale=1.2]
\draw[ ->, thin] (-2.2,0)--(2.2,0);
\draw[ ->, thin] (0,-0.6)--(0,1);
\draw[thick, domain=-1.8:1.8, samples=50] plot (\x,{(\x)^4/4-(\x)^2/2});
 \fill (-1.414,0) circle (0.40mm);
 \fill (1.414,0) circle (0.40mm);
 \fill (0,0) circle (0.40mm);
 \fill (-1,-0.25) circle (0.40mm);
 \fill (1,-0.25) circle (0.40mm);
 \draw (-1,-0.25) node[below]{$\scriptstyle{(-1,-\tfrac{1}{4})}$};
 \draw (1,-0.25) node[below]{$\scriptstyle{(+1,-\tfrac{1}{4})}$};
\end{tikzpicture}\end{bmlimage}
\end{center}
\eqref{itestudfunceleme21}: $f(x)=2x^3-3x^2-12x+1$ é
crescente em $(-\infty,-1]\cup[2,\infty)$, decrescente em $[-1,2]$:
\begin{center}
\begin{bmlimage}\begin{tikzpicture}[scale=0.8]
\newcommand{\funcao}[1]{(2*(#1)^3-3*(#1)^2-12*(#1)+1)/10}
\draw[ ->, thin] (-2.2,0)--(4.2,0);
\draw[ ->, thin] (0,-1)--(0,1);
\draw[thick, domain=-2.5:3.5, samples=50] plot (\x,{\funcao{\x}});
\fill (-1,{0.8}) circle (0.5mm);
\draw (-1,0.8) node[above]{$\scriptstyle{(-1,8)}$};
\draw (2,-1.9) node[below]{$\scriptstyle{(2,-19)}$};
\fill (2,{-1.9}) circle (0.5mm);
\end{tikzpicture}\end{bmlimage}
\end{center}
Observe que nesse caso, a identificação dos pontos em que o gráfico corta o
eixo $x$ é mais difícil (precisa resolver uma equação do terceiro grau).
\eqref{itestudfunceleme22}: $f$ decresce em $(-\infty,-1]$, cresce em
$[-1,\infty)$. Observe que $f$ não é derivável em $x=-1$.
\eqref{itestudfunceleme3}: Já encontramos o gráfico dessa função no Exercício
\ref{Ex:graficosbasicos}. Observe que
$f(x)=||x|-1|$ não é derivável em $x=-1,0,+1$, então é melhor estudar a variação
sem a derivada: $f$ é decrescente em $(-\infty,-1]$ e em $[0,1]$,
crescente em $[-1,0]$ e em $[1,\infty)$.
\eqref{itestudfunceleme4} Como $(\sen x)'=\cos x$, vemos que o seno é crescente
em cada intervalo em que o cosseno é positivo, e decrescente em cada intervalo
em que o cosseno é negativo. Por exemplo, no intervalo $[-\pisobredois,
\pisobredois]$, $\cos x>0$, logo $\sen x$ é crescente:
\begin{center}
\begin{bmlimage}\begin{tikzpicture}[scale=0.7]
\draw[thin,  ->] (-6.2,0)--(6.2,0);
\draw[thin,  ->] (0,-1.2)--(0,1.3);
\draw[color=gray, domain=-6:6, samples=50] plot (\x,{cos(\x r)});
\draw[thick, domain=-6:6, samples=50] plot (\x,{sin(\x r)});
\draw[dotted] (-1.57,-1.1)
node[below]{$\scriptstyle{-\tfrac{\pi}{2}}$}--(-1.57,1.1);
\draw[dotted]
(1.57,-1.1)node[below]{$\scriptstyle{\tfrac{\pi}{2}}$}--(1.57,1.1);
\end{tikzpicture}\end{bmlimage}
\end{center}
\eqref{itestudfunceleme5}:
$f(x)=\sqrt{x^2-1}$ tem domínio $(-\infty,-1]\cup[1,\infty)$, é sempre
não-negativa, e $f(-1)=f(1)=0$. Temos $f'(x)=\frac{x}{\sqrt{x^2-1}}$. Logo,
a variação de $f$ é dada por:
\begin{center}
\begin{bmlimage}\begin{tikzpicture}[scale=0.8]
\tkzTabInit[nocadre, espcl=2,  color, colorV=lightgray!5, colorL=gray!15,
colorC=gray!15]
{$x$ /.6, $f'(x)$ /.9, Variaç. de $f$ /1.5}%
{,$-1$, $+1$,}%
%\tkzTabLine{+,z,h,z,+}
\tkzTabLine{,-,t,h,t,+,}
\tkzTabVar{+/,-H/,-/,+/,}
%\tkzTabLine{,\searrow,\text{mín.},h,\text{mín.},\nearrow,}
\end{tikzpicture}\end{bmlimage}
\end{center}
Assim, o gráfico é do tipo:
\begin{center}
\begin{bmlimage}\begin{tikzpicture}
\newcommand{\func}[1]{sqrt((#1)^2-1)}
\draw[ ->] (-2.5,0)--(2.5,0);
\draw[ ->] (0,-0.2)--(0,1.5);
\draw[thick, domain=-2:-1] plot (\x,{\func{\x}});
\draw[thick, domain=1:2] plot (\x,{\func{\x}});
\foreach \k in {-1,+1} {
\draw (\k,0) node[below]{$\k$};
}
\end{tikzpicture}\end{bmlimage}
\end{center}
Observe que $\lim_{x\to -1^-}f'(x)=-\infty$, $\lim_{x\to +1^+}f'(x)=+\infty$
\eqref{itestudfunceleme5}:
Considere $f(x)=\frac{x+1}{x+2}$. Como
$\lim_{x\to \pm\infty}f(x)=1$, $y=1$ é assíntota horizontal, e como $\lim_{x\to
-2^-}f(x)=+\infty$, $\lim_{x\to -2^+}f(x)=-\infty$, $x=-2$ é assíntota vertical.
Como $f'(x)=\frac{1}{(x+2)^2}>0$ para todo $x\neq 2$, $f$ é crescente em
$(-\infty,-2)$ e em $(-2,\infty)$. Isso permite montar o gráfico:
\begin{center}
\begin{bmlimage}\begin{tikzpicture}[scale=0.7]
\draw[ ->] (-5,0)--(4,0);
\draw[dashed] (-2,-1)node[left]{$\scriptstyle{x=-2}$}--(-2,3);
\draw[dashed] (-5,1)--(4,1) node[above]{$\scriptstyle{y=1}$};
\draw[ ->] (0,-1)--(0,2.5);
\pgfmathsetmacro{\e}{0.5};
\draw[thick, domain=-5:{-2-\e}, samples=50] plot (\x,{(\x+1)/(\x+2)});
\draw[thick, domain={-2+\e}:4, samples=50] plot (\x,{(\x+1)/(\x+2)});
\end{tikzpicture}\end{bmlimage}
\end{center}
\eqref{itestudfunceleme61}: Um estudo parecido dá
\begin{center}
\begin{bmlimage}\begin{tikzpicture}[scale=0.7]
\draw[ ->] (-4,0)--(4,0);
\draw[dashed] (0.5,-2)node[right]{$\scriptstyle{x=\tfrac{1}{2}}$}--(0.5,1.5);
\draw[dashed] (-4,-0.5)--(4,-0.5) node[below]{$\scriptstyle{y=\tfrac{1}{2}}$};
\draw[ ->] (0,-2)--(0,1.5);
\pgfmathsetmacro{\e}{0.16};
\draw[thick, domain=-4:{0.5-\e}, samples=50] plot (\x,{(\x-1)/(1-2*\x)});
\draw[thick, domain={0.5+\e}:4, samples=50] plot (\x,{(\x-1)/(1-2*\x)});
\end{tikzpicture}\end{bmlimage}
\end{center}
\eqref{itestudfunceleme7}: Como $f'(x)=-xe^{-\frac{x^2}{2}}$,
$f$ é crescente em $(-\infty,0]$, decrescente em $[0,\infty)$.
Como $f(x)\to 0$ quando $x\to \pm \infty$, temos:
\begin{center}
\begin{bmlimage}\begin{tikzpicture}[scale=0.7]
\draw[ ->] (-4,0)--(4,0);
\draw[ ->] (0,-0.2)--(0,1.3);
\draw[thick, domain=-4:4, samples=50] plot (\x,{exp(-\x*\x*0.5)});
\end{tikzpicture}\end{bmlimage}
\end{center}
\eqref{itestudfunceleme9}: Observe que $\ln(x^2)$ tem domínio
$D=\bR\setminus\{0\}$, e $(\ln(x^2))'=\frac{2}{x}$. Logo, $\ln(x^2)$ é
decrescente em $(-\infty,0)$, crescente em $(0,\infty)$:
\begin{center}
\begin{bmlimage}\begin{tikzpicture}[scale=0.5]
\draw[ ->] (-4,0)--(4,0);
\draw[ ->] (0,-2.5)--(0,2);
\draw[thick, domain=0.3:4, samples=50] plot (\x,{2*ln(\x)});
\draw[thick, domain=0.3:4, samples=50] plot (-\x,{2*ln(\x)});
\end{tikzpicture}\end{bmlimage}
\end{center}
\eqref{itestudfunceleme10}
Lembre que o domínio da tangente é formado pela união dos intervalos da forma
$I_k=]-\pisobredois+k\pi,\pisobredois+k\pi[$.
Como $(\tan x)'=1+\tan^2x>0$ para todo $x\in I_k$, $\tan x$ é crescente em cada
intervalo do seu domínio (veja o esboço na Seção \ref{Sec:GraficosTrigo}).
\end{Solution}
\begin{Solution}{6.33}
  Em $t=0$, a partícula está na origem, onde ela fica até o instante
$t_1$. Durante $[t_1,t_2]$, ela anda em direção ao ponto $x=d_1$, com
velocidade constante $v=\frac{d_1}{t_2-t_1}$ e aceleração $a=0$. No tempo $t_2$
ela chega em $d_1$
e fica lá até o tempo $t_3$. No tempo $t_3$ ela começa a andar em direção ao
ponto $x=d_2$ (isto é, ela \emph{recua}), com velocidade constante
$v=\frac{d_2-d_1}{t_4-t_3}<0$. Quando chegar em $d_1$ no tempo $t_4$, para, fica
lá até $t_5$. No tempo $t_5$, começa a acelerar com uma aceleração $a>0$, até
o tempo $t_6$.
\end{Solution}
\begin{Solution}{6.34}
Como $v(t)=t-1$, temos $v(0)=-1<0$, $v(1)=0$, $v(2)=1>0$, $v(10)=9$.
Quando $t\to \infty$, $v(t)\to\infty$.
Observando a partícula, significa que no tempo $t=0$ ela está em
$x(0)=0$, recuando com uma velocidade de $-1$ metros por segundo. No instante
$t=1$, ela está com velocidade nula em $x(1)=-\frac12$. No instante $t=2$ ela
está de volta em $x(2)=0$, mas dessa vez com uma velocidade de $+1$ metro por
segundo.
A aceleração é \emph{constante}: $a(t)=v'(t)=+1$.
\end{Solution}
\begin{Solution}{6.35}
Temos $v(t)=x'(t)=A\omega \cos(\omega t)$, e $a(t)=v'(t)=-A\omega^2\sen (\omega
t)\equiv -\omega^2 x(t)$.
\begin{center}
\begin{bmlimage}\begin{tikzpicture}
\pgfmathsetmacro{\o}{1};
\pgfmathsetmacro{\A}{1};
\pgfmathsetmacro{\l}{12.7};
\pgfmathsetmacro{\omeg}{1};
\draw[ ->] (0,0)--(\l,0);
\draw[ ->] (0,-\A-0.2)--(0,\A+0.3);
\draw[thick, domain=0:\l-1.8, samples=80] plot (\x,{\A*sin(\omeg*\x r)})
node[right]{$x(t)$};
\draw[dashed, domain=0:\l-0.5, samples=80] plot (\x,{\A*\omeg*cos(\omeg*\x r)})
node[right]{$v(t)$};
\draw[dotted, domain=0:\l-2, samples=80] plot (\x,{-\A*\omeg^2*sin(\omeg*\x
r)})
node[right]{$a(t)$};
\foreach \k in {1,2,3} {
\draw ({\k*3.1414/\omeg},0) node{$\shortmid$} node[above]{$\frac{\k
\pi}{\omega}$};
}
\end{tikzpicture}\end{bmlimage}
\end{center}
Observe que $v(t)$ é máxima quando $x(t)=0$, e é mínima quando $x(t)=\pm A$.
Por sua vez, $a(t)$ é nula quando $x(t)=0$ e máxima quando $x(t)=\pm A$.
\end{Solution}
\begin{Solution}{6.36}
A taxa de variação no mês $t$ é dada por $P'(t)=2t+20$. Logo, hoje,
$P'(0)=+20$ hab./mês, o que significa que a população hoje cresce a medida de
$20$ habitantes por mês. Daqui a $15$ meses, $P'(15)=+50$ hab./mês. A
variação real da população durante o $16$-ésimo mês será $P(16)-P(15)=+51$
habitantes.
\end{Solution}
\begin{Solution}{6.37}
Como $V=L^3$, $V'=3L^2L'=\frac32 L^2$.
Logo, quando $L=10$, $V'=150$ $m^3/s$, e quando
$L=20$, $V'=600$ $m^3/s$.
\end{Solution}
\begin{Solution}{6.38}
O volume do balão no tempo $t$ é dado por $V(t)=\tfrac43 \pi R(t)^3$.
Logo, $R(t)=(\frac{3}{4\pi}V(t))^{1/3}$, e pela regra da cadeia,
$R'(t)=\tfrac13(\frac{3}{4\pi}V(t))^{-2/3}\frac{3}{4\pi}V'(t)$.
No instante $t_*$ que interessa, $V(t_*)=\frac{4\pi}{3}m^3$, e como
$V'(t)=2m^3/s$ para todo $t$, obtemos
$$
R'(t_*)=\tfrac13(\frac{3}{4\pi}\frac{4\pi}{3})^{-2/3}\frac{3}{4\pi}2\,m/s=\frac{
1}{2\pi}m/s\,.
$$
\end{Solution}
\begin{Solution}{6.39}
Seja $x$ a distância de $I$ até a parede, e $y$ a distância de $S$ até o chão:
$x^2+y^2=4$. Quando a vassoura começa a escorregar, $x$ e $y$ ambos se
tornam funções do tempo: $x=x(t)$ com $x'(t)=0.8\,m/s$, e $y=y(t)$. Derivando
implicitamente com respeito a $t$,
$2xx'+2yy'=0$. Portanto,
$y'=-\frac{xx'}{y}=-0.8\frac{x}{y}=-\frac{0.8x}{\sqrt{4-x^2}}$.
1) Quando $x=1\,m$, $y'=-0.46\,m/s$ (da onde vém esse sinal ``-''?)
2) Quando $x\to 2^-$, $y'\searrow -\infty$.
Obs: Quando $I$ estiver a $2-7.11\cdot 10^{-22}\,m$ da parede,
$S$ ultrapassa a velocidade da luz.
\end{Solution}
\begin{Solution}{6.40}
Definamos $\theta$ e $x$ da seguinte maneira:
\begin{center}
\begin{bmlimage}\begin{tikzpicture}
\draw (-5,0)--(5,0);
\pgfmathsetmacro{\teta}{60};
\pgfmathsetmacro{\h}{2};
\pgfmathsetmacro{\p}{-\h*tan(\teta)};
\fill (\p,0) circle (0.50mm);
\draw (\p,0) node[above]{$P$};
\draw[thick, ->] (\p,0)--(\p-0.4,0);
\draw[dotted] (0,0)--(0,\h) node[right]{$L$};
\draw[dashed] (\p,0)--(0,\h);
\draw[->] (0,\h-0.8) arc (270:270-\teta:0.8);
\draw (-0.5,1.05) node{$\theta$};
\draw[decorate, decoration=brace] (0,-0.2)--(\p,-0.2) node[midway, below]{$x$};
\pgfmathsetmacro{\e}{0.2};
\draw (0,0) node[above right]{$A$};
\pgfmathsetmacro{\f}{\e*sin(\teta)};
\pgfmathsetmacro{\g}{\e*cos(\teta)};
\draw[line width=4pt] (-\f,\h-\g)--(\f,\h+\g);
\end{tikzpicture}\end{bmlimage}
\end{center}
Temos $\tan \theta=\frac{x}{10}$ e como $\theta'=0.5$ rad/s, temos
$x'=10(1+\tan^2\theta)\theta'=5(1+\tan^2\theta)$.
1) Se $P=A$, então $\tan \theta=0$, logo $x'=5$ m/s. 2) Se $x=10\,m$, então
$\tan \theta=1$ e $x'=10\,m/s$.
3) Se  $x=100\,m$, então $\tan \theta=10$ e $x'=505\,m/s$ (mais rápido que a
velocidade do som, que fica em torno de $343\, m/s$).
\end{Solution}
\begin{Solution}{6.41}
Seja $H$ a altura do balão e $\theta$ o ângulo sob o qual o observador vê o
balão. Temos $H'=5$, e $\tan \theta=\frac{H}{50}$. Como ambos $H$ e
$\theta$ dependem do tempo, ao derivar com respeito a $t$ dá
$(1+\tan^2\theta)\theta'=\frac{H'}{50}=\frac{1}{10}$, isto é:
$\theta'=\frac{1}{10(1+\tan^2\theta)}$.
1) No instante em que o balão estiver a $30$ metros do chão, $\tan
\theta=\frac{30}{50}=\tfrac35$, assim $\theta'=\frac{5}{68}\simeq 0.0735$
rad/s.
2) No instante em que o balão estiver a $1000$ metros do chão, $\tan
\theta=\frac{1000}{50}=20$, assim $\theta'=\frac{1}{4010}\simeq 0.0025$ rad/s.
\end{Solution}
\begin{Solution}{6.42}
Como $P=\frac{nkT}{V}$, $P'=-\frac{nkT}{V^2}V'$. Logo,
no instante em que $V=V_0$,
$P'=-\frac{3nkT}{V_0^2}$.
\end{Solution}
\begin{Solution}{6.43}
\eqref{itexolinearizac1} $f(x)\simeq x+1$, $f(x)\simeq e^{-1}x+2e{^-1}$
\eqref{itexolinearizac2} $f(x)\simeq x$,
\eqref{itexolinearizac3} $f(x)\simeq -x$,
\eqref{itexolinearizac4} $f(x)\simeq 1$,
\eqref{itexolinearizac5} $f(x)\simeq x$, $f(x)\simeq 1$, $f(x)\simeq -x+\pi$
\eqref{itexolinearizac6} $f(x)\simeq 1+\frac{x}{2}$.
\end{Solution}
\begin{Solution}{6.44}
Como $\sqrt{4+x}\simeq 2+\frac{x}{4}$, temos $\sqrt{3.99}=\sqrt{4-0.01}\simeq
2+\frac{-0.01}{4}=1.9975$ (HP: $\sqrt{3.99}=1.997498...$).
Como $\ln (1+x)\simeq x$, temos $\ln(1.0123)=\ln(1+0.123)\simeq 0.123$ (HP:
$\ln(1.123)=0.1160...$).
Como $\sqrt{101}=10\sqrt{1+\frac{1}{100}}$ e que $\sqrt{1+x}\simeq
1+\frac{x}{2}$, temos $\sqrt{101}\simeq 10\cdot(1+\frac{1/100}{2})=10.05$ (HP:
$\sqrt{101}=10.04987...$).
\end{Solution}
\begin{Solution}{6.45}
\eqref{itderivimplicit1} $y'=\frac{3\cos(3x+y)}{1-\cos(3x+y)}$.
\eqref{itderivimplicit2} $y'=\frac{2xy^3+3x^2y^2}{1-3x^2y^2-2x^3y}$
\eqref{itderivimplicit3} Atenção: o único par $(x,y)$ solução da
equação $x=\sqrt{x^2+y^2}$ é $(0,0)$! Logo, não há jeito de escrever $y$ como
\emph{função} de $x$, assim não faz sentido derivar com respeito a $x$.
\eqref{itderivimplicit4} $y'=\frac{1-3x^2-4x-y}{3y^2+x+2}$
\eqref{itderivimplicit5} $y'=\frac{-\sen x-x\cos x}{\cos y-y\sen y}$
\eqref{itderivimplicit6} $y'=\frac{\cos y-\cos(x+y)}{x\sen y+\cos(x+y)}$
\end{Solution}
\begin{Solution}{6.46}
\eqref{itderivimplicitB1} Com $y'=1-\frac{2x}{3(y-x)^2}$,  $y=\frac56
x+\frac{13}{6}$.
\eqref{itderivimplicitB2} Com $y'=\frac{2-2xy}{x^2+4y^3}$, $y=\frac45
x+\frac95$.
\eqref{itderivimplicitB3} $y=-x+2$.
Obs: curvas definidas implicitamente por equações do tipo acima podem ser
representadas usando qualquer programa simples de esboço de funções, por exemplo
\verb|kmplot|.
\end{Solution}
\begin{Solution}{6.47}
\eqref{itfonctionsconvexes1} Queremos verificar que $\sqrt{\frac{x+y}{2}}\geq
\frac{\sqrt{x}+\sqrt{y}}{2}$ para todo $x,y\geq 2$.
Elevando ambos lados ao quadrado (essa operação é permitida, já que ambos
lados são positivos), $\frac{x+y}{2}\geq
(\frac{\sqrt{x}+\sqrt{y}}{2})^2
=\frac{x+2\sqrt{x}\sqrt{y}+y}{4}$, e rearranjando os termos obtemos $0\leq
\frac{(\sqrt{x}-\sqrt{y})^2}{4}$, que é sempre verdadeira.
\eqref{itfonctionsconvexes2}
Se $x,y>0$, $\frac{1}{\frac{x+y}{2}}\leq \frac{\frac{1}{x}+\frac{1}{y}}{2}$ é
equivalente a $4xy\leq (x+y)^2$, que por sua vez é equivalente a $0\leq
(x-y)^2$, que é sempre verdadeira. Logo, $\frac1x$ é convexa em $(0,\infty)$.
Como $\frac1x$ é ímpar, a concavidade em $(-\infty,0)$ segue imediatamente.
\end{Solution}
\begin{Solution}{6.48}
\eqref{itexconvexidadeA1}
$\frac{x^3}{3}-x$ é côncava em $(-\infty,0]$, convexa em $[0,\infty)$.
O gráfico se encontra na solução do Exercício \ref{Ex:variacoesbasicas}.
\eqref{itexconvexidadeA2} $-x^3+5x^2-6x$ é convexa em $(-\infty,\tfrac53]$,
côncava em $[\frac53,\infty)$:
\begin{center}
\begin{bmlimage}\begin{tikzpicture}[scale=0.5]
\draw[ ->] (-1,0)--(3.6,0);
\draw[ ->] (0,-2)--(0,1.7);
\newcommand{\funcao}[1]{-1*(#1)^3+5*(#1)^2-6*(#1)}
\draw[thick, domain=-0.1:3.4] plot (\x,{\funcao{\x}});
\pgfmathsetmacro{\a}{5/3};
\coordinate (I) at (\a,{\funcao{\a}});
\draw[dotted] (\a,0)node[above]{$\scriptstyle{\tfrac53}$}--(I);
\fill (I) circle (0.70mm);
\end{tikzpicture}\end{bmlimage}
\end{center}
\eqref{itexconvexidadeA3} Se $f(x)=3x^4-10x^3-12x^2+10x+9$, então
$f''(x)=12(3x^2-5x-2)$.
Logo, $f(x)$ é convexa em $(-\infty,-\tfrac13]$ e em $[2,\infty)$,
côncava em $[-\tfrac13,2]$.
\begin{center}
\begin{bmlimage}\begin{tikzpicture}[scale=0.7]
\draw[ ->] (-1,0)--(3.6,0);
\draw[ ->] (0,-1)--(0,1.7);
\newcommand{\funcao}[1]{(3*(#1)^4-10*(#1)^3-12*(#1)^2+10*(#1)+9)/100}
\draw[thick, domain=-2:4.5, samples=50] plot (\x,{\funcao{\x}});
\pgfmathsetmacro{\a}{-0.333};
\coordinate (I) at (\a,{\funcao{\a}});
\draw[dotted] (\a,0)node[below]{$\scriptstyle{-\tfrac13}$}--(I);
\fill (I) circle (0.70mm);
\pgfmathsetmacro{\b}{2};
\coordinate (J) at (\b,{\funcao{\b}});
\draw[dotted] (\b,0)node[above]{$\scriptstyle{2}$}--(J);
\fill (J) circle (0.70mm);
\end{tikzpicture}\end{bmlimage}
\end{center}
\eqref{itexconvexidadeA4} Como $(\frac{1}{x})''=\frac{2}{x^3}$, $\frac{1}{x}$ é
côncava em $(-\infty,0)$, convexa em $(0,\infty)$ (confere no gráfico do
Capítulo \ref{Cap:Funcoes}).
\eqref{itexconvexidadeA5}: Como $f''(x)=(x+2)e^x$, $f$ é côncava em
$(-\infty, -2]$, convexa em $[-2,\infty)$:
\begin{center}
\begin{bmlimage}\begin{tikzpicture}[scale=0.7]
\draw[ ->] (-4,0)--(2,0);
\draw[ ->] (0,-0.6)--(0,1.7);
\newcommand{\funcao}[1]{(#1)*exp(#1)}
\draw[thick, domain=-4:0.8, samples=50] plot (\x,{\funcao{\x}});
 \pgfmathsetmacro{\a}{-2};
 \coordinate (I) at (\a,{\funcao{\a}});
 \draw[dotted] (\a,0)node[above]{$\scriptstyle{-2}$}--(I);
 \fill (I) circle (0.70mm);
\end{tikzpicture}\end{bmlimage}
\end{center}
\eqref{itexconvexidadeA6}:
$f(x)=\frac{x^2+9}{(x-3)^2}$ é bem definida em $D=(-\infty,3)\cup
(3,+\infty)$. Como $f''(x)=\frac{12(x+6)}{(x-3)^4}$, $f(x)$ é côncava em
$(-\infty, -6]$, convexa em $(-6,3)$ e $(3,+\infty)$:
\begin{center}
\begin{bmlimage}\begin{tikzpicture}[scale=0.2]
 \newcommand{\funcao}[1]{( (#1)^2+ 9 )/( ( (#1) - 3)^2 )}
\draw[ ->] (-15,0)--(12,0);
\draw[ ->] (0,-0.6)--(0,12);
\draw[dashed] (3,0)node[below]{$\scriptstyle{3}$}--(3,12);
\draw[dashed] (-15,1)node[left]{$\scriptstyle{y=1}$}--(12,1);
\draw[thick, domain=-15:1.9, samples=50] plot (\x,{\funcao{\x}});
\draw[thick, domain=4.4:12, samples=50] plot (\x,{\funcao{\x}});
\pgfmathsetmacro{\a}{-6};
\coordinate (I) at (\a,{\funcao{\a}});
\draw[dotted] (\a,0)node[below]{$\scriptstyle{-6}$}--(I);
\fill (I) circle (3mm);
\end{tikzpicture}\end{bmlimage}
\end{center}
\eqref{itexconvexidadeA7} Com $f(x)=xe^{-3x}$ temos $f''(x)=(9x-6)e^{-3x}$.
Logo, $f$ é côncava em $(-\infty,\tfrac23]$, convexa em $[\tfrac23,\infty)$:
\begin{center}
\begin{bmlimage}\begin{tikzpicture}[scale=0.7]
 \newcommand{\funcao}[1]{ 5*(#1)*exp(-3*(#1))}
\draw[ ->] (-2,0)--(3,0);
\draw[ ->] (0,-0.6)--(0,2);
% \draw[dashed] (3,0)node[below]{$\scriptstyle{3}$}--(3,12);
% \draw[dashed] (-15,1)node[left]{$\scriptstyle{y=1}$}--(12,1);
\draw[thick, domain=-0.1:2, samples=50] plot (\x,{\funcao{\x}});
%\draw[thick, domain=4.4:12, samples=50] plot (\x,{\funcao{\x}});
\pgfmathsetmacro{\a}{0.66666};
\coordinate (I) at (\a,{\funcao{\a}});
\draw[dotted] (\a,0)node[below]{$\scriptstyle{\tfrac23}$}--(I);
\fill (I) circle (1mm);
\end{tikzpicture}\end{bmlimage}
\end{center}
\eqref{itexconvexidadeA10} $f(x)=|x|-x$ é $=0$ se $x\geq 0$, e $=-2x$ se
$x\leq 0$. Logo, $f$ é convexa. Obs: como $|x|$ não é derivável em $x=0$, a
convexidade não pode ser obtida com o Teorema \ref{Teo:Sinalfseconde}.
\eqref{itexconvexidadeA11} Se $f(x)=\arctan x$, então $f'(x)=\frac{1}{x^2+1}$,
e $f''(x)=\frac{-2x}{(x^2+1)^2}$. Logo, $\arctan x$ é convexa em $]-\infty,0]$,
côncava em $[0,\infty)$ (confere no gráfico da Seção
\ref{Sec:Functriginversas}).
\eqref{itexconvexidadeA12} $f(x)=e^{-\frac{x^2}{2}}$ tem
$f''(x)=(x^2-1)e^{-\frac{x^2}{2}}$. Logo, $f$ é convexa em $]-\infty,1]$ e
$[1,\infty)$, e côncava em $[-1,1]$ (veja o gráfico do Exercício
\ref{Ex:variacoesbasicas}).
\eqref{itexconvexidadeA13} $f(x)=\frac{1}{x^2+1}$ é convexa em
$(-\infty,-\frac{1}{\sqrt{3}}]$ e $[\frac{1}{\sqrt{3}},\infty)$, côncava em
$[-\frac{1}{\sqrt{3}},\frac{1}{\sqrt{3}}]$.
\begin{center}
\begin{bmlimage}\begin{tikzpicture}[scale=0.7]
 \newcommand{\funcao}[1]{ 1/( (#1)^2 +1)}
\draw[ ->] (-3,0)--(3,0);
\draw[ ->] (0,-0.6)--(0,1.3);
% \draw[dashed] (3,0)node[below]{$\scriptstyle{3}$}--(3,12);
% \draw[dashed] (-15,1)node[left]{$\scriptstyle{y=1}$}--(12,1);
\draw[thick, domain=-2.5:2.5, samples=50] plot (\x,{\funcao{\x}});
%\draw[thick, domain=4.4:12, samples=50] plot (\x,{\funcao{\x}});
\pgfmathsetmacro{\a}{0.5777};
\coordinate (I) at (\a,{\funcao{\a}});
\draw[dotted] (\a,0)node[below]{$\scriptstyle{\tfrac{1}{\sqrt{3}}}$}--(I);
\fill (I) circle (0.5mm);
\coordinate (J) at (-\a,{\funcao{-\a}});
\draw[dotted] (-\a,0)node[below]{$\scriptstyle{-\tfrac{1}{\sqrt{3}}}$}--(J);
\fill (J) circle (0.5mm);
\end{tikzpicture}\end{bmlimage}
\end{center}
\end{Solution}
\begin{Solution}{6.49}
Nos dois primeiros e último exemplos, as hipóteses do Teorema \ref{Teo:BH1} são
verificadas, dando
 $$\lim_{s\to 0}\frac{\log(1+s)}{e^{2s}-1}=
\frac{(\log(1+s))'|_{s=0}}{(e^{2s})'|_{s=0}}
=\frac{\frac{1}{1+s}|_{s=0}}{2e^{2s}|_{s=0}}=\frac{1}{2}$$
$$
\lim_{t\to \pi}\frac{\cos t+1}{\pi-t}=-(\cos t)'|_{t=\pi}=\sen t|_{t=0}=0\,.$$
$$
\lim_{x\to 0}\frac{\sen x}{x^2+3x}=\frac{(\sen
x)'|_{x=0}}{(x^2+3x)'|_{x=0}}
=\frac{\cos 0}{2\cdot 2+3}=\frac{1}{3}\,.
$$
No terceiro, o teorema não se aplica: apesar das funções $1-\cos(\alpha)$ e
$\sen(\alpha+\frac{\pi}{2})$ serem deriváveis em $\alpha=0$, temos
$\sen (0+\pi/2)=1\neq 0$. Logo o limite se calcula sem a regra de B.H.:
$\lim_{\alpha\to 0}\frac{1-\cos(\alpha)}{\sen (\alpha+\pi/2)}=\tfrac01=0$.
\end{Solution}
\begin{Solution}{6.50}
\eqref{itexBH1} $0$ (B.H. não se aplica)
\eqref{itexBH2} $\tfrac37$
\eqref{itexBH2b} $+\infty$ (B.H. não se aplica)
\eqref{itexBH3} $\lim_{x\to 0}\frac{(\sen x)^2}{x^2}=(\lim_{x\to 0}\frac{\sen
x}{x})^2=1^2=1$ (não precisa de B.H.)
\eqref{itexBH4} Usando B.H.,  $\lim_{x\to 0}\frac{\ln\frac{1}{1+x}}{\sen x}
=-\lim_{x\to 0}\frac{\ln(1+x)}{\sen x}=-\lim_{x\to 0}\frac{\frac{1}{x+1}}{\cos
x}=-1$.
\eqref{itexBH14} $1$
\eqref{itexBH7} $0$
\eqref{itexBH5} $0$
\eqref{itexBH10} $-\frac{1}{6}$
\eqref{itexBH9} $\tfrac13$
\eqref{itexBH922} $1$
\eqref{itexBH11} $2$
\eqref{itexBH12} $0$ (B.H. não se aplica)
\eqref{itexBH12aa} $0$
\eqref{itexBH12a} $0$ (aplicando duas vezes B.H.)
\eqref{itexBH12ab} $0$
\eqref{itexBH12b} Como $e^{\ln x}=x$, o limite é $1$ (B.H. se aplica mas não
serve para nada!)
\eqref{itexBH12c} Esse limite se calcula como no Capítulo \ref{Cap:Limites}:
$\lim_{x\to \infty}\frac{\sqrt{x+1}}{\sqrt{x-1}}=
\lim_{x\to \infty}\frac{\sqrt{x}\sqrt{1+\frac1x}}{\sqrt{x}\sqrt{1-\frac1x}}=
1$.
\eqref{itexBH12d} $-1/3$ (sem B.H.!)
\eqref{itexBH13} $2$
\eqref{itexBH15} $0$ (B.H. não se aplica)
\eqref{itexBH16} $\lim_{x\to \infty}\frac{x+\sen x}{x}=\lim_{x\to
\infty}(1+\frac{\sen x}{x})=1+0=1$ (Obs: Aqui B.H. não se aplica, porqué
$\lim_{x\to \infty}\frac{(x+\sen x)'}{(x)'}=\lim_{x\to
\infty}(1+\cos x)$, que  não existe.)
\eqref{itexBH6} $\tfrac13$
\eqref{itexBH17} $\lim_{x\to 0^+}\frac{x^2 \sen \frac{1}{x}}{x}=
\lim_{x\to 0^+} x\sen \frac{1}{x}=0$, com um ``sanduíche''. Aqui B.H. não se
aplica, porqué o limite $\lim_{x\to 0^+}(x^2 \sen \frac{1}{x})'$ não existe.
\eqref{itexBH18} $\frac13$. \eqref{itexBH20} (Segunda prova, Segundo semestre de
2011) Como $\lim_{y \to
\infty}\arctan y=\frac{\pi}{2}$, o limite
é da forma $\frac00$. As funções são deriváveis em $x>0$, logo pela
regra de B.H.,
$$
\lim_{x\to 0^+}\frac{\arctan(\frac1x)-\tfrac{\pi}{2}}{x}=
\lim_{x\to 0^+}\frac{\frac{1}{1+(\frac{1}{x})^2}(-\frac{1}{x^2})}{1}=
\lim_{x\to 0^+}\frac{-1}{1+x^2}=-1\,.
$$
\eqref{itexBHww8} $1/2$.
\end{Solution}
\begin{Solution}{6.51}
\eqref{itexBHB1} $\sqrt{e}$
\eqref{itexBHB2} $\lim_{x\to 0^+}x^x=\exp(\lim_{x\to 0^+}x\ln x)=e^0=1$.
\eqref{itexBHB3} $e^2$
\eqref{itexBHB5} $1$
\eqref{itexBHB58} $e$
\eqref{itexBHB6} $1$
\eqref{itexBHB7} $1$
\eqref{itexBHB75} $1$
\eqref{itexBHB8} $e^{-1}$
\eqref{itexBHB2bis} $0$
\eqref{itexBHB4} $-e/2$
\end{Solution}
\begin{Solution}{6.52}
 Para o primeiro,
\begin{align*}
 \lim_{z\to \infty}\bigl(\frac{z+9}{z-9}\bigr)^z&=\exp \Bigl(\lim_{z\to \infty}
z \ln \frac{z+9}{z-9}\Bigr)\\
&=\exp \Bigl(\lim_{z\to \infty} \frac{\ln (z+9)-\ln
(z-9)}{\frac{1}{z}}\Bigr)\text{ e as hipót. de BH satisfeitas, logo}\\
&=\exp \Bigl(\lim_{z\to \infty}
\frac{\frac{1}{z+9}-\frac{1}{z-9}}{\frac{-1}{z^2}}\Bigr)\\
&=\exp \Bigl(\lim_{z\to \infty} \frac{18 z^2}{z^2-81}\Bigr)\\
&=e^{18}\,.
\end{align*}
Para o segundo,
\begin{align*}
 \lim_{x\to \infty}x^{\ln x}e^{-x}&=\exp \Bigl(\lim_{x\to \infty} \big((\ln
x)^2-x \big)\Bigr)
=\exp \Bigl(\lim_{x\to \infty} x\big(\frac{(\ln x)^2}{x}-1
\big)\Bigr)
\end{align*}
Usando BH duas vezes, verifica-se que $\lim_{x\to \infty}\frac{(\ln
x)^2}{x}=0$,
o que implica $\lim_{x\to \infty} x(\frac{(\ln
x)^2}{x}-1)=-\infty$.
Logo, $\lim_{x\to \infty}x^{\ln x}e^{-x}=0$.
O último limite se calcula sem usar B.H.:
$$\lim_{x\to \infty}\frac{\sqrt{2x+1}}{\sqrt{x-1000}}=\sqrt{2}\lim_{x\to
\infty}\frac{\sqrt{1+\frac{1}{2x}}}{\sqrt{1-\frac{1000}{x}}}=\sqrt{2}\frac{1}{1}
=\sqrt{2}\,.$$
\end{Solution}
\protect \section *{Capítulo \ref {Cap:Extremos}}
\begin{Solution}{7.1}
\eqref{itminmaxbasico1} As hipóteses do teorema não são satisfeitas, pois o
domínio não é um intervalo finito e fechado. Mesmo assim, qualquer $x\in \bR$ é
ponto de máximo e mínimo global ao mesmo tempo.
\eqref{itminmaxbasico101} As hipóteses não são satisfeitas: o intervalo
não é limitado. Tém um
mínimo global em $x=1$, não tem máximo global.
\eqref{itminmaxbasico10} Hipóteses não satisfeitas (domínio não limitado).
Máximo global em $x=0$, não tem mínimo global.
\eqref{itminmaxbasico2} Hipóteses não satisfeitas (o intervalo não é fechado).
Tém mínimo global em $x=2$, não tem máximo global.
\eqref{itminmaxbasico3} Hipóteses satisfeitas: mínimo global em $x=2$, máximos
globais em $x=0$ e $x=2$.
\eqref{itminmaxbasico4} Hipóteses satisfeitas:
mínímos globais em $1,-1$ e $0$, máximos globais em
$-\tfrac32$ e $\tfrac32$.
\begin{center}
\begin{bmlimage}\begin{tikzpicture}
\pgfmathsetmacro{\a}{1.5};
\draw [thick, domain=-\a:\a, samples=150] plot (\x,{abs(1-\x^2)+abs(\x)-1});
\pgfmathsetmacro{\x}{0.6};
\draw [ ->] (-\a-0.2,0)--(\a+0.2,0);
\draw [ ->] (0,-0.2)--(0,2);
\foreach \k in {-1.5,1.5}{
\draw (\k,0) node{$\shortmid$};
\draw[dotted] (\k,0)--(\k,{abs(1-\k^2)+abs(\k)-1});
\fill (\k,{abs(1-\k^2)+abs(\k)-1}) circle (0.45mm);
}
\draw (-1.5,0) node[below]{$-\tfrac32$};
\draw (1.5,0) node[below]{$\tfrac32$};
\end{tikzpicture}\end{bmlimage}
\end{center}
\eqref{itminmaxbasico5} Hipóteses satisfeitas: mínimos globais em $x=-2$ e
$+1$, máximos globais em $x=-1$ e $+2$.
\eqref{itminmaxbasico6} Hipóteses satisfeitas: mínimo global em $x=+1$, máximo
global em $x=-1$.
\eqref{itminmaxbasico7} Hipóteses não satisfeitas ($f$ não é contínua). Não tem
máximo global, tem mínimos globais em $x=0$ e $+3$.
\eqref{itminmaxbasico8} Hipóteses satisfeitas: mínimo global em $x=0$, máximos
locais em $x=2$ e $4$.
\eqref{itminmaxbasico9} Hipóteses não satisfeitas ($f$ é contínua, mas o
domínio não é limitado). Tém mínimo global em $x=0$, não possui máximo global.
\eqref{itminmaxbasico11} Hipóteses não satisfeitas (intervalo não
limitado). No entanto, tem infinitos mínimos
globais, em todos os pontos da forma $x=-\pisobredois+k2\pi$, e infinitos
máximos globais, em todos os pontos da forma $x=\pisobredois+k2\pi$.
\end{Solution}
\begin{Solution}{7.2}
\eqref{itextremoslocais1} Máximo local no ponto $(-2,25)$, um mínimo local (e
global) em $(1,-2)$.
\eqref{itextremoslocais2} Sem mín./máx.
\eqref{itextremoslocais3} Mínimo local (e global) em
$(-1,-\frac{1}{12})$ (Atenção: a derivada é nula em $x=0$, mas não é nem
máximo nem mínimo pois a derivada não muda de sinal).
\eqref{itextremoslocais30} $f'(x)=-\frac{1-x^2}{x^2+x+1}$, tem um mínimo local
(em global) em $(1,f(1))$, um máximo local (e global) em $(-1,f(-1))$.
\eqref{itextremoslocais31} Máximo local (e global) em $(0,1)$.
\eqref{itextremoslocais32} Máximo local em $(1,e^{-1})$.
\eqref{itextremoslocais5} Mínimo local em $(-1,-\frac12)$, máximo local em
$(1,\frac12)$.
\eqref{itextremoslocais6} Mínimo local em $(e^{-1},e^{-1/e})$.
\eqref{itextremoslocais4} Máximo local em $(e^{-2}, 4e^{-2})$, mínimo local em
$(1,0)$.
\end{Solution}
\begin{Solution}{7.3}
$a=-b=3$.
\end{Solution}
\begin{Solution}{7.4}
\eqref{itLJ1} $r_0=\sigma$, \eqref{itLJ2} $r_*=\sqrt[6]{2}\sigma$.
Como $\lim_{r\to 0^+}V(r)=+\infty$, $V$ não possui máximo global.
$V$ decresce em $(0,r_*]$, cresce em $[r_*,\infty)$:
\begin{center}
\begin{bmlimage}\begin{tikzpicture}
\pgfmathsetmacro{\e}{1};
\pgfmathsetmacro{\s}{1};
\newcommand{\funcao}[1]{4*\e*(\s/(#1))^(12)-4*\e*(\s/(#1))^(6)}
\draw[->] (0,0)--(4.4,0)node[right]{$r$};
\draw[->] (0,-1)--(0,2)node[left]{$V(r)$};
\pgfmathsetmacro{\rzer}{\s};
\pgfmathsetmacro{\ret}{\s*1.122};
\draw[thick, domain=0.95:3.8, samples=50] plot (\x,{\funcao{\x}});
\draw[dotted] (\ret,0)node[above right]{$r_*$}--(\ret,{\funcao{\ret}});
\end{tikzpicture}\end{bmlimage}
\end{center}
Obs: O potencial de Lennard-Jones $V(r)$ descreve a energia de interação entre
dois átomos neutros a distância $r$.
Quando $0<r<r_0$ essa energia é positiva (os átomos se repelem), e quando
$r_0<r<\infty$ essa energia é negativa (os átomos se atraem).
Vemos que quando $r\to \infty$, a energia tende a zero e que ela tende a
$+\infty$ quando $r\to 0^+$: a distâncias longas, os átomos não interagem, e a
distâncias curtas a energia diverge (caroço duro).
A posição mais estável é quando a distância entre os dois átomos é
$r=r_*$.
\end{Solution}
\begin{Solution}{7.5}
 \eqref{itexoretanginscrito1}
A função área é dada por $A(x)=4x\sqrt{R^2-x^2}$, $x\in [0,R]$. O leitor pode
verificar que o seu máximo global em $[0,R]$ é atingido em
$x_*=\frac{R}{\sqrt{2}}$. Logo, o retângulo de maior área inscrito no círculo
tem largura $2x_*=\sqrt{2}R$, e altura $2\sqrt{R^2-x_*^2}=\sqrt{2}R$. Logo, é
um quadrado!
\eqref{itexoretanginscrito2} Usaremos a variável $h\in [0,4]$ definida da
seguinte maneira
\begin{center}
\begin{bmlimage}\begin{tikzpicture}[yscale=0.3]
\newcommand{\funcao}[1]{-2*(#1)+12}
\draw[ ->] (-0.2,0)--(6.5,0);
\draw[ ->] (0,-0.2)--(0,13);
\draw (-0.3,{\funcao{-0.3}})node[left]{$y=-2x+12$}--(7,{\funcao{7}});
\draw (-0.3,-0.3)--(5,5)node[right]{$y=x$};
\fill (4,4) circle (0.40mm);
\pgfmathsetmacro{\h}{2.2};
\draw[decorate, decoration=brace] (0,0)--(0,\h) node[midway, left]{$h$};
\draw[dotted] (0,\h)--(\h,\h);
\fill[color=gray!15] (\h,0) rectangle ({-\h/2+6},\h);
\draw[thick] (\h,0) rectangle ({-\h/2+6},\h);
\draw (\h,0) node[below]{$x_1$};
\draw ({-\h/2+6},0) node[below]{$x_2$};
\draw (4,4) node{$\bullet$};
% \fill (\h,\h) circle (0.50mm);
% \fill ({-\h/2+6},\h) circle (0.50mm);
\draw (4,4) node[above]{$(4,4)$};
\end{tikzpicture}\end{bmlimage}
\end{center}
A área do retângulo é dada por $A(h)=h(x_2-x_2)$. Ora, $x_1=h$ e
$x_2=6-\frac{h}{2}$. Logo, $x_2-x_1=6-\frac{3h}{2}$. Portanto,
queremos maximizar $A(h)=h(6-\frac{3h}{2})$ em
$h\in [0,4]$.
É fácil ver que o de máximo é atingido em $h_*=2$. Logo o maior retângulo tem
altura $h_*=2$, e largura $6-\frac{3h_*}{2}=3$.
\end{Solution}
\begin{Solution}{7.6}
A altura do triângulo de abertura
$\theta\in [0,\pi]$ é $\cos \frac{\theta}{2}$, a sua base é $2\sen
\frac{\theta}{2}$, logo a sua área é dada por
$$A(\theta)=\cos(\frac{\theta}{2})\sen (\frac{\theta}{2})=\frac12 \sen
\theta\,.\pt{3}$$
Queremos maximizar $A(\theta)$ quando $\theta\in [0,\pi]$.
Ora, $A(0)=A(\pi)=0$, e como $A'(\theta)=\frac12\cos \theta$,
$A'(\theta)=0$ se e somente se $\cos
\theta=0$, isto é, se e somente se $\theta=\frac{\pi}{2}$ $pt{1}$. Ora, como
$A'(\theta)>0$ se $\theta<\frac{\pi}{2}$,
$A'(\theta)<0$ se $\theta>\frac{\pi}{2}$, $\frac{\pi}{2}$ é um máximo de $A$
$\pt{2}$.
Logo, {o triângulo que tem maior área é aquele cuja abertura vale
$\frac{\pi}{2}$ $\pt{2}$.} Obs: pode também expressar a área em função do lado
horizontal $x$, $A(x)=\tfrac12 x\sqrt{1-(\tfrac{x}{2})^2}$.
Obs: Pode também introduzir a variável $h$, definida como
\begin{center}
\begin{bmlimage}\begin{tikzpicture}
\draw[thick] (0,0)--(1,0)node[midway,
below]{$\scriptstyle{1}$}--(1.5,0.866)node[midway,
below]{$\scriptstyle{1}$}--cycle;
\draw[dotted] (1.5,0)--(1.5,0.866) node[midway, right]{$h$};
\fill (1.5,0) circle (0.40mm);
%\draw (1,0.5) node{$\theta$};
\end{tikzpicture}\end{bmlimage}
\end{center}
e fica claro que o triângulo de maior área é aquele que tem maior altura $h$,
isto é, $h=1$ (aqui nem precisa calcular uma derivada...), o que acontece quando
a abertura vale $\frac{\pi}{2}$.
\end{Solution}
\begin{Solution}{7.7}
Seja $x$ o tamanho do lado horizontal do retângulo, e $y$ o seu lado vertical.
A área vale $A=xy$.
Como o perímetro é fixo e vale $2x+2y=L$, podemos expressar $y$ em função de
$x$, $y=\frac{L}{2}-x$, e expressar tudo em termos de $x$:
$A(x)=x(\frac{L}{2}-x)$. Maximizar essa função em $x\in [0,L/2]$ mostra que $A$
é máxima quando $x=x_*=\frac{L}{4}$. Como $y_*=\frac{L}{2}-x_*=\frac{L}{4}$, o
retângulo com maior área é um quadrado!
\end{Solution}
\begin{Solution}{7.8}
Suponha que a corda seja cortada em dois pedaços. Com o primeiro pedaço, de
tamanho $x\in [0,L]$, façamos um quadrado: cada um dos seus lados tem lado
$\frac{x}{4}$, e a sua área vale $(\frac{x}{4})^2$. Com o outro pedaço façamos
um círculo, de perímetro $L-x$, logo o seu raio é $\frac{L-x}{2\pi}$, e a sua
área $\pi(\frac{L-x}{2\pi})^2$. Portanto, queremos maximizar a função
$$
A(x)\pardef \frac{x^2}{16}+\frac{(L-x)^2}{4\pi}\,,\quad \text{ com }x\in
[0,L]\,.
$$
Na fronteira, $A(0)=\frac{L^2}{4\pi}$ (a corda inteira usada para fazer um
círculo), $A(L)=\frac{L^2}{16}$ (a corda inteira para fazer um quadrado).
Procuremos os pontos críticos de $A$: é fácil ver que $A'(x)=0$ se e somente
$x=x_*=\frac{L}{1+\frac{\pi }{4}}\in (0,L)$.
Como $A(x_*)=\frac{L^2}{4(4+\pi)}$, temos que $A(x_*)<A(L)<A(0)$. Logo,
a área total mínima é obtida fazendo um quadrado com o primeiro pedaço de
tamanho $x_*\simeq 0.56 L$, e  um círculo com o outro pedaço ($L-x_*\simeq
0.43 L$). A área total máxima é obtida usando a corda toda para fazer um
círculo.
\end{Solution}
\begin{Solution}{7.9}
$Q_*=(2,4)$
\end{Solution}
\begin{Solution}{7.10}
Seja $C=(x,0)$, com $1\leq x\leq 8$. É preciso minimizar
$f(x)=\sqrt{(x-1)^2+3^2}+\sqrt{(x-8)^2+4^2}$
para $x\in [1,8]$.
Os pontos críticos de $f$ são soluções de $7x^2+112x-560=0$ (em $[1,8]$), isto é,
$x=4$. Como $f''(4)>0$, $x=4$ é um mínimo de $f$ (pode verificar calculando os
valores $f(1)$, $f(8)$).
Logo, $C=(4,0)$ é tal que o perímetro de $ABC$ seja mínimo.
\end{Solution}
\begin{Solution}{7.11}
$\alpha=\pm 1$.
\end{Solution}
\begin{Solution}{7.12}
Considere a variável $x$ definida da seguinte maneira:
\begin{center}
\begin{bmlimage}\begin{tikzpicture}
\draw[ ->] (-0.2,0)--(4,0);
\draw[ ->] (0,-0.2)--(0,2);
\pgfmathsetmacro{\a}{1.5};
\pgfmathsetmacro{\b}{0.5};
\coordinate (P) at (\a,\b);
\pgfmathsetmacro{\d}{1.4};
\coordinate (Q) at (\a+\d,0);
\coordinate (Qp) at (0,{\b*(\d+\a)/\d});
\fill (P) circle (0.4mm);
\draw (P) node[above right]{$P=(a,b)$};
\fill (Q) circle (0.4mm);
\draw (Q) node[below]{$Q$};
\draw[dashed] (Q)--(Qp);
\draw[decorate, decoration=brace] (0,0)--(Qp) node[midway, left]{$h$};
\draw[decorate, decoration=brace] (\a,0)--(P) node[midway, left]{$b$};
\draw[decorate, decoration=brace] (\a,0)--(0,0) node[midway, below]{$a$};
\draw[decorate, decoration=brace] (Q)--(\a,0) node[midway, below]{$x$};
\end{tikzpicture}\end{bmlimage}
\end{center}
Assim temos que a área do triângulo em função de $x$, $A(x)$, é dada por
$A(x)=\half (a+x)\cdot h$. Mas, como $\frac{h}{a+x}=\frac{b}{x}$, temos
$h=\frac{b(x+a)}{x}$, que dá
$A(x)=\frac{b}{2}\frac{(x+a)^2}{x}$.
Procuremos o mínimo de $A(x)$ para $x\in (0,\infty)$.
Como $A$ é derivável em todo $x>0$, $A'(x)=\frac{b}{2}\frac{(x-a)(x+a)}{x^2}$,
vemos que $A$ possui dois pontos críticos, em $-a$ e $+a$, e $A'(x)>0$ se
$x<-a$, $A'(x)<0$ se $-a<x<a$, e $A'(x)>0$ se $x>a$. Desconsideremos o $-a$ pois
queremos um ponto em $(0,\infty)$. Assim, o mínimo de $A$ é atingido em $x=a$,
e nesse ponto $A(a)=2ab$:
\begin{center}
\begin{bmlimage}\begin{tikzpicture}
\pgfmathsetmacro{\a}{1.5};
\pgfmathsetmacro{\b}{0.5};
\draw[ ->] (0,0)--(4,0)node[right]{$x$};
\draw[ ->] (0,-0.2)--(0,2.2) node[left]{$A(x)$};
\draw[thick, domain=0.4:4] plot (\x,{(\b*(\x^2+2*\a*\x+\a^2))/(2*\x)});
\draw[dotted] (\a,0)node[below]{$a$}--(\a,{2*\a*\b})--(0,{2*\a*\b})
node[left]{$2ab$};
\fill (\a,{2*\a*\b}) circle (0.45mm);
\end{tikzpicture}\end{bmlimage}
\end{center}
\end{Solution}
\begin{Solution}{7.13}
 Representamos o triângulo da seguinte maneira:
\begin{center}
\begin{bmlimage}\begin{tikzpicture}
\pgfmathsetmacro{\r}{1};
\draw (0,0) circle(\r cm);
\draw[->] (-\r-0.2,0)--(\r+0.2,0);
\draw[->] (0,-\r-0.2)--(0,\r+0.2);
\coordinate (A) at (0,\r);
\coordinate (B) at (0.5*\r,-0.866*\r);
\coordinate (T) at (0,-0.866*\r);
\coordinate (C) at  (-0.5*\r,-0.866*\r);
\fill[color=gray!30, opacity=0.8] (A)--(B)--(C)--cycle;
\draw[thick] (A)--(B)--(C)--cycle;
\draw[decorate, decoration={brace, raise=1pt}] (B)--(T)
node[midway, below]{$x$};
\end{tikzpicture}\end{bmlimage}
\end{center}
Parametrizando o triângulo usando a variável $x$ acima (pode
também usar um ângulo),
obtemos a área como sendo a função
$A(x)=x(R+\sqrt{R^2-x^2})$, com $x\in [0,R]$.
Observe que não é necessário considerar os triângulos cuja
base fica acima do eixo $x$. (Por qué?)
Deixamos o leitor verificar que o máximo da função $A(x)$ é
atingido no ponto $x_*=\tfrac{\sqrt{3}}{2}R$, e que esse $x_*$
corresponde ao triângulo equilátero.
\end{Solution}
\begin{Solution}{7.14}
O único ponto crítico de $\sigma(x)$ é $x_*=\frac{x_1+\dots+ x_n}{n}$ (isto é,
a média aritmética). Como $\sigma''(x)=2n>0$, $x_*$ é mínimo global.
\end{Solution}
\begin{Solution}{7.15}
Seja $F$ a formiga, $S$ (respectivamente $I$) a extremidade superior
(respectivamente inferior) do telão, $\theta$ o ângulo $SFI$, e $x$ a distância
de $F$ à parede:
\begin{center}
\begin{bmlimage}\begin{tikzpicture}[yscale=0.3]
\draw (-1,0)--(0,0)--(0,10);
\coordinate (F) at (-7,0);
\coordinate (S) at (0,8);
\coordinate (I) at (0,3);
\draw[thick] (S)--(F)--(I);
\draw (F) node{$\bullet$} node[below]{$F$};
\draw (S) node[right]{$S$};
\draw (I) node[right]{$I$};
\draw (0,0) node[right]{$O$};
\draw[decorate, decoration=brace] (I)--(0,0) node[midway, right]{$3$};
\draw[decorate, decoration=brace] (S)--(I) node[midway, right]{$5$};
\draw[decorate, decoration=brace] (0,0)--(F) node[midway, below]{$x$};
\end{tikzpicture}\end{bmlimage}
\end{center}
Se $x$ é a distância de $F$ à parede, precisamos expressar $\theta$ em função
de $x$. Para começar, $\theta=\alpha-\beta$, em que $\alpha$ é o ângulo $SFO$,
e $\beta$ o ângulo $IFO$. Mas $\tan \alpha =\frac{8}{x}$ e $\tan
\beta=\frac{3}{x}$. Logo, precisamos achar o máximo da função
$$
\theta(x)=\arctan\tfrac{8}{x}-\arctan \tfrac{3}{x}\,,\quad \text{ com }x>0\,.
$$
Observe que $\lim_{x\to 0^+}\theta(x)=0$ (indo infinitamente perto da
parede, a formiga vê o telão sob um ângulo nulo) e $\lim_{x\to
\infty}\theta(x)=0$ (indo infinitamente longe da parede, a
formiga também vê o telão sob um ângulo nulo), é claro que deve existir (pelo
menos) um $0<x_*<\infty$ que maximize $\theta(x)$. Como $\theta$ é derivável,
procuremos os seus pontos críticos:
$$
\theta'(x)=\frac{1}{1+(\tfrac8x)^2}(\frac{-8}{x^2})
-\frac{1}{1+(\tfrac3x)^2}(\frac{-3}{x^2})=(\cdots)=\frac{120-5x^2}{
(x^2+8^2)(x^2+3^2)}\,.
$$
Logo o único ponto crítico de $\theta$ no intervalo $(0,\infty)$ é
$x_*=\sqrt{24}$. Vemos também que $\theta'(x)>0$ se $x<x_*$ e
$\theta'(x)<0$ se $x>x_*$, logo $x_*$ é o ponto onde $\theta$ atinge o seu
valor máximo.
Logo, para ver o telão sob um ângulo máximo, a formiga precisa ficar a uma
distância de $\sqrt{24}\simeq 4.9$ metros da parede.
\end{Solution}
\begin{Solution}{7.16}
Seja $R$ o raio da base do cone, $H$ a sua altura, $r$ o raio da base do
cilíndro e $h$ a sua altura.
Para o cilíndro ser inscrito, $\frac{h}{H}=\frac{R-r}{R}$ (para entender essa
relação, faça um desenho de um corte vertical).
Logo, expressando o volume do cilíndro em função de $r$, $V(r)=\frac{\pi
H}{R}r^2(R-r)$. É fácil ver que essa função possui um máximo local em $[0,R]$
atingido em $r_*=\frac{2}{3}R$. A altura do cilíndro correspondente é
$h_*=\frac{H}{3}$.
(Obs: pode também expressar $V$ em função de $h$: $V(h)=\pi
R^2h(1-\frac{h}{H})^2$.)
\end{Solution}
\begin{Solution}{7.17}
Seja $r$ o raio da base do cone, $h$ a sua altura.
O volume do cone é dado por $V=\tfrac13 \times \pi r^2\times h$. Como $h$ e
$r$ são ligados pela relação $(h-R)^2+r^2=R^2$, podemos expressar $V$ somente
em termos de $h$:
$$V(h)=\tfrac{\pi}{3}h(R^2-(h-R)^2)=\tfrac{\pi}{3}(2Rh^2-h^3)\,,$$
onde $h\in [0,2R]$.
Os valores na fronteira são $V(0)=0$, $V(2R)=0$.
Procurando os pontos críticos dentro do intervalo: $V'(h)=0$ se e somente se
$4Rh-3h^2=0$. Como $h=0$ não está \emph{dentro} do intervalo, somente
consideramos o ponto crítico $h_*=\tfrac{4}{3}R$. (Como $V''(h_*)<0$, é máximo
local.) Comparando $V(h_*)$ com os valores na fronteira, vemos que $h_*$ é
máximo global de $V$ em $[0,2R]$, e que tem dois mínimos globais, em $h=0$ e
$h=2R$.
{O maior cone, portanto, tem altura $\tfrac{4}{3}R$, e raio
$\sqrt{R^2-(\tfrac{4}{3}R-R)^2}=\frac{\sqrt{8}}{3}R$.}
\end{Solution}
\begin{Solution}{7.19}
Cada quadrado retirado deve ter os seus lados iguas a
$\tfrac12(1-\frac{1}{\sqrt{3}})$.
\end{Solution}
\begin{Solution}{7.20}
Como no exemplo anterior, $T(x)=\frac{\sqrt{x^2+h^2}}{v_1}+\frac{L-x}{v_2}$.
Procuremos o mínimo global de $T$ em $[0,L]$.
O ponto crítico $x_*$ é solução de
$\frac{x}{v_1\sqrt{x^2+h^2}}-\frac{1}{v_2}=0$. Isto é,
$x_*=\frac{h}{\sqrt{(v_2/v_1)^2-1}}$.
Se $v_1\geq v_2$, $T$ não tem ponto critico no intervalo, e  $T$ atinge o seu
mínimo global em $x=L$ (a melhor estratégia é de nadar diretamente até $B$). Se
$v_1<v_2$, e se $\frac{h}{\sqrt{(v_2/v_1)^2-1}}<L$, então $T$ tem um mínimo
global em $x_*$ (como $T''(x)=\frac{h^2}{v_1(x^2+h^2)}>0$
para todo $x$, $T$ é convexa, logo $x_*\in (0,L)$ é bem um ponto de
mínimo global).
Por outro lado, se $\frac{h}{\sqrt{(v_2/v_1)^2-1}}\geq L$, então $x_*$ não
pertence a $(0,L)$, e o mínimo global de $T$ é atingido em $x=L$.
\end{Solution}
\begin{Solution}{7.21}
Seja $O$ o centro da piscina. Uma estratégia que minimize o tempo de
viagem é de nadar em linha
reta de $A$ até um ponto $C$ na beirada tal que o ângulo $COB$ seja igual a
$\frac{\pi}{3}$ (ou $-\frac{\pi}{3}$). Depois, andar na beirada de $C$
até $B$.
\end{Solution}
\begin{Solution}{7.22}
%%%%%%%%%
A maior vara corresponde ao menor segmento que passa por $C$ e
encosta nas paredes em dois pontos $P$ e $Q$ (ver imagem
abaixo).
\begin{center}
\begin{bmlimage}\begin{tikzpicture}
\pgfmathsetmacro{\L}{1};
\pgfmathsetmacro{\M}{2};
\pgfmathsetmacro{\m}{2};
\pgfmathsetmacro{\l}{4};
\coordinate (A) at (0,{\M+\m});
\coordinate (B) at (\L,{\M+\m});
\coordinate (C) at (\L,{\M});
\coordinate (D) at ({\L+\l},\M);
\coordinate (E) at ({\L+\l},0);
\draw (A)--(0,0)--(E);
\draw (B)--(C)--(D);
\pgfmathsetmacro{\t}{30};
\coordinate (P) at (0,{\M+(\L*sin(\t)/cos(\t))});
\draw (P) node[left]{$P$};
\coordinate (Q) at ({\L+(\M*cos(\t)/sin(\t))},0);
\draw (Q) node[below]{$Q$};
\draw[thick] (P)--(C);
\draw[thick] (Q)--(C);
\draw (C) node[below left]{$C$};
\draw (D) node[right]{$D$};
%\coordinate (X) at (1.8,1.5);
%\coordinate (Y) at (4.2,0.5);
%\fill (X) circle (0.3mm);
%\fill (Y) circle (0.3mm);
%\draw[thick] (X)--(Y) node[midway, above]{$\ell$};
%\draw[dotted, <->] (A)--(B) node[midway, above]{$L$};
%\draw[dotted, <->] (D)--(E) node[midway, right]{$M$};
%\fill[color=gray!15] (0,0)--(A)--(B)--(C)--(D)--(E)--cycle;
\fill (P) circle (0.4mm);
\fill (Q) circle (0.4mm);
\fill (C) circle (0.4mm);
\end{tikzpicture}\end{bmlimage}
\end{center}
Seja $\theta$ o ângulo $QCD$. Quando $\theta$ é fixo, a
distância de $P$ a $Q$ vale
$$
f(\theta)=\frac{L}{\cos \theta}+\frac{M}{\sen \theta}\,.
$$
Precisamos minimizar $f$ no intervalo $(0,\pisobredois)$.
(Observe que $\lim_{\theta\to 0^+}f(\theta)=+\infty$,
$\lim_{\theta\to {\pisobredois}^-}f(\theta)=+\infty$.)
Resolvendo $f'(\theta)=0$, vemos que o único ponto crítico
$\theta_*$ satisfaz $\tan^3\theta_*=M/L$. É fácil verificar
que $f$ é convexa, logo $\theta_*$ é um ponto de mínimo global
de $f$.
Assim, o tamanho da maior vara possível é igual a
$$
f(\theta_*)=\cdots=L\bigl(1+(M/L)^{2/3}\bigr)^{3/2}\,.
$$
Observe que quando $L=M$, a maior vara tem tamanho
$2\sqrt{2}L$, e quando $M\to 0^+$, a maior vara tende a ter
tamanho igual a $L$.
\end{Solution}
\protect \section *{Capítulo \ref {Cap:Estudos}}
\begin{Solution}{8.1}
(Já vimos no Exemplo \ref{Ex:logsurx} que a afirmação vale para $p=1$, $q=1$.)
Observe que
$\frac{(\ln
x)^p}{x^q}=(\frac{(\ln
x)^{p/q}}{x})^q$. Logo, basta provar a afirmação para $q=1$ e $p>0$ qualquer:
$\lim_{x\to \infty}\frac{(\ln
x)^p}{x}=0$.
Mostremos por indução que se a afirmação vale para $p>0$
($\lim_{x\to \infty}\frac{(\ln
x)^{p}}{x}=0$), então ela vale para $p+1$. De fato, pela regra de B.H.,
$$
\lim_{x\to \infty}\frac{(\ln x)^{p+1}}{x}=\lim_{x\to \infty}\frac{(p+1)(\ln
x)^{p}\tfrac{1}{x}}{1}=
(p+1)\lim_{x\to \infty}\frac{(\ln
x)^{p}}{x}=0\,.
$$
Então, a afirmação estará provada para qualquer $p>0$ se ela for provada para
$0<p\leq 1$. Mas para tais $p$, $(\ln x)^p\leq \ln x$ para todo $x>1$, logo,
$$
\lim_{x\to \infty}\frac{(\ln x)^p}{x}\leq \lim_{x\to \infty}\frac{\ln x}{x}=0\,,
$$
pelo Exemplo \ref{Ex:logsurx}.
\end{Solution}
\begin{Solution}{8.3}
\eqref{itexBHlast1} $0$
\eqref{itexBHlast2} $0$
\eqref{itexBHlast3} $-\infty$
\eqref{itexBHlast4} $0$
\eqref{itexBHlast5_a} $0$
\eqref{itexBHlast5_b} $\infty$
\eqref{itexBHlast6} $0$
\eqref{itexBHlast24} $\infty$
\end{Solution}
\begin{Solution}{8.4}
\eqref{itasobl1} A função é a sua própria assíntota oblíqua.
\eqref{itasobl11} Não possui ass.
\eqref{itasobl6} $y=-2$ (vertical), $y=x-2$ em $\pm\infty$.
\eqref{itasobl12} Não possui ass.
\eqref{itasobl2} $y=0$ em $-\infty$, $y=x$ em $+\infty$.
\eqref{itasobl3} $y=x$ em $+\infty$.
\eqref{itasobl4} $y=x-\ln 2$ em $+\infty$, $y=-x-\ln 2$ em
$-\infty$.
\eqref{itasobl5} Não possui assíntotas: apesar de
$m=\lim_{x\to \infty}\frac{e^{\sqrt{\ln^2x+1}}}{x}$
existir e valer $1$,
$\lim_{x\to\infty}\{e^{\sqrt{\ln^2x+1}}-x\}=\infty$.
\end{Solution}
\begin{Solution}{8.5}
Em geral, náo.
Por exemplo, $f(x)=x+\tfrac{1}{x}\sen (x^2)$ possui $y=x$ como assíntota
oblíqua em $+\infty$,
mas $f'(x)=1-\frac{\sen x^2}{x^2}+2\cos (x^2)$
não possui limite quando $x\to\infty$.
Na verdade, uma função pode possuir uma assíntota (oblíqua ou
outra)
sem sequer ser derivável.
\end{Solution}
\begin{Solution}{8.6}
%%%%%%%%%%%%%%%%%%%%%%%%%%%%%%%%5
\eqref{itexoEstudA1}:
O domínio de $\bigl(\frac{x-1}{x}\bigr)^2$ é $D=\bR\setminus \{0\}$, o sinal é
sempre não-negativo, tem um zero
em $x=1$. $f$ não é par, nem ímpar.
Os limites relevantes são $\lim_{x\to 0^{\pm}}f(x)=+\infty$, logo $x=0$ é
assíntota vertical, e
$$\lim_{x\to \pm\infty}\bigl(\frac{x-1}{x}\bigr)^2=\Bigl(\lim_{x\to \pm
\infty}\frac{x-1}{x}\Bigr)^2==\Bigl(\lim_{x\to \pm
\infty}\bigl(1-\frac{1}{x}\bigr)\Bigr)^2=1^2=1\,.$$
Logo, $y=1$ é assíntota horizontal.
$f$ é derivável em $D$, e $f'(x)=\frac{2(x-1)}{x^3}$.
\begin{center}
\begin{bmlimage}\begin{tikzpicture}
\tkzTabInit[nocadre,espcl=2,  color, colorV=lightgray!5, colorL=gray!15,
colorC=gray!15]
{$x$ /.6,  $f'(x)$ /.6, Var. de $f$ /1.3}%
{,$0$, $1$,}%
%\tkzTabLine{,+,z,+,,+,}
\tkzTabLine{,+,d,-,z,+,}
\tkzTabVar{-/,+D+/$+\infty$/$+\infty$,-/mín,+/,}
%\tkzTabLine{,\searrow,\text{mín.},h,\text{mín.},\nearrow,}
\end{tikzpicture}\end{bmlimage}
\end{center}
$f$ possui um mínimo global em $(1,0)$.
A segunda derivada é dada por $f''(x)=\frac{2(3-2x)}{x^4}$. Ela se anula em
$x=\tfrac32$, e muda de sinal neste ponto:
\begin{center}
\begin{bmlimage}\begin{tikzpicture}
\tkzTabInit[nocadre,espcl=2,  color, colorV=lightgray!5, colorL=gray!15,
colorC=gray!15]
{$x$ /.6,  $f''(x)$ /.7, Conv. de $f$ /1.2}%
{,$0$, $\tfrac32$,}%
\tkzTabLine{,+,d,+,z,-,}%
\tkzTabLine{,\smile,d,\smile,z,\frown,}%
\end{tikzpicture}\end{bmlimage}
\end{center}
Logo, $f$ é convexa em $(-\infty,0)$ e $(0,\frac32)$,  côncava em
$(\frac32,\infty)$, e possui um ponto de inflexão em
$(\tfrac{3}{2},f(\tfrac{3}{2}))=(\tfrac{3}{2},\tfrac19)$.
\begin{center}
\begin{bmlimage}\begin{tikzpicture}
\draw [thick, domain=-4:-1.2, samples=100] plot
(\x,{((\x)-1)^2/((\x)^2)});
\draw [thick, domain=0.4:4, samples=100] plot (\x,{(\x-1)^2/((\x)^2)});
\draw [ ->] (-4,0)--(4,0) node[right] {$x$};
\draw [ ->] (0,-0.1)--(0,3) node[left] {$f(x)$};
\draw [dotted] (-4,1)--(4,1) node[above] {$y=1$};
\draw [dotted] (0,0)--(0,3.5) node[right] {$x=0$};
\fill (1,0) circle (0.35mm);
\draw (1,0) node[below] {$(1,0)$};
\fill (1.5,0.1111) circle (0.35mm);
\draw [ <-] (1.52,0.0911)--(2,-0.3) node[right]
{$(\tfrac{3}{2},\tfrac{1}{9})$};
\end{tikzpicture}\end{bmlimage}
\end{center}
\eqref{itexoEstudA3}:
O domínio de  $f(x)=x(\ln x)^2$ é
$D=(0,+\infty)$, e o seu sinal é: $f(x)\geq 0$ para todo $x\in D$.
A função não é { par} nem { ímpar}.
Como $\lim_{x\to \infty}f(x)=+\infty$, não tem assintota horizontal.
Para ver se tem assíntota vertical em $x=0$, calculemos
$\lim_{x\to 0^+}f(x)=\lim_{x\to 0^+}\frac{(\ln x)^2}{1/x}$. Como ambas funções
$(\ln x)^2$ e $1/x$ são deriváveis em $(0,1)$ e tendem a $+\infty$ quando $x\to
0^+$, apliquemos a regra de B.H.:
$$
\lim_{x\to 0^+}\frac{(\ln x)^2}{1/x}=
\lim_{x\to 0^+}\frac{2(\ln x)1/x}{-1/x^2}=
-2\lim_{x\to 0^+}x\ln x\,.
$$
Usando a regra de B.H. de novo, pode ser mostrado que esse segundo limite é
zero (ver Exemplo \ref{Ex:xlogxemzero}). Logo, $\lim_{x\to 0^+}f(x)=0$: não
tem assíntota vertical em $x=0$.
A derivada é dada por $f'(x)=\ln x(\ln x+2)$.
\begin{center}
\begin{bmlimage}\begin{tikzpicture}[scale=0.8]
\tkzTabInit[nocadre, espcl=2,  color, colorV=lightgray!5, colorL=gray!15,
colorC=gray!15]
{$x$ /.6, $f'(x)$ /.6, Variaç. de $f$ /1.2}%
{,$e^{-2}$, $1$, }%
\tkzTabLine{,+,z,-,z,+}
\tkzTabVar{-/,+/{máx.},-/{mín.},+/}
%\tkzTabLine{,\searrow,\text{mín.},h,\text{mín.},\nearrow,}
\end{tikzpicture}\end{bmlimage}
\end{center}
O máximo local está em
$(e^{-2},f(e^{-2}))=(e^{-2},4e^{- 2})$, e o
mínimo global em $(1,f(1))=(1,0)$.
A {segunda derivada} de $f$ é dada por
$f''(x)=\frac{2(\ln x+1)}{x}$.
\begin{center}
\begin{bmlimage}\begin{tikzpicture}[scale=0.8]
\tkzTabInit[nocadre, espcl=2,  color, colorV=lightgray!5, colorL=gray!15,
colorC=gray!15]
{$x$ /.6, $f''(x)$ /.6, Conv. de $f$ /1.2}%
{,$e^{-1}$, }%
\tkzTabLine{,-,z,+,}
\tkzTabLine{,\frown,,\smile,}
\end{tikzpicture}\end{bmlimage}
\end{center}
Logo, $f$ é côncava em $(0,e^{-1})$, possui um ponto de inflexão em
$(e^{-1},f(e^{-1}))=(e^{-1},e^{-1})$, e é convexa em $(e^{-1},+\infty)$.
\begin{center}
\begin{bmlimage}\begin{tikzpicture}[scale=1.3]
\draw [thick, domain=0.001:2.5, samples=100] plot (\x,{(\x)*(ln(\x))^2});
 \draw [ ->] (0,0)--(2.5,0) node[right] {$x$};
 \draw [ ->] (0,-0.1)--(0,2);
% \draw [dotted] (-4,1)--(4,1) node[above left] {Assíntota horiz.: $y=1$};
 \fill (1,0) circle (0.35mm);
 \draw (1,0) node[below] {$\scriptscriptstyle{(1,0)}$};
 \fill (0.367,0.367) circle (0.35mm);
 \draw[<-] (0.39,0.39)--(0.9,0.5) node[above]
{$\scriptscriptstyle{(e^{-1},e^{-1})}$};
 \fill (0.1353,0.541) circle (0.35mm);
 \draw[<-] (0.14,0.58)--(0.9,1.5) node[above]
{$\scriptscriptstyle{(e^{-2},4e^{-2})}$};
\end{tikzpicture}\end{bmlimage}
\end{center}
Podemos também notar que $\lim_{x\to 0^+}f'(x)=+\infty$.
\end{Solution}
\begin{Solution}{8.7}
$D=\bR\backslash \{\pm 4\}$. Os zeros de $f(x)\pardef\frac{x^2-4}{x^2-16}$ são
$x=-2$, $x=+2$, e o seu sinal:
\begin{center}
\begin{bmlimage}\begin{tikzpicture}
\tkzTabInit[lgt=3, nocadre, espcl=2]
{ /.6,  $x^2-4$ /.6, $x^2-16$ /.6, $f(x)$ /.8}%
{,$-4$, $-2$, $2$, $4$,}%
%\tkzTabLine{,+,z,+,,+,}
\tkzTabLine{,+,,+,z,-,z,+,,+,}
\tkzTabLine{,+,z,-,,-,,-,z,+,}
\tkzTabLine{,+,d,-,z,+,z,-,d,+,}
%\tkzTabLine{,+,z,-,z,+,}
%\tkzTabVar{-/,+/\text{a.v.},-/$0$,+/,}
%\tkzTabLine{,\searrow,\text{mín.},h,\text{mín.},\nearrow,}
\end{tikzpicture}\end{bmlimage}
\end{center}
Como
$$
\lim_{x\to \pm\infty}f(x)=\lim_{x\to
\pm \infty}\frac{1-\frac{4}{x^2}}{1-\frac{16}{x^2}}
=1\,,$$
a reta $y=1$ é assíntota horizontal.
Como
$$
\lim_{x\to -4^\pm}f(x)=\mp \infty\,,\quad \lim_{x\to +4^\pm}f(x)=\pm \infty\,,$$
as retas $x=-4$ e $x=+4$ são assíntotas verticais.
A primeira derivada se calcula facilmente: $f'(x)=\frac{-24 x }{(x^2-16)^2}$,
logo a variação de $f$ é dada por:
\begin{center}
\begin{bmlimage}\begin{tikzpicture}[scale=0.8]
\tkzTabInit[nocadre, espcl=2,  color, colorV=lightgray!5, colorL=gray!15,
colorC=gray!15]
{$x$ /.6, $f'(x)$ /.6, Variaç. de $f$ /1.2}%
{,$-4$,$0$,$4$, }%
\tkzTabLine{,+,d,+,z,-,d,-}
\tkzTabVar{-/,+D-/{},+/{máx.},-D+/{},-/}
%\tkzTabLine{,\searrow,\text{mín.},h,\text{mín.},\nearrow,}
\end{tikzpicture}\end{bmlimage}
\end{center}
A posição do máximo local é: $(0,f(0))=(0,\tfrac14)$.
O gráfico:
\begin{center}
\begin{bmlimage}\begin{tikzpicture}[scale=0.7]
\pgfmathsetmacro{\a}{10}
\pgfmathsetmacro{\b}{4}
\newcommand{\funcao}[1]{ ( (#1)^2-4 )/( (#1)^2-16) }
\draw[->] (-\a,0)--(\a,0);
\draw[->] (0,-\b)--(0,\b);
\draw[thick, domain=-\a:-4.5, samples=50] plot (\x,{\funcao{\x}});
\draw[thick, domain=-3.5:3.5, samples=50] plot (\x,{\funcao{\x}});
\draw[thick, domain=4.5:\a, samples=50] plot (\x,{\funcao{\x}});
\draw[dashed] (-\a,1)node[below]{$y=1$}--(\a,1);
\draw[dashed] (-4,-\b)node[left]{$x=-4$}--(-4,\b);
\draw[dashed] (4,-\b)node[right]{$x=+4$}--(4,\b);
\draw[<-] (0.1,0.3)--(2,2)node[above]{máx.: $(0,\frac14)$};
\draw (-2,0) node{$\shortmid$} node[above]{$-2$};]
\draw (2,0) node{$\shortmid$} node[above]{$+2$};]
\end{tikzpicture}\end{bmlimage}
\end{center}
A segunda derivada: $f''(x)=24\frac{16+3x^2}{(x^2-16)^3}$, e a convexidade é
dada por
\begin{center}
\begin{bmlimage}\begin{tikzpicture}[scale=0.8]
\tkzTabInit[nocadre, espcl=2,  color, colorV=lightgray!5, colorL=gray!15,
colorC=gray!15]
{$x$ /.6, $f''(x)$ /.6, Conv. $f$ /1.2}%
{,$-4$,$4$, }%
\tkzTabLine{,+,d,-,d,+}
\tkzTabLine{,\smile,d, \frown,d,\smile,}
%\tkzTabLine{,\searrow,\text{mín.},h,\text{mín.},\nearrow,}
\end{tikzpicture}\end{bmlimage}
\end{center}
\end{Solution}
\begin{Solution}{8.8}
%%%%%%%%%%%%%%%%%%
% \eqref{itEstBas9a}
%  { Domínio}: $D=\R\backslash \{-1\}$. { Sinal}:
% $f(x)\geq 0$ para todo $x\in D$, e $f(x)=0$ se e somente se $x=0$.
% Assíntotas:
% como $\lim_{x\to -1^-}f(x)=\lim_{x\to -1^+}f(x)=+\infty$, { a reta $x=-1$ é
% assíntota vertical}
% (é a única). Como
% $$\lim_{x\to \pm\infty}\frac{x^2}{(x+1)^2}=\lim_{x\to
% \pm\infty}\frac{1}{(1+\frac{1}{x})^2}=1\,,$$
%  { a reta $y=1$ é assíntota
% horizontal} (a direita e a esquerda). Como
% $f'(x)=\frac{2x}{(x+1)^3}$,
% \begin{center}
% \begin{bmlimage}\begin{tikzpicture}[scale=0.8]
% \tkzTabInit[nocadre, espcl=2,  color, colorV=lightgray!5, colorL=gray!15,
% colorC=gray!15]
% {$x$ /.6, $f'(x)$ /.6, Variaç. de $f$ /1.2}%
% {,$-1$, $0$, }%
% \tkzTabLine{,+,d,-,z,+}
% \tkzTabVar{-/,+D+/,-/mín.,+/}
% %\tkzTabLine{,\searrow,\text{mín.},h,\text{mín.},\nearrow,}
% \end{tikzpicture}\end{bmlimage}
% \end{center}
% $f$ possui um mínimo local no ponto $(0,f(0))=(0,0)$.
% Como $f''(x)=\frac{2(1-2x)}{(x+1)^4}$, temos:
% \begin{center}
% \begin{bmlimage}\begin{tikzpicture}[scale=0.8]
% \tkzTabInit[nocadre, espcl=2,  color, colorV=lightgray!5, colorL=gray!15,
% colorC=gray!15]
% {$x$ /.6, $f''(x)$ /.6, Conv. de $f$ /1.2}%
% {,$-1$, $\tfrac12$, }%
% \tkzTabLine{,+,d,+,z,-}
% \tkzTabLine{,\smile,d,\smile,z,\frown}
% %\tkzTabVar{-/,+D+/${+\infty}$/${+\infty}$,-/mín.,+/}
% %\tkzTabLine{,\searrow,\text{mín.},h,\text{mín.},\nearrow,}
% \end{tikzpicture}\end{bmlimage}
% \end{center}
% Logo, $f$ é convexa nos intervalos $]-\infty,-1[$ e $]-1,\tfrac12[$, possui um
% ponto de inflexão em $(\tfrac12,f(\tfrac12))=(\tfrac12,\tfrac19)$, e é côncava
% em $(\tfrac12,+\infty)$. Gráfico:
% \begin{center}
% \begin{bmlimage}\begin{tikzpicture}
% \draw[ ->] (-4,0)--(4,0);
% \draw[dashed] (-5,1)node[below]{$y=1$}--(4,1);
% \draw[ ->] (0,-0.5)--(0,4);
% \draw[dashed] (-1,-0.5)node[below]{$x=-1$}--(-1,4);
% \draw[thick, domain=-5:-1.9, samples=50] plot (\x,{\x^2/(\x+1)^2});
% \draw[thick, domain=-0.65:4, samples=100] plot (\x,{\x^2/(\x+1)^2});
% \end{tikzpicture}\end{bmlimage}
% \end{center}
% Observe que esse gráfico é o gráfico da função $(\frac{x}{x-1})$
OBS: Para as demais funções, colocamos somente um \emph{resumo} das
soluções, na forma de um gráfico no qual o leitor pode verificar os resultados
do seu estudo.

\eqref{itEstBas1} Ass. vert.: $x=0$. Ass. oblíqua: $y=x$.
\begin{center}
\begin{bmlimage}\begin{tikzpicture}[yscale=0.7]
\draw [thick, domain=-3:-0.3, samples=100] plot (\x,{(\x)+1/(\x)});
\draw [thick, domain=0.3:3, samples=100] plot (\x,{(\x)+1/(\x)});
 \draw [ ->] (-3,0)--(3,0) node[right] {$x$};
 \draw [ ->] (0,-3)--(0,3) node[left]{$x+\tfrac{1}{x}$};
 \fill (1,2) circle (0.45mm);
  \draw (1,2) node[below] {$\scriptscriptstyle{(1,2)}$};
  \fill (-1,-2) circle (0.45mm);
  \draw (-1,-2) node[above] {$\scriptscriptstyle{(-1,-2)}$};
\end{tikzpicture}\end{bmlimage}
\end{center}


\eqref{itEstBas6}
Ass. vert.: $x=0$. Ass. obl.: $y=x$.
\begin{center}
\begin{bmlimage}\begin{tikzpicture}[yscale=0.7]
\draw [thick, domain=-3:-0.5, samples=100] plot
(\x,{\x+1/((\x)^2)});
\draw [thick, domain=0.6:3, samples=100] plot
(\x,{\x+1/((\x)^2)})node[right]{$x+\tfrac{1}{x^2}$};
 \draw [ ->] (-3,0)--(3,0) node[right] {$x$};
 \draw [ ->] (0,-3)--(0,3);
 \fill (1.256,1.88) circle (0.45mm);
  \draw (1.256,1.88) node[below]
{$\scriptscriptstyle{(2^{1/3},2^{1/3}+2^{-2/3})}$};
\draw (-1,0) node{$\shortmid$} node[above left]{$-1$};
\end{tikzpicture}\end{bmlimage}
\end{center}

\eqref{itEstBas9}
\begin{center}
\begin{bmlimage}\begin{tikzpicture}
\draw [thick, domain=-3:3, samples=100] plot
(\x,{1/((\x)^2+1)});
\draw [ ->] (-3,0)--(3,0);
\draw [ ->] (0,-0.5)--(0,1.5)node[right]{$\tfrac{1}{x^2+1}$};
\fill (0.577,0.75) circle (0.45mm);
\draw[<-] (0.6,0.8)--(1.3,1)
node[right]{inflex: $(\tfrac{1}{\sqrt{3}},\tfrac34)$};
\fill (-0.577,0.75) circle (0.45mm);
\draw[<-] (-0.6,0.8)--(-1.3,1)
node[left]{inflex: $(-\tfrac{1}{\sqrt{3}},\tfrac34)$};
\draw (-5,2) node[left]{$\displaystyle{f'(x)=\frac{-2x}{(x^2+1)^2}}$};
\draw (-5,0.5) node[left]{$\displaystyle{f''(x)=\frac{2(3x^2-1}{(x^2+1)^3}}$};
\end{tikzpicture}\end{bmlimage}
\end{center}



\eqref{itEstBas2}
\begin{center}
\begin{bmlimage}\begin{tikzpicture}[yscale=0.7]
\draw [ ->] (-3,0)--(3,0) node[right] {$x$};
\draw [ ->] (0,-3)--(0,3) node[left]{$\frac{x}{x^2-1}$};
\draw [thick, domain=-3:-1.2, samples=100] plot (\x,{\x/((\x)^2-1)});
\draw[dashed] (-1,-2.5)--(-1,2.5) node[below left]{$\scriptscriptstyle{x=-1}$};
\draw [thick, domain=-0.8:0.8, samples=100] plot (\x,{\x/((\x)^2-1)});
\draw [thick, domain=1.2:3, samples=100] plot (\x,{\x/((\x)^2-1)});
\draw[dashed] (1,-2.5)node[right]{$\scriptscriptstyle{x=1}$}--(1,2.5) ;
\draw[<-] (0.3,-0.1)--(1.5,-1)
node[right]{pt. inflex.: $\scriptstyle{(0,0)}$};
\draw (4,2) node[right]{$\displaystyle{f'(x)=\frac{-(1+x^2)}{(x^2-1)^2}}$};
\draw (4,0.5)
node[right]{$\displaystyle{f''(x)=\frac{-2x(3x^2+1)}{(x^2-1)^3}}$};
\end{tikzpicture}\end{bmlimage}
\end{center}

\eqref{itEstBas3}
\begin{center}
\begin{bmlimage}\begin{tikzpicture}
\draw [ ->] (-3,0)--(3,0) node[right] {$x$};
\draw [ ->] (0,-1)--(0,1) node[above]{$xe^{-x^2}$};
\draw [thick, domain=-2.5:2.5, samples=100] plot (\x,{\x*exp(-(\x)^2)});
\fill (0.707,0.428) circle (0.45mm);
  \draw[<-] (0.71,0.44)-- (0.9,1) node[right]
{$\scriptstyle{(\tfrac{1}{\sqrt{2}},\tfrac{1}{\sqrt{2}}e^{-\tfrac12})}$};
\fill (-0.707,-0.428) circle (0.45mm);
  \draw[<-] (-0.71,-0.5)-- (-0.9,-1) node[left]
{$\scriptstyle{(-\tfrac{1}{\sqrt{2}},-\tfrac{1}{\sqrt{2}}e^{-\tfrac12})}$};
\draw[<-] (0.1,-0.1)--(0.5,-1.3) node[right]{pt. inflex. $\scriptstyle{(0,0)}$};
\fill (1.225,0.273) circle (0.40mm);
\fill (-1.225,-0.273) circle (0.40mm);
\draw[<-] (1.225,0.24)--(1.5,-0.6)
node[right]{pt. inflex.: $\scriptstyle{(\sqrt{3/2},f(\sqrt{3/2}))}$};
\draw[<-] (-1.225,-0.24)--(-1.5,0.6)
node[left]{pt. inflex.: $\scriptstyle{(-\sqrt{3/2},f(\sqrt{3/2}))}$};
\draw (4,1.2) node[right]{$\displaystyle{f'(x)=(1-2x^2)e^{-x^2}}$};
\draw (4,0.5)
node[right]{$\displaystyle{f''(x)=-2x(3-2x^2)e^{-x^2}}$};
\end{tikzpicture}\end{bmlimage}
\end{center}

\eqref{itEstBas7}, \eqref{itEstBas8},
\eqref{itEstBas8t}:
\begin{center}
\begin{bmlimage}\begin{tikzpicture}[scale=0.5]
\draw [ ->] (0,-0.1)--(0,3);
\pgfmathsetmacro{\a}{2};
\draw [ ->] (-\a,0)--(\a,0);
\draw [thick, domain=-\a:\a, samples=100] plot (\x,{(exp(\x)+exp(-\x))/2})
node[right]{$\cosh x$};

\begin{scope}[xshift=9cm, yshift=1cm]
\draw [ ->] (0,-2)--(0,2);
\pgfmathsetmacro{\a}{1.6};
\draw [ ->] (-\a,0)--(\a,0);
\draw [thick, domain=-\a:\a, samples=100] plot (\x,{(exp(\x)-exp(-\x))/2})
node[right]{$\senh x$};
\end{scope}

\begin{scope}[xshift=18cm, yshift=1cm]
\draw [ ->] (0,-1.5)--(0,1.5);
\pgfmathsetmacro{\a}{3};
\draw [ ->] (-\a,0)--(\a,0);
\draw [thick, domain=-\a:\a, samples=100] plot
(\x,{(exp(\x)-exp(-\x))/(exp(\x)+exp(-\x))})
node[below right]{$\tanh x$};
\draw[dashed] (0,1)--(\a,1) node[above]{$x=+1$};
\draw[dashed] (0,-1)--(-\a,-1) node[below]{$x=-1$};
\end{scope}

\end{tikzpicture}\end{bmlimage}
\end{center}

\eqref{itEstBas13}
\begin{center}
\begin{bmlimage}\begin{tikzpicture}[yscale=0.7]
\draw [ ->] (-3,0)--(3,0);
\draw [ ->] (0,-3)--(0,3) node[right]{$\frac{x^3-1}{x^3+1}$};
\draw [thick, domain=-3:-1.2, samples=100] plot (\x,{((\x)^3-1)/((\x)^3+1)});
\draw [thick, domain=-0.8:3, samples=100] plot (\x,{((\x)^3-1)/((\x)^3+1)});
\draw[dashed] (-1,-3)node[left]{$\scriptscriptstyle{x=-1}$}--(-1,3) ;
\draw[dashed] (-3,1) node[below]{$\scriptscriptstyle{x=1}$}--(3,1) ;
\fill (0,-1) circle (0.45mm);
\fill (0.793, -0.3333) circle (0.45mm);
\draw[<-] (0.1,-1.1)--(1,-3)node[right]{Pt. de inflexão e crítico: $(0,-1)$};
\draw[<-] (0.8, -0.4)--(1.2,-1) node[right]{Pt. de inflexão:
$(2^{-1/3},-1/3)$};
\draw (1,0) node{$\shortmid$} node[above]{$1$};
\draw (4,2) node[right]{$\displaystyle{f'(x)=\frac{6x^2}{(x^3+1)^2}}$};
\draw (4,0.5)
node[right]{$\displaystyle{f''(x)=\frac{12x(1-2x^3)}{(x^3+1)^3}}$};
\end{tikzpicture}\end{bmlimage}
\end{center}

\eqref{itEstBas14}:
\begin{center}
\begin{bmlimage}\begin{tikzpicture}
\draw [ ->] (-0.2,0)--(2*pi+0.5,0);
\draw [ ->] (0,-1.5)--(0,1.5) node[right]{$\scriptstyle{\tfrac12\sen
(2x)-\sen(x)}$};
\draw [color=gray!20, domain=-1:2*pi+1, samples=100] plot (\x,{0.5*sin(2*\x
r)-sin(\x r)});
\draw [thick, domain=0:2*pi, samples=100] plot (\x,{0.5*sin(2*\x r)-sin(\x r)});
\foreach \k in {0, 0.666, 1.333, 2} {
\draw ({\k*pi},0) node{$\shortmid$};
}
\draw (0.666*pi,0) node[above]{$\tfrac{2\pi}{3}$};
\draw (1.333*pi,0) node[below]{$\tfrac{4\pi}{3}$};
\fill (1.318,-0.726) circle (0.40mm);
\fill (4.965,0.726) circle (0.40mm);
\fill (0,0) circle (0.40mm);
\fill (pi,0) circle (0.40mm);
\fill (2*pi,0) circle (0.40mm);
% \draw[dashed] (-1,-3)node[left]{$\scriptscriptstyle{x=-1}$}--(-1,3) ;
% \draw[dashed] (-3,1) node[below]{$\scriptscriptstyle{x=1}$}--(3,1) ;
% \fill (0,-1) circle (0.45mm);
% \fill (0.793, -0.3333) circle (0.45mm);
% \draw[<-] (0.1,-1.1)--(1,-3)node[right]{Pt. de inflexão e crítico: $(0,-1)$};
% \draw[<-] (0.8, -0.4)--(1.2,-1) node[right]{Pt. de inflexão:
% $(2^{1/3},f(2^{1/2}))$};
% \draw (1,0) node{$\shortmid$} node[above]{$1$};
\end{tikzpicture}\end{bmlimage}
\end{center}

\eqref{itEstBas15}:
\begin{center}
\begin{bmlimage}\begin{tikzpicture}
\draw [ ->] (-5,0)--(5,0);
\draw [ ->] (0,-1.3)--(0,1.3) node[left]{$\frac{x}{\sqrt{x^2+1}}$};
\draw [thick, domain=-5:5, samples=50] plot (\x,{\x/(sqrt((\x)^2+1))});
 \draw[dashed] (0,1)--(5,1)node[above]{$\scriptscriptstyle{y=1}$};
 \draw[dashed] (-5,-1) node[below]{$\scriptscriptstyle{y=-1}$}--(0,-1) ;
 \draw[<-] (0.2, -0.2)--(0.5,-0.7) node[right]{Pt. de inflexão: $(0,0)$};
\draw (6,0.5) node[right]{$\displaystyle{f'(x)=\frac{1}{(x^2+1)^{3/2}}}$};
\draw (6,-0.5)
node[right]{$\displaystyle{f''(x)=\frac{-3x}{(x^2+1)^{5/2}}}$};
\end{tikzpicture}\end{bmlimage}
\end{center}


\end{Solution}
\begin{Solution}{8.9}
\eqref{itEstFuncB1}
\begin{center}
\begin{bmlimage}\begin{tikzpicture}[yscale=0.8]
\draw [ ->] (-4,0)--(4,0);
\draw [ ->] (0,-1.3)--(0,2.3) node[above]{$\ln |2-5x|$};
\draw [thick, domain=-4:0.3, samples=100] plot (\x,{ln(abs(2-5*(\x)))});
\draw[dashed] (0.4,-1.5)node[right]{$\scriptscriptstyle{x=\frac25}$}--(0.4,1.5);
\draw [thick, domain=0.5:4, samples=100] plot (\x,{ln(abs(2-5*\x))});
%  \draw[dashed] (-5,-1) node[below]{$\scriptscriptstyle{y=-1}$}--(0,-1) ;
% \fill (0,-1) circle (0.45mm);
% \fill (0.793, -0.3333) circle (0.45mm);
% \draw[<-] (0.1,-1.1)--(1,-3)node[right]{Pt. de inflexão e crítico: $(0,-1)$};
% \draw[<-] (0.2, -0.2)--(0.5,-0.7) node[right]{Pt. de inflexão: $(0,0)$};
% \draw (1,0) node{$\shortmid$} node[above]{$1$};
\end{tikzpicture}\end{bmlimage}
\end{center}


\eqref{itEstFuncB3}
\begin{center}
\begin{bmlimage}\begin{tikzpicture}[yscale=0.7]
\draw [ ->] (0,0)--(5,0);
\draw [ ->] (0,-1.3)--(0,2.3) node[left]{$\ln(\ln x)$};
\draw [thick, domain=1.2:5, samples=100] plot (\x,{ln(ln(\x))});
% \draw[dashed]
%(0.4,-1.5)node[right]{$\scriptscriptstyle{x=\frac25}$}--(0.4,1.5);
% \draw [thick, domain=0.5:4, samples=100] plot (\x,{ln(abs(2-5*\x))});
 \draw[dashed] (1,-2) node[left]{$\scriptscriptstyle{x=1}$}--(1,2) ;
% \fill (-0.693,1.012) circle (0.45mm);
% \fill (-1.365,1.033) circle (0.45mm);
% \fill (2.46,4.86) circle (0.45mm);
% \fill (0.793, -0.3333) circle (0.45mm);
% \draw[<-] (0.1,-1.1)--(1,-3)node[right]{Pt. de inflexão e crítico: $(0,-1)$};
% \draw[<-] (0.2, -0.2)--(0.5,-0.7) node[right]{Pt. de inflexão: $(0,0)$};
% \draw (1,0) node{$\shortmid$} node[above]{$1$};
\end{tikzpicture}\end{bmlimage}
\end{center}

\eqref{itEstFuncB7}
\begin{center}
\begin{bmlimage}\begin{tikzpicture}
\newcommand{\funcao}[1]{2.5*exp( -1*(#1) )*( (#1)^2 - 2*(#1))}
\draw [ ->] (-1,0)--(6.5,0);
\draw [ ->] (0,-1)--(0,2.3) node[right]{$e^{-x}(x^2-2x)$};
\draw [thick, domain=-0.35:5.5, samples=100] plot (\x,{\funcao{\x}});
\fill ({2-sqrt(2)},{\funcao{2-sqrt(2)}}) circle (0.40mm);
\draw ({2-sqrt(2)},{\funcao{2-sqrt(2)}})
node[below]{$\scriptstyle{(2-\sqrt{2},f(2-\sqrt{2}))}$};
\draw ({2+sqrt(2)},{\funcao{2+sqrt(2)}})
node[above]{$\scriptstyle{(2+\sqrt{2},f(2+\sqrt{2}))}$};
\fill ({2+sqrt(2)},{\funcao{2+sqrt(2)}}) circle (0.40mm);
\fill ({(6+sqrt(10))/2},{\funcao{(6+sqrt(10))/2}}) circle (0.40mm);
\draw[<-] ({(6+sqrt(10))/2+0.1},{\funcao{(6+sqrt(10))/2}+0.1})--
({(6+sqrt(10))/2+0.5},{\funcao{(6+sqrt(10))/2}+0.3})
node[right]{$\scriptstyle{(3+\sqrt{10}/2,f(3+\sqrt{10}/2))}$};
\fill ({(6-sqrt(10))/2},{\funcao{(6-sqrt(10))/2}}) circle (0.40mm);
\draw[<-] ({(6-sqrt(10))/2-0.1},{\funcao{(6-sqrt(10))/2}+0.1})--
({(6-sqrt(10))/2-0.5},{\funcao{(6-sqrt(10))/2}+2})
node[right]{$\scriptstyle{(3-\sqrt{10}/2,f(3-\sqrt{10}/2))}$};
\draw[<-] (5,-0.2)--(4.5,-0.5)node[below]{ass. horiz.: $y=0$};
\draw (6,2)
node[right]{$\displaystyle{f'(x)=-(x^2-4x+2)e^{-x}}$};
\draw (6,1.5)
node[right]{$\displaystyle{f''(x)=(x^2-6x+6)e^{-x}}$};
\end{tikzpicture}\end{bmlimage}
\end{center}




\eqref{itEstFuncB70}


\begin{center}
\begin{bmlimage}\begin{tikzpicture}[yscale=0.7]
\draw [ ->] (0,0)--(2.5,0);
\draw [ ->] (0,-0.3)--(0,1.5) node[left]{$\sqrt[x]{x}$};
\draw [thick, domain=0.2:6, samples=100, <-] plot
(\x,{exp(ln(\x)/\x)});
\fill (2.718,1.444) circle (0.45mm) node[above]{máx. glob.:
$(e,\sqrt[e]{e})$};
\coordinate (A) at (0.539,0.318);
\coordinate (B) at (5.04,1.37);
\fill (A) circle (0.45mm);
\fill (B) circle (0.45mm);
\draw[<-] (A)--(1.2,-0.3) node[right]{pt. infl.:
$(x_1,f(x_1))$};
\draw[<-] (B)--(5.2,1.9) node[right]{pt. infl.:
$(x_2,f(x_2))$};
%\draw[<-] (-0.67,0.9)--(0.2,0.4)node[right]{mín. global: $(\ln \tfrac12,f(\ln
%\tfrac12))$};
%\fill (-1.365,1.033) circle (0.45mm);
%\draw[<-] (-1.4,0.9)--(-1.6,0.5)node[left]{pt. infl.};
%\fill (2.46,4.86) circle (0.45mm);
%\draw[<-] (2.55,4.7)--(3,4)node[right]{pt. infl.};
\draw[dashed] (0,1)--(6,1);
\draw (2,1) node[below right]{Ass. Horiz.: $y=1$};
%\draw (5,2.5)
%node[right]{$\displaystyle{f'(x)=\frac{e^x(2e^x-1)}{e^{2x}-e^x+3}}$};
%\draw (5,0.5)
%node[right]{$\displaystyle{f''(x)=\frac{e^x(12e^x-e^{2x}
%-3)}{(e^{2x}-e^x+3)^2}}$};
\end{tikzpicture}\end{bmlimage}
\end{center}
Os pontos de inflexão são soluções da equação $(1-\ln
x)^2-3x+2x\ln x=0$. Pode ser mostrado que esses satisfazem
$x_1\simeq 0.58$, $x_1\simeq 4.37$.

\eqref{itEstFuncB4}
\begin{center}
\begin{bmlimage}\begin{tikzpicture}[yscale=0.7]
\newcommand{\funcao}[1]{ln( 5*(#1) )/sqrt( 5*(#1) ) }
\draw [ ->] (0,0)--(8,0);
\draw [ ->] (0,-1.3)--(0,2.3) node[left]{$\frac{\ln x}{\sqrt{x}}$};
\draw [thick, domain=0.1:8, samples=100] plot (\x,{\funcao{\x}});
% \draw[dashed] (1,-2) node[left]{$\scriptscriptstyle{x=1}$}--(1,2) ;
\fill ({2.718^2/5},{\funcao{2.718^2/5}}) circle (0.40mm);
\draw ({2.718^2/5},{\funcao{2.718^2/5}}) node[above]{$(e^2,2/e)$};
\fill ({2.718^(8/3)/5},{\funcao{2.718^(8/3)/5}}) circle (0.40mm);
\draw[<-] ({2.718^(8/3)/5+0.1},{\funcao{2.718^(8/3)/5}+0.2})--
({2.718^(8/3)/5+0.3},{\funcao{2.718^(8/3)/5}+1.3})
node[above]{pt. infl.: $(e^{8/3},f(e^{8/3}))$};
\draw (6,2.8)
node[right]{$\displaystyle{f'(x)=\frac{2-\ln x}{2x^{3/2}}}$};
\draw (6,1.5)
node[right]{$\displaystyle{f''(x)=-\frac{\sqrt{x}}{2}\frac{4-\tfrac32 \ln
x}{|x|^3}}$};
\draw[<-] (5,-0.2)--(4,-0.6) node[below]{ass. horiz.: $y=0$};

\end{tikzpicture}\end{bmlimage}
\end{center}

\eqref{itEstFuncB5}
%$\frac{\ln x-2}{(\ln x)^2}$
\begin{center}
\begin{bmlimage}\begin{tikzpicture}[yscale=0.5]
\newcommand{\funcao}[1]{( ln( (#1) ) -2)/ ( (ln( (#1) ))^2 ) }
\draw [ ->] (0,0)--(12,0);
\draw [ ->] (0,-1.3)--(0,2.3) node[left]{$\frac{\ln x-2}{(\ln x)^2}$};
\draw[dashed] (1,1)node[above]{$x=1$}--(1,-5);
\draw [thick, domain=0.1:0.5, samples=100] plot (\x,{\funcao{\x}});
\draw [thick, domain=1.6:11, samples=100] plot (\x,{\funcao{\x}});
% \draw[dashed] (1,-2) node[left]{$\scriptscriptstyle{x=1}$}--(1,2) ;
% \fill ({2.718^2/5},{\funcao{2.718^2/5}}) circle (0.40mm);
% \draw ({2.718^2/5},{\funcao{2.718^2/5}}) node[above]{$(e^2,2/e)$};
% \fill ({2.718^(8/3)/5},{\funcao{2.718^(8/3)/5}}) circle (0.40mm);
% \draw[<-] ({2.718^(8/3)/5+0.1},{\funcao{2.718^(8/3)/5}+0.2})--
% ({2.718^(8/3)/5+0.3},{\funcao{2.718^(8/3)/5}+1.3})
% node[above]{pt. infl.: $(e^{8/3},f(e^{8/3}))$};
% \draw (6,2.8)
% node[right]{$\displaystyle{f'(x)=\frac{2-\ln x}{2x^{3/2}}}$};
% \draw (6,1.5)
% node[right]{$\displaystyle{f''(x)=-\frac{\sqrt{x}}{2}\frac{4-\tfrac32 \ln
% x}{|x|^3}}$};
\draw[<-] (6,0.2)--(5,0.6) node[above]{ass. horiz.: $y=0$};
\draw (7,-1.5) node[right]{máx. global em $(e^4,f(e^4))$};
\draw (7,-3) node[right]{pt. infl. em
$(e^{1+\sqrt{13}},f(e^{1+\sqrt{13}})$};
\draw (5,-4.5) node[right]{$f'(x)=\frac{4-\ln x}{x(\ln x)^3}$,
$f''(x)=\frac{(\ln x)^2-2\ln x-12}{x^2(\ln x)^4}$};
\end{tikzpicture}\end{bmlimage}
\end{center}


\eqref{itEstFuncB2}
Ass. horiz.: $y=\ln 3$. Ass. obl.: $y=2x$.
\begin{center}
\begin{bmlimage}\begin{tikzpicture}[yscale=0.7]
\draw [ ->] (-4,0)--(2.5,0);
\draw [ ->] (0,-1.3)--(0,2.3) node[left]{$\ln(e^{2x}-e^x+3)$};
\draw [thick, domain=-4:2.6, samples=100] plot (\x,{ln(exp(2*\x)-exp(\x)+3)});
\fill (-0.693,1.012) circle (0.45mm);
\draw[<-] (-0.67,0.9)--(0.2,0.4)node[right]{mín. global: $(\ln \tfrac12,f(\ln
\tfrac12))$};
\fill (-1.365,1.033) circle (0.45mm);
\draw[<-] (-1.4,0.9)--(-1.6,0.5)node[left]{pt. infl.};
\fill (2.46,4.86) circle (0.45mm);
\draw[<-] (2.55,4.7)--(3,4)node[right]{pt. infl.};
\draw[dashed] (-4,{ln(3)})node[above]{$y=\ln 3$}--(-1,{ln(3)});
\draw (5,2.5)
node[right]{$\displaystyle{f'(x)=\frac{e^x(2e^x-1)}{e^{2x}-e^x+3}}$};
\draw (5,0.5)
node[right]{$\displaystyle{f''(x)=\frac{e^x(12e^x-e^{2x}
-3)}{(e^{2x}-e^x+3)^2}}$};
\end{tikzpicture}\end{bmlimage}
\end{center}

\eqref{itEstFuncB29} Observe que $(e^{|x|}-2)^3$ é par, e não
é derivável em $x=0$.
\begin{center}
\begin{bmlimage}\begin{tikzpicture}[yscale=0.7]
\draw [ ->] (-2,0)--(2,0);
\draw [ ->] (0,-1.3)--(0,2.3) node[above]{$(e^{|x|}-2)^3$};
\draw [thick, domain=0:1.2, samples=50] plot
(\x,{(exp(\x)-2)^3});
\draw [thick, domain=0:1.2, samples=50] plot
(-\x,{(exp(\x)-2)^3});
%\fill (-0.693,1.012) circle (0.45mm);
\draw[<-] (0.1,-1.05)--(1,-1.3) node[right]{mín. global: $(0,-1)$};
%\fill (-1.365,1.033) circle (0.45mm);
\draw[<-] (0.683,-0.1)--({0.693+0.5},-0.5)
node[right]{pt. infl.: $(\ln 2,0)$};
\draw[<-] (-0.683,-0.1)--({-0.693-0.5},-0.5)
node[left]{pt. infl.: $(-\ln 2,0)$};
\fill (0.693,0) circle (0.45mm);
\fill (-0.693,0) circle (0.45mm);
%\draw[<-] (2.55,4.7)--(3,4)node[right]{pt. infl.};
%\draw[dashed] (-4,{ln(3)})node[above]{$y=\ln 3$}--(-1,{ln(3)});
%\draw (5,2.5)
%node[right]{$\displaystyle{f'(x)=\frac{e^x(2e^x-1)}{e^{2x}-e^x+3}}$};
%\draw (5,0.5)
%node[right]{$\displaystyle{f''(x)=\frac{e^x(12e^x-e^{2x}
%-3)}{(e^{2x}-e^x+3)^2}}$};
\end{tikzpicture}\end{bmlimage}
\end{center}

\eqref{itEstFuncB33}
\begin{center}
\begin{bmlimage}\begin{tikzpicture}[yscale=0.7]
\draw [ ->] (-3,0)--(3,0);
\draw [ ->] (0,-0.3)--(0,1.4) node[above]{$\frac{e^x}{e^x-x}$};
\draw [thick, domain=-3:3, samples=50] plot
(\x,{exp(\x)/(exp(\x)-\x)});
\draw[dashed] (0,1)--(3,1);
%\draw [thick, domain=0:1.2, samples=50] plot
%(-\x,{(exp(\x)-2)^3});
%\fill (-0.693,1.012) circle (0.45mm);
%\draw[<-] (0.1,-1.05)--(1,-1.3) node[right]{mín. global: $(0,-1)$};
%\fill (-1.365,1.033) circle (0.45mm);
%\draw[<-] (0.683,-0.1)--({0.693+0.5},-0.5)
%node[right]{pt. infl.: $(\ln 2,0)$};
%\draw[<-] (-0.683,-0.1)--({-0.693-0.5},-0.5)
%node[left]{pt. infl.: $(-\ln 2,0)$};
%\fill (0.693,0) circle (0.45mm);
%\fill (-0.693,0) circle (0.45mm);
%\draw[<-] (2.55,4.7)--(3,4)node[right]{pt. infl.};
%\draw[dashed] (-4,{ln(3)})node[above]{$y=\ln 3$}--(-1,{ln(3)});
%\draw (5,2.5)
%node[right]{$\displaystyle{f'(x)=\frac{e^x(2e^x-1)}{e^{2x}-e^x+3}}$};
%\draw (5,0.5)
%node[right]{$\displaystyle{f''(x)=\frac{e^x(12e^x-e^{2x}
%-3)}{(e^{2x}-e^x+3)^2}}$};
\end{tikzpicture}\end{bmlimage}
\end{center}

\eqref{itEstFuncB33a}
\begin{center}
\begin{bmlimage}\begin{tikzpicture}[yscale=1]
\draw [ ->] (-4,0)--(4,0);
\draw [ ->] (0,-0.3)--(0,3.3) node[above right]{$\displaystyle{
\arcos(\frac{1-x^2}{1+x^2})}$};
\draw [thick, domain=-3.7:3.7, samples=51] plot
(\x,{3.1415/180*acos((1-\x*\x)/(1+\x*\x))});
\draw[dashed] (-4,3.14)--(4,3.14) node[right]{$y=\pi$};
\draw (5,1.7) node{Obs: a função não é derivável em $x=0$!};
\end{tikzpicture}\end{bmlimage}
\end{center}

\eqref{itEstFuncB36}

\begin{center}
\begin{bmlimage}\begin{tikzpicture}[yscale=0.9]
%\newcommand{\funcao}[1]{(abs(#1))^(0.8)*(abs((#1)-1))^(0.2)*(-1)};
\draw [ ->] (-3,0)--(3,0);
\draw [ ->] (0,-0.3)--(0,1.4) node[above]{
$\sqrt[5]{x^4(x-1)}$};
\pgfmathsetmacro{\e}{0.002};
\coordinate (A) at (0.8,-0.606);
\fill (A) circle (0.45mm);
\draw[<-] (0.8,-0.73)--(1.3,-1) node[right]{mín. loc.:
$(\tfrac45,f(\tfrac45))$};
\draw [thick, domain=-2:-\e, samples=50] plot
%%PROBLEMA:
(\x,{-exp(0.8*ln(abs(\x))+0.2*ln(abs(\x-1)))});
\draw [thick, domain=\e:{1-\e}, samples=50] plot
%(\x,{(abs(\x))^(0.8)*(abs(\x-1))^(0.2)*(-1)});
(\x,{-exp(0.8*ln(abs(\x))+0.2*ln(abs(\x-1)))});
\draw [thick, domain={1+\e}:3, samples=50] plot
(\x,{exp(0.8*ln(abs(\x))+0.2*ln(abs(\x-1)))});
\draw[<-] (-0.1,0.1)--(-1.5,1) node[left]{máx. loc.: $(0,0)$};
\draw[thick]
({1-\e},{-exp(0.8*ln(abs(1-\e))+0.2*ln(abs(1-\e-1)))})--
({1+\e},{exp(0.8*ln(abs(1+\e))+0.2*ln(abs(1+\e-1)))});
\draw[dashed] (-2,-2.2)--(3,2.8) node[right]{Ass. obl.:
$y=x-\tfrac15$.};
\end{tikzpicture}\end{bmlimage}
\end{center}
Obs: $f'(x)=f(x)\varphi(x)$, onde
$\varphi(x)=\tfrac15(\tfrac{4}{x}+\tfrac{1}{x-1})$.
A função não é derivável nem em $x=0$, nem em $x=1$
(apesar de ser contínua nesses pontos).
$f''(x)=(\varphi(x)^2+\varphi'(x))f(x)=-\tfrac{4}{25}
\frac{f(x)}{x^2(x-1)^2}$, logo, $f$ é convexa em
$(-\infty,0)$ e $(0,1)$, côncava em $(1,\infty)$.
Essa função possui uma assíntota \emph{oblíqua}:
$y=x-\tfrac15$.
\end{Solution}
\protect \section *{Capítulo \ref {CAP:Integral}}
\begin{Solution}{9.3}
A soma associada dá, usando a fórmula sugerida,
\[
\text{área}(R_n)=\frac{e^0}{n}+\frac{e^{1/n}}{n}
+\frac{e^{2/n}}{n}+\dots+\frac{e^{(n-1)/n}}{n}
=\frac{e-1}{\frac{e^{1/n}-1}{1/n}}\,.
\]
Mas $\lim_{n\to\infty}\frac{e^{1/n}-1}{1/n}=\lim_{t\to
0^+}\frac{e^t-1}{t}=1$. Logo,
$\text{área}(R)=e-1$.
\end{Solution}
\begin{Solution}{9.5}
\eqref{itExFuncArea1} $I(x)=0$ se $x\leq \frac12$, $I(x)=(x-\frac12)$ se
$x>\frac12$
\eqref{itExFuncArea2} $I(x)=-\frac{x^2}{2}+x$
\eqref{itExFuncArea3} $I(x)=x^2-x$.
\end{Solution}
\begin{Solution}{9.6}
\eqref{itExoPrimitTriv0} $-2x+C$
\eqref{itExoPrimitTriv1} $\frac{x^2}{2}+C$
\eqref{itExoPrimitTriv2} $\frac{x^3}{3}+C$
\eqref{itExoPrimitTriv3} $\frac{x^{n+1}}{n+1}+C$
\eqref{itExoPrimitTriv35} $\tfrac{2}{3}(1+x)^{3/2}+C$
\eqref{itExoPrimitTriv5} $\sen x+C$
\eqref{itExoPrimitTriv6} $-\cos x+C$
\eqref{itExoPrimitTriv7} $\frac{1}{2}\sen (2x)+C$
\eqref{itExoPrimitTriv9} $e^x+C$
\eqref{itExoPrimitTriv95} $x+e^{-x}+C$
\eqref{itExoPrimitTriv10} $\tfrac12 e^{2x}+C$
\eqref{itExoPrimitTriv105} $-\tfrac32e^{-x^2}+C$
\eqref{itExoPrimitTriv8} $2\sqrt{x}+C$
\eqref{itExoPrimitTriv4} $\ln x+C$
\eqref{itExoPrimitTriv11} $\arctan x+C$
\eqref{itExoPrimitTriv12} Com $-1<x<1$, $\arcsen x+C$
\end{Solution}
\begin{Solution}{9.8}
Como $\tfrac{x^2}{2}-x$ é primitiva de $f(x)=x-1$, temos
$\int_0^2(x-1)\,dx=(\tfrac{x^2}{2}-x)|_0^2=0$.
Esse resultado pode ser interpretando decompondo a integral em duas partes:
$\int_0^2f(x)\,dx=\int_0^1f(x)\,dx+\int_1^2f(x)\,dx$.
Esboçando o gráfico de $f(x)$ entre $0$ e $2$,
\begin{center}
\begin{bmlimage}\begin{tikzpicture}
\fill[areagrafico] (1,0)--(2,1)--(2,0)--cycle;
\fill[areafuncaoarea] (1,0)--(0,-1)--(0,0)--cycle;
\draw (1.65,0.25) node{$+$};
\draw (0.35,-0.3) node{$-$};
\draw (1,0) node{$\shortmid$} node[above]{$1$};
\draw[dashed] (2,0)node[below]{$2$}--(2,1);
\draw[dashed] (0,0)--(0,-1);
\draw[>=latex, ->] (-0.3,0)--(2.4,0);
\draw[>=latex, ->] (0,-1.2)--(0,1.3);
\draw[thick] (0,-1)--(2,1);
\end{tikzpicture}\end{bmlimage}
\end{center}
Vemos que a primeira parte
$\int_0^1f(x)\,dx=-\tfrac12$ é a contribuição do intervalo em
que $f$ é \emph{negativa}, e é exatamente
compensada pela contribuição da parte \emph{positiva}
$\int_1^2f(x)\,dx=+\tfrac12$.
\end{Solution}
\begin{Solution}{9.9}
Não, a conta não está certa. É porqué a função $\frac{1}{x^2}$ não é
contínua (nem definida) em $0$, ora $0$ pertence ao intervalo de
integração. Logo, o Teorema Fundamental não se aplica.
No entanto, será possível dar um sentido a
$\int_{-1}^2\frac{1}{x^2}\,dx$, usando \emph{integrais impróprias}.
\end{Solution}
\begin{Solution}{9.10}
\eqref{itareaRbas1} $5$,
\eqref{itareaRbas2} $\frac{16}{3}$,
\eqref{itareaRbas3} $\frac{1}{3}$,
\eqref{itareaRbas4} $1$.
\eqref{itareaRbas5} $\tfrac{125}{6}$.
\end{Solution}
\begin{Solution}{9.11}
\mbox{}
\begin{center}
\begin{bmlimage}\begin{tikzpicture}[scale=0.6]
\fill[areagrafico]
(0,-1)--(0.368,-1)--plot[domain=0.368:7.38](\x,{ln(\x)})--(0,2)--cycle;
\draw [thick, domain=0.3:8, samples=80] plot (\x,{ln(\x)}) node[right] {$\ln
x$};
\draw [>=latex, ->] (-0.5,0)--(4,0) node[right] {$x$};
\draw [>=latex, ->] (0,-1.2)--(0,2.5);
\draw [dotted] (0,-1)--(0.368,-1);
\draw [dotted] (0,2)--(7.38,2);
\draw (0,-1) node[left]{$-1$};
\draw (0,2) node[left]{$2$};
\draw (8,0) node[right]{$A=\int_{-1}^2e^ydy=e^2-e^{-1}\,.$};
\end{tikzpicture}\end{bmlimage}
\end{center}
Observe que expressando a área com uma integral com respeito a $x$,
$$A=\int_0^{e^{-1}}(2-(-1))dx+\int_{e^{-1}}^{e^2}(2-\ln x)
dx\,.$$
Essa integral requer a primitiva de $\ln x$, o que não
sabemos (ainda) fazer.
\end{Solution}
\begin{Solution}{9.12}
Consideremos $f_\alpha$ para diferentes valores de $\alpha$:
\begin{center}
\begin{bmlimage}\begin{tikzpicture}[scale=1.3]
\newcommand{\funcao}[2]{ ( exp(-1*(#1))/((#1)^2) )*( (#1)^2 - (#2)^2)}

\foreach \a in {0.3, 0.6,1,2} {
\fill[areagrafico, opacity=0.8] (-\a,0)--
plot[domain=-\a:\a] (\x,{\funcao{\a}{\x}})--(\a,0)--cycle;
}

\foreach \a in {0.3, 0.6,1,2} {
\draw[thick, domain=-\a:\a, samples=50] plot (\x,{\funcao{\a}{\x}});
}

\draw[>=latex,->] (-2.3,0)--(2.3,0);
\draw[>=latex,->] (0,-0.1)--(0,1.3);
\end{tikzpicture}\end{bmlimage}
\end{center}
A área debaixo do gráfico de $f_\alpha$ é dada pela integral
$$
I_\alpha=\int_{-\alpha}^\alpha f_\alpha(x)\,dx=\frac{e^{-\alpha}}{\alpha^2}
\int_{-\alpha}^\alpha(\alpha^2-x^2)\,dx=(\cdots)=\tfrac43 \alpha
e^{-\alpha}\,.$$
Um simples estudo de $\alpha\mapsto I_\alpha$ mostra que o seu máximo é
atingido em $\alpha=1$.
\end{Solution}
\begin{Solution}{9.13}
Como $I_n=\frac{n}{n+1}a^{\frac{n+1}{n}}$, temos $\lim_{n\to \infty}I_n=a$.
Quando $n\to \infty$, o gráfico de $x\mapsto x^{1/n}$ em $\bR_+$ tende
ao gráfico da função constante $f(x)\equiv 1$. Ora, $\int_0^a f(x)\,dx=a$!
\end{Solution}
\begin{Solution}{9.14}
\eqref{itprimitsubst1}
$-\frac{x^4}{4}-\frac{x^3}{3}+\frac{x^2}{2}+x+C$,
\eqref{itprimitsubst2} $\frac{-1}{2x^2}-\frac{\sen (2x)}{2}+C$,
\eqref{itprimitsubst3} $-\frac{1}{7x^7}-\frac{5}{x}+C$,
%\eqref{itprimitsubst4} $-\frac{1}{2}\cos(x^2)+C$,
%\eqref{itprimitsubst5} $\frac{x}{2}+\frac{\sen x \cos x}{2}+C$,
\eqref{itprimitsubst6} $2\tan x+C$.
%\eqref{itprimitsubst7} $\tfrac12\ln(1+x^2)+C$
%\eqref{itprimitsubst8} $-\ln(\cos x)+C$
\end{Solution}
\begin{Solution}{9.15}
\eqref{itprimitsubst40} $\frac{1}{8}(x+1)^8+C$ (Obs: aqui, basta fazer a
substituição $u=x+1$. Pode também fazer sem, mas implica desenvolver um
polinômio de grau $7$!)
\eqref{itprimitsubst400} $\frac{-1}{2(2x+1)}+C$
\eqref{itprimitsubst401} $\frac{1}{8(1-4x)^2}+C$
\eqref{itprimitsubst4} $-\frac{1}{2}\cos(x^2)+C$,
\eqref{itprimitsubst4000} $\frac{1}{2}\sen^2(x)+C$, ou $-\frac{1}{2}\cos^2(x)+C$
\eqref{itprimitsubst45} $2\sen(\sqrt{x})+C$,
\eqref{itprimitsubst5} $\frac{x}{2}+\tfrac14\sen (2x)+C$,
\eqref{itprimitsubst7} $\tfrac12\ln(1+x^2)+C$,
\eqref{itprimitsubst71} $\frac{2}{3}(1+\sen x)^{\frac{3}{2}}+C$
\eqref{itprimitsubst8} $\int \tan x\,dx=\int\frac{\sen x}{\cos
x}\,dx=-\int\frac{(\cos x)'}{\cos
x}\,dx -\ln|\cos x|+C$.
\eqref{itprimitsubst9} $\tfrac32 \ln(1+x^2)+5\arctan x+C$
\eqref{itprimitsubst10} $\frac{1}{\sqrt{2}}\arctan(\frac{x+1}{\sqrt{2}})+C$
\eqref{itprimitsubst12}
Com a substituição $u:=e^x$, $du=e^x dx$,
$\int e^x\tan(e^x)dx=\int \tan u du=-\ln|\cos u|+C=-\ln|\cos(e^x)|+C$.
\eqref{itprimitsubst13} $\frac{1}{2(1+y)^2}-\frac{1}{1+y}+C$
\eqref{itprimitsubst14} $\frac{1}{3}(1+x^2)^{\frac{3}{2}}+C$
\eqref{itprimitsubst15} $\frac{-1}{2(1+x^2)}+C$
\eqref{itprimitsubst11} $-\frac{1}{3\sen^3t}+\frac{1}{\sen t}+C$ (a ideia aqui
é escrever $\frac{\cos^3t}{\sen^4t}=\frac{\cos^2t}{\sen^4t}\cos
t=\frac{1-\sen^2t}{\sen^4t}\cos t$)
\eqref{itprimitsubst16} $\frac{(\sen x)^4}{4}-\frac{(\sen x)^6}{6}$
\end{Solution}
\begin{Solution}{9.16}
\eqref{itttit1}
Com $u=1-x^2$, $du=-2x\,dx$, temos
\begin{align*}
\int \frac{2x^3dx}{\sqrt{1-x^2}}\,dx=-\int
\frac{x^2}{\sqrt{1-x^2}}(-2x)\,dx
&=-\int \frac{1-u}{\sqrt{u}}\,du\\
&=-2\sqrt{u}+\tfrac23 u^{3/2}+C\\
&=-2\sqrt{1-x^2}+\tfrac23 (1-x^2)^{3/2}+C\,.
\end{align*}
\eqref{itttit2}
Completando o quadrado, e fazendo a substituição $u=2x-1$,
\begin{align*}
\int \frac{dx}{\sqrt{x-x^2}}=\int
\frac{dx}{\sqrt{\tfrac14-(x-\tfrac12)^2}}&=
\int \frac{2 dx}{\sqrt{1-(2x-1)^2}}\\
&=\int \frac{du}{\sqrt{1-u^2}}=\arcsen u+C=\arcsen (2x-1)+C\,.
\end{align*}
\eqref{itttit3} Com $u=\ln t$, $\int \frac{\ln x}{x}\,dx=\int
u\,du=\tfrac{u^2}{2}+C=\tfrac12(\ln x)^2+C$
\eqref{itttit4} Com $u=e^x$, $\int e^{e^x}e^x\,dx=\int e^u\,du=e^u+C=e^{e^x}+C$.
\eqref{itttit5} $\int \frac{\sqrt{x}}{1+\sqrt{x}}\,dx=x-2\sqrt{x}+2\ln
(1+\sqrt{x})+C$.
\eqref{itttit6} $\int \tan^2x\,dx=\int(1+\tan^2x-1)\,dx=\tan x-x+C$.
\end{Solution}
\begin{Solution}{9.17}
\eqref{itintpartes1} $\sen x-x\cos x+C$,
\eqref{itintpartes2} $\frac{1}{5}x\sen(5x)+\frac{1}{25}\cos(5x)+C$
\eqref{itintpartes3} Integrando duas vezes por partes:
$$
\int x^2\cos x\,dx=x^2\sen x-\int (2x)\sen x\,dx=x^2\sen x-2\Bigl\{
x(-\cos x)-\int (-\cos x)\,dx\,.
\Bigr\}$$
Portanto $\int x^2\cos x\,dx=x^2\sen x-2(\sen x-x\cos x)+C$.
\eqref{itintpartes4} $(x-1)e^x+C$
\eqref{itintpartes5} $-\tfrac13 e^{-3x}(x^2-\tfrac23 x-\tfrac29)+C$
\eqref{itintpartes6}
\begin{align*}
\int x^3\cos (x^2)\,dx=\int x^2 (x\cos(x^2))\,dx&=x^2(\tfrac12
\sen(x^2))-\int(2x)(\tfrac12 \sen (x^2))\,dx\,\\
&=\tfrac12 x^2 \sen(x^2)+\tfrac12 \cos (x^2)+C\,.
\end{align*}
%\eqref{itintpartes7} $x^2(\ln x-\tfrac12)+C$
\end{Solution}
\begin{Solution}{9.18}
 \eqref{ititnpartmntriv1}
$\int \arctan x dx=x\arctan
x-\int\frac{x}{1+x^2}\,dx=x\arctan x-\tfrac12 \ln (1+x^2)+C$.
\eqref{ititnpartmntriv2} $x(\ln x)^2-2x(\ln x-1)+C$
\eqref{ititnpartmntriv3} $x\arcsen x+\sqrt{1-x^2}+C$
\eqref{ititnpartmntriv4} $\int x\arctan x\,dx=\frac12(x^2\arctan x-x+\arctan x)+C$
\end{Solution}
\begin{Solution}{9.19}
\eqref{itintpartestruc1} $-\frac{e^{-x}}{2}(\sen x+\cos x)+C$
\eqref{itintpartestruc2} $\frac{e^{-st}}{1+s^2}(\sen t- s\cos t)+C$
\eqref{itintpartestruc3} $\frac{x}{2}(\sen (\ln x)-\cos (ln x))+C$
\end{Solution}
\begin{Solution}{9.20}
Chamando $u=\sqrt{x+1}$, temos
$$
\int_0^3e^{\sqrt{x+1}}\,dx=\int_1^2
2ue^u\,du=2\bigl\{ue^u-e^u\bigr\}\big|_1^2=2e^2\,.
$$
Chamando $u=\ln x$, temos $e^u\,du=dx$, e
$$
\int x(\ln x)^2\,dx=\int
u^2e^{2u}\,du=\tfrac{u^2}{2}e^{2u}-\tfrac{u}{2}e^{2u}+\tfrac14 e^{2u}+C\,.
$$
Logo, $\int x(\ln x)^2\,dx=\tfrac12 x^2(\ln x)^2-\tfrac12 x^2\ln
x+\tfrac14x^2+C$.
\end{Solution}
\begin{Solution}{9.21}
Para ter $\frac{1}{x(x^2+1)}=\frac{A}{x}+\frac{B}{x^2+1}$, isto é
$1=A(x^2+1)+Bx$, $A$ e $B$
devem satisfazer às três condições $A=0$, $B=0$, $A=1$, que obviamente é
impossível.
\end{Solution}
\begin{Solution}{9.22}
Para ter $\frac{1}{x(x+1)^2}=\frac{A}{x}+\frac{B}{(x+1)^2}$, isto é
$1=A(x+1)^2+Bx$, $A$ e $B$ precisariam satisfazer às três condições $A=0$,
$2A+B=0$, $A=1$, que obviamente é impossível.
\end{Solution}
\begin{Solution}{9.23}
\eqref{itfracparciais1} $\tfrac{1}{\sqrt{2}}\arctan(\sqrt{2}x)+C$
\eqref{itfracparciais2} Como $\frac{x^5}{x^2+1}=x^3-x+\frac{x}{x^2+1}$, temos
$\int\frac{x^5}{x^2+1}\,dx=\tfrac{x^4}{4}-\tfrac{x^2}{2}+\tfrac12\ln (x^2+1)+C$.
\eqref{itfracparciais3} $\frac{-1}{x+2}+C$

\eqref{itfracparciais30}
A decomposição em frações parciais é da forma
$\frac{1}{x(x+1)}=\frac{A}{x}+\frac{B}{x+1}$.
Colocando no mesmo denominador, $A$ e $B$
tem que satisfazer $1=(A+B)x+A$ para todo $x$. Logo, $A=1$ e $B=-1$. Isto é,
$\frac{1}{x^2+x}=\frac{1}{x}-\frac{1}{x+1}$. Logo,
\begin{align*}
\int \frac{1}{x^2+x}\,dx&=\int \frac{1}{x}\,dx-\int\frac{1}{x+1}\,dx\\
&=\ln |x|-\ln |x+1|+C\,,\quad\quad
\end{align*}
\eqref{itfracparciais31}
O integrante é da forma $\frac{P(x)}{Q(x)}$, em que o grau
de $P$ é menor do que o de $Q$. Além disso, podemos fatorar $x^3+x=x(x^2+1)$. O
polimômio de ordem $2$ tem discriminante negativo. Logo, é irredutível,
e podemos tentar uma decomposição da forma
$$
\frac{1}{x(x^2+1)}=\frac{A}{x}+\frac{Bx+C}{x^2+1}\quad \forall x\,.
$$
Colocando no mesmo denominador, $A$ $B$ e $C$
tem que satisfazer $1=(A+B)x^2+Cx+A$ para todo $x$. Logo, $A=1$, $C=0$, e
$B=-A=-1$. Isto é,
\begin{align*}
\int \frac{1}{x^3+x}\,dx=\int \frac{1}{x}\,dx-\int\frac{x}{x^2+1}\,dx
&=\ln |x|-\int\frac{x}{x^2+1}\,dx\\
&=\ln |x|-\tfrac{1}{2}\ln (x^2+1)+C\,,\quad\quad
\end{align*}
Nesta última integral, fizemos $u=x^2+1$, $du=2x\,dx$.
\eqref{itfracparciais4} Como $\Delta=16>0$, podemos procurar fatorar e fazer uma
separação em frações parciais,
$$\int\frac{dx}{x^2+2x-3}=\int\frac{dx}{(x+3)(x-1)}=-\tfrac14\int\frac{dx}{x+3}
+\tfrac14\int\frac{dx}{x-1}=\tfrac14\ln \Bigl|\frac{x-1}{x+3}\Bigr|+C\,.
$$
\eqref{itfracparciais5} Como $\Delta=-8<0$, o denominador não se fatora.
Completando o quadrado,
$$
\int\frac{dx}{x^2+2x+3}=\int\frac{dx}{(x+1)^2+2}=\tfrac12\int\frac{dx}{(\frac{x+
1}{\sqrt{2}})^2+1}=\tfrac{1}{\sqrt{2}}\arctan\bigl(\frac{x+
1}{\sqrt{2}}\bigr)+C\,.
$$
\eqref{itfracparciais50} Como
$\frac{1}{x(x-2)^2}=\frac{1}{4x}-\frac{1}{4(x-2)}+\frac{1}{2(x-2)^2}$, temos
$$
\int\frac{dx}{x(x-2)^2}=\tfrac14\ln|x|-\tfrac14\ln|x-2|-\frac{1}{2(x-2)}+C\,.
$$
\eqref{itfracparciais51}
$\frac{1}{x^2(x+1)}=\frac{A}{x}+\frac{B}{x^2}+\frac{C}{x+1}$, com $A=-1$,
$B=1$, $C=1$. Logo,
$$
\int\frac{dx}{x^2(x+1)}=-\ln |x|-\tfrac1x+\ln|x+1|+C'\,.
$$

\eqref{itfracparciais7}
Como $t^4+t^3=t^3(t+1)$, procuramos uma separação da forma
$$
\frac{1}{t^4+t^3}=\frac{A}{t}+\frac{B}{t^2}+\frac{C}{t^3}+\frac{D}{t+1}\,\quad
\forall t.
$$
Colocando no mesmo denominador e juntando os termos vemos que $A,B,C,D$ têm que
satisfazer
$$
1=(A+D)t^3+(A+B)t^2+(B+C)t+C\quad\forall t\,.
$$
Identificando os coeficientes obtemos $C=1$, $B=-C=-1$, $A=-B=+1$, e
$D=-A=-1$. Isso implica
\begin{align*}
\int \frac{1}{t^4+t^3}dt&=\int\frac{dt}{t}-\int \frac{dt}{t^2}+\int
\frac{dt}{t^3}-\int \frac{dt}{t+1}\\
&=\ln|t|+\frac{1}{t}-\frac{1}{2t^2}-\ln|t+1|+C\,.
\end{align*}
\eqref{itfracparciais52}
\begin{align*}
\int\frac{dx}{x(x+1)^3}
&=\int
\frac{dx}{x}-\int\frac{dx}{x+1}-\int\frac{dx}{(x+1)^2}-\int\frac{dx}{(x+1)^3}\\
&=\ln|x|-\ln|x+1|+\frac{1}{x+1}+\frac{1}{2(x+1)^2}+C\,.
\end{align*}
\eqref{itfracparciais9} $\int\frac{x^2+1}{x^3+x}\,dx=\int \frac{dx}{x}=\ln|x|+C$
\eqref{itfracparciais10} Com
$u=x^4-1$, $\int\frac{x^3}{x^4-1}\,dx=\tfrac14\ln|x^4-1|+C$ (é bem mais simples do que começar uma
decomposição em frações parciais...)
\eqref{itfracparciais104} Começando com uma integração por partes,
\[
\int \frac{x\ln x}{(x^2+1)^2}\,dx=\frac{-1}{2(x^2+1)}\ln x+\frac12\int
\frac{1}{(x^2+1)x}\,dx\,,
\]
e essa última integral se calcula como no Exemplo \ref{Ex:decomppp}.
\eqref{itfracparciais6} Primeiro, observe que $x^3+1$ possui $x=-1$ como raiz.
Logo, ele pode ser fatorado como $x^3+1=(x+1)(x^2-x+1)$.
Como $x^2-x+1$ tem um discriminante negativo,
procuremos uma decomposição da forma
$$
\frac{1}{x^3+1}=\frac{A}{x+1}+\frac{Bx+C}{x^2-x+1}\,.
$$
É fácil ver que $A$, $B$ e $C$ satisfazem às três condições $A+B=0$,
$-A+B+C=0$, $A+C=1$. Logo, $A=\frac13$, $B=-\frac13$, $C=\frac23$. Escrevendo
\begin{align*}
 \int\frac{dx}{x^3+1}&=\tfrac{1}{3}\int\frac{dx}{x+1}-\tfrac13\int
\frac{x-2}{x^2-x+1}\,dx\\
&=\tfrac{1}{3}\ln|x+1|-\tfrac13\int
\frac{x-2}{x^2-x+1}\,dx\\
\end{align*}
Agora,
\begin{align*}
\int \frac{x-2}{x^2-x+1}\,dx&=\tfrac12\int \frac{2x-1}{x^2-x+1}\,dx-\tfrac{3}{2}
\int\frac{dx}{x^2-x+1}\\
&=\tfrac12 \ln|x^2-x+1|-\tfrac{3}{2}
\int\frac{dx}{x^2-x+1}\\
&=\tfrac12 \ln|x^2-x+1|-\tfrac{4}{\sqrt{3}}\arctan\bigl(\tfrac{2}{\sqrt{3}}
(x-\tfrac12) \bigr)+C\,.
\end{align*}
Juntando,
$$
\int\frac{dx}{x^3+1}=\tfrac{1}{3}\ln|x+1|-\tfrac16\ln|x^2-x+1|+\tfrac{4}{3\sqrt{
3}}\arctan\bigl(\tfrac{2}{\sqrt{3}}(x-\tfrac12) \bigr)+C\,.
$$
\end{Solution}
\begin{Solution}{9.24}
Com a dica, e a substituição $u=\sen x$,
\begin{align*}
\int \frac{dx}{\cos x}=\int\frac{\cos x}{1-\sen^2
x}dx=\int\frac{du}{1-u^2}&=-\int\frac{du}{u^2-1}\\
&=-\tfrac{1}{2}\ln\Bigl|\frac{u-1}{u+1}\Bigr|+C\\
&=\tfrac{1}{2}\ln\Bigl|\frac{1+\sen x}{1-\sen x}\Bigr|+C
\end{align*}
Observe que essa última expressão pode ser transformada da seguinte maneira:
\begin{align*}
\tfrac{1}{2}\ln\Bigl|\frac{\sen x+1}{\sen x-1}\Bigr|=
\tfrac{1}{2}\ln\Bigl|\frac{(1+\sen x)^2}{\cos^2x}\Bigr|=
\ln\Bigl|\frac{1+\sen x}{\cos x}\Bigr|=
\ln\Bigl|\frac{1}{\cos x}+\tan x\Bigr|\,.
\end{align*}
\end{Solution}
\begin{Solution}{9.25}
Como $\Delta=4^2-4\cdot 13<0$, o polinômio $x^2+4x+13$ tem discriminante
negativo. Logo, completando o quadrado:\index{completar um
quadrado}
$x^2+4x+13=(x+2)^2-4+13=(x+2)^2+9$, e
\begin{align*}
\int \frac{x}{x^2+4x+13}dx=\int
\frac{x}{(x+2)^2+9}dx=\tfrac19\int\frac{x}{(\tfrac13(x+2))^2+1}dx
\end{align*}
Com $u=\frac{1}{3}(x+2)$, $x=3u-2$, $3du=dx$,
\begin{align*}
 \tfrac19\int\frac{x}{(\tfrac13(x+2))^2+1}dx&=\tfrac13\int\frac{3u-2}{u^2+1}du\\
&=\tfrac12\int\frac{2u}{u^2+1}du-\tfrac23\int\frac{du}{u^2+1}\\
&=\tfrac12\ln (u^2+1)-\tfrac23 \arctan(u)+C\\
&=\tfrac12\ln (x^2+4x+13)-\tfrac23\arctan(\frac{1}{3}(x+2))+C
\end{align*}
\end{Solution}
\begin{Solution}{9.26}
\eqref{itPotTrig0} $-\cos x+\tfrac13\cos^3x+C$
\eqref{itPotTrig01} Com $u=\sen x$, $\int \cos^5x\,dx=\int
(1-u^2)^2\,du=\cdots=\sen x-\tfrac23\sen^3x+\tfrac15\sen^5x+C$
\eqref{itPotTrig1} Escrevemos
$\int (\cos x\sen x)^5dx=\int
\sen^5x(1-\sen^2x)^2\cos xdx$.
Com $u=\sen x$ dá
\begin{align*}
 \int \sen^5x(1-\sen^2x)^2\cos xdx&=
\int u^5(1-u^2)^2du\\
&=\int (u^5-2u^7+u^9)du\\
&=\frac{u^6}{6}-2\frac{u^8}{8}+\frac{u^{10}}{10}+C\\
&=\frac{\sen^6x}{6}-\frac{\sen^8x}{4}+\frac{\sen^{10}x}{10}+C\,.
\end{align*}
\eqref{itPotTrig10} $-\frac{\cos^{1001}x}{1001}+C$
\eqref{itPotTrig2}
Com $u=\sen t$,
$\int (\sen^2t\cos t) e^{\sen t}dt=\int u^2e^udu$.
Integrando duas vezes por partes e voltando para a variável $t$,
\begin{align*}
 \int u^2e^udu&=u^2e^u-\int (2u)e^udu\\
&=u^2e^u-2\big\{ue^u-\int e^udu\big\}\\
&=u^2e^u-2\{ue^u-e^u\}+C\\
&=e^u(u^2-2u+2)+C\\
&=e^{\sen t}(\sen^2 t-2\sen t+2)+C\,.
\end{align*}
\eqref{itPotTrig3} Com $u=\cos x$,
$\int \sen^3x \sqrt{\cos x}\,dx=-\int(1-u^2)\sqrt{u}\,du=-\int
(u^{1/2}-u^{5/2})\,du=-\tfrac23 u^{3/2}+\tfrac27 u^{7/2}+C=-\tfrac23
(\cos x)^{3/2}+\tfrac27 (\cos x)^{7/2}+C$.
\eqref{itPotTrig4} $\int
\sen^2x\cos^2x\,dx=\int(1-\cos^2x)\cos^2x\,dx=\int\cos^2x\,dx-\int\cos^4x\,dx$,
e essas duas primitivas já foram calculadas anteriormente.
\end{Solution}
\begin{Solution}{9.27}
\eqref{itInttansec3} $\int \sec^2x\,dx=\tan x+C$.
\eqref{itInttansec1} $\int\tan^2x \,dx=\int(\tan^2x+1-1)\,dx=\tan x-x+C$.
\eqref{itInttansec1a} $\int\tan^3x \,dx=\int\tan x(1+\tan^2x)\,dx-\int \tan x\,dx=\tfrac12\tan^2 x-\ln
|\cos x|+C$.
\eqref{itInttansec6} $\int \tan x\sec x\,dx=\sec x+C$.
\eqref{itInttansec2} $\int\tan^4 x\sec^4x\,dx=\int
\tan^4x(\tan^2x+1)\sec^2x\,dx=\int u^4(u^2+1)\,du=
\tfrac17u^7+\tfrac15u^5+C=\tfrac17\tan^7x+\tfrac15\tan^5x+C$.
\eqref{itInttansec4} $\int\cos^5x\tan^5x\,dx=\int
\sen^5x\,dx=\int(1-\cos^2x)^2\sen x\,dx=
-\int(1-u^2)^2\,du=-u+\tfrac23u^3-\tfrac15 u^5+C=
-\cos x+\tfrac23\cos^3x-\tfrac15 \cos^5x+C$.
\eqref{itInttansec7} $\int \sec^5x\tan^3x\,dx=\int \sec^4x(\sec^2x-1)(\tan x\sec
x)\,dx=\int w^4(w^2-1)\,dw=\tfrac17 w^7-\tfrac15 w^5+C=\tfrac17 \sec^7x-\tfrac15
\sec^5x+C$.
\eqref{itInttansec8} Por partes (lembra que
$(\sec\theta)'=\tan\theta\sec\theta$):
\begin{align*}
 \int\sec^2\theta\sec\theta\,d\theta
&=\tan \theta\sec\theta-\int\tan^2\theta\sec\theta\,d\theta\\
&=\tan \theta\sec\theta-\int(\sec^2\theta-1)\sec\theta\,d\theta\,.
\end{align*}
Logo,
$$
\int\sec^3\theta\,d\theta=
\tfrac12\tan\theta\sec\theta+\tfrac12\int\sec\theta\,d\theta\,.
$$
Já calculamos a primitiva de $\sec \theta$ no Exercício
\ref{exo:primunsurseno}:
$\int\sec\theta\,d\theta=\ln\bigl|\sec\theta+\tan \theta\bigr|+C$. Logo,
$$
\int\sec^3\theta\,d\theta=
\tfrac12\tan\theta\sec\theta+\tfrac12\ln\bigl|\sec\theta+\tan \theta\bigr|+C\,.
$$
\end{Solution}
\begin{Solution}{9.28}
De fato,
\begin{align*}
\bigl(\tfrac12\arcsen x+\tfrac12x\sqrt{1-x^2}\bigr)'&=
\tfrac12\frac{1}{\sqrt{1-x^2}}+\tfrac12
\sqrt{1-x^2}+\tfrac12 x\frac{-2x}{2\sqrt{1-x^2}} \\
&=\tfrac12\frac{1-x^2}{\sqrt{1-x^2}}+\tfrac12
\sqrt{1-x^2}\\
&=\tfrac12 \sqrt{1-x^2}+\tfrac12 \sqrt{1-x^2}=\sqrt{1-x^2}\,.
\end{align*}
\end{Solution}
\begin{Solution}{9.29}
A área é dada por $A=4\int_0^{\alpha}\beta\sqrt{1-\frac{x^2}{\alpha^2}}\,dx$.
Com $x=\alpha\sen \theta$,
$$A=4\beta\int_0^{\alpha}\sqrt{1-\frac{x^2}{\alpha^2}}\,dx=
4\alpha\beta \int_0^{\pisobredois}\cos^2\theta\,d\theta=\pi \alpha\beta\,.
$$
Quando $\alpha=\beta=R$, a elipse é um disco de raio $R$, de área $\pi
R\cdot R=\pi R^2$.
\end{Solution}
\begin{Solution}{9.30}
\eqref{itPrimSubSinus1}
Sabemos que $\int \frac{dx}{\sqrt{1-x^2}}=\arcsen x+C$, mas isso pode ser
verificado de novo fazendo a substituição $x=\sen \theta$:
$\frac{dx}{\sqrt{1-x^2}}=\int\frac{1}{\sqrt{1-\sen^2\theta}}\cos\theta\,
d\theta\int d\theta=\theta+C=\arcsen x+C$.
%\eqref{itSubstitTrig00} $\int\frac{x}{\sqrt{3-x^2}}\,dx=...$
\eqref{itSubstitTrig2} Com $x=\sqrt{10}\sen t$ dá
\begin{align*}
\int\frac{x^7}{\sqrt{10-x^2}}dx=\int\frac{\sqrt{10}^7\sen^7t}{\sqrt{10}\cos
t}\sqrt{10}\cos tdt
&=\sqrt{10}^7\int \sen^7tdt
\end{align*}
Uma segunda substituição $u=\cos t$ dá
\begin{align*}
\int \sen^7tdt&=\int (1-\cos^2t)^3\sen tdt\\
\quad&=-\int(1-u^2)^3du\\
&=-\int(1-3u^2+3u^4-u^6)du \\
&=-\Big\{u-u^3+\frac{3}{5}u^5-\frac{1}{7}u^7\Big\}+C
\end{align*}
Para voltar para $x$, observe que $u=\cos
t=\sqrt{1-\sen^2t}=\sqrt{1-(x/\sqrt{10})^2}$.
Logo,
$$
\int\frac{x^7}{\sqrt{10-x^2}}dx=\sqrt{10}^7\Bigl\{-\sqrt{1-\frac{x^2}{10}}
+\sqrt{1-\frac{x^2}{10}}^3-\frac{3}{5}\sqrt{1-\frac{x^2}{10}}^5
+\frac{1}{7}\sqrt{1-\frac{x^2}{10}}^7\Bigr\}+C
$$
\eqref{itPrimSubSinus3} Observe que $\sqrt{1-x^3}$ \emph{não é da forma
$\sqrt{a^2-b^2x^2}$!} Mas com a substituição $u=1-x^3$,
$\int \frac{x^2}{\sqrt{1-x^3}}\,dx=-\tfrac13\int
\frac{du}{\sqrt{u}}=-\tfrac23\sqrt{u}+C=-\tfrac23\sqrt{1-x^3}+C$.
\eqref{itPrimSubSinus2} Aqui uma simples substituição $u=1-x^2$ dá
$\int x\sqrt{1-x^2}\,dx=-\tfrac13(1-x^2)^{3/2}+C$. (Pode também fazer $x=\sen
\theta$, é um pouco mais longo.)
\eqref{itSubstitTrig6} Completando o quadrado,
$3-2x-x^2=4-(x+1)^2$. Chamando $x+1=2\sen \theta$,
\begin{align*}
\int\frac{x}{\sqrt{3-2x-x^2}}\,dx=\int\frac{2\sen
\theta-1}{\sqrt{4-4\sen^2\theta}}2\cos \theta\,d\theta&=
2\int \sen \theta\,d\theta-\int \,d\theta\\
&=-2\cos\theta-\theta+C\,.
\end{align*}
Voltando para $x$, temos
$$
\int\frac{x}{\sqrt{3-2x-x^2}}\,dx=-2\sqrt{1-(\tfrac{x+1}{2})^2}
-\arcsen(\tfrac{x+1}{2})+C\,.$$
\eqref{itSubstitTrig000} Com $x=3\sen \theta$ obtemos
$\int x^2{\sqrt{9-x^2}}\,dx=3^4\int \sen^2\theta\cos^2\theta\,d\theta$.
\end{Solution}
\begin{Solution}{9.31}
\eqref{itSubstitTrig3}
fazendo $x=\tfrac12\tan \theta$ dá
\begin{align*}
 \int \frac{x^3}{\sqrt{4x^2+1}}dx&=\int \frac{(\tfrac12 \tan
\theta)^3}{\sqrt{\sec^2\theta}}
\half\sec^2\theta d\theta\\
&=\tfrac{1}{16} \int\tan^3\theta \sec\theta d\theta\\
&=\tfrac{1}{16} \int (\sec^2\theta-1)\sec\theta \tan\theta d\theta
\end{align*}
Com $w=\sec \theta$, obtemos
$\int (\sec^2\theta-1)\sec\theta \tan\theta
d\theta=\frac{\sec^3\theta}{3}-{\sec\theta}+C$.
Mas $\tan \theta=2x$ implica $\sec \theta=\sqrt{\tan^2\theta+1}=\sqrt{1+4x^2}$.
Logo,
$$
\int \frac{x^3}{\sqrt{4x^2+1}}dx=\frac{(1+4x^2)^{\frac{3}{2}}}{48}
-\frac{\sqrt{1+4x^2}}{16}+C\,.
$$
Observe que pode também rearranjar um pouco a função e fazer por partes:
\begin{align*}
 \int \frac{x^3}{\sqrt{4x^2+1}}dx&=
 \tfrac14\int x^2\frac{8x}{2\sqrt{4x^2+1}}dx\\
&=\tfrac14\Bigl\{x^2\sqrt{4x^2+1}-\int (2x)\sqrt{4x^2+1}dx\Bigr\}\\
&=\tfrac14\Bigl\{x^2\sqrt{4x^2+1}-\tfrac14\frac{(4x^2+1)^{3/2}}{3/2}\Bigr\}+C\,,
\end{align*}
dá na mesma!
\eqref{itSubstitTrig31} Com $x=\tan \theta$, temos
\begin{align*}
\int x^3\sqrt{x^2+1}\,dx&=\int \tan^3\theta\sec^3\theta\,d\theta\\
&=\int (\sec^2\theta-1)\sec^2\theta(\tan\theta\sec\theta)\,d\theta\\
(\text{via }w=\sec\theta)\,
&=\tfrac15\sec^5\theta-\tfrac13\sec^3\theta+C\\
&=\tfrac15(x^2+1)^{5/2}-\tfrac13(x^2+1)^{3/2}+C\,.
\end{align*}
\eqref{itSubstitTrig32} Aqui não precisa fazer substituição trigonométrica:
$u=x^2+a^2$ dá $\int
x\sqrt{x^2+a^2}\,dx=\tfrac12\int\sqrt{u}\,du=\tfrac13u^{3/2}+C=
\tfrac13(x^2+a^2)^{3/2}+C$.
\eqref{itSubstitTrig33} Como $x^2+2x+2=(x+1)^2+1$, a substituição
$x+1=\tan\theta$ dá
$\int
\frac{dx}{\sqrt{x^2+2x+2}}=\int\frac{\sec^2\theta}{\sec\theta}\,
d\theta=\int\sec\theta\,d\theta=\ln|\sec\theta+\tan\theta|+C=\ln\bigl|x+1+\sqrt{
x^2+2x+2}\bigr|+C$.
\eqref{itSubstitTrig10} Apesar da função $\frac{1}{(x^2+1)^3}$ não possuir
raizes, façamos a substituição $x=\tan\theta$:
\begin{align*}
\int\frac{dx}{(x^2+1)^3}&=\int\frac{\sec^2\theta}{(\tan^2\theta+1)^3}\,
d\theta=\int\frac{d\theta}{\sec^4\theta}=\int\cos^4\theta\,d\theta\,.
\end{align*}
Essa última primitiva já foi calculada em \eqref{eq:intcosquatre}:
$\int \cos^4\theta\,d \theta=\tfrac14\sen
\theta\cos^3\theta+\tfrac{3\theta}{8}+\tfrac{3}{16}
\sen(2\theta)+C$. Ora, se $\tan\theta=x$, então
$\sen\theta=\frac{x}{\sqrt{1+x^2}}$ e $\cos\theta=\frac{1}{\sqrt{1+x^2}}$.
Logo,
$$
\int\frac{dx}{(x^2+1)^3}=\frac{x}{4(1+x^2)^2}+\frac38\Bigl\{\arctan
x+\frac{x}{1+x^2}\Bigr\}+C\,.
$$
\eqref{itSubstitTrig100} Com $x=2\tan \theta$,
$\int\frac{dx}{x^2\sqrt{x^2+4}}=\tfrac14\int
\frac{\cos\theta}{\sen^2\theta}\,d\theta=-\frac{1}{4\sen\theta}+C$.
Agora observe que $2\tan \theta=x$ implica $\sen\theta=\frac{x}{\sqrt{x^2+4}}$.
Logo,
$\int\frac{dx}{x^2\sqrt{x^2+4}}=-\frac{\sqrt{x^2+4}}{4x}+C$.
\end{Solution}
\begin{Solution}{9.32}
Já montamos a integral no Exemplo \ref{ex:comprparabdifiss}, e esta pode ser
calculada com os métodos dessa seção:
$L=2\int_0^1\sqrt{1+4x^2}\,dx=\frac{\sqrt{5}}{4}+\frac12\ln(\frac12+\frac{\sqrt{5}}{2})$.
\end{Solution}
\begin{Solution}{9.33}
\eqref{itPrimSubSecante1}
Seja $x=\sqrt{3}\sec \theta$. Então $dx=\sqrt{3}\sec \theta\tan \theta$, e
\begin{align*}
 \int x^3\sqrt{x^2-3}dx&=\int (\sqrt{3}\sec \theta)^3 \sqrt{3}  \tan \theta
\sqrt{3}\sec \theta\tan \theta d\theta\\
&=\sqrt{3}^5\int \{\sec^2\theta \tan^2 \theta\}\sec^2 \theta d\theta\\
(\text{ com }u=\tan \theta)&=\sqrt{3}^5\int (u^2+1)u^2du\\
&=\sqrt{3}^5(u^5/5+u^3/3)+C
\end{align*}
Mas como $\cos \theta=\sqrt{3}/x$, temos (fazer um desenho) $u=\tan
\theta=\sqrt{x^2-3}/\sqrt{3}$. Logo,
$$
\int x^3\sqrt{x^2-3}dx=\tfrac15\sqrt{x^2-3}^5+\sqrt{x^2-3}^3+C
$$
Um outro jeito de calcular essa primitiva é de começar com uma integração por
partes:
\begin{align*}
\int x^3\sqrt{x^2-3}dx=
 \tfrac12\int x^2\,
\big\{2x\sqrt{x^2-3}\big\}dx&=\tfrac12\Big\{x^2\frac{(x^2-3)^{3/2}}{3/2}-\int
2x\frac{(x^2-3)^{3/2}}{3/2}dx\Big\}\\
&=\tfrac12\Big\{x^2\frac{(x^2-3)^{3/2}}{3/2}-\tfrac23\int
2x{(x^2-3)^{3/2}}dx\Big\}\\
&=\tfrac12\Big\{x^2\frac{(x^2-3)^{3/2}}{3/2}-\tfrac23\frac{(x^2-3)^{5/2}}{5/2}
\Big\}+C\\
&=\tfrac13x^2{(x^2-3)^{3/2}}-\tfrac{2}{15}{(x^2-3)^{5/2}}+C\,\,)
\end{align*}
\eqref{itSubstitTrig8} Com $x=a\sec\theta$,
$\int
\frac{dx}{\sqrt{x^2-a^2}}\,dx=\int\sec\theta\,
d\theta=\ln|\sec\theta+\tan\theta|+C$. Como $\cos\theta=\frac{a}{x}$,
\begin{center}
\begin{bmlimage}\begin{tikzpicture}[scale=1.5]
\draw (0,0)--(2,1) node[midway, above, sloped]{$x$}--(2,0)
node[midway, right]{$\sqrt{x^2-a^2}$} --(0,0) node[midway,
below]{$a$};
\draw (0.4,0) arc (0:26.56:0.4);
\draw (0.56,0.13) node{$\theta$};
\draw (4.8,0.5) node{$\Rightarrow\,\,\tan\theta=\frac{\sqrt{x^2-a^2}}{a}$};
\end{tikzpicture}\end{bmlimage}
\end{center}
Logo,
$\int
\frac{dx}{\sqrt{x^2-a^2}}\,dx=\ln|\tfrac{x}{a}+\tfrac{\sqrt{x^2-a^2}}{a}|+C$.
\eqref{itSubstitTrig1}
Com $x=\sec \theta$, $dx=\sec \theta\tan \theta
d\theta$:
\begin{align*}
\int\frac{x^3}{\sqrt{x^2-1}}dx&=\int
\frac{\sec^3\theta}{\tan \theta}\sec \theta\tan \theta d\theta\\
&=\int \sec^2\theta \sec^2\theta d\theta\\
&=\int (\tan^2\theta+1)\sec^2\theta d\theta\\
(u\pardef \tan \theta)\quad&=\int (u^2+1)du\\
&=\frac{u^3}{3}+u+C\\
&=\frac{\tan^3\theta}{3}+\tan\theta+C\,.
\end{align*}
Mas $\sec \theta=x$ implica $\tan\theta=\sqrt{x^2-1}$.
Logo,
$$
\int\frac{x^3}{\sqrt{x^2-1}}dx=
\frac{1}{3}(x^2-1)^{\tfrac32}+\sqrt{x^2-1}+C\,.
$$

\end{Solution}
\protect \section *{Capítulo \ref {CAP:Applicacoes}}
\begin{Solution}{10.1}
 Representando a metade superior do círculo de raio $R$ centrado na origem com a função $f(x)=\sqrt{R^2-x^2}$, podemos expressar o
comprimento da circunferência como
$$
2\int_{-R}^R\sqrt{1+[(\sqrt{R^2-x^2})']^2}\,dx=2R\int_{-R}^R\frac{dx}{\sqrt{R^2-x^2}}=2R\int_{-1}^1\frac{du}{\sqrt{1-u^2}}=2\pi R\,.
$$
\end{Solution}
\begin{Solution}{10.2}
Lembrando que $\cosh'(x)=\senh x$, que $\cosh^2 x-\senh^2x=1$, e que $\cosh x$ é par,
\begin{align*}
 L=\int_{-1}^1\sqrt{1+(\senh x)^2}\,dx=2\int_{0}^1\cosh x\,dx=2\senh (1)=e-e^{-1}\,.
\end{align*}
\end{Solution}
\begin{Solution}{10.3}
O comprimento é dado por $L=\int_0^1\sqrt{1+e^{2x}}\,dx$.
Se $u=\sqrt{1+e^{2x}}$, então $dx=\frac{u}{u^2-1}du$, logo
$$L=\int_{\sqrt{2}}^{\sqrt{1+e^4}}\frac{u^2}{u^2-1}du
=\int_{\sqrt{2}}^{\sqrt{1+e^4}}1\,du+\int_{\sqrt{2}}^{\sqrt{1+e^4}}\frac{du}{
u^2-1}\,.
$$
Essa última integral pode ser calculada como no Exemplo \ref{Ex:unsurxdeuxmun}:
$\int\frac{du}{
u^2-1}=\tfrac{1}{2}\ln\Bigl|\frac{u-1}{u+1}\Bigr|+C$. Logo,
$$
L=\sqrt{1+e^4}-\sqrt{2}+\tfrac12\ln\Bigl[\frac{\sqrt{1+e^4}-1}{\sqrt{1+e^4}
+1}\cdot\frac{\sqrt{2}+1}{\sqrt{2}-1}\Bigr]\,.
$$
\end{Solution}
\begin{Solution}{10.4}
\eqref{itexsolrev1} A esfera pode ser obtida girando o semi-disco,
delimitado pelo gráfico da função
$f(x)=\sqrt{r^2-x^2}$, $x\in [-r,r]$, em torno do eixo $x$.
\eqref{itexsolrev2} O cilíndro pode ser obtido girando o gráfico da função
constante $f(x)=r$, no intervalo $[0,h]$.
\eqref{itexsolrev3} O cubo não é um sólido de revolução.
\eqref{itexsolrev4} O cone pode ser obtido girando o gráfico da função
$f(x)=\frac{r}{h}x$ (ou $f(x)=r-\frac{r}{h}x$), no intervalo $[0,h]$.
\end{Solution}
\begin{Solution}{10.5}
 $11\pi$
 
\end{Solution}
\begin{Solution}{10.6}
$\tfrac{\pi}{6}$.
\end{Solution}
\begin{Solution}{10.8}
A área é dada por
$$\int_{\pi/2}^\pi\sen (x)dx=-\cos (x)|^{\pi}_{\pi/2}=-(-1)-0=1\,.$$
Girando em torno do eixo $x$:
$V_1=\int_{\pi/2}^{\pi}\pi(\sen x)^2\,dx$.
Ou, com as cascas: $V_1=\int_0^12\pi y (\pi/2-\arcsen y)\,dy$.
Em torno da reta $x=\pi$, usando as cascas:
$V_2=\int_{\pi/2}^\pi2\pi(\pi-x)\sen x\,dx$.
Sem usar as cascas:
$V_2=\pi(\tfrac{\pi}{2})^2\cdot 1-\int_0^1\pi (\arcsen y)^2\,.dy$.
\end{Solution}
\begin{Solution}{10.9}
O cone pode ser (tem vários jeitos, mas esse é o mais simples)
obtido girando o gráfico da função $f(x)=\frac{R}{H}x$, $0\leq x\leq H$, em
torno do eixo $x$. Logo,
$$
V=\int_0^H\pi\big(\frac{R}{H}x\Big)^2dx=\pi\frac{R^2}{H^2}\int_0^Hx^2dx=
\pi\frac{R^2}{H^2}\frac{H^3}{3}=\frac{1}{3}\pi R^2H \,\,$$
Obs: pode também rodar o gráfico da função $f(x)=-\frac{H}{R}x+H$, $0\leq x\leq
R$, em torno do eixo $y$.
\end{Solution}
\begin{Solution}{10.10}
O volume é dado por $V=\int_1^e\pi(\sqrt{x}\ln x)^2dx$. Integrando duas vezes
por partes, obtem-se
\begin{align*}
\int x(\ln x)^2dx&=\frac{x^2}{2}(\ln x)^2-\int \frac{x^2}{2}2(\ln
x)\frac{1}{x}dx\\
&=\frac{x^2}{2}(\ln x)^2-\int x\ln xdx\\
&=\frac{x^2}{2}(\ln x)^2-\big\{\frac{x^2}{2}\ln x-\int\frac{x^2}{2}\frac{1}{x}dx
\big\}\\
&=\frac{x^2}{2}(\ln x)^2-\frac{x^2}{2}\ln x+\frac{x^2}{4}+C
\end{align*}
Logo, $V=\pi\frac{e^2-1}{4}$.
\end{Solution}
\begin{Solution}{10.11}
\eqref{itexorotsol1}
Cil.: $\int_0^1\pi (x^2)^2\,dx$,
Casc.:
$\int_0^12\pi y(1-\sqrt{y})\,dy$.
\eqref{itexorotsol2}
Cil.: $\int_0^1\pi(1^2-(1-x^2)^2)\,dx$
Casc.: $\int_0^12\pi(1-y)(1-\sqrt{y})\,dy$,
\eqref{itexorotsol3}
Cil.: $\int_0^1\pi((1+x^2)^2-1^2)\,dx$
Casc.: $\int_0^1 2\pi(1+y)(1-\sqrt{y})\,dy$
\eqref{itexorotsol4}
Cil.: $\int_0^1\pi(1^2-\sqrt{y}^2)\,dy$
Casc.: $\int_0^12\pi x\cdot x^2\,dx$
\eqref{itexorotsol5}
Cil. $\int_0^1\pi(1-\sqrt{y})^2\,dy$
Casc.: $\int_0^12\pi(1-x)x^2\,dx$
\eqref{itexorotsol6}
Cil.: $\int_0^1\pi(2^2-(1+\sqrt{y})^2)\,dy$
Casc. $\int_0^12\pi(1+x)x^2\,dx$
\end{Solution}
\begin{Solution}{10.12}
Com o método dos cilíndros,
$$V=\int_1^3\pi 2^2dx-\int_1^3\pi\big(2-(1-(x-2)^2)\big)^2dx\,\,.$$
OU, usando o método das cascas,
$$
V=\int_0^12\pi(2-y)2\sqrt{1-y}dy\,.
$$
OU, transladando o gráfico da função, e girando a nova região (finita,
delimitada pela nova curva $y=-1-x^2$ e o eixo $x$),
$$V=\int_{-1}^{+1}\pi 2^2dx-\int_{-1}^{+1}\pi(-1-x^2)^2dx\,.$$
\end{Solution}
\begin{Solution}{10.13}
O volume é dado pela integral
\begin{align*}
V=\int_{-1}^{+1}\pi \cosh^2xdx&=\pi\int_{-1}^{+1}\frac{e^{2x}+2+e^{-2x}}{4}dx\\
&=\frac{\pi}{4}\Big\{
\frac{e^{2x}}{2}+2x-\frac{e^{-2x}}{2}
\Big\}_{-1}^{+1}\\
&=\frac{\pi}{4}\big\{e^2+4-e^{-2}\big\}
\end{align*}
\end{Solution}
\begin{Solution}{10.14}
Em torno da reta $x=\pi$:
$$
V=\int_{\pi/2}^{\pi}2\pi(\pi-x)|\cos x|\,dx\,,\quad\text{ ou }
\quad V=\int_{-1}^0\pi(\pi-\arcos y)^2\,dy\,.
$$
Em torno da reta $y=-1$:
$$
V=\int_{\pi/2}^\pi \pi\cdot 1^2\,dx-\int_{\pi/2}^\pi\pi(\cos
x -(-1))^2\,dx\,,\quad\text{ ou }
\quad V=\int_{-1}^02\pi (y-(-1))(\pi-\arcos y)\,dy\,.
$$
\end{Solution}
\begin{Solution}{10.16}
Se trata de mostrar que
a área lateral de um cone truncado de raios $r\leq R$
e de altura $h$ é dada por
$$
A=\pi (R+r)\sqrt{h^2+(R-r)^2}\,.
$$
De fato, fazendo o corte,
\begin{center}
\begin{bmlimage}\begin{tikzpicture}
\coordinate (A) at (0,0);
\coordinate (B) at (0,1);
\coordinate (C) at (0,3);
\coordinate (D) at (1.333,1);
\coordinate (E) at (2,0);
\draw (A)--(B) node[midway, left]{$h$}--
(C)--(E)--(A) node[midway, above]{$R$};
\draw (B)--(D) node[midway, above]{$r$};
\draw (C) node[left]{$C$};
\draw (D) node[right]{$D$};
\draw (E) node[right]{$E$};
\end{tikzpicture}\end{bmlimage}
\end{center}
Chamando a distância $CD$ de $l$, e a distância $CE$ de $L$, temos
$A=\pi R L-\pi rl$. Uma conta elementar mostra que
$l=\frac{r}{R-r}\sqrt{h^2+(R-r)^2}$, e que
$L=\frac{R}{R-r}\sqrt{h^2+(R-r)^2}$.
Isso dá a fórmula desejada.
\end{Solution}
\begin{Solution}{10.17}
Como a esfera é obtida girando o gráfico de
$f(x)=\sqrt{R^2-x^2}$, a sua área é dada por
$$
A=2\pi\int_{-R}^R\sqrt{R^2-x^2}\sqrt{1+\bigl(\sqrt{R^2-x^2}'\bigr)^2}\,dx
=2\pi R\int_{-R}^R\,dx= 4\pi R^2\,.
$$
\end{Solution}
\protect \section *{Capítulo \ref {CAP:Improprias}}
\begin{Solution}{11.1}
\eqref{itImproprias0} Com $u=x-2$,
$\int_3^\infty\frac{dx}{x-2}=\lim_{L\to \infty}\int_3^L\frac{dx}{x-2}=
\lim_{L\to \infty}\int_1^{L-2}\frac{du}{u}=\lim_{L\to \infty}\ln(L-2)=\infty
$, diverge.
\eqref{itImproprias1} Diverge (é a área da região contida entre a parábola
$x^2$ e o eixo $x$!)
\eqref{itImproprias2} $\int_{1}^\infty
\tfrac{dx}{x^7}=\lim_{L\to\infty}\int_{1}^L \tfrac{dx}{x^7}=
\tfrac16\lim_{L\to\infty}\{1-\frac{1}{L^6}\}=\tfrac16$, logo converge.
\eqref{itImproprias3} Como $\int_0^L\cos x\,dx=\sen L$, e que $\sen L$ não
possui limite quando $L\to \infty$, a integral imprópria $\int_0^\infty\cos
x\,dx$ diverge.
\eqref{itImproprias4} $\int_{0}^\infty \frac{dx}{x^2+1}=\frac{\pi}{2}$, logo
converge.
\eqref{itImproprias5} Temos
$\frac{1}{x^2+x}=\frac{1}{x(x+1)}=\frac{1}{x}-\frac{1}{x+1}$, logo
$$\int_1^L\frac{dx}{x^2+x}=\{\ln x\}|_1^L-\{\ln(x+1)\}|_1^L
=\ln L-\ln(L+1)+\ln 2\,.
$$
Mas como $\lim_{L\to \infty}\{\ln L-\ln(L+1)\}=
\lim_{L\to \infty}\ln\frac{L}{L+1}=\ln 1=0$, temos
$\int_{1}^\infty \frac{dx}{x^2+x}=\ln 2<\infty$, logo converge.
\eqref{itImproprias6} converge.
\eqref{itImproprias65} Com $u=\ln x$, $\int\frac{\ln x}{x}\,dx=\int
u\,du=\frac{u^2}{2}+C$, logo $\int_{3}^\infty \frac{\ln x}{x}\,dx$ diverge.
\eqref{itImproprias7} converge (pode escrever $x^4=u^2$, onde $u=x^2$)
\end{Solution}
\begin{Solution}{11.2}
\eqref{itTRansfLapl1} $L(s)=\frac{k}{s}$.
\eqref{itTRansfLapl2} $L(s)=\frac{1}{s^2}$.
\eqref{itTRansfLapl4} Integrando duas vezes por partes, é fácil
verificar que $L(s)$ satisfaz $L(s)=\frac{1}{s}(\frac{1}{s}-\frac{1}{s}L(s))$.
Logo, $L(s)=\frac{1}{1+s^2}$.
\eqref{itTRansfLapl5} $L(s)=\frac{1}{s+\alpha}$.
\end{Solution}
\begin{Solution}{11.3}
A função tem domínio $\bR$, é ímpar e possui a assíntota horizontal
$y=0$, a
direita e esquerda.
A sua derivada vale $f'(x)=\frac{1-x^2}{(x^2+1)^2}$. Logo, $f$ decresce em
$(-\infty,-1]$, possui um mínimo local em $(-1,\tfrac{1}{2})$, cresce em
$[-1,+1]$, possui um
um máximo local em $(+1,\tfrac{1}{2})$, e decresce em $[1,+\infty)$.
A derivada segunda vale $f''(x)=\frac{2x(x^2-3)}{(x^2+1)^3}$. Logo, $f$ possui
três pontos de inflexão: em $(-\sqrt{3},-\frac{\sqrt{3}}{4})$,
$(0,0)$ e $(\sqrt{3},\frac{\sqrt{3}}{4})$, e é côncava em
$(-\infty,-\frac{\sqrt{3}}{4}]$, convexa em $[-\tfrac{\sqrt{3}}{4},0]$, côncava
em $[0,\tfrac{\sqrt{3}}{4}]$, e convexa em
$[\tfrac{\sqrt{3}}{4},+\infty)$.
\begin{center}
\begin{bmlimage}\begin{tikzpicture}
\newcommand{\funcao}[1]{( 2*(#1)/( (#1)^2 + 1 ))}
\draw[>=latex, ->] (-6,0)--(6,0);
\draw[>=latex, ->] (0,-1.5)--(0,1.5);
\draw[thick, domain=-6:6, samples=70] plot (\x,{\funcao{\x}});
\fill[areagrafico] (0,0)--plot[domain=0:5.5,
samples=50](\x,{\funcao{\x}})--(5.5,0)--cycle;
\draw[thick, domain=-6:6, samples=70] plot (\x,{\funcao{\x}});
\end{tikzpicture}\end{bmlimage}
\end{center}
Vemos que a área procurada é dada pela integral imprópria
$$
\int_0^\infty\frac{x}{x^2+1}dx=\lim_{L\to\infty}\int_0^L\frac{x}{x^2+1}dx=\lim_{
L\to\infty} \ln (L^2+1)=+\infty\,.
$$
\end{Solution}
\begin{Solution}{11.4}
$f$ tem domínio $\bR$, e é sempre positiva. Já que
$$
\lim_{x\to +\infty} \frac{e^x}{1+e^x}=\lim_{x\to+\infty}
\frac{1}{1+e^{-x}}=1\,,\quad
\lim_{x\to-\infty} \frac{e^x}{1+e^x}=0\,,
$$
$f$ tem duas assíntotas horizontais: a reta $y=0$ a esquerda, e a reta $y=1$ a
direita.
Como $f'(x)=\frac{e^x}{(1+e^x)^2}$ é sempre positiva, $f$ é crescente em todo
$x$ (não possui mínimos ou máximos locais). Como
$f''(x)=\frac{e^x(1-e^x)}{(1+e^x)^2}$,
e que essa é positiva quando $x\leq 0$, negativa quando $x\geq 0$, temos que $f$
é convexa em $(-\infty,0]$, côncava em $[0,\infty)$, e possui um ponto de
inflexão em $(0,\tfrac12)$:
\begin{center}
\begin{bmlimage}\begin{tikzpicture}
\newcommand{\funcao}[1]{(exp(#1))/(1+ exp(#1))}
\draw[>=latex, ->] (-6,0)--(6,0);
\draw[>=latex, ->] (0,-0.2)--(0,1.5);
\draw[dashed] (0,1)--(6,1);
\fill[areagrafico] (0,0.5)--plot[domain=0:5.5,
samples=50](\x,{\funcao{\x}})--(5.5,1)--(0,1)--cycle;
\draw[thick, domain=-6:6, samples=60] plot (\x,{\funcao{\x}});
\end{tikzpicture}\end{bmlimage}
\end{center}

A área procurada é dada pela integral imprópria
$$\int_0^\infty\Big\{1-\frac{e^x}{1+e^x}\Big\}dx=\int_0^\infty\frac{1}{1+e^x}
dx$$
Com $u=e^x+1$ dá $du=e^x\,dx=(u-1)\,dx$, e
$$
\int \frac{1}{1+e^x}dx=\int\frac{1}{u(u-1)}du\,.
$$
A decomposição desta última fração dá
$$
\int\frac{1}{u(u-1)}du=-\int\frac{du}{u}+\int \frac{du}{u-1}=-\ln|u|+\ln|u-1|+C
$$
Logo,
\begin{align*}
\int_0^\infty\frac{1}{1+e^x}dx=\lim_{L\to\infty}\int_0^L\frac{1}{1+e^x}dx=
&\lim_{L\to\infty}\Big\{
-\ln (e^x+1)+\ln e^x
\Big\}_0^L\\
&=\lim_{L\to\infty}\Big\{
-\ln (1+e^{-x})
\Big\}_0^L\\
&=\ln 2
\end{align*}
\end{Solution}
\begin{Solution}{11.5}
Considere por exemplo a seguinte função $f$:
\begin{center}
\begin{bmlimage}\begin{tikzpicture}
\pgfmathsetmacro{\n}{7};
\draw[>=latex, ->] (-0.2,0)--({\n+0.5},0);
\draw[>=latex, ->] (0,-0.2)--(0,1.3);
\draw (0,1) node{$-$} node[left]{$1$};
\foreach \k in {1,...,\n} {
\draw (\k,0) node{$\shortmid$} node[below]{$\k$};
\coordinate (A) at (\k,1);
\coordinate (B) at ({\k-(1/(2^(\k)))},0);
\coordinate (C) at ({\k+(1/(2^(\k)))},0);
\fill[areagrafico] (B)--(A)--(C)--cycle;
\draw[thick] (B)--(A)--(C);
}
\end{tikzpicture}\end{bmlimage}
\end{center}
Fora dos triângulos, $f$ vale zero.
O primeiro triângulo tem base de largura $1$, o segundo $\frac{1}{2}$, o
$k$-ésimo $\frac{1}{2^{k-1}}$, etc. Logo, a integral de $f$ é igual à soma
das áreas dos triângulos:
$$
\int_0^\infty f(x)\,dx=\tfrac12+\tfrac14+\tfrac18+\tfrac{1}{16}+\dots=1\,.
$$
Assim, a integral imprópria converge. Por outro lado, já que $f(k)=1$ para
todo inteiro positivo $k$, $f(x)$ não tende a zero quando $x\to \infty$.
\end{Solution}
\begin{Solution}{11.7}
\eqref{itIntImpropp1} Como $\frac{1}{\sqrt{x}^\alpha}=\frac{1}{x^{p}}$ com
$p=\alpha/2$, a integral
converge se e somente se $\alpha>2$.
\eqref{itIntImpropp2} Defina $p:=\alpha^2-3$.
Pelo Teorema \ref{Teo:ConvSerieHarmon}, sabemos que a integral converge se
$p>1$, diverge caso contrário. Logo, a integral converge se
$\alpha>2$ ou $\alpha<-2$, e ela diverge se $-2\leq \alpha\leq 2$.
\eqref{itIntImpropp3} Converge se e somente se $\alpha>1/2$ (pode fazer $u=\ln
x$).
\end{Solution}
\begin{Solution}{11.8}
O volume do sólido é dado pela integral imprópria
$$
V=\pi\int_1^\infty\Bigl(\frac{1}{x^q}\Bigr)^2\,dx=\pi\int_1^\infty\frac{dx}{
x^{2q}}\,.
$$
Pelo Teorema \ref{Teo:ConvSerieHarmon}, essa integral converge se $2q>1$ (isto
é se $q>\tfrac12$), diverge caso contrário.
\end{Solution}
\begin{Solution}{11.9}
\eqref{itIntCompar1} Como $x^2+x\geq x^2$ para
todo $x\in [1,\infty)$, temos também $\frac{1}{x^2+x}\leq \frac{1}{x^2}$ neste
intervalo, logo
$\int_1^\infty\frac{dx}{x^2+x}\leq
\int_1^\infty\frac{dx}{x^2}<\infty$, converge.
\eqref{itIntCompar2} Como $x+1\geq x$ para todo $x\geq 1$,
$\int_1^\infty
\frac{dx}{\sqrt{x}(x+1)}\leq
\int_1^\infty \frac{dx}{\sqrt{x}x}= \int_1^\infty \frac{dx}{x^{3/2}}<\infty$,
converge.
\eqref{itIntCompar3} $\int_0^\infty \frac{dx}{1+e^x}\leq \int_0^\infty
e^{-x}\,dx<\infty$, converge.
\eqref{itIntCompar4} $\int_1^\infty \frac{e^x}{e^x-1}\,dx\geq
\int_1^\infty \frac{e^x}{e^x}\,dx=\int_1^\infty dx=\infty$, diverge.
\eqref{itIntCompar5} Como
$\int_0^\infty\frac{dx}{2x^2+1}=\int_0^1\frac{dx}{2x^2+1}
 +\int_1^\infty\frac{dx}{2x^2+1}$ e
$\int_1^\infty\frac{dx}{2x^2+1}\leq
\int_1^\infty\frac{dx}{2x^2}<\infty$, temos que $\int_0^\infty\frac{dx}{2x^2+1}$
converge.
\eqref{itIntCompar6} Escrevendo
$\tfrac{1}{x^2-1}=\tfrac{1}{x^2}\tfrac{x^2}{x^2-1}$, e observando que o máximo
da função $\tfrac{x^2}{x^2-1}$ no intervalo $[3,\infty)$ é $\tfrac98$, temos
$\int_3^\infty\frac{dx}{x^2-1}\leq \tfrac98
\int_3^\infty\frac{dx}{x^2}<\infty$, logo a integral converge.
Um outro jeito de fazer é de observar que se $x\geq 3$, então $x^2-1\geq x^{3/2}$.
\eqref{itIntCompar22}
Como $\sqrt{x^2+1}\geq \sqrt{x^2}=x$ em todo o intervalo de integração,
$\int_1^\infty\frac{\sqrt{x^2+1}}{x^2}dx\geq
\int_1^\infty\frac{x}{x^2}dx=\int_1^\infty\frac{1}{x}dx$.
Como aqui é uma integral do tipo $\int_1^\infty\frac{1}{x^p}dx$ com $p=1$, ela é
divergente. Logo, pelo critério de comparação,
$\int_1^\infty\frac{\sqrt{x^2+1}}{x^2}dx$ {diverge} também.\\
% (Obs: pode também calcular a primitiva da função, o que dá mais trabalho. Por
% exemplo, por partes,
% \begin{align*}
% \int\frac{1}{x^2}\sqrt{x^2+1}dx&=\frac{-1}{x}\sqrt{x^2+1}-\int(\frac{-1}{x}
% )\frac{2x}{2\sqrt{x^2+1}}dx\\
% &=\frac{-1}{x}\sqrt{x^2+1}+\int\frac{1}{\sqrt{x^2+1}}dx
% \end{align*}
% Usando $x=\sec\theta$ nesta última integral dá
% $\int\frac{1}{\sqrt{x^2+1}}dx=\int \sec\theta
% d\theta=(\cdots)=\ln|x+\sqrt{x^2+1}|+C$. Logo,
% $$\int\frac{1}{x^2}\sqrt{x^2+1}dx=\ln|x+\sqrt{x^2+1}|-\frac{\sqrt{x^2+1}}{x}+C\,
% .$$
% Calculando aquele limite, obtem-se  também que a integral diverge.)
\eqref{itIntCompar7} $\int_1^\infty\frac{x^2-1}{x^4+1}\,dx\leq
\int_1^\infty\frac{x^2}{x^4}\,dx=\int_1^\infty\frac{1}{x^2}\,dx<\infty$,
converge.
\eqref{itIntCompar8} Como $\sen x\geq -1$,
$\int_1^\infty\frac{x^2+1+\sen x}{x}\,dx\geq
\int_1^\infty\frac{x^2}{x}\,dx
=\int_1^\infty x\,dx=\infty$, diverge.
\eqref{itIntCompar9}
Como $\ln x \geq 2$ para todo $x\geq e^2$, temos que
$\int_{e^2}^\infty e^{-(\ln x)^2}\,dx\leq
\int_{e^2}^\infty e^{-2\ln x}\,dx=\int_{e^2}^\infty\frac{dx}{x^2}$, que converge.
\end{Solution}
\begin{Solution}{11.10}
 Observe que se $0\leq x<1$, então $e^{-x^2/2t}\leq 1$,
e se $x\geq 1$, então $x^2\geq x$, logo
$e^{-x^2/2t}\leq e^{-x/2t}$. Logo,
$$\int_{0}^\infty
e^{-\frac{x^2}{2t}}\,dx= \int_{0}^1
e^{-\frac{x^2}{2t}}\,dx+ \int_{1}^\infty
e^{-\frac{x^2}{2t}}\,dx \leq \int_0^1 \,dx+\int_1^\infty
e^{-x/2t}\,dx\,.$$
Como essa última integral converge (ela pode ser calculada
explicitamente), por comparação $\int_{0}^\infty
e^{-\frac{x^2}{2t}}\,dx$ converge também. Como $x\mapsto
e^{-x^2/2t}$ é par, isso implica que $f(t)$ é bem definida.
Com a mudança $y=x/\sqrt{t}$, temos
$$
 \frac{1}{\sqrt{2\pi t}}\int_{0}^\infty
e^{-\frac{x^2}{2t}}\,dx= \frac{1}{\sqrt{2\pi}}\int_{0}^\infty
e^{-\frac{y^2}{2}}\,dy\,,
$$
que não depende de $t$. Assim, $f$ é constante.
\end{Solution}
\begin{Solution}{11.11}
\eqref{itIntimpropr0} Por definição,
$\int_{0}^{1^-}\frac{dx}{\sqrt{1-x}}=\lim_{\epsilon\to
0^+}\int_0^{1-\epsilon}\frac{dx}{\sqrt{1-x}}=
\lim_{\epsilon\to 0^+}\{-2\sqrt{1-x}\}_0^{1-\epsilon}=2$. Logo, a integral
converge.
\eqref{itIntimpropr1} $\int_{0^+}^1\frac{\ln(x)}{\sqrt{x}}dx=\lim_{\epsilon\to
0^+}\int_\epsilon^1\frac{\ln(x)}{\sqrt{x}}dx$.
Integrando por partes, definindo $f'(x)\pardef \frac{1}{\sqrt{x}}$, $g(x)\pardef
\ln
(x)$, temos $f(x)=2\sqrt{x}$, $g'(x)=\frac{1}{x}$, e
\begin{align*}
\int \frac{\ln(x)}{\sqrt{x}}dx
=2\sqrt{x}\ln (x)-2\int \frac{\sqrt{x}}{x}dx
&=2\sqrt{x}\ln (x)-2\int \frac{1}{\sqrt{x}}dx\\
&=2\sqrt{x}\ln (x)-4\sqrt{x}+C\,.
\end{align*}
(Obs: pode também começar com $u=\sqrt{x}$, e acaba calculando $4\int
\ln(u)du$.)
Logo,
\begin{align*}
\int_{0^+}^1\frac{\ln(x)}{\sqrt{x}}dx&=\lim_{\epsilon\to 0^+}
\big\{
2\sqrt{x}\ln (x)-4\sqrt{x}+C
\big\}_\epsilon^1\\
&=\lim_{\epsilon\to 0^+}
-4-2\sqrt{\epsilon}\ln (\epsilon)+4\sqrt{\epsilon}=-4\,.\\
\end{align*}
Este último passo é justificado porqué $\lim_{\epsilon\to
0^+}\sqrt{\epsilon}=0$, e porqué uma simples aplicação da Regra de
Bernoulli-l'Hôpital dá $\lim_{\epsilon\to 0^+}\sqrt{\epsilon}\ln
(\epsilon)=-\lim_{y\to +\infty}\frac{\ln (y)}{\sqrt{y}}=0$.
Como o limite existe e é finito, a integral imprópria acima { converge e o
seu valor é $-4$}.

\eqref{itIntimpropr2}
Observe que a função $\frac{1}{\sqrt{e^t-1}}$ não é definida em $t=0$, logo é
necessário dividir a integral em duas integrais impróprias:
\begin{align*}
\int_{0^+}^\infty\frac{1}{\sqrt{e^t-1}}dt&=\int_{0^+}^1\frac{1}{\sqrt{e^t-1}}
dt+\int_1^\infty\frac{1}{\sqrt{e^t-1}}dt\\
&=\lim_{\epsilon\to 0^+}\int_\epsilon^1\frac{1}{\sqrt{e^t-1}}dt+\lim_{L\to
\infty}\int_1^L\frac{1}{\sqrt{e^t-1}}dt\,.
\end{align*}
Para calcular a primitiva, seja $u=\sqrt{e^t-1}$,
$du=\frac{e^t}{2\sqrt{e^t-1}}dt$, i.é. $dt=\frac{2u}{u^2+1}du$, e
\begin{align*}
\int\frac{1}{\sqrt{e^t-1}}dt=2\int\frac{du}{u^2+1} &=2\arctan (u)+C\\
&=2\arctan\sqrt{e^t-1}+C
\end{align*}
Logo,
$$\lim_{\epsilon\to
0^+}\int_\epsilon^1\frac{1}{\sqrt{e^t-1}}dt=2\lim_{\epsilon\to
0^+}\arctan\sqrt{e^t-1}\big|_\epsilon^1=2\arctan\sqrt{e-1}$$
$$\lim_{L\to \infty}\int_1^L\frac{1}{\sqrt{e^t-1}}dt=2\lim_{L\to
\infty}\arctan\sqrt{e^t-1}\big|_1^L=\pi-2\arctan\sqrt{e-1}$$
Como esses dois limites existem,
$\int_0^\infty\frac{dt}{\sqrt{e^t-1}}$ {converge, e o seu valor é
$\pi$}.
\end{Solution}
\par
}
\fi


\cleardoublepage
\phantomsection
\iflatexml
%\chapter*{(vazio)}
\else
\addcontentsline{toc}{chapter}{\indexname}
\fi
\printindex


\end{document}

