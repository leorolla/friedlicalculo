
% !TeX spellcheck = pt_BR
% !TEX encoding = UTF-8 Unicode

\chapter{Integral}\label{CAP:Integral}

\ifdefined\updateans
% Only need to run once in a lifetime, when the file ans.tex needs to be updated.
\Writetofile{ans}{\protect\section*{Capítulo \ref{CAP:Integral}}}
\fi

O problema original e fundamental do \emph{cálculo integral} era
de \emph{calcular comprimentos, áreas, e volumes} de objetos geométricos no
plano ou no espaço, em particular de objetos 
mais gerais do que aqueles
considerados em geometria elementar que são retângulos,
triângulos, círculos (no plano), ou paralelepípedos, cones,
esferas (no espaço).\\

O maior avanço no cálculo integral 
veio com os trabalhos de Newton e Leibniz
no fim do século XVI, em que a noção de derivada tem papel fundamental.
Os métodos desenvolvidos por Newton e Leibniz
tornaram a integral uma ferramenta com inúmeras aplicações, bem além da 
geometria, em todas as áreas da ciência e da engenharia.
\\

Nesse capítulo introduziremos a noção de \emph{integral} para uma função $f$ de
uma variável real~\footnote{Integrais \emph{múltiplas} serão estudadas em
Cálculo III.} $x$, a partir da Seção \ref{Sec:IntRiemann}.
O \emph{Teorema Fundamental do Cálculo} 
(Teoremas \ref{Teo:TFC} e \ref{Teo:TFC2}) será provado na Seção
\ref{Sec:TeoremaFundamental}.

\section{Introdução}

\emph{Como calcular, em geral, a área de uma região limitada do plano?}
Para sermos um pouco mais específicos, faremos a mesma pergunta para áreas
delimitadas pelo gráfico de uma função. \emph{Dada uma função positiva
$f:[a,b]\to \bR$,
como calcular a área debaixo do seu gráfico, isto é, a área da região $R$,
delimitada pelo gráfico de $f$, pelo eixo $x$, e pelas retas $x=a$, $x=b$?}
\index{gráfico! área debaixo de um}
\begin{center}
\begin{bmlimage}\begin{tikzpicture}[scale=1]
%\clip (0.5,0.5) rectangle (1.5,1.5);
\newcommand{\funcao}[1]{(0.05*(#1)^3-0.2*(#1)^2+2)}
\fill[areagrafico] (1,0)--plot[domain=1:3](\x,{\funcao{\x}})--(3,0)--cycle;
\draw [dotted, domain=0.5:3.5] plot (\x,{\funcao{\x}});
\draw [thick, domain=1:3] plot (\x,{\funcao{\x}});
\draw (2,0.8) node {$R$};
\draw (1,-0.22) node{$a$};
\draw (3,-0.2) node{$b$};
\draw (4,1.8) node {$f(x)$};
\draw [>=latex, ->] (0.5,0)--(3.5,0) node[right] {$x$};
\draw [dashed] (3,-0.05)--(3,1.52);
\draw [dashed] (1,-0.05)--(1,{\funcao{1}});
\end{tikzpicture}\end{bmlimage}
\end{center}


Para as funções elementares a seguir, a resposta pode ser dada sem muito
esforço.
Por exemplo, se $f$ é constante, $f(x)=h>0$, $R$ é um retângulo, logo

\begin{center}
\begin{bmlimage}\begin{tikzpicture}[scale=0.9]
%\clip (0.5,0.5) rectangle (1.5,1.5);
\fill[areagrafico] (1,0)--plot[domain=1:3](\x,{1.5})--(3,0)--cycle;
\draw [domain=0.5:3.5] plot (\x,{1.5});
\draw (2,0.8) node {$R$};
%\draw (1,0) node {$\shortmid$};
\draw (1,-0.22) node{$a$};
%\draw (3,0) node {$\shortmid$};
\draw (3,-0.2) node{$b$};
\draw (3.8,1.5) node {$h$};
\draw [>=latex, ->] (0.5,0)--(3.5,0) node[right] {$x$};
\draw [dashed] (1,-0.05)--(1,1.5);
\draw [dashed] (3,-0.05)--(3,1.5);
\draw [>=latex, ->] (0.7,-0.2)--(0.7,1.9) ; 
%\draw[very thin, gray] (0,0) grid[step=1] (5,2);
\draw (5,0.8) node[right] {$\Rightarrow \,\text{área}(R)=\text{base}\times 
\text{altura}=(b-a)h$};
\end{tikzpicture}\end{bmlimage}
\end{center}

Por outro lado, se o gráfico de $f$ for uma reta, por exemplo $f(x)=mx$ 
com $m>0$, e se $0<a<b$, então $R$ é um trapézio, e a sua área pode ser
 escrita como a diferença das áreas de dois triângulos (lembre o
Exercício \ref{Exo:primeiraarea}):

\begin{center}
\begin{bmlimage}\begin{tikzpicture}[scale=0.9]
%\clip (0.5,0.5) rectangle (1.5,1.5);
\fill[areagrafico] (1,0)--plot[domain=1:3](\x,0.6*\x)--(3,0)--cycle;
\draw [dotted, domain=-0.1:3.2] plot (\x,{0.6*\x});
\draw [thick, domain=1:3] plot (\x,{0.6*\x});
\draw (2,0.5) node {$R$};
%\draw (1,0) node {$\shortmid$};
\draw (1,-0.22) node{$a$};
%\draw (3,0) node {$\shortmid$};
\draw (3,-0.2) node{$b$};
%\draw (3.8,1.5) node {$h$};
\draw [>=latex, ->] (0,0)--(3.5,0) node[right] {$x$};
\draw [dashed] (1,-0.05)--(1,0.6);
\draw [dashed] (3,-0.05)--(3,1.8);
\draw [>=latex, ->] (0,0)--(0,2) ; 
%\draw[thick, gray] (0,0) grid[step=1] (5,2); \draw[very thin, gray] (0,0) 
%grid[step=0.2] (5,2);
\draw[dotted]  (1,0.6)--(0,0.6) node[left]{$ma$};
\draw[dotted]  (3,1.8)--(0,1.8) node[left]{$mb$};
\draw (4.2,0.8) node[right] {$\Rightarrow \,\text{área}(R)=\half b\times
mb-\half
 a\times ma=\half m(b^2-a^2)$};
\end{tikzpicture}\end{bmlimage}
\end{center}

O nosso último exemplo ``simples'' 
será $f(x)=\sqrt{1-x^2}$, com $a=0$, $b=1$. Neste caso 
reconhecemos a região $R$ como 
a sendo o quarto do disco de raio $1$ centrado na origem, contido no primeiro
quadrante:
\index{disco}
\begin{center}
\begin{bmlimage}\begin{tikzpicture}[scale=0.8]
%\clip (0.5,0.5) rectangle (1.5,1.5);
%\fill[color=gray!30] (1,0)--plot[domain=1:3](\x,0.6*\x)--(3,0)--cycle;
%\draw [domain=-0.1:3.2] plot (\x,{0.6*\x});
%\draw (1,0) node {$\shortmid$};
\draw (0,-0.2) node{$0$};
%\draw (3,0) node {$\shortmid$};
\draw (2,-0.22) node{$1$};
%\draw (0,-1)--(0,0)--(-1,0) ;
\fill[areagrafico] (0,0)--(2,0) arc (0:90:2)--cycle;
\draw[dotted] (2,0) arc (0:360:2);
\draw (2,0) arc (0:90:2);
\draw (0.8,0.8) node {$R$};
%\draw (-0.7,-0.7) node {$\pi/2$};
%\draw (0,0) node {$\bullet$};
%\draw (3.8,1.5) node {$h$};
\draw [>=latex, ->] (0,0)--(2.5,0) node[right] {$x$};
%\draw [dashed] (1,-0.05)--(1,0.6);
%\draw [dashed] (3,-0.05)--(3,1.8);
\draw [>=latex, ->] (0,0)--(0,2.2) ; 
%\draw[thick, gray] (0,0) grid[step=1] (5,2); \draw[very thin, gray] (0,0) grid[step=0.2] (5,2);
%\draw[dotted]  (1,0.6)--(0,0.6) node[left]{$ma$};
%\draw[dotted]  (3,1.8)--(0,1.8) node[left]{$mb$};
\draw (3.2,1) node[right] {$\Rightarrow \,\text{área}(R)=\tfrac14 \times\pi
1^2=\tfrac{\pi}{4}$};
\end{tikzpicture}\end{bmlimage}
\end{center}

Consideremos agora $f(x)=1-x^2$, 
também com $a=0$, $b=1$:

\begin{center}
\begin{bmlimage}\begin{tikzpicture}[scale=1.5]
\fill[areagrafico] (0,1)--plot[domain=0:1](\x,{1-(\x)^2})--(0,0)--cycle;
\draw [dotted, domain=-0.5:1.1] plot (\x,{1-(\x)^2});
\draw [domain=0:1] plot (\x,{1-(\x)^2});
\draw (0,0) node[below]{$0$};
\draw (0,1) node[left]{$1$};
\draw (1,0) node[below]{$1$};
\draw (0.4,0.4) node {$R$};
\draw [>=latex, ->] (-0.5,0)--(1.3,0) node[right] {$x$};
\draw [>=latex, ->] (0,0)--(0,1.2); 
%\draw[thick, gray] (0,0) grid[step=1] (2,2); \draw[very thin, gray] (0,0) grid[step=0.2] (2,2);
%\draw[dotted]  (1,0.6)--(0,0.6) node[left]{$ma$};
%\draw[dotted]  (3,1.8)--(0,1.8) node[left]{$mb$};
\draw (1.7,0.5) node[right] {$\Rightarrow \,R=\,?$};
\end{tikzpicture}\end{bmlimage}
\end{center}

Apesar da função $f(x)=1-x^2$ ser elementar, não vemos um jeito simples
de decompor $R$ em um número finito de regiões simples do tipo
retângulo, triângulo, ou disco.\\

No entanto, o que pode ser feito 
é \emph{aproximar $R$ por regiões mais
simples}, a começar com retângulos~\footnote{Já encontramos esse tipo de
construção, mas com triângulos, no Exercício
\ref{Exo:DecomporCircemTriang}.}.
Começemos aproximando $R$ de maneira grosseira, usando uma região $R_2$
formada por dois retângulos, da seguinte maneira:
\index{aproximação! por retângulos}
\begin{center}
\begin{bmlimage}\begin{tikzpicture}[scale=1.8]
\pgfmathsetmacro{\numretangulos}{2}
\foreach \k in {1,...,\numretangulos} {
\pgfmathsetmacro{\cantinho}{\k/\numretangulos}
\pgfmathsetmacro{\altura}{1-((\k-1)/\numretangulos)^2}
\fill[corretangulos] (\cantinho-1/\numretangulos,0) rectangle (\cantinho,\altura);
\draw (\cantinho-1/\numretangulos,0) rectangle (\cantinho,\altura);
\fill (\cantinho-1/\numretangulos,\altura) circle (0.15mm);
}
\draw [dotted, domain=0:1] plot (\x,{1-(\x)^2});
\draw (0,0) node[below]{$0$};
\draw (1,0) node[below]{$1$};
\draw (0.5,0) node[below]{$\half$};
\draw[dotted]  (0.5,0.75)--(0,0.75) node[left]{$1-(\half)^2=\tfrac34$};
\draw [>=latex, ->] (0,0)--(1.3,0) node[right] {$x$};
\draw [>=latex, ->] (0,0)--(0,1.2); 
\draw (1.7,0.5) node[right] {$\Rightarrow \,
\text{área}(R_2)=\bigl\{\half\times
1\big\}+\bigl\{\half \times \tfrac34\big\}=\tfrac78$};
\end{tikzpicture}\end{bmlimage}
\end{center}

A área de $R_2$ é a soma das áreas dos dois retângulos de bases
iguais $\tfrac12$ mas de alturas diferentes: 
o canto esquerdo superior do primeiro retângulo está em $(0,1)$, e o do segundo
foi escolhido \emph{no gráfico de $1-x^2$}, no ponto $(\tfrac12,\tfrac34)$.
Logo, $\text{área}(R_2)=\tfrac78$.
É claro que $\text{área}R_2$ somente dá uma \emph{estimativa}:
$\text{área}(R)<\text{área}R_2$.\\

Tentaremos agora melhorar 
essa aproximação: fixemos um inteiro $n\in \bN$, e
aproximemos
$R$ pela região $R_n$ formada pela união de $n$ retângulos de larguras iguais a
$1/n$, mas com alturas 
escolhidas tais que o canto superior esquerdo esteja sempre \emph{na curva}
$1-x^2$. Por exemplo, 
se $n=5$, ${15}$ e ${25}$,

\begin{center}
\begin{bmlimage}\begin{tikzpicture}[scale=1.8]

\newcommand{\parabole}[1]{
\pgfmathsetmacro{\numretangulos}{#1}
\foreach \k in {1,...,\numretangulos}
{\pgfmathsetmacro{\cantinho}{\k/\numretangulos}
\pgfmathsetmacro{\altura}{1-((\k-1)/\numretangulos)^2}
\fill[corretangulos] (\cantinho-1/\numretangulos,0) rectangle
(\cantinho,\altura);
\draw (\cantinho-1/\numretangulos,0) rectangle (\cantinho,\altura);
\fill (\cantinho-1/\numretangulos,\altura) circle (0.15mm);
}
\draw [dotted, domain=0:1] plot (\x,{1-(\x)^2});
%\draw (0,1) node[left]{$1$};
%\draw (0,0) node[below]{$0$};
%\draw (1,0) node[below]{$1$};
\draw [>=latex, ->] (0,0)--(1.1,0);
\draw [>=latex, ->] (0,0)--(0,1.2);
}

\begin{scope}
 \parabole{5}
\end{scope}

\begin{scope}[xshift=2cm]
 \parabole{15}
\end{scope}

\begin{scope}[xshift=4cm]
 \parabole{25}
\end{scope}

\end{tikzpicture}\end{bmlimage}
\end{center}

Vemos que quanto maior o número de retângulos $n$, melhor a aproximação da
verdadeira área de $R$.
Logo, tentaremos calcular $\text{área}(R)$ via um \emph{limite}:
$$\text{área}(R)=\lim_{n\to \infty}\text{área}(R_n)\,.$$
Olhemos os retângulos de mais perto. Por exemplo, para calcular
$\text{área}(R_5)$, calculemos a soma das áreas de $5$ retângulos:
\begin{align*}
\text{área}(R_5)&=\tfrac15\big(1-(\tfrac{0}{5})^2)
+\tfrac15\big(1-(\tfrac{1}{5})^2)
+\tfrac15\big(1-(\tfrac{2}{5})^2)
+\tfrac15\big(1-(\tfrac{3}{5})^2)
+\tfrac15\big(1-(\tfrac{4}{5})^2)\\
&=1-\tfrac{1^2+2^2+3^2+4^2}{5^3}(=0.76)\,.
\end{align*}

Para um $n$ qualquer,
\begin{align}
\text{área}(R_n)&=\tfrac1n\big(1-(\tfrac{0}{n})^2)
+\tfrac1n\big(1-(\tfrac{1}{n})^2)+\dots
+\tfrac1n\big(1-(\tfrac{n-2}{n})^2)
+\tfrac1n\big(1-(\tfrac{n-1}{n})^2)\nonumber\\
&=1-\tfrac{1^2+2^2+\dots+(n-2)^2+(n-1)^2}{n^3}\,.\label{eq:somaquadrados}
\end{align}

Pode ser mostrado (ver Exercício \ref{exo:provaporinducao}) que para todo
$k\geq 1$,
\eq{\label{eq:inducnnn}1^2+2^2+\dots+k^2=\frac{k(k+1)(2k+1)}{6}\,.}
Usando essa expressão em \eqref{eq:somaquadrados} com $k=n-1$, obtemos
\begin{align*}
\text{área}(R)=\lim_{n\to \infty}\text{área}(R_n)&=1-\lim_{n\to \infty}
\frac{(n-1)((n-1)+1)(2(n-1)+1)}{6n^3}\\
&=1-\lim_{n\to \infty}
\frac{n(n-1)(2n-1)}{6n^3}\\
&=1-\tfrac{1}{3}\\
&=\tfrac{2}{3}\,.
\end{align*}

\begin{obs}
É interessante observar que no limite $n\to\infty$, o número de retângulos que
aproxima $R$ tende ao infinito,
mas que a área de cada um tende a zero. Assim podemos dizer, informalmente, que
depois do processo de limite, a área exata de $R$
é obtida ``somando infinitos retângulos de largura zero''.
\end{obs}


\begin{exo}\label{exo:provaporinducao}
Mostre por indução que para todo $n\geq 1$, 
$$1+2+3+\dots+n=\frac{n(n+1)}{2}\,,\quad
1^2+2^2+\dots+n^2=\frac{n(n+1)(2n+1)}{6}\,.$$
\end{exo}

\begin{exo}
Considere a aproximação da área $R$ tratada acima, usando retângulos cujo canto
superior \emph{direito} sempre fica na curva $y=1-x^2$, e mostre que quando 
$n\to\infty$, o limite é o mesmo: $\tfrac23$.
\end{exo}


O método usado para calcular a área debaixo de $1-x^2$ 
funcionou graças à fórmula \eqref{eq:inducnnn}, que
permitiu transformar a soma dos $k$ primeiros quadrados em um
polinômio de grau $3$ em $k$. Essa fórmula foi particularmente bem adaptada à
função $1-x^2$, mas não será útil em outras situações. 
Na verdade, são poucos casos em que a conta pode ser feita ne maneira
explícita.

\begin{ex}
Considere $f(x)=\cos(x)$ entre $a=0$ e $b=\pi/2$.
\begin{center}
\begin{bmlimage}\begin{tikzpicture}[scale=1.5]
\fill[areagrafico] (0,1)--plot[domain=0:1.57](\x,{cos(\x r)})--(0,0)--cycle;
\draw [dotted, domain=0:1.57] plot (\x,{cos(\x r)});
\draw [domain=-0.1:1.59] plot (\x,{cos(\x r)});
\draw (0,0) node[below]{$0$};
\draw (1.57,0) node[below]{$\tfrac{\pi}{2}$};
\draw (0.6,0.4) node {$R$};
\draw [>=latex, ->] (-0.1,0)--(1.7,0) node[right] {$x$};
\draw [>=latex, ->] (0,0)--(0,1.2); 
\end{tikzpicture}\end{bmlimage}
\end{center}
Neste caso, uma aproximação da área $R$ debaixo do gráfico por 
retângulos de largura $\tfrac1n$ dá:
\begin{align}
\text{área}(R_n)&=\tfrac1n\cos(\tfrac{1}{n})
+\tfrac1n\cos(\tfrac{2}{n})+\dots
+\tfrac1n\cos(\tfrac{\frac{n\pi}{2}}{n})\,.
\end{align}
Para calcular o limite $n\to\infty$ desta soma, o leitor interessado 
pode começar verificando por indução~\footnote{Fonte: Folhetim de
Educação Matemática, Feira de Santana, Ano 18, Número 166, junho de
2012.} que para todo $a>0$ e todo inteiro $k$,
\[
\tfrac12+\cos(a)+\cos(2a)+\cos(3a)+\dots+\cos(ka)=\frac{\sen(\frac{2k+1}{2}a)}{2\sen(\frac{a}{2})}\,.
\]
Usando esta fórmula com $a$ e $n$ bem escolhidos, pode 
mostrar que  $\lim_{n\to \infty}\text{área}(R_n)=1$. Portanto,
$\text{área}(R)=1$.
\end{ex}

\begin{exo}
Considere $f(x)=e^x$ entre $a=0$ e $b=1$.
Monte $\text{área}(R_n)$ usando retângulos de largura $\tfrac1n$.
Usando
\[
1+r+r^2+\dots+r^n=\frac{1-r^n}{1-r}\,,
\]
calcule $\lim_{n\to\infty}\text{área}(R_n)$.
\begin{sol}
A soma associada dá, usando a fórmula sugerida,
\[
\text{área}(R_n)=\frac{e^0}{n}+\frac{e^{1/n}}{n}
+\frac{e^{2/n}}{n}+\dots+\frac{e^{(n-1)/n}}{n}
=\frac{e-1}{\frac{e^{1/n}-1}{1/n}}\,.
\]
Mas $\lim_{n\to\infty}\frac{e^{1/n}-1}{1/n}=\lim_{t\to
0^+}\frac{e^t-1}{t}=1$. Logo,
$\text{área}(R)=e-1$.
\end{sol}
\end{exo}

O que foi feito nesses últimos exemplos foi calcular uma área por um
procedimento chamado \emph{integração}. 
Mais tarde, desenvolveremos um método que permite calcular
integrais usando um método completamente diferente. Mas
antes disso precisamos definir o que significa \emph{integrar} de
maneira mais geral.

\section{A integral de Riemann}\label{Sec:IntRiemann}

De modo geral, a área da região $R$ delimitada pelo gráfico de uma função
$f:[a,b]\to \bR$ pode ser definida via um processo de limite,
como visto acima no caso de $f(x)=1-x^2$.\\

Primeiro, 
escolhemos um inteiro $n$, e escolhemos pontos distintos em $(a,b)$:
$x_0\equiv a<x_1<x_2<\dots<x_{n-1}<x_n\equiv b$. 
Esses pontos formam uma \grasA{partição} de $[a,b]$.
Em seguida, escolhemos um
ponto $x_j^*$ em cada intervalo $[x_{j-1},x_{j}]$, e definimos a \grasA{soma
de \index{Riemann (Georg Friedrich)} 
Riemann~\footnote{Georg Friedrich Bernhard Riemann, 1826 – 1866.}} $I_n$ por:
\begin{center}
 \begin{bmlimage}\begin{tikzpicture}[scale=3]
\newcommand{\funcao}[1]{( 0.5+ ((#1)^2-(#1)^3)/2 )}
\pgfmathsetmacro{\numintervalos}{15}
\pgfmathsetmacro{\a}{-0.6}
\pgfmathsetmacro{\b}{1.2}
\pgfmathsetmacro{\incr}{(\b-\a)/\numintervalos}
\foreach \k in {1,...,\numintervalos}
{\pgfmathsetmacro{\cantinho}{\a+(\k-1)*\incr}
\pgfmathsetmacro{\pontoaleat}{\cantinho+(rnd*\incr)}
\pgfmathsetmacro{\altura}{\funcao{\pontoaleat}}
\fill[corretangulos] (\cantinho,0) rectangle (\cantinho+\incr,\altura);
\draw (\cantinho,0) rectangle (\cantinho+\incr,\altura);
\draw[dotted] (\pontoaleat,0)--(\pontoaleat,\altura);
\fill (\pontoaleat,\altura) circle (0.15mm);
}
\draw (\a,0) node{$\shortmid$} node[below] {$a$};
\draw (\b,0) node{$\shortmid$} node[below] {$b$};
\draw (\a-0.2,0)--(\b+0.2,0) ;
\draw [thick, domain=\a-0.1:\b+0.1] plot (\x,{\funcao{\x}}) node[right]{$f$};
\draw (-1,0.3) node[left]{$\displaystyle{I_n\pardef
\sum_{j=1}^nf(x_j^*)\Delta x_{j}}\,,$};
 \end{tikzpicture}\end{bmlimage}
\end{center}
$I_n$ aproxima a área debaixo do gráfico pela soma das áreas dos retângulos, em
que o $j$-ésimo retângulo tem como base $\Delta x_j\pardef x_{j}-x_{j-1}$, e
como altura \emph{o valor da função no ponto $x_j^*$}: $f(x_j^*)$. (Na imagem
acima os pontos $x_i$ foram escolhidos equidistantes,
$\Delta x_{j}=\tfrac{b-a}{n}$.)\\

A \emph{integral} de $f$ é obtida considerando $I_n$ para uma sequência de
partições em que o tamanho dos intervalos $\Delta x_j$
tendem a zero:
\begin{defin}\index{função! integrável}
A função $f:[a,b]\to \bR$ é \grasA{integrável} se o limite $\lim_{n\to
\infty}I_n$ existir, qualquer que seja a 
sequência de partições em que $\max_j\Delta x_j\to 0$, e
qualquer que seja a escolha de $x_j^*\in [x_{j-1},x_{j}]$. Quando $f$ é
integrável, o limite $\lim_{n\to \infty}I_n$
é chamado de \grasA{integral (de Riemann) de $f$}, ou \grasA{integral definida
de $f$}, e denotado\index{integral! de Riemann}
\eq{\label{eq:DefinIntRiem}\lim_{n\to \infty}I_n\equiv \int_a^bf(x)dx\,.}
Os números $a$ e $b$ são chamados de \grasA{limites de integração}.
\end{defin}\index{limite! de integração}

Inventada por Newton, 
a notação  ``$\int_a^bf(x)dx$'' lembra que a integral é definida a partir de
uma
\emph{soma} (o ``$\int$'' é parecido com um ``s'') de retângulos
contidos entre $a$ e $b$, de áreas $f(x^*_j)\Delta x_j$ (o
``$f(x)dx$''). 

\begin{obs}
É importante lembrar que $\int_a^bf(x)dx$ \emph{é um número, não uma função}:
a variável ``$x$'' que aparece em $\int_a^bf(x)dx$ é usada somente
para indicar que $f$ está sendo integrada, com a sua variável
varrendo o intervalo $[a,b]$. 
Logo, seria equivalente escrever essa
integral $\int_a^bf(t)dt$, $\int_a^bf(z)dz$, etc., ou simplesmente $\int_a^bf\,
dx$. Por isso, a variável $x$ que aparece em \eqref{eq:DefinIntRiem} é
\index{variável! muda}
chamada de \emph{muda}.
\end{obs}

\begin{obs} A definição de integrabilidade faz sentido mesmo se $f$ não é
positiva.
Neste caso, o termo $f(x_j^*)\Delta x_{j}$ da soma de Riemann não pode ser mais
interpretado
como a área do $j$-ésimo retângulo, e $\int_a^bf\,dx$
não possui necessariamente uma interpretação geométrica.
O Exercício \ref{Exo:IntegraleNegative} abaixo esclarece esse ponto.
\end{obs}

Enunciemos algumas propriedades básicas da integral, que podem ser provadas a
partir da definição.
\index{integral! propriedades da}
\begin{pro}\label{Prop:ProprIntegral} Seja $f:[a,b]\to \bR$ integrável.
\begin{enumerate}
 \item\label{itProprIntegr1} Se $\lambda\in \bR$ é uma
 constante, então $\lambda f$ é
integrável, e $\int_a^b\lambda f\, dx=\lambda\int_a^bf\,dx$.
\item\label{itProprIntegr2} Se $g:[a,b]\to \bR$ também é integrável, então 
$f+g$ é integrável e
$\int_a^b(f+g)dx=\int_a^bf\,dx+\int_a^bg\,dx$.
\item\label{itProprIntegr3} Se $a<c<b$, então
$\int_a^cf\,dx+\int_c^bf\,dx=\int_a^bf\,dx$.
\end{enumerate}
\end{pro}

Observe que se $f$ é uma constante, $f(x)=c$, então qualquer soma de Riemann
pode ser calculada via um retângulo só, e
\eq{\label{eq:integrconstante} \int_a^bf(x)\,dx=c(b-a)\,.}
Mais tarde precisaremos da seguinte propriedade:

\begin{pro} Se $f$ e $g:[a,b]\to \bR$ são integráveis,
e se $f\leq g$, então 
 \eq{\label{eq:comparacaointegrais}\int_a^bf\,dx\leq \int_a^bg\,dx\,.}
Em particular, se $f$ é limitada,
$M_-\leq f(x)\leq M_+$ para todo $x\in [a,b]$, então 
\eq{\label{ineq_estim_int}M_-(b-a)\leq \int_a^bf\,dx\leq M_+(b-a)\,.}
\end{pro}
Para funções positivas, a interpretação de \eqref{eq:comparacaointegrais} em
termos de áreas é imediata: se o gráfico de $f$ está sempre abaixo do gráfico
de $g$, então a área debaixo de $f$ é menor do que a área abaixo de $g$.

\begin{exo}
Justifique as seguintes afirmações:
\begin{enumerate}
\item Se $f$ é par, $\int_{-a}^af(x)\,dx=2\int_0^af(x)\,dx$.
\item Se $f$ é ímpar, $\int_{-a}^af(x)\,dx=0$.
\end{enumerate}
\end{exo}

Em geral, verificar se uma função é integrável pode ser difícil. O seguinte
resultado garante que
as maioria das funções consideradas no restante do
curso \emph{são} integráveis.

\begin{teo}
 Se $f:[a,b] \to \bR$ é contínua, então ela é integrável.
\end{teo}

Por exemplo, $f(x)=1-x^2$ é contínua, logo integrável, e vimos na
introdução que
$$\int_0^1(1-x^2)dx=\tfrac23\,.$$
Sabendo que uma função contínua é integrável, queremos um jeito de 
\emph{calcular} a sua integral.
Mas como já foi dito, o procedimento de limite descrito acima (calcular a soma
de Riemann, tomar o limite $n\to \infty$, etc.) é díficil de
se implementar, mesmo se $f$ é simples.

\section{O Teorema Fundamental do Cálculo}\label{Sec:TeoremaFundamental}

Suponha que se queira calcular a integral de uma função
contínua $f:[a,b]\to \bR$: 
\begin{center}
\begin{bmlimage}\begin{tikzpicture}[scale=0.9]
\newcommand{\funcao}[1]{( 2- (0.2*( ( (#1) -1.4))^2))}
\fill[areagrafico]
(0.5,0)--plot[domain=0.5:3.5](\x,{\funcao{\x}})--(3.5,0)--cycle;
\draw [dotted, domain=0:4] plot (\x,{\funcao{\x}});
\draw [thick, domain=0.5:3.5] plot (\x,{\funcao{\x}});
\draw[dotted] (0.5,-0.1)--(0.5,{\funcao{0.5}});
 \draw (0.5,0) node[below]{$a$};
\draw[dotted] (3.5,-0.1)--(3.5,{\funcao{3.5}});
 \draw (3.5,0) node[below]{$b$};
\draw[>=latex, ->] (0,0)--(4.3,0);
 \draw (0,1.7) node[left]{$f(x)$};
\draw (6,1) node[right]{$\displaystyle{I= \int_a^bf(t)dt\,.}$};
\end{tikzpicture}\end{bmlimage}
\end{center}
Podemos supor sem perda de generalidade que $f\geq
0$, o que deve ajudar a entender geometricamente alguns dos raciocínios a
seguir. Para calcular $I$ passaremos pelo estudo de uma função auxiliar,
chamada de \grasA{função área}, definida da seguinte maneira:
\index{função área}
\begin{center}
\begin{bmlimage}\begin{tikzpicture}[scale=0.9]
\newcommand{\funcao}[1]{( 2- (0.2*( ( (#1) -1.4))^2))}
%\fill[areagrafico]
%(0.5,0)--plot[domain=0.5:3.5](\x,{\funcao{\x}})--(3.5,0)--cycle;
\fill[areafuncaoarea]
(0.5,0)--plot[domain=0.5:2.5](\x,{\funcao{\x}})--(2.5,
0)--cycle;
\draw [dotted, domain=0:4] plot (\x,{\funcao{\x}});
\draw [thick, domain=0.5:3.5] plot (\x,{\funcao{\x}});
\draw[dotted] (0.5,-0.1)--(0.5,{\funcao{0.5}});
 \draw (0.5,0) node[below]{$a$};
\draw[dotted] (3.5,-0.1)--(3.5,{\funcao{3.5}});
 \draw (3.5,0) node[below]{$b$};
\draw (2.5,-0.1)--(2.5,{\funcao{2.5}});
 \draw (2.5,0) node[below]{$x$};
\draw[>=latex, ->] (0,0)--(4.3,0);
 \draw (0,1.7) node[left]{$f(x)$};
 \draw (1.5,1) node{$I(x)$};
\draw (6,1) node[right]{$\displaystyle{I(x)\pardef \int_a^xf(t)dt\,.}$};
\end{tikzpicture}\end{bmlimage}
\end{center}
Isto é, $I(x)$ representa a área debaixo do gráfico de $f$,
entre as retas verticais em $a$ (fixa) e em $x$ (móvel).
Como $f$ é positiva, $x\mapsto I(x)$ é crescente.
Além disso, $I(a)=0$, e a integral original procurada é
$I(b)\equiv I$.

\begin{ex}
Se $f(x)=mx$, a função área pode ser calculada explicitamente:
\begin{center}
\begin{bmlimage}\begin{tikzpicture}[scale=1]
\pgfmathsetmacro{\m}{0.6};
\fill[areafuncaoarea] (1,0)--plot[domain=1:2.5](\x,\m*\x)--(2.5,0)--cycle;
\draw [dashed, domain=-0.1:3.2] plot (\x,{\m*\x});
\draw [thick, domain=1:2.5] plot (\x,{\m*\x});
\draw (1.8,0.5) node {$I(x)$};
%\draw (1,0) node {$\shortmid$};
\draw (1,-0.22) node{$a$};
%\draw (3,0) node {$\shortmid$};
\draw (3,-0.2) node{$b$};
\draw (2.5,\m*2.5)--(2.5,0) node[below]{$x$};
\draw [>=latex, ->] (-0.5,0)--(3.5,0);
\draw [>=latex, ->] (0,-0.5,0)--(0,2);
\draw [dotted] (1,-0.05)--(1,0.6);
\draw [dotted] (3,-0.05)--(3,1.8);
\draw (4.2,0.8) node[right] {$I(x)=\half m(x^2-a^2)$};
\end{tikzpicture}\end{bmlimage}
\end{center}
Podemos observar que 
$$I'(x)=\bigl(\tfrac12 m(x^2-a^2)\bigr)'=mx\equiv f(x)\,!$$
\end{ex}

\begin{exo} Calcule as funções área associadas às 
funções $f:[0,1]\to \bR$ abaixo.
\begin{multicols}{3}
\begin{enumerate}
\item\label{itExFuncArea1} $\displaystyle{f(x)=
\begin{cases}
 0&\text{ se }x\leq \frac12\,,\\
 1&\text{ se }x> \frac12\,.
\end{cases}
}$
%$f(x)=0$ se $x\leq \frac12$, $1$ se $x>\frac12$
\item\label{itExFuncArea2} $f(x)=-x+1$
\item\label{itExFuncArea3} $f(x)=2x-1$
\end{enumerate}
\end{multicols}
\vspace{0.01cm}
\begin{sol}
\eqref{itExFuncArea1} $I(x)=0$ se $x\leq \frac12$, $I(x)=(x-\frac12)$ se
$x>\frac12$
\eqref{itExFuncArea2} $I(x)=-\frac{x^2}{2}+x$
\eqref{itExFuncArea3} $I(x)=x^2-x$.
\end{sol}
\end{exo}

A relação entre $I$ e $f$ é surpreendentemente simples:
\index{Teorema Fundamental do Cálculo|textbf}
\index{função!contínua}
\begin{teo}[Teorema Fundamental do Cálculo]\label{Teo:TFC}
 Seja $f:[a,b]\to \bR$ contínua. Então a função área $I:[a,b]\to \bR$,
definida por $I(x)\pardef \int_a^xf(t)dt$
é derivável em todo $x\in (a,b)$, e a sua derivada é igual a $f$:
\eq{I'(x)=f(x)\,.}
\end{teo}
O seguinte desenho deve ajudar a entender a prova:
\begin{center}
\begin{bmlimage}\begin{tikzpicture}[scale=2]
\newcommand{\funcao}[1]{ 2- (0.2*( ( (#1) -1.4))^2)}
\pgfmathsetmacro{\e}{1.5};
\pgfmathsetmacro{\c}{2.5};
\pgfmathsetmacro{\a}{2.9};
\fill[areafuncaoarea]
(\e-0.2,0)--plot[domain=\e-0.2:\a](\x,{\funcao{\x}})--(\a,
0)--cycle;
\draw[dotted] (\c,0) rectangle (\a,{\funcao{\c}});
\draw [thick, domain=\e-0.4:\a+0.2] plot (\x,{\funcao{\x}});
\draw[dashed] (\c,0)--(\c,{\funcao{\c}});
\draw[decorate, decoration=brace] (\c,0)--(\c,{\funcao{\c}})
node[midway, left]{$f(x)$};
\draw[dashed] (\a,0)--(\a,{\funcao{\a}});
\draw (\c,0) node[below]{$\scriptstyle{x}$};
\draw (\a,0) node[below]{$\scriptstyle{x+h}$};
\draw[decorate, decoration=brace]
(\c,{\funcao{\c}})--(\a,{\funcao{\c}})node[above, midway]{$h$};
\draw[>=latex, ->] ({\e-0.4},0)--({\a+0.5},0);
\draw (4,1) node[right]{$\displaystyle{\Rightarrow\, I(x+h)\simeq I(x)+f(x)\cdot
h}$};
\draw (4,0.3) node[right]{$\displaystyle{\Rightarrow\,
\frac{I(x+h)-I(x)}{h}\simeq f(x)}$};
\end{tikzpicture}\end{bmlimage}
\end{center}
De fato, entre $x$ e $x+h$, a função área $I$ cresce de uma
quantidade que pode ser aproximada, quando $h>0$ é pequeno, 
pela área do retângulo pontilhado, cuja base é $h$ e altura $f(x)$.
Isso sugere 
\eq{\label{eq:letrucmuch}\lim_{h\to 0^+}\frac{I(x+h)-I(x)}{h}=f(x)\,.}
\begin{proof}
Seja $x\in (a,b)$. Provemos \eqref{eq:letrucmuch} 
(o limite $h\to 0^-$ se trata da mesma maneira).
Pela propriedade \eqref{itProprIntegr3} da Proposição \ref{Prop:ProprIntegral},
$$I(x+h)=\int_a^{x+h}f(t)\,dt=\int_a^xf(t)\,dt+\int_{x}^{x+h}f(t)\,dt=
I(x)+\int_{x}^{x+h}f(t)\,dt\,.$$
Observe também que por \eqref{eq:integrconstante}, $f(x)$ pode
ser escrito como a diferença
$f(x)=\frac{1}{h}f(x)\int_x^{x+h}\,dt=\frac{1}{h}\int_x^{x+h}f(x)\,dt$.
Logo, \eqref{eq:letrucmuch} é equivalente a mostrar que
\eq{\label{eq:maaachin}
\frac{I(x+h)-I(x)}{h}-f(x)=\tfrac{1}{h}\int_x^{x+h}(f(t)-f(x))dt}
tende a zero quando $h\to 0$.
Como $f$ é contínua em $x$, sabemos que para todo $\epsilon>0$,
$-\epsilon\leq f(t)-f(x)\leq +\epsilon$, desde que $t$ seja suficientemente
perto de $x$.
Logo, para $h>0$ suficientemente pequeno, a integral em
\eqref{eq:maaachin} pode ser limitada por 
$$
-\epsilon =\tfrac1h\int_x^{x+h}(-\epsilon)\,dt
\leq \tfrac{1}{h}\int_x^{x+h}(f(t)-f(x))dt\leq 
\tfrac1h\int_x^{x+h}(+\epsilon)\,dt=+\epsilon\,.
$$
(Usamos \eqref{ineq_estim_int}.)
Isso mostra que \eqref{eq:maaachin} fica arbitrariamente pequeno  quando $h\to
0^+$, o que prova \eqref{eq:letrucmuch}.
\end{proof}

Assim, provamos que integral e derivada são duas noções intimamente ligadas, já
que a função área é \emph{uma função derivável cuja
derivada é igual a $f$}. 

\begin{defin}
Seja $f$ uma função. Se $F$ é uma função derivável tal que $$F'(x)=f(x)$$ 
para todo $x$, então $F$ é chamada \grasA{primitiva de} $f$.
\index{primitiva|textbf}
\end{defin}

\begin{ex} Se $f(x)=x$, então
$F(x)=\frac{x^2}{2}$ é primitiva de $f$, já que
$$F'(x)=\bigl(\frac{x^2}{2}\bigr)'=\tfrac12(x^2)'=\tfrac12 2x=x\,.$$
Observe que como $(\frac{x^2}{2}+1)'=x$,
$G(x)=\frac{x^2}{2}+1$ é \emph{também} primitiva de $f$.
\end{ex}

\begin{ex}
Se $f(x)=\cos x$, então $F(x)=\sen x$ é primitiva de $f$. Observe que
$G(x)=\sen x+14$ e $H(x)=\sen x-7$ também são primitivas de $f$.
\end{ex}

Os dois exemplos acima mostram que \emph{uma função admite infinitas
primitivas}, e que aparentemente duas primitivas de uma mesma função somente
diferem por uma constante:

\begin{lem}
 Se $F$ e $G$ são duas primitivas de uma mesma função $f$, então existe uma
constante $C$ tal que $F(x)-G(x)=C$ para todo $x$.
\end{lem}
\begin{proof}
Defina $m(x)\pardef F(x)-G(x)$. Como $F'(x)=f(x)$ e $G'(x)=f(x)$, temos 
$m'(x)=0$ para todo
$x$. Considere dois pontos $x_1<x_2$ quaisquer. Aplicando o Corólário
\eqref{Corol:ValorIntermDeriv} a $m$ no intervalo $[x_1,x_2]$: existe $c\in
[x_1,x_2]$ tal que $\frac{m(x_2)-m(x_1)}{x_2-x_1}=m'(c)$. Como $m'(c)=0$, temos
$m(x_2)=m(x_1)$. Como isso pode ser feito para qualquer ponto $x_2<x_1$, temos
que $m$ toma o mesmo valor em qualquer ponto, o que implica que é uma função
constante.
\end{proof}

Em geral, escreveremos uma primitiva genérica de $f(x)$ como
$$F(x)=\text{primitiva}+C\,,$$ 
para indicar que é sempre possível adicionar uma constante $C$ arbitrária.

\begin{exo}\label{Exo:PrimitivasBasicas}
Ache as primitivas das funções abaixo.
\begin{multicols}{4}
\begin{enumerate}
\item\label{itExoPrimitTriv0} $-2$
\item\label{itExoPrimitTriv1} $x$
\item\label{itExoPrimitTriv2} $x^2$
\item\label{itExoPrimitTriv3} $x^n$ ($n\neq -1$)
\item\label{itExoPrimitTriv35} $\sqrt{1+x}$
\item\label{itExoPrimitTriv5} $\cos x$
\item\label{itExoPrimitTriv6} $\sen x$
\item\label{itExoPrimitTriv7} $\cos (2x)$
\item\label{itExoPrimitTriv9} $e^x$
\item\label{itExoPrimitTriv95} $1-e^{-x}$
\item\label{itExoPrimitTriv10} $e^{2x}$
\item\label{itExoPrimitTriv105} $3xe^{-x^2}$
\item\label{itExoPrimitTriv8} $\frac{1}{\sqrt{x}}$
\item\label{itExoPrimitTriv4} $\frac1x$, $x>0$
\item\label{itExoPrimitTriv11} $\frac{1}{1+x^2}$
\item\label{itExoPrimitTriv12} $\frac{1}{\sqrt{1-x^2}}$
\end{enumerate}
\end{multicols}
\vspace{0.01cm}
\begin{sol}
\eqref{itExoPrimitTriv0} $-2x+C$
\eqref{itExoPrimitTriv1} $\frac{x^2}{2}+C$
\eqref{itExoPrimitTriv2} $\frac{x^3}{3}+C$
\eqref{itExoPrimitTriv3} $\frac{x^{n+1}}{n+1}+C$
\eqref{itExoPrimitTriv35} $\tfrac{2}{3}(1+x)^{3/2}+C$
\eqref{itExoPrimitTriv5} $\sen x+C$
\eqref{itExoPrimitTriv6} $-\cos x+C$
\eqref{itExoPrimitTriv7} $\frac{1}{2}\sen (2x)+C$
\eqref{itExoPrimitTriv9} $e^x+C$
\eqref{itExoPrimitTriv95} $x+e^{-x}+C$
\eqref{itExoPrimitTriv10} $\tfrac12 e^{2x}+C$
\eqref{itExoPrimitTriv105} $-\tfrac32e^{-x^2}+C$
\eqref{itExoPrimitTriv8} $2\sqrt{x}+C$
\eqref{itExoPrimitTriv4} $\ln x+C$
\eqref{itExoPrimitTriv11} $\arctan x+C$
\eqref{itExoPrimitTriv12} Com $-1<x<1$, $\arcsen x+C$
\end{sol}
\end{exo}

\begin{exo}
Mostre que $(2x^2-2x+1)e^{2x}$ é primitiva da função $4x^2e^{2x}$.
\end{exo}


Mais tarde olharemos de mais perto o problema de calcular primitivas.
Voltemos agora ao nosso problema:
\index{Teorema Fundamental do Cálculo}
\begin{teo}[Teorema Fundamental do Cálculo]\label{Teo:TFC2}
Seja $f:[a,b]\to \bR$ uma função contínua, e $F$ uma primitiva de $f$.
Então
\eq{\label{eq:TFUNDAM}\int_a^bf(t)\,dt=F(b)-F(a)\equiv F(x)\big|_{a}^b\,.}
\end{teo}

\begin{proof} Lembre que $\int_a^bf(t)\,dt=I(b)$, onde $I(x)$ é a função
área. Ora, sabemos pelo Teorema \ref{Teo:TFC} que $I(x)$ é primitiva
de $f$. Assim, $I(x)=F(x)+C$, onde $F(x)$ é uma primitiva qualquer de $f$, e
onde se trata de achar o valor de $C$.
Mas $I(a)=0$ implica $F(a)+C=0$, logo $C=-F(a)$, e $I(x)=F(x)-F(a)$. Em
particular, $I(b)=F(b)-F(a)$.
\end{proof}

\begin{ex}
Considere $I=\int_0^1x^2dx$, que representa
a área debaixo do gráfico da parábola $y=f(x)=x^2$, entre $x=0$ e $x=1$. 
Como $F(x)=\frac{x^3}{3}$ é primitiva de $f$, temos 
$$\int_0^1x^2\,dx=\frac{x^3}{3}\Big|_{0}^1=\frac{1^3}{3}-\frac{0^3}{3}=\frac{1}{
3} \,.$$
Podemos também calcular a integral da introdução, dessa vez usando o Teorema
Fundamental:
$$\int_0^1(1-x^2)\,dx=\int_0^11\,dx-\int_0^1x^2\,dx=1-\tfrac13=\tfrac23\,.$$
\end{ex}


\begin{exo}\label{Exo:IntegraleNegative}
Mostre que $\int_0^2(x-1)\,dx=0$. Como interpretar esse resultado
geometricamente?
\begin{sol}
Como $\tfrac{x^2}{2}-x$ é primitiva de $f(x)=x-1$, temos
$\int_0^2(x-1)\,dx=(\tfrac{x^2}{2}-x)|_0^2=0$.  
Esse resultado pode ser interpretando decompondo a integral em duas partes: 
$\int_0^2f(x)\,dx=\int_0^1f(x)\,dx+\int_1^2f(x)\,dx$.
Esboçando o gráfico de $f(x)$ entre $0$ e $2$,
\begin{center}
\begin{bmlimage}\begin{tikzpicture}
\fill[areagrafico] (1,0)--(2,1)--(2,0)--cycle;
\fill[areafuncaoarea] (1,0)--(0,-1)--(0,0)--cycle;
\draw (1.65,0.25) node{$+$};
\draw (0.35,-0.3) node{$-$};
\draw (1,0) node{$\shortmid$} node[above]{$1$};
\draw[dashed] (2,0)node[below]{$2$}--(2,1);
\draw[dashed] (0,0)--(0,-1);
\draw[>=latex, ->] (-0.3,0)--(2.4,0);
\draw[>=latex, ->] (0,-1.2)--(0,1.3);
\draw[thick] (0,-1)--(2,1);
\end{tikzpicture}\end{bmlimage}
\end{center}
Vemos que a primeira parte 
$\int_0^1f(x)\,dx=-\tfrac12$ é a contribuição do intervalo em
que $f$ é \emph{negativa}, e é exatamente
compensada pela contribuição da parte \emph{positiva}
$\int_1^2f(x)\,dx=+\tfrac12$.
\end{sol} 
\end{exo}

\begin{exo}\label{exo_TFCnaoseaplica}
A seguinte conta está certa? Justifique.
\[
\int_{-1}^2\frac{1}{x^2}\,dx=\bigl(-\frac{1}{x}\bigr)\Big|_{-1}^2=-\tfrac32\,.
\]
\begin{sol}
Não, a conta não está certa. É porqué a função $\frac{1}{x^2}$ não é
contínua (nem definida) em $0$, ora $0$ pertence ao intervalo de
integração. Logo, o Teorema Fundamental não se aplica.
No entanto, será possível dar um sentido a
$\int_{-1}^2\frac{1}{x^2}\,dx$, usando \emph{integrais impróprias}.
\end{sol}
\end{exo}

O Teorema Fundamental mostra que se uma primitiva de $f$ é
conhecida, então a integral de $f$ em qualquer intervalo $[c,d]$ pode ser
obtida, calculando simplesmente $F(d)-F(c)$. 
Isto é, o problema de calcular integral é reduzido ao de achar uma primitiva de
$f$. 
Ora, \emph{calcular uma primitiva} é uma operação mais complexa do que calcular
uma derivada. De fato, calcular
uma derivada significa simplesmente aplicar mecanicamente as regras de
derivação descritas no Capítulo \ref{Cap:Derivacao}, enquanto uma certa
ingeniosidade pode ser necessária para achar uma primitiva, mesmo de uma
função simples como $\sqrt{1+x^2}$ ou $\ln x$.\\

Portanto, estudaremos \emph{técnicas} para calcular primitivas, ao longo
do capítulo. Por enquanto, vejamos primeiro como usar integrais para calcular
áreas mais gerais do plano.

\section{Áreas de regiões do plano}
\index{área! de região do plano}


Sejam $f$ e $g$ duas funções definidas no mesmo intervalo $[a,b]$, tais que
$g(x)\leq f(x)$ para todo $x\in [a,b]$. Como calcular a área da região $R$
contida entre os gráficos das duas funções, delimitada lateralmente pelas
retas verticais $x=a$ e $x=b$?

\begin{center}
\begin{bmlimage}\begin{tikzpicture}[scale=0.8]
\newcommand{\funcaof}[1]{ 4+sin(((#1)-1) r) }
\newcommand{\funcaog}[1]{ 1+0.2*(#1) }
\draw[>=latex, ->] (0,0)--(5,0);
\pgfmathsetmacro{\a}{1};
\pgfmathsetmacro{\b}{4};
\coordinate (A) at (\a,{\funcaof{\a}});
\coordinate (B) at (\b,{\funcaof{\b}});
\coordinate (C) at (\a,{\funcaog{\a}});
\coordinate (D) at (\b,{\funcaog{\b}});
\coordinate (P) at ({\a+0.4*(\a+\b)/2},{\funcaof{\a+0.4*(\a+\b)/2}});
\draw (P) node[above]{$f$}; 
\coordinate (Q) at ({\a+0.7*(\a+\b)/2},{\funcaog{\a+0.7*(\a+\b)/2}});
\draw (Q) node[below]{$g$}; 
\draw[thick, domain=\a:\b] plot (\x,{\funcaof{\x}});
\draw[thick, domain=\a:\b] plot (\x,{\funcaog{\x}});
\fill[areagrafico, opacity=0.8]
(C)--(A)--plot[domain=\a:\b] (\x,{\funcaof{\x}})--(B)--
(D)--plot[domain=\b:\a] (\x,{\funcaog{\x}})--cycle;
\draw[dashed] (A)--(C); \draw[dashed] (B)--(D);
\draw[>=latex, ->] (0,0)--(5,0);
\draw (\a,0) node{$\shortmid$} node[below]{$a$};
\draw (\b,0) node{$\shortmid$} node[below]{$b$};
\draw[dotted] (\a,0)--(A);
\draw[dotted] (\b,0)--(B);
\end{tikzpicture}\end{bmlimage}
\end{center}

Por uma translação vertical, sempre podemos supor que 
$0\leq g\leq f$. Logo, a área de
$R$ pode ser obtida calculando primeiro a área debaixo do gráfico de $f$, que
vale $\int_a^bf\,dx$, da qual se subtrai a área debaixo do gráfico de $g$,
que vale $\int_a^bg\,dx$. 
\eq{\label{eq:areaentregraficos}\text{área}(R)=\int_a^bf\,dx-\int_a^b
g\,dx\equiv \int_a^b(f-g)\,dx\,.}

\begin{ex}
Considere a região finita $R$ delimitada pela parábola $y=2-x^2$ e pela
reta $y=-x$:
\begin{center}
\begin{bmlimage}\begin{tikzpicture}[scale=0.8]
\newcommand{\funcaof}[1]{ 2-(#1)^2 }
\newcommand{\funcaog}[1]{ -1*(#1) }
\draw[>=latex, ->] (-2,0)--(3,0);
\draw[>=latex, ->] (0,-2)--(0,2.5);
\pgfmathsetmacro{\a}{-1};
\pgfmathsetmacro{\b}{2};
\coordinate (A) at (\a,{\funcaof{\a}});
\coordinate (B) at (\b,{\funcaof{\b}});
\coordinate (C) at (\a,{\funcaog{\a}});
\coordinate (D) at (\b,{\funcaog{\b}});
\coordinate (P) at ({\a+0.7*(\b-\a)},{\funcaof{\a+0.7*(\b-\a)}});
\draw (P) node[above right]{$y=2-x^2$}; 
\coordinate (Q) at ({\a+0.7*(\b-\a)},{\funcaog{\a+0.7*(\b-\a)}});
%\draw (Q) node[below]{$g$};
\fill[areagrafico, opacity=0.8]
(C)--(A)--plot[domain=\a:\b] (\x,{\funcaof{\x}})--(B)--
(D)--plot[domain=\b:\a] (\x,{\funcaog{\x}})--cycle;
\draw[thick, domain=-1.5:2.2] plot (\x,{\funcaof{\x}});
\draw[thick, domain=2.3:-1.5] plot (\x,{\funcaog{\x}}) node[left]{$y=-x$};
% \draw (C) node[left]{$C$};
% \draw (B) node[right]{$B$};
% \draw (D) node[right]{$D$};
\draw[dashed] (A)--(C); \draw[dashed] (B)--(D);
\draw[>=latex, ->] (0,0)--(5,0);
\draw (\a,0) node{$\shortmid$} node[below]{$-1$};
\draw (\b,0) node{$\shortmid$} node[above]{$2$};
\draw[dotted] (\a,0)--(A);
\draw[dotted] (\b,0)--(B);
 \draw (0.5,0.5) node{$R$};
\end{tikzpicture}\end{bmlimage}
\end{center}
Pode ser verificado que os pontos de interseção entre as duas
curvas são $x=-1$ e $x=2$. Observe também que no intervalo $[-1,2]$, a parábola
está sempre \emph{acima} da reta.
Logo, por \eqref{eq:areaentregraficos}, a área de
$R$ é dada pela integral 
$$
\int_{-1}^2\bigl((2-x^2)-(-x)\bigr)\,dx=
\int_{-1}^2\bigl(-x^2+x+2\bigr)\,dx=\Bigl(
-\frac{x^3}{3}+\frac{x^2}{2}+2x
\Bigr)\Big|_{-1}^2=\tfrac92\,.
$$
\end{ex}

\begin{exo}
Esboce e calcule a área da região delimitada pelas curvas abaixo.
\begin{multicols}{2}
\begin{enumerate}
\item\label{itareaRbas1}  $y=-2$, $x=2$, $x=4$, $y=\half x-1$.
\item\label{itareaRbas2} $y=-2$, $x=2$, $x=4$, $y=\half (x-2)^2$.
\item\label{itareaRbas3} $y=x^2$, $y=-(x+1)^2+1$.
\item \label{itareaRbas4} $y=0$, $x=1$, $x=e$, $y=\tfrac1x$.
\item \label{itareaRbas5} $y=-2$,  $y=4+x-x^2$.
\end{enumerate}
\end{multicols}
\vspace{0.01cm}
\begin{sol}
\eqref{itareaRbas1} $5$, 
\eqref{itareaRbas2} $\frac{16}{3}$,
\eqref{itareaRbas3} $\frac{1}{3}$,
\eqref{itareaRbas4} $1$.
\eqref{itareaRbas5} $\tfrac{125}{6}$.
\end{sol}
\end{exo}



\begin{ex} Considere a área da região finita delimitada pelas curvas $x=1-y^2$
e $x=5-5y^2$. 
\begin{center}
\begin{bmlimage}\begin{tikzpicture}
\newcommand{\funcaof}[1]{ 1-(#1)^2 }
\newcommand{\funcaog}[1]{ 5-5*(#1)^2 }
\draw[>=latex, ->] (-1,0)--(6,0);
\draw[>=latex, ->] (0,-1.5)--(0,1.5);
\pgfmathsetmacro{\a}{-1};
\pgfmathsetmacro{\b}{1};
\coordinate (A) at ({\funcaof{\a}},\a);
\coordinate (B) at ({\funcaof{\b}},\b);
\coordinate (C) at ({\funcaog{\a}},\a);
\coordinate (D) at ({\funcaog{\b}},\b);
\coordinate (P) at ({\funcaof{\a+0.7*(\b-\a)}},{\a+0.7*(\b-\a)});
\coordinate (Q) at ({\funcaog{\a+0.7*(\b-\a)}},{\a+0.7*(\b-\a)});
%\draw (Q) node[below]{$g$};
\fill[areagrafico, opacity=0.8]
(C)--(A)--plot[domain=\a:\b] ({\funcaof{\x}},\x)--(B)--
(D)--plot[domain=\b:\a] ({\funcaog{\x}},\x)--cycle;
\draw[thick, domain=-1.3:1.3] plot ({\funcaof{\x}},\x);
\draw[thick, domain=-1.1:1.1] plot ({\funcaog{\x}},\x);
\draw[dashed] (A)--(C); \draw[dashed] (B)--(D);
\draw[>=latex, ->] (0,0)--(5,0);
\draw (0,\a) node{-} node[above left]{$-1$};
\draw (0, \b) node{-} node[below left]{$1$};
\draw[dotted] (0,\a)--(A);
\draw[dotted] (0,\b)--(B);
\draw (P) node[right]{$\scriptstyle{x=1-y^2}$}; 
\draw (Q) node[above right]{$\scriptstyle{x=5-5y^2}$};
% \draw (0.5,0.5) node{$R$};
\end{tikzpicture}\end{bmlimage}
\end{center}
%\begin{sol}
Neste caso, é mais natural expressar a área procurada como um
integral \emph{com respeito a $y$}. Como função de $y$, as curvas são parábolas:
$x=f(y)$ com $f(y)=5-5y^2$ e $x=g(y)$ com $f(y)=1-y^2$, e o gráfico de $f(y)$
está sempre acima do gráfico de $g(y)$. Logo, 
a área procurada é dada por 
 $\int_a^b[f(y)-g(y)]dy$, que vale 
$$\int_{-1}^{1}\big\{(5-5y^2)-(1-y^2)\big\}dy=\int_{-1}^{1}\big\{4-4y^2\big\}
dy=\big\{4y-\tfrac43y^3\big\}\Big|_{-1}^1=\tfrac{16}{3}\,.$$
%\end{sol}
\end{ex}


\begin{exo}(3a prova, primeiro semestre de 2011)
Calcule a área da região finita delimitada pelo gráfico da função $y=\ln x$ e
pelas retas $y=-1$, $y=2$, $x=0$.
\begin{sol}\mbox{}
\begin{center}
\begin{bmlimage}\begin{tikzpicture}[scale=0.6]
\fill[areagrafico]
(0,-1)--(0.368,-1)--plot[domain=0.368:7.38](\x,{ln(\x)})--(0,2)--cycle;
\draw [thick, domain=0.3:8, samples=80] plot (\x,{ln(\x)}) node[right] {$\ln
x$};
\draw [>=latex, ->] (-0.5,0)--(4,0) node[right] {$x$};
\draw [>=latex, ->] (0,-1.2)--(0,2.5);
\draw [dotted] (0,-1)--(0.368,-1);
\draw [dotted] (0,2)--(7.38,2);
\draw (0,-1) node[left]{$-1$};
\draw (0,2) node[left]{$2$};
\draw (8,0) node[right]{$A=\int_{-1}^2e^ydy=e^2-e^{-1}\,.$};
\end{tikzpicture}\end{bmlimage}
\end{center}
Observe que expressando a área com uma integral com respeito a $x$, 
$$A=\int_0^{e^{-1}}(2-(-1))dx+\int_{e^{-1}}^{e^2}(2-\ln x)
dx\,.$$
Essa integral requer a primitiva de $\ln x$, o que não
sabemos (ainda) fazer.
\end{sol}
\end{exo}

%\newpage
\begin{exo} Fixe $\alpha>0$.
Considere $f_\alpha(x)\pardef \alpha^{-2}e^{-\alpha}(\alpha^2-x^2)$. Esboce
$x\mapsto f_\alpha(x)$ para diferentes valores de $\alpha$ (em particular para
$\alpha$
pequeno e grande). Determine o valor de $\alpha$ que maximize a área delimitada
pelo gráfico de $f_\alpha$ e pelo eixo $x$.
\begin{sol}
Consideremos $f_\alpha$ para diferentes valores de $\alpha$:
\begin{center}
\begin{bmlimage}\begin{tikzpicture}[scale=1.3]
\newcommand{\funcao}[2]{ ( exp(-1*(#1))/((#1)^2) )*( (#1)^2 - (#2)^2)}

\foreach \a in {0.3, 0.6,1,2} {
\fill[areagrafico, opacity=0.8] (-\a,0)--
plot[domain=-\a:\a] (\x,{\funcao{\a}{\x}})--(\a,0)--cycle;
}

\foreach \a in {0.3, 0.6,1,2} {
\draw[thick, domain=-\a:\a, samples=50] plot (\x,{\funcao{\a}{\x}});
}

\draw[>=latex,->] (-2.3,0)--(2.3,0);
\draw[>=latex,->] (0,-0.1)--(0,1.3);
\end{tikzpicture}\end{bmlimage}
\end{center}
A área debaixo do gráfico de $f_\alpha$ é dada pela integral 
$$
I_\alpha=\int_{-\alpha}^\alpha f_\alpha(x)\,dx=\frac{e^{-\alpha}}{\alpha^2}
\int_{-\alpha}^\alpha(\alpha^2-x^2)\,dx=(\cdots)=\tfrac43 \alpha
e^{-\alpha}\,.$$
Um simples estudo de $\alpha\mapsto I_\alpha$ mostra que o seu máximo é
atingido em $\alpha=1$.
\end{sol}
\end{exo}

\begin{exo}
Se $a>0$, calcule $I_n=\int_0^ax^{1/n}dx$. Calcule $\lim_{n\to \infty}I_n$, e dê
a interpretação geométrica da solução. (Dica: lembre dos esboços das funções
$x\mapsto x^{1/p}$, no Capítulo \ref{CAP:Funcoes}.)
\begin{sol}
Como $I_n=\frac{n}{n+1}a^{\frac{n+1}{n}}$, temos $\lim_{n\to \infty}I_n=a$.
Quando $n\to \infty$, o gráfico de $x\mapsto x^{1/n}$ em $\bR_+$ tende
ao gráfico da função constante $f(x)\equiv 1$. Ora, $\int_0^a f(x)\,dx=a$!
\end{sol}
\end{exo}

\section{Primitivas}
\index{primitiva}
O Teorema Fundamental mostra a importância de saber calcular
primitivas. Por isso, será útil desenvolver \emph{técnicas de
integração}.
Mas antes de apresentarmos essas técnicas, 
faremos alguns comentários sobre as notações usadas para denotar
primitivas.\\

Para uma dada função $f$, queremos \emph{achar uma primitiva} $F$, isto é 
uma função cuja derivada $F'$ é igual a $f$. Essa operação, \emph{inversa da
derivada}~\footnote{Às vezes, essa operação é naturalmente chamada de
\emph{antiderivada}.}, será chamada de \grasA{integrar $f$}.
Por isso, é útil introduzir uma notação que mostra que
$F$ é o resultado de uma transformação aplicada a $f$:
$$F(x)=\int f(x)dx+C\,,$$
em que $C$ é uma constante arbitrária.
Ao invés da integral definida $\int_a^bf(x)\,dx$, a integral \emph{indefinida}
$\int f(x)\,dx$ \emph{é uma função de $x$}, que por definição satisfaz 
$$\Bigl(\int f(x)\,dx\Bigr)'=f(x)\,.$$
Como a operação ``integrar com respeito a $x$'' é a operação
inversa da derivada, temos 
\eq{\label{eq:integrderivegalf}\int f'(x)\,dx=f(x)+C\,.}
Além disso, as seguintes propriedades são satisfeitas ($\lambda\in \bR$ é uma
constante):
$$\int\lambda f(x)\, dx=\lambda\int f(x)\,dx\,,\quad
\int(f(x)+g(x))dx=\int f(x)\,dx+\int g(x)\,dx\,.$$

As seguintes primitivas fundamentais foram calculadas no Exercício
\ref{Exo:PrimitivasBasicas}:
\begin{multicols}{2}
\begin{enumerate}
\item $\int k\,dx=kx+C$
\item $\int x\,dx=\frac{x^2}{2}+C$
\item\label{itPrimitFund3} $\int x^p\,dx=\frac{x^{p+1}}{p+1}+C$ ($p\neq -1$)
\item $\int \cos x\,dx=\sen x+C$
\item $\int \sen x\,dx=-\cos x+C$
\item $\int e^x\,dx=e^x+C$
\item $\int \frac{dx}{1+x^2}=\arctan x+C$
\item $\int \frac{dx}{\sqrt{1-x^2}}=\arcsen x+C$
\end{enumerate}
\end{multicols}
\vspace{0.01cm}

O caso $p=-1$ em \eqref{itPrimitFund3} corresponde a $\int \frac1x\,dx$, que
obviamente é definida somente para $x\neq 0$. Ora, se $x>0$, temos $(\ln
(x))'=\tfrac1x$, e se $x<0$, temos $(\ln (-x))'=\tfrac{-1}{-x}=\tfrac1x$. Logo,
%\framebox[1.1\width]
$$\boxed{\int \frac1x \,dx=\ln|x|+C\,\quad (x\neq 0).}$$

\begin{exo}
Calcule as primitivas das seguintes funções.
\begin{multicols}{3}
\begin{enumerate}
\item\label{itprimitsubst1} $(1-x)(1+x)^2$
\item\label{itprimitsubst2} $\frac{1}{x^3}-\cos(2x)$
\item\label{itprimitsubst3} $\frac{x+5x^7}{x^9}$
%\item\label{itprimitsubst4} $x\sen(x^2)$
%\item\label{itprimitsubst5} $\cos^2(t)$
\item\label{itprimitsubst6} $2+2\tan^2(x)$
%\item\label{itprimitsubst7} $\frac{x}{1+x^2}$
%\item\label{itprimitsubst8} $\tan x$
\end{enumerate}
\end{multicols}
\vspace{0.01cm}
\begin{sol}
\eqref{itprimitsubst1}
$-\frac{x^4}{4}-\frac{x^3}{3}+\frac{x^2}{2}+x+C$, 
\eqref{itprimitsubst2} $\frac{-1}{2x^2}-\frac{\sen (2x)}{2}+C$,
\eqref{itprimitsubst3} $-\frac{1}{7x^7}-\frac{5}{x}+C$, 
%\eqref{itprimitsubst4} $-\frac{1}{2}\cos(x^2)+C$,
%\eqref{itprimitsubst5} $\frac{x}{2}+\frac{\sen x \cos x}{2}+C$,
\eqref{itprimitsubst6} $2\tan x+C$.
%\eqref{itprimitsubst7} $\tfrac12\ln(1+x^2)+C$
%\eqref{itprimitsubst8} $-\ln(\cos x)+C$
\end{sol}
\end{exo}

Vamos agora apresentar os métodos clássicos usados para 
calcular primitivas.
O leitor interessado em \emph{usar} a integral de Riemann para
resolver problemas concretos pode pular para o
Capítulo~\ref{CAP:Applicacoes}, e voltar depois para as Seçoes 
\ref{Sec:MetodoSubstit} até \ref{Sec:MetodoSubstitTrig} abaixo para
exercitar a sua habilidade a calcular primitivas.

\subsection{Integração por Substituição}\label{Sec:MetodoSubstit}
\index{integração!por substituição}
\begin{ex}
Suponha que se queira calcular $$\int x\cos (x^2)\,dx\,.$$
Apesar da função $x\cos (x^2)$ não ser a derivada de uma função elementar,
ela possui uma estrutura particular: o ``$x$'' que multiplica o cosseno é um
polinômio cujo grau é um a menos do que o polinômio ``$x^2$'' contido dentro do
cosseno. Ora, sabemos que a derivada diminui o grau de um polinômio. No nosso
caso: $(x^2)'=2x$. 
Logo, ao multiplicar e dividir a primitiva por $2$, podemos escrever
$$\int x\cos (x^2)\,dx=\tfrac12\int (2x)\cos (x^2)\,dx
=\tfrac12\int (x^2)'\cos (x^2)\,dx\,.$$
Agora, reconhecemos em $(x^2)'\cos (x^2)$ uma derivada. De fato, pela regra da
cadeia, $(\sen (x^2))'=\cos(x^2)\cdot (x^2)'$. Logo, usando
\eqref{eq:integrderivegalf},
$$
\int (x^2)'\cos (x^2)\,dx=\int (\sen (x^2))'\,dx=
\sen(x^2)+C\,.
$$
Portanto,
$$\int x\cos (x^2)\,dx=\tfrac12\sen(x^2)+C\,.$$
Do mesmo jeito, 
$$\int x^2 \cos (x^3)\,dx=\tfrac13\int 3x^2 \cos (x^3)\,dx=
\tfrac13\int (x^3)' \cos (x^3)\,dx= \tfrac13\sen (x^3)+C\,.$$
\end{ex}

A ideia apresentada nesse último exemplo consiste em conseguir 
escrever a função integrada na forma da derivada 
de uma função composta; é a base do método de integração chamado
\emph{integração por substituição}.
Lembremos a regra da cadeia:
\index{regra!da cadeia}
$$\bigl(f(g(x))\bigr)'=f'(g(x))g'(x)\,.$$
Integrando ambos lados dessa identidade com respeito a $x$ e usando de
novo \eqref{eq:integrderivegalf} obtemos
$f(g(x))=\int f'(g(x))g'(x)\,dx+\text{constante}$, que é equivalente à
\grasA{fórmula de integração por substituição}:
\eq{\boxed{\int f'(g(x))g'(x)\,dx=f(g(x))+C\,.}}
Existem vários jeitos de escrever a mesma fórmula. Por exemplo, se $H$ é
primitiva de $h$,
\eq{\int h(g(x))g'(x)\,dx=H(g(x))+C\,.}
Senão, a função $g(x)$ pode ser considerada como
\emph{uma nova váriavel}: $u\pardef g(x)$. 
Derivando com respeito a $x$,
$\frac{du}{dx}=g'(x)$, que pode ser simbolicamente escrita como $du=g'(x)dx$.
Assim, a primitiva inicial pode ser escrita
somente em termos da variável $u$, \emph{substituindo $g(x)$ por $u$}:
\eq{\label{eq:substitporu}\int h(g(x))g'(x)\,dx=\int h(u)\,du\,.}
Em seguida, se trata de calcular uma primitiva de $h$, e no final voltar para a
variável $x$. O objetivo é sempre tornar $\int h(u)\,du$ o mais próximo possível
de uma primitiva elementar como as descritas no início da seção.

\begin{ex}
Considere $\int \frac{\cos x}{\sen^2x}\,dx$. Aqui queremos usar o fato do
$\cos x$ ser a derivada da função $\sen x$. Façamos então a {substituição}
$u=\sen x$, que implica $du=(\sen x)'dx=\cos x\,dx$, o que implica
$$\int \frac{\cos x}{\sen^2x}\,dx=\int \frac{1}{u^2}\,du\equiv \int
h(u)\,du\,.$$ 
Mas $h(u)=\tfrac{1}{u^2}$, é a derivada (com respeito a $u$!) de 
$H(u)=-\frac{1}{u}$. Logo, 
$$\int \frac{\cos x}{\sen^2x}\,dx=\int h(u)\,du=
H(u)+C=-\frac{1}{\sen x}+C\,.$$
\end{ex}

\begin{ex}
Para calcular $\int \frac{x}{1+x}\,dx$, definemos $u\pardef 1+x$. Logo, $du=dx$
e $x=u-1$. Assim,
\begin{align*}
\int\frac{x}{1+x}\,dx=\frac{u-1}{u}\,du=\int\bigl\{1-\tfrac{1}{u}\bigr\}\,du
&=\int du-\int\tfrac{1}{u}\,du\\
&=u-\ln u+C=1+x-\ln(1+x)+C\,.
\end{align*}
\end{ex}

\begin{ex}
Calculemos agora $\int\frac{x+1}{\sqrt{1-x^2}}dx$.
Para começar, separemos a primitiva em dois termos:
$$
\int\frac{x+1}{\sqrt{1-x^2}}\,dx=\int\frac{x}{\sqrt{1-x^2}}\,dx+\int\frac{1}{
\sqrt {1-x^2}}\,dx\,.
$$
Para o primeiro termo, vemos que com $u=g(x)\pardef 1-x^2$, cuja derivada é
$g'(x)=-2x$, temos $du=-2x\,dx$, e
$$
\int\frac{x}{\sqrt{1-x^2}}\,dx=
%-\int\frac{-2x}{2\sqrt{1-x^2}}\,dx=
-\int \frac{1}{2\sqrt{u}}du=-\sqrt{u}+C=-\sqrt{1-x^2}+C\,.$$
No segundo termo reconhecemos a derivada da função arcseno. Logo, somando,
\eq{\label{eq:rienderien}\int\frac{x+1}{\sqrt{1-x^2}}dx
=-\sqrt{1-x^2}+{\mathrm{arcsen}}\,x+C\,.}
\end{ex}

\begin{obs}
Lembra que um cálculo de primitiva pode sempre
ser \emph{verificado}, derivando o resultado obtido! Por exemplo, não
perca a oportunidade de verificar que derivando o lado direito de
\eqref{eq:rienderien}, obtém-se $\frac{x+1}{\sqrt{1-x^2}}$!
\end{obs}

Às vezes, é preciso transformar a função integrada antes de fazer
uma substituição útil, como visto nos três próximos exemplos.

\begin{ex}
Para calcular $\int \frac{1}{9+x^2}\,dx$
podemos colocar $9$ em evidência no denominador, e em seguida fazer a
substituição $u=\tfrac{x}{3}$:
\begin{align*}\int \frac{1}{9+x^2}\,dx=
\tfrac19\int \frac{1}{1+(\tfrac{x}{3})^2}\,dx&=\tfrac19
\int\frac{3}{1+u^2}\,dx\\
&=\tfrac13\int \frac{1}{1+u^2}\,du =\tfrac13\arctan u+C
=\tfrac13\arctan(\tfrac{x}{3})+C\,.
\end{align*}
\end{ex}

\begin{ex} Para calcular $\int\frac{1}{x^2+2x+2}\,dx$ comecemos 
completando o quadrado: $x^2+2x+2=\{(x+1)^2-1\}+2=1+(x+1)^2$. Logo,
usando $u\pardef x+1$,
\begin{align*}
\int\frac{1}{x^2+2x+2}\,dx&=\int\frac{1}{1+(x+1)^2}\,dx\\
&=\int\frac{1}{1+u^2}\,du=\arctan u+C=
\arctan(x+1)+C\,.
\end{align*}
\end{ex}

\begin{ex}\label{Ex:sencarre}
Considere $\int \sen^2x\,dx$. Lembrando a identidade trigonométrica
$\sen^2x=\frac{1-\cos(2x)}{2}$, 
$$
\int\sen^2x\,dx=\tfrac12 \int \,dx-\tfrac12 \int \cos(2x)\,dx=
\tfrac{x}{2}-\tfrac12 \int \cos(2x)\,dx\,.
$$ 
Agora com $u=2x$ obtemos $\int \cos(2x)\,dx=\tfrac{1}{2}\int
\cos(u)\,du=\tfrac12 \sen u+\text{constante}$. Logo,
$$
\int \sen^2x\,dx=\tfrac{x}{2}-\tfrac14\sen(2x)+C\,.
$$
\end{ex}




\begin{exo}
Calcule as primitivas das seguintes funções.
\begin{multicols}{3}
\begin{enumerate}
\item\label{itprimitsubst40} $(x+1)^7$
\item\label{itprimitsubst400} $\frac{1}{(2x+1)^2}$
\item\label{itprimitsubst401} $\frac{1}{(1-4x)^3}$
\item\label{itprimitsubst4} $x\sen(x^2)$
\item \label{itprimitsubst4000} $\sen x \cos x$
\item\label{itprimitsubst45} $\tfrac{1}{\sqrt{x}}\cos (\sqrt{x})$
\item\label{itprimitsubst5} $\cos^2(t)$
\item\label{itprimitsubst7} $\frac{x}{1+x^2}$
\item\label{itprimitsubst71} $\cos x\sqrt{1+\sen x}$
\item\label{itprimitsubst8} $\tan x$
\item\label{itprimitsubst9} $\frac{3x+5}{1+x^2}$
\item\label{itprimitsubst10} $\frac{1}{x^2+2x+3}$
\item\label{itprimitsubst12} $e^x\tan(e^x)$
%\item\label{itprimitsubst121} $\frac{x}{1+x}$
\item\label{itprimitsubst13} $\frac{y}{(1+y)^3}$
\item\label{itprimitsubst14} $x\sqrt{1+x^2}$
\item\label{itprimitsubst15} $\frac{x}{(1+x^2)^2}$
\item\label{itprimitsubst11} $\frac{\cos^3t}{\sen^4t}$
\item\label{itprimitsubst16} $\sen^3x\cos^3x$
\end{enumerate}
\end{multicols}
\vspace{0.01cm}
\begin{sol}
\eqref{itprimitsubst40} $\frac{1}{8}(x+1)^8+C$ (Obs: aqui, basta fazer a
substituição $u=x+1$. Pode também fazer sem, mas implica desenvolver um
polinômio de grau $7$!)
\eqref{itprimitsubst400} $\frac{-1}{2(2x+1)}+C$
\eqref{itprimitsubst401} $\frac{1}{8(1-4x)^2}+C$
\eqref{itprimitsubst4} $-\frac{1}{2}\cos(x^2)+C$,
\eqref{itprimitsubst4000} $\frac{1}{2}\sen^2(x)+C$, ou $-\frac{1}{2}\cos^2(x)+C$
\eqref{itprimitsubst45} $2\sen(\sqrt{x})+C$,
\eqref{itprimitsubst5} $\frac{x}{2}+\tfrac14\sen (2x)+C$,
\eqref{itprimitsubst7} $\tfrac12\ln(1+x^2)+C$,
\eqref{itprimitsubst71} $\frac{2}{3}(1+\sen x)^{\frac{3}{2}}+C$
\eqref{itprimitsubst8} $\int \tan x\,dx=\int\frac{\sen x}{\cos
x}\,dx=-\int\frac{(\cos x)'}{\cos
x}\,dx -\ln|\cos x|+C$.
\eqref{itprimitsubst9} $\tfrac32 \ln(1+x^2)+5\arctan x+C$
\eqref{itprimitsubst10} $\frac{1}{\sqrt{2}}\arctan(\frac{x+1}{\sqrt{2}})+C$
\eqref{itprimitsubst12} 
Com a substituição $u:=e^x$, $du=e^x dx$, 
$\int e^x\tan(e^x)dx=\int \tan u du=-\ln|\cos u|+C=-\ln|\cos(e^x)|+C$.
\eqref{itprimitsubst13} $\frac{1}{2(1+y)^2}-\frac{1}{1+y}+C$
\eqref{itprimitsubst14} $\frac{1}{3}(1+x^2)^{\frac{3}{2}}+C$
\eqref{itprimitsubst15} $\frac{-1}{2(1+x^2)}+C$
\eqref{itprimitsubst11} $-\frac{1}{3\sen^3t}+\frac{1}{\sen t}+C$ (a ideia aqui
é escrever $\frac{\cos^3t}{\sen^4t}=\frac{\cos^2t}{\sen^4t}\cos
t=\frac{1-\sen^2t}{\sen^4t}\cos t$)
\eqref{itprimitsubst16} $\frac{(\sen x)^4}{4}-\frac{(\sen x)^6}{6}$
\end{sol}
\end{exo}

A fórmula \eqref{eq:substitporu} mostra que a primitiva (ou integral 
indefinida) de uma função 
da forma $h(g(x))g'(x)$ se reduz a achar uma primitiva de $h$.
Aquela fórmula pode também ser usada para integrais definidas: se
$h(g(x))g'(x)$ é integrada com $x$ percorrendo o intervalo $[a,b]$, então
$u=g(x)$ percorre o intervalo $[g(a),g(b)]$, logo
\eq{\label{eq:substitporudefin}
\int_a^bh(g(x))g'(x)\,dx=\int_{g(a)}^{g(b)}h(u)\,du\,.}


\begin{exo}
%(Gilcione, 13/11/2009) 
Calcule as primitivas
\begin{multicols}{3}
\begin{enumerate}
\item\label{itttit1} $\int \frac{2x^3dx}{\sqrt{1-x^2}}\,dx$
\item\label{itttit2} $\int \frac{dx}{\sqrt{x-x^2}}$
\item\label{itttit3} $\int \frac{\ln x}{x}\,dx$
\item\label{itttit4} $\int e^{e^x}e^x\,dx$
\item\label{itttit5} $\int \frac{\sqrt{x}}{1+\sqrt{x}}\,dx$
\item\label{itttit6} $\int \tan^2x\,dx$
\end{enumerate}
\end{multicols}
\vspace{0.01cm}

\begin{sol}
\eqref{itttit1}
Com $u=1-x^2$, $du=-2x\,dx$, temos
\begin{align*}
\int \frac{2x^3dx}{\sqrt{1-x^2}}\,dx=-\int
\frac{x^2}{\sqrt{1-x^2}}(-2x)\,dx
&=-\int \frac{1-u}{\sqrt{u}}\,du\\
&=-2\sqrt{u}+\tfrac23 u^{3/2}+C\\
&=-2\sqrt{1-x^2}+\tfrac23 (1-x^2)^{3/2}+C\,.
\end{align*}
\eqref{itttit2}
Completando o quadrado, e fazendo a substituição $u=2x-1$,
\begin{align*}
\int \frac{dx}{\sqrt{x-x^2}}=\int 
\frac{dx}{\sqrt{\tfrac14-(x-\tfrac12)^2}}&=
\int \frac{2 dx}{\sqrt{1-(2x-1)^2}}\\
&=\int \frac{du}{\sqrt{1-u^2}}=\arcsen u+C=\arcsen (2x-1)+C\,.
\end{align*}
\eqref{itttit3} Com $u=\ln t$, $\int \frac{\ln x}{x}\,dx=\int
u\,du=\tfrac{u^2}{2}+C=\tfrac12(\ln x)^2+C$
\eqref{itttit4} Com $u=e^x$, $\int e^{e^x}e^x\,dx=\int e^u\,du=e^u+C=e^{e^x}+C$.
\eqref{itttit5} $\int \frac{\sqrt{x}}{1+\sqrt{x}}\,dx=x-2\sqrt{x}+2\ln
(1+\sqrt{x})+C$.
\eqref{itttit6} $\int \tan^2x\,dx=\int(1+\tan^2x-1)\,dx=\tan x-x+C$.
\end{sol}
\end{exo}

\subsection{Integração por Partes}\label{Sec:MetodoPartes}
\index{integração!por partes}
Vimos que o método de integração por substituição decorreu da regra da cadeia.
Vejamos agora qual método pode ser obtido a partir da regra de derivação de um
produto.

\begin{ex}
Suponha que se queira calcular a primitiva
$$\int x\cos x\,dx\,.$$
Aqui não vemos (e na verdade: não há) uma substituição que seja útil para
transformar essa primitiva.
O que pode ser útil é escrever $x\cos x=x(\sen x)'$, e de interpretar
$x(\sen x)'$ como o segundo termo da derivada
$$(x\sen x)'=(x)'\sen x+x(\sen x)'=\sen x+x(\sen x)'\,.$$
Assim,
\begin{align*}
\int x\cos x\,dx=\int\bigl\{(x\sen x)'-\sen x\bigr\}\,dx&=
x\sen x-\int\sen x\,dx\\
&=x\sen x+\cos x+C
\end{align*}
\end{ex}
A ideia usada no último exemplo pode ser generalizada da seguinte maneira. Pela
regra de Leibniz,
$$(f(x)g(x))'=f'(x)g(x)+f(x)g'(x)\,.$$
Integrando com respeito a $x$ em ambos lados,
$$
f(x)g(x)=\int f'(x)g(x)\,dx+\int f(x)g'(x)\,dx\,.
$$
Essa última expressão pode ser reescrita como
\eq{\label{eq:integrporpartes}
\boxed{\int f'(x)g(x)\,dx=f(x)g(x)-\int f(x)g'(x)\,dx\,,}
}
(ou a mesma trocando os papéis de $f$ e $g$)
% correcao de "papeis":  Hugo Reis <hugofreitasreis@gmail.com>
chamada \grasA{fórmula de integração por partes}. Ela possui uma forma definida
também:
\eq{\label{eq:integrporpartesdefin}
\boxed{\int_a^b f'(x)g(x)\,dx=f(x)g(x)\big|_a^b-\int_a^b
f(x)g'(x)\,dx\,.}}

A fórmula \eqref{eq:integrporpartes} acima será usada com o intuito de
transformar a integral $\int f'(x)g(x)\,dx$ numa integral (mais simples,
espera-se) $\int f(x)g'(x)\,dx$.

\begin{ex}
Considere $\int x\ln x\,dx$. Aqui definamos $f$ e $g$ da seguinte maneira:
$f'(x)=x$, $g(x)=\ln x$. Assim, $f(x)=\tfrac{x^2}{2}$, $g'(x)=(\ln
x)'=\tfrac{1}{x}$. Usando \eqref{eq:integrporpartes},
\begin{align*}
\int x\ln x\,dx&\equiv \int f'(x)g(x)\,dx\\
&=f(x)g(x)-\int f(x)g'(x)\,dx\\
&\equiv (\tfrac{x^2}{2})(\ln x)-\int (\tfrac{x^2}{2})(\tfrac1x)\,dx=
\tfrac{x^2}{2}\ln x-\tfrac12\int x\,dx=\tfrac{x^2}{2}\ln x-\tfrac{x^2}{4}+C\,
\end{align*}
\end{ex}



\begin{exo} Calcule as primitivas das funções abaixo. 
(Obs: às vezes, pode precisar integrar por partes duas vezes.)
\begin{multicols}{3}
\begin{enumerate}
\item\label{itintpartes1} $x\sen x$
\item\label{itintpartes2} $x\cos(5x)$
\item\label{itintpartes3} $x^2\cos x$
\item\label{itintpartes4} $xe^x$
\item\label{itintpartes5} $x^2e^{-3x}$
\item\label{itintpartes6} $x^3\cos (x^2)$
%\item\label{itintpartes7} $x\ln x$
\end{enumerate}
\end{multicols}
\vspace{0.01cm}
\begin{sol}
\eqref{itintpartes1} $\sen x-x\cos x+C$,
\eqref{itintpartes2} $\frac{1}{5}x\sen(5x)+\frac{1}{25}\cos(5x)+C$
\eqref{itintpartes3} Integrando duas vezes por partes:
$$
\int x^2\cos x\,dx=x^2\sen x-\int (2x)\sen x\,dx=x^2\sen x-2\Bigl\{
x(-\cos x)-\int (-\cos x)\,dx\,.
\Bigr\}$$
Portanto $\int x^2\cos x\,dx=x^2\sen x-2(\sen x-x\cos x)+C$.
\eqref{itintpartes4} $(x-1)e^x+C$
\eqref{itintpartes5} $-\tfrac13 e^{-3x}(x^2-\tfrac23 x-\tfrac29)+C$
\eqref{itintpartes6} 
\begin{align*}
\int x^3\cos (x^2)\,dx=\int x^2 (x\cos(x^2))\,dx&=x^2(\tfrac12
\sen(x^2))-\int(2x)(\tfrac12 \sen (x^2))\,dx\,\\
&=\tfrac12 x^2 \sen(x^2)+\tfrac12 \cos (x^2)+C\,.
\end{align*}
%\eqref{itintpartes7} $x^2(\ln x-\tfrac12)+C$ 
\end{sol}
\end{exo}

Às vezes, escrevendo ``$1$'' como $1=(x)'$, integração por partes pode ser usada mesmo quando não tem duas partes:
\begin{ex}
Considere $\int \ln x\, dx$. Escrevendo $\ln x=1\cdot \ln x=(x)'\ln x$, 
$$
\int \ln x\,dx=\int (x)'\ln x\,dx=x\ln x-\int x(\ln x)'\,dx=x\ln x-\int x\cdot
\tfrac1x\,dx=x\ln x-x+C\,.
$$
\end{ex}

\begin{exo} Calcule 
\begin{multicols}{2}
\begin{enumerate}
\item\label{ititnpartmntriv1} $\int \arctan x \,dx$
\item\label{ititnpartmntriv2} $\int (\ln x)^2\,dx$
\item\label{ititnpartmntriv3} $\int \arcsen x\,dx$
\item\label{ititnpartmntriv4} $\int x\arctan x\,dx$
\end{enumerate}
\end{multicols}
\vspace{0.01cm}
\begin{sol} \eqref{ititnpartmntriv1}
$\int \arctan x dx=x\arctan
x-\int\frac{x}{1+x^2}\,dx=x\arctan x-\tfrac12 \ln (1+x^2)+C$.
\eqref{ititnpartmntriv2} $x(\ln x)^2-2x(\ln x-1)+C$
\eqref{ititnpartmntriv3} $x\arcsen x+\sqrt{1-x^2}+C$
\eqref{ititnpartmntriv4} $\int x\arctan x\,dx=\frac12(x^2\arctan x-x+\arctan x)+C$
\end{sol}
\end{exo}



Consideremos agora um mecanismo particular que pode aparecer quando se aplica
integração por partes:
\begin{ex}
Considere $\int\sen(x)\cos (3x)\,dx$. Integrando duas vezes por partes:
\begin{align*}
\int \sen(x)\cos (3x)dx&=(-\cos x) \cos 3x-\int (-\cos x)(-3\sen 3x)dx\\
&=-\cos x\cos 3x-3\int \cos x\sen 3x\,dx\\
&=-\cos x\cos 3x-3\Big\{
\sen x\sen 3x-\int \sen x(3\cos 3x)\,dx
\Big\}\\
&=-\cos x\cos 3x-3
\sen x\sen 3x+9\int \sen x\cos 3x\,dx\,.
\end{align*}
Assim, a primitiva procurada $I(x)=\int\sen(x)\cos (3x)\,dx$
é solução da equação
$$
I(x)=-\cos x\cos 3x-3\sen x\sen 3x+9 I(x)\,.
$$
Isolando $I(x)$ obtemos
$I(x)=\frac{1}{8}\big\{\cos x\cos 3x+3\sen x\sen 3x \big\}$. Isto é,
$$\int \sen(x)\cos (3x)\,dx=\frac{1}{8}\big\{
\cos x\cos 3x+3\sen x\sen 3x \big\}+C\,.$$
\end{ex}

\begin{exo}\label{Exo:porpartesmach}
Calcule 
\begin{multicols}{3}
\begin{enumerate}
\item\label{itintpartestruc1} $\int e^{-x}\sen x\,dx$
\item\label{itintpartestruc2} $\int e^{-st}\cos t\,dt$
\item\label{itintpartestruc3} $\int \sen (\ln x)\,dx$
\end{enumerate}
\end{multicols}
\vspace{0.01cm}
\begin{sol}
\eqref{itintpartestruc1} $-\frac{e^{-x}}{2}(\sen x+\cos x)+C$
\eqref{itintpartestruc2} $\frac{e^{-st}}{1+s^2}(\sen t- s\cos t)+C$
\eqref{itintpartestruc3} $\frac{x}{2}(\sen (\ln x)-\cos (ln x))+C$
\end{sol}
\end{exo}

Integração por partes pode ser combinada com substituição:

\begin{ex}
Considere $\int x\ln(1+x)\,dx$. Integrando primeiro por partes,
$$
\int x\ln(1+x)\,dx=\tfrac{x^2}{2}\ln(1+x)-\tfrac{1}{2}\int \frac{x^2}{1+x}\,dx\,.
$$
Essa segunda pode ser calculada substituindo $1+x$ por $u$:
\begin{align*}
\int\frac{x^2}{1+x}\,dx=\int\frac{(u-1)^2}{u}\,du&=
\int\{u-2+\tfrac1u\}\,du\\
&=\tfrac{u^2}{2}-2u+\ln|u|+C\\
&=\tfrac12(1+x)^2-2x+\ln|1+x|+C'\,.
\end{align*}
Logo,
$$
\int x\ln (1+x)\,dx=\tfrac{x^2}{2}\ln(1+x)-\tfrac14(1+x)^2
+x-\tfrac12\ln |x|+C'\,.
$$
\end{ex}

\begin{exo}\label{Exo:Bonnardpartessubsit}
Calcule $\int_0^3e^{\sqrt{x+1}}\,dx$, $\int x(\ln x)^2\,dx$.
\begin{sol}
Chamando $u=\sqrt{x+1}$, temos 
$$
\int_0^3e^{\sqrt{x+1}}\,dx=\int_1^2
2ue^u\,du=2\bigl\{ue^u-e^u\bigr\}\big|_1^2=2e^2\,.
$$
Chamando $u=\ln x$, temos $e^u\,du=dx$, e
$$
\int x(\ln x)^2\,dx=\int
u^2e^{2u}\,du=\tfrac{u^2}{2}e^{2u}-\tfrac{u}{2}e^{2u}+\tfrac14 e^{2u}+C\,.
$$
Logo, $\int x(\ln x)^2\,dx=\tfrac12 x^2(\ln x)^2-\tfrac12 x^2\ln
x+\tfrac14x^2+C$. 
\end{sol}
\end{exo}



\subsection{Integração de funções racionais}\label{Sec:FracParciais}
\index{integração!de funções racionais}
Nesta seção estudaremos métodos para calcular primitivas da forma 
$$
\int \frac{dx}{1-x^2}\,,\quad \int \frac{dx}{(1-x)(x+1)^2}\,,
\quad \int\frac{x^2}{x^2+1}\,dx\,,\quad
\int\frac{x^4}{x^3+1}\,dx\,.
$$
Essas primitivas são todas da forma
\eq{\label{eq:primitPQ}
\int\frac{ P(x)}{Q(x)}\,dx\,,}
em que $P(x)$ e $Q(x)$ são polinômios em $x$. Lembramos que um \emph{polinômio em $x$} 
é uma soma finita de potências inteiras e não negativas de $x$: $a_0+a_1x+a_2x^2+\dots+ a_nx^n$, em que
os $a_i$ são constantes. Por exemplo, $x^3-x+1$ é um polinômio, mas
$x^{2/3}+\sqrt{x}$ não é. Lembramos que o \emph{grau} de um polinômio $a_0+a_1x+a_2x^2+\dots+ a_nx^n$ é
o maior índice $i$ tal que $a_i\neq 0$.\\

Existe uma teoria geral que descreve os
métodos que permitem calcular primitivas da forma \eqref{eq:primitPQ}. 
Aqui ilustraremos somente as
ideias principais em casos simples.\\

A primeira etapa tem como objetivo simplificar a expressão para ser
integrada:
\begin{itemize}
 \item \emph{Se o grau de $P$ for maior ou igual ao grau de $Q$, divide
$P$ por $Q$.}
\end{itemize}


\begin{ex}
Considere $\int \frac{x^2}{x^2+1}\,dx$. Aqui, $P(x)=x^2$ é de grau
$2$, que é igual ao grau de $Q(x)=x^2+1$. Logo, como a divisão de $P(x)$ por
$Q(x)$ dá $1$ com um resto de $-1$, temos
$\frac{x^2}{x^2+1}=1-\frac{1}{x^2+1}$. Logo,
$$
\int \frac{x^2}{x^2+1}\,dx=\int\Big\{
1-\frac{1}{x^2+1}\Big\}\,dx=
x-\arctan x+C\,.
$$
(Observe que em vez de fazer uma divisão, podia ter observado que 
$\frac{x^2}{x^2+1}=\frac{x^2+1-1}{x^2+1}=\frac{x^2+1}{x^2+1}-\frac{1}{x^2+1}
=1-\frac{1}{x^2+1}$.)
\end{ex}

\begin{ex}
Considere $\int \frac{x^3}{x^2+1}\,dx$. Aqui, $P(x)=x^3$ é de grau
$3$, que é maior do que o grau de $Q(x)=x^2+1$. Logo, como a divisão de $P(x)$
por
$Q(x)$ dá $x$ com um resto de $-x$, temos
$\frac{x^3}{x^2+1}=x-\frac{x}{x^2+1}$. Logo,
\begin{align*}
\int \frac{x^3}{x^2+1}\,dx=\int\Big\{
x-\frac{x}{x^2+1}\Big\}\,dx&=
\tfrac{x^2}{2}-\tfrac12\int\frac{2x}{x^2+1}\,dx\\
&=\tfrac{x^2}{2}- \tfrac{1}{2}\ln (x^2+1)+C\,.
\end{align*}
\end{ex}

Em geral, quando $\mathrm{grau}(P)\geq \mathrm{grau}(Q)$, 
a divisão de $P$ por $Q$ dá
$$\frac{P(x)}{Q(x)}=\text{polinômio em $x$ }+\frac{\widetilde{P}(x)}{Q(x)}\,,
$$
em 
que $\mathrm{grau}(\widetilde{P})<\mathrm{grau}(Q)$. A primitiva do primeiro 
polimômio é imediata, e o próximo passo é de estudar a primitiva da razão
$\frac{\widetilde{P}(x)}{Q(x)}$.\\

Portanto, é preciso agora desenvolver técnicas para calcular primitivas de
frações de polinômios, em que o grau do numerador é \emph{estritamente menor}
que o grau do denominador. Já sabemos tratar casos do tipo:
$$\int\frac{dx}{x^3}=-\frac{1}{2x^2}+C\,,\quad \int\frac{dx}{x^2+1}=\arctan
x+C\,,\quad \int\frac{x}{x^2+1}dx=\tfrac12 \ln (x^2+1)+C\,.$$

O objetivo será de sempre \emph{decompor} a fração
$\frac{\widetilde{P}(x)}{Q(x)}$ numa soma de frações elementares desse tipo.
O método geral, descrito abaixo em exemplos simples, pode
ser resumido da seguinte maneira:
\begin{itemize}
\item \emph{Fatore completamente o polinômio $Q$, o escrevendo como um produto
de fatores de grau $2$, possivelmente repetidos.} Em seguida,
\item \emph{Procure uma decomposição de $\frac{\widetilde{P}(x)}{Q(x)}$ em
frações parciais.}
\end{itemize}

\begin{ex}\label{Ex:unsurxdeuxmun}
Considere $\int \frac{dx}{x^2-1}$. Aqui, $x^2-1$ tem discriminante $\Delta>0$,
logo ele pode ser \emph{fatorado}: $x^2-1=(x-1)(x+1)$. Procuremos agora um
jeito de escrever a função integrada na forma de uma
soma de frações elementares:
\eq{\label{eq:decompelem}\frac{1}{x^2-1}=\frac{1}{(x-1)(x+1)}=\frac{A}{x-1}
+\frac{B}{x+1}\,.}
Observe que \emph{se} tiver um jeito de achar duas
constantes (isto é: \emph{números que não dependem de $x$}) $A$ e $B$ tais
que a expressão acima seja verificada para todo $x$, então a primitiva será
fácil de se calcular:
$$
\int\frac{dx}{x^2-1}=A\int\frac{dx}{x-1}+B\int\frac{dx}{x+1}=A\ln
|x-1|+B\ln|x+1|+C\,.
$$
Verifiquemos então que as constantes $A$ e $B$ existem. Colocando no mesmo
denominador no lado direito de \eqref{eq:decompelem} e igualando os
numeradores, vemos que $A$ e $B$ devem ser escolhidos tais que
\eq{\label{eq:machiiii}1=A(x+1)+B(x-1)\,.} 
Rearranjando os coeficientes, 
\eq{(A+B)x+A-B-1=0\,.}
Para essa expressão valer para todo $x$, é necessário ter 
$$
A+B=0\,,\quad A-B-1=0\,.
$$
Essas expressões representam um \emph{sistema} de duas equações nas
incógnitas $A$ e $B$, cuja solução pode ser calculada facilmente:
$A=\tfrac12$, $B=-\tfrac12$.
Verifiquemos que os valores calculados para $A$ e $B$ são
corretos:
$$
\frac{\tfrac12}{x-1}
+\frac{-\tfrac12}{x+1}=\frac{\tfrac12 (x+1)-\tfrac12(x-1)}{(x-1)(x+1)}
\equiv \frac{1}{(x-1)(x+1)}\,.
$$
Portanto, 
$$\int\frac{dx}{x^2-1}=\tfrac12\ln
|x-1|-\tfrac12\ln|x+1|+C=\tfrac12\ln \Bigl|\frac{x-1}{x+1}\Bigr|+C\,.$$
\end{ex}

\begin{obs}
Às vezes, os 
valores de $A$ e $B$ podem ser achados de um outro jeito. Por exemplo,
tomando o limite $x\to -1$ em \eqref{eq:machiiii} obtemos
$$1=-2B\,,$$
isto é $B=-\frac12$. Tomando agora $x\to +1$ em \eqref{eq:machiiii} obtemos 
$$1=2A\,,$$ isto é $A=\tfrac12$. 
\end{obs}

A decomposição \eqref{eq:decompelem} é chamada de \grasA{decomposição em
frações parciais}. \index{decomposição em frações parciais}
Esta decomposição pode ser feita a cada vez que o
denominador se encontra na forma de um produto de fatores irredutíveis de
grau $2$. A decomposição deve às vezes ser adaptada. 

\begin{ex}\label{Ex:decomppp}
Considere $\int\frac{dx}{x(x^2+1)}$. 
Vendo o que foi feito acima, uma 
decomposição natural seria de decompor a fração da seguinte maneira:
\eq{\label{eq:decomposp22}
\frac{1}{x(x^2+1)}=\frac{A}{x}+\frac{B}{x^2+1}\,.}
Infelizmente, pode ser verificado (veja o Exercício \ref{Exo:naopode} abaixo) que \emph{não existem} constantes $A$
e $B$ tais que a relação acima valha para todo $x$. O problema é que o denominador da fração original
contém $x^2+1$, que é \emph{irredutível} (isto é: possui um discriminante
negativo), de grau $2$. Assim, procuremos uma decomposição da forma
\eq{\label{eq:decomposp}
\frac{1}{x(x^2+1)}=\frac{A}{x}+\frac{Bx+C}{x^2+1}\,.}
Igualando os numeradores,
$1=A(x^2+1)+(Bx+C)x$, o que equivale a dizer que o polinômio $(A+B)x^2+Cx+A-1=0$
é nulo para todo $x$. Isto é: todos os seus coeficientes são nulos:
$$
A+B=0\,,\quad C=0 \,,\quad A-1=0\,.$$
Assim vemos que $A=1$, $B=-1$, $C=0$.
Verificando:
$$
\frac{1}{x}+\frac{-x}{x^2+1}=\frac{1(x^2+1)+(-x)x}{x(x^2+1)}
\equiv \frac{1}{x(x^2+1)}\,.
$$
Logo,
$$\int\frac{dx}{x(x^2+1)}=\int\frac{dx}{x}-\int \frac{x}{x^2+1}\,dx
=\ln |x|-\tfrac12\ln(x^2+1)+c\,.
$$
\end{ex}


\begin{exo}\label{Exo:naopode}
No Exemplo \ref{Ex:decomppp}, verifique que \emph{não tem}
decomposição da forma 
$\frac{1}{x(x^2+1)}=\frac{A}{x}+\frac{B}{x^2+1}$.
\begin{sol}
Para ter $\frac{1}{x(x^2+1)}=\frac{A}{x}+\frac{B}{x^2+1}$, isto é
$1=A(x^2+1)+Bx$, $A$ e $B$
devem satisfazer às três condições $A=0$, $B=0$, $A=1$, que obviamente é
impossível.
\end{sol}

\end{exo}

\begin{obs}
O esquema de decomposição usado em 
\eqref{eq:decomposp} pode ser generalizado:
$$
\frac{1}{(\alpha_1x^2+\beta_1)(\alpha_2x^2+\beta_2)\cdots
(\alpha_nx^2+\beta_n)}=\frac{A_1x+C_1}{\alpha_1x^2+\beta_1}+
\frac{A_2x+C_2}{\alpha_2 x^2+\beta_2}+\dots
+\frac{A_nx+C_n}{\alpha_n x^2+\beta_n}\,.
$$
Na expressão acima, todos os $\alpha_k>0$ e $\beta_k>0$.
\end{obs}

\begin{ex}\label{Ex:decompppp}
Considere $\int\frac{dx}{x(x+1)^2}$. Aqui o denominador contém o polinômio
irredutível $x+1$ elevado à potência $2$. Assim procuremos uma decomposição da
forma 
\eq{\label{eq:decompospp}
\frac{1}{x(x+1)^2}=\frac{A}{x}+\frac{B}{x+1}+\frac{C}{(x+1)^2}\,.
}
Igualando os numeradores, $1=A(x+1)^2+Bx(x+1)+Cx$, isto é
$(A+B)x^2+(2A+B+C)x+A-1=0$. Para isso valer para todo $x$, é preciso que sejam
satisfeitas as seguintes relações:
$$
A+B=0\,,\quad 2A+B+C=0\,,\quad A-1=0
$$
Assim vemos que $A=1$, $B=-1$, $C=-1$. Deixemos o leitor verificar a
decomposição.
Logo,
\begin{align*}
 \int\frac{dx}{x(x+1)^2}&=\int\Big\{\frac1x-\frac{1}{x+1}-\frac{1}{(x+1)^2}\}\,
dx\\
&=\ln|x|-\ln|x+1|+\frac{1}{x+1}+c\,.
\end{align*}
\end{ex}
\begin{obs}
A decomposição \eqref{eq:decompospp} pode ser usada a cada vez que aparece uma
potência de um fator irredutível. Por exemplo,
$$
\frac{1}{x(x+1)^4}=\frac{A}{x}+\frac{B}{x+1}+\frac{C}{(x+1)^2}+\frac{D}{(x+1)^3}
+\frac{E}{(x+1)^4}\,.
$$
\end{obs}

\begin{exo}
No Exemplo \ref{Ex:decompppp}, verifique que \emph{não tem} decomposição da
forma 
$\frac{1}{x(x+1)^2}=\frac{A}{x}+\frac{B}{(x+1)^2}$.
\begin{sol}
Para ter $\frac{1}{x(x+1)^2}=\frac{A}{x}+\frac{B}{(x+1)^2}$, isto é
$1=A(x+1)^2+Bx$, $A$ e $B$ precisariam satisfazer às três condições $A=0$,
$2A+B=0$, $A=1$, que obviamente é impossível.
\end{sol}
\end{exo}

Os métodos acima podem ser combinados:

\begin{ex}
Para $\int \frac{dx}{x^2(x^2+4)}$, procuremos uma decomposição da
forma
$$
\frac{1}{x^2(x^2+4)}=\frac{A}{x}+\frac{B}{x^2}+\frac{Cx+D}{x^2+4}\,.
$$
Igualando os numeradores e expressando os coeficientes do polinômio em função
de $A,B,C,D$ obtemos o seguinte sistema:
$$
A+C=0\,,\quad B+D=0\,,\quad 4A=0\,,\quad 4B=1\,.
$$
A solução é obtida facilmente: $A=0$, $B=\frac14$, $C=0$, $D=-\frac14$.
Logo,
$$\int \frac{dx}{x^2(x^2+4)}=
\tfrac14\int\frac{dx}{x^2}-\tfrac14\int\frac{dx}{x^2+4}=-\tfrac{1}{4x}
-\tfrac18\arctan(\tfrac{x}{2})+c\,.
$$
\end{ex}

\begin{exo}\label{Exo:PrimitivasDecomposicao}
Calcule as primitivas.
\begin{multicols}{4}
\begin{enumerate}
\item\label{itfracparciais1} $\int \frac{dx}{2x^2+1}$
\item\label{itfracparciais2} $\int\frac{x^5}{x^2+1}\,dx$
\item\label{itfracparciais3} $\int\frac{dx}{(x+2)^2}$ 
\item\label{itfracparciais30} $\int\frac{1}{x^2+x}\,dx$
\item\label{itfracparciais31} $\int\frac{1}{x^3+x}\,dx$
\item\label{itfracparciais4} $\int\frac{dx}{x^2+2x-3}$
\item\label{itfracparciais5} $\int\frac{dx}{x^2+2x+3}$
\item\label{itfracparciais50} $\int\frac{dx}{x(x-2)^2}$ 
\item\label{itfracparciais51} $\int\frac{dx}{x^2(x+1)}$
\item\label{itfracparciais7} $\int\frac{1}{t^4+t^3}dt$
\item\label{itfracparciais52} $\int\frac{dx}{x(x+1)^3}$
\item\label{itfracparciais9} $\int \frac{x^2+1}{x^3+x}\,dx$
\item\label{itfracparciais10} $\int \frac{x^3}{x^4-1}\,dx$
\item\label{itfracparciais104} $\int \frac{x\ln x}{(x^2+1)^2}\,dx$
\item\label{itfracparciais6} $\int\frac{dx}{x^3+1}$
\end{enumerate}
\end{multicols}
\vspace{0.01cm}
\begin{sol}
\eqref{itfracparciais1} $\tfrac{1}{\sqrt{2}}\arctan(\sqrt{2}x)+C$
\eqref{itfracparciais2} Como $\frac{x^5}{x^2+1}=x^3-x+\frac{x}{x^2+1}$, temos
$\int\frac{x^5}{x^2+1}\,dx=\tfrac{x^4}{4}-\tfrac{x^2}{2}+\tfrac12\ln (x^2+1)+C$.
\eqref{itfracparciais3} $\frac{-1}{x+2}+C$

\eqref{itfracparciais30}
A decomposição em frações parciais é da forma
$\frac{1}{x(x+1)}=\frac{A}{x}+\frac{B}{x+1}$.
Colocando no mesmo denominador, $A$ e $B$
tem que satisfazer $1=(A+B)x+A$ para todo $x$. Logo, $A=1$ e $B=-1$. Isto é,
$\frac{1}{x^2+x}=\frac{1}{x}-\frac{1}{x+1}$. Logo,
\begin{align*}
\int \frac{1}{x^2+x}\,dx&=\int \frac{1}{x}\,dx-\int\frac{1}{x+1}\,dx\\
&=\ln |x|-\ln |x+1|+C\,,\quad\quad 
\end{align*}
\eqref{itfracparciais31} 
O integrante é da forma $\frac{P(x)}{Q(x)}$, em que o grau 
de $P$ é menor do que o de $Q$. Além disso, podemos fatorar $x^3+x=x(x^2+1)$. O
polimômio de ordem $2$ tem discriminante negativo. Logo, é irredutível,
e podemos tentar uma decomposição da forma
$$
\frac{1}{x(x^2+1)}=\frac{A}{x}+\frac{Bx+C}{x^2+1}\quad \forall x\,.
$$
Colocando no mesmo denominador, $A$ $B$ e $C$ 
tem que satisfazer $1=(A+B)x^2+Cx+A$ para todo $x$. Logo, $A=1$, $C=0$, e
$B=-A=-1$. Isto é,
\begin{align*}
\int \frac{1}{x^3+x}\,dx=\int \frac{1}{x}\,dx-\int\frac{x}{x^2+1}\,dx
&=\ln |x|-\int\frac{x}{x^2+1}\,dx\\
&=\ln |x|-\tfrac{1}{2}\ln (x^2+1)+C\,,\quad\quad 
\end{align*}
Nesta última integral, fizemos $u=x^2+1$, $du=2x\,dx$.
\eqref{itfracparciais4} Como $\Delta=16>0$, podemos procurar fatorar e fazer uma
separação em frações parciais, 
$$\int\frac{dx}{x^2+2x-3}=\int\frac{dx}{(x+3)(x-1)}=-\tfrac14\int\frac{dx}{x+3}
+\tfrac14\int\frac{dx}{x-1}=\tfrac14\ln \Bigl|\frac{x-1}{x+3}\Bigr|+C\,.
$$
\eqref{itfracparciais5} Como $\Delta=-8<0$, o denominador não se fatora.
Completando o quadrado,
$$
\int\frac{dx}{x^2+2x+3}=\int\frac{dx}{(x+1)^2+2}=\tfrac12\int\frac{dx}{(\frac{x+
1}{\sqrt{2}})^2+1}=\tfrac{1}{\sqrt{2}}\arctan\bigl(\frac{x+
1}{\sqrt{2}}\bigr)+C\,.
$$
\eqref{itfracparciais50} Como
$\frac{1}{x(x-2)^2}=\frac{1}{4x}-\frac{1}{4(x-2)}+\frac{1}{2(x-2)^2}$, temos
$$
\int\frac{dx}{x(x-2)^2}=\tfrac14\ln|x|-\tfrac14\ln|x-2|-\frac{1}{2(x-2)}+C\,.
$$
\eqref{itfracparciais51}
$\frac{1}{x^2(x+1)}=\frac{A}{x}+\frac{B}{x^2}+\frac{C}{x+1}$, com $A=-1$,
$B=1$, $C=1$. Logo,
$$
\int\frac{dx}{x^2(x+1)}=-\ln |x|-\tfrac1x+\ln|x+1|+C'\,.
$$ 

\eqref{itfracparciais7} 
Como $t^4+t^3=t^3(t+1)$, procuramos uma separação da forma 
$$
\frac{1}{t^4+t^3}=\frac{A}{t}+\frac{B}{t^2}+\frac{C}{t^3}+\frac{D}{t+1}\,\quad
\forall t.
$$
Colocando no mesmo denominador e juntando os termos vemos que $A,B,C,D$ têm que
satisfazer 
$$
1=(A+D)t^3+(A+B)t^2+(B+C)t+C\quad\forall t\,.
$$
Identificando os coeficientes obtemos $C=1$, $B=-C=-1$, $A=-B=+1$, e
$D=-A=-1$. Isso implica
\begin{align*}
\int \frac{1}{t^4+t^3}dt&=\int\frac{dt}{t}-\int \frac{dt}{t^2}+\int
\frac{dt}{t^3}-\int \frac{dt}{t+1}\\
&=\ln|t|+\frac{1}{t}-\frac{1}{2t^2}-\ln|t+1|+C\,.
\end{align*}
\eqref{itfracparciais52} 
\begin{align*}
\int\frac{dx}{x(x+1)^3}
&=\int
\frac{dx}{x}-\int\frac{dx}{x+1}-\int\frac{dx}{(x+1)^2}-\int\frac{dx}{(x+1)^3}\\
&=\ln|x|-\ln|x+1|+\frac{1}{x+1}+\frac{1}{2(x+1)^2}+C\,.
\end{align*}
\eqref{itfracparciais9} $\int\frac{x^2+1}{x^3+x}\,dx=\int \frac{dx}{x}=\ln|x|+C$
\eqref{itfracparciais10} Com
$u=x^4-1$, $\int\frac{x^3}{x^4-1}\,dx=\tfrac14\ln|x^4-1|+C$ (é bem mais simples do que começar uma
decomposição em frações parciais...)
\eqref{itfracparciais104} Começando com uma integração por partes, 
\[ 
\int \frac{x\ln x}{(x^2+1)^2}\,dx=\frac{-1}{2(x^2+1)}\ln x+\frac12\int
\frac{1}{(x^2+1)x}\,dx\,,
\]
e essa última integral se calcula como no Exemplo \ref{Ex:decomppp}.
\eqref{itfracparciais6} Primeiro, observe que $x^3+1$ possui $x=-1$ como raiz.
Logo, ele pode ser fatorado como $x^3+1=(x+1)(x^2-x+1)$. 
Como $x^2-x+1$ tem um discriminante negativo,
procuremos uma decomposição da forma
$$
\frac{1}{x^3+1}=\frac{A}{x+1}+\frac{Bx+C}{x^2-x+1}\,.
$$
É fácil ver que $A$, $B$ e $C$ satisfazem às três condições $A+B=0$,
$-A+B+C=0$, $A+C=1$. Logo, $A=\frac13$, $B=-\frac13$, $C=\frac23$. Escrevendo
\begin{align*}
 \int\frac{dx}{x^3+1}&=\tfrac{1}{3}\int\frac{dx}{x+1}-\tfrac13\int
\frac{x-2}{x^2-x+1}\,dx\\
&=\tfrac{1}{3}\ln|x+1|-\tfrac13\int
\frac{x-2}{x^2-x+1}\,dx\\
\end{align*}
Agora, 
\begin{align*}
\int \frac{x-2}{x^2-x+1}\,dx&=\tfrac12\int \frac{2x-1}{x^2-x+1}\,dx-\tfrac{3}{2}
\int\frac{dx}{x^2-x+1}\\
&=\tfrac12 \ln|x^2-x+1|-\tfrac{3}{2}
\int\frac{dx}{x^2-x+1}\\
&=\tfrac12 \ln|x^2-x+1|-\tfrac{4}{\sqrt{3}}\arctan\bigl(\tfrac{2}{\sqrt{3}}
(x-\tfrac12) \bigr)+C\,.
\end{align*}
Juntando,
$$
\int\frac{dx}{x^3+1}=\tfrac{1}{3}\ln|x+1|-\tfrac16\ln|x^2-x+1|+\tfrac{4}{3\sqrt{
3}}\arctan\bigl(\tfrac{2}{\sqrt{3}}(x-\tfrac12) \bigr)+C\,.
$$ 
\end{sol}
\end{exo}


\begin{exo}\label{exo:primunsurseno}
Calcule $\int \frac{1}{\cos x}\,dx$. (Dica: multiplique e divida por $\cos x$.)
\begin{sol}
Com a dica, e a substituição $u=\sen x$,
\begin{align*}
\int \frac{dx}{\cos x}=\int\frac{\cos x}{1-\sen^2
x}dx=\int\frac{du}{1-u^2}&=-\int\frac{du}{u^2-1}\\
&=-\tfrac{1}{2}\ln\Bigl|\frac{u-1}{u+1}\Bigr|+C\\
&=\tfrac{1}{2}\ln\Bigl|\frac{1+\sen x}{1-\sen x}\Bigr|+C
\end{align*}
Observe que essa última expressão pode ser transformada da seguinte maneira:
\begin{align*}
\tfrac{1}{2}\ln\Bigl|\frac{\sen x+1}{\sen x-1}\Bigr|=
\tfrac{1}{2}\ln\Bigl|\frac{(1+\sen x)^2}{\cos^2x}\Bigr|=
\ln\Bigl|\frac{1+\sen x}{\cos x}\Bigr|=
\ln\Bigl|\frac{1}{\cos x}+\tan x\Bigr|\,.
\end{align*}
\end{sol}
\end{exo}

\begin{exo} (3a Prova 2010, Turmas N) Calcule  $\int\frac{x}{x^2+4x+13}dx$.
\begin{sol}
Como $\Delta=4^2-4\cdot 13<0$, o polinômio $x^2+4x+13$ tem discriminante
negativo. Logo, completando o quadrado:\index{completar um
quadrado}
$x^2+4x+13=(x+2)^2-4+13=(x+2)^2+9$, e
\begin{align*}
\int \frac{x}{x^2+4x+13}dx=\int
\frac{x}{(x+2)^2+9}dx=\tfrac19\int\frac{x}{(\tfrac13(x+2))^2+1}dx
\end{align*}
Com $u=\frac{1}{3}(x+2)$, $x=3u-2$, $3du=dx$,
\begin{align*}
 \tfrac19\int\frac{x}{(\tfrac13(x+2))^2+1}dx&=\tfrac13\int\frac{3u-2}{u^2+1}du\\
&=\tfrac12\int\frac{2u}{u^2+1}du-\tfrac23\int\frac{du}{u^2+1}\\
&=\tfrac12\ln (u^2+1)-\tfrac23 \arctan(u)+C\\
&=\tfrac12\ln (x^2+4x+13)-\tfrac23\arctan(\frac{1}{3}(x+2))+C
\end{align*}
\end{sol}
\end{exo}

%\fi
%\newpage

\subsection{Integrar potências de funções trigonométricas}
\index{integração!de funções trigonométricas}
Nesta seção estudaremos primitivas de funções que envolvem 
funções trigonométricas. Essas
aparecem em geral após ter feito uma \emph{substituição
trigonométrica}, que é o nosso último método de integração, e que será
apresentado na próxima seção.

\subsubsection{Primitivas das funções $\sen^mx\cos^nx$}\label{Sec:primsincos}
Aqui estudaremos primitivas da forma
$$\int \sen^mx\cos^nx\,dx\,.$$
Consideremos primeiro integrais contendo somente potências de $\sen x$, ou de
$\cos x$. Além dos casos triviais $\int \sen x\,dx=-\cos x+C$ e $\int \cos
x\,dx=\sen x+C$ já encontramos, no Exemplo \ref{Ex:sencarre},
$$\int\sen^2x\,dx=\int \tfrac{1-\cos(2x)}{2}\,dx=
\tfrac{x}{2}-\tfrac14 \sen(2x)+C\,.$$ 
Consequentemente,
\eq{\label{eq:intcoscarre}\int\cos^2x\,dx=\int\{1-\sen^2x\}\,dx=x-\int
\sen^2x\,dx=
\tfrac{x}{2}+\tfrac14 \sen(2x)+C\,.}
Potências \emph{ímpares} podem ser tratadas da seguinte maneira:
$$
\int \cos^3x\,dx=\int (\cos x)^2 \cos x\,dx=\int (1-\sen^2x)\cos x\,dx\,.
$$
Chamando $u\pardef \sen x$, obtemos
$$
\int \cos^3x\,dx=\int (1-u^2)\,du=u-\tfrac13 u^3+C=\sen x-\tfrac13\sen^3x+C\,.
$$
A mesma ideia pode ser usada para integrar
$\int \sen^mx\cos^nx\,dx$ quando \emph{pelo menos um dos expoentes, $m$ ou $n$,
é ímpar}. Por exemplo, 
\begin{align*}
\int \sen^2x\cos^3x\,dx&= \int \sen^2x\cos^2x\cos x\,dx\\
&=\int \sen^2x(1-\sen^2x)\cos x\,dx=\int u^2(1-u^2)\,du\,,
\end{align*}
onde $u=\sen x$. Logo,
$$
\int \sen^2x\cos^3x\,dx=\tfrac13u^3-\tfrac15 u^5+C=
\tfrac13\sen^3x-\tfrac15 \sen^5x+C\,.
$$
Para tratar potências \emph{pares}, comecemos usando uma
integração por partes. Por exemplo,
\begin{align*}
\int \cos^4x\,dx=\int \cos x\cos^3x\,dx&=\sen x\cos^3x-\int \sen x
(-3\cos^2x\sen x)\,dx \\
&=\sen x\cos^3x+3\int \sen^2 x
\cos^2x\,dx \\
&=\sen x\cos^3x+3\int (1-\cos^2x)
\cos^2x\,dx \\
&=\sen x\cos^3x+3\int\cos^2x\,dx
-3\int \cos^4x\,dx \\
\end{align*}
Isolando $\int \cos^4x\,dx$ nessa última expressão e usando
\eqref{eq:intcoscarre},
\eq{\label{eq:intcosquatre}
\int \cos^4x\,dx=\tfrac14\sen x\cos^3x+\tfrac{3x}{8}+\tfrac{3}{16}
\sen(2x)+C\,.
}


\begin{exo} Calcule as primitivas.
\begin{multicols}{3}
\begin{enumerate}
\item\label{itPotTrig0} $\int \sen^3x\,dx$
\item\label{itPotTrig01} $\int \cos^5x\,dx$
\item\label{itPotTrig1} $\int (\cos x\sen x)^5\,dx$
\item\label{itPotTrig10} $\int \cos^{1000} x\sen x\,dx$
\item\label{itPotTrig2} $\int (\sen^2t\cos t) e^{\sen t}\,dt$
\item\label{itPotTrig3} $\int \sen^3x \sqrt{\cos x}\,dx$
\item\label{itPotTrig4} $\int \sen^2x\cos^2x\,dx$
\end{enumerate}
\end{multicols}
\vspace{0.01cm}
\begin{sol}
\eqref{itPotTrig0} $-\cos x+\tfrac13\cos^3x+C$
\eqref{itPotTrig01} Com $u=\sen x$, $\int \cos^5x\,dx=\int
(1-u^2)^2\,du=\cdots=\sen x-\tfrac23\sen^3x+\tfrac15\sen^5x+C$
\eqref{itPotTrig1} Escrevemos
$\int (\cos x\sen x)^5dx=\int
\sen^5x(1-\sen^2x)^2\cos xdx$. 
Com $u=\sen x$ dá 
\begin{align*}
 \int \sen^5x(1-\sen^2x)^2\cos xdx&=
\int u^5(1-u^2)^2du\\
&=\int (u^5-2u^7+u^9)du\\
&=\frac{u^6}{6}-2\frac{u^8}{8}+\frac{u^{10}}{10}+C\\
&=\frac{\sen^6x}{6}-\frac{\sen^8x}{4}+\frac{\sen^{10}x}{10}+C\,.
\end{align*}
\eqref{itPotTrig10} $-\frac{\cos^{1001}x}{1001}+C$
\eqref{itPotTrig2}
Com $u=\sen t$,
$\int (\sen^2t\cos t) e^{\sen t}dt=\int u^2e^udu$.
Integrando duas vezes por partes e voltando para a variável $t$,
\begin{align*}
 \int u^2e^udu&=u^2e^u-\int (2u)e^udu\\
&=u^2e^u-2\big\{ue^u-\int e^udu\big\}\\
&=u^2e^u-2\{ue^u-e^u\}+C\\
&=e^u(u^2-2u+2)+C\\
&=e^{\sen t}(\sen^2 t-2\sen t+2)+C\,.
\end{align*}
\eqref{itPotTrig3} Com $u=\cos x$, 
$\int \sen^3x \sqrt{\cos x}\,dx=-\int(1-u^2)\sqrt{u}\,du=-\int
(u^{1/2}-u^{5/2})\,du=-\tfrac23 u^{3/2}+\tfrac27 u^{7/2}+C=-\tfrac23
(\cos x)^{3/2}+\tfrac27 (\cos x)^{7/2}+C$.
\eqref{itPotTrig4} $\int
\sen^2x\cos^2x\,dx=\int(1-\cos^2x)\cos^2x\,dx=\int\cos^2x\,dx-\int\cos^4x\,dx$,
e essas duas primitivas já foram calculadas anteriormente. 
\end{sol}
\end{exo}


\subsubsection{Primitivas das funções $\tan^mx\sec^nx$}
Nesta seção estudaremos primitivas da forma
$$\int \tan^mx\sec^nx\,dx\,,$$
onde lembramos que a função \grasA{secante} é definida como 
$$
\sec x\pardef \frac{1}{\cos x}\,.$$
Como $1+\tan^2x=1+\frac{\sen^2x}{\cos^2x}=\frac{1}{\cos^2x}$,
a seguinte relação vale: 
$$1+\tan^2x=\sec^2x\,.$$
Lembramos que
$(\tan x)' =1+\tan^2x=\sec^2x$.
Então, para calcular por exemplo 
\eq{\label{eq:primtances}\int\tan x\sec^2x\,dx\,,}
podemos chamar $u=\tan x$, $du=\sec^2x \,dx$, e escrever
$$
\int\tan x\sec^2x\,dx=
\int u\,du=\tfrac{1}{2}u^2+C=\tfrac12 \tan^2x+C\,.
$$
Na verdade, é facil ver que a mesma substituição pode ser usada \emph{a cada
vez que a potência da secante é par}. Por exemplo,
\begin{align*}
\int\tan x\sec^4x\,dx=\int \tan x\sec^2 x(\sec^2x)\,dx&=
\int \tan x(1+\tan^2x)(\sec^2x)\,dx\\
&=\int u(1+u^2)\,du\\
&=\tfrac12 u^2+\tfrac14 u^4+C\\
&=\tfrac12 (\tan x)^2+\tfrac14(\tan x)^4+C\,.
\end{align*}

Por outro lado, a relação
$$(\sec
x)'=\frac{\sen x}{\cos^2x}\equiv \tan x\sec x\,$$
permite um outro tipo de substituição. 
Por exemplo, \eqref{eq:primtances} pode ser calculada também via a
mudança de variável $w=\sec x$, $dw=\tan x\sec x\,dx$:
$$
\int\tan x\sec^2x\,dx=\int \sec x(\tan x\sec x)\,dx=\int w\,dw=\tfrac12 w^2+C
=\tfrac12 \sec^2x+C\,.
$$
A mesma mudança de variável $w=\sec x$ se aplica \emph{a cada vez que a potência
da tangente é ímpar (e que a potência da secante é pelo menos $1$)}. Por
exemplo,
\begin{align*}
\int  \tan^3x\sec x\,dx&=\int \tan^2 x(\tan x\sec x)\,dx\\
&=\int (\sec^2x-1) (\tan x\sec x)\,dx\\
&=\int(w^2-1)\,dw\\
&=\tfrac13 w^3-w+C\\
&=\tfrac13 \sec^3x-\sec x+C\,.
\end{align*}
Os casos em que a potência da tangente é ímpar e que não tem secante
são tratados separadamente. 
Por exemplo, lembramos que 
$$
\int \tan x\,dx=\int\frac{\sen x}{\cos x}\,dx=-\ln|\cos x|+C\,.
$$
Ou, 
\begin{align*}
\int \tan^3 x\,dx&=\int \tan x(\tan^2 x)\,dx\\
&=\int\tan x(\sec^2x-1)\,dx=\int \tan x \sec^2x\,dx-\int \tan x\,dx\,,
\end{align*}
e essas duas primitivas já foram calculadas acima.
Finalmente, deixemos o leitor fazer o Exercício
\ref{exo:primunsurseno} para mostrar que
$$
\int \sec x\,dx=\ln\bigl|\sec x+\tan x\bigr|+C\,.
$$
\begin{exo}\label{Exo:PrimitTangSec}
Calcule as primitivas.
\begin{multicols}{3}
\begin{enumerate}
\item\label{itInttansec3} $\int \sec^2x\,dx$
\item\label{itInttansec1} $\int\tan^2x \,dx$
\item\label{itInttansec1a} $\int\tan^3x \,dx$
\item\label{itInttansec6} $\int \tan x\sec x\,dx$
\item\label{itInttansec2} $\int\tan^4 x\sec^4x\,dx$
\item\label{itInttansec4} $\int \cos^5x\tan^5x\,dx$
%\item\label{itInttansec5} $\int \sec^4x\tan^4x\,dx$
\item\label{itInttansec7} $\int \sec^5x\tan^3x\,dx$
\item\label{itInttansec8} $\int \sec^3x\,dx$
\end{enumerate}
\end{multicols}
\vspace{0.01cm}
\begin{sol}
\eqref{itInttansec3} $\int \sec^2x\,dx=\tan x+C$.
\eqref{itInttansec1} $\int\tan^2x \,dx=\int(\tan^2x+1-1)\,dx=\tan x-x+C$.
\eqref{itInttansec1a} $\int\tan^3x \,dx=\int\tan x(1+\tan^2x)\,dx-\int \tan x\,dx=\tfrac12\tan^2 x-\ln
|\cos x|+C$.
\eqref{itInttansec6} $\int \tan x\sec x\,dx=\sec x+C$.
\eqref{itInttansec2} $\int\tan^4 x\sec^4x\,dx=\int
\tan^4x(\tan^2x+1)\sec^2x\,dx=\int u^4(u^2+1)\,du=
\tfrac17u^7+\tfrac15u^5+C=\tfrac17\tan^7x+\tfrac15\tan^5x+C$.
\eqref{itInttansec4} $\int\cos^5x\tan^5x\,dx=\int
\sen^5x\,dx=\int(1-\cos^2x)^2\sen x\,dx=
-\int(1-u^2)^2\,du=-u+\tfrac23u^3-\tfrac15 u^5+C=
-\cos x+\tfrac23\cos^3x-\tfrac15 \cos^5x+C$.
\eqref{itInttansec7} $\int \sec^5x\tan^3x\,dx=\int \sec^4x(\sec^2x-1)(\tan x\sec
x)\,dx=\int w^4(w^2-1)\,dw=\tfrac17 w^7-\tfrac15 w^5+C=\tfrac17 \sec^7x-\tfrac15
\sec^5x+C$.
\eqref{itInttansec8} Por partes (lembra que
$(\sec\theta)'=\tan\theta\sec\theta$):
\begin{align*}
 \int\sec^2\theta\sec\theta\,d\theta
&=\tan \theta\sec\theta-\int\tan^2\theta\sec\theta\,d\theta\\
&=\tan \theta\sec\theta-\int(\sec^2\theta-1)\sec\theta\,d\theta\,.
\end{align*}
Logo,
$$
\int\sec^3\theta\,d\theta=
\tfrac12\tan\theta\sec\theta+\tfrac12\int\sec\theta\,d\theta\,.
$$
Já calculamos a primitiva de $\sec \theta$ no Exercício
\ref{exo:primunsurseno}: 
$\int\sec\theta\,d\theta=\ln\bigl|\sec\theta+\tan \theta\bigr|+C$. Logo,
$$
\int\sec^3\theta\,d\theta=
\tfrac12\tan\theta\sec\theta+\tfrac12\ln\bigl|\sec\theta+\tan \theta\bigr|+C\,.
$$
\end{sol}
\end{exo}

\subsection{Substituições trigonométricas}\label{Sec:MetodoSubstitTrig}
\index{substituição! trigonométrica}
Nesta seção final apresentaremos métodos para calcular primitivas de funções
particulares onde aparecem raizes de polinômio do segundo grau:
$$
\int \sqrt{1-x^2}\,dx\,,\quad
\int x^3{\sqrt{1-x^2}}\,dx\,,\quad
\int \frac{dx}{\sqrt{x^2+2x+2}}\,,\quad
\int x^3\sqrt{x^2-3}dx\,,\dots
$$

O nosso objetivo é fazer uma substituição \emph{que transforme o
polinômio que está dentro da raiz em um quadrado perfeito}. 
Essas substituições serão baseadas nas seguintes idenditades trigonométricas:

\eq{\label{eq:relatsinusss}1-\sen^2\theta=\cos^2\theta\,,}
\eq{\label{eq:relattangente} 1+\tan^2\theta=\sec^2\theta\,.}

Ilustraremos os métodos em três exemplos elementares, integrando
$\sqrt{1-x^2}$, $\sqrt{1+x^2}$ e $\sqrt{x^2-1}$. Em seguida aplicaremos as
mesmas ideias em casos mais gerais.

\subsubsection{A primitiva $\int \sqrt{1-x^2}\,dx$}

Observe primeiro que $\sqrt{1-x^2}$ é bem definido se $x\in [-1,1]$.
Para calcular $\int \sqrt{1-x^2}\,dx$
usaremos \eqref{eq:relatsinusss} para
transformar $1-x^2$ em um quadrado perfeito.
Portanto, consideremos a substituição 
$$x=\sen \theta\,,\quad dx=\cos \theta\,d\theta\,.$$
Como $x\in [-1,1]$, 
essa substituição é bem definida, e implica que $\theta$ pode ser
escolhido $\theta\in [-\pisobredois,\pisobredois]$:
\begin{center}
\begin{bmlimage}\begin{tikzpicture}[scale=1]
\pgfmathsetmacro{\a}{2.3};
\draw[->] (0,-\a) -- (0,\a);
\draw[->] (0,0)-- (\a,0);
\draw (0,0)--(1.1,1.665);
\draw [thick] (1.1,1.665)--(1.1,0) node[midway, above, sloped]{$x$};
%\draw [color=\coultang, thick] (2,3)--(2,0) node[midway, above, sloped]{$\tan \alpha$};
%\draw [color=\coulcoseno, thick] (1.1,0)--(0,0) node[midway, below]{$\cos \alpha$};
%\draw (1.1,1.665) node[above]{$B$};
\draw[dotted] (0,-2) arc (-90:90:2);
%\draw(0,0)--(2,3);
%\draw (0,0)--(2,0);
\draw (0.5,1.1) node{$1$};
\draw[ ->] (0.5,0) arc (0:1 r:0.5);
\draw (0.4,0.3) node[right]{$\theta$};
\fill (1.1,1.665) circle (0.45mm);
\draw (0,-2) node[left]{$-1$};
\draw (0,2) node[left]{$+1$};
\end{tikzpicture}\end{bmlimage}
\end{center}

Expressemos agora a primitiva somente em termos de $\theta$:
$$
\int \sqrt{1-x^2}\,dx=\int \sqrt{1-\sen^2\theta}\cos \theta\,d\theta=
\int \sqrt{\cos^2\theta}\cos \theta\,d\theta=\int \cos^2\theta\,d\theta\,.
$$
De fato, como $\theta\in [-\pisobredois,\pisobredois]$, 
$\cos \theta\geq 0$, o que
significa $\sqrt{\cos^2\theta}=\cos \theta$. Mas a primitiva de $\cos^2\theta$ é

$$
\int\cos^2\theta\,d\theta=\tfrac{1}{2}\theta+\tfrac{1}{4}\sen (2\theta)+C\,.
$$
Agora precisamos voltar para a variável $x$. Primeiro,
$x=\sen \theta$ implica $\theta=\arcsen x$. Por outro lado, $\sen
(2\theta)=2\sen \theta\cos\theta=2x\sqrt{1-x^2}$. Logo, 
$$
\boxed{\int \sqrt{1-x^2}\,dx=\tfrac12\arcsen
x+\tfrac12x\sqrt{1-x^2}+C\,.}
$$

\begin{exo}
Verifique esse último resultado, derivando com respeito a $x$.
\begin{sol}
De fato,
\begin{align*}
\bigl(\tfrac12\arcsen x+\tfrac12x\sqrt{1-x^2}\bigr)'&=
\tfrac12\frac{1}{\sqrt{1-x^2}}+\tfrac12
\sqrt{1-x^2}+\tfrac12 x\frac{-2x}{2\sqrt{1-x^2}} \\
&=\tfrac12\frac{1-x^2}{\sqrt{1-x^2}}+\tfrac12
\sqrt{1-x^2}\\
&=\tfrac12 \sqrt{1-x^2}+\tfrac12 \sqrt{1-x^2}=\sqrt{1-x^2}\,.
\end{align*}
\end{sol}
\end{exo}

O método descrito acima costuma ser eficiente
a cada vez que se quer integrar uma função 
que contém uma raiz da forma $\sqrt{a^2-b^2x^2}$, com $a,b>0$.
Para transformar o polinómio $a^2-b^2x^2$ em um quadrado perfeito, podemos
tentar as seguintes subsituições: 
$$x\pardef \tfrac{a}{b}\sen \theta\,,\text{  ou  }\quad x\pardef
\tfrac{a}{b}\cos \theta\,.$$
De fato, uma substituição desse tipo permite cancelar a raiz:
$$
\sqrt{a^2-b^2(\tfrac{a}{b}\sen \theta)^2}=
\sqrt{a^2-a^2\sen^2\theta}=a\sqrt{1-\sen^2\theta}=a\cos \theta\,.
$$
Depois de ter feito a substituição, aparece em geral uma primitiva 
de potências de funções trigonométricas, parecidas com aquelas encontradas na
Seção \ref{Sec:primsincos}.

\begin{ex} Neste exemplo verificaremos que a área de um disco de raio $R$ é
igual a $\pi R^2$.
\begin{center}
\begin{bmlimage}\begin{tikzpicture}
\pgfmathsetmacro{\R}{2};
\pgfmathsetmacro{\alf}{-40};
\pgfmathsetmacro{\bet}{50};
\filldraw[areagrafico] (0,0)--(\R,0) arc (0:90:\R)--cycle;
\draw[>=latex, ->] ({-\R-0.3},0)--({\R+0.3},0);
\draw[>=latex, ->] (0,{-\R-0.3})--(0,{\R+0.3});
\draw (\R,0) arc (0:360:\R);
\draw[thick] (\R,0) arc (0:90:\R);
\draw[->] (0,0)--({\R*cos(\alf)},{\R*sin(\alf)}) node[midway, above]{$R$};
\draw[<-]
({\R*cos(\bet)},{\R*sin(\bet)})--(\R,{\R})node[right]{$y=f(x)=\sqrt{R^2-x^
2}$};
\end{tikzpicture}\end{bmlimage}
\end{center}
A área do disco completo é dada pela integral
$$
A=4\int_{0}^R\sqrt{R^2-x^2}\,dx.
$$
Usemos a substituição trigonométrica $x=R\sen \theta$, $dx=R\cos
\theta\,d\theta$. Se $x=0$, então $\theta=0$, e se $x=R$
então $\theta=\pisobredois$. Logo,
\begin{align*}
 \int_{0}^R\sqrt{R^2-x^2}\,dx&=\int_0^{\pisobredois}\sqrt{R^2-(R\sen
\theta)^2}R\cos \theta\,d\theta\\
&=R^2\int_0^{\pisobredois} \cos^2\theta\,d\theta\\
&=R^2\bigl\{\tfrac12\theta+\tfrac14\sen(2\theta)\bigr\}_0^{\pisobredois}\\
&=R^2\frac{\pi}{4}\,.
\end{align*}
Logo, $A=4 R^2\frac{\pi}{4}=\pi R^2$.
\end{ex}

\begin{ex}
Calculemos a primitiva $\int x^3{\sqrt{4-x^2}}\,dx$. 
Usemos a substituição $x=2\sen \theta$, $dx=2\cos \theta\,d\theta$.
Como $x\in [-2,2]$, temos $\theta\in [-\pisobredois, \pisobredois]$.
$$
\int x^3{\sqrt{4-x^2}}\,dx=\int(2\sen \theta)^3\sqrt{4-(2\sen \theta)^2}2\cos
\theta\,d\theta=32\int\sen^3\theta\cos^2\theta\,d\theta\,.
$$
A última primitiva se calcula feito na seção anterior: com $u=\cos \theta$,
\begin{align*}
\int\sen^3\theta\cos^2\theta\,d\theta&=\int
(1-\cos^2\theta)\cos^2\theta\sen\theta\,d\theta\\
&=-\int (1-u^2)u^2\,du=-\tfrac13 u^3+\tfrac15u^5+C
=-\tfrac13 \cos^3\theta+\tfrac15\cos^5\theta+C\,.
\end{align*}
Para voltar para a variável $x$, observe que $x=2\sen \theta$ implica $\cos
\theta=\sqrt{1-\sen^2\theta}=\sqrt{1-(\tfrac{x}{2})^2}=\sqrt{1-\tfrac{x^2}{4}}$.
Logo,
$$
\int
x^3{\sqrt{4-x^2}}\,dx=-\tfrac{32}{3}\sqrt{1-\tfrac{x^2}{4}}^3+\tfrac{32}{5}\sqrt{1-\tfrac{
x^2}{4}}^5+C\,.
$$
\end{ex}

\begin{exo}
Calcule a área da região delimitada pela elipse cuja equação é
dada por
$$
\frac{x^2}{\alpha^2}+\frac{y^2}{\beta^2}=1\,,
$$
Em seguida, verifique que quando a elipse é um círculo, $\alpha=\beta=R$, a sua área é
$\pi R^2$.  
\begin{sol}
A área é dada por $A=4\int_0^{\alpha}\beta\sqrt{1-\frac{x^2}{\alpha^2}}\,dx$.
Com $x=\alpha\sen \theta$, 
$$A=4\beta\int_0^{\alpha}\sqrt{1-\frac{x^2}{\alpha^2}}\,dx=
4\alpha\beta \int_0^{\pisobredois}\cos^2\theta\,d\theta=\pi \alpha\beta\,.
$$
Quando $\alpha=\beta=R$, a elipse é um disco de raio $R$, de área $\pi 
R\cdot R=\pi R^2$.
\end{sol}
\end{exo}


\begin{ex}
Considere $\int \frac{dx}{x\sqrt{5-x^2}}$.
Com $x=\sqrt{5}\sen \theta$, obtemos
$$\int \frac{dx}{x\sqrt{5-x^2}}=
\int\frac{\sqrt{5}\cos \theta}{(\sqrt{5}\sen \theta)\sqrt{5-(\sqrt{5}\sen
\theta)^2}}\,d\theta
=\tfrac{1}{\sqrt{5}}\int\frac{d\theta}{\sen \theta}\,.
$$
Essa última primitiva pode ser tratada como no Exercício
\ref{exo:primunsurseno}:
$$
\int\frac{d\theta}{\sen \theta}=\tfrac12\ln \Bigl|\frac{1-\cos
\theta}{1+\cos\theta}\Bigr|+C=\tfrac{1}{2}\ln\Bigl|\frac{1-\sqrt{1-\frac{x^2}{5}
} } { 1+\sqrt{1-\frac{x^2}{5}
} } \Bigr|+C\,.
$$
Logo,
$$
\int
\frac{dx}{x\sqrt{5-x^2}}=\tfrac{1}{2\sqrt{5}}
\ln\Bigl|\frac{\sqrt{5}-\sqrt{5-{x^2}
} } { \sqrt{5}+\sqrt{5-{x^2}
} } \Bigr|+C\,.
$$
%\ref{Ex:unsurxdeuxmun}
\end{ex}

\begin{exo} Calcule as primitivas
\begin{multicols}{3}
\begin{enumerate}
\item\label{itPrimSubSinus1} $\int \frac{dx}{\sqrt{1-x^2}}$
%\item\label{itSubstitTrig0} $\int\frac{dx}{\sqrt{3-x^2}}$
%\item\label{itSubstitTrig00} $\int\frac{x}{\sqrt{3-x^2}}\,dx$
%\item\label{itSubstitTrig4} $\int x^3\sqrt{3-x^2}\,dx$.
\item\label{itSubstitTrig2} $\int\frac{x^7}{\sqrt{10-x^2}}\,dx$.
\item\label{itPrimSubSinus3} $\int \frac{x^2}{\sqrt{1-x^3}}\,dx$
\item\label{itPrimSubSinus2} $\int x\sqrt{1-x^2}\,dx$
\item\label{itSubstitTrig6} $\int \frac{x}{\sqrt{3-2x-x^2}}\,dx$.
\item\label{itSubstitTrig000} $\int x^2{\sqrt{9-x^2}}\,dx$
\end{enumerate}
\end{multicols}
\vspace{0.01cm}
\begin{sol}
\eqref{itPrimSubSinus1}
Sabemos que $\int \frac{dx}{\sqrt{1-x^2}}=\arcsen x+C$, mas isso pode ser
verificado de novo fazendo a substituição $x=\sen \theta$:
$\frac{dx}{\sqrt{1-x^2}}=\int\frac{1}{\sqrt{1-\sen^2\theta}}\cos\theta\,
d\theta\int d\theta=\theta+C=\arcsen x+C$.
%\eqref{itSubstitTrig00} $\int\frac{x}{\sqrt{3-x^2}}\,dx=...$
\eqref{itSubstitTrig2} Com $x=\sqrt{10}\sen t$ dá
\begin{align*}
\int\frac{x^7}{\sqrt{10-x^2}}dx=\int\frac{\sqrt{10}^7\sen^7t}{\sqrt{10}\cos
t}\sqrt{10}\cos tdt
&=\sqrt{10}^7\int \sen^7tdt
\end{align*}
Uma segunda substituição $u=\cos t$ dá
\begin{align*}
\int \sen^7tdt&=\int (1-\cos^2t)^3\sen tdt\\
\quad&=-\int(1-u^2)^3du\\
&=-\int(1-3u^2+3u^4-u^6)du \\
&=-\Big\{u-u^3+\frac{3}{5}u^5-\frac{1}{7}u^7\Big\}+C
\end{align*}
Para voltar para $x$, observe que $u=\cos
t=\sqrt{1-\sen^2t}=\sqrt{1-(x/\sqrt{10})^2}$.
Logo,
$$
\int\frac{x^7}{\sqrt{10-x^2}}dx=\sqrt{10}^7\Bigl\{-\sqrt{1-\frac{x^2}{10}}
+\sqrt{1-\frac{x^2}{10}}^3-\frac{3}{5}\sqrt{1-\frac{x^2}{10}}^5
+\frac{1}{7}\sqrt{1-\frac{x^2}{10}}^7\Bigr\}+C
$$
\eqref{itPrimSubSinus3} Observe que $\sqrt{1-x^3}$ \emph{não é da forma
$\sqrt{a^2-b^2x^2}$!} Mas com a substituição $u=1-x^3$, 
$\int \frac{x^2}{\sqrt{1-x^3}}\,dx=-\tfrac13\int
\frac{du}{\sqrt{u}}=-\tfrac23\sqrt{u}+C=-\tfrac23\sqrt{1-x^3}+C$.
\eqref{itPrimSubSinus2} Aqui uma simples substituição $u=1-x^2$ dá
$\int x\sqrt{1-x^2}\,dx=-\tfrac13(1-x^2)^{3/2}+C$. (Pode também fazer $x=\sen
\theta$, é um pouco mais longo.)
\eqref{itSubstitTrig6} Completando o quadrado, 
$3-2x-x^2=4-(x+1)^2$. Chamando $x+1=2\sen \theta$,
\begin{align*}
\int\frac{x}{\sqrt{3-2x-x^2}}\,dx=\int\frac{2\sen
\theta-1}{\sqrt{4-4\sen^2\theta}}2\cos \theta\,d\theta&=
2\int \sen \theta\,d\theta-\int \,d\theta\\
&=-2\cos\theta-\theta+C\,.
\end{align*}
Voltando para $x$, temos
$$
\int\frac{x}{\sqrt{3-2x-x^2}}\,dx=-2\sqrt{1-(\tfrac{x+1}{2})^2}
-\arcsen(\tfrac{x+1}{2})+C\,.$$
\eqref{itSubstitTrig000} Com $x=3\sen \theta$ obtemos 
$\int x^2{\sqrt{9-x^2}}\,dx=3^4\int \sen^2\theta\cos^2\theta\,d\theta$.
\end{sol}
\end{exo}


\subsubsection{A primitiva $\int \sqrt{1+x^2}\,dx$}

Para calcular $\int \sqrt{1+x^2}\,dx$
usaremos \eqref{eq:relattangente} para
transformar $1+x^2$ em um quadrado perfeito.
Portanto, consideremos a substituição 
$$x=\tan \theta\,,\quad dx=\sec^2 \theta\,d\theta\,.$$
Expressemos agora a primitiva somente em termos de $\theta$:
$$
\int \sqrt{1+x^2}\,dx=\int \sqrt{1+\tan^2\theta}\sec^2 \theta\,d\theta=
\int \sqrt{\sec^2\theta}\sec^2\theta\,d\theta=\int \sec^3\theta\,d\theta\,.
$$
Vimos no Exercício \ref{Exo:PrimitTangSec} que
$$
\int\sec^3\theta\,d\theta=
\tfrac12\tan\theta\sec\theta+\tfrac12\ln\bigl|\sec\theta+\tan
\theta\bigr|+C\,.
$$
Para voltar à variável $x$: $\sec\theta=x$,
$\tan\theta=\sqrt{1+\sec^2\theta}=\sqrt{1+x^2}$.
Logo,
$$\boxed{\int \sqrt{1+x^2}dx=\tfrac12
x\sqrt{1+x^2}+\tfrac12 \ln |x+\sqrt{1+x^2}|
+C\,.}$$

O método descrito acima se aplica a cada vez que se quer integrar uma função 
que contém uma raiz da forma $\sqrt{a^2+b^2x^2}$, com $a,b>0$.
Para transformar o polinómio $a^2+b^2x^2$ em um quadrado perfeito, podemos
tentar as seguintes subsituições: 
$$x\pardef \tfrac{a}{b}\tan \theta \,.$$
De fato, uma substituição desse tipo permite cancelar a raiz:
$$
\sqrt{a^2+b^2(\tfrac{a}{b}\tan \theta)^2}=
\sqrt{a^2+a^2\tan^2\theta}=a\sqrt{1+\tan^2\theta}=a\sec \theta\,.
$$

\begin{exo}
Calcule as primitivas
\begin{multicols}{3}
\begin{enumerate}
\item\label{itSubstitTrig3} $\int\frac{x^3}{\sqrt{4x^2+1}}\,dx$.
\item\label{itSubstitTrig31} $\int x^3\sqrt{x^2+1}\,dx$
\item\label{itSubstitTrig32} $\int x\sqrt{x^2+a^2}\,dx$
\item\label{itSubstitTrig33} $\int \frac{dx}{\sqrt{x^2+2x+2}}$ 
\item\label{itSubstitTrig10} $\int\frac{dx}{(x^2+1)^3}$
\item\label{itSubstitTrig100} $\int\frac{dx}{x^2\sqrt{x^2+4}}$
\end{enumerate}
\end{multicols}
\vspace{0.01cm}
\begin{sol}
\eqref{itSubstitTrig3}
fazendo $x=\tfrac12\tan \theta$ dá
\begin{align*}
 \int \frac{x^3}{\sqrt{4x^2+1}}dx&=\int \frac{(\tfrac12 \tan
\theta)^3}{\sqrt{\sec^2\theta}}
\half\sec^2\theta d\theta\\
&=\tfrac{1}{16} \int\tan^3\theta \sec\theta d\theta\\
&=\tfrac{1}{16} \int (\sec^2\theta-1)\sec\theta \tan\theta d\theta
\end{align*}
Com $w=\sec \theta$, obtemos 
$\int (\sec^2\theta-1)\sec\theta \tan\theta
d\theta=\frac{\sec^3\theta}{3}-{\sec\theta}+C$. 
Mas $\tan \theta=2x$ implica $\sec \theta=\sqrt{\tan^2\theta+1}=\sqrt{1+4x^2}$.
Logo,
$$
\int \frac{x^3}{\sqrt{4x^2+1}}dx=\frac{(1+4x^2)^{\frac{3}{2}}}{48}
-\frac{\sqrt{1+4x^2}}{16}+C\,.
$$
Observe que pode também rearranjar um pouco a função e fazer por partes:
\begin{align*}
 \int \frac{x^3}{\sqrt{4x^2+1}}dx&=
 \tfrac14\int x^2\frac{8x}{2\sqrt{4x^2+1}}dx\\
&=\tfrac14\Bigl\{x^2\sqrt{4x^2+1}-\int (2x)\sqrt{4x^2+1}dx\Bigr\}\\
&=\tfrac14\Bigl\{x^2\sqrt{4x^2+1}-\tfrac14\frac{(4x^2+1)^{3/2}}{3/2}\Bigr\}+C\,,
\end{align*}
dá na mesma!
\eqref{itSubstitTrig31} Com $x=\tan \theta$, temos
\begin{align*}
\int x^3\sqrt{x^2+1}\,dx&=\int \tan^3\theta\sec^3\theta\,d\theta\\
&=\int (\sec^2\theta-1)\sec^2\theta(\tan\theta\sec\theta)\,d\theta\\
(\text{via }w=\sec\theta)\,
&=\tfrac15\sec^5\theta-\tfrac13\sec^3\theta+C\\
&=\tfrac15(x^2+1)^{5/2}-\tfrac13(x^2+1)^{3/2}+C\,.
\end{align*}
\eqref{itSubstitTrig32} Aqui não precisa fazer substituição trigonométrica:
$u=x^2+a^2$ dá $\int
x\sqrt{x^2+a^2}\,dx=\tfrac12\int\sqrt{u}\,du=\tfrac13u^{3/2}+C=
\tfrac13(x^2+a^2)^{3/2}+C$.
\eqref{itSubstitTrig33} Como $x^2+2x+2=(x+1)^2+1$, a substituição
$x+1=\tan\theta$ dá 
$\int
\frac{dx}{\sqrt{x^2+2x+2}}=\int\frac{\sec^2\theta}{\sec\theta}\,
d\theta=\int\sec\theta\,d\theta=\ln|\sec\theta+\tan\theta|+C=\ln\bigl|x+1+\sqrt{
x^2+2x+2}\bigr|+C$.
\eqref{itSubstitTrig10} Apesar da função $\frac{1}{(x^2+1)^3}$ não possuir
raizes, façamos a substituição $x=\tan\theta$:
\begin{align*}
\int\frac{dx}{(x^2+1)^3}&=\int\frac{\sec^2\theta}{(\tan^2\theta+1)^3}\,
d\theta=\int\frac{d\theta}{\sec^4\theta}=\int\cos^4\theta\,d\theta\,.
\end{align*}
Essa última primitiva já foi calculada em \eqref{eq:intcosquatre}:
$\int \cos^4\theta\,d \theta=\tfrac14\sen
\theta\cos^3\theta+\tfrac{3\theta}{8}+\tfrac{3}{16}
\sen(2\theta)+C$. Ora, se $\tan\theta=x$, então
$\sen\theta=\frac{x}{\sqrt{1+x^2}}$ e $\cos\theta=\frac{1}{\sqrt{1+x^2}}$.
Logo,
$$
\int\frac{dx}{(x^2+1)^3}=\frac{x}{4(1+x^2)^2}+\frac38\Bigl\{\arctan
x+\frac{x}{1+x^2}\Bigr\}+C\,.
$$
\eqref{itSubstitTrig100} Com $x=2\tan \theta$, 
$\int\frac{dx}{x^2\sqrt{x^2+4}}=\tfrac14\int
\frac{\cos\theta}{\sen^2\theta}\,d\theta=-\frac{1}{4\sen\theta}+C$.
Agora observe que $2\tan \theta=x$ implica $\sen\theta=\frac{x}{\sqrt{x^2+4}}$.
Logo,
$\int\frac{dx}{x^2\sqrt{x^2+4}}=-\frac{\sqrt{x^2+4}}{4x}+C$.
\end{sol}
\end{exo}

\begin{exo}\label{exo:comprparabola}
Calcule o comprimento do arco da parábola $y=x^2$, contido 
entre as retas $x=-1$ e  $x=1$.
\begin{sol}
Já montamos a integral no Exemplo \ref{ex:comprparabdifiss}, e esta pode ser
calculada com os métodos dessa seção: 
$L=2\int_0^1\sqrt{1+4x^2}\,dx=\frac{\sqrt{5}}{4}+\frac12\ln(\frac12+\frac{\sqrt{5}}{2})$.
\end{sol}
\end{exo}

\subsubsection{A primitiva $\int \sqrt{x^2-1}\,dx$}
Finalmente, consideremos a primitiva $\int \sqrt{x^2-1}\,dx$. Para transformar
$x^2-1$ num quadrado perfeito, usaremos a relação \eqref{eq:relattangente}:
$\sec^2\theta-1=\tan^2\theta$. Assim, chamando $x=\sec\theta$, temos
$dx=\tan\theta\sec\theta\,d\theta$, portanto

$$\int
\sqrt{x^2-1}\,dx=\int\sqrt{\sec^2\theta-1}\tan\theta\sec\theta\,d\theta
=\int \tan^2\theta\sec\theta\,d\theta\,.$$
Integrando por partes,
\begin{align*}
\int
(\tan\theta\sec\theta)\tan\theta\,
d\theta&=\sec\theta\tan\theta-\int\sec^3\theta\, d\theta\\
&=\sec\theta\tan\theta-\Bigl\{
\tfrac12\tan\theta\sec\theta+\tfrac12\ln\bigl|\sec\theta+\tan
\theta\bigr|
\Bigr\}+C\\
&=\tfrac12\sec\theta\tan\theta-\tfrac12\ln\bigl|\sec\theta+\tan
\theta\bigr|+C\,.
\end{align*}
Como $\sec\theta=x$ implica
$\tan\theta=\sqrt{\sec^2\theta-1}=\sqrt{x^2-1}$, obtemos
$$\boxed{
\int\sqrt{x^2-1}\,dx=\tfrac12
x\sqrt{x^2-1}-\tfrac12 \ln\bigl|x+\sqrt{x^2-1}\bigr|+C\,.}
$$
O método apresentado acima sugere que para integrar uma função que contém um
polinômio do segundo grau da forma $\sqrt{a^2x^2-b^2}$, pode-se tentar fazer a
substituição $$x\pardef \frac{b}{a}\sec \theta\,.$$

\begin{ex}
Consideremos a primitiva $\int\frac{dx}{x^2\sqrt{x^2-9}}$, fazendo a
substituição $x=3\sec\theta$, $dx=3\tan\theta\sec\theta\,d\theta$:
$$
\int\frac{dx}{x^2\sqrt{x^2-9}}=\int\frac{3\tan\theta\sec\theta}{(3\sec\theta)^2
\sqrt{(3\sec\theta)^2-9}}\,d\theta
=\tfrac{1}{9}\int\frac{d\theta}{\sec\theta}=
\tfrac19\int\cos\theta\,d\theta
=\tfrac19\sen\theta+C\,.
$$
Para voltar à variável $x$, façamos uma interpretação geométrica da nossa
substituição. A relação $x=3\sec\theta$, isto é $\cos\theta =\frac{3}{x}$, se
concretiza no seguinte triângulo:
\begin{center}
\begin{bmlimage}\begin{tikzpicture}[scale=1.5]
\draw (0,0)--(2,1) node[midway, above, sloped]{$x$}--(2,0) 
node[midway, right]{$\sqrt{x^2-9}$} --(0,0) node[midway,
below]{$3$};
\draw (0.4,0) arc (0:26.56:0.4);
\draw (0.56,0.13) node{$\theta$};
\draw (4.8,0.5) node{$\Rightarrow\,\,\sen\theta=\frac{\sqrt{x^2-9}}{x}$};
\end{tikzpicture}\end{bmlimage}
\end{center}
Assim, 
$$
\int\frac{dx}{x^2\sqrt{x^2-9}}=\frac{\sqrt{x^2-9}}{9x}+C\,.
$$
\end{ex}

\begin{exo} Calcule as primitivas.
\begin{multicols}{3}
\begin{enumerate}
\item\label{itPrimSubSecante1} $\int x^3\sqrt{x^2-3}dx$
\item\label{itSubstitTrig8} $\int \frac{dx}{\sqrt{x^2-a^2}}\,dx$.
\item\label{itSubstitTrig1} $\int\frac{x^3}{\sqrt{x^2-1}}\,dx$
\end{enumerate}
\end{multicols}
\vspace{0.01cm}
\begin{sol}
\eqref{itPrimSubSecante1}
Seja $x=\sqrt{3}\sec \theta$. Então $dx=\sqrt{3}\sec \theta\tan \theta$, e
\begin{align*}
 \int x^3\sqrt{x^2-3}dx&=\int (\sqrt{3}\sec \theta)^3 \sqrt{3}  \tan \theta
\sqrt{3}\sec \theta\tan \theta d\theta\\
&=\sqrt{3}^5\int \{\sec^2\theta \tan^2 \theta\}\sec^2 \theta d\theta\\
(\text{ com }u=\tan \theta)&=\sqrt{3}^5\int (u^2+1)u^2du\\
&=\sqrt{3}^5(u^5/5+u^3/3)+C
\end{align*}
Mas como $\cos \theta=\sqrt{3}/x$, temos (fazer um desenho) $u=\tan
\theta=\sqrt{x^2-3}/\sqrt{3}$. Logo,
$$
\int x^3\sqrt{x^2-3}dx=\tfrac15\sqrt{x^2-3}^5+\sqrt{x^2-3}^3+C
$$
Um outro jeito de calcular essa primitiva é de começar com uma integração por
partes:
\begin{align*}
\int x^3\sqrt{x^2-3}dx=
 \tfrac12\int x^2\,
\big\{2x\sqrt{x^2-3}\big\}dx&=\tfrac12\Big\{x^2\frac{(x^2-3)^{3/2}}{3/2}-\int
2x\frac{(x^2-3)^{3/2}}{3/2}dx\Big\}\\
&=\tfrac12\Big\{x^2\frac{(x^2-3)^{3/2}}{3/2}-\tfrac23\int
2x{(x^2-3)^{3/2}}dx\Big\}\\
&=\tfrac12\Big\{x^2\frac{(x^2-3)^{3/2}}{3/2}-\tfrac23\frac{(x^2-3)^{5/2}}{5/2}
\Big\}+C\\
&=\tfrac13x^2{(x^2-3)^{3/2}}-\tfrac{2}{15}{(x^2-3)^{5/2}}+C\,\,)
\end{align*}
\eqref{itSubstitTrig8} Com $x=a\sec\theta$, 
$\int
\frac{dx}{\sqrt{x^2-a^2}}\,dx=\int\sec\theta\,
d\theta=\ln|\sec\theta+\tan\theta|+C$. Como $\cos\theta=\frac{a}{x}$,
\begin{center}
\begin{bmlimage}\begin{tikzpicture}[scale=1.5]
\draw (0,0)--(2,1) node[midway, above, sloped]{$x$}--(2,0) 
node[midway, right]{$\sqrt{x^2-a^2}$} --(0,0) node[midway,
below]{$a$};
\draw (0.4,0) arc (0:26.56:0.4);
\draw (0.56,0.13) node{$\theta$};
\draw (4.8,0.5) node{$\Rightarrow\,\,\tan\theta=\frac{\sqrt{x^2-a^2}}{a}$};
\end{tikzpicture}\end{bmlimage}
\end{center}
Logo,
$\int
\frac{dx}{\sqrt{x^2-a^2}}\,dx=\ln|\tfrac{x}{a}+\tfrac{\sqrt{x^2-a^2}}{a}|+C$.
\eqref{itSubstitTrig1}
Com $x=\sec \theta$, $dx=\sec \theta\tan \theta
d\theta$:
\begin{align*}
\int\frac{x^3}{\sqrt{x^2-1}}dx&=\int 
\frac{\sec^3\theta}{\tan \theta}\sec \theta\tan \theta d\theta\\
&=\int \sec^2\theta \sec^2\theta d\theta\\
&=\int (\tan^2\theta+1)\sec^2\theta d\theta\\
(u\pardef \tan \theta)\quad&=\int (u^2+1)du\\
&=\frac{u^3}{3}+u+C\\
&=\frac{\tan^3\theta}{3}+\tan\theta+C\,.
\end{align*}
Mas $\sec \theta=x$ implica $\tan\theta=\sqrt{x^2-1}$.
Logo,
$$
\int\frac{x^3}{\sqrt{x^2-1}}dx=
\frac{1}{3}(x^2-1)^{\tfrac32}+\sqrt{x^2-1}+C\,.
$$

\end{sol}
\end{exo}



%\fi


