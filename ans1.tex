\protect \section *{Capítulo \ref {Cap_Fundam}}
\begin{Solution}{1.1}
\eqref{it0} $S=\{0\}$
\eqref{it01} $S=\{\pm 1\}$
 \eqref{it02} Observe primeiro que $0$ não é solução (a divisão por zero no lado esquerdo
não é nem definida). Assim, multiplicando por $x$ e rearranjando obtemos $x^2+x-1=0$. Como
$\Delta=5>0$, obtemos duas soluções: $S=\{\tfrac{-1\pm \sqrt{5}}{2}\}$. (Obs: o número
$\tfrac{-1+\sqrt{5}}{2}=0.618033989...$ é às vezes chamado de \grasA{razão áurea}. Veja
$\verb|http://pt.wikipedia.org/wiki/Proporção_áurea|$)
 \eqref{it1} Para ter $(x+1)(x-7)=0$, é necessário que pelo menos um dos fatores, $(x+1)$
ou $(x-7)$, seja nulo. Isto é, basta ter $x=-1$ ou $x=7$. Assim, $S=\{-1,7\}$. Obs:
querendo aplicar a fórmula $x=\frac{-b\pm\sqrt{b^2-4ac}}{2a}$ de qualquer jeito, um aluno
com pressa pode querer expandir o produto $(x+1)(x-7)$ para ter $x^2-6x-7=0$, calcular
$\Delta=(-6)^2-4\cdot 1\cdot (-7)=64$, e obter
$S=\{\frac{-(-6)\pm\sqrt{64}}{2\cdot 1}\}=\{-1,7\}$.
 Mas além de mostrar uma falta de compreensão (pra que expandir uma expressão já
fatorada!?), isso implica aplicar uma fórmula e fazer \emph{contas}, o que cria várias
oportunidades de errar!)
\eqref{it2} $S=\bR$ (qualquer $x$ torna a equação verdadeira!)
 \eqref{it20} $S=\{0,1\}$
\eqref{it3} $S=\varnothing$
\eqref{it4} $S=\{-\tfrac13\}$
\eqref{it500} $S=\{\frac{-7\pm \sqrt{29}}{2}\}$.
\end{Solution}
\begin{Solution}{1.3}
 Resposta: não.
Sejam $a$ e $b$ os catetos do triângulo. Para ter uma área de $7$, é preciso ter
$\frac{ab}{2}=7$. Para ter um perímetro de $12$, é preciso ter $a+b+\sqrt{a^2+b^2}=12$
(o comprimento da hipotenusa foi calculada com o Teorema de Pitágoras).
Essa última expressão é equivalente a $12-a-b=\sqrt{a^2+b^2}$, isto é (tomando o quadrado
em ambos lados) $144-24(a+b)+2ab=0$. Como $b=\frac{14}{a}$, esta equação se reduz a uma
equação da única incógnita $a$: $6a^2-43a+84=0$. Como essa equação tem $\Delta=-167<0$,
não existe triângulo retângulo com aquelas propriedades.
\end{Solution}
\begin{Solution}{1.4}
$A=[-2,2]$, $B=[0,1)$, $C=(-\infty,0)$, $D=\varnothing$, $E=\bR$, $F=\{1\}$, $G=\{0\}$,
$H=\bR_+$.
\end{Solution}
\begin{Solution}{1.5}
 \eqref{itinequ1} $(-1,\infty)$
 \eqref{itinequ2} $(-\infty,\tfrac12]$
 \eqref{itinequ3} $(-\tfrac34,\infty)$
 \eqref{itinequ4} $(0,\infty)$
 \eqref{itinequ5} $(-\infty,-1]\cup [1,\infty)$
 \eqref{itinequ6} $\varnothing$
 \eqref{itinequ7} $\varnothing$
 \eqref{itinequ8} $\bR$
 \eqref{itinequ9} $(-\infty,0]\cup [1,\infty)$ Obs: aqui, um erro comum é de começar
dividindo ambos lados de $x\leq x^2$ por $x$, o que dá $1\leq x$. Isso dá somente uma
parte do conjunto das soluções, $[1,\infty)$, porque ao dividir por $x$, é preciso
considerar também os casos em que $x$ é negativo. Se $x$ é negativo, dividir por $x$ dá
$1\geq x$ (invertemos o sentido da desigualdade), o que fornece o outro pedaço das
soluções: $(-\infty,0]$.
 \eqref{itinequ10} $(-\infty,2)\cup (3,\infty)$
\eqref{itinequ10b} $(-\infty,-7]\cup \{0\}$
 \eqref{itinequ11} $(-1,+1)\cup (2,+\infty)$
\eqref{itinequ12} $[0,+\infty[$
\eqref{itinequ13} $S=(-\infty,-1]\cup (1,3]$. Cuidado: tem que excluir o valor
$x=1$ para evitar a divisão por zero\index{divisão por zero} e a inequação ser bem
definida.
\eqref{itt6} Primeiro observemos que os
valores $x=0$ e $x=-2$ são proibidos. Em seguida, colocando no mesmo denominador,
queremos resolver $\frac{2}{x(x+2)}\geq 0$. Isso é equivalente a resolver $x(x+2)\geq 0$,
cujo conjunto de soluções é dado por $(-\infty,-2]\cup [0,\infty)$. Logo,
$S=(-\infty,-2)\cup (0,\infty)$ (tiramos os dois valores proibidos).
\eqref{itt7} $S=(-\infty,0)\cup(2,\infty)$.
\eqref{itt7a} $S=(-\infty,-2]\cup [0,\tfrac43]\cup [3,+\infty)$.
\end{Solution}
\begin{Solution}{1.6}
Um só: $n=1$.
\end{Solution}
\begin{Solution}{1.7}
Resolvendo $0\leq 2x-3$ obtemos $S_1=[\tfrac32,\infty)$, e resolvendo $2x-3\leq
x+8$ obtemos $S_2=(-\infty,11]$. Logo, $S=S_1\cap S_2=[\tfrac32,11]$ é solução
das duas inequações no mesmo tempo. Mas esse intervalo contém os primos
$p=2,3,5,7,11$. Logo, a resposta é: $5$.
\end{Solution}
\begin{Solution}{1.8}
A expressão correta é a terceira, e vale para qualquer $x\in \bR$.
A primeira está certa quando $x\geq 0$, mas errada quando $x<0$ (por exemplo,
$\sqrt{(-3)^2}=\sqrt{9}=3(\neq -3)$). A segunda também está certa quando $x\geq
0$, mas $\sqrt{x}$ não é nem definido quando $x<0$.
\end{Solution}
\begin{Solution}{1.9}
\eqref{itt1} Observe que como um valor absoluto é sempre $\geq 0$,
qualquer $x$ é solução de $|x+27|\geq 0$. Logo, $S=\bR$. \eqref{itt2} Como no
item anterior, $|x-2|\geq 0$ para qualquer $x$. Logo, não tem nenhum $x$ tal que
$|x-2|<0$, o que implica $S=\varnothing$. \eqref{itt3} Para ter $|2x+3|>0$, a
única possibilidade é de excluir $|2x+3|=0$. Como isso acontece se e somente se
$2x+3=0$, isto é se e somente se $x=-\tfrac32$, temos $S=\bR\setminus
\{-\tfrac32\}=(-\infty,-\tfrac32)\cup(-\tfrac32,+\infty)$.
\eqref{itt4} Considere primeiro o caso em que $3-x\geq 0$ (isto é $x\leq 3$). A inequação
se torna $3<3-x$, isto é $x<0$. Logo, $S_1=(-\infty,0)$. No caso em que $3-x\leq 0$ (isto
é $x\geq 3$), a inequação se torna $3<-(3-x)$, isto é $x>6$. Assim, $S_2=(6,+\infty)$.
Finalmente, $S=S_1\cup S_2=(-\infty,0)\cup ]6,+\infty)$.
\eqref{itt4bis} $S=\varnothing$
\eqref{itt5} $S=[-\sqrt{2},\sqrt{2}]$. Observe que por
\eqref{eq:consequvalabsol}, $|x^2-1|\leq 1$ se e somente se
$-1\leq x^2-1\leq 1$. Assim, resolvendo separadamente as inequações $-1\leq x^2-1$ e
$x^2-1\leq 1$ leva ao mesmo conjunto de soluções.
\eqref{itt8} $S=(\tfrac43,2)\cup (2,4)$.
\end{Solution}
\begin{Solution}{1.10}
\eqref{itexsinal1} $<0$ se $x<-5$, $>0$ se $x>-5$, nula se $x=-5$.
\eqref{itexsinal2} $>0$ para todo $x\in \bR$.
\eqref{itexsinal21} $>0$ se $x\in\bR\setminus \{5\}$, nula se $x=5$.
\eqref{itexsinal3} $>0$ se $x\in (-\infty,-\sqrt{5})\cup (\sqrt{5},\infty)$, $<0$ se
$x\in (-\sqrt{5},\sqrt{5})$, nula se $x=\pm \sqrt{5}$
\eqref{itexsinal4} $>0$ se $x\in (-\infty,-8)\cup (2,6)$, $<0$ se $x\in (-8,2)\cup
(6,\infty)$, nula se $x\in \{-8,6\}$. Observe que a expressão \emph{não é definida em
$x=2$}.
\eqref{itexsinal5} $>0$ se $x\in (-1,1)\cup(1,\infty)$, $<0$ se $x<-1$, nula se $x\in
\{-1,1\}$.
\end{Solution}
\begin{Solution}{1.11}
\eqref{itplano1} $\{(x,y):y> 0\}$,
\eqref{itplano2} $\{(x,y):x< 0\}$,
\eqref{itplano3} $\{(x,y):|x|\leq \half, |y|\leq \half\}$,
\eqref{itplano4} $\{(x,y): x=2\}$,
\eqref{itplano5} $\{(x,y): y=-5\}$,
\eqref{itplano6} $\{(x,y): y=-5\}$,
\eqref{itplano7} $\{(x,y): 0\leq x\leq 2\}$,
\eqref{itplano8} $\{P=(x,y): d(P,(0,0))=1\}=\{(x,y):x^2+y^2=1\}$,
\eqref{itplano9} $\{P=(x,y): d(P,(1,-2))\leq 2\}=\{(x,y):(x-1)^2+(y+2)^2\leq 4\}$,
\end{Solution}
\begin{Solution}{1.12}
$R=(-\frac{391}{3},100)$, $T=(6,-\frac{9}{4})$.
\end{Solution}
\begin{Solution}{1.13}
\eqref{itreta1} $y=x$,
\eqref{itreta2} $y=1$,
\eqref{itreta3} $x=-3$,
\eqref{itreta4} $y=-\tfrac{5}{2}x+\tfrac12$,
\eqref{itreta5} $y=\tfrac{2}{3}x+5$.
\end{Solution}
\begin{Solution}{1.14}

\mbox{}
\begin{center}
\begin{bmlimage}\begin{tikzpicture}[scale=0.6]
\draw [ ->] (0,-3.3)--(0,2.4) node[left]{$y$};
\draw [ ->] (-5,0)--(5.4,0) node[right]{$x$};
\foreach \k in {-5,...,5}
{\draw ({\k},0) node{$\shortmid$};}
\foreach \k in {-3,...,2}
{\draw (0,{\k}) node{$-$};}
%%%%%%%%
\draw [very thick] (4,-2.5)--(4,2) node[right]{$r_1$};
%%%%%%%%%%%%
%\draw (4.2,0.2) node{$4$};
\draw [very thick] (4.8,-1.5)--(-4.5,-1.5) node[left]{$r_2$};
%\draw (0,-1.5) node[above left]{$-\tfrac{3}{2}$};
%%%%%%%%%%%%
%\draw (4.2,0.2) node{$4$};
\pgfmathsetmacro{\a}{-0.1}
\pgfmathsetmacro{\b}{2.3}
\draw [very thick] (\a,{(2*\a)-3})--(\b,{2*\b-3}) node[above]{$r_4$};
%%%
\pgfmathsetmacro{\c}{-3}
\pgfmathsetmacro{\d}{4.5}
\draw [very thick] (\d,{(-\d)/2})--(\c,{(-\c)/2}) node[left]{$r_3$};
\end{tikzpicture}\end{bmlimage}
\end{center}
\end{Solution}
\begin{Solution}{1.15}
\eqref{ittexreta1} $r':\,y=5x+10$.
\eqref{ittexreta2} $r':\,y=\tfrac{4}{3}x-9$
\end{Solution}
\begin{Solution}{1.16}
 Comecemos com um exemplo: considere a reta $r_1$ de inclinação $m_1=\tfrac13$
que passa pela origem. Qual é a equação da reta $r_2$, perpendicular a $r_1$, que passa
pela origem?
\begin{center}
 \begin{bmlimage}\begin{tikzpicture}
  \draw [ ->] (0,-0.1)--(0,3.2) node[right]{$y$};
\draw [ ->] (-1.5,0)--(3.5,0) node[right]{$x$};
\draw[thick] (-1,-0.33)--(3.3,1.1) node[right]{$r_1$};
\draw[dashed] (0.2,-0.6)--(-1.1,3.3) node[left]{$r_2$};
\fill (3,1) circle (0.55mm);
\draw (3,1) node[above left]{$P_1$};
\fill (-1,3) circle (0.55mm);
\draw (-1,3) node[above right]{$P_2$};
\foreach \k in {-1,...,3}
{\draw ({\k},0) node{$\shortmid$};}
\foreach \k in {0,...,3}
{\draw (0,{\k}) node{$-$};}
 \end{tikzpicture}\end{bmlimage}
\end{center}
 Observe que se $P_1=(3,1)\in r_1$, então o ponto $P_2=(-1,3)\in r_2$, já que o segmento
$OP_1$ precisa ser perpendicular a $OP_2$. Logo, a inclinação de $r_2$ pode ser obtida
usando o ponto $P_2$:
$m_2=\frac{0-3}{0-(-1)}=-3$,
 o que prova $m_2=-\frac{1}{m_1}$. Escolhendo qualquer outro ponto $P_1=(x,y)$ em $r_1$,
obteríamos um ponto $P_2=(-y,x)$, e $m_2$ seria calculada da mesma maneira.\\

 Para uma reta de inclinação $m_1$ qualquer, podemos escolher $P_1=(1,m_1)$ e
$P_2=(-m_1,1)$, assim $m_2=\frac{0-1}{0-(-m_1)}=-\frac{1}{m_1}$ é sempre verificada.
\end{Solution}
\begin{Solution}{1.17}
$r_2$ e $r_4$ são paralelas, e ambas são perpendiculares a $r_3$.
\end{Solution}
\begin{Solution}{1.18}
\eqref{itexcirc1} $C=(0,-1)$, $R=3$.
\eqref{itexcirc2} não é círculo: $-1$ não é um quadrado.
\eqref{itexcirc3} $C=(3,0)$, $R=3$.
 \eqref{itexcirc4} não é círculo: depois de ter completado o quadrado obtemos
$(x+\half)^2+(y+\half)^2=-\half$, que não é um quadrado.
 \eqref{itexcirc5} não é círculo: depois de ter completado o quadrado obtemos
$(x+1)^2+y^2=0$ (que poderia ser interpretado como um círculo de raio $R=0$ centrado em
$(-1,0)$).
\eqref{itexcirc6} não é círculo ($x^2-y^2=1$ é
uma \emph{hipérbole}).
\end{Solution}
\begin{Solution}{1.19}
 Durante uma hora e quinze minutos, o ponteiro dos segundos
dá $75$ voltas.
Como uma volta representa uma distância percorrida (pela ponta) de
$2\times \pi\times
20\simeq 125.66$ centímetros, a distância total é de $\simeq 9424.5$
centímetros, o que corresponde a $\simeq 94.25$ metros.
\end{Solution}
\begin{Solution}{1.21}
\mbox{}

\begin{center}
\begin{bmlimage}\begin{tikzpicture}[scale=2]
 \draw (0,0)--(0.5,0) node[midway, below]{$\tfrac12$} --(0.5,0.866)--(0,0) node[midway,
above left]{$1$};
\draw (0.6,0.35) node{$\frac{\sqrt{3}}{2}$};
\draw (0.225,0.15) node{$\tfrac{\pi}{3}$};
\draw (0.2,0) arc (0:60:0.2);
\draw (0.4,0.5) node{$\tfrac{\pi}{6}$};
\draw (0.4,0.7) arc (240:265:0.25);
\draw (1.5,0.7) node[right]{$\Rightarrow\,\sen \tfrac{\pi}{3}=\frac{\sqrt{3}}{{2}}\,, \quad
\cos \tfrac{\pi}{3}=\frac{1}{{2}}\,,\quad \tan \tfrac{\pi}{3}=\sqrt{3}$\,.};
\draw (1.5,0.2) node[right]{$\Rightarrow\,\sen \tfrac{\pi}{6}=\frac{1}{{2}}\,, \quad
 \cos \tfrac{\pi}{6}=\frac{\sqrt{3}}{{2}}\,,\quad \tan
\tfrac{\pi}{6}=\tfrac{1}{\sqrt{3}}$\,.};
\draw[dotted] (0.5,0)--(1,0)--(0.5,0.866)--cycle;
\end{tikzpicture}\end{bmlimage}
\end{center}
\end{Solution}
\begin{Solution}{1.22}
$H=\frac{d(\tan\beta-\tan\alpha)}{\tan \alpha\tan\beta}$.
\end{Solution}
\begin{Solution}{1.23}
Todas essas identidades seguem da observação do círculo trigonométrico. Por exemplo,
o desenho
\begin{center}
\begin{bmlimage}\begin{tikzpicture}[scale=3]
\pgfmathsetmacro{\a}{1};
\draw[dotted] (\a,0) arc (0:180:\a);
\draw[ ->, color=gray!70] (-1.1*\a,0) -- (1.1*\a,0);
\draw[ ->, color=gray!70] (0,0) -- (0,1.1*\a);

\pgfmathsetmacro{\alf}{35};

%DESSINER LES ANGLES:
\draw[ ->] ({0.4*\a},0) arc (0:\alf:{0.4*\a});
%\draw ({(\alf)/2}:{(\a)*(0.45)}) node{$\alpha$};
\draw ({\alf/2}:{\a*0.45}) node{$\alpha$};
\draw[ ->] ({0.3*\a},0) arc (0:180-\alf:{0.3*\a});
\draw ({(180-\alf)/2}:{0.35*\a}) node[above]{$\pi-\alpha$};

%DEFINIR LES POINTS:
\coordinate (B) at ({\a*cos(\alf)},{\a*sin(\alf)});
\draw (B) node[above right]{$B$};
\fill (B) circle (0.15 mm);
\draw (0,0)--(B);

\coordinate (C) at ({\a*cos(180-\alf)},{\a*sin(180-\alf)});
%\draw (C) node[above left]{$C$};
\fill (C) circle (0.15 mm);
\draw (0,0)--(C);

\draw[dotted] (C)--(B);

 \draw [color=\coulseno, thick] (B)--({\a*cos(\alf)},0) node[midway, above, sloped]{$\sen
\alpha$};
 \draw [color=\coulcoseno, thick] ({\a*cos(\alf)},0)--(0,0) node[midway, below]{$\cos
\alpha$};
 \draw [color=\coulseno, thick] (C)--({\a*cos(180-\alf)},0) node[midway, above,
sloped]{$\sen (\pi-\alpha)$};
 \draw [color=\coulcoseno, thick] ({\a*cos(180-\alf)},0)--(0,0) node[midway, below]{$\cos
(\pi-\alpha)$};

\end{tikzpicture}\end{bmlimage}
\end{center}
mostra que $\cos(\pi-\alpha)=-\cos\alpha$ e $\sen(\pi-\alpha)=\sen\alpha$.
 Como consequência,
$$\tan(\pi-\alpha)=\frac{\sen(\pi-\alpha)}{\cos(\pi-\alpha)}=-\tan \alpha\,.$$
Deixemos o leitor provar as identidades parecidas com $\pi+\alpha$.
Por outro lado, o desenho
\begin{center}
\begin{bmlimage}\begin{tikzpicture}[scale=3]
\pgfmathsetmacro{\a}{1};
\draw[dotted] (\a,0) arc (0:90:\a);
\draw[ ->, color=gray!70] (0,0) -- (1.1*\a,0);
\draw[ ->, color=gray!70] (0,0) -- (0,1.1*\a);

\pgfmathsetmacro{\alf}{35};

%DESSINER LES ANGLES:
\coordinate (P) at ({0.4*\a*cos(\alf)},{0.4*\a*sin(\alf)});
\draw[ <-] (P) arc (\alf:90:{0.4*\a});
\draw[ ->] ({0.4*\a},0) arc (0:\alf:{0.4*\a});
\draw ({\alf/2}:{0.45*\a}) node{$\alpha$};
\draw ({\alf+(90-\alf)/2}:{0.35*\a}) node[above right]{$\tfrac{\pi}{2}-\alpha$};

%DEFINIR LES POINTS:
\coordinate (B) at ({\a*cos(\alf)},{\a*sin(\alf)});
\coordinate (Bx) at ({\a*cos(\alf)},0);
\coordinate (By) at (0,{\a*sin(\alf)});

\draw (B) node[above right]{$B$};
\draw (0,0)--(B);
\draw [color=\coulseno, thick] (B)--(Bx) node[midway, above, sloped]{$\sen \alpha$};
\draw [color=\coulcoseno, thick] (Bx)--(0,0) node[midway, below]{$\cos \alpha$};
 \draw [color=\coulseno, thick] (By)--(B) node[midway, above,
sloped]{$\sen(\tfrac{\pi}{2}- \alpha)$};
 \draw [color=\coulcoseno, thick] (By)--(0,0) node[midway, below,
sloped]{$\cos(\tfrac{\pi}{2}- \alpha)$};

\fill (B) circle (0.15 mm);
\end{tikzpicture}\end{bmlimage}
\end{center}
 mostra que $\cos(\tfrac{\pi}{2}-\alpha)=\sen\alpha$ e
$\sen(\tfrac{\pi}{2}-\alpha)=\cos\alpha$.
Como consequência,
$$
 \tan
(\tfrac{\pi}{2}-\alpha)=\frac{\sen(\tfrac{\pi}{2}-\alpha)}{\cos(\tfrac{\pi}{2}-\alpha)}=
\frac{\cos\alpha}{\sen\alpha}\equiv \frac{1}{\tan \alpha}=\cot \alpha\,.
$$
\end{Solution}
\begin{Solution}{1.25}
 \eqref{eqsensomabis} segue de \eqref{eqsensoma} trocando $\beta$ por $-\beta$ e usando
\eqref{eqtrigo0}. Para provar
\eqref{eqcossoma}, basta usar  \eqref{eqsensomabis} da seguinte maneira:
\begin{align*}
 \cos(\alpha+\beta)&=\sen\bigl(\tfrac{\pi}{2}-(\alpha+\beta)\bigr)\\
&=\sen\bigl((\tfrac{\pi}{2}-\alpha)-\beta)\bigr)\\
&=\sen(\tfrac{\pi}{2}-\alpha)\cos\beta-\cos (\tfrac{\pi}{2}-\alpha)\sen \beta\\
&=\cos \alpha\cos\beta-\sen\alpha\sen\beta\,.
\end{align*}
Para \eqref{eqtansoma},
\begin{align*}
 \tan(\alpha+\beta)=\frac{\sen(\alpha+\beta)}{\cos(\alpha+\beta)}=
\frac{\sen\alpha\cos\beta+\cos \alpha\sen\beta}{
\cos\alpha\cos\beta-\sen \alpha\sen\beta}
=\frac{\tan \alpha+\tan\beta}{1-\tan\alpha\tan\beta}\,.
\end{align*}
 A última igualdade foi obtida dividindo o numerador e o denominador por
$\cos\alpha\cos\beta$.
\end{Solution}
\begin{Solution}{1.26}
As duas primeiras seguem das identidades anteriores, com $\beta=\alpha$.
A terceira obtem-se escrevendo:
$$
\sen\alpha=\sen(2\tfrac{\alpha}{2})=2\sen\tfrac{\alpha}{2}\cos\tfrac{\alpha}{2}=
2\tan\tfrac{\alpha}{2}\cos^2\tfrac{\alpha}{2}=\tan\tfrac{\alpha}{2}(\cos\alpha+1)\,.
$$
 Será que você consegue provar \eqref{eqidentandoisalpha} somente a partir do círculo
trigonométrico?

A última, \eqref{eqidentantreixx}, se obtem facilmente a partir de $\cos(\alpha\pm \beta)$. Observe que
a relação \eqref{eqidentantreixx} é a base da técnica chamada \emph{ring modulation} em música
eletrônica.
\end{Solution}
\begin{Solution}{1.27}
 A inclinação é dada por $\tan 60^o=\tan \frac{\pi}{3}=\sqrt{3}$ (Exercício
\ref{exo:calculsimple60}). Logo, a equação é $y=\sqrt{3}x-1-2\sqrt{3}$.
\end{Solution}
\begin{Solution}{1.28}
Observe que boa parte das equações desse exercício possuem \emph{infinitas} soluções!
As soluções obtêm-se essencialmente olhando para o círculo trigonométrico.
\eqref{itOinequ1} $S=\{\pisobredois\pm k\pi,\,k\in \bZ\}$.
\eqref{itOinequ10} $S=\{\pisobreseis\pm k2\pi\}\cup \{\tfrac{5\pi}{6}\pm k2\pi\}$
\eqref{itOinequ11} $S=\{\pisobrequatro\pm k\pi,\,k\in \bZ\}$.
\eqref{itOinequ2} $S=\{\pm k\pi\}\cup \{\pisobredois+2k\pi\}$.
\eqref{itOinequ3}  Observe que $z\pardef \sen x$ satisfaz $z^2+\tfrac{3}{2}z-1=0$, isto é
$z=\tfrac{1}{2}$ ou $-2$. Como o seno somente toma valores entre $-1$ e $1$, $\sen x=-2$
não possui soluções. Por outro lado, $\sen x=\half$ possui as soluções $\{\pisobreseis\pm
k2\pi\}\cup \{\tfrac{5\pi}{6}\pm k2\pi\}$, como visto em \eqref{itOinequ10}.
Portanto, $S=\{\pisobreseis\pm k2\pi\}\cup \{\tfrac{5\pi}{6}\pm k2\pi\}$.
\eqref{itOinequ4}  $S=[\tfrac{\pi}{6},\tfrac{5\pi}{6}]$ e as suas translações de $\pm
2k\pi$.
\eqref{itOinequ5}
$S=(\tfrac{\pi}{4},\tfrac{3\pi}{4})\cup(\tfrac{5\pi}{4},\tfrac{7\pi}{4})$ e as suas
translações de $\pm 2k\pi$.
 \eqref{itOinequ6} Rearranjando obtemos $\sen (2x)=-\tfrac12$, o que significa $2x\in
\{\frac{7\pi}{6}\pm 2k\pi\}\cup \{\frac{11\pi}{6}\pm 2k\pi\}$. Logo,
$S= \{\frac{7\pi}{12}\pm k\pi\}\cup \{\frac{11\pi}{12}\pm k\pi\}$
 \eqref{itOinequ7} $S=\{k\pi,k\in\bZ\}\cup \{\pisobretres+2k\pi,k\in\bZ\}\cup
\{\tfrac{5\pi}{3}+2k\pi,k\in\bZ\}$.
\end{Solution}
\protect \section *{Capítulo \ref {Cap:Funcoes}}
\begin{Solution}{2.1}
\eqref{itte1} $D=\bR\setminus\{-8,5\}$
\eqref{itte2} $D=\bR\setminus\{0\}$
\eqref{itte21} $D=\bR$
\eqref{itte3} $D=\bR$
\eqref{itte4} $D=\bR\setminus\{0,\tfrac12\}$
\eqref{itte45} $D=[1,\infty)$
\eqref{itte46} $D=(-\infty,-1]\cup [1,\infty)$
\eqref{itte5} $D=[1,\infty ) \setminus \{2\}$
\eqref{itdominio1} $D=\bR\setminus\{\pm 1\}$
\eqref{itdominio2} $D=(-1,+1)$
\eqref{itdominio3} $D=\{1\}$
 \eqref{itdominio4} $D=[0,1)$ (Atenção: é necessário que o numerador \emph{e} o
denominador sejam bem definidos.)
\eqref{itdominio41} $D=\bR\setminus \{\pisobredois+k\pi, k\in \bZ\}$
\eqref{itdominio5} $D=$união dos intervalos $[k2\pi,\pi +k2\pi]$, para $k\in \bZ$.
 \eqref{itdominio6} $D=\bR_+$. Observe que apesar da função ser identicamente nula, o seu
domínio não é a reta toda.
\eqref{itdominio7} $D=\{0\}$ (e não $D=\varnothing$!).
\end{Solution}
\begin{Solution}{2.2}
\eqref{itlimitacao1} $x^2$ é limitada inferiormente ($M_-=0$) mas não
superiormente: toma valores arbitrariamente grandes quando
$x$ toma valores grandes. \eqref{itlimitacao2} Não-limitada. De fato, $\tan x=\frac{\sen
x}{\cos x}$, e quando $x$ se aproxima por exemplo de $\pisobredois$, $\sen x$ se aproxima
de $1$ e $\cos x$ de $0$, o que dá uma divisão por zero. (Dê uma olhada no gráfico da
função tangente mais longe no capítulo.) \eqref{itlimitacao3} É limitada:
$\tfrac{1}{x^2+1}\geq 0\equiv M_-$, e como $x^2+1\geq
1$, temos
$\frac{1}{x^2+1}\leq \tfrac11=1\equiv M_+$. \eqref{itlimitacao4} Limitada
inferiormente ($M_-=0$), mas não superiormente: o domínio
dessa função é $(-\infty,1)$, e quando $x<1$ se aproxima de $1$, $\sqrt{1-x}$ se aproxima
de zero, o que implica que $\frac{1}{\sqrt{1-x}}$ toma valores arbitrariamente grandes.
\eqref{itlimitacao5} Observe que o denominador $x^3-x^2+x-1$ se anula em $x=1$. Logo, o
domínio da função é $\bR\setminus \{1\}$. Fatorando (ou fazendo a divisão),
$x^3-x^2+x-1=(x-1)(x^2+1)$. Portanto, quando $x\neq 1$,
$\frac{x-1}{x^3-x^2+x-1}=\frac{x-1}{(x-1)(x^2+1)}=\frac{1}{x^2+1}$. Como $\frac{1}{x^2+1}$
é limitada (item \eqref{itlimitacao3}), $\frac{x-1}{x^3-x^2+x-1}$ é limitada.
\eqref{itlimitacao6} Não-limitada. Apesar de $\sen x$ ser limitado por $-1$ e $+1$,
o ``$x$'' pode tomar valores arbitrariamente grandes.
\end{Solution}
\begin{Solution}{2.3}
\eqref{itgrafunc0} $f(x)=-1$, $D=\bR$
\eqref{itgrafunc1} $f(x)=-\sqrt{81-(x-5)^2}-4$, $D=[-4,14]$.
\eqref{itgrafunc2} $f(x)=\sqrt{25-x^2}$, $D=(-4,4)$
\eqref{itgrafunc3} $f(x)=-\sqrt{25-x^2}$, $D=[0,5]$
\end{Solution}
\begin{Solution}{2.4}
\mbox{}

\begin{bmlimage}\begin{tikzpicture}
\begin{scope}[xshift=-0.5cm]
\draw[thick] (-0.7,1)--(1,1);
\fill (1,1) circle (0.45mm);
\draw [thick, domain=1:1.5] plot (\x,{(\x)^2});
\draw [ ->] (-1.2,0)--(1.2,0);
\draw [ ->] (0,-0.4)--(0,{1.2});
\draw (0,1) node[right]{$1$};
\draw (-0.5,0.5) node{$f(x)$};
\end{scope}

\begin{scope}[xshift=1.8cm, yshift=1cm]
\draw [ ->] (-0.2,0)--(2.2,0);
\draw [ ->] (0,-1)--(0,0.5);
\draw (0,0.4) node[right]{$g(x)$};
\draw (1,0) node[above]{$1$};
\draw[thick] (0.2,-0.8)--(1,0)--(2,-1);
\end{scope}

\begin{scope}[xshift=5.8cm, yshift=0.5cm]
\draw [ ->] (-1.5,0)--(1.5,0);
\draw [ ->] (0,-1)--(0,1);
\draw (0,0.8) node[left]{$h(x)$};
\foreach \k in {-3,...,3}{
\pgfmathsetmacro{\a}{\k/3};
\fill (\a,\a) circle (0.45mm);
\draw[thick] (\a,\a)--({\a+0.333},\a);
\fill[intaberto] ({\a+0.333},\a) circle (0.45mm);
}
\end{scope}

\begin{scope}[xshift=9cm, yshift=0.5cm]
\draw [ ->] (-1.3,0)--(1.3,0);
\draw [ ->] (0,-1)--(0,1);
\draw (0,0.8) node[left]{$i(x)$};
\foreach \k in {-3,...,3}{
\pgfmathsetmacro{\a}{\k/3};
\fill (\a,0) circle (0.45mm);
\draw[thick] (\a,0)--({\a+0.333},0.333);
\fill[intaberto] ({\a+0.333},0.333) circle (0.45mm);
}
\end{scope}

\begin{scope}[xshift=12.5cm, yshift=0.1cm, scale=0.7]
\draw [ ->] (-1.3,0)--(1.3,0);
\draw [ ->] (0,-0.3)--(0,1.3) node[left]{$j(x)$};
%\draw [thick, domain=-1.5:1.5, samples=20] plot (\x,{abs(abs(\x)-1)});
\draw[thick] (-2.2,1.2)--(-1,0)--(0,1)--(1,0)--(2.2,1.2);
\end{scope}

\end{tikzpicture}\end{bmlimage}
\end{Solution}
\begin{Solution}{2.5}
A primeira curva é o gráfico da função $f(x)=-1$ se $x\leq 1$, $f(x)=2-x$ se $x>1$.
 A segunda não é um gráfico, pois os pontos $-\tfrac12 <x\leq 0$ têm duas saídas, o que
não é descrito por uma função (lembra que uma função é um mecanismo que a um entrada $x$
do domínio associa \emph{um (único)} número $f(x)$). No entanto, seria possível
interpretar aquela curva como a união dos gráficos de duas funções distintas: uma função
$f$ com domínio $(-\infty,0]$, e uma outra função $g$ com domínio $(-\tfrac12,\infty)$.
A terceira é o gráfico da função $f(x)=0$ se $x\in \bZ$, $f(x)=1$ caso contrário.
\end{Solution}
\begin{Solution}{2.6}
\eqref{itparidade1} É par: $f(-x)=\tfrac{(-x)}{(-x)^3-(-x)^5}=\tfrac{-x}{-(x^3-x^5)}=f(x)$.
\eqref{itparidade2} É par: $f(-x)=\sqrt{1-(-x)^2}=\sqrt{1-x^2}=f(x)$.
\eqref{itparidade3} É ímpar: $f(-x)=(-x)^2\sen (-x)=x^2(-\sen x)=-f(x)$.
\eqref{itparidade4} É par: $f(-x)=\sen (\cos(-x))=\sen(\cos x)=f(x)$.
\eqref{itparidade41} É ímpar: $f(-x)=\sen (\sen(-x))=\sen(-\sen x)=-\sen(\sen x)=-f(x)$.
\eqref{itparidade5} É par: $f(-x)=(\sen(-x))^2-\cos(-x)=(-\sen x)^2-\cos x=f(x)$.
 \eqref{itparidade6} Não é par nem ímpar, pois $f(\pisobrequatro)=\sqrt{2}$,
$f(-\pisobrequatro)=0$.
\eqref{itparidade7} Como $f(x)\equiv 0$, ela é par \emph{e} ímpar.
\end{Solution}
\begin{Solution}{2.7}
\eqref{itexodecr1} cresce na reta toda.
\eqref{itexodecr2} decrescce (estritamente) em $(-\infty,0]$, cresce (estritamente) em $[0,\infty)$.
\eqref{itexodecr3} decrescce (estritamente) em $(-\infty,0]$, cresce (estritamente) em $[0,\infty)$.
\eqref{itexodecr4}  cresce (estritamente) na reta toda.
\eqref{itexodecr5} decrescce (estritamente) em $(-\infty,0)$, decresce (estritamente) em $(0,\infty)$.
\eqref{itexodecr6} crescce (estritamente) em $(-\infty,0)$, decresce (estritamente) em $(0,\infty)$.
\eqref{itexodecr7} crescce (estritamente) em $(-\infty,\tfrac12]$, decresce (estritamente) em
$[\tfrac12,\infty$. (Será mais fácil resolver esse item depois de saber esboçar o gráfico de $x-x^2$,
veja o Exemplo \ref{exemplo_Funcoes_grafparabdesloc}.)
\eqref{itexodecr8} decrescce (estritamente) em $(-\infty,-1]$ e em $[0,1]$,
cresce (estritamente) em $[-1,0]$ e $[1,\infty)$.
\end{Solution}
\begin{Solution}{2.8}
Se a reta for vertical ($x=a$): $g(x)\pardef f(2a-x)$.
Se a reta for horizontal ($y=b$): $g(x)\pardef 2b-f(x)$.
\end{Solution}
\begin{Solution}{2.10}
\mbox{}
\begin{center}
 \begin{bmlimage}\begin{tikzpicture}[scale=0.5]
\draw [thick, domain=-8:8, samples=200] plot (\x,{1-abs(sin(\x r))});
\draw [ ->] (-9,0)--(9,0) node[right]{$x$};
\draw [ ->] (0,-0.3)--(0,{1.3}) node[left]{$f(x)$};
\pgfmathsetmacro{\Pi}{3.1415};
\draw ({-2*\Pi},0) node[below]{$\scriptstyle{ -2\pi}\,\,\,$};
\draw ({-2*\Pi},0) node{$\shortmid$};
\draw ({2*\Pi},0) node[below]{$\scriptstyle{ 2\pi}$};
\draw ({2*\Pi},0) node{$\shortmid$};
 \end{tikzpicture}\end{bmlimage}
\end{center}
Observe que o período de $f$ é $\pi$. Completando o quadrado\index{completar um quadrado},
$g(x)=-(x-\tfrac12)^2+\tfrac{5}{4}$:
\begin{center}
\begin{bmlimage}\begin{tikzpicture}
\draw [thick, domain=-0.8:1.8] plot (\x,{1.25-(\x-0.5)^2}) ;
\draw[dotted] (0,1.25)--(0.5,1.25)--(0.5,0);
\draw (0.5,1.25) node[above]{$\scriptstyle{(\tfrac12,\tfrac54)}$};
\fill (0.5,1.25) circle (0.45mm);
\draw [ ->] (-1,0)--(1.9,0);
\draw [ ->] (0,-0.4)--(0,1.7) node[left]{$g(x)$};
\end{tikzpicture}\end{bmlimage}
\end{center}
 Observe que a parábola corta o eixo $x$ nos pontos solução da equação $g(x)=0$, que são
$\frac{1\pm \sqrt{5}}{2}$.
 O gráfico da função $h$ já foi esboçado no Exercício \ref{Ex:graficosbasicos}. Mas aqui
vemos que ele pode ser obtido a partir do gráfico de $|x|$ por uma translação de $1$ para
baixo, composta por uma reflexão das partes negativas.
Como $i(x)$ é igual ao dobro de $\sen x$ e $j(x)$ à metade de $\sen x$, temos:
\begin{center}
 \begin{bmlimage}\begin{tikzpicture}[scale=0.5]
 \draw [color=gray!30, domain=-14:14, samples=100] plot (\x,{sin(\x r)})
node[color=gray!30, right]{$\sen x$};
\draw [thick, domain=-14:14, samples=100] plot (\x,{2*sin(\x r)}) node[right]{$i(x)$};
\draw [thick, domain=-14:14, samples=100] plot (\x,{0.5*sin(\x r)}) node[right]{$j(x)$};
\draw [ ->] (-15,0)--(15,0) node[right]{$x$};
\draw [ ->] (0,-1.3)--(0,{1.3});
\pgfmathsetmacro{\Pi}{3.1415};
 \end{tikzpicture}\end{bmlimage}
\end{center}
Completando o quadrado do numerador:
$k(x)=\frac{1-(x-1)^2}{(x-1)^2}=\frac{1}{(x-1)^2}-1$. Portanto, o gráfico pode ser obtido
a partir do gráfico de $\frac{1}{x^2}$:
\begin{center}
\begin{bmlimage}\begin{tikzpicture}[scale=0.7]
\pgfmathsetmacro{\a}{3.5}
\draw [thick, domain=-\a:-0.5, samples=100] plot (\x,{1/((\x)^2)});
\draw [thick, domain=0.5:\a, samples=100] plot (\x,{1/((\x)^2)});
\draw [ ->] (-1,-1)--(-1,2) node[left]{$y$};
\draw [ ->] (-3,1)--(2,1) node[right]{$x$};
\draw[dotted] (-3,0)--(4,0);
\draw[dotted] (0,-1)--(0,3);
\fill (0,0) circle (0.45mm);
\draw (0,0) node[below]{$(1,-1)$};
\end{tikzpicture}\end{bmlimage}
\end{center}
\end{Solution}
\begin{Solution}{2.11}
A trajetória é uma \emph{parábola}.
Resolvendo $y(x)=0$ para $x$, obtemos os pontos onde a parábola toca o chão: $x_1=0$
(ponto de partida), e
$x_2=\frac{2v_{\textsf{v}}v_{\textsf{h}}}{g}$ (distância na qual a partícula vai cair no
chão).
É claro que se o campo de gravitação é mais fraco (na lua por exemplo), $g$ é menor, logo
$x_2$ é maior: o objeto vai mais longe.
Por simetria sabemos que a abcissa do ponto mais alto da trajetória é
$x_*=\frac{x_2}{2}=\frac{v_{\textsf{v}}v_{\textsf{h}}}{g}$, e a sua abcissa é dada por
$y_*=y(x_*)=\tfrac12 \frac{v_{\textsf{v}}^2}{g}$. Observe que $y_*$ \emph{não depende de
$v_{\textsf{h}}$}.
O ponto $(x_*,y_*)$ pode também ser calculado a partir da trajetória $y(x)$, completando
o quadrado.
\end{Solution}
\begin{Solution}{2.12}
 \eqref{itineqgraf1} Se $f(x)=1-|x-1|$, $g(x)=|x|$,
\begin{center}
\begin{bmlimage}\begin{tikzpicture}[scale=0.7]
\draw [ ->] (-1.7,0)--(2,0) node[right]{$x$};
\draw [ ->] (0,-1)--(0,2) node[left]{$y$};
\draw[thick, dashed] (-1.5,1.5)--(0,0)--(1.5,1.5) node[right]{$g$};
\draw[thick] (-1,-1)--(1,1)--(3,-1) node[right]{$f$};
\pgfmathsetmacro{\x}{-1};
\pgfmathsetmacro{\y}{abs(\x-2)};
\draw (1,0) node{$\shortmid$};
\draw (1,0) node[below]{$\scriptstyle{1}$};
\end{tikzpicture}\end{bmlimage}
\end{center}
Logo, $S=[0,1]$. Para \eqref{itineqgraf2}, $S=\varnothing$.
\eqref{itineqgraf3} Se $f(x)=|x^2-1|$ (veja o gráfico do Exemplo
\ref{Ex:modulodografico}), vemos que $S=(-\sqrt{2},0)\cup(0,\sqrt{2})$.
\end{Solution}
\begin{Solution}{2.13}
Tinta: Como a esfera tem superfície igual a $4\pi r^2$, temos $T(r)=40\pi r^2$
(onde $r$ é medido em metros).
Concreto: Como o volume é dado por $V=\tfrac43\pi r^3$, o custo de concreto em
função do raio é $C(r)=40\pi r^3$. Como a superfície $s=4\pi r^2$ temos
$r=\sqrt{s/4\pi}$. Portanto,
$C(s)=40\pi(\tfrac{s}{4\pi})^{3/2}$.
\end{Solution}
\begin{Solution}{2.14}
 Por definição, $d(P,Q)=\sqrt{(a-1)^2+(b+2)^2}$.
Como $2a+b=2$, temos $d(a)=\sqrt{\tfrac54a^2-5a+10}$, e $d(b)=\sqrt{5b^2+5}$.
\end{Solution}
\begin{Solution}{2.15}
Perímetro: $P(n,r)=2nr\sen (\tfrac{\pi}{n})$.
Área: $A(n,r)=\tfrac12 nr^2\sen(\tfrac{2\pi}{n})$.
\end{Solution}
\begin{Solution}{2.16}
Suponha que o cone fique cheio de água, até uma altura de $h$ metros. Isso representa um
volume de
$V(h)=\tfrac13 (\pi h^2)\times h$ metros cúbicos. Logo, $h(V)=(\tfrac{3 V}{\pi})^{1/3}$.
Assim, a marca para $1m^3$ deve ficar na altura $h(1)\simeq 0.98$, para $2$ metros
cúbicos, $h(2)\simeq 1.24$, etc.
\begin{center}
\begin{bmlimage}\begin{tikzpicture}
 \draw (-3,3)--(0,0)--(3,3)--cycle;
 \fill[areagrafico] (-3,3)--(0,0)--(3,3)--cycle;
\draw (0,0)--(0,3);
\foreach \k in {1,...,5}{
\pgfmathsetmacro{\h}{(((3*\k)/3.141)^(0.333333))};
\draw (0,{\h}) node{$-$};
\draw[dotted] ({-\h},\h)--(\h,\h) node[right]{$\scriptscriptstyle{\k m^3}$};
}
\foreach \k in {6,...,28}{
\pgfmathsetmacro{\h}{(3*\k/3.1414)^(0.333333)};
\draw (0,{\h}) node{$-$};
\draw[dotted] ({-\h},{\h})--({\h},{\h});
}
\end{tikzpicture}\end{bmlimage}
\end{center}

\end{Solution}
\begin{Solution}{2.17}
Seja $x$ o tamanho do primeiro pedaço. Como os lados do quadrado medem
$\tfrac{x}{4}$, a área do quadrado é $\tfrac{x^2}{16}$. O círculo tem
circunferência igual a $L-x$, logo o seu raio vale $\tfrac{L-x}{2\pi}$, e a sua
área
$\pi(\tfrac{L-x}{2\pi})^2=\tfrac{(L-x)^2}{4\pi}$. Portanto a área total é dada por
$A(x)=\tfrac{x^2}{4}+\tfrac{(L-x)^2}{4\pi}$, e o seu domínio é $D=[0,L]$.
\end{Solution}
\begin{Solution}{2.18}
Seja $\alpha$ o ângulo entre $AB$ e $AC$.
Área: $A(\alpha)=\sen \tfrac{\alpha}{2}\cos\tfrac{\alpha}{2}=\tfrac12\sen \alpha$, com
$D=(0,\pi)$.
Logo, (olhe para a função $\sen \alpha$), a área é máxima para $\alpha=\tfrac{\pi}{2}$.
\end{Solution}
\begin{Solution}{2.19}
A área pode ser calculada via uma diferença de dois triângulos:
\begin{center}
\begin{bmlimage}\begin{tikzpicture}
\fill[areagrafico] (1,0)--(1,2)--plot[domain=1:2.5](\x,{\x+1})--(2.5,0)--cycle;
\draw[dotted] (1,0)--(1,2);
\draw (1,0) node{$\shortmid$};
\draw (1,0) node[below]{$1$};
\draw[dotted] (2.5,0)--(2.5,3.5);
\draw (2.5,0) node{$\shortmid$};
\draw (2.5,0) node[below]{$t$};
\draw (2.5,-0.1) node[below left]{$\leftarrow$};
\draw (2.5,-0.1) node[below right]{$\rightarrow$};
\draw [ ->] (0,-0.2)--(0,3);
\draw [ ->] (-0.2,0)--(3.5,0);
\draw[thick] (-0.5,0.5)--(3,4) node[right]{$r:\,y=x+1$};
\draw (5,2) node[right]{$A(t)=\tfrac{t^2}{2}+t-\tfrac32$};
\draw (1.7,1.3) node{$R_t$};
\end{tikzpicture}\end{bmlimage}
\end{center}
%\caption{Truc}
%\end{figure}
\end{Solution}
\begin{Solution}{2.21}
 Como $f(g(x))=\frac{1}{(x+1)^2}$, $g(f(x))=\frac{1}{x^2+1}$, temos
$(f\circ g)(0)=1$, $(g\circ f)(0)=1$, $(f\circ g)(1)=\frac14$, $(g\circ f)(1)=\frac12$.
Como $f(g(h(x)))=\frac{1}{(x+2)^2}$ e $h(f(g(x)))=\frac{1}{(x+1)^2}+1$,
 $f(g(h(-1)))=1$,
$h(f(g(3)))=\frac{17}{16}$.
\end{Solution}
\begin{Solution}{2.22}
\eqref{itexcompos1} $\sen (2x)=f(g(x))$, onde $g(x)=2x$, $f(x)=\sen x$.
\eqref{itexcompos2} $\frac{1}{\sen x}=f(g(x))$, onde $g(x)=\sen x$, $f(x)=\frac1x$.
\eqref{itexcompos3} $\sen(\frac{1}{x})=f(g(x))$, onde $f(x)=\sen x$, $g(x)=\frac1x$.
\eqref{itexcompos4} $\sqrt{\frac{1}{\tan (x)}}=f(g(h(x)))$, onde $f(x)=\sqrt{x}$,
$g(x)=\frac{1}{x}$, $h(x)=\tan x$.
\end{Solution}
\begin{Solution}{2.23}
$$
(g\circ f)(x)=
\begin{cases}
 2x+7&\text{ se }x\geq 0\,,\\
x^2&\text{ se }-\sqrt{3}<x<0\,,\\
2x^2+1&\text{ se }x\leq -\sqrt{3}\,.
\end{cases}
\quad\quad
(f\circ g)(x)=
\begin{cases}
 2x+4&\text{ se }x\geq 3\,,\\
x+3&\text{ se }0\leq x<3\,,\\
x^2&\text{ se }x<0\,.
\end{cases}
$$
\end{Solution}
\begin{Solution}{2.24}
\eqref{itconjimagem1} $\imagem(f)=\bR$,
\eqref{itconjimagem2} $\imagem(f)=[-1,3]$,
\eqref{itconjimagem21} Se $p>0$ então $D=\bR$ e $\imagem(f)=\bR$. Se
$p<0$ então $D=\bR\setminus\{0\}$ e $\imagem(f)=\bR\setminus\{0\}$
\eqref{itconjimagem22} $\imagem(f)=[0,\infty)$ se $p>0$, $\imagem(f)=(0,\infty)$ se $p<0$,
\eqref{itconjimagem3} $\imagem(f)=\bR\setminus \{0\}$,
\eqref{itconjimagem4} $\imagem(f)=(0,\infty)$,
\eqref{itconjimagem5a} $\imagem(f)=[1,\infty)$,
\eqref{itconjimagem5} $\imagem(f)=(-\infty,1]$,
\eqref{itconjimagem51} $\imagem(f)=[-1,\infty)$,
\eqref{itconjimagem6} $\imagem(f)=\bR$,
\eqref{itconjimagem7} $\imagem(f)=[-1,1]$,
\eqref{itconjimagem8} $\imagem(f)=(0,1]$,
\eqref{itconjimagem801} $\imagem(f)=[-\tfrac13,\tfrac13]$,
\eqref{itconjimagem81} $\imagem(f)=[-\tfrac{1}{\sqrt{2}},\tfrac{1}{\sqrt{2}}]$,
\eqref{itconjimagem9} $\imagem(f)=(0,1]$. De fato, $0<\frac{1}{1+x^2}\leq 1$. Melhor:
se $y\in (0,1]$ então $y=\frac{1}{1+x^2}$ possui uma única solução, dada por
$x=\sqrt{\frac{1-y}{y}}$.
\eqref{itconjimagem10} $\imagem(f)=(-\infty,-\tfrac12)\cup [1,\infty)$.

Para as funções do Exercício \ref{ExoEsbocosElementares}:
$\imagem(f)=(0,\infty)$,
$\imagem(g)=(-\infty,0]$,
$\imagem(h)=\bZ$,
$\imagem(i)=[0,1)$,
$\imagem(j)=[0,\infty)$.
\end{Solution}
\begin{Solution}{2.25}
Se trata de achar todos os $y\in \bR$ para os quais existe pelo menos um $x\in
\bR$ tal que $f(x)=y$. Isso corresponde a resolver a equação do segundo grau em
$x$: $yx^2-2x+25y=0$. Se $y=0$, então $x=0$. Se $y\neq 0$,
$x=\frac{1\pm\sqrt{1-25y^2}}{y}$, que tem solução se e somente se
$|y|\leq\tfrac15$.
Logo, $\imagem(f)=[-\tfrac15,\tfrac15]$. O ponto $y=0$ é o único que possui uma única preimagem, qualquer outro ponto de $\imagem(f)$ possui duas preimagens. Isso pode ser verificado no gráfico:
\begin{center}
 \begin{bmlimage}\begin{tikzpicture}[scale=0.5]
\draw[->] (-13,0)--(13,0);
\draw[->] (0,-3)--(0,3);
\draw[thick, domain=-12.5:12.5, samples=100] plot (\x,{(60*\x)/(9*(\x)^2+25)});
\draw[dotted] (0,2)node[left]{$\tfrac15$}--(5/3,2);
\draw[dotted] (0,-2)node[right]{$\tfrac15$}--(-5/3,-2);
\draw[dotted,->] (0,1)node[left]{$y$}--(0.5,1)--(0.5,0);
\draw (0,1) node{$-$};
\draw[dotted,->] (0,1)--(0.5,1)--(0.5,0);
\draw[dotted,->] (0,1)--(0.5,1)--(0.5,0);
\draw[dotted,->] (0,1)--(6,1)--(6,0);
 \end{tikzpicture}\end{bmlimage}
\end{center}
\end{Solution}
\begin{Solution}{2.26}
Observe que se $x\in (-1,0)$, então $f(x)\in (0,1)$. Por outro lado, se $y\in (0,1)$,
então existe um único $x\in (-1,0)$ tal que $f(x)=y$: $x=-\sqrt{1-y^2}$.
Logo, $f^{-1}:(0,1)\to(-1,0)$, $f^{-1}(x)=-\sqrt{1-x^2}$.
\begin{center}
\begin{bmlimage}\begin{tikzpicture}
 \draw[ ->] (-1.1,0)--(1.1,0) node[right]{$\scriptstyle{x}$};
 \draw[ ->] (0,-0.1)--(0,1.2) node[right]{$\scriptstyle{f(x)}$};
\draw[thick, <->] (0,1) arc (90:180:1);
\pgfmathsetmacro{\x}{-0.8};
\draw (\x,0) node[below]{$\scriptstyle{f^{-1}(y)}$};
\draw[dotted, <-] (\x,0)--(\x,{sqrt(1-(\x)^2)})--(0,{sqrt(1-(\x)^2)}) node[right]{$\scriptstyle{y}$};

\begin{scope}[xshift=5cm, yshift=1cm]
  \draw[ ->] (-1.1,0)--(1.3,0)node[right]{$\scriptstyle{x}$};
  \draw[ ->] (0,-1.2)--(0,0.4)node[left]{$\scriptstyle{f^{-1}(x)}$};
\draw[thick, <->] (0,-1) arc (270:360:1);
\pgfmathsetmacro{\x}{0.6};
\draw (\x,0) node[above]{$\scriptstyle{x}$};
\draw[dotted, ->] (\x,0)--(\x,{-sqrt(1-(\x)^2)})--(0,{-sqrt(1-(\x)^2)}) node[left]{$\scriptstyle{f^{-1}(x)}$};
\end{scope}
 \end{tikzpicture}\end{bmlimage}
\end{center}
\end{Solution}
\begin{Solution}{2.27}
O gráfico de $\frac{1}{x+1}$ é o de $\tfrac1x$ transladado de uma unidade para a esquerda.
O conjunto imagem é $(0,\infty)$. De fato, para todo $y\in (0,\infty)$, a equação $y=\frac{1}{x+1}$
possui uma solução dada por $x=\frac{1-y}{y}$. Logo, $f^{-1}:(0,\infty)\to (-1,\infty)$,
$f^{-1}(x)=\frac{1-x}{x}$.
\end{Solution}
\begin{Solution}{2.28}
Para verificar que $f^{-1}(-y)=-f^{-1}(y)$, usemos a definição: seja $x$ o único
$x$ tal que $f^{-1}(-y)=x$. Pela definição de função inversa ($(f\circ
f^{-1})(y)=y$), aplicando
$f$ temos $-y=f(x)$. Portanto, $y=-f(x)=f(-x)$ (pela imparidade de $f$).
Aplicando agora $f^{-1}$ obtemos $f^{-1}(y)=-x$, isto é, $x=-f^{-1}(y)$. Isso
mostra que
$f^{-1}(-y)=-f^{-1}(y)$.
\end{Solution}
\begin{Solution}{2.29}
Exemplos:
\eqref{itexobijecoes1} $f(x)=bx$
\eqref{itexobijecoes2} $f(x)=a+(b-a)x$
\eqref{itexobijecoes3} $f(x)=\tan \pisobredois{x}$, ou $f(x)=\tfrac{1}{(x-1)^2}-1$
\eqref{itexobijecoes4} $f(x)=\tan (\tfrac{2}{\pi}(x-\tfrac12))$
\end{Solution}
\begin{Solution}{2.32}
\ref{itexoresolvagraf1} $S=(\frac{1+\sqrt{5}}{2},+\infty)$
\ref{itexoresolvagraf2} $S=[0,1]$
\ref{itexoresolvagraf2} $S=\{-\tfrac52\}$
\end{Solution}
\begin{Solution}{2.33}
Por definição, $\sen y=\tfrac35$. Logo, $\cos y=+\sqrt{1-\sen^2 y}=\tfrac45$ (a raiz
positiva é escolhida, já que $y\in (0,\pisobredois)$). Portanto, $\tan y=\tfrac34$.
\end{Solution}
\begin{Solution}{2.34}
\eqref{itdomintriginv1} $[-1,1]$,
\eqref{itdomintriginv2} $[-\tfrac12,\tfrac12]$,
\eqref{itdomintriginv3} $(-1,1)$,
\eqref{itdomintriginv4} $(-\infty,-\tfrac{1}{\sqrt{2}}]\cup
[\tfrac{1}{\sqrt{2}},+\infty)$.
\end{Solution}
\begin{Solution}{2.35}
Seja $A$ a posição do topo da tela, $B$ a sua base, e $Q$ o ponto onde a parede toca o chão.
Seja $\alpha$ o ângulo $APQ$ e $\beta$ o ângulo $BPQ$.
Temos $\tan \alpha=\tfrac8x$, $\tan \beta=\tfrac3x$. Logo, em a):
$\theta(x)=\arctan\tfrac8x-\arctan\tfrac3x$. Em
b), $\theta(x)=\arctan\tfrac6x-\arctan\tfrac1x$.
\end{Solution}
\begin{Solution}{2.36}
\eqref{itexoinvtrig1} $x=\frac{1}{2}$
\eqref{itexoinvtrig2} $x=\sqrt{3}+1$
\eqref{itexoinvtrig3} $x=\tfrac16$
\eqref{itexoinvtrig4} $x=\tfrac{\sqrt{\pi}}{3}$
\end{Solution}
\begin{Solution}{2.37}
\eqref{itidenttriginv1} $\cos(2\arcos x)=2\cos^2(\arcos x)-1=2x^2-1$
\eqref{itidenttriginv2} $\cos(2\arcsin x)=1-2\sen^2(\arcsen x)=1-2x^2$
\eqref{itidenttriginv3} $\sen(2\arcos x)=2\sen (\arcos x)\cos (\arcos x)=2x\sqrt{1-x^2}$
\eqref{itidenttriginv4} $\cos(2\arctan x)=2\cos^2(\arctan x)-1=\tfrac{1-x^2}{1+x^2}$
\eqref{itidenttriginv5} $\sen (2\arctan x)=\frac{2x}{1+x^2}$
\eqref{itidenttriginv6} $\tan (2\arcsen x)=\frac{2x\sqrt{1-x^2}}{1-2x^2}$
\end{Solution}
\begin{Solution}{2.38}
Chamando $\alpha=\arcsen x$, $\beta=\arcos x$, temos $x=\sen \alpha$, $x=\cos \beta$:
\begin{center}
\begin{bmlimage}\begin{tikzpicture}[scale=2]
\pgfmathsetmacro{\a}{1};
\draw[dotted] (\a,0) arc (0:90:\a);
\draw[ ->, color=gray!70] (0,0) -- (1.1*\a,0);
\draw[ ->, color=gray!70] (0,0) -- (0,1.1*\a);
\pgfmathsetmacro{\alf}{35};
\coordinate (P) at ({0.4*\a*cos(\alf)},{0.4*\a*sin(\alf)});
\draw[<-] (P) arc (\alf:90:{0.4*\a});
\draw[->] ({0.4*\a},0) arc (0:\alf:{0.4*\a});
\draw ({\alf/2}:{0.45*\a}) node{$\alpha$};
\draw ({\alf+(90-\alf)/2}:{0.35*\a}) node[above right]{$\beta$};
\coordinate (B) at ({\a*cos(\alf)},{\a*sin(\alf)});
\coordinate (Bx) at ({\a*cos(\alf)},0);
\coordinate (By) at (0,{\a*sin(\alf)});
\draw (0,0)--(B);
\draw [thick] (B)--(Bx) node[midway, above, sloped]{$x$};
\draw [thick] (By)--(B);
\draw [thick] (By)--(0,0) node[midway, below, sloped]{$x$};
\end{tikzpicture}\end{bmlimage}
\end{center}
\end{Solution}
\protect \section *{Capítulo \ref {CAP:ExponLog}}
\begin{Solution}{3.1}
 Todos os gráficos podem ser obtidos por transformações de
$2^x$,\index{gráfico! transformação de}
$3^x$ e $(\frac32)^x$:
\begin{center}
 \begin{bmlimage}\begin{tikzpicture}[scale=0.7]
\draw[->] (-4.5,0)--(4.3,0) node[right]{$x$};
\draw[->] (0,-3)--(0,5);
\draw[domain=-2.1:4.1]  plot (\x,{1-exp(-\x*ln(2))}) node[right]{$1-2^{-x}$};
\draw[domain=-3.4:2.5]  plot (\x,{exp((\x-1)*ln(3))}) node[above]{$3^{x-1}$};
\draw[domain=-3.7:2.4]  plot (-\x,{exp(\x*ln(1.5)))})
node[above]{$(\tfrac{3}{2})^{-x}$};
\draw[domain=-3:3]  plot (\x,{(-1)*(exp(abs(\x)*ln(1.5)))})
node[right]{$-(\tfrac{3}{2})^{|x|}$};
\draw[dotted] (-3,1)--(4,1);
\draw (0,1) node[above right]{$1$};
 \end{tikzpicture}\end{bmlimage}
\end{center}
\end{Solution}
\begin{Solution}{3.2}
\eqref{itexoresolexp1} $S=\{0,2\}$.
\eqref{itexoresolexp2}  Tomando a raiz: $2^x=\pm 4$, mas como a função
exponencial somente toma valores positivos, $2^x=-4$ não possui soluções. Logo,
$S=\{2\}$.
\eqref{itexoresolexp3} Escrevendo a inequação como $2^{x+1}\leq 2^4$, vemos que
$S=\{x:x+1\leq 4\}=(-\infty,3]$.
\eqref{itexoresolexp6} $S=(-2,\infty)$.
\eqref{itexoresolexp4} $S=(-\infty,0)\cup (1,\infty)$.
\eqref{itexoresolexp5} $S=(\log_{20}\tfrac52,\infty)$.
\end{Solution}
\begin{Solution}{3.3}
Se $z=\log_a(x^y)$, então $z$ satisfaz $a^z=x^y$.
Por~\eqref{eq_ExpLog_debase},
podemos sempre escrever $x$ como $x=a^{\log_a x}$, o que permite
escrever $x^y=(a^{\log_ax})^y=a^{y\log_a x}$. Assim temos $a^z=a^{y\log_a x}$, o que implica
$z=y\log_ax$.
Se $z=\log_a\frac{x}{y}$, então
$$
a^z=\frac{x}{y}=\frac{a^{\log_ax}}{a^{\log_ay}}=a^{\log_ax-\log_ay},
$$
logo $z=\log_ax-\log_ay$.
\end{Solution}
\begin{Solution}{3.4}
$\log_4 16=2$,
$\log_\pi 1=0$,
$\log_2\frac{1}{16}=-4$,
$\log_{\tfrac12}8=-3$,
$7^{2\log_75}=25$.
\end{Solution}
\begin{Solution}{3.5}
Se $N(n)$ é o número de baratas depois de $n$ meses, temos $N(1)=3\cdot 2$,
$N(2)=3\cdot 2\cdot 2$, etc. Logo, $N(n)=3\cdot 2^n$. No fim de julho se
passaram $7$ meses, logo são $N(7)=3\cdot 2^7=384$ baratas. No fim do mês
seguinte são $384\times 2=768$ baratas.
Para saber quando a casa terá mais de um milhão de baratas, é preciso resolver
$N(n)>1000000$, isto é, $3\cdot 2^n>1000000$, que dá
$n>\log_2(1000000/3)=18,34...$,
isto é, no fim do $19$-ésimo mês, o que significa julho de $2012$...
\end{Solution}
\begin{Solution}{3.6}
\eqref{iteqdomlog2} $D=(-2,\infty)$
\eqref{iteqdomlog3} $D=(-\infty,2)$
\eqref{iteqdomlog1} Para $\log_6(1-x^2)$ ser definido, precisa $1-x^2>0$, que dá
$(-1,1)$. Por outro lado,
para evitar uma divisão por zero, precisa $\log_6(1-x^2)\neq 0$, isto é,
$1-x^2\neq 1$, isto é, $x\neq 0$. Logo, $D=(-1,0)\cup(0,1)$.
\eqref{iteqdomlog4} $D=(0,7]$
\eqref{iteqdomlog5} $D=(0,8)$
\eqref{iteqdomlog6} $D=(-\tfrac15,\infty)$
\eqref{iteqdomlog7} $D=\bR_+^*$
\end{Solution}
\begin{Solution}{3.7}
\eqref{itlogeq_1} $S=\{-3\}$,
\eqref{itlogeq_2} $S=\{997\}$,
\eqref{itlogeq_3} $S=\{0,1\}$,
\eqref{itlogeq_4} $S=\{\frac{\log_25}{1+\log_25}\}$,
\eqref{itlogeq_5} $S=\varnothing$,
\eqref{itlogeq_6} $S=\{-\tfrac{13}{8}\}$.
\eqref{itlogeq_7} $S=(-\infty,-1)$,
\eqref{itlogeq_8} $S=(-1,0)\cup(2,3)$,
\end{Solution}
\begin{Solution}{3.8}
As populações respectivas de bactérias depois de $n$ horas são:
$N_A(n)=123456\cdot 3^{\tfrac{n}{24}}$, $N_B(n)=20\cdot 2^n$.
Procuremos o $n_*$ tal que $N_A(n)=N_B(n)$, isto é (o logaritmo pode ser em
qualquer base):
$$n_*=\frac{\log_{10}123456-\log_{10}
20}{\log_{10}2-\tfrac{1}{24}\log_{10}3}=13.48...\,.$$
Isto é, depois de aproximadamente $13$ horas e meia, as duas colônias têm o
mesmo número de indivíduos.
Depois desse instante, as bactérias do tipo $B$ são sempre maiores em número.
De fato (verifique!), $N_A(n)<N_B(n)$ para todo $n>n_*$.
\end{Solution}
\begin{Solution}{3.9}
Se $y\in \bR_+^*$, procuremos uma solução de
$y=\frac{3^x+2}{3^{-x}}$. Essa equação se reduz a $(3^x)^2+2\cdot 3^x-y=0$, isto
é $3^x=-1\pm \sqrt{1+y}$. Como $y>0$, vemos que a solução positiva dá uma única
preimagem $x=\log_3(-1+\sqrt{1+y})\in \bR$. Logo $f$ é uma bijeção e
$f^{-1}:\bR_+^*\to \bR$ é dada por $f^{-1}(y)=\log_3(-1+\sqrt{1+y})$.
\end{Solution}
\begin{Solution}{3.10}
\eqref{itbanco1} Se $r=5\%$, $C_n=C_0\cdot 1,05^n$.
Logo, seu eu puser $1000$ hoje, daqui a $5$ anos terei
$C_5\simeq 1276$, e
para ter $2000$ daqui a $5$ anos, preciso por hoje $C_0\simeq 1814$.
Para por $1$ hoje e ter um milhão, preciso esperar
$n=\log_{1,05}(1000000/1)\simeq 283$ anos.
\eqref{itbanco2} Para ter um lucro de $600$ em $5$ anos, começando de $1000$,
preciso achar o $r$ tal que
$1000+600=1000(1+r/100)^5$. Isto é, $r=100\times
(10^{\frac{\log_{10}1,6}{5}}-1)\simeq 9,8\%$.
\end{Solution}
\begin{Solution}{3.11}
\eqref{itDobrafolha1}
Um pacote de $500$ folhas $A4$ para impressora tem uma espessura de
aproximadamente $5$ centímetros. Logo, uma folha tem uma espessura de
$E_0=5/500=0,01$ centrímetros. Como a espessura dobra a cada dobra, a espessura
depois de $n$ dobras é de $E_n=E_02^n$. Assim, $E_6=0,64$cm, $E_7=1.28$cm
\eqref{itDobrafolha1} a) Para ter $E_n=180$, são necessárias
$n=\log_{2}\frac{180}{0,01}\simeq 14$ dobras.
b) A distância média da terra à lua é de $D=384'403$km. Em centímetros:
$D=3,84403\times 10^{10}$cm. Assim, depois da $41$-ésima dobra, a distância
terra-lua já é ultrapassada.
Observe que depois desse tanto de dobras, o a largura do pacote de papel é
microscópica.
\end{Solution}
\begin{Solution}{3.12}
Se uma fonte é de $120$dB, a potência $P$ que ela produz se acha
isolando $P$ em $120=10\cdot \log_{10}(\tfrac{P}{P_{min}})$, o que dá
$P=10^{-2}W/m^2$.
Como duas fontes produzem o dobro da potência, isto é $2P$, o que representa
\[L=10\cdot
\log_{10}\Bigl(\frac{2P}{P_{min}}\Bigr)=120+\log_{10}2\simeq 120.3\text{dB}\]
\end{Solution}
\begin{Solution}{3.13}
Para ter $N_T=\tfrac{N_0}{2}$, significa que $e^{-\alpha T}=\tfrac12$. Isto é:
$T=\tfrac{\ln 2}{\lambda}$.
Depois de duas meia-vidas, $N_{2T}=N_0e^{-\lambda\tfrac{2 \ln
2}{\lambda}}=\frac{N_0}{4}$ ($>0$: logo, duas meia-vidas não são suficientes
para acabar com a substância!).
Para quatro, $N_{4T}=\frac{N_0}{16}$. Depois de $k$ meia-vidas,
$N_{kT}=\frac{N_0}{2^k}$:
depois de um número qualquer de meia-vidas, sempre sobre alguma coisa...
Para o uranio $235$, a meia-vida vale $T=\frac{\ln 2}{9.9\cdot 10^{-10}}$, isto
é aproximadamente: $700$ milhões de anos.
\end{Solution}
\begin{Solution}{3.14}
\eqref{iteqln0} $S=\{-e^2\}$
\eqref{iteqln1} $S=\{\pm 1\}$ Obs: aqui, se escrever $\ln(x^2)=2\ln x$, perde-se
a solução negativa! Lembre que $\ln (x^y)=y\ln x$ vale se $x$ é positivo! Então
aqui poderia escrever $\ln(x^2)=\ln (|x|^2)=2\ln |x|$.
\eqref{iteqln11} $S=\{e^{-\tfrac15}-1\}$
\eqref{iteqln2} $S=\varnothing$
\eqref{iteqln3} $S=...$
\eqref{iteqln4} $S=(-\infty,\tfrac34)$
\eqref{iteqln5} $S=(-\infty,-\tfrac13)\cup (-\tfrac18,\infty)$
\eqref{iteqln6} $S=(-\infty,-\tfrac23)\cup (\tfrac12,\infty)$
\eqref{iteqln7} $S=\{-5,-2,-1,2\}$
\eqref{iteqln8} $S=(0,e^{-1}]\cup [1,+\infty)$
\end{Solution}
\begin{Solution}{3.15}
\eqref{itparidadelog1} Nem par nem ímpar.
\eqref{itparidadelog2} Nem par nem ímpar (aqui, tem um problema de domínio: o
domínio do $\ln$ é $(0,\infty)$, então nem faz sentido verificar se
$\ln (-x)=\ln (x)$).
\eqref{itparidadelog3} Par: $e^{(-x)^2-(-x)^4}=e^{x^2-x^4}$.
\eqref{itparidadelog4} Par.
\eqref{itparidadelog5} Ímpar.
\eqref{itparidadelog6} Par (cuidado com o domínio: $\bR\setminus\{0\}$)
\eqref{itparidadelog7} Par.
\end{Solution}
\begin{Solution}{3.16}
Sabemos que o gráfico de $\frac{1}{(x-1)^2}$ é obtido transladando o de
$\frac{1}{x^2}$ de uma unidade para direita.
\begin{center}
\begin{bmlimage}\begin{tikzpicture}[scale=0.7]
\draw [ ->] (0,-1)--(0,2) node[left]{$y$};
\draw [ ->] (-3,0)--(4,0) node[right]{$x$};
\pgfmathsetmacro{\a}{3.5}
\pgfmathsetmacro{\e}{0.5}
\draw [thick, domain=-2:1-\e, samples=100] plot (\x,{1/((\x-1)^2)});
\draw [thick, domain=1+\e:4, samples=100] plot (\x,{1/((\x-1)^2)});
%\draw [thick, domain=0.5:\a, samples=100] plot (\x,{\x^{-2}});
%\draw[dotted] (-3,0)--(4,0);
\draw[dotted] (1,-1)--(1,3);
\fill (0,1) circle (0.55mm);
\fill (2,1) circle (0.55mm);
\end{tikzpicture}\end{bmlimage}
\end{center}
Ao tomar o logaritmo de $g(x)$, $f(x)=\ln(g(x))$, é bom ter o gráfico da função
$\ln x$ debaixo dos olhos.
Quando $x$ é grande (positivo ou negativo),
$g(x)$ é próximo de zero, logo $f(x)$ vai tomar valores grandes e negativos.
Quando $x$ cresce, $g(x)$ cresce até atingir o valor $1$ em $x=0$, logo $f(x)$
cresce até atingir o valor $0$ em $0$. Entre $x=0$ e $x=1$ ($x<1$), $g(x)$
diverge, logo $f(x)$ diverge também.
Entre $x=1$ ($x>1$) e $x=2$, $g(x)$ decresce até atingir o valor $1$ em $x=2$,
logo $f(x)$ decresce até atingir o valor $0$ em $x=2$.
Para $x>2$, $g(x)$ continua decrescendo, e toma valores que se aproximam de $0$,
logo $f(x)$ se toma valores negativos, e decresce para tomar valores
arbitrariamente grandes negativos.
\begin{center}
\begin{bmlimage}\begin{tikzpicture}[scale=0.7]
\draw [ ->] (0,-1)--(0,2) node[left]{$y$};
\draw [ ->] (-3,0)--(4,0) node[right]{$x$};
\pgfmathsetmacro{\a}{3.5}
\pgfmathsetmacro{\e}{0.5}
\draw [color=gray!60, domain=-2:1-\e, samples=100] plot (\x,{1/((\x-1)^2)});
\draw [color=gray!60, domain=1+\e:4, samples=100] plot (\x,{1/((\x-1)^2)});
\pgfmathsetmacro{\e}{0.2}
\draw [thick, domain=-1.8:1-\e, samples=100] plot (\x,{ln(1/((\x-1)^2))});
\draw [thick, domain=1+\e:3.8, samples=100] plot (\x,{ln(1/((\x-1)^2))});
\draw[dotted] (1,-1)--(1,3);
\fill (0,0) circle (0.55mm);
\fill (2,0) circle (0.55mm);
\end{tikzpicture}\end{bmlimage}
\end{center}
Observe que é também possível observar que $f(x)=-2\ln|x-1|$, e obter o seu
gráfico a partir do gráfico da função $\ln |x|$!

\end{Solution}
\begin{Solution}{3.17}
Lembramos que $y\in \bR$ pertence ao conjunto imagem de $f$ se e somente
se existe um $x$ (no domínio de $f$) tal que $f(x)=y$.
Ora $\frac{e^{x}}{e^{x}+1}=y$ implica $e^x=\frac{y}{1-y}$. Para ter uma
solução, é necessário ter $\frac{y}{1-y}>0$. É fácil ver que
$\frac{y}{1-y}>0$ se e somente se $y\in (0,1)$. Logo,
$\mathrm{Im}(f)=(0,1)$.
\end{Solution}
\begin{Solution}{3.18}
Por exemplo, $\senh (-x)=\frac{e^{(-x)}-e^{-(-x)}}{2}=\frac{e^{-x}-e^{x}}{2}
=-\frac{e^{x}-e^{-x}}{2}=-\senh (x).$
\end{Solution}
\protect \section *{Capítulo \ref {Cap:Limites}}
\begin{Solution}{4.1}
Em cada caso, fixemos uma tolerância $\epsilon>0$ e procuremos resolver uma
desigualdade elementar.
(1) Observe que $\frac{500}{x}>0$ para todo $x>0$. Seja $\epsilon>0$. Procuremos
quais são os $x>0$ grandes, positivos, para os quais
$0<\frac{500}{x}\leq \epsilon$.
Resolvendo a desigualdade achamos: $x\geq \frac{500}{\epsilon}$.
(2) Seja $\epsilon>0$. Procuremos resolver $0<\frac{9}{x^2}\leq \epsilon$, que dá $x\geq
\frac{3}{\sqrt{\epsilon}}$.
(3) Observe que $\frac{2}{3-x}<0$ quando $x$ for grande, positivo.
Fixemos $\epsilon>0$, e procuremos resolver
$-\epsilon\leq \frac{2}{3-x}<0$, e achamos $x\geq 3+\frac{2}{\epsilon}$.
\end{Solution}
\begin{Solution}{4.3}
%Primeiro, precisamos decidir qual deve ser o valor do limite. Podemos por exemplo
%observar os valores da função para alguns valores de $x$, grandes e
%positivos:
%\begin{center}
%\begin{tabular}{c|c|c|c|c}
%$x=$&10&100&1000&10'000\\
%\hline
%$\frac{2x-1}{3x+5}\simeq $&$0.5428$&$0.6524$&$0.6652$&$0.6665$
%\end{tabular}
%\end{center}
%Esses números parecem indicar que $\frac{2x-1}{3x+5}$ se
%aproxima de $0.6666\dots=\tfrac23$.
%Podemos também argumentar da seguinte maneira: na
%fração $\tfrac{2x-1}{3x+5}$, quando $x$ é grande,
%o numerador $2x-1$ e o denominador $3x+5$ são ambos grandes.
%No entanto, o ``$-1$'' no
%numerador se torna desprezível comparado com $2x$ (que é
%\emph{grande}!), logo $2x-1$ pode ser aproximado por $2x$. No denominador,
%o ``$5$'' é desprezível comparado com o ``$3x$'', logo
%$3x+5$ pode ser aproximado por $3x$. Portanto, para $x$ grande,
%$$
%\frac{2x-1}{3x+5}\quad\text{ pode ser aproximado por }
%\quad\frac{2x}{3x}=\frac23\,.
%$$
%Atenção: esse tipo de raciocínio ajuda a adivinhar qual deve ser o valor do
%limite (no caso
%$\tfrac23$) quando $x\to \infty$, mas não sempre funciona, e é ainda preciso
%\emph{mostrar} que o limite é $\tfrac23$ mesmo.

%Para tornar o argumento rigoroso, basta colocar $x$ em evidência no
%numerador e denominador, e \emph{simplificar por $x$}:
%$$
%\frac{2x-1}{3x+5}=\frac{x(2-\frac{1}{x})}{x(3+\frac{5}{x})}=
%\frac{2-\frac{1}{x}}{3+\frac{5}{x}}\,.
%$$
%Agora vemos que quando $x\to\infty$, o numerador dessa fração,
%$2-\frac{1}{x}$, tende a
%$2$ (pois já sabemos que $\frac{1}{x}$ tende a zero) e que o
%denominador, $3+\frac{5}{x}$, tende a $3$.
%Assim podemos escrever (as operações com limites serão justificadas mais
%tarde)
%$$\lim_{x\to\infty}f(x)=
%\lim_{x\to\infty}\frac{2-\frac{1}{x}}{3+\frac{5}{x}}
%=\frac{\lim_{x\to\infty}(2-\frac{1}{x})}{\lim_{x\to\infty}(3+\frac{5}{
%x})}
%=\frac{2-\lim_{x\to\infty}\frac{1}{x}}{3+5\lim_{x\to\infty}\frac{1}{x}
%}=\frac{2-0}{3+5\cdot 0}=\frac{2}{3}\,.
%$$
%
Vamos mostrar que
\begin{equation}\label{eq_ftendeadoistercos}
\lim_{x\to \infty}\frac{2x-1}{3x+5} =\frac23\,.
\end{equation}
Para isso fixemos uma tolerância $\epsilon>0$ (arbitrariamente pequena),
e verifiquemos que
\[
\Bigl|\frac{2x-1}{3x+5}-\frac23\Bigr|\leq \epsilon
\]
vale sempre que $x$ for tomado suficientemente grande.
Para começar, calculemos o valor absoluto da diferença:
\begin{equation}\label{eq_calculdifff}
\Bigl|\frac{2x-1}{3x+5}-\frac23\Bigr|=
\Bigl|\frac{-13}{3(3x+5)}\Bigr|=\frac{13}{3}
\frac{1}{3x+5}\,.
\end{equation}
Agora resolvemos a desigualdade (para $x$ grande, positivo)
\[ \frac{13}{3} \frac{1}{3x+5}
\leq \epsilon\,,
\]
e achamos a solução:
$x\geq 13\epsilon-15$. Assim, provamos
\eqref{eq_ftendeadoistercos}.
Deixamos o leitor tratar o limite
$x\to-\infty$.
Usando um computador, podemos verificar que de fato, os valores
de $\frac{2x-1}{3x+5}$, longe da origem, se aproximam
de $\tfrac23$:

\begin{center}
\begin{bmlimage}\begin{tikzpicture}
\pgfmathsetmacro{\a}{5.5}
\draw[dashed] (-\a,0.6666)--(\a,0.6666) node[above]{$y=\tfrac{2}{3}$};
\draw [thick, domain=-\a:-2.2, samples=100] plot
(\x,{(2*\x-1)/(3*\x+5)});
\draw [thick, domain=-0.8:\a, samples=100] plot
(\x,{(2*\x-1)/(3*\x+5)});
\draw [ ->] (-\a-0.3,0)--(\a+0.3,0) node[right]{$x$};
\draw [ ->] (0,-1)--(0,{3})
node[right]{$f(x)=\tfrac{2x-1}{3x+5}$};
\pgfmathsetmacro{\x}{4.5};
\draw[dotted] (\x,0) node[below]{$x$}
--(\x,{(2*\x-1)/(3*\x+5)})--(0,{(2*\x-1)/(3*\x+5)})
node[left]{$\scriptstyle{f(x)}$};
\fill (\x,{(2*\x-1)/(3*\x+5)}) circle (0.45mm);
\pgfmathsetmacro{\x}{-4.5};
\draw[dotted] (\x,0) node[below]{$x$}
--(\x,{(2*\x-1)/(3*\x+5)})--(0,{(2*\x-1)/(3*\x+5)})
node[right]{$\scriptstyle{f(x)}$};
\fill (\x,{(2*\x-1)/(3*\x+5)}) circle (0.45mm);
\end{tikzpicture}\end{bmlimage}
\end{center}
\end{Solution}
\begin{Solution}{4.4}
\eqref{itexoformallim2}
Vamos mostrar que o limite é $\tfrac15$.
Calculemos então
$\bigl|\frac{x^2-1}{5x^2}-\tfrac15\bigr|=\frac{1}{5x^2}$.
Seja $\epsilon>0$. Para ter $\tfrac{1}{5x^2}\leq \epsilon$, podemos tomar
$x\geq N$, onde $N=\frac{1}{\sqrt{5\epsilon}}$.
Logo, como isso pode ser feito com qualquer $\epsilon>0$, isso mostra que
$\lim_{x\to \pm\infty}\frac{x^2-1}{5x^2}=\tfrac15$.
\eqref{itexoformallim4}
Como a função é \emph{constante e igual a $1$} nos positivos, temos
$\lim_{x\to\infty}f(x)=1$. Observe aqui que para qualquer tolerância $\epsilon>0$,
podemos sempre tomar o mesmo $N$, por exemplo $N=0$. De fato, para todo $x\geq 0$,
$|f(x)-1|=0\leq \epsilon$, qualquer que seja a tolerância.
Esse exemplo mostra que uma função pode coincidir com a sua assíntota.
\eqref{itexoformallim3}
Como a função é a divisão de $1$ por um número grande, o limite deve ser zero.
De fato, seja $\epsilon>0$. Precisamos mostrar que
\[ \Bigl|\frac{1}{x^3+\sen^2 x}\Bigr|\leq \epsilon
\]
para todo $x$ suficientemente grande. Mas como não dá para resolver essa desigualdade
(isto é: isolar o $x$), podemos começar observando que
$\bigl|\frac{1}{x^3+\sen^2 x}\bigr|\leq	\frac{1}{x^3}$, e procurar
resolver $\frac{1}{x^3}\leq \epsilon$.
Vemos que se $x\geq N\pardef \epsilon^{-1/3}$, então essa desigualdade será
verificada, e $\bigl|\frac{1}{x^3+\sen^2 x}\bigr|\leq \epsilon$.
Isso mostra que $\lim_{x\to \infty}\frac{1}{x^3+\sen^2 x}=0$.
\end{Solution}
\begin{Solution}{4.5}
Seja $\epsilon>0$. Queremos mostrar que $|\frac{1}{f(x)}|\leq \epsilon$ para todo
$x$ suficientemente grande. Como $\lim_{x\to\infty}f(x)=\pm\infty$, sabemos que
se $A=\tfrac1\epsilon$, então existe $N$ tal que $f(x)\geq A$ para todo $x\geq
N$ (em particular, $f(x)>0$ para esses $x$). Mas isso implica também
$\frac{1}{f(x)}\leq \frac{1}{A}=\epsilon$, o que queríamos.
\end{Solution}
\begin{Solution}{4.7}
\eqref{itexliminfini2} Como $\lim_{x\to\pm\infty}\frac{1}{x^q}=0$ para
qualquer $q>0$, usando \eqref{eq:proprliminfty1} dá
$\lim_{x\to
\pm\infty}\{\frac{1}{x}+\frac{1}{x^2}+\frac{1}{x^3}\}=0$.
\eqref{itexliminfini3} $\lim_{x\to \pm\infty}\frac{x^2-1}{x^2}=1$
\eqref{itexliminfini6} $\lim_{x\to\pm\infty}\frac{1-x^2}{x^2-1}=-1$.
\eqref{itexliminfini7} Colocando $x^3$ em evidência e usando
\eqref{eq:proprliminfty3},
$$\lim_{x\to\pm\infty}\frac{2x^3+x^2+1}{x^3+x}=\lim_{x\to
\pm\infty}\frac{x^3(2+\frac{1}{x}+\frac{1}{x^3})}{x^3(1+\frac{1}{x^2})
} =\lim_{x\to
\pm\infty}\frac{2+\frac{1}{x}+\frac{1}{x^3}}{1+\frac{1}{x^2}}=\frac{2}
{1}=2\,.$$
\eqref{itexliminfini8} $\lim_{x\to
\pm\infty}\frac{2x^3-2}{x^4+x}=0$
\eqref{itexliminfini9} Colocando $x^4$ em
evidência no denominador, $x^2$ no numerador,
$\frac{1+x^4}{x^2+4}=x^2\frac{\frac{1}{x^4}+1}{1+\frac{4}{x^2}}$.
Como $x^2\to\infty$ e que a fração tende a $1$, temos
$\lim_{x\to\pm\infty}\frac{1+x^4}{x^2+4}=\infty$.
\eqref{itexliminfini10} ``$\lim_{x\to
-\infty}\frac{\sqrt{x+1}}{\sqrt{x}}$'' não é definido.
Por outro lado, colocando $\sqrt{x}$ em evidência,
$$\lim_{x\to+\infty}\frac{\sqrt{x+1}}{\sqrt{x}}=
\lim_{x\to +\infty}\frac{\sqrt{1+\frac{1}{{x}}}}{1}=1\,.
$$
\eqref{itexliminfini15}
Lembrando que $\sqrt{x^2}=|x|$ (Exercício
\ref{Exo:valorabscorreto}!), temos
$\frac{\sqrt{4x^2+1}}{x}=\frac{\sqrt{x^2(4+\frac{1}{x^2})}}{x}=\frac{
|x|}{x}\sqrt{4+\frac{1}{x^2}}$.
Como $\frac{|x|}{x}=+1$ se $x>0$, $=-1$ se $x<0$, temos $\lim_{x\to\pm\infty}\frac{|x|}{x}=\pm 1$. Como
$\lim_{x\to\pm\infty}\sqrt{4+\frac{1}{x^2}}=\sqrt{4}=2$, temos
$\lim_{x\to\pm\infty}\frac{\sqrt{4x^2+1}}{x}=\pm 2$.
\eqref{itexliminfini11}
Do mesmo jeito,
$\sqrt{x^2+3}=|x|\sqrt{1+\frac{3}{x^2}}$. Assim,
$$
\frac{3x+2}{\sqrt{x^2+3}-4}=\frac{x}{|x|}\frac{3+\frac{2}{x}}{\sqrt{1+
\frac{3}{x^2}}-\frac{4}{|x|}}
$$
Como $\lim_{x\to\pm\infty}\frac{x}{|x|}=\pm 1$, e que a razão tende a
$3$, temos
$$\lim_{x\to+\infty}\frac{3x+2}{\sqrt{x^2+3}-4}=+3\,,\quad
\lim_{x\to-\infty}\frac{3x+2}{\sqrt{x^2+3}-4}=-3\,.
$$
\eqref{itexliminfini12} O limite $x\to-\infty$
não é definido, e $\lim_{x\to+\infty}
\frac{\sqrt{x+\sqrt{x+\sqrt{x}}}}{\sqrt{x+1}}=1$.
\eqref{itexliminfini13} $\lim_{x\to\pm\infty}\frac{|x|}{x^2+1}=0$
\eqref{itexliminfini14} $\lim_{x\to\pm\infty}\sqrt{x^2+1}=+\infty$
\eqref{itexliminfini141} Como $\frac{1}{2^x}=2^{-x}$, temos $\lim_{x\to+\infty}\frac{1}{2^x}=0$,
$\lim_{x\to-\infty}\frac{1}{2^x}=+\infty$.
\eqref{itexliminfini16}
$\lim_{x\to+\infty}\frac{e^x+100}{e^{-x}-1}=-\infty$,
$\lim_{x\to-\infty}\frac{e^x+100}{e^{-x}-1}=0$.
\eqref{itexliminfini161}  Primeiro mostre (usando os mesmos métodos do que os que foram
usados nos outros itens) que $\lim_{x\to \pm \infty}(1+\frac{x+1}{x^2})=1$. Em seguida,
observe que
se $z$ se aproxima de $1$,  então $\ln(z)$ se aproxima de $\ln(1)=0$. Logo, $\lim_{x\to
\pm \infty}\ln(1+\frac{x+1}{x^2})=0$. Obs: dizer que ``se $z$ se aproxima de $1$, então
$\ln(z)$ se aproxima de $\ln(1)$'' presupõe que a função $\ln$ é \emph{contínua} em $1$.
Continuidade será estudada no fim do capítulo.
\eqref{itexliminfini162}  Escreve $(1+e^x)=e^x(1+e^{-x})$, logo
$\frac{\ln(1+e^x)}{x}=\frac{\ln
e^x}{x}+\frac{\ln(1+e^{-x})}{x}=1+\frac{\ln(1+e^{-x})}{x}$. Mas $\lim_{x\to
\infty}\frac{\ln(1+e^{-x})}{x}=0$, logo
$\lim_{x\to \infty}\frac{\ln(1+e^{x})}{x}=1$.
Por outro lado, $\ln(1+e^x)\to 0$  quando $x\to-\infty$, logo $\lim_{x\to
-\infty}\frac{\ln(1+e^{x})}{x}=0$.
\eqref{itexliminfini5} Como $\lim_{x\to\pm \infty}\frac{1}{x}=0$
temos, $\lim_{x\to\pm \infty}e^{\frac{1}{x}}=e^0=1$.
\eqref{itexliminfini17} $``\lim_{x\to \pm\infty}\sen^2x''$ não existe.
\eqref{itexliminfini19} $\lim_{x\to \pm\infty}\arctan
x=\pm\pisobredois$.
\eqref{itexliminfini20}
Por definição, $\senh x=\frac{e^x-e^{-x}}{2}$. Para estudar
$x\to\infty$, coloquemos $e^x$ em evidência:
$\frac{e^x-e^{-x}}{2}=e^x\frac{1-e^{-2x}}{2}$. Como $e^x\to\infty$ e
$1-e^{-2x}\to 1$ temos $\lim_{x\to \infty}\senh x=+\infty$. Como
$\senh x$ é ímpar, temos $\lim_{x\to -\infty}\senh x=-\infty$.
\eqref{itexliminfini21} $\lim_{x\to \pm\infty}\cosh x=+\infty$
\eqref{itexliminfini22} Para estudar, $x\to\infty$:
$\tanh x=\frac{e^x-e^{-x}}{e^x+e^{-x}}=\frac{e^x}{e^x}\frac{1-e^{-2x}}{1+e^{
-2x }}=\frac{1-e^{-2x}}{1+e^{
-2x }}$, logo $\lim_{x\to +\infty}\tanh x=+1$. Como $\tanh$ é ímpar,
$\lim_{x\to -\infty}\tanh x=-1$.
\end{Solution}
\begin{Solution}{4.8}
Pelo gráfico de $x\mapsto \tanh x$, vemos que $V(t)$ cresce e tende a
um valor limite, dado por
$$
V_{\rm lim}=\lim_{t\to\infty}V(t)=\sqrt{\frac{m
g}{k}}\lim_{t\to\infty}\tanh\Bigl(\sqrt{\frac{gk}{m}}t\Bigr)
$$
Vimos no Exercício \ref{Exo:limitesinfini} que
$\lim_{x\to\infty}\tanh x=1$. Portanto,
$$V_{\rm lim}=\sqrt{\frac{m
g}{k}}\,.$$
Observe que $V(t)<V_{\rm lim}$ para todo $t$, então o paraquedista
nunca atinge a velocidade limite, mesmo se ele cair um tempo infinito!
Com os valores propostos,
$V_{\rm lim}=\sqrt{80\cdot 9,81/0.1}\simeq 89m/s\simeq 318km/h$.
\end{Solution}
\begin{Solution}{4.9}
\eqref{itexliminfini1} $\lim_{x\to\infty}(7-x)=-\infty$,
$\lim_{x\to-\infty}(7-x)=+\infty$.
\eqref{itexliminfini4} ``$\lim_{x\to +\infty}\sqrt{1-x}$'' não é
definida, pois o domínio de $\sqrt{1-x}$ é $(-\infty,1]$.
$\lim_{x\to -\infty}\sqrt{1-x}=+\infty$.
\eqref{itexliminfini18} Como $\lim_{x\to\pm\infty}
x=\pm\infty$, e que $\cos x$ é limitado por $-1\leq \cos x\leq 1$,
temos $\lim_{x\to \pm\infty}x+\cos x=\pm\infty$.
\eqref{itexoinfinf1} $-\infty$.
\eqref{itexoinfinf2} $0$.
\eqref{itexoinfinf3} $+\infty$.
\eqref{itexoinfinf311} $-\infty$
\eqref{itexoinfinf31} $+\infty$
\eqref{itexoinfinf5} $\frac12$.
Esse ítem (e o próximo) mostram que argumentos informais do tipo
``$x^2+1\simeq x^2$
quando $x$ é grande'' não sempre são eficazes! De fato, aqui daria
$\sqrt{x^2+1}-\sqrt{x^2-x}\simeq \sqrt{x^2}-\sqrt{x^2}=0$...
\eqref{itexoinfinf51} $\frac32$.
\eqref{itexoinfinf4} Aqui não precisa multiplicar pelo conjugado: pode
simplesmente colocar $\sqrt{x}$ em evidência:
$\sqrt{2x}-\sqrt{x+1}=\sqrt{x}(\sqrt{2}-\sqrt{1+\frac1x})$. Como
$\sqrt{x}\to\infty$ e $\sqrt{2}-\sqrt{1+\frac1x}\to \sqrt{2}-1>0$,
temos $\sqrt{x}(\sqrt{2}-\sqrt{1+\frac1x})\to +\infty$.
\eqref{itexoinfinf6} $-\infty$ (Obs: pode observar que $e^x-e^{2x}=z-z^2$, em que $z=e^x$. Como $z\to \infty$
quando $x\to\infty$, temos $z-z^2\to \infty$, como no item \eqref{itexoinfinf1}.)
\eqref{itexoinfinf7} Como $\ln x-\ln(2x)=-\ln 2$, o limite é $-\ln 2$.
\eqref{itexoinfinf8} $\lim_{x\to \infty}\{\ln x-\ln(x+1)\}=
\lim_{x\to \infty}\ln (\frac{x}{x+1})=\ln 1=0$.
\end{Solution}
\begin{Solution}{4.10}
\eqref{itexosanduiche2} Para todo $x$, $-1\leq
\cos(x^2+3x)\leq +1$, logo
$0\leq \frac{1+\cos(x^2+3x)}{x^2}\leq \frac{2}{x^2}$.
Como $\frac{2}{x^2}$ tende a zero,
$\lim_{x\to\infty}\frac{1+\cos(x^2+3x)}{x^2}=0$.
\eqref{itexosanduiche1} Como $\frac{x+\sen x}{x-\cos
x}=\frac{1+\frac{\sen x}{x}}{1-\frac{\cos x}{x}}$, e como
$\lim_{x\to\infty}\frac{\sen x}{x}=0$, $\lim_{x\to\infty}\frac{\cos
x}{x}=0$ (mesmo método), temos que $\lim_{x\to\infty}\frac{x+\sen
x}{x-\cos x}=1$.
\eqref{itexosanduiche3}
Como $-e^{-x}\leq e^{-x}\sen x\leq e^{-x}$ e
$\lim_{x\to\infty}-e^{-x}=\lim_{x\to\infty}e^{-x}=0$, o limite
procurado vale $0$.
\eqref{itexosanduiche4} Como $0\leq x-\lfloor x\rfloor\leq 1$, temos
$\lim_{x\to\infty}\frac{x-\lfloor x\rfloor}{x}=0$.
\eqref{itexosanduiche5} Como
$-\frac{\pi/2}{\ln x}\leq \frac{\arctan(\sen x)}{\ln x}\leq
\frac{\pi/2}{\ln x}$, e $\lim_{x\to\infty}\frac{1}{\ln x}=0$, o
limite procurado é $0$.
\eqref{itexosanduiche6} Para todo $x$,
$-\frac{1}{x^2+4}\leq \frac{\sen x}{x^2+4}\leq
\frac{1}{x^2+4}$. Como
$\lim_{x\to \infty}\frac{1}{x^2+4}=0$, o limite procurado vale
$1$.
\end{Solution}
\begin{Solution}{4.11}
 A divisão dá $\frac{x^4-1}{x-1}=x^3+x^2+x+1$. Logo, como cada termo tende a $1$,
$\lim_{x\to 1^{\pm}}\frac{x^4-1}{x-1}=4$.
 No caso geral, $\frac{x^n-1}{x-1}=x^{n-1}+\dots+x+1$. Como são $n$ termos e que cada um
tende a $1$, temos $\lim_{x\to 1^{\pm}}\frac{x^n-1}{x-1}=n$.
\end{Solution}
\begin{Solution}{4.12}
$\lim_{x\to 0^+}f(x)=\lim_{x\to 0^+}\frac{x}{2}=0$,
$\lim_{x\to 0^-}f(x)=\lim_{x\to 0^-}\frac{x}{2}=0$.
$\lim_{x\to 2^+}f(x)=\lim_{x\to 2^+}5-x=3$.
$\lim_{x\to 2^-}f(x)=\lim_{x\to 2^-}\frac{x}{2}=1$, logo $f$ é descontínua em
$x=2$.
$\lim_{x\to 5^+}f(x)=\lim_{x\to 5^+}5-x=0$,
$\lim_{x\to 5^-}f(x)=\lim_{x\to 5^-}5-x=0$.
\end{Solution}
\begin{Solution}{4.13}
Escolha um ponto $a\in \bR$ qualquer.
Como os racionais diádicos \index{racionais diádicos}
são densos em $\bR$, existem infinitos diádicos $x_D>a$,
 arbitrariamente próximos de $a$, tais que $f(x_D)=1$. Mas existem também infinitos
irracionais $x_I>a$ arbitrariamente próximos de $a$ tais que $f(x_I)=0$. Portanto, $f(x)$
não pode tender a um valor quando $x\to a^+$. O mesmo raciocínio vale para $x\to a^-$.
Logo, a função $f$ não possui limites laterais em nenhum ponto da reta.
\end{Solution}
\begin{Solution}{4.14}
$\lim_{x\to \half^+}f(x)=\lim_{x\to \half^-}f(x)=0$,
$\lim_{x\to \frac{1}{3}^+}f(x)=\lim_{x\to \frac{1}{3}^-}f(x)=0$.
$\lim_{x\to 1^+}f(x)=1$, $\lim_{x\to 1^-}f(x)=0$. Para
$n\in \bZ$, $\lim_{x\to n^+}f(x)=n$, $\lim_{x\to n^-}f(x)=n-1$.
(Pode verificar essas afirmações também no seu esboço do
Exercício \ref{ExoEsbocosElementares}!)
\end{Solution}
\begin{Solution}{4.15}
\eqref{itlimbasic1} $0$
\eqref{itlimbasic2} $0$ (O limite é bem definido, no seguinte
sentido: como $\sqrt{x}$ é definida para $x>0$, o limite
somente pode ser do tipo $x\to 0^+$.)
\eqref{itlimbasic3} $1$
\eqref{itlimbasic4} $\frac45$
\eqref{itlimbasic5} $1$
\eqref{itlimbasic6} Sabemos que $\frac{|x-4|}{x-4}=+1$ se $x>4$, e
$=-1$ se $x<4$. Logo, $\lim_{x\to 4^+}\frac{|x-4|}{x-4}=+1$,
$\lim_{x\to 4^-}\frac{|x-4|}{x-4}=-1$, mas $\lim_{x\to
4}\frac{|x-4|}{x-4}$ não existe.
\eqref{itlimbasic61} $-1$
\eqref{itlimbasic7} $-\frac12$
\eqref{itexolimelem20} Como $\ln x$ muda de sinal em $1$, é preciso
que $x$ tenda a $1$ pela direita para $\sqrt{\ln x}$ ser bem definida,
e escrever esse limite como $\lim_{x\to 1^+}\sqrt{\ln x}=0$.
$\lim_{x\to 1^-}\sqrt{\ln x}$ não é definido.
\eqref{itexolimelem201} Não definido pois $\sqrt{x-2}$ não é definido perto de $x=-2$.
\end{Solution}
\begin{Solution}{4.16}
No primeiro caso, podemos comparar $0\leq f(x)\leq x^2$ para todo $x$.
Logo,
pelo Teorema \ref{Teo:Sanduichefinito},
$\lim_{x\to 0}f(x)$ existe e vale $0$.
No segundo caso,
$\lim_{x\to 0^-}g(x)=\lim_{x\to 0^-}\frac{1+x}{1+x^2}=1$, e
$\lim_{x\to 0^+}g(x)=\lim_{x\to 0^+}\sen(\frac{\pi}{2}+x)=\sen
\pisobredois=1$. Logo, $\lim_{x\to 0}g(x)$ existe e vale $1$.
\end{Solution}
\begin{Solution}{4.17}
\eqref{itlimzerozero1} $-4$.
\eqref{itlimzerozero2} $6$.
\eqref{itlimzerozero24} $-\tfrac12$.
\eqref{itlimzerozero4} $\frac{b}{2a}$.
\eqref{itlimzerozero44} $0$.
\eqref{itlimzerozero3} $\tfrac12$.
\end{Solution}
\begin{Solution}{4.18}
Observe que quando $x\to -2$, o denominador tende a $0$.
Para o limite existir, a única possibilidade é do numerador também
tender a zero quando $x\to -2$. Mas como $3x^2+ax+a+3$ tende a $15-a$
quando $x\to -2$, $a$ precisa satisfazer $15-a=0$, isto é: $a=15$.
Neste caso (e somente neste caso), o limite existe e vale
$$
\lim_{x\to -2}\frac{3x^2+15x+18}{x^2+x-2}
\lim_{x\to -2}\frac{(3x+9)(x+2)}{(x-1)(x+2)}=
\lim_{x\to -2}\frac{3x+9}{x-1}=-1\,.
$$
\end{Solution}
\begin{Solution}{4.19}
 \eqref{itexosinxx1}
Como $\frac{\tan x}{x}=\frac{\sen x}{x}\frac{1}{\cos x}$,
temos $\lim_{x\to 0}\tfrac{\tan x}{x}=1$.
\eqref{itexosinxx2}
Como $\frac{\sen x}{\tan x}=\cos x$, temos $\lim_{x\to 0}\tfrac{\sen
x}{\tan x}=1$.
\eqref{itexosinxx3} Como ${\sen 2x}\to 0$ e ${\cos x}\to 1$, temos $\lim_{x\to 0}\frac{\sen 2x}{\cos x}=\frac{0}{1}=0$ (não é um limite do tipo ``$\frac00$'').
\eqref{itexosinxx4}
Como $\frac{\sen 2x}{x\cos x}=2\frac{\sen x}{x}$,
temos $\lim_{x\to 0}\frac{\sen 2x}{x\cos x}=2$.
\eqref{itexosinxx5} Como
$$\frac{1-\cos x}{x^2}=\frac{1-\cos x}{x^2}\frac{1+\cos
x}{1+\cos x}=\frac{1-\cos^2x}{x^2}\cdot \frac{1}{1+\cos x}=\Bigl(\frac{\sen x}{x}\Bigr)^2\cdot \frac{1}{1+\cos x}\,,$$
temos
$\lim_{x\to 0}\tfrac{1-\cos x}{x^2}=(1)^2\cdot \frac12=\frac12$.
\eqref{itexosinxx6} $+\infty$
\eqref{itexosinxx7}  $\lim_{x\to 0^+}\tfrac{\sen (x^2)}{x}=\lim_{x\to
0^+}x\cdot\tfrac{\sen(x^2)}{x^2}=0\cdot 1=0$.
\end{Solution}
\begin{Solution}{4.20}
``$\lim_{x\to a^+}f(x)=+\infty$'' significa que $f(x)$ ultrapassa
qualquer valor dado (arbitrariamente grande), desde que $x>a$ esteja
suficientemente perto de $a$. Isto é: para todo $M>0$ (arbitrariamente
grande), existe um $\delta>0$ tal que se $a<x\leq a+\delta$, então
$f(x)\geq M$.
Por outro lado, $\lim_{x\to a^+}f(x)=-\infty$ significa que
para todo $M>0$ (arbitrariamente grande),
existe um $\delta>0$ tal que se $a<x\leq a+\delta$, então $f(x)\leq
-M$.
\end{Solution}
\begin{Solution}{4.21}
\eqref{itlimbasic9a} $5$
\eqref{itlimbasic9b} $1$
\eqref{itlimbasic9c} $\mp \infty$
\eqref{itlimbasic10} Observe que enquanto $x^2-4>0$, $\frac{x-2}{(
\sqrt{x^2-4})^2}=\frac{1}{x+2}$. Logo, $\lim_{x\to
2^+}\frac{x-2}{(\sqrt{x^2-4})^2}=\frac14$, e
\eqref{itlimbasic12} $\lim_{x\to -2^-}\frac{x-2}{(\sqrt{x^2-4})^2}=
-\infty$
\eqref{itlimbasic14} $-\infty$
\eqref{itlimbasic14b} Não é definido.
\eqref{itlimbasic15a}   $\lim_{t\to 0^+}\frac{1}{\sen t}=+\infty$,
$\lim_{t\to 0^-}\frac{1}{\sen t}=-\infty$
\eqref{itlimbasic15b}   $\lim_{t\to 0^\pm}\frac{t}{\sen t}=\lim_{t\to
0^\pm}\frac{1}{\frac{\sen t}{t}}=1$.
\eqref{itlimbasic15} Não existe, porqué quando $t\to 0^+$, $\sen
\frac1t$
oscila entre $+1$ e $-1$, enquanto $\frac1t$ tende a $+\infty$:
\begin{center}
\begin{bmlimage}\begin{tikzpicture}[scale=0.3]
\draw[->] (0,0)--(3,0) node[right]{$t$};
\draw[->] (0,-3)--(0,3) node[left]{$\frac{\sen \frac1t}{t}$};
\pgfmathsetmacro{\e}{0.2};
\draw[thick, domain=\e:{3}, samples=1000] plot (\x,{(sin(5/\x r))/\x});
\end{tikzpicture}\end{bmlimage}
\end{center}
\eqref{itlimbasic16} $\li{z}{0^+}9^{\frac{1}{z}}=+\infty$,
$\li{z}{0^-}9^{\frac{1}{z}}=0$.
\eqref{itlimbasic18b} $+\infty$
\eqref{itlimbasic19b} $-\infty$
\eqref{itlimbasic13} $1$ (veremos mais tarde como calcular esse
limite...)
\end{Solution}
\begin{Solution}{4.22}
A função $v\mapsto m_v$ tem domínio $[0,c)$, e a reta $v=c$ é
assíntota vertical:
\begin{center}
\begin{bmlimage}\begin{tikzpicture}
\draw[->] (0,0)--(4,0) node[right]{$v$};
\draw[->] (0,0)--(0,3) node[left]{$m_v$};
\pgfmathsetmacro{\c}{3.2};
\pgfmathsetmacro{\m}{1};
\fill (0,\m) circle (0.45mm);
\draw (0,\m) node[left]{$m_0$};
\draw[thick, domain=0:{\c-0.2}] plot (\x,{\m/(sqrt(1-(\x/\c)^2))});
\draw[dotted] (\c,0) node[below]{$c$}--(\c,3);
\draw (3.8,2.5) node[right]{$\displaystyle{\lim_{v\to c^-}m_v}=
+\infty$};
\end{tikzpicture}\end{bmlimage}
\end{center}
\end{Solution}
\begin{Solution}{4.23}
Observe que $\lim_{x\to \pm\infty}f(x)=+1$, logo $y=1$ é assíntota
horizontal.
Por outro lado, $\lim_{x\to 1^+}f(x)=+\infty$ e $\lim_{x\to 1^-}f(x)
=-\infty$. Portanto, $x=1$ é assíntota vertical.
Temos então: 1) o gráfico se aproxima da sua assintota horizontal em
$-\infty$, e ele tende a $-\infty$ quando $x\to 1^-$,
2) o gráfico se aproxima da sua assintota horizontal em $+\infty$, e
ele tende a $+\infty$ quando $x\to 1^+$.
Somente com essas informações, um esboço razoável pode ser montado:
\begin{center}
\begin{bmlimage}\begin{tikzpicture}[scale=0.5]
\draw[->] (-3,0)--(3,0) node[right]{$x$};
\draw[->] (0,-2)--(0,3) node[left]{$y$};
\pgfmathsetmacro{\e}{0.6};
%\pgfmathsetmacro{\m}{1};
%\fill (0,\m) circle (0.45mm);
%\draw (0,\m) node[left]{$m_0$};
\draw[thick, domain=-4:1-\e] plot (\x,{(\x+1)/(\x-1)});
\draw[thick, domain=1+\e:4] plot (\x,{(\x+1)/(\x-1)});
\draw[dotted] (1,-2) node[right]{$\scriptstyle{x=1}$}--(1,3);
\draw[dotted] (-4,1) node[above]{$\scriptstyle{y=1}$}--(4,1);
%\draw (3.8,2.5) node[right]{$\displaystyle{\lim_{v\to c^-}m_v}=+\infty$};
\end{tikzpicture}\end{bmlimage}
\end{center}
Observe que pode também escrever $\frac{x+1}{x-1}=\frac{2}{x-1}+1$,
logo o gráfico pode ser obtido a partir de transformações elementares
do gráfico de $\frac1x$...
\end{Solution}
\begin{Solution}{4.24}
\eqref{itexassint1} $D=\bR$, sem assíntotas.
\eqref{itexassint2} $D=\bR\setminus\{-1\}$. Horiz: $y=0$, Vertic: $x=-1$.
\eqref{itexassint3} $D=\bR\setminus\{3\}$. sem assíntotas.
\eqref{itexassint4} $D=\bR\setminus\{0\}$. Horiz: $y=2$, Vertic: $x=0$.
\eqref{itexassint5} $D=\bR\setminus\{-3\}$. Horiz: $y=-1$, Vertic: $x=-3$.
\eqref{itexassint6} $D=\bR\setminus\{0\}$. Horiz: $y=1$, Vertic: não tem.
\eqref{itexassint6b} $D=(-\infty,2)$. Horiz: não tem, Vertic: $x=2$.
\eqref{itexassint7} $D=\bR\setminus\{0\}$. Horiz: não tem, Vertic: $x=0$.
\eqref{itexassint8} $D=\bR\setminus\{0\}$. Horiz: $y=0$, Vertic: não tem.
\eqref{itexassint9} $D=\bR\setminus\{0\}$. Horiz: $y=0$, Vertic: $x=0$.
\eqref{itexassint10} $D=\bR$. Horiz: $y=1$, Vertic: não tem.
\eqref{itexassint11} Para garantir $1-x^2>0$, $D=(-1,1)$ Horiz: não
tem (já que o domínio é $(-1,1)$...), Vertic: $x=-1$ (porqué
$\lim_{x\to -1^+}\ln (1-x^2)=-\infty$), $x=+1$ (porqué $\lim_{x\to
+1^-}\ln (1-x^2)=-\infty$).
\eqref{itexassint12} $D=(-1,1)$. Horiz: não tem, Vertic: $x=-1$, $x=+1$.
\eqref{itexassint13} $D=\bR\setminus\{\pm 1, 3\}$. Horiz: $y=0$,
Vertic: $x=+1$, $x=-1$.
\eqref{itexassint14} $D=(-1,+1)\setminus\{ 0\}$. Horiz: não tem,
Vertic: $x=0$.
\eqref{itexassint17} $D=\bR\setminus\{0\}$. Horiz: $y=+1$, $y=-1$,
Vertic: $x=0$.
\eqref{itexassint15b} $D=(-1,1)$. Horiz: não tem, Vertic: $x=-1$, $x=+1$.
\eqref{itexassint18} $D=\bR\setminus\{0\}$. Horiz: $y=1$ (a direita), $y=0$ (a esquerda),
Vertic: $x=0$.
\end{Solution}
\begin{Solution}{4.26}
Por exemplo: $f(x)=\frac{1-x^2}{(x+1)(x-3)}$, ou $f(x)=\frac{1}{x+1}
+\frac{1}{x-3}-\frac{x^2}{x^2+1}$.
\end{Solution}
\begin{Solution}{4.27}
\eqref{itmudvarlim1} Com $z\pardef x-1$, $\lim_{x\to 1}\frac{\sen
(x-1)}{3x-3}=\lim_{z\to 0}\frac{\sen
z}{3z}=\frac13$.
\eqref{itmudvarlim11} $\frac35$ (Escreve $\frac{\sen (3x)}{\sen (5x)}=\frac{\sen (3x)}{3x}\frac{1}{\frac{\sen (5x)}{5x}}\frac{3x}{5x}$.)
\eqref{itmudvarlim2} Com $z\pardef x+1$, $\lim_{x\to
-1}\frac{\sen(x+1)}{1-x^2}=\lim_{z\to
0}\frac{\sen z}{z}\frac{1}{2-z}=\frac12$.
\eqref{itmudvarlim3} Com $h\pardef x-a$,
$\lim_{x\to a}\frac{x^n-a^n}{x-a}=\lim_{h\to
0}\frac{(a+h)^n-a^n}{h}=na^{n-1}$.
\eqref{itmudvarlim31} Chamando $t\pardef \sqrt{x}$,
$$\lim_{x\to 4}\frac{x-4}{x-\sqrt{x}-2}=\lim_{t\to 2}\frac{t^2-4}{t^2-t-2}=\lim_{t\to 2}\frac{(t-2)(t+2)}{(t-2)(t+1)}=\lim_{t\to 2}\frac{(t+2)}{(t+1)}=\tfrac43\,.$$
\eqref{itmudvarlim4} Com $z\pardef \frac{1}{x}$, temos (lembre o
item \eqref{itexliminfini22} do Exercício \ref{Exo:limitesinfini})
 $\lim_{x\to 0^+}\tanh \frac{1}{x}=\lim_{z\to +\infty}\tanh z=+1$,
$\lim_{x\to 0^-}\tanh \frac{1}{x}=\lim_{z\to -\infty}\tanh z=-1$.
\eqref{itmudvarlim5} Com a mesma mudança,
$\lim_{x\to 0^\pm}x\tanh \frac{1}{x}=\lim_{z\to \pm
\infty}\frac{1}{z}\tanh{z}=0\cdot (\pm 1)=0$.
\end{Solution}
\begin{Solution}{4.28}
Pela fórmula \eqref{eq:mudancabaselog} de mudança de base para o
logaritmo, $\log_a(1+h)=\frac{\ln(1+h)}{\ln a}$. Logo, por
\eqref{eq:derivlnenun},
$$\lim_{h\to 0}\frac{\log_a(1+h)}{h}=\frac{1}{\ln a}\lim_{h\to
0}\frac{\ln(1+h)}{h}=\frac{1}{\ln a}\,.$$
Por outro lado, chamando $z\pardef a^x$, $x\to 0$ implica $z\to 1$.
Mas $x=\log_az$, logo
$$
\lim_{x\to
0}\frac{a^x-1}{x}=\lim_{z\to 1}\frac{z-1}{\log_a
z}=\frac{1}{\lim_{z\to 1}\frac{\log_az}{z-1}}\,.
$$
Definindo $h\pardef z-1$ obtemos $\lim_{z\to
1}\frac{\log_az}{z-1}=\lim_{h\to 0}\frac{\log_a(1+h)}{h}=\frac{1}{\ln
a}$, o que prova a identidade desejada.
\end{Solution}
\begin{Solution}{4.29}
\eqref{it_exsuplAldo_1} $\infty$
\eqref{it_exsuplAldo_2} $\infty$
\eqref{it_exsuplAldo_3} $0$
\eqref{it_exsuplAldo_4} $\infty$
\eqref{it_exsuplAldo_5} $0$
\eqref{it_exsuplAldo_6} $1$
\end{Solution}
\begin{Solution}{4.30}
$\lim_{x\to 0^-}f(x)=\lim_{x\to 0^-}(2x+2)=2$,
$\lim_{x\to 0^+}f(x)=\lim_{x\to 0^+}(x^2-2)=-2$,
Já que esses dois limites laterais são diferentes, $\lim_{x\to 0}f(x)$ não
existe.
$\lim_{x\to 2^-}f(x)=\lim_{x\to 2^-}(x^2-2)=2$.
$\lim_{x\to 2^+}f(x)=\lim_{x\to 2^+}2=2$. Como
$\lim_{x\to 2^-}f(x)=\lim_{x\to 2^+}f(x)$, $\lim_{x\to 2}f(x)$ existe e vale
$2$.
\begin{center}
\begin{bmlimage}\begin{tikzpicture}[scale=0.7]
\draw[->] (-1.5,0)--(3,0)node[right]{$x$};
\draw[->] (0,-2)--(0,2.4);
\draw[->, thick] (-2,-2)--(0,2);
\fill (0,-2) circle (0.45mm);
\draw[thick, domain=0:2] plot (\x,{\x^2-2});
\fill (2,2) circle (0.45mm);
\draw[thick] (2,2)--(3,2);
\end{tikzpicture}\end{bmlimage}
\end{center}
\end{Solution}
\begin{Solution}{4.31}
O ponto $Q$ é da forma $Q=(\lambda,\lambda^2)$, e $Q\to O$
corresponde a $\lambda\to 0$.
Temos $M=(\frac{\lambda}{2},\frac{\lambda^2}{2})$.
É fácil ver que a equação da reta $r$ é
$y=-\frac{1}{\lambda}x+\frac{\lambda^2}{2}+\frac12$. Logo,
$R=(0,\frac{\lambda^2}{2}+\frac12)$. Quando $Q$ se aproxima da origem,
isto é, quando $\lambda$ se aproxima de $0$, $\lambda^2$ decresce,
o que significa que $R$ \emph{desce}. Quando $\lambda\to 0$, $R\to
(0,\frac12)$. (Pode parecer contra-intuitivo, já que o segmento $OQ$ tende a
ficar sempre mais horizontal, logo o segmento $MR$ fica mais vertical, à medida
que $Q\to O$.)
\end{Solution}
\begin{Solution}{4.32}
\mbox{}
\begin{center}
\begin{bmlimage}\begin{tikzpicture}
\pgfmathsetmacro{\r}{2};
\pgfmathsetmacro{\n}{10};
\pgfmathsetmacro{\incrang}{2*3.14152/\n};
\draw (0,0) circle (\r);
\foreach \k in {0,...,\n} {
\coordinate (Pk) at ({\r*cos(\k*\incrang r)},{\r*sin(\k*\incrang r)});
\coordinate (Pkm) at ({\r*cos((\k-1)*\incrang r)},{\r*sin((\k-1)*\incrang
r)});
\fill[color=gray!10] (0,0)--(Pk)--(Pkm)--cycle;
\draw (0,0)--(Pk)--(Pkm);
}
\coordinate (Pk) at ({\r*cos(\incrang r)},{\r*sin(\incrang r)});
\coordinate (Pkm) at ({\r*cos(0 r)},{\r*sin(0 r)});
\coordinate (M) at ($(Pk)!(0,0)!(Pkm)$);
\fill[color=gray!30] (0,0)--(Pk)--(Pkm)--cycle;
\draw[thick] (0,0)--(Pk)--(Pkm)--cycle;
\draw[thick] (0,0)--(M);
%\draw (M) node[above right]{$M$};
\end{tikzpicture}\end{bmlimage}
\end{center}
Como um setor tem abertura $\alpha_n=\frac{2\pi}{n}$,
a área de cada triângulo se calcula facilmente:
$$2\times \frac12
\times(r\cos\tfrac{\alpha_n}{2})\times
(r\sen\tfrac{\alpha_n}{2})=\frac{r^2}{2}\sen
{\alpha_n}=\frac{r^2}{2}\sen \tfrac{2\pi}{n}\,.$$
Logo, a área do polígono é dada por $A_n=n\times \frac{r^2}{2}\sen
\frac{2\pi}{n}$. No limite $n\to \infty$ obtemos
$$
\lim_{n\to \infty}A_n=r^2\lim_{n\to \infty}\frac{n}{2}\sen \frac{2\pi}{n}
=\pi r^2\lim_{n\to \infty}\frac{1}{\frac{2\pi}{n}}\sen \tfrac{2\pi}{n}
=\pi r^2\lim_{t\to 0^+}\frac{\sen t}{t}=\pi r^2\,.
$$
\end{Solution}
\begin{Solution}{4.33}
\eqref{itrevisaolimites1} $32$
\eqref{itrevisaolimites11} $\frac13$
\eqref{itrevisaolimites2} $2$
\eqref{itrevisaolimites3} $0$
\eqref{itrevisaolimites31} $1$
\eqref{itrevisaolimites4} $-1$
\eqref{itrevisaolimites5} Com a mudança $y=x+1$, $\frac12$
\eqref{itrevisaolimites6} $0$
\eqref{itrevisaolimites7} $-\infty$
\eqref{itrevisaolimites8} $0$
\eqref{itrevisaolimites9} $0$
\eqref{itrevisaolimites10} $\tfrac12$ (Pois é, esse limite é
um pouco mais difícil. Calcularemos ele no
Capítulo~\ref{Cap:Derivacao} usando a regra de
Bernoulli-l'Hôpital.)
\eqref{itrevisaolimites12} $\mp \pisobredois$
\eqref{itrevisaolimites13} Como $\sen$ é contínua em $\pisobredois$,
$\li{x}{+\infty}\sen(\frac{\pi}{2}+\frac{1}{1+x^2})=\sen(\frac{
\pi}{2}+\li{x}{+\infty}\frac{1}{1+x^2})=\sen \pisobredois =1$.
\eqref{itrevisaolimites14} $0$
\eqref{itrevisaolimites15} $-\frac{1}{10}$
\eqref{itrevisaolimites16} $\frac{\sqrt{3}}{2}$
\eqref{itrevisaolimites161} $\frac{2}{3}$
\eqref{itrevisaolimites17} $0$
\eqref{itrevisaolimites20} $1$
\eqref{itrevisaolimites21} $1$
\end{Solution}
\begin{Solution}{4.34}
Seja $\epsilon>0$ e $N$ grande o suficiente, tal que $|g(x)-\ell|\leq \epsilon$ e $|h(x)-\ell|\leq \epsilon$ para todo $x\geq N$.
Para esses $x$, podemos escrever $f(x)-\ell \leq h(x)-\ell\leq |h(x)-\ell|\leq \epsilon$, e $f(x)-\ell\geq g(x)-\ell\geq -|g(x)-\ell|\geq -\epsilon$. Logo, $|f(x)-\ell|\leq \epsilon$.
\end{Solution}
\begin{Solution}{4.35}
\eqref{itlimdificeis0} Como $\sqrt{1-\cos^2x}=\sqrt{\sen^2x}=|\sen x|$ e
$x\mapsto |x|$ é contínua,
$$\lim_{x\to 0}\frac{\sqrt{1-\cos x}}{|x|}=\lim_{x\to
0}\frac{1}{\sqrt{1+\cos x}}\frac{|\sen x|}{|x|}
=\Bigl(\lim_{x\to 0}\frac{1}{\sqrt{1+\cos x}}\Bigr)
\cdot \Bigl|
\lim_{x\to 0}\frac{\sen x}{x}\Bigr|=\frac{1}{\sqrt{2}}\,.
$$
\eqref{itlimdificeis1} Como $\sen (a+h)=\sen a\cos h+\sen h\cos a$, temos
$$
\lim_{h\to 0}\frac{\sen (a+h)-\sen a}{h}=\sen a \Bigl(\lim_{h\to 0}\frac{\cos
h-1}{x}\Bigr)+\cos a\Bigl(\lim_{h\to 0}\frac{\sen h}{h}\Bigr)=\cos a\,.
$$
\eqref{itlimdificeis2} Escrevendo
$$
\frac{x^3-\alpha^3}{\sen (\tfrac{\pi}{\alpha}x)}=
\frac{x^3-\alpha^3}{x-\alpha}\frac{1}{\frac{\sen
(\tfrac{\pi}{\alpha}x)}{x-\alpha}}\,.
$$
Já calculamos $\lim_{x\to \alpha}\frac{x^3-\alpha^3}{x-\alpha}= 3\alpha^2$, e
chamando $y\pardef \tfrac{\pi}{\alpha}x$ seguido por $y'\pardef y-\pi$,
$$\lim_{x\to \alpha}\frac{\sen(\tfrac{\pi}{\alpha}x)}{x-\alpha}
=\lim_{y\to \pi}\frac{\sen(y)}{\frac{\alpha}{\pi}(y-\pi)}
=\frac{\pi}{\alpha}\lim_{y'\to 0}\frac{\sen(y'+\pi)}{y'}
=-\frac{\pi}{\alpha}\lim_{y'\to 0}\frac{\sen(y')}{y'}=-\frac{\pi}{\alpha}\,.$$
Logo,
$$
\lim_{x\to \alpha}\frac{x^3-\alpha^3}{\sen
(\tfrac{\pi}{\alpha}x)}=(3\alpha^2)/ (-\frac{\pi}{\alpha})=-3\alpha^3/ \pi\,.
$$
\eqref{itlimdificeis3} Comecemos definindo $t$ tal que $\pi-3x=3t$, isto é:
$t\pardef \pisobretres-x$:
$$\lim_{x\to \pisobretres}\frac{1-2\cos
x}{\sen(\pi-3x)}=\lim_{t\to 0}\frac{1-2\cos (\pisobretres-t)}{\sen (3t)}\,.$$
Mas $\cos (\pisobretres-t)=\cos \pisobretres\cos t+\sen \pisobretres\sen
t=\frac12\cos t+\frac{\sqrt{3}}{2}\sen t$,
\begin{align*}
\lim_{t\to 0}\frac{1-2\cos (\pisobretres-t)}{\sen (3t)}&=
\lim_{t\to 0}\frac{1-\cos t}{\sen (3t)}-\sqrt{3}\lim_{t\to 0}\frac{\sen
(t)}{\sen (3t)}\\
&=\lim_{t\to 0}\frac{1-\cos t}{t}\frac{1}{
3\frac{\sen (3t)}{3t}}-
\sqrt{3}\lim_{t\to
0}\frac{\sen (t)}{t}\frac{1}{3\frac{\sen
(3t)}{3t}}=0-\sqrt{3}\frac{1}{3}=-\frac{1}{\sqrt{3}}\,.
\end{align*}
\eqref{itlimdificeis34}
Se $a\geq b$, é melhor escrever $a^x+b^x=a^x(1+(b/a)^x)$, logo
\[
\lim_{x\to\infty}\frac{1}{x}\ln(a^x+b^x)
=\ln a+
\lim_{x\to\infty}\frac{\ln(1+(b/a)^x)}{x}
=\ln a\,.
\]
O caso $a<b$ se trata da mesma maneira. Obtemos:
\[
\lim_{x\to\infty}\frac{1}{x}\ln(a^x+b^x)
=
\begin{cases}
\ln a&\text{ se }a\geq b\,,\\
\ln b&\text{ se }a< b\,.\\
\end{cases}
\]
\eqref{itlimdificeis37}
O caso $n=1$ é trivial: $(x_0+h)^1=x_0+h$. Quando $n=2$,
$(x_0+h)^2=x_0^2+2x_0h+h^2$, logo (veja o Exemplo
\ref{Ex:derivxissdoisemum})
$$
\lim_{h\to 0}\frac{(x_0+h)^2-x_0^2}{h}=
\lim_{h\to 0}(2x_0+h)=2x_0\,.
$$
Para $n=3,4,\dots$, usaremos a fórmula do binômio de
Newton:
$$(x_0+h)^n=x_0^n+\binom{n}{1}x_0^{n-1}h+\binom{n}{2}x_0^{n-2}
h^2+\dots+\binom{n}{k}x_0^{n-k} h^k+\dots+h^n\,,
$$
onde $\binom{n}{k}=\frac{n!}{(n-k)!k!}$. Portanto,
$$
\frac{(x_0+h)^n-x_0^n}{h}=\binom{n}{1}x_0^{n-1}+\binom{n}{2}x_0^{n-2}
h+\dots+\binom{n}{k}x_0^{n-k} h^{k-1}+\dots+h^{n-1}\,.
$$
Observe que cada termo dessa soma, a partir do segundo, contém
uma potência de $h$. Logo, quando $h\to 0$, só sobra
o primeiro termo: $\binom{n}{1}x_0^{n-1}=nx_0^{n-1}$. Logo,
\[
\lim_{h\to 0}\frac{(x_0+h)^n-x_0^n}{h}=nx_0^{n-1}\,.
\]
Esse limite será usado para \emph{derivar} polinômios, no próximo capítulo.
\end{Solution}
\protect \section *{Capítulo \ref {Cap:Continuidade}}
\begin{Solution}{5.1}
Em qualquer ponto $a\neq 0$, os limites laterais nem existem, então
$f$ é descontínua. Por outro lado vimos que $\lim_{x\to
0^+}f(x)=\lim_{x\to 0^- }f(x)=0$. Logo,
$\lim_{x\to 0}f(x)=f(0)$: $f$ é contínua em $0$.
\end{Solution}
\begin{Solution}{5.2}
$D=\bR$, $C=\bR_*$.
\end{Solution}
\begin{Solution}{5.3}
Considere um $a\neq 2$. $f$ sendo uma razão de polinómios, e como o denumerador não se
anula em $a$, a Proposição
\ref{Prop:continuidadechiante} implica que $f$ é contínua em $a$.
Na verdade, quando $x\neq 2$, $f(x)=\frac{x^2-3x+2}{x-2}=\frac{(x-1)(x-2)}{x-2}=x-1$.
Logo, $\lim_{x\to 2}f(x)=\lim_{x\to 2}(x-1)=1$. Como $1\neq f(2)=0$, $f$ é descontínua em
$2$.
Para tornar $f$ contínua na reta toda, é so redefiní-la em $x=2$, da seguinte maneira:
$$
\tilde{f}(x)\pardef
\begin{cases}
\frac{x^2-3x+2}{x-2}&\text{ se }x\neq 2\,,\\
1&\text{ se }x=2\,.
\end{cases}
$$
Agora, $\tilde{f}(x)=x-1$ para todo $x\in \bR$.
\end{Solution}
\begin{Solution}{5.4}
Como $\lim_{x\to 1}f(x)=1-a$ e que $f(1)=5+a$, é preciso ter $1-a=5+a$, o que implica
$a=-2$.
\end{Solution}
\begin{Solution}{5.5}
Por um lado, como $\tanh \tfrac1x$ é a composição de duas funções contínuas, ela é
contínua em todo $a\neq 0$.
Um raciocínio parecido implica que $g$ é contínua em todo $a\neq 0$.
Por outro lado,
vimos no item \eqref{itmudvarlim4} do Exercício \ref{Exo:mudvarlimites} que $\lim_{x\to
0^{\pm}}\tanh \frac{1}{x}=\pm 1$, o que implica que $f$ é descontínua em $a=0$.
Vimos no item \eqref{itmudvarlim5} do mesmo exercício que $\lim_{x\to
0^{\pm}}x\tanh \frac{1}{x}=0$, logo $\lim_{x\to 0}g(x)$ existe e vale $g(0)$.
Logo, $g$ é contínua em $a=0$.
\begin{center}
\begin{bmlimage}\begin{tikzpicture}
\pgfmathsetmacro{\a}{3};
\pgfmathsetmacro{\e}{0.12};
\draw[ ->] (-\a,0)--(\a,0)node[right]{$x$};
\draw[ ->] (0,-1.3)--(0,1.3)node[left]{$\tanh\frac1x$};
\draw[->, thick, domain=-\a:-\e] plot (\x,
{(exp(1/\x)-exp(-1/\x))/(exp(1/\x)+exp(-1/\x))});
\draw[<-, thick, domain=\e:\a] plot (\x, {(exp(1/\x)-exp(-1/\x))/(exp(1/\x)+exp(-1/\x))});
\fill (0,0) circle (0.50mm);

\begin{scope}[xshift=7cm]
\draw[ ->] (-\a,0)--(\a,0)node[right]{$x$};
\draw[ ->] (0,-1.3)--(0,1.3)node[left]{$x\tanh\frac1x$};
\draw[->, thick, domain=-\a:-\e] plot (\x,
{\x*(exp(1/\x)-exp(-1/\x))/(exp(1/\x)+exp(-1/\x))});
\draw[<-, thick, domain=\e:\a] plot (\x,
{\x*(exp(1/\x)-exp(-1/\x))/(exp(1/\x)+exp(-1/\x))});
\fill (0,0) circle (0.50mm);
\end{scope}
\end{tikzpicture}\end{bmlimage}
\end{center}
\end{Solution}
\begin{Solution}{5.6}
(Esboçar os gráficos de $f,g,h$ ajuda a compreensão do exercício).

Temos $f(-1)=1$, $f(2)=4$.
Como $f$ é contínua, o Teorema \eqref{Teo:ValInterm} se aplica:
se $1\leq h\leq 4$, o gráfico de $f$ corta a reta horizontal de
altura $y=h$ pelo menos uma vez. Na verdade, ele corta a reta
exatamente uma vez se $1<h\leq 4$, e duas vezes se $h=1$.

Temos $g(-1)=-1$, $g(1)=1$.
Como $g$ é descontínua em $x=0$, o teorema não se aplica. Por
exemplo, o gráfico de $g$ nunca corta a reta horizontal $y=\frac12$.

Temos $h(0)=-1$, $h(2)=1$. Apesar de $h$ não ser contínua, ela
satisfaz à propriedade do valor intermediário. De fato, o gráfico de
$h$ corta a reta $y=h_*$ duas vezes se $-1\leq h_*<1$, e uma vez se $h_*=1$.
\end{Solution}
\begin{Solution}{5.7}
$a=1$, $b=3$, $c=\pm 2$.
\end{Solution}
\begin{Solution}{5.8}
Seja $y\in \bR$ fixo, qualquer. Como $\lim_{x\to
+\infty}f(x)=+\infty$, existe $b>0$ grande o suficiente tal que
$f(b)>y$.
Como $\lim_{x\to -\infty}f(x)=-\infty$, existe $a<0$ grande o
suficiente tal que $f(a)<y$.
Pelo Teorema do Valor Intermediário, existe $c\in [a,b]$ tal que
$f(c)=y$. Isto implica que $y\in \imagem(f)$.
\end{Solution}
\begin{Solution}{5.9}
Considere $\lim_{x\to 0^-}f(x)$. Chamando $y\pardef -x$, $x\to 0^-$ corresponde
a $y\to 0^+$.
Logo, $$\lim_{x\to 0^-}f(x)=\lim_{y\to 0^+}f(-y)=-\lim_{y\to 0^+}f(y)\equiv-
\lim_{x\to 0^+}f(x)\,.$$
Portanto, para uma função ímpar ser contínua em $0$, é preciso ter
$\lim_{x\to 0^+}f(x)=f(0)=0$ (não pode ser $L>0$).
\end{Solution}
\protect \section *{Capítulo \ref {Cap:Derivacao}}
\begin{Solution}{6.1}
Se $P=(a,a^2)$, $Q=(\lambda,\lambda^2)$, a equação da reta $r^{PQ}$ é dada por
$y=(\lambda+a)x-a\lambda$. Quando $\lambda\to a$ obtemos
a equação da reta tangente à parábola em $P$: $y=2a x-a^2$.
Por exemplo, se $a=0$, a equação da reta tangente é $y=0$, se $a=2$, é
$y=4x-4$,
$a=-1$, é $y=-2x-1$ (o que foi calculado no Exemplo
\ref{Ex:primeiraretatangente}).
\end{Solution}
\begin{Solution}{6.2}
Como $x^2-x=(x-\frac12)^2-\frac14$, o gráfico obtém-se a partir do gráfico de
$x\mapsto x^2$ por duas translações.
Usando a definição de derivada, podemos calcular para todo $a$:
$$f'(a)=\lim_{x\to a}\frac{f(x)-f(a)}{x-a}
=\lim_{x\to a}\frac{(x^2-x)-(a^2-a)}{x-a}=
\lim_{x\to a}\Bigl\{\frac{x^2-a^2}{x-a}-1\Bigr\}=2a-1\,.$$
Aplicando essa fórmula para $a=0,\frac12,1$, obtemos $f'(0)=-1$,
$f'(\frac12)=0$, $f'(1)=+1$.
Esses valores correspondem às inclinações das retas
tangentes ao gráfico nos pontos $(0,f(0))=(0,0)$,
$(\frac12,f(\frac12))=(\frac12,-\frac14)$ e $(1,f(1))=(1,0)$:
\begin{center}
\begin{bmlimage}\begin{tikzpicture}[scale=1.7]
\newcommand{\funcao}[1]{((#1)^2-(#1))}
\newcommand{\dfuncao}[2]{ ((\funcao{#1+#2})/(#2)-(\funcao{#1})/(#2)) }
\draw[ ->] (0,-0.2)--(0,1)node[right]{$\scriptstyle{x^2-x}$};
\draw[ ->] (-0.5,0)--(1.5,0);
\draw[thick, domain=-0.5:1.5] plot (\x,{\funcao{\x}});
\foreach \a in {0,0.5,1} {
\draw[thick,  domain={\a-0.3}:{\a+0.3}] plot
(\x,{(\dfuncao{\a}{0.01})*(\x-\a)+\funcao{\a}});
\fill (\a,{\funcao{\a}}) circle (0.40mm);
}
\draw[dotted]
(0.5,{\funcao{0.5}})--(0.5,0)node[above]{$\scriptstyle{\tfrac12}$};

\draw (1,0) node[above]{$\scriptstyle{1}$};
\end{tikzpicture}\end{bmlimage}
\end{center}
\end{Solution}
\begin{Solution}{6.3}
\eqref{itderivelem11} $f'(1)=\half$,
\eqref{itderivelem1} $f'(0)=\half$ (a mesma do item anterior, pois o
gráfico de $\sqrt{1+x}$ é o de $\sqrt{x}$ transladado de $1$ para a esquerda!),
\eqref{itderivelem2} $f'(0)=1$,
\eqref{itderivelem3} $f'(-1)=-4$,
\eqref{itderivelem4} $f'(2)=-\frac{1}{4}$.
\end{Solution}
\begin{Solution}{6.4}
\eqref{iteqretang1} $y=3x+9$,
\eqref{iteqretang2} $y=\frac{1}{4}$,
\eqref{iteqretang3} $y=\half x+1$,
\eqref{iteqretang4} $y=-x-2$, $y=-x+2$
\eqref{iteqretang5} Observe que a função descreve a metade superior de um
circulo de raio $1$ centrado na origem. As retas tangentes são, em $(-1,0)$:
$x=-1$, em $(1,-1)$: não existe (o ponto nem pertence ao círculo!), em $(0,1)$:
$y=1$, e em $(1,0)$: $x=1$.
\eqref{iteqretang6} Mesmo sem saber ainda como calcular a derivada da
função seno: $y=x$, $y=1$.
\end{Solution}
\begin{Solution}{6.5}
Primeiro é preciso ter uma função para representar o círculo na vizinhança de
$P_1$: $f(x)\pardef \sqrt{25-x^2}$. A inclinação da tangente em $P_1$ é dada por
\begin{align*}
f'(3)=\lim_{x\to 3}\frac{f(x)-f(3)}{x-3}&=
\lim_{x\to 3}\frac{\sqrt{25-x^2}-\sqrt{16}}{x-3}\\
&=\lim_{x\to 3}\frac{(25-x^2)-{16}}{(x-3)(\sqrt{25-x^2}+\sqrt{16})}
=\lim_{x\to 3}\frac{-(3+x)}{\sqrt{25-x^2}+\sqrt{16}}=-\tfrac34\,.
\end{align*}
(Essa inclinação poderia ter sido obtido observando que a reta
procurada é perpendicular ao segmento $OP$, cuja inclinação é
$\frac43$...)
Portanto, a equação da reta tangente em $P_1$ é $y=-\frac34
x+\frac{25}{4}$.  No ponto $P_2$, é preciso tomar a função
$f(x)\pardef -\sqrt{25-x^2}$. Contas parecidas dão a equação
da tangente ao círculo em $P_2$: $y=\frac34 x-\frac{25}{4}$.
\begin{center}
\begin{bmlimage}\begin{tikzpicture}[scale=1]
\draw[ ->] (0,-1.2)--(0,1.2);
\draw[ ->] (-1.2,0)--(1.2,0);
\draw[dotted]
(0.6,0)node[below]{$\scriptstyle{3}$} -- (0.6,0.8) -- (0,0.8)node[left]
{$\scriptstyle{4}$};
\pgfmathsetmacro{\a}{0.6};
\draw[very thick, domain={\a-0.4}:{\a+0.4}] plot
(\x,{-0.75*(\x-\a)+0.8});
\draw[very thick,  domain={\a-0.4}:{\a+0.4}] plot
(\x,{+0.75*(\x-\a)-0.8});
\draw (0,0) circle (1cm);
\fill (0.6,0.8) circle (0.40mm);
\fill (0.6,-0.8) circle (0.40mm);
\draw (0.6,0.8) node[above right]{$P_1$};
\draw (0.6,-0.8) node[below right]{$P_2$};
\draw[very thick] (1,-0.4)--(1,0.4);
\fill (1,0) circle (0.40mm);
\draw (1,0) node[above right]{$P_3$};
\end{tikzpicture}\end{bmlimage}
\end{center}
A reta tangente ao círculo no ponto $P_3$ é vertical, e tem equação $x=5$.
Aqui podemos observar que a derivada de $f$ em $a=5$ \emph{não existe}, porqué
a inclinação de uma reta vertical não é definida (o que não impede achar a sua
equação...)!
\end{Solution}
\begin{Solution}{6.6}
Se $f(x)=\sqrt{x}$, temos que para todo $a>0$,
$f'(a)=\frac{1}{2\sqrt{a}}$.
Como a reta $8x-y- 1 = 0$ tem inclinação $8$, precisamos achar um $a$ tal que
$f'(a)=8$, isto é, tal que $\frac{1}{2\sqrt{a}}=8$: $a=\frac{1}{256}$.
Logo, o ponto procurado é $P=(a,f(a))=(\frac{1}{256},\frac{1}{16})$.
\end{Solution}
\begin{Solution}{6.7}
Para a reta $y=x-1$ (cuja inclinação é $1$) poder ser tangente ao gráfico de
$f$ em algum ponto $(a,f(a))$, esse $a$ deve satisfazer $f'(a)=1$. Ora, é fácil
ver que para um $a$ qualquer, $f'(a)=2a-2$. Logo, $a$ deve satisfazer $2a-2=1$,
isto é: $a=\frac32$. Ora, a reta e a função devem ambas passar pelo ponto
$(a,f(a))$, logo $f(a)=a-1$, isto é:
$(\frac32)^2-2\cdot\frac32+\beta=\frac32-1$. Isolando:
$\beta=\frac{5}{4}$.
\begin{center}
\begin{bmlimage}\begin{tikzpicture}
\newcommand{\funcao}[1]{(#1)^2-2*(#1)+1.25}
\newcommand{\dfuncao}[2]{ (\funcao{#1+#2})/(#2)-(\funcao{#1})/(#2)}
\draw[ ->] (0,-0.2)--(0,2.5) node[right]{$y$};
\draw[ ->] (-1,0)--(3,0) node[right]{$x$};
\draw[thick, domain=-0.5:2.5] plot (\x,{\funcao{\x}})
node[right]{$y=x^2-2x+\frac54$};
\pgfmathsetmacro{\a}{1.5};
\draw[thick,  domain={\a-1}:{\a+1}] plot
(\x,{(\dfuncao{\a}{0.01})*(\x-\a)+\funcao{\a}}) node[right]{$y=x-1$};
\fill (\a,{\funcao{\a}}) circle (0.40mm);
\end{tikzpicture}\end{bmlimage}
\end{center}
Esse problema pode ser resolvido sem usar derivada:
para a parábola $y=x^2-2x+\beta$ ter $y=x-1$ como reta tangente, a única
possibilidade é que as duas se intersectem em um ponto só, isto é, que a
equação $x^2-2x+\beta=x-1$ possua uma única solução. Rearranjando:
$x^2-3x+\beta+1=0$. Para essa equação ter uma única solução, é preciso que o
seu $\Delta=5-4\beta=0$. Isso implica $\beta=\frac{5}{4}$.
\end{Solution}
\begin{Solution}{6.8}
Seja $P=(a,\frac1a)$ um ponto qualquer do gráfico. Como
$f'(a)=-\frac{1}{a^2}$, a reta tangente ao gráfico em $P$ é
$y=f'(a)(x-a)+f(a)=-\frac{1}{a^2}(x-a)+\frac1a$. Para essa reta passar pelo
ponto $(0,3)$, temos $3=-\frac{1}{a^2}(0-a)+\frac1a$, o que
significa que $a=\frac{2}{3}$.
Logo, a reta tangente ao gráfico de $\frac1x$ no ponto $P=(\frac23,\frac32)$
passa pelo ponto $(0,3)$.
\end{Solution}
\begin{Solution}{6.9}
$P=(-1,2)$.
\end{Solution}
\begin{Solution}{6.10}
Por exemplo, $f(x)\pardef |x+1|/2-|x|+|x-1|$.
Mais explicitamente,
\begin{center}
\begin{bmlimage}\begin{tikzpicture}
\draw (-8,0.8) node{$\displaystyle{
f(x)=
\begin{cases}
\frac{1-x}{2}&\text{ se }x\leq -1\\
\frac{x+3}{2}&\text{ se }-1\leq x\leq 0\\
\frac{3-3x}{2}&\text{ se }0\leq x\leq 1\\
\frac{x-1}{2}&\text{ se }x\geq 1\,.
\end{cases}
}$};
\draw [thick, domain=-2:2, samples=200]plot
(\x,{abs(\x+1)/2-abs(\x)+abs(\x-1)});
\draw [->](-2,0)--(2,0) ;
\draw (2.2,0) node {$x$};
\draw [->](0,-0.5)--(0,2);
\draw (-0.5,1.8) node {$f(x)$};
\draw (-1,0) node {$\shortmid$};
\draw (-1,-0.4) node {$-1$};
\draw (1,0) node {$\shortmid$};
\draw (1,-0.4) node {$1$};
\end{tikzpicture}\end{bmlimage}
\end{center}
$f$ não é derivável em $x=1$, porqué
$\lim_{x\to 1^+}\frac{f(x)-f(1)}{x-1}=\lim_{x\to
1^+}\frac{\frac{x-1}{2}-0}{x-1}=\frac12$,
enquanto
$\lim_{x\to 1^-}\frac{f(x)-f(1)}{x-1}=\lim_{x\to
1^-}\frac{\frac{3-3x}{2}-0}{x-1}=-\frac32\neq \frac12$.
A não-derivabilidade nos pontos $-1$ e $0$ obtem-se da mesma maneira.
\end{Solution}
\begin{Solution}{6.11}
De fato, se $f$ é par,
\begin{align*}
f'(-x)=\lim_{h\to 0}\frac{f(-x+h)-f(-x)}{h}
&=\lim_{h\to 0}\frac{f(x-h)-f(x)}{h}\\
&=-\lim_{h'\to 0}\frac{f(x+h')-f(x)}{h'}=-f'(x)\,.
\end{align*}
\end{Solution}
\begin{Solution}{6.12}
$af'(a)-f(a)$
\end{Solution}
\begin{Solution}{6.13}
$(\sqrt{x})'=\lim_{h\to 0}\frac{\sqrt{x+h}-\sqrt{x}}{h}=
\lim_{h\to 0}\frac{1}{\sqrt{x+h}+\sqrt{x}}=\frac{1}{2\sqrt{x}}$.
O outro limite se calcula de maneira parecida:
$$(\frac{1}{\sqrt{x}})'=\lim_{h\to
0}\frac{\frac{1}{\sqrt{x+h}}-\frac{1}{\sqrt{x}}}{h}=
\lim_{h\to
0}\frac{\sqrt{x}-\sqrt{x+h}}{h\sqrt{x}\sqrt{x+h}}=\cdots=-\frac{1}{2\sqrt{x^3}}
\,.
$$
\end{Solution}
\begin{Solution}{6.14}
Como $(\sen)'(x)=\cos x$, a inclinação da reta tangente em $P_1$ é $\cos(0)=1$,
em $P_2$ é $\cos(\pisobredois)=0$, e em $P_3$ é $\cos(\pi)=-1$. Logo, as
equações das respectivas retas tangentes são $r_1$: $y=x$, $r_2$: $y=1$, $r_3$:
$y=-(x-\pi)$:
\begin{center}
\begin{bmlimage}\begin{tikzpicture}[scale=1]
\newcommand{\funcao}[1]{sin(#1 r)}
\newcommand{\dfuncao}[2]{ (\funcao{#1+#2})/{#2}-(\funcao{#1})/{#2} }
\draw[ ->] (0,-0.2)--(0,1)node[left]{$\scriptstyle{\sen x}$};
\draw[ ->] (-2,0)--(5,0);
\draw[thick, domain=-1.8:4.8] plot (\x,{\funcao{\x}});

\pgfmathsetmacro{\e}{0.75};
\pgfmathsetmacro{\a}{0};
\draw[thick,  domain={\a*1.5707-\e}:{\a*1.5707+\e}] plot
(\x,{\x});
\pgfmathsetmacro{\a}{1};
\draw[thick,  domain={\a*1.5707-\e}:{\a*1.5707+\e}] plot
(\x,{1});
\pgfmathsetmacro{\a}{2};
\draw[thick,  domain={\a*1.5707-\e}:{\a*1.5707+\e}] plot
(\x,{3.1415-\x});

\foreach \a in {0,1,2} {
\fill ({\a*1.5707},{sin(\a*1.5707 r)}) circle (0.50mm);
}
% \draw[dotted]
% (0.5,{\funcao{0.5}})--(0.5,0)node[above]{$\scriptstyle{\tfrac12}$};
%
% \draw (1,0) node[above]{$\scriptstyle{1}$};
\end{tikzpicture}\end{bmlimage}
\end{center}
\end{Solution}
\begin{Solution}{6.16}
Por exemplo, se $f(x)=g(x)=x$, temos $(f(x)g(x))'=(x\cdot x)'=(x^2)'=2x$,
e $f'(x)g'(x)=1\cdot 1=1$. Isto é, $(f(x)g(x))'\neq f'(x)g'(x)$.
\end{Solution}
\begin{Solution}{6.17}
Já sabemos que $(x)'=1$, e que $(x^2)'=2x$, o que prova a fórmula para $n=1$ e $n=2$.
Supondo que a fórmula foi provada para $n$, provaremos que ela vale para $n+1$
também.
De fato, usando a regra de Leibniz e a hipótese de indução,
\[
(x^{n+1})'=
(x\cdot x^n)'=1\cdot x^n+x\cdot nx^{n-1}=x^n+nx^n=(n+1)x^n\,.
\]
\end{Solution}
\begin{Solution}{6.18}
\eqref{itderivbas0} $-5$
\eqref{itderivbas1} $(x^3-x^7)'=(x^3)'-(x^7)'=3x^2-7x^6$.
\eqref{itderivbas111}
$(1+x+\frac{x^2}{2}+\frac{x^3}{3})'=(1)'+(x)'+(\frac{x^2}{2})'+(\frac{x^3}{3})'
=1+x+x^2$.
\eqref{itderivbas2}
$(\frac{1}{1-x})'=-\frac{1}{(1-x)^2}\cdot(1-x)'=\frac{1}{(1-x)^2}$
\eqref{itderivbas15} $\sen x+x\cos x$
\eqref{itderivbas151} Usando duas vezes a regra de Leibniz:
$((x^2+1)\sen x\cos x)'=2x\sen x\cos x+(x^2+1)(\cos^2x-\sen^2x)$
\eqref{itderivbas3} $\frac{x\cos x-\sen x}{x^2}$
\eqref{itderivbas4} $(\frac{x+1}{x^2-1})'=(\frac{1}{x-1})'=\frac{-1}{(x-1)^2}$.
\eqref{itderivbas1111} $(x+1)^5=f(g(x))$ com $f(x)=x^5$ e $g(x)=x+1$.
Logo, $((x+1)^5)'=f'(g(x))g'(x)=5(x+1)^4$. Obs: poderia também expandir
$(x+1)^5=x^5+\cdots$, derivar termo a termo, mas é muito mais longo, e a
resposta não é fatorada.
\eqref{itderivbas6} Como $(3+\frac{1}{x})^2=f(g(x))$ com
$f(x)=x^2$ e $g(x)=3+\frac{1}{x}$, e que $f'(x)=2x$,
$g'(x)=(3+\frac{1}{x})'=0-\frac{1}{x^2}$, temos
$((3+\frac{1}{x})^2)'=2(3+\frac{1}{x})\cdot(\frac{-1}{x^2})=-2\frac{3+\frac{1}
{x}}{x^2}$.
\eqref{itderivbas5} Como $\sqrt{1-x^2}=f(g(x))$, com $f(x)=\sqrt{x}$,
$g(x)=1-x^2$, e que $f'(x)=\frac{1}{2\sqrt{x}}$, $g'(x)=-2x$,
temos $(\sqrt{1-x^2})'=\frac{-x}{\sqrt{1-x^2}}$-
\eqref{itderivbas7} $3\sen^2x\cos x+7\cos^6x\sen x$
\eqref{itderivbas8} $\frac{\sen x}{(1-\cos x)^2}$
\eqref{itderivbas8meio} $\frac{2\sen (2x-1)}{(\cos(2x-1))^2}$
\eqref{itderivbas9}
$(\frac{1}{\sqrt{1+x^2}})'=((1+x^2)^{-\frac12})'=-\frac12(1+x^2)^{-\frac32}
\cdot (2x)=-\frac{x}{(1+x^2)^{\frac32}}=\frac{-x}{\sqrt{(1+x^2)^3}}$.
\eqref{itderivbas10}
$(\frac{(x^2-1)^2}{\sqrt{x^2-1}})'=((x^2-1)^{\frac32})'=\frac{3}{2}(x^2-1)^{
\frac12}
\cdot(2x)=3x\sqrt{x^2-1}$ Obs: vale a pena simplificar a fração antes de
usar a regra do quociente!
\eqref{itderivbas11} $\frac{9}{\sqrt{9+x^2}(x+\sqrt{9+x^2})^2}$
\eqref{itderivbas12} $\frac{1}{4\sqrt{x}\sqrt{1+\sqrt{x}}}$
\eqref{itderivbas16} $\frac{\cos x+x\sen x}{(\cos x)^2}$
\eqref{itderivbas17} Usando duas vezes a regra da cadeia:
$(\cos\sqrt{1+x^2})'=(-\sen \sqrt{1+x^2})(\sqrt{1+x^2})'=\frac{-x\sen
\sqrt{1+x^2}}{\sqrt{1+x^2}}$
\eqref{itderivbas18} $\cos(\sen x)\cdot\cos x$
\end{Solution}
\begin{Solution}{6.19}
\eqref{itderivexpon1} $(2e^{-x})'=2(e^{-x})'=2(e^{-x}\cdot(-x)')=-2e^{-x}$.
\eqref{itderivexpon2} $\frac{1}{x+1}$
\eqref{itderivexpon3} $(\ln (e^{3x}))'=(3x)'=3$
\eqref{itderivexpon61} $e^x(\sen x+\cos x)$
\eqref{itderivexpon4} $\cos x \cdot e^{\sen x}$
\eqref{itderivexpon5} $e^{e^x}\cdot e^x$
\eqref{itderivexpon6} $\frac{2e^{2x}}{1+e^{2x}}$
\eqref{itderivexpon7} $\ln x+x\frac{1}{x}=\ln x+1$
\eqref{itderivexpon8} $\frac{-e^{\frac1x}}{x^2}$
\eqref{itderivexpon12} $-\tan x$
\eqref{itderivexpon13} $\frac{-1}{\sen x}$
\end{Solution}
\begin{Solution}{6.20}
$(\senh
x)'=(\frac{e^x-e^{-x}}{2})'=\frac{e^x+e^{-x}}{2}\equiv \cosh x$.
Do mesmo jeito, $(\cosh x)'=\senh x$.
Para $\tanh$, basta usar a regra do quociente.
Observe as semelhanças entre as derivadas das funções trigonométricas
hiperbólicas e as funções trigonométricas.
\end{Solution}
\begin{Solution}{6.21}
\eqref{itlimbargeaotviaderiv1}
Sabemos que o limite $\lim_{x\to 1}\frac{x^{999}-1}{x-1}$ dá a inclinação da
reta tangente ao gráfico da função $f(x)=x^{999}$ no ponto $a=1$, isto é:
$\lim_{x\to 1}\frac{x^{999}-1}{x-1}=f'(1)$. Mas como
$f'(x)=999x^{998}$, temos $f'(1)=999$.
\eqref{itlimbargeaotviaderiv2}
Da mesma maneira, $\lim_{x\to \pi}\frac{\cos x+1}{x-\pi}=
\lim_{x\to \pi}\frac{\cos x-\cos(\pi)}{x-\pi}$ dá a inclinação da reta tangente
ao gráfico do $\cos$ no ponto $\pi$. Como $(\cos x)'=-\sen x$, o limite vale
$0$.
\eqref{itlimbargeaotviaderiv3} $2\pi \cos(\pi^2)$
\eqref{itlimbargeaotviaderiv4} $\frac12$
\eqref{itlimbargeaotviaderiv5} $\lambda$
\end{Solution}
\begin{Solution}{6.22}
Fora de $x=0$, $g$ é derivável e a sua derivada se calcula facilmente:
$g'(x)=(x^2\sen \frac1x)'=2x\sen \frac1x-\cos\frac1x$.
Do mesmo jeito $f$ é derivável fora de $x=0$.
Em $x=0$,
$$
g'(0)=\lim_{h\to 0}\frac{g(h)-g(0)}{h}=\lim_{h\to 0} h\sen \tfrac1h=0\,.
$$
(O último limite pode ser calculado como no Exemplo \ref{Ex:sanduicheseno},
escrevendo
$-h\leq h\sen \tfrac1h\leq +h$.)
Assim, $g$ é derivável também em $x=0$. No entanto, como
$$\lim_{h\to 0}\frac{f(h)-f(0)}{h}=\lim_{h\to 0} \sen \tfrac1h\,,$$
$f'(0)$ não existe: $f$ não é derivável em $x=0$.
\end{Solution}
\begin{Solution}{6.23}
\eqref{itderivfelevg1}
$(x^{\sqrt{x}})'=(e^{\sqrt{x}\ln x})'=(\frac{\ln x}{2}+1){
x^{\sqrt{x}-\frac12}}$.
\eqref{itderivfelevg3} $((\sen x)^x)'=(\ln \sen x+x\cot x)(\sen x)^x$.
\eqref{itderivfelevg4} $(x^{\sen x})'=(\cos x\ln x+\frac{\sen x}{x})x^{\sen x}$.
\eqref{itderivfelevg2} $(x^{x^x})'=\bigl((\ln x+1)\ln
x+\frac1x\bigr)x^xx^{x^x}$.
\end{Solution}
\begin{Solution}{6.24}
 As derivadas são dadas por:
\eqref{itderitruclog1}
$\frac{(x+1)(x+2)(x+3)}{(x+4)(x+5)(x+6)}
(\frac{1}{x+1}+\frac{1}{x+2}+\frac{1}{x+3}-\frac{1}{x+4}-\frac{1}{x+5}-\frac{1}{
x+6})$
\eqref{itderitruclog2}
$\frac{x\sen^3x}{\sqrt{1+\cos^2x}}\bigl(\frac{1}{x}+3\cot x+\frac{\sen
x\cos x}{{1+\cos^2x}}\bigr)$
\eqref{itderitruclog3}
$\bigl(\prod_{k=1}^n(1+x^k)\bigr)\sum_{k=1}^n\frac{kx^{k-1}}{1+x^k}$
\end{Solution}
\begin{Solution}{6.26}
\eqref{itderivfuncinv1} $\frac{-2x}{(\ln a)(1-x^2)}$
\eqref{itderivfuncinv2} $\frac{-2x}{\sqrt{1-(1-x^2)^2}}$
\eqref{itderivfuncinv3} $1$
\eqref{itderivfuncinv4} $-1$
\eqref{itderivfuncinv5} $\frac{-x}{\sqrt{1-x^2}}$
\end{Solution}
\begin{Solution}{6.28}
(O gráfico da função pode ser usado para interpretar o resultado.)
\eqref{itRolleA1} Temos $f(-2)=f(1)$, e como $f'(x)=2x+1$, vemos que a derivada
se anula em $c=-\frac{1}{2}\in (-2,1)$.
\eqref{itRolleA2} Aqui são três pontos possíveis: $c=-\pi$, $c=0$ e $c=+\pi$.
\eqref{itRolleA3} Temos $f(-1)=f(0)$ e $f'(x)=4x^3+1$, cuja raiz é
$-(\frac14)^{1/3}\in (-1,0)$.
\end{Solution}
\begin{Solution}{6.29}
Vemos que existem dois pontos $C$ em que a inclinação é igual à inclinação do
segmento $AB$:
\begin{center}
\begin{bmlimage}\begin{tikzpicture}
\newcommand{\funcao}[1]{sin(#1 r)}
\newcommand{\dfuncao}[2]{ (\funcao{#1+#2})/{#2}-(\funcao{#1})/{#2}}
\pgfmathsetmacro{\a}{-1.57};
\pgfmathsetmacro{\b}{1.57};
\pgfmathsetmacro{\c}{0.8};
\pgfmathsetmacro{\cc}{-\c};
\draw[ ->] (-2,0)--(2,0);
\draw[ ->] (0,-1)--(0,1);
% \draw[color=gray, domain=\a-1.3:\b+1.3] plot (\x,{\funcao{\x}});
\draw[thick, domain=\a:\b] plot (\x,{\funcao{\x}});
\coordinate (A) at (\a,{\funcao{\a}});
\coordinate (B) at (\b,{\funcao{\b}});
\coordinate (C) at (\c,{\funcao{\c}});
\coordinate (Cc) at (\cc,{\funcao{\cc}});
\fill (A) circle (0.4mm);
\fill (B) circle (0.4mm);
\pgfmathsetmacro{\m}{2/3.1415};
\draw[thick,  domain={\c-0.5}:{\c+0.5}] plot
(\x,{\funcao{\c}+\m*(\x-\c)});
\draw[thick,  domain={\cc-0.5}:{\cc+0.5}] plot
(\x,{\funcao{\cc}+\m*(\x-\cc)});
\draw (A) node[left]{$A$};
\draw (B) node[right]{$B$};
\draw (C) node[above]{$C$};
\fill (C) circle (0.4mm);
\draw (Cc) node[below]{$C'$};
\fill (Cc) circle (0.4mm);
\draw[dashed] (A)--(B);
\end{tikzpicture}\end{bmlimage}
\end{center}
O ponto $c\in [-\pisobredois,\pisobredois]$ é tal que
$f'(c)=\frac{f(b)-f(a)}{b-a}=\frac{\sen
(\pisobredois)-\sen(0)}{\pisobredois-0}=\frac{2}{\pi}$. Como $f'(x)=\cos x$, $c$
é solução de $\cos c=\frac{2}{\pi}$. Com a calculadora obtemos duas
soluções: $c=\pm \arcos(\frac{2}{\pi})\simeq \pm 0.69$.
\end{Solution}
\begin{Solution}{6.30}
Como $f$ não é derivável no ponto $2\in [0,3]$, o teorema não se aplica. Não
existe ponto $C$ com as desejadas propriedades:
\begin{center}
\begin{bmlimage}\begin{tikzpicture}[scale=0.7]
\draw[->] (-0.5,0)--(3.5,0);
\draw[->] (0,-0.2)--(0,2.3);
\coordinate (A) at (0,0);
\coordinate (B) at (3,2);
\draw (-0.3,-0.15)--(2,1)--(3.3,2.3);
\draw[thick] (A)--(2,1)--(B);
\fill (A) circle (0.45mm);
\fill (B) circle (0.45mm);
\draw (A)node[above left]{$A$};
\draw (B)node[below right]{$B$};
\draw[dotted] (2,1)--(2,0) node[below]{$\scriptstyle{2}$};
\end{tikzpicture}\end{bmlimage}
\end{center}

\end{Solution}
\begin{Solution}{6.31}
Sejam $x_1<x_2$. Pelo Corolário \ref{Corol:ValorIntermDeriv}, existe $c\in
(x_2,x_2)$ tal que
\[
\frac{\sen x_2-\sen x_1}{x_2-x_1}=\cos(c)\,.
\]
Como $|\cos (c)|\leq 1$, isso dá \eqref{eq_DERIV_sinussLipshh}.
Por ser derivável, já sabemos que $\sen x$ é contínua, mas \eqref{eq_DERIV_sinussLipshh}
permite ver continuidade de uma maneira mais concreta. De fato,
seja $a$ um ponto qualquer da reta. Para mostrar que $\sen x$ é contínua em
$a$, precisamos escolher um $\epsilon>0$ qualquer, e mostrar que se $x$ for
suficientemente perto de $a$, $|x-a|\leq \delta$ (para um certo $\delta$) então
\[
|\sen x-\sen a|\leq \epsilon\,.
\]
Mas, usando \eqref{eq_DERIV_sinussLipshh}, vemos que a condição acima
vale se $\delta\equiv \epsilon$.
\end{Solution}
\begin{Solution}{6.32}

\eqref{itestudfunceleme2}: Como $f'(x)=x^3-x=x(x^2-1)$,
$f(x)$ é crescente em $[-1,0]\cup [1,\infty)$,
decrescente em $(-\infty,-1]\cup[0,1]$:
\begin{center}
\begin{bmlimage}\begin{tikzpicture}[scale=1.2]
\draw[ ->, thin] (-2.2,0)--(2.2,0);
\draw[ ->, thin] (0,-0.6)--(0,1);
\draw[thick, domain=-1.8:1.8, samples=50] plot (\x,{(\x)^4/4-(\x)^2/2});
 \fill (-1.414,0) circle (0.40mm);
 \fill (1.414,0) circle (0.40mm);
 \fill (0,0) circle (0.40mm);
 \fill (-1,-0.25) circle (0.40mm);
 \fill (1,-0.25) circle (0.40mm);
 \draw (-1,-0.25) node[below]{$\scriptstyle{(-1,-\tfrac{1}{4})}$};
 \draw (1,-0.25) node[below]{$\scriptstyle{(+1,-\tfrac{1}{4})}$};
\end{tikzpicture}\end{bmlimage}
\end{center}
\eqref{itestudfunceleme21}: $f(x)=2x^3-3x^2-12x+1$ é
crescente em $(-\infty,-1]\cup[2,\infty)$, decrescente em $[-1,2]$:
\begin{center}
\begin{bmlimage}\begin{tikzpicture}[scale=0.8]
\newcommand{\funcao}[1]{(2*(#1)^3-3*(#1)^2-12*(#1)+1)/10}
\draw[ ->, thin] (-2.2,0)--(4.2,0);
\draw[ ->, thin] (0,-1)--(0,1);
\draw[thick, domain=-2.5:3.5, samples=50] plot (\x,{\funcao{\x}});
\fill (-1,{0.8}) circle (0.5mm);
\draw (-1,0.8) node[above]{$\scriptstyle{(-1,8)}$};
\draw (2,-1.9) node[below]{$\scriptstyle{(2,-19)}$};
\fill (2,{-1.9}) circle (0.5mm);
\end{tikzpicture}\end{bmlimage}
\end{center}
Observe que nesse caso, a identificação dos pontos em que o gráfico corta o
eixo $x$ é mais difícil (precisa resolver uma equação do terceiro grau).
\eqref{itestudfunceleme22}: $f$ decresce em $(-\infty,-1]$, cresce em
$[-1,\infty)$. Observe que $f$ não é derivável em $x=-1$.
\eqref{itestudfunceleme3}: Já encontramos o gráfico dessa função no Exercício
\ref{Ex:graficosbasicos}. Observe que
$f(x)=||x|-1|$ não é derivável em $x=-1,0,+1$, então é melhor estudar a variação
sem a derivada: $f$ é decrescente em $(-\infty,-1]$ e em $[0,1]$,
crescente em $[-1,0]$ e em $[1,\infty)$.
\eqref{itestudfunceleme4} Como $(\sen x)'=\cos x$, vemos que o seno é crescente
em cada intervalo em que o cosseno é positivo, e decrescente em cada intervalo
em que o cosseno é negativo. Por exemplo, no intervalo $[-\pisobredois,
\pisobredois]$, $\cos x>0$, logo $\sen x$ é crescente:
\begin{center}
\begin{bmlimage}\begin{tikzpicture}[scale=0.7]
\draw[thin,  ->] (-6.2,0)--(6.2,0);
\draw[thin,  ->] (0,-1.2)--(0,1.3);
\draw[color=gray, domain=-6:6, samples=50] plot (\x,{cos(\x r)});
\draw[thick, domain=-6:6, samples=50] plot (\x,{sin(\x r)});
\draw[dotted] (-1.57,-1.1)
node[below]{$\scriptstyle{-\tfrac{\pi}{2}}$}--(-1.57,1.1);
\draw[dotted]
(1.57,-1.1)node[below]{$\scriptstyle{\tfrac{\pi}{2}}$}--(1.57,1.1);
\end{tikzpicture}\end{bmlimage}
\end{center}
\eqref{itestudfunceleme5}:
$f(x)=\sqrt{x^2-1}$ tem domínio $(-\infty,-1]\cup[1,\infty)$, é sempre
não-negativa, e $f(-1)=f(1)=0$. Temos $f'(x)=\frac{x}{\sqrt{x^2-1}}$. Logo,
a variação de $f$ é dada por:
\begin{center}
\begin{bmlimage}\begin{tikzpicture}[scale=0.8]
\tkzTabInit[nocadre, espcl=2,  color, colorV=lightgray!5, colorL=gray!15,
colorC=gray!15]
{$x$ /.6, $f'(x)$ /.9, Variaç. de $f$ /1.5}%
{,$-1$, $+1$,}%
%\tkzTabLine{+,z,h,z,+}
\tkzTabLine{,-,t,h,t,+,}
\tkzTabVar{+/,-H/,-/,+/,}
%\tkzTabLine{,\searrow,\text{mín.},h,\text{mín.},\nearrow,}
\end{tikzpicture}\end{bmlimage}
\end{center}
Assim, o gráfico é do tipo:
\begin{center}
\begin{bmlimage}\begin{tikzpicture}
\newcommand{\func}[1]{sqrt((#1)^2-1)}
\draw[ ->] (-2.5,0)--(2.5,0);
\draw[ ->] (0,-0.2)--(0,1.5);
\draw[thick, domain=-2:-1] plot (\x,{\func{\x}});
\draw[thick, domain=1:2] plot (\x,{\func{\x}});
\foreach \k in {-1,+1} {
\draw (\k,0) node[below]{$\k$};
}
\end{tikzpicture}\end{bmlimage}
\end{center}
Observe que $\lim_{x\to -1^-}f'(x)=-\infty$, $\lim_{x\to +1^+}f'(x)=+\infty$
\eqref{itestudfunceleme5}:
Considere $f(x)=\frac{x+1}{x+2}$. Como
$\lim_{x\to \pm\infty}f(x)=1$, $y=1$ é assíntota horizontal, e como $\lim_{x\to
-2^-}f(x)=+\infty$, $\lim_{x\to -2^+}f(x)=-\infty$, $x=-2$ é assíntota vertical.
Como $f'(x)=\frac{1}{(x+2)^2}>0$ para todo $x\neq 2$, $f$ é crescente em
$(-\infty,-2)$ e em $(-2,\infty)$. Isso permite montar o gráfico:
\begin{center}
\begin{bmlimage}\begin{tikzpicture}[scale=0.7]
\draw[ ->] (-5,0)--(4,0);
\draw[dashed] (-2,-1)node[left]{$\scriptstyle{x=-2}$}--(-2,3);
\draw[dashed] (-5,1)--(4,1) node[above]{$\scriptstyle{y=1}$};
\draw[ ->] (0,-1)--(0,2.5);
\pgfmathsetmacro{\e}{0.5};
\draw[thick, domain=-5:{-2-\e}, samples=50] plot (\x,{(\x+1)/(\x+2)});
\draw[thick, domain={-2+\e}:4, samples=50] plot (\x,{(\x+1)/(\x+2)});
\end{tikzpicture}\end{bmlimage}
\end{center}
\eqref{itestudfunceleme61}: Um estudo parecido dá
\begin{center}
\begin{bmlimage}\begin{tikzpicture}[scale=0.7]
\draw[ ->] (-4,0)--(4,0);
\draw[dashed] (0.5,-2)node[right]{$\scriptstyle{x=\tfrac{1}{2}}$}--(0.5,1.5);
\draw[dashed] (-4,-0.5)--(4,-0.5) node[below]{$\scriptstyle{y=\tfrac{1}{2}}$};
\draw[ ->] (0,-2)--(0,1.5);
\pgfmathsetmacro{\e}{0.16};
\draw[thick, domain=-4:{0.5-\e}, samples=50] plot (\x,{(\x-1)/(1-2*\x)});
\draw[thick, domain={0.5+\e}:4, samples=50] plot (\x,{(\x-1)/(1-2*\x)});
\end{tikzpicture}\end{bmlimage}
\end{center}
\eqref{itestudfunceleme7}: Como $f'(x)=-xe^{-\frac{x^2}{2}}$,
$f$ é crescente em $(-\infty,0]$, decrescente em $[0,\infty)$.
Como $f(x)\to 0$ quando $x\to \pm \infty$, temos:
\begin{center}
\begin{bmlimage}\begin{tikzpicture}[scale=0.7]
\draw[ ->] (-4,0)--(4,0);
\draw[ ->] (0,-0.2)--(0,1.3);
\draw[thick, domain=-4:4, samples=50] plot (\x,{exp(-\x*\x*0.5)});
\end{tikzpicture}\end{bmlimage}
\end{center}
\eqref{itestudfunceleme9}: Observe que $\ln(x^2)$ tem domínio
$D=\bR\setminus\{0\}$, e $(\ln(x^2))'=\frac{2}{x}$. Logo, $\ln(x^2)$ é
decrescente em $(-\infty,0)$, crescente em $(0,\infty)$:
\begin{center}
\begin{bmlimage}\begin{tikzpicture}[scale=0.5]
\draw[ ->] (-4,0)--(4,0);
\draw[ ->] (0,-2.5)--(0,2);
\draw[thick, domain=0.3:4, samples=50] plot (\x,{2*ln(\x)});
\draw[thick, domain=0.3:4, samples=50] plot (-\x,{2*ln(\x)});
\end{tikzpicture}\end{bmlimage}
\end{center}
\eqref{itestudfunceleme10}
Lembre que o domínio da tangente é formado pela união dos intervalos da forma
$I_k=]-\pisobredois+k\pi,\pisobredois+k\pi[$.
Como $(\tan x)'=1+\tan^2x>0$ para todo $x\in I_k$, $\tan x$ é crescente em cada
intervalo do seu domínio (veja o esboço na Seção \ref{Sec:GraficosTrigo}).
\end{Solution}
\begin{Solution}{6.33}
  Em $t=0$, a partícula está na origem, onde ela fica até o instante
$t_1$. Durante $[t_1,t_2]$, ela anda em direção ao ponto $x=d_1$, com
velocidade constante $v=\frac{d_1}{t_2-t_1}$ e aceleração $a=0$. No tempo $t_2$
ela chega em $d_1$
e fica lá até o tempo $t_3$. No tempo $t_3$ ela começa a andar em direção ao
ponto $x=d_2$ (isto é, ela \emph{recua}), com velocidade constante
$v=\frac{d_2-d_1}{t_4-t_3}<0$. Quando chegar em $d_1$ no tempo $t_4$, para, fica
lá até $t_5$. No tempo $t_5$, começa a acelerar com uma aceleração $a>0$, até
o tempo $t_6$.
\end{Solution}
\begin{Solution}{6.34}
Como $v(t)=t-1$, temos $v(0)=-1<0$, $v(1)=0$, $v(2)=1>0$, $v(10)=9$.
Quando $t\to \infty$, $v(t)\to\infty$.
Observando a partícula, significa que no tempo $t=0$ ela está em
$x(0)=0$, recuando com uma velocidade de $-1$ metros por segundo. No instante
$t=1$, ela está com velocidade nula em $x(1)=-\frac12$. No instante $t=2$ ela
está de volta em $x(2)=0$, mas dessa vez com uma velocidade de $+1$ metro por
segundo.
A aceleração é \emph{constante}: $a(t)=v'(t)=+1$.
\end{Solution}
\begin{Solution}{6.35}
Temos $v(t)=x'(t)=A\omega \cos(\omega t)$, e $a(t)=v'(t)=-A\omega^2\sen (\omega
t)\equiv -\omega^2 x(t)$.
\begin{center}
\begin{bmlimage}\begin{tikzpicture}
\pgfmathsetmacro{\o}{1};
\pgfmathsetmacro{\A}{1};
\pgfmathsetmacro{\l}{12.7};
\pgfmathsetmacro{\omeg}{1};
\draw[ ->] (0,0)--(\l,0);
\draw[ ->] (0,-\A-0.2)--(0,\A+0.3);
\draw[thick, domain=0:\l-1.8, samples=80] plot (\x,{\A*sin(\omeg*\x r)})
node[right]{$x(t)$};
\draw[dashed, domain=0:\l-0.5, samples=80] plot (\x,{\A*\omeg*cos(\omeg*\x r)})
node[right]{$v(t)$};
\draw[dotted, domain=0:\l-2, samples=80] plot (\x,{-\A*\omeg^2*sin(\omeg*\x
r)})
node[right]{$a(t)$};
\foreach \k in {1,2,3} {
\draw ({\k*3.1414/\omeg},0) node{$\shortmid$} node[above]{$\frac{\k
\pi}{\omega}$};
}
\end{tikzpicture}\end{bmlimage}
\end{center}
Observe que $v(t)$ é máxima quando $x(t)=0$, e é mínima quando $x(t)=\pm A$.
Por sua vez, $a(t)$ é nula quando $x(t)=0$ e máxima quando $x(t)=\pm A$.
\end{Solution}
\begin{Solution}{6.36}
A taxa de variação no mês $t$ é dada por $P'(t)=2t+20$. Logo, hoje,
$P'(0)=+20$ hab./mês, o que significa que a população hoje cresce a medida de
$20$ habitantes por mês. Daqui a $15$ meses, $P'(15)=+50$ hab./mês. A
variação real da população durante o $16$-ésimo mês será $P(16)-P(15)=+51$
habitantes.
\end{Solution}
\begin{Solution}{6.37}
Como $V=L^3$, $V'=3L^2L'=\frac32 L^2$.
Logo, quando $L=10$, $V'=150$ $m^3/s$, e quando
$L=20$, $V'=600$ $m^3/s$.
\end{Solution}
\begin{Solution}{6.38}
O volume do balão no tempo $t$ é dado por $V(t)=\tfrac43 \pi R(t)^3$.
Logo, $R(t)=(\frac{3}{4\pi}V(t))^{1/3}$, e pela regra da cadeia,
$R'(t)=\tfrac13(\frac{3}{4\pi}V(t))^{-2/3}\frac{3}{4\pi}V'(t)$.
No instante $t_*$ que interessa, $V(t_*)=\frac{4\pi}{3}m^3$, e como
$V'(t)=2m^3/s$ para todo $t$, obtemos
$$
R'(t_*)=\tfrac13(\frac{3}{4\pi}\frac{4\pi}{3})^{-2/3}\frac{3}{4\pi}2\,m/s=\frac{
1}{2\pi}m/s\,.
$$
\end{Solution}
\begin{Solution}{6.39}
Seja $x$ a distância de $I$ até a parede, e $y$ a distância de $S$ até o chão:
$x^2+y^2=4$. Quando a vassoura começa a escorregar, $x$ e $y$ ambos se
tornam funções do tempo: $x=x(t)$ com $x'(t)=0.8\,m/s$, e $y=y(t)$. Derivando
implicitamente com respeito a $t$,
$2xx'+2yy'=0$. Portanto,
$y'=-\frac{xx'}{y}=-0.8\frac{x}{y}=-\frac{0.8x}{\sqrt{4-x^2}}$.
1) Quando $x=1\,m$, $y'=-0.46\,m/s$ (da onde vém esse sinal ``-''?)
2) Quando $x\to 2^-$, $y'\searrow -\infty$.
Obs: Quando $I$ estiver a $2-7.11\cdot 10^{-22}\,m$ da parede,
$S$ ultrapassa a velocidade da luz.
\end{Solution}
\begin{Solution}{6.40}
Definamos $\theta$ e $x$ da seguinte maneira:
\begin{center}
\begin{bmlimage}\begin{tikzpicture}
\draw (-5,0)--(5,0);
\pgfmathsetmacro{\teta}{60};
\pgfmathsetmacro{\h}{2};
\pgfmathsetmacro{\p}{-\h*tan(\teta)};
\fill (\p,0) circle (0.50mm);
\draw (\p,0) node[above]{$P$};
\draw[thick, ->] (\p,0)--(\p-0.4,0);
\draw[dotted] (0,0)--(0,\h) node[right]{$L$};
\draw[dashed] (\p,0)--(0,\h);
\draw[->] (0,\h-0.8) arc (270:270-\teta:0.8);
\draw (-0.5,1.05) node{$\theta$};
\draw[decorate, decoration=brace] (0,-0.2)--(\p,-0.2) node[midway, below]{$x$};
\pgfmathsetmacro{\e}{0.2};
\draw (0,0) node[above right]{$A$};
\pgfmathsetmacro{\f}{\e*sin(\teta)};
\pgfmathsetmacro{\g}{\e*cos(\teta)};
\draw[line width=4pt] (-\f,\h-\g)--(\f,\h+\g);
\end{tikzpicture}\end{bmlimage}
\end{center}
Temos $\tan \theta=\frac{x}{10}$ e como $\theta'=0.5$ rad/s, temos
$x'=10(1+\tan^2\theta)\theta'=5(1+\tan^2\theta)$.
1) Se $P=A$, então $\tan \theta=0$, logo $x'=5$ m/s. 2) Se $x=10\,m$, então
$\tan \theta=1$ e $x'=10\,m/s$.
3) Se  $x=100\,m$, então $\tan \theta=10$ e $x'=505\,m/s$ (mais rápido que a
velocidade do som, que fica em torno de $343\, m/s$).
\end{Solution}
\begin{Solution}{6.41}
Seja $H$ a altura do balão e $\theta$ o ângulo sob o qual o observador vê o
balão. Temos $H'=5$, e $\tan \theta=\frac{H}{50}$. Como ambos $H$ e
$\theta$ dependem do tempo, ao derivar com respeito a $t$ dá
$(1+\tan^2\theta)\theta'=\frac{H'}{50}=\frac{1}{10}$, isto é:
$\theta'=\frac{1}{10(1+\tan^2\theta)}$.
1) No instante em que o balão estiver a $30$ metros do chão, $\tan
\theta=\frac{30}{50}=\tfrac35$, assim $\theta'=\frac{5}{68}\simeq 0.0735$
rad/s.
2) No instante em que o balão estiver a $1000$ metros do chão, $\tan
\theta=\frac{1000}{50}=20$, assim $\theta'=\frac{1}{4010}\simeq 0.0025$ rad/s.
\end{Solution}
\begin{Solution}{6.42}
Como $P=\frac{nkT}{V}$, $P'=-\frac{nkT}{V^2}V'$. Logo,
no instante em que $V=V_0$,
$P'=-\frac{3nkT}{V_0^2}$.
\end{Solution}
\begin{Solution}{6.43}
\eqref{itexolinearizac1} $f(x)\simeq x+1$, $f(x)\simeq e^{-1}x+2e{^-1}$
\eqref{itexolinearizac2} $f(x)\simeq x$,
\eqref{itexolinearizac3} $f(x)\simeq -x$,
\eqref{itexolinearizac4} $f(x)\simeq 1$,
\eqref{itexolinearizac5} $f(x)\simeq x$, $f(x)\simeq 1$, $f(x)\simeq -x+\pi$
\eqref{itexolinearizac6} $f(x)\simeq 1+\frac{x}{2}$.
\end{Solution}
\begin{Solution}{6.44}
Como $\sqrt{4+x}\simeq 2+\frac{x}{4}$, temos $\sqrt{3.99}=\sqrt{4-0.01}\simeq
2+\frac{-0.01}{4}=1.9975$ (HP: $\sqrt{3.99}=1.997498...$).
Como $\ln (1+x)\simeq x$, temos $\ln(1.0123)=\ln(1+0.123)\simeq 0.123$ (HP:
$\ln(1.123)=0.1160...$).
Como $\sqrt{101}=10\sqrt{1+\frac{1}{100}}$ e que $\sqrt{1+x}\simeq
1+\frac{x}{2}$, temos $\sqrt{101}\simeq 10\cdot(1+\frac{1/100}{2})=10.05$ (HP:
$\sqrt{101}=10.04987...$).
\end{Solution}
\begin{Solution}{6.45}
\eqref{itderivimplicit1} $y'=\frac{3\cos(3x+y)}{1-\cos(3x+y)}$.
\eqref{itderivimplicit2} $y'=\frac{2xy^3+3x^2y^2}{1-3x^2y^2-2x^3y}$
\eqref{itderivimplicit3} Atenção: o único par $(x,y)$ solução da
equação $x=\sqrt{x^2+y^2}$ é $(0,0)$! Logo, não há jeito de escrever $y$ como
\emph{função} de $x$, assim não faz sentido derivar com respeito a $x$.
\eqref{itderivimplicit4} $y'=\frac{1-3x^2-4x-y}{3y^2+x+2}$
\eqref{itderivimplicit5} $y'=\frac{-\sen x-x\cos x}{\cos y-y\sen y}$
\eqref{itderivimplicit6} $y'=\frac{\cos y-\cos(x+y)}{x\sen y+\cos(x+y)}$
\end{Solution}
\begin{Solution}{6.46}
\eqref{itderivimplicitB1} Com $y'=1-\frac{2x}{3(y-x)^2}$,  $y=\frac56
x+\frac{13}{6}$.
\eqref{itderivimplicitB2} Com $y'=\frac{2-2xy}{x^2+4y^3}$, $y=\frac45
x+\frac95$.
\eqref{itderivimplicitB3} $y=-x+2$.
Obs: curvas definidas implicitamente por equações do tipo acima podem ser
representadas usando qualquer programa simples de esboço de funções, por exemplo
\verb|kmplot|.
\end{Solution}
\begin{Solution}{6.47}
\eqref{itfonctionsconvexes1} Queremos verificar que $\sqrt{\frac{x+y}{2}}\geq
\frac{\sqrt{x}+\sqrt{y}}{2}$ para todo $x,y\geq 2$.
Elevando ambos lados ao quadrado (essa operação é permitida, já que ambos
lados são positivos), $\frac{x+y}{2}\geq
(\frac{\sqrt{x}+\sqrt{y}}{2})^2
=\frac{x+2\sqrt{x}\sqrt{y}+y}{4}$, e rearranjando os termos obtemos $0\leq
\frac{(\sqrt{x}-\sqrt{y})^2}{4}$, que é sempre verdadeira.
\eqref{itfonctionsconvexes2}
Se $x,y>0$, $\frac{1}{\frac{x+y}{2}}\leq \frac{\frac{1}{x}+\frac{1}{y}}{2}$ é
equivalente a $4xy\leq (x+y)^2$, que por sua vez é equivalente a $0\leq
(x-y)^2$, que é sempre verdadeira. Logo, $\frac1x$ é convexa em $(0,\infty)$.
Como $\frac1x$ é ímpar, a concavidade em $(-\infty,0)$ segue imediatamente.
\end{Solution}
\begin{Solution}{6.48}
\eqref{itexconvexidadeA1}
$\frac{x^3}{3}-x$ é côncava em $(-\infty,0]$, convexa em $[0,\infty)$.
O gráfico se encontra na solução do Exercício \ref{Ex:variacoesbasicas}.
\eqref{itexconvexidadeA2} $-x^3+5x^2-6x$ é convexa em $(-\infty,\tfrac53]$,
côncava em $[\frac53,\infty)$:
\begin{center}
\begin{bmlimage}\begin{tikzpicture}[scale=0.5]
\draw[ ->] (-1,0)--(3.6,0);
\draw[ ->] (0,-2)--(0,1.7);
\newcommand{\funcao}[1]{-1*(#1)^3+5*(#1)^2-6*(#1)}
\draw[thick, domain=-0.1:3.4] plot (\x,{\funcao{\x}});
\pgfmathsetmacro{\a}{5/3};
\coordinate (I) at (\a,{\funcao{\a}});
\draw[dotted] (\a,0)node[above]{$\scriptstyle{\tfrac53}$}--(I);
\fill (I) circle (0.70mm);
\end{tikzpicture}\end{bmlimage}
\end{center}
\eqref{itexconvexidadeA3} Se $f(x)=3x^4-10x^3-12x^2+10x+9$, então
$f''(x)=12(3x^2-5x-2)$.
Logo, $f(x)$ é convexa em $(-\infty,-\tfrac13]$ e em $[2,\infty)$,
côncava em $[-\tfrac13,2]$.
\begin{center}
\begin{bmlimage}\begin{tikzpicture}[scale=0.7]
\draw[ ->] (-1,0)--(3.6,0);
\draw[ ->] (0,-1)--(0,1.7);
\newcommand{\funcao}[1]{(3*(#1)^4-10*(#1)^3-12*(#1)^2+10*(#1)+9)/100}
\draw[thick, domain=-2:4.5, samples=50] plot (\x,{\funcao{\x}});
\pgfmathsetmacro{\a}{-0.333};
\coordinate (I) at (\a,{\funcao{\a}});
\draw[dotted] (\a,0)node[below]{$\scriptstyle{-\tfrac13}$}--(I);
\fill (I) circle (0.70mm);
\pgfmathsetmacro{\b}{2};
\coordinate (J) at (\b,{\funcao{\b}});
\draw[dotted] (\b,0)node[above]{$\scriptstyle{2}$}--(J);
\fill (J) circle (0.70mm);
\end{tikzpicture}\end{bmlimage}
\end{center}
\eqref{itexconvexidadeA4} Como $(\frac{1}{x})''=\frac{2}{x^3}$, $\frac{1}{x}$ é
côncava em $(-\infty,0)$, convexa em $(0,\infty)$ (confere no gráfico do
Capítulo \ref{Cap:Funcoes}).
\eqref{itexconvexidadeA5}: Como $f''(x)=(x+2)e^x$, $f$ é côncava em
$(-\infty, -2]$, convexa em $[-2,\infty)$:
\begin{center}
\begin{bmlimage}\begin{tikzpicture}[scale=0.7]
\draw[ ->] (-4,0)--(2,0);
\draw[ ->] (0,-0.6)--(0,1.7);
\newcommand{\funcao}[1]{(#1)*exp(#1)}
\draw[thick, domain=-4:0.8, samples=50] plot (\x,{\funcao{\x}});
 \pgfmathsetmacro{\a}{-2};
 \coordinate (I) at (\a,{\funcao{\a}});
 \draw[dotted] (\a,0)node[above]{$\scriptstyle{-2}$}--(I);
 \fill (I) circle (0.70mm);
\end{tikzpicture}\end{bmlimage}
\end{center}
\eqref{itexconvexidadeA6}:
$f(x)=\frac{x^2+9}{(x-3)^2}$ é bem definida em $D=(-\infty,3)\cup
(3,+\infty)$. Como $f''(x)=\frac{12(x+6)}{(x-3)^4}$, $f(x)$ é côncava em
$(-\infty, -6]$, convexa em $(-6,3)$ e $(3,+\infty)$:
\begin{center}
\begin{bmlimage}\begin{tikzpicture}[scale=0.2]
 \newcommand{\funcao}[1]{( (#1)^2+ 9 )/( ( (#1) - 3)^2 )}
\draw[ ->] (-15,0)--(12,0);
\draw[ ->] (0,-0.6)--(0,12);
\draw[dashed] (3,0)node[below]{$\scriptstyle{3}$}--(3,12);
\draw[dashed] (-15,1)node[left]{$\scriptstyle{y=1}$}--(12,1);
\draw[thick, domain=-15:1.9, samples=50] plot (\x,{\funcao{\x}});
\draw[thick, domain=4.4:12, samples=50] plot (\x,{\funcao{\x}});
\pgfmathsetmacro{\a}{-6};
\coordinate (I) at (\a,{\funcao{\a}});
\draw[dotted] (\a,0)node[below]{$\scriptstyle{-6}$}--(I);
\fill (I) circle (3mm);
\end{tikzpicture}\end{bmlimage}
\end{center}
\eqref{itexconvexidadeA7} Com $f(x)=xe^{-3x}$ temos $f''(x)=(9x-6)e^{-3x}$.
Logo, $f$ é côncava em $(-\infty,\tfrac23]$, convexa em $[\tfrac23,\infty)$:
\begin{center}
\begin{bmlimage}\begin{tikzpicture}[scale=0.7]
 \newcommand{\funcao}[1]{ 5*(#1)*exp(-3*(#1))}
\draw[ ->] (-2,0)--(3,0);
\draw[ ->] (0,-0.6)--(0,2);
% \draw[dashed] (3,0)node[below]{$\scriptstyle{3}$}--(3,12);
% \draw[dashed] (-15,1)node[left]{$\scriptstyle{y=1}$}--(12,1);
\draw[thick, domain=-0.1:2, samples=50] plot (\x,{\funcao{\x}});
%\draw[thick, domain=4.4:12, samples=50] plot (\x,{\funcao{\x}});
\pgfmathsetmacro{\a}{0.66666};
\coordinate (I) at (\a,{\funcao{\a}});
\draw[dotted] (\a,0)node[below]{$\scriptstyle{\tfrac23}$}--(I);
\fill (I) circle (1mm);
\end{tikzpicture}\end{bmlimage}
\end{center}
\eqref{itexconvexidadeA10} $f(x)=|x|-x$ é $=0$ se $x\geq 0$, e $=-2x$ se
$x\leq 0$. Logo, $f$ é convexa. Obs: como $|x|$ não é derivável em $x=0$, a
convexidade não pode ser obtida com o Teorema \ref{Teo:Sinalfseconde}.
\eqref{itexconvexidadeA11} Se $f(x)=\arctan x$, então $f'(x)=\frac{1}{x^2+1}$,
e $f''(x)=\frac{-2x}{(x^2+1)^2}$. Logo, $\arctan x$ é convexa em $]-\infty,0]$,
côncava em $[0,\infty)$ (confere no gráfico da Seção
\ref{Sec:Functriginversas}).
\eqref{itexconvexidadeA12} $f(x)=e^{-\frac{x^2}{2}}$ tem
$f''(x)=(x^2-1)e^{-\frac{x^2}{2}}$. Logo, $f$ é convexa em $]-\infty,1]$ e
$[1,\infty)$, e côncava em $[-1,1]$ (veja o gráfico do Exercício
\ref{Ex:variacoesbasicas}).
\eqref{itexconvexidadeA13} $f(x)=\frac{1}{x^2+1}$ é convexa em
$(-\infty,-\frac{1}{\sqrt{3}}]$ e $[\frac{1}{\sqrt{3}},\infty)$, côncava em
$[-\frac{1}{\sqrt{3}},\frac{1}{\sqrt{3}}]$.
\begin{center}
\begin{bmlimage}\begin{tikzpicture}[scale=0.7]
 \newcommand{\funcao}[1]{ 1/( (#1)^2 +1)}
\draw[ ->] (-3,0)--(3,0);
\draw[ ->] (0,-0.6)--(0,1.3);
% \draw[dashed] (3,0)node[below]{$\scriptstyle{3}$}--(3,12);
% \draw[dashed] (-15,1)node[left]{$\scriptstyle{y=1}$}--(12,1);
\draw[thick, domain=-2.5:2.5, samples=50] plot (\x,{\funcao{\x}});
%\draw[thick, domain=4.4:12, samples=50] plot (\x,{\funcao{\x}});
\pgfmathsetmacro{\a}{0.5777};
\coordinate (I) at (\a,{\funcao{\a}});
\draw[dotted] (\a,0)node[below]{$\scriptstyle{\tfrac{1}{\sqrt{3}}}$}--(I);
\fill (I) circle (0.5mm);
\coordinate (J) at (-\a,{\funcao{-\a}});
\draw[dotted] (-\a,0)node[below]{$\scriptstyle{-\tfrac{1}{\sqrt{3}}}$}--(J);
\fill (J) circle (0.5mm);
\end{tikzpicture}\end{bmlimage}
\end{center}
\end{Solution}
\begin{Solution}{6.49}
Nos dois primeiros e último exemplos, as hipóteses do Teorema \ref{Teo:BH1} são
verificadas, dando
 $$\lim_{s\to 0}\frac{\log(1+s)}{e^{2s}-1}=
\frac{(\log(1+s))'|_{s=0}}{(e^{2s})'|_{s=0}}
=\frac{\frac{1}{1+s}|_{s=0}}{2e^{2s}|_{s=0}}=\frac{1}{2}$$
$$
\lim_{t\to \pi}\frac{\cos t+1}{\pi-t}=-(\cos t)'|_{t=\pi}=\sen t|_{t=0}=0\,.$$
$$
\lim_{x\to 0}\frac{\sen x}{x^2+3x}=\frac{(\sen
x)'|_{x=0}}{(x^2+3x)'|_{x=0}}
=\frac{\cos 0}{2\cdot 2+3}=\frac{1}{3}\,.
$$
No terceiro, o teorema não se aplica: apesar das funções $1-\cos(\alpha)$ e
$\sen(\alpha+\frac{\pi}{2})$ serem deriváveis em $\alpha=0$, temos
$\sen (0+\pi/2)=1\neq 0$. Logo o limite se calcula sem a regra de B.H.:
$\lim_{\alpha\to 0}\frac{1-\cos(\alpha)}{\sen (\alpha+\pi/2)}=\tfrac01=0$.
\end{Solution}
\begin{Solution}{6.50}
\eqref{itexBH1} $0$ (B.H. não se aplica)
\eqref{itexBH2} $\tfrac37$
\eqref{itexBH2b} $+\infty$ (B.H. não se aplica)
\eqref{itexBH3} $\lim_{x\to 0}\frac{(\sen x)^2}{x^2}=(\lim_{x\to 0}\frac{\sen
x}{x})^2=1^2=1$ (não precisa de B.H.)
\eqref{itexBH4} Usando B.H.,  $\lim_{x\to 0}\frac{\ln\frac{1}{1+x}}{\sen x}
=-\lim_{x\to 0}\frac{\ln(1+x)}{\sen x}=-\lim_{x\to 0}\frac{\frac{1}{x+1}}{\cos
x}=-1$.
\eqref{itexBH14} $1$
\eqref{itexBH7} $0$
\eqref{itexBH5} $0$
\eqref{itexBH10} $-\frac{1}{6}$
\eqref{itexBH9} $\tfrac13$
\eqref{itexBH922} $1$
\eqref{itexBH11} $2$
\eqref{itexBH12} $0$ (B.H. não se aplica)
\eqref{itexBH12aa} $0$
\eqref{itexBH12a} $0$ (aplicando duas vezes B.H.)
\eqref{itexBH12ab} $0$
\eqref{itexBH12b} Como $e^{\ln x}=x$, o limite é $1$ (B.H. se aplica mas não
serve para nada!)
\eqref{itexBH12c} Esse limite se calcula como no Capítulo \ref{Cap:Limites}:
$\lim_{x\to \infty}\frac{\sqrt{x+1}}{\sqrt{x-1}}=
\lim_{x\to \infty}\frac{\sqrt{x}\sqrt{1+\frac1x}}{\sqrt{x}\sqrt{1-\frac1x}}=
1$.
\eqref{itexBH12d} $-1/3$ (sem B.H.!)
\eqref{itexBH13} $2$
\eqref{itexBH15} $0$ (B.H. não se aplica)
\eqref{itexBH16} $\lim_{x\to \infty}\frac{x+\sen x}{x}=\lim_{x\to
\infty}(1+\frac{\sen x}{x})=1+0=1$ (Obs: Aqui B.H. não se aplica, porqué
$\lim_{x\to \infty}\frac{(x+\sen x)'}{(x)'}=\lim_{x\to
\infty}(1+\cos x)$, que  não existe.)
\eqref{itexBH6} $\tfrac13$
\eqref{itexBH17} $\lim_{x\to 0^+}\frac{x^2 \sen \frac{1}{x}}{x}=
\lim_{x\to 0^+} x\sen \frac{1}{x}=0$, com um ``sanduíche''. Aqui B.H. não se
aplica, porqué o limite $\lim_{x\to 0^+}(x^2 \sen \frac{1}{x})'$ não existe.
\eqref{itexBH18} $\frac13$. \eqref{itexBH20} (Segunda prova, Segundo semestre de
2011) Como $\lim_{y \to
\infty}\arctan y=\frac{\pi}{2}$, o limite
é da forma $\frac00$. As funções são deriváveis em $x>0$, logo pela
regra de B.H.,
$$
\lim_{x\to 0^+}\frac{\arctan(\frac1x)-\tfrac{\pi}{2}}{x}=
\lim_{x\to 0^+}\frac{\frac{1}{1+(\frac{1}{x})^2}(-\frac{1}{x^2})}{1}=
\lim_{x\to 0^+}\frac{-1}{1+x^2}=-1\,.
$$
\eqref{itexBHww8} $1/2$.
\end{Solution}
\begin{Solution}{6.51}
\eqref{itexBHB1} $\sqrt{e}$
\eqref{itexBHB2} $\lim_{x\to 0^+}x^x=\exp(\lim_{x\to 0^+}x\ln x)=e^0=1$.
\eqref{itexBHB3} $e^2$
\eqref{itexBHB5} $1$
\eqref{itexBHB58} $e$
\eqref{itexBHB6} $1$
\eqref{itexBHB7} $1$
\eqref{itexBHB75} $1$
\eqref{itexBHB8} $e^{-1}$
\eqref{itexBHB2bis} $0$
\eqref{itexBHB4} $-e/2$
\end{Solution}
\begin{Solution}{6.52}
 Para o primeiro,
\begin{align*}
 \lim_{z\to \infty}\bigl(\frac{z+9}{z-9}\bigr)^z&=\exp \Bigl(\lim_{z\to \infty}
z \ln \frac{z+9}{z-9}\Bigr)\\
&=\exp \Bigl(\lim_{z\to \infty} \frac{\ln (z+9)-\ln
(z-9)}{\frac{1}{z}}\Bigr)\text{ e as hipót. de BH satisfeitas, logo}\\
&=\exp \Bigl(\lim_{z\to \infty}
\frac{\frac{1}{z+9}-\frac{1}{z-9}}{\frac{-1}{z^2}}\Bigr)\\
&=\exp \Bigl(\lim_{z\to \infty} \frac{18 z^2}{z^2-81}\Bigr)\\
&=e^{18}\,.
\end{align*}
Para o segundo,
\begin{align*}
 \lim_{x\to \infty}x^{\ln x}e^{-x}&=\exp \Bigl(\lim_{x\to \infty} \big((\ln
x)^2-x \big)\Bigr)
=\exp \Bigl(\lim_{x\to \infty} x\big(\frac{(\ln x)^2}{x}-1
\big)\Bigr)
\end{align*}
Usando BH duas vezes, verifica-se que $\lim_{x\to \infty}\frac{(\ln
x)^2}{x}=0$,
o que implica $\lim_{x\to \infty} x(\frac{(\ln
x)^2}{x}-1)=-\infty$.
Logo, $\lim_{x\to \infty}x^{\ln x}e^{-x}=0$.
O último limite se calcula sem usar B.H.:
$$\lim_{x\to \infty}\frac{\sqrt{2x+1}}{\sqrt{x-1000}}=\sqrt{2}\lim_{x\to
\infty}\frac{\sqrt{1+\frac{1}{2x}}}{\sqrt{1-\frac{1000}{x}}}=\sqrt{2}\frac{1}{1}
=\sqrt{2}\,.$$
\end{Solution}
\protect \section *{Capítulo \ref {Cap:Extremos}}
\begin{Solution}{7.1}
\eqref{itminmaxbasico1} As hipóteses do teorema não são satisfeitas, pois o
domínio não é um intervalo finito e fechado. Mesmo assim, qualquer $x\in \bR$ é
ponto de máximo e mínimo global ao mesmo tempo.
\eqref{itminmaxbasico101} As hipóteses não são satisfeitas: o intervalo
não é limitado. Tém um
mínimo global em $x=1$, não tem máximo global.
\eqref{itminmaxbasico10} Hipóteses não satisfeitas (domínio não limitado).
Máximo global em $x=0$, não tem mínimo global.
\eqref{itminmaxbasico2} Hipóteses não satisfeitas (o intervalo não é fechado).
Tém mínimo global em $x=2$, não tem máximo global.
\eqref{itminmaxbasico3} Hipóteses satisfeitas: mínimo global em $x=2$, máximos
globais em $x=0$ e $x=2$.
\eqref{itminmaxbasico4} Hipóteses satisfeitas:
mínímos globais em $1,-1$ e $0$, máximos globais em
$-\tfrac32$ e $\tfrac32$.
\begin{center}
\begin{bmlimage}\begin{tikzpicture}
\pgfmathsetmacro{\a}{1.5};
\draw [thick, domain=-\a:\a, samples=150] plot (\x,{abs(1-\x^2)+abs(\x)-1});
\pgfmathsetmacro{\x}{0.6};
\draw [ ->] (-\a-0.2,0)--(\a+0.2,0);
\draw [ ->] (0,-0.2)--(0,2);
\foreach \k in {-1.5,1.5}{
\draw (\k,0) node{$\shortmid$};
\draw[dotted] (\k,0)--(\k,{abs(1-\k^2)+abs(\k)-1});
\fill (\k,{abs(1-\k^2)+abs(\k)-1}) circle (0.45mm);
}
\draw (-1.5,0) node[below]{$-\tfrac32$};
\draw (1.5,0) node[below]{$\tfrac32$};
\end{tikzpicture}\end{bmlimage}
\end{center}
\eqref{itminmaxbasico5} Hipóteses satisfeitas: mínimos globais em $x=-2$ e
$+1$, máximos globais em $x=-1$ e $+2$.
\eqref{itminmaxbasico6} Hipóteses satisfeitas: mínimo global em $x=+1$, máximo
global em $x=-1$.
\eqref{itminmaxbasico7} Hipóteses não satisfeitas ($f$ não é contínua). Não tem
máximo global, tem mínimos globais em $x=0$ e $+3$.
\eqref{itminmaxbasico8} Hipóteses satisfeitas: mínimo global em $x=0$, máximos
locais em $x=2$ e $4$.
\eqref{itminmaxbasico9} Hipóteses não satisfeitas ($f$ é contínua, mas o
domínio não é limitado). Tém mínimo global em $x=0$, não possui máximo global.
\eqref{itminmaxbasico11} Hipóteses não satisfeitas (intervalo não
limitado). No entanto, tem infinitos mínimos
globais, em todos os pontos da forma $x=-\pisobredois+k2\pi$, e infinitos
máximos globais, em todos os pontos da forma $x=\pisobredois+k2\pi$.
\end{Solution}
\begin{Solution}{7.2}
\eqref{itextremoslocais1} Máximo local no ponto $(-2,25)$, um mínimo local (e
global) em $(1,-2)$.
\eqref{itextremoslocais2} Sem mín./máx.
\eqref{itextremoslocais3} Mínimo local (e global) em
$(-1,-\frac{1}{12})$ (Atenção: a derivada é nula em $x=0$, mas não é nem
máximo nem mínimo pois a derivada não muda de sinal).
\eqref{itextremoslocais30} $f'(x)=-\frac{1-x^2}{x^2+x+1}$, tem um mínimo local
(em global) em $(1,f(1))$, um máximo local (e global) em $(-1,f(-1))$.
\eqref{itextremoslocais31} Máximo local (e global) em $(0,1)$.
\eqref{itextremoslocais32} Máximo local em $(1,e^{-1})$.
\eqref{itextremoslocais5} Mínimo local em $(-1,-\frac12)$, máximo local em
$(1,\frac12)$.
\eqref{itextremoslocais6} Mínimo local em $(e^{-1},e^{-1/e})$.
\eqref{itextremoslocais4} Máximo local em $(e^{-2}, 4e^{-2})$, mínimo local em
$(1,0)$.
\end{Solution}
\begin{Solution}{7.3}
$a=-b=3$.
\end{Solution}
\begin{Solution}{7.4}
\eqref{itLJ1} $r_0=\sigma$, \eqref{itLJ2} $r_*=\sqrt[6]{2}\sigma$.
Como $\lim_{r\to 0^+}V(r)=+\infty$, $V$ não possui máximo global.
$V$ decresce em $(0,r_*]$, cresce em $[r_*,\infty)$:
\begin{center}
\begin{bmlimage}\begin{tikzpicture}
\pgfmathsetmacro{\e}{1};
\pgfmathsetmacro{\s}{1};
\newcommand{\funcao}[1]{4*\e*(\s/(#1))^(12)-4*\e*(\s/(#1))^(6)}
\draw[->] (0,0)--(4.4,0)node[right]{$r$};
\draw[->] (0,-1)--(0,2)node[left]{$V(r)$};
\pgfmathsetmacro{\rzer}{\s};
\pgfmathsetmacro{\ret}{\s*1.122};
\draw[thick, domain=0.95:3.8, samples=50] plot (\x,{\funcao{\x}});
\draw[dotted] (\ret,0)node[above right]{$r_*$}--(\ret,{\funcao{\ret}});
\end{tikzpicture}\end{bmlimage}
\end{center}
Obs: O potencial de Lennard-Jones $V(r)$ descreve a energia de interação entre
dois átomos neutros a distância $r$.
Quando $0<r<r_0$ essa energia é positiva (os átomos se repelem), e quando
$r_0<r<\infty$ essa energia é negativa (os átomos se atraem).
Vemos que quando $r\to \infty$, a energia tende a zero e que ela tende a
$+\infty$ quando $r\to 0^+$: a distâncias longas, os átomos não interagem, e a
distâncias curtas a energia diverge (caroço duro).
A posição mais estável é quando a distância entre os dois átomos é
$r=r_*$.
\end{Solution}
\begin{Solution}{7.5}
 \eqref{itexoretanginscrito1}
A função área é dada por $A(x)=4x\sqrt{R^2-x^2}$, $x\in [0,R]$. O leitor pode
verificar que o seu máximo global em $[0,R]$ é atingido em
$x_*=\frac{R}{\sqrt{2}}$. Logo, o retângulo de maior área inscrito no círculo
tem largura $2x_*=\sqrt{2}R$, e altura $2\sqrt{R^2-x_*^2}=\sqrt{2}R$. Logo, é
um quadrado!
\eqref{itexoretanginscrito2} Usaremos a variável $h\in [0,4]$ definida da
seguinte maneira
\begin{center}
\begin{bmlimage}\begin{tikzpicture}[yscale=0.3]
\newcommand{\funcao}[1]{-2*(#1)+12}
\draw[ ->] (-0.2,0)--(6.5,0);
\draw[ ->] (0,-0.2)--(0,13);
\draw (-0.3,{\funcao{-0.3}})node[left]{$y=-2x+12$}--(7,{\funcao{7}});
\draw (-0.3,-0.3)--(5,5)node[right]{$y=x$};
\fill (4,4) circle (0.40mm);
\pgfmathsetmacro{\h}{2.2};
\draw[decorate, decoration=brace] (0,0)--(0,\h) node[midway, left]{$h$};
\draw[dotted] (0,\h)--(\h,\h);
\fill[color=gray!15] (\h,0) rectangle ({-\h/2+6},\h);
\draw[thick] (\h,0) rectangle ({-\h/2+6},\h);
\draw (\h,0) node[below]{$x_1$};
\draw ({-\h/2+6},0) node[below]{$x_2$};
\draw (4,4) node{$\bullet$};
% \fill (\h,\h) circle (0.50mm);
% \fill ({-\h/2+6},\h) circle (0.50mm);
\draw (4,4) node[above]{$(4,4)$};
\end{tikzpicture}\end{bmlimage}
\end{center}
A área do retângulo é dada por $A(h)=h(x_2-x_2)$. Ora, $x_1=h$ e
$x_2=6-\frac{h}{2}$. Logo, $x_2-x_1=6-\frac{3h}{2}$. Portanto,
queremos maximizar $A(h)=h(6-\frac{3h}{2})$ em
$h\in [0,4]$.
É fácil ver que o de máximo é atingido em $h_*=2$. Logo o maior retângulo tem
altura $h_*=2$, e largura $6-\frac{3h_*}{2}=3$.
\end{Solution}
\begin{Solution}{7.6}
A altura do triângulo de abertura
$\theta\in [0,\pi]$ é $\cos \frac{\theta}{2}$, a sua base é $2\sen
\frac{\theta}{2}$, logo a sua área é dada por
$$A(\theta)=\cos(\frac{\theta}{2})\sen (\frac{\theta}{2})=\frac12 \sen
\theta\,.\pt{3}$$
Queremos maximizar $A(\theta)$ quando $\theta\in [0,\pi]$.
Ora, $A(0)=A(\pi)=0$, e como $A'(\theta)=\frac12\cos \theta$,
$A'(\theta)=0$ se e somente se $\cos
\theta=0$, isto é, se e somente se $\theta=\frac{\pi}{2}$ $pt{1}$. Ora, como
$A'(\theta)>0$ se $\theta<\frac{\pi}{2}$,
$A'(\theta)<0$ se $\theta>\frac{\pi}{2}$, $\frac{\pi}{2}$ é um máximo de $A$
$\pt{2}$.
Logo, {o triângulo que tem maior área é aquele cuja abertura vale
$\frac{\pi}{2}$ $\pt{2}$.} Obs: pode também expressar a área em função do lado
horizontal $x$, $A(x)=\tfrac12 x\sqrt{1-(\tfrac{x}{2})^2}$.
Obs: Pode também introduzir a variável $h$, definida como
\begin{center}
\begin{bmlimage}\begin{tikzpicture}
\draw[thick] (0,0)--(1,0)node[midway,
below]{$\scriptstyle{1}$}--(1.5,0.866)node[midway,
below]{$\scriptstyle{1}$}--cycle;
\draw[dotted] (1.5,0)--(1.5,0.866) node[midway, right]{$h$};
\fill (1.5,0) circle (0.40mm);
%\draw (1,0.5) node{$\theta$};
\end{tikzpicture}\end{bmlimage}
\end{center}
e fica claro que o triângulo de maior área é aquele que tem maior altura $h$,
isto é, $h=1$ (aqui nem precisa calcular uma derivada...), o que acontece quando
a abertura vale $\frac{\pi}{2}$.
\end{Solution}
\begin{Solution}{7.7}
Seja $x$ o tamanho do lado horizontal do retângulo, e $y$ o seu lado vertical.
A área vale $A=xy$.
Como o perímetro é fixo e vale $2x+2y=L$, podemos expressar $y$ em função de
$x$, $y=\frac{L}{2}-x$, e expressar tudo em termos de $x$:
$A(x)=x(\frac{L}{2}-x)$. Maximizar essa função em $x\in [0,L/2]$ mostra que $A$
é máxima quando $x=x_*=\frac{L}{4}$. Como $y_*=\frac{L}{2}-x_*=\frac{L}{4}$, o
retângulo com maior área é um quadrado!
\end{Solution}
\begin{Solution}{7.8}
Suponha que a corda seja cortada em dois pedaços. Com o primeiro pedaço, de
tamanho $x\in [0,L]$, façamos um quadrado: cada um dos seus lados tem lado
$\frac{x}{4}$, e a sua área vale $(\frac{x}{4})^2$. Com o outro pedaço façamos
um círculo, de perímetro $L-x$, logo o seu raio é $\frac{L-x}{2\pi}$, e a sua
área $\pi(\frac{L-x}{2\pi})^2$. Portanto, queremos maximizar a função
$$
A(x)\pardef \frac{x^2}{16}+\frac{(L-x)^2}{4\pi}\,,\quad \text{ com }x\in
[0,L]\,.
$$
Na fronteira, $A(0)=\frac{L^2}{4\pi}$ (a corda inteira usada para fazer um
círculo), $A(L)=\frac{L^2}{16}$ (a corda inteira para fazer um quadrado).
Procuremos os pontos críticos de $A$: é fácil ver que $A'(x)=0$ se e somente
$x=x_*=\frac{L}{1+\frac{\pi }{4}}\in (0,L)$.
Como $A(x_*)=\frac{L^2}{4(4+\pi)}$, temos que $A(x_*)<A(L)<A(0)$. Logo,
a área total mínima é obtida fazendo um quadrado com o primeiro pedaço de
tamanho $x_*\simeq 0.56 L$, e  um círculo com o outro pedaço ($L-x_*\simeq
0.43 L$). A área total máxima é obtida usando a corda toda para fazer um
círculo.
\end{Solution}
\begin{Solution}{7.9}
$Q_*=(2,4)$
\end{Solution}
\begin{Solution}{7.10}
Seja $C=(x,0)$, com $1\leq x\leq 8$. É preciso minimizar
$f(x)=\sqrt{(x-1)^2+3^2}+\sqrt{(x-8)^2+4^2}$
para $x\in [1,8]$.
Os pontos críticos de $f$ são soluções de $7x^2+112x-560=0$ (em $[1,8]$), isto é,
$x=4$. Como $f''(4)>0$, $x=4$ é um mínimo de $f$ (pode verificar calculando os
valores $f(1)$, $f(8)$).
Logo, $C=(4,0)$ é tal que o perímetro de $ABC$ seja mínimo.
\end{Solution}
\begin{Solution}{7.11}
$\alpha=\pm 1$.
\end{Solution}
\begin{Solution}{7.12}
Considere a variável $x$ definida da seguinte maneira:
\begin{center}
\begin{bmlimage}\begin{tikzpicture}
\draw[ ->] (-0.2,0)--(4,0);
\draw[ ->] (0,-0.2)--(0,2);
\pgfmathsetmacro{\a}{1.5};
\pgfmathsetmacro{\b}{0.5};
\coordinate (P) at (\a,\b);
\pgfmathsetmacro{\d}{1.4};
\coordinate (Q) at (\a+\d,0);
\coordinate (Qp) at (0,{\b*(\d+\a)/\d});
\fill (P) circle (0.4mm);
\draw (P) node[above right]{$P=(a,b)$};
\fill (Q) circle (0.4mm);
\draw (Q) node[below]{$Q$};
\draw[dashed] (Q)--(Qp);
\draw[decorate, decoration=brace] (0,0)--(Qp) node[midway, left]{$h$};
\draw[decorate, decoration=brace] (\a,0)--(P) node[midway, left]{$b$};
\draw[decorate, decoration=brace] (\a,0)--(0,0) node[midway, below]{$a$};
\draw[decorate, decoration=brace] (Q)--(\a,0) node[midway, below]{$x$};
\end{tikzpicture}\end{bmlimage}
\end{center}
Assim temos que a área do triângulo em função de $x$, $A(x)$, é dada por
$A(x)=\half (a+x)\cdot h$. Mas, como $\frac{h}{a+x}=\frac{b}{x}$, temos
$h=\frac{b(x+a)}{x}$, que dá
$A(x)=\frac{b}{2}\frac{(x+a)^2}{x}$.
Procuremos o mínimo de $A(x)$ para $x\in (0,\infty)$.
Como $A$ é derivável em todo $x>0$, $A'(x)=\frac{b}{2}\frac{(x-a)(x+a)}{x^2}$,
vemos que $A$ possui dois pontos críticos, em $-a$ e $+a$, e $A'(x)>0$ se
$x<-a$, $A'(x)<0$ se $-a<x<a$, e $A'(x)>0$ se $x>a$. Desconsideremos o $-a$ pois
queremos um ponto em $(0,\infty)$. Assim, o mínimo de $A$ é atingido em $x=a$,
e nesse ponto $A(a)=2ab$:
\begin{center}
\begin{bmlimage}\begin{tikzpicture}
\pgfmathsetmacro{\a}{1.5};
\pgfmathsetmacro{\b}{0.5};
\draw[ ->] (0,0)--(4,0)node[right]{$x$};
\draw[ ->] (0,-0.2)--(0,2.2) node[left]{$A(x)$};
\draw[thick, domain=0.4:4] plot (\x,{(\b*(\x^2+2*\a*\x+\a^2))/(2*\x)});
\draw[dotted] (\a,0)node[below]{$a$}--(\a,{2*\a*\b})--(0,{2*\a*\b})
node[left]{$2ab$};
\fill (\a,{2*\a*\b}) circle (0.45mm);
\end{tikzpicture}\end{bmlimage}
\end{center}
\end{Solution}
\begin{Solution}{7.13}
 Representamos o triângulo da seguinte maneira:
\begin{center}
\begin{bmlimage}\begin{tikzpicture}
\pgfmathsetmacro{\r}{1};
\draw (0,0) circle(\r cm);
\draw[->] (-\r-0.2,0)--(\r+0.2,0);
\draw[->] (0,-\r-0.2)--(0,\r+0.2);
\coordinate (A) at (0,\r);
\coordinate (B) at (0.5*\r,-0.866*\r);
\coordinate (T) at (0,-0.866*\r);
\coordinate (C) at  (-0.5*\r,-0.866*\r);
\fill[color=gray!30, opacity=0.8] (A)--(B)--(C)--cycle;
\draw[thick] (A)--(B)--(C)--cycle;
\draw[decorate, decoration={brace, raise=1pt}] (B)--(T)
node[midway, below]{$x$};
\end{tikzpicture}\end{bmlimage}
\end{center}
Parametrizando o triângulo usando a variável $x$ acima (pode
também usar um ângulo),
obtemos a área como sendo a função
$A(x)=x(R+\sqrt{R^2-x^2})$, com $x\in [0,R]$.
Observe que não é necessário considerar os triângulos cuja
base fica acima do eixo $x$. (Por qué?)
Deixamos o leitor verificar que o máximo da função $A(x)$ é
atingido no ponto $x_*=\tfrac{\sqrt{3}}{2}R$, e que esse $x_*$
corresponde ao triângulo equilátero.
\end{Solution}
\begin{Solution}{7.14}
O único ponto crítico de $\sigma(x)$ é $x_*=\frac{x_1+\dots+ x_n}{n}$ (isto é,
a média aritmética). Como $\sigma''(x)=2n>0$, $x_*$ é mínimo global.
\end{Solution}
\begin{Solution}{7.15}
Seja $F$ a formiga, $S$ (respectivamente $I$) a extremidade superior
(respectivamente inferior) do telão, $\theta$ o ângulo $SFI$, e $x$ a distância
de $F$ à parede:
\begin{center}
\begin{bmlimage}\begin{tikzpicture}[yscale=0.3]
\draw (-1,0)--(0,0)--(0,10);
\coordinate (F) at (-7,0);
\coordinate (S) at (0,8);
\coordinate (I) at (0,3);
\draw[thick] (S)--(F)--(I);
\draw (F) node{$\bullet$} node[below]{$F$};
\draw (S) node[right]{$S$};
\draw (I) node[right]{$I$};
\draw (0,0) node[right]{$O$};
\draw[decorate, decoration=brace] (I)--(0,0) node[midway, right]{$3$};
\draw[decorate, decoration=brace] (S)--(I) node[midway, right]{$5$};
\draw[decorate, decoration=brace] (0,0)--(F) node[midway, below]{$x$};
\end{tikzpicture}\end{bmlimage}
\end{center}
Se $x$ é a distância de $F$ à parede, precisamos expressar $\theta$ em função
de $x$. Para começar, $\theta=\alpha-\beta$, em que $\alpha$ é o ângulo $SFO$,
e $\beta$ o ângulo $IFO$. Mas $\tan \alpha =\frac{8}{x}$ e $\tan
\beta=\frac{3}{x}$. Logo, precisamos achar o máximo da função
$$
\theta(x)=\arctan\tfrac{8}{x}-\arctan \tfrac{3}{x}\,,\quad \text{ com }x>0\,.
$$
Observe que $\lim_{x\to 0^+}\theta(x)=0$ (indo infinitamente perto da
parede, a formiga vê o telão sob um ângulo nulo) e $\lim_{x\to
\infty}\theta(x)=0$ (indo infinitamente longe da parede, a
formiga também vê o telão sob um ângulo nulo), é claro que deve existir (pelo
menos) um $0<x_*<\infty$ que maximize $\theta(x)$. Como $\theta$ é derivável,
procuremos os seus pontos críticos:
$$
\theta'(x)=\frac{1}{1+(\tfrac8x)^2}(\frac{-8}{x^2})
-\frac{1}{1+(\tfrac3x)^2}(\frac{-3}{x^2})=(\cdots)=\frac{120-5x^2}{
(x^2+8^2)(x^2+3^2)}\,.
$$
Logo o único ponto crítico de $\theta$ no intervalo $(0,\infty)$ é
$x_*=\sqrt{24}$. Vemos também que $\theta'(x)>0$ se $x<x_*$ e
$\theta'(x)<0$ se $x>x_*$, logo $x_*$ é o ponto onde $\theta$ atinge o seu
valor máximo.
Logo, para ver o telão sob um ângulo máximo, a formiga precisa ficar a uma
distância de $\sqrt{24}\simeq 4.9$ metros da parede.
\end{Solution}
\begin{Solution}{7.16}
Seja $R$ o raio da base do cone, $H$ a sua altura, $r$ o raio da base do
cilíndro e $h$ a sua altura.
Para o cilíndro ser inscrito, $\frac{h}{H}=\frac{R-r}{R}$ (para entender essa
relação, faça um desenho de um corte vertical).
Logo, expressando o volume do cilíndro em função de $r$, $V(r)=\frac{\pi
H}{R}r^2(R-r)$. É fácil ver que essa função possui um máximo local em $[0,R]$
atingido em $r_*=\frac{2}{3}R$. A altura do cilíndro correspondente é
$h_*=\frac{H}{3}$.
(Obs: pode também expressar $V$ em função de $h$: $V(h)=\pi
R^2h(1-\frac{h}{H})^2$.)
\end{Solution}
\begin{Solution}{7.17}
Seja $r$ o raio da base do cone, $h$ a sua altura.
O volume do cone é dado por $V=\tfrac13 \times \pi r^2\times h$. Como $h$ e
$r$ são ligados pela relação $(h-R)^2+r^2=R^2$, podemos expressar $V$ somente
em termos de $h$:
$$V(h)=\tfrac{\pi}{3}h(R^2-(h-R)^2)=\tfrac{\pi}{3}(2Rh^2-h^3)\,,$$
onde $h\in [0,2R]$.
Os valores na fronteira são $V(0)=0$, $V(2R)=0$.
Procurando os pontos críticos dentro do intervalo: $V'(h)=0$ se e somente se
$4Rh-3h^2=0$. Como $h=0$ não está \emph{dentro} do intervalo, somente
consideramos o ponto crítico $h_*=\tfrac{4}{3}R$. (Como $V''(h_*)<0$, é máximo
local.) Comparando $V(h_*)$ com os valores na fronteira, vemos que $h_*$ é
máximo global de $V$ em $[0,2R]$, e que tem dois mínimos globais, em $h=0$ e
$h=2R$.
{O maior cone, portanto, tem altura $\tfrac{4}{3}R$, e raio
$\sqrt{R^2-(\tfrac{4}{3}R-R)^2}=\frac{\sqrt{8}}{3}R$.}
\end{Solution}
\begin{Solution}{7.19}
Cada quadrado retirado deve ter os seus lados iguas a
$\tfrac12(1-\frac{1}{\sqrt{3}})$.
\end{Solution}
\begin{Solution}{7.20}
Como no exemplo anterior, $T(x)=\frac{\sqrt{x^2+h^2}}{v_1}+\frac{L-x}{v_2}$.
Procuremos o mínimo global de $T$ em $[0,L]$.
O ponto crítico $x_*$ é solução de
$\frac{x}{v_1\sqrt{x^2+h^2}}-\frac{1}{v_2}=0$. Isto é,
$x_*=\frac{h}{\sqrt{(v_2/v_1)^2-1}}$.
Se $v_1\geq v_2$, $T$ não tem ponto critico no intervalo, e  $T$ atinge o seu
mínimo global em $x=L$ (a melhor estratégia é de nadar diretamente até $B$). Se
$v_1<v_2$, e se $\frac{h}{\sqrt{(v_2/v_1)^2-1}}<L$, então $T$ tem um mínimo
global em $x_*$ (como $T''(x)=\frac{h^2}{v_1(x^2+h^2)}>0$
para todo $x$, $T$ é convexa, logo $x_*\in (0,L)$ é bem um ponto de
mínimo global).
Por outro lado, se $\frac{h}{\sqrt{(v_2/v_1)^2-1}}\geq L$, então $x_*$ não
pertence a $(0,L)$, e o mínimo global de $T$ é atingido em $x=L$.
\end{Solution}
\begin{Solution}{7.21}
Seja $O$ o centro da piscina. Uma estratégia que minimize o tempo de
viagem é de nadar em linha
reta de $A$ até um ponto $C$ na beirada tal que o ângulo $COB$ seja igual a
$\frac{\pi}{3}$ (ou $-\frac{\pi}{3}$). Depois, andar na beirada de $C$
até $B$.
\end{Solution}
\begin{Solution}{7.22}
%%%%%%%%%
A maior vara corresponde ao menor segmento que passa por $C$ e
encosta nas paredes em dois pontos $P$ e $Q$ (ver imagem
abaixo).
\begin{center}
\begin{bmlimage}\begin{tikzpicture}
\pgfmathsetmacro{\L}{1};
\pgfmathsetmacro{\M}{2};
\pgfmathsetmacro{\m}{2};
\pgfmathsetmacro{\l}{4};
\coordinate (A) at (0,{\M+\m});
\coordinate (B) at (\L,{\M+\m});
\coordinate (C) at (\L,{\M});
\coordinate (D) at ({\L+\l},\M);
\coordinate (E) at ({\L+\l},0);
\draw (A)--(0,0)--(E);
\draw (B)--(C)--(D);
\pgfmathsetmacro{\t}{30};
\coordinate (P) at (0,{\M+(\L*sin(\t)/cos(\t))});
\draw (P) node[left]{$P$};
\coordinate (Q) at ({\L+(\M*cos(\t)/sin(\t))},0);
\draw (Q) node[below]{$Q$};
\draw[thick] (P)--(C);
\draw[thick] (Q)--(C);
\draw (C) node[below left]{$C$};
\draw (D) node[right]{$D$};
%\coordinate (X) at (1.8,1.5);
%\coordinate (Y) at (4.2,0.5);
%\fill (X) circle (0.3mm);
%\fill (Y) circle (0.3mm);
%\draw[thick] (X)--(Y) node[midway, above]{$\ell$};
%\draw[dotted, <->] (A)--(B) node[midway, above]{$L$};
%\draw[dotted, <->] (D)--(E) node[midway, right]{$M$};
%\fill[color=gray!15] (0,0)--(A)--(B)--(C)--(D)--(E)--cycle;
\fill (P) circle (0.4mm);
\fill (Q) circle (0.4mm);
\fill (C) circle (0.4mm);
\end{tikzpicture}\end{bmlimage}
\end{center}
Seja $\theta$ o ângulo $QCD$. Quando $\theta$ é fixo, a
distância de $P$ a $Q$ vale
$$
f(\theta)=\frac{L}{\cos \theta}+\frac{M}{\sen \theta}\,.
$$
Precisamos minimizar $f$ no intervalo $(0,\pisobredois)$.
(Observe que $\lim_{\theta\to 0^+}f(\theta)=+\infty$,
$\lim_{\theta\to {\pisobredois}^-}f(\theta)=+\infty$.)
Resolvendo $f'(\theta)=0$, vemos que o único ponto crítico
$\theta_*$ satisfaz $\tan^3\theta_*=M/L$. É fácil verificar
que $f$ é convexa, logo $\theta_*$ é um ponto de mínimo global
de $f$.
Assim, o tamanho da maior vara possível é igual a
$$
f(\theta_*)=\cdots=L\bigl(1+(M/L)^{2/3}\bigr)^{3/2}\,.
$$
Observe que quando $L=M$, a maior vara tem tamanho
$2\sqrt{2}L$, e quando $M\to 0^+$, a maior vara tende a ter
tamanho igual a $L$.
\end{Solution}
\protect \section *{Capítulo \ref {Cap:Estudos}}
\begin{Solution}{8.1}
(Já vimos no Exemplo \ref{Ex:logsurx} que a afirmação vale para $p=1$, $q=1$.)
Observe que
$\frac{(\ln
x)^p}{x^q}=(\frac{(\ln
x)^{p/q}}{x})^q$. Logo, basta provar a afirmação para $q=1$ e $p>0$ qualquer:
$\lim_{x\to \infty}\frac{(\ln
x)^p}{x}=0$.
Mostremos por indução que se a afirmação vale para $p>0$
($\lim_{x\to \infty}\frac{(\ln
x)^{p}}{x}=0$), então ela vale para $p+1$. De fato, pela regra de B.H.,
$$
\lim_{x\to \infty}\frac{(\ln x)^{p+1}}{x}=\lim_{x\to \infty}\frac{(p+1)(\ln
x)^{p}\tfrac{1}{x}}{1}=
(p+1)\lim_{x\to \infty}\frac{(\ln
x)^{p}}{x}=0\,.
$$
Então, a afirmação estará provada para qualquer $p>0$ se ela for provada para
$0<p\leq 1$. Mas para tais $p$, $(\ln x)^p\leq \ln x$ para todo $x>1$, logo,
$$
\lim_{x\to \infty}\frac{(\ln x)^p}{x}\leq \lim_{x\to \infty}\frac{\ln x}{x}=0\,,
$$
pelo Exemplo \ref{Ex:logsurx}.
\end{Solution}
\begin{Solution}{8.3}
\eqref{itexBHlast1} $0$
\eqref{itexBHlast2} $0$
\eqref{itexBHlast3} $-\infty$
\eqref{itexBHlast4} $0$
\eqref{itexBHlast5_a} $0$
\eqref{itexBHlast5_b} $\infty$
\eqref{itexBHlast6} $0$
\eqref{itexBHlast24} $\infty$
\end{Solution}
\begin{Solution}{8.4}
\eqref{itasobl1} A função é a sua própria assíntota oblíqua.
\eqref{itasobl11} Não possui ass.
\eqref{itasobl6} $y=-2$ (vertical), $y=x-2$ em $\pm\infty$.
\eqref{itasobl12} Não possui ass.
\eqref{itasobl2} $y=0$ em $-\infty$, $y=x$ em $+\infty$.
\eqref{itasobl3} $y=x$ em $+\infty$.
\eqref{itasobl4} $y=x-\ln 2$ em $+\infty$, $y=-x-\ln 2$ em
$-\infty$.
\eqref{itasobl5} Não possui assíntotas: apesar de
$m=\lim_{x\to \infty}\frac{e^{\sqrt{\ln^2x+1}}}{x}$
existir e valer $1$,
$\lim_{x\to\infty}\{e^{\sqrt{\ln^2x+1}}-x\}=\infty$.
\end{Solution}
\begin{Solution}{8.5}
Em geral, náo.
Por exemplo, $f(x)=x+\tfrac{1}{x}\sen (x^2)$ possui $y=x$ como assíntota
oblíqua em $+\infty$,
mas $f'(x)=1-\frac{\sen x^2}{x^2}+2\cos (x^2)$
não possui limite quando $x\to\infty$.
Na verdade, uma função pode possuir uma assíntota (oblíqua ou
outra)
sem sequer ser derivável.
\end{Solution}
\begin{Solution}{8.6}
%%%%%%%%%%%%%%%%%%%%%%%%%%%%%%%%5
\eqref{itexoEstudA1}:
O domínio de $\bigl(\frac{x-1}{x}\bigr)^2$ é $D=\bR\setminus \{0\}$, o sinal é
sempre não-negativo, tem um zero
em $x=1$. $f$ não é par, nem ímpar.
Os limites relevantes são $\lim_{x\to 0^{\pm}}f(x)=+\infty$, logo $x=0$ é
assíntota vertical, e
$$\lim_{x\to \pm\infty}\bigl(\frac{x-1}{x}\bigr)^2=\Bigl(\lim_{x\to \pm
\infty}\frac{x-1}{x}\Bigr)^2==\Bigl(\lim_{x\to \pm
\infty}\bigl(1-\frac{1}{x}\bigr)\Bigr)^2=1^2=1\,.$$
Logo, $y=1$ é assíntota horizontal.
$f$ é derivável em $D$, e $f'(x)=\frac{2(x-1)}{x^3}$.
\begin{center}
\begin{bmlimage}\begin{tikzpicture}
\tkzTabInit[nocadre,espcl=2,  color, colorV=lightgray!5, colorL=gray!15,
colorC=gray!15]
{$x$ /.6,  $f'(x)$ /.6, Var. de $f$ /1.3}%
{,$0$, $1$,}%
%\tkzTabLine{,+,z,+,,+,}
\tkzTabLine{,+,d,-,z,+,}
\tkzTabVar{-/,+D+/$+\infty$/$+\infty$,-/mín,+/,}
%\tkzTabLine{,\searrow,\text{mín.},h,\text{mín.},\nearrow,}
\end{tikzpicture}\end{bmlimage}
\end{center}
$f$ possui um mínimo global em $(1,0)$.
A segunda derivada é dada por $f''(x)=\frac{2(3-2x)}{x^4}$. Ela se anula em
$x=\tfrac32$, e muda de sinal neste ponto:
\begin{center}
\begin{bmlimage}\begin{tikzpicture}
\tkzTabInit[nocadre,espcl=2,  color, colorV=lightgray!5, colorL=gray!15,
colorC=gray!15]
{$x$ /.6,  $f''(x)$ /.7, Conv. de $f$ /1.2}%
{,$0$, $\tfrac32$,}%
\tkzTabLine{,+,d,+,z,-,}%
\tkzTabLine{,\smile,d,\smile,z,\frown,}%
\end{tikzpicture}\end{bmlimage}
\end{center}
Logo, $f$ é convexa em $(-\infty,0)$ e $(0,\frac32)$,  côncava em
$(\frac32,\infty)$, e possui um ponto de inflexão em
$(\tfrac{3}{2},f(\tfrac{3}{2}))=(\tfrac{3}{2},\tfrac19)$.
\begin{center}
\begin{bmlimage}\begin{tikzpicture}
\draw [thick, domain=-4:-1.2, samples=100] plot
(\x,{((\x)-1)^2/((\x)^2)});
\draw [thick, domain=0.4:4, samples=100] plot (\x,{(\x-1)^2/((\x)^2)});
\draw [ ->] (-4,0)--(4,0) node[right] {$x$};
\draw [ ->] (0,-0.1)--(0,3) node[left] {$f(x)$};
\draw [dotted] (-4,1)--(4,1) node[above] {$y=1$};
\draw [dotted] (0,0)--(0,3.5) node[right] {$x=0$};
\fill (1,0) circle (0.35mm);
\draw (1,0) node[below] {$(1,0)$};
\fill (1.5,0.1111) circle (0.35mm);
\draw [ <-] (1.52,0.0911)--(2,-0.3) node[right]
{$(\tfrac{3}{2},\tfrac{1}{9})$};
\end{tikzpicture}\end{bmlimage}
\end{center}
\eqref{itexoEstudA3}:
O domínio de  $f(x)=x(\ln x)^2$ é
$D=(0,+\infty)$, e o seu sinal é: $f(x)\geq 0$ para todo $x\in D$.
A função não é { par} nem { ímpar}.
Como $\lim_{x\to \infty}f(x)=+\infty$, não tem assintota horizontal.
Para ver se tem assíntota vertical em $x=0$, calculemos
$\lim_{x\to 0^+}f(x)=\lim_{x\to 0^+}\frac{(\ln x)^2}{1/x}$. Como ambas funções
$(\ln x)^2$ e $1/x$ são deriváveis em $(0,1)$ e tendem a $+\infty$ quando $x\to
0^+$, apliquemos a regra de B.H.:
$$
\lim_{x\to 0^+}\frac{(\ln x)^2}{1/x}=
\lim_{x\to 0^+}\frac{2(\ln x)1/x}{-1/x^2}=
-2\lim_{x\to 0^+}x\ln x\,.
$$
Usando a regra de B.H. de novo, pode ser mostrado que esse segundo limite é
zero (ver Exemplo \ref{Ex:xlogxemzero}). Logo, $\lim_{x\to 0^+}f(x)=0$: não
tem assíntota vertical em $x=0$.
A derivada é dada por $f'(x)=\ln x(\ln x+2)$.
\begin{center}
\begin{bmlimage}\begin{tikzpicture}[scale=0.8]
\tkzTabInit[nocadre, espcl=2,  color, colorV=lightgray!5, colorL=gray!15,
colorC=gray!15]
{$x$ /.6, $f'(x)$ /.6, Variaç. de $f$ /1.2}%
{,$e^{-2}$, $1$, }%
\tkzTabLine{,+,z,-,z,+}
\tkzTabVar{-/,+/{máx.},-/{mín.},+/}
%\tkzTabLine{,\searrow,\text{mín.},h,\text{mín.},\nearrow,}
\end{tikzpicture}\end{bmlimage}
\end{center}
O máximo local está em
$(e^{-2},f(e^{-2}))=(e^{-2},4e^{- 2})$, e o
mínimo global em $(1,f(1))=(1,0)$.
A {segunda derivada} de $f$ é dada por
$f''(x)=\frac{2(\ln x+1)}{x}$.
\begin{center}
\begin{bmlimage}\begin{tikzpicture}[scale=0.8]
\tkzTabInit[nocadre, espcl=2,  color, colorV=lightgray!5, colorL=gray!15,
colorC=gray!15]
{$x$ /.6, $f''(x)$ /.6, Conv. de $f$ /1.2}%
{,$e^{-1}$, }%
\tkzTabLine{,-,z,+,}
\tkzTabLine{,\frown,,\smile,}
\end{tikzpicture}\end{bmlimage}
\end{center}
Logo, $f$ é côncava em $(0,e^{-1})$, possui um ponto de inflexão em
$(e^{-1},f(e^{-1}))=(e^{-1},e^{-1})$, e é convexa em $(e^{-1},+\infty)$.
\begin{center}
\begin{bmlimage}\begin{tikzpicture}[scale=1.3]
\draw [thick, domain=0.001:2.5, samples=100] plot (\x,{(\x)*(ln(\x))^2});
 \draw [ ->] (0,0)--(2.5,0) node[right] {$x$};
 \draw [ ->] (0,-0.1)--(0,2);
% \draw [dotted] (-4,1)--(4,1) node[above left] {Assíntota horiz.: $y=1$};
 \fill (1,0) circle (0.35mm);
 \draw (1,0) node[below] {$\scriptscriptstyle{(1,0)}$};
 \fill (0.367,0.367) circle (0.35mm);
 \draw[<-] (0.39,0.39)--(0.9,0.5) node[above]
{$\scriptscriptstyle{(e^{-1},e^{-1})}$};
 \fill (0.1353,0.541) circle (0.35mm);
 \draw[<-] (0.14,0.58)--(0.9,1.5) node[above]
{$\scriptscriptstyle{(e^{-2},4e^{-2})}$};
\end{tikzpicture}\end{bmlimage}
\end{center}
Podemos também notar que $\lim_{x\to 0^+}f'(x)=+\infty$.
\end{Solution}
\begin{Solution}{8.7}
$D=\bR\backslash \{\pm 4\}$. Os zeros de $f(x)\pardef\frac{x^2-4}{x^2-16}$ são
$x=-2$, $x=+2$, e o seu sinal:
\begin{center}
\begin{bmlimage}\begin{tikzpicture}
\tkzTabInit[lgt=3, nocadre, espcl=2]
{ /.6,  $x^2-4$ /.6, $x^2-16$ /.6, $f(x)$ /.8}%
{,$-4$, $-2$, $2$, $4$,}%
%\tkzTabLine{,+,z,+,,+,}
\tkzTabLine{,+,,+,z,-,z,+,,+,}
\tkzTabLine{,+,z,-,,-,,-,z,+,}
\tkzTabLine{,+,d,-,z,+,z,-,d,+,}
%\tkzTabLine{,+,z,-,z,+,}
%\tkzTabVar{-/,+/\text{a.v.},-/$0$,+/,}
%\tkzTabLine{,\searrow,\text{mín.},h,\text{mín.},\nearrow,}
\end{tikzpicture}\end{bmlimage}
\end{center}
Como
$$
\lim_{x\to \pm\infty}f(x)=\lim_{x\to
\pm \infty}\frac{1-\frac{4}{x^2}}{1-\frac{16}{x^2}}
=1\,,$$
a reta $y=1$ é assíntota horizontal.
Como
$$
\lim_{x\to -4^\pm}f(x)=\mp \infty\,,\quad \lim_{x\to +4^\pm}f(x)=\pm \infty\,,$$
as retas $x=-4$ e $x=+4$ são assíntotas verticais.
A primeira derivada se calcula facilmente: $f'(x)=\frac{-24 x }{(x^2-16)^2}$,
logo a variação de $f$ é dada por:
\begin{center}
\begin{bmlimage}\begin{tikzpicture}[scale=0.8]
\tkzTabInit[nocadre, espcl=2,  color, colorV=lightgray!5, colorL=gray!15,
colorC=gray!15]
{$x$ /.6, $f'(x)$ /.6, Variaç. de $f$ /1.2}%
{,$-4$,$0$,$4$, }%
\tkzTabLine{,+,d,+,z,-,d,-}
\tkzTabVar{-/,+D-/{},+/{máx.},-D+/{},-/}
%\tkzTabLine{,\searrow,\text{mín.},h,\text{mín.},\nearrow,}
\end{tikzpicture}\end{bmlimage}
\end{center}
A posição do máximo local é: $(0,f(0))=(0,\tfrac14)$.
O gráfico:
\begin{center}
\begin{bmlimage}\begin{tikzpicture}[scale=0.7]
\pgfmathsetmacro{\a}{10}
\pgfmathsetmacro{\b}{4}
\newcommand{\funcao}[1]{ ( (#1)^2-4 )/( (#1)^2-16) }
\draw[->] (-\a,0)--(\a,0);
\draw[->] (0,-\b)--(0,\b);
\draw[thick, domain=-\a:-4.5, samples=50] plot (\x,{\funcao{\x}});
\draw[thick, domain=-3.5:3.5, samples=50] plot (\x,{\funcao{\x}});
\draw[thick, domain=4.5:\a, samples=50] plot (\x,{\funcao{\x}});
\draw[dashed] (-\a,1)node[below]{$y=1$}--(\a,1);
\draw[dashed] (-4,-\b)node[left]{$x=-4$}--(-4,\b);
\draw[dashed] (4,-\b)node[right]{$x=+4$}--(4,\b);
\draw[<-] (0.1,0.3)--(2,2)node[above]{máx.: $(0,\frac14)$};
\draw (-2,0) node{$\shortmid$} node[above]{$-2$};]
\draw (2,0) node{$\shortmid$} node[above]{$+2$};]
\end{tikzpicture}\end{bmlimage}
\end{center}
A segunda derivada: $f''(x)=24\frac{16+3x^2}{(x^2-16)^3}$, e a convexidade é
dada por
\begin{center}
\begin{bmlimage}\begin{tikzpicture}[scale=0.8]
\tkzTabInit[nocadre, espcl=2,  color, colorV=lightgray!5, colorL=gray!15,
colorC=gray!15]
{$x$ /.6, $f''(x)$ /.6, Conv. $f$ /1.2}%
{,$-4$,$4$, }%
\tkzTabLine{,+,d,-,d,+}
\tkzTabLine{,\smile,d, \frown,d,\smile,}
%\tkzTabLine{,\searrow,\text{mín.},h,\text{mín.},\nearrow,}
\end{tikzpicture}\end{bmlimage}
\end{center}
\end{Solution}
\begin{Solution}{8.8}
%%%%%%%%%%%%%%%%%%
% \eqref{itEstBas9a}
%  { Domínio}: $D=\R\backslash \{-1\}$. { Sinal}:
% $f(x)\geq 0$ para todo $x\in D$, e $f(x)=0$ se e somente se $x=0$.
% Assíntotas:
% como $\lim_{x\to -1^-}f(x)=\lim_{x\to -1^+}f(x)=+\infty$, { a reta $x=-1$ é
% assíntota vertical}
% (é a única). Como
% $$\lim_{x\to \pm\infty}\frac{x^2}{(x+1)^2}=\lim_{x\to
% \pm\infty}\frac{1}{(1+\frac{1}{x})^2}=1\,,$$
%  { a reta $y=1$ é assíntota
% horizontal} (a direita e a esquerda). Como
% $f'(x)=\frac{2x}{(x+1)^3}$,
% \begin{center}
% \begin{bmlimage}\begin{tikzpicture}[scale=0.8]
% \tkzTabInit[nocadre, espcl=2,  color, colorV=lightgray!5, colorL=gray!15,
% colorC=gray!15]
% {$x$ /.6, $f'(x)$ /.6, Variaç. de $f$ /1.2}%
% {,$-1$, $0$, }%
% \tkzTabLine{,+,d,-,z,+}
% \tkzTabVar{-/,+D+/,-/mín.,+/}
% %\tkzTabLine{,\searrow,\text{mín.},h,\text{mín.},\nearrow,}
% \end{tikzpicture}\end{bmlimage}
% \end{center}
% $f$ possui um mínimo local no ponto $(0,f(0))=(0,0)$.
% Como $f''(x)=\frac{2(1-2x)}{(x+1)^4}$, temos:
% \begin{center}
% \begin{bmlimage}\begin{tikzpicture}[scale=0.8]
% \tkzTabInit[nocadre, espcl=2,  color, colorV=lightgray!5, colorL=gray!15,
% colorC=gray!15]
% {$x$ /.6, $f''(x)$ /.6, Conv. de $f$ /1.2}%
% {,$-1$, $\tfrac12$, }%
% \tkzTabLine{,+,d,+,z,-}
% \tkzTabLine{,\smile,d,\smile,z,\frown}
% %\tkzTabVar{-/,+D+/${+\infty}$/${+\infty}$,-/mín.,+/}
% %\tkzTabLine{,\searrow,\text{mín.},h,\text{mín.},\nearrow,}
% \end{tikzpicture}\end{bmlimage}
% \end{center}
% Logo, $f$ é convexa nos intervalos $]-\infty,-1[$ e $]-1,\tfrac12[$, possui um
% ponto de inflexão em $(\tfrac12,f(\tfrac12))=(\tfrac12,\tfrac19)$, e é côncava
% em $(\tfrac12,+\infty)$. Gráfico:
% \begin{center}
% \begin{bmlimage}\begin{tikzpicture}
% \draw[ ->] (-4,0)--(4,0);
% \draw[dashed] (-5,1)node[below]{$y=1$}--(4,1);
% \draw[ ->] (0,-0.5)--(0,4);
% \draw[dashed] (-1,-0.5)node[below]{$x=-1$}--(-1,4);
% \draw[thick, domain=-5:-1.9, samples=50] plot (\x,{\x^2/(\x+1)^2});
% \draw[thick, domain=-0.65:4, samples=100] plot (\x,{\x^2/(\x+1)^2});
% \end{tikzpicture}\end{bmlimage}
% \end{center}
% Observe que esse gráfico é o gráfico da função $(\frac{x}{x-1})$
OBS: Para as demais funções, colocamos somente um \emph{resumo} das
soluções, na forma de um gráfico no qual o leitor pode verificar os resultados
do seu estudo.

\eqref{itEstBas1} Ass. vert.: $x=0$. Ass. oblíqua: $y=x$.
\begin{center}
\begin{bmlimage}\begin{tikzpicture}[yscale=0.7]
\draw [thick, domain=-3:-0.3, samples=100] plot (\x,{(\x)+1/(\x)});
\draw [thick, domain=0.3:3, samples=100] plot (\x,{(\x)+1/(\x)});
 \draw [ ->] (-3,0)--(3,0) node[right] {$x$};
 \draw [ ->] (0,-3)--(0,3) node[left]{$x+\tfrac{1}{x}$};
 \fill (1,2) circle (0.45mm);
  \draw (1,2) node[below] {$\scriptscriptstyle{(1,2)}$};
  \fill (-1,-2) circle (0.45mm);
  \draw (-1,-2) node[above] {$\scriptscriptstyle{(-1,-2)}$};
\end{tikzpicture}\end{bmlimage}
\end{center}


\eqref{itEstBas6}
Ass. vert.: $x=0$. Ass. obl.: $y=x$.
\begin{center}
\begin{bmlimage}\begin{tikzpicture}[yscale=0.7]
\draw [thick, domain=-3:-0.5, samples=100] plot
(\x,{\x+1/((\x)^2)});
\draw [thick, domain=0.6:3, samples=100] plot
(\x,{\x+1/((\x)^2)})node[right]{$x+\tfrac{1}{x^2}$};
 \draw [ ->] (-3,0)--(3,0) node[right] {$x$};
 \draw [ ->] (0,-3)--(0,3);
 \fill (1.256,1.88) circle (0.45mm);
  \draw (1.256,1.88) node[below]
{$\scriptscriptstyle{(2^{1/3},2^{1/3}+2^{-2/3})}$};
\draw (-1,0) node{$\shortmid$} node[above left]{$-1$};
\end{tikzpicture}\end{bmlimage}
\end{center}

\eqref{itEstBas9}
\begin{center}
\begin{bmlimage}\begin{tikzpicture}
\draw [thick, domain=-3:3, samples=100] plot
(\x,{1/((\x)^2+1)});
\draw [ ->] (-3,0)--(3,0);
\draw [ ->] (0,-0.5)--(0,1.5)node[right]{$\tfrac{1}{x^2+1}$};
\fill (0.577,0.75) circle (0.45mm);
\draw[<-] (0.6,0.8)--(1.3,1)
node[right]{inflex: $(\tfrac{1}{\sqrt{3}},\tfrac34)$};
\fill (-0.577,0.75) circle (0.45mm);
\draw[<-] (-0.6,0.8)--(-1.3,1)
node[left]{inflex: $(-\tfrac{1}{\sqrt{3}},\tfrac34)$};
\draw (-5,2) node[left]{$\displaystyle{f'(x)=\frac{-2x}{(x^2+1)^2}}$};
\draw (-5,0.5) node[left]{$\displaystyle{f''(x)=\frac{2(3x^2-1}{(x^2+1)^3}}$};
\end{tikzpicture}\end{bmlimage}
\end{center}



\eqref{itEstBas2}
\begin{center}
\begin{bmlimage}\begin{tikzpicture}[yscale=0.7]
\draw [ ->] (-3,0)--(3,0) node[right] {$x$};
\draw [ ->] (0,-3)--(0,3) node[left]{$\frac{x}{x^2-1}$};
\draw [thick, domain=-3:-1.2, samples=100] plot (\x,{\x/((\x)^2-1)});
\draw[dashed] (-1,-2.5)--(-1,2.5) node[below left]{$\scriptscriptstyle{x=-1}$};
\draw [thick, domain=-0.8:0.8, samples=100] plot (\x,{\x/((\x)^2-1)});
\draw [thick, domain=1.2:3, samples=100] plot (\x,{\x/((\x)^2-1)});
\draw[dashed] (1,-2.5)node[right]{$\scriptscriptstyle{x=1}$}--(1,2.5) ;
\draw[<-] (0.3,-0.1)--(1.5,-1)
node[right]{pt. inflex.: $\scriptstyle{(0,0)}$};
\draw (4,2) node[right]{$\displaystyle{f'(x)=\frac{-(1+x^2)}{(x^2-1)^2}}$};
\draw (4,0.5)
node[right]{$\displaystyle{f''(x)=\frac{-2x(3x^2+1)}{(x^2-1)^3}}$};
\end{tikzpicture}\end{bmlimage}
\end{center}

\eqref{itEstBas3}
\begin{center}
\begin{bmlimage}\begin{tikzpicture}
\draw [ ->] (-3,0)--(3,0) node[right] {$x$};
\draw [ ->] (0,-1)--(0,1) node[above]{$xe^{-x^2}$};
\draw [thick, domain=-2.5:2.5, samples=100] plot (\x,{\x*exp(-(\x)^2)});
\fill (0.707,0.428) circle (0.45mm);
  \draw[<-] (0.71,0.44)-- (0.9,1) node[right]
{$\scriptstyle{(\tfrac{1}{\sqrt{2}},\tfrac{1}{\sqrt{2}}e^{-\tfrac12})}$};
\fill (-0.707,-0.428) circle (0.45mm);
  \draw[<-] (-0.71,-0.5)-- (-0.9,-1) node[left]
{$\scriptstyle{(-\tfrac{1}{\sqrt{2}},-\tfrac{1}{\sqrt{2}}e^{-\tfrac12})}$};
\draw[<-] (0.1,-0.1)--(0.5,-1.3) node[right]{pt. inflex. $\scriptstyle{(0,0)}$};
\fill (1.225,0.273) circle (0.40mm);
\fill (-1.225,-0.273) circle (0.40mm);
\draw[<-] (1.225,0.24)--(1.5,-0.6)
node[right]{pt. inflex.: $\scriptstyle{(\sqrt{3/2},f(\sqrt{3/2}))}$};
\draw[<-] (-1.225,-0.24)--(-1.5,0.6)
node[left]{pt. inflex.: $\scriptstyle{(-\sqrt{3/2},f(\sqrt{3/2}))}$};
\draw (4,1.2) node[right]{$\displaystyle{f'(x)=(1-2x^2)e^{-x^2}}$};
\draw (4,0.5)
node[right]{$\displaystyle{f''(x)=-2x(3-2x^2)e^{-x^2}}$};
\end{tikzpicture}\end{bmlimage}
\end{center}

\eqref{itEstBas7}, \eqref{itEstBas8},
\eqref{itEstBas8t}:
\begin{center}
\begin{bmlimage}\begin{tikzpicture}[scale=0.5]
\draw [ ->] (0,-0.1)--(0,3);
\pgfmathsetmacro{\a}{2};
\draw [ ->] (-\a,0)--(\a,0);
\draw [thick, domain=-\a:\a, samples=100] plot (\x,{(exp(\x)+exp(-\x))/2})
node[right]{$\cosh x$};

\begin{scope}[xshift=9cm, yshift=1cm]
\draw [ ->] (0,-2)--(0,2);
\pgfmathsetmacro{\a}{1.6};
\draw [ ->] (-\a,0)--(\a,0);
\draw [thick, domain=-\a:\a, samples=100] plot (\x,{(exp(\x)-exp(-\x))/2})
node[right]{$\senh x$};
\end{scope}

\begin{scope}[xshift=18cm, yshift=1cm]
\draw [ ->] (0,-1.5)--(0,1.5);
\pgfmathsetmacro{\a}{3};
\draw [ ->] (-\a,0)--(\a,0);
\draw [thick, domain=-\a:\a, samples=100] plot
(\x,{(exp(\x)-exp(-\x))/(exp(\x)+exp(-\x))})
node[below right]{$\tanh x$};
\draw[dashed] (0,1)--(\a,1) node[above]{$x=+1$};
\draw[dashed] (0,-1)--(-\a,-1) node[below]{$x=-1$};
\end{scope}

\end{tikzpicture}\end{bmlimage}
\end{center}

\eqref{itEstBas13}
\begin{center}
\begin{bmlimage}\begin{tikzpicture}[yscale=0.7]
\draw [ ->] (-3,0)--(3,0);
\draw [ ->] (0,-3)--(0,3) node[right]{$\frac{x^3-1}{x^3+1}$};
\draw [thick, domain=-3:-1.2, samples=100] plot (\x,{((\x)^3-1)/((\x)^3+1)});
\draw [thick, domain=-0.8:3, samples=100] plot (\x,{((\x)^3-1)/((\x)^3+1)});
\draw[dashed] (-1,-3)node[left]{$\scriptscriptstyle{x=-1}$}--(-1,3) ;
\draw[dashed] (-3,1) node[below]{$\scriptscriptstyle{x=1}$}--(3,1) ;
\fill (0,-1) circle (0.45mm);
\fill (0.793, -0.3333) circle (0.45mm);
\draw[<-] (0.1,-1.1)--(1,-3)node[right]{Pt. de inflexão e crítico: $(0,-1)$};
\draw[<-] (0.8, -0.4)--(1.2,-1) node[right]{Pt. de inflexão:
$(2^{-1/3},-1/3)$};
\draw (1,0) node{$\shortmid$} node[above]{$1$};
\draw (4,2) node[right]{$\displaystyle{f'(x)=\frac{6x^2}{(x^3+1)^2}}$};
\draw (4,0.5)
node[right]{$\displaystyle{f''(x)=\frac{12x(1-2x^3)}{(x^3+1)^3}}$};
\end{tikzpicture}\end{bmlimage}
\end{center}

\eqref{itEstBas14}:
\begin{center}
\begin{bmlimage}\begin{tikzpicture}
\draw [ ->] (-0.2,0)--(2*pi+0.5,0);
\draw [ ->] (0,-1.5)--(0,1.5) node[right]{$\scriptstyle{\tfrac12\sen
(2x)-\sen(x)}$};
\draw [color=gray!20, domain=-1:2*pi+1, samples=100] plot (\x,{0.5*sin(2*\x
r)-sin(\x r)});
\draw [thick, domain=0:2*pi, samples=100] plot (\x,{0.5*sin(2*\x r)-sin(\x r)});
\foreach \k in {0, 0.666, 1.333, 2} {
\draw ({\k*pi},0) node{$\shortmid$};
}
\draw (0.666*pi,0) node[above]{$\tfrac{2\pi}{3}$};
\draw (1.333*pi,0) node[below]{$\tfrac{4\pi}{3}$};
\fill (1.318,-0.726) circle (0.40mm);
\fill (4.965,0.726) circle (0.40mm);
\fill (0,0) circle (0.40mm);
\fill (pi,0) circle (0.40mm);
\fill (2*pi,0) circle (0.40mm);
% \draw[dashed] (-1,-3)node[left]{$\scriptscriptstyle{x=-1}$}--(-1,3) ;
% \draw[dashed] (-3,1) node[below]{$\scriptscriptstyle{x=1}$}--(3,1) ;
% \fill (0,-1) circle (0.45mm);
% \fill (0.793, -0.3333) circle (0.45mm);
% \draw[<-] (0.1,-1.1)--(1,-3)node[right]{Pt. de inflexão e crítico: $(0,-1)$};
% \draw[<-] (0.8, -0.4)--(1.2,-1) node[right]{Pt. de inflexão:
% $(2^{1/3},f(2^{1/2}))$};
% \draw (1,0) node{$\shortmid$} node[above]{$1$};
\end{tikzpicture}\end{bmlimage}
\end{center}

\eqref{itEstBas15}:
\begin{center}
\begin{bmlimage}\begin{tikzpicture}
\draw [ ->] (-5,0)--(5,0);
\draw [ ->] (0,-1.3)--(0,1.3) node[left]{$\frac{x}{\sqrt{x^2+1}}$};
\draw [thick, domain=-5:5, samples=50] plot (\x,{\x/(sqrt((\x)^2+1))});
 \draw[dashed] (0,1)--(5,1)node[above]{$\scriptscriptstyle{y=1}$};
 \draw[dashed] (-5,-1) node[below]{$\scriptscriptstyle{y=-1}$}--(0,-1) ;
 \draw[<-] (0.2, -0.2)--(0.5,-0.7) node[right]{Pt. de inflexão: $(0,0)$};
\draw (6,0.5) node[right]{$\displaystyle{f'(x)=\frac{1}{(x^2+1)^{3/2}}}$};
\draw (6,-0.5)
node[right]{$\displaystyle{f''(x)=\frac{-3x}{(x^2+1)^{5/2}}}$};
\end{tikzpicture}\end{bmlimage}
\end{center}


\end{Solution}
\begin{Solution}{8.9}
\eqref{itEstFuncB1}
\begin{center}
\begin{bmlimage}\begin{tikzpicture}[yscale=0.8]
\draw [ ->] (-4,0)--(4,0);
\draw [ ->] (0,-1.3)--(0,2.3) node[above]{$\ln |2-5x|$};
\draw [thick, domain=-4:0.3, samples=100] plot (\x,{ln(abs(2-5*(\x)))});
\draw[dashed] (0.4,-1.5)node[right]{$\scriptscriptstyle{x=\frac25}$}--(0.4,1.5);
\draw [thick, domain=0.5:4, samples=100] plot (\x,{ln(abs(2-5*\x))});
%  \draw[dashed] (-5,-1) node[below]{$\scriptscriptstyle{y=-1}$}--(0,-1) ;
% \fill (0,-1) circle (0.45mm);
% \fill (0.793, -0.3333) circle (0.45mm);
% \draw[<-] (0.1,-1.1)--(1,-3)node[right]{Pt. de inflexão e crítico: $(0,-1)$};
% \draw[<-] (0.2, -0.2)--(0.5,-0.7) node[right]{Pt. de inflexão: $(0,0)$};
% \draw (1,0) node{$\shortmid$} node[above]{$1$};
\end{tikzpicture}\end{bmlimage}
\end{center}


\eqref{itEstFuncB3}
\begin{center}
\begin{bmlimage}\begin{tikzpicture}[yscale=0.7]
\draw [ ->] (0,0)--(5,0);
\draw [ ->] (0,-1.3)--(0,2.3) node[left]{$\ln(\ln x)$};
\draw [thick, domain=1.2:5, samples=100] plot (\x,{ln(ln(\x))});
% \draw[dashed]
%(0.4,-1.5)node[right]{$\scriptscriptstyle{x=\frac25}$}--(0.4,1.5);
% \draw [thick, domain=0.5:4, samples=100] plot (\x,{ln(abs(2-5*\x))});
 \draw[dashed] (1,-2) node[left]{$\scriptscriptstyle{x=1}$}--(1,2) ;
% \fill (-0.693,1.012) circle (0.45mm);
% \fill (-1.365,1.033) circle (0.45mm);
% \fill (2.46,4.86) circle (0.45mm);
% \fill (0.793, -0.3333) circle (0.45mm);
% \draw[<-] (0.1,-1.1)--(1,-3)node[right]{Pt. de inflexão e crítico: $(0,-1)$};
% \draw[<-] (0.2, -0.2)--(0.5,-0.7) node[right]{Pt. de inflexão: $(0,0)$};
% \draw (1,0) node{$\shortmid$} node[above]{$1$};
\end{tikzpicture}\end{bmlimage}
\end{center}

\eqref{itEstFuncB7}
\begin{center}
\begin{bmlimage}\begin{tikzpicture}
\newcommand{\funcao}[1]{2.5*exp( -1*(#1) )*( (#1)^2 - 2*(#1))}
\draw [ ->] (-1,0)--(6.5,0);
\draw [ ->] (0,-1)--(0,2.3) node[right]{$e^{-x}(x^2-2x)$};
\draw [thick, domain=-0.35:5.5, samples=100] plot (\x,{\funcao{\x}});
\fill ({2-sqrt(2)},{\funcao{2-sqrt(2)}}) circle (0.40mm);
\draw ({2-sqrt(2)},{\funcao{2-sqrt(2)}})
node[below]{$\scriptstyle{(2-\sqrt{2},f(2-\sqrt{2}))}$};
\draw ({2+sqrt(2)},{\funcao{2+sqrt(2)}})
node[above]{$\scriptstyle{(2+\sqrt{2},f(2+\sqrt{2}))}$};
\fill ({2+sqrt(2)},{\funcao{2+sqrt(2)}}) circle (0.40mm);
\fill ({(6+sqrt(10))/2},{\funcao{(6+sqrt(10))/2}}) circle (0.40mm);
\draw[<-] ({(6+sqrt(10))/2+0.1},{\funcao{(6+sqrt(10))/2}+0.1})--
({(6+sqrt(10))/2+0.5},{\funcao{(6+sqrt(10))/2}+0.3})
node[right]{$\scriptstyle{(3+\sqrt{10}/2,f(3+\sqrt{10}/2))}$};
\fill ({(6-sqrt(10))/2},{\funcao{(6-sqrt(10))/2}}) circle (0.40mm);
\draw[<-] ({(6-sqrt(10))/2-0.1},{\funcao{(6-sqrt(10))/2}+0.1})--
({(6-sqrt(10))/2-0.5},{\funcao{(6-sqrt(10))/2}+2})
node[right]{$\scriptstyle{(3-\sqrt{10}/2,f(3-\sqrt{10}/2))}$};
\draw[<-] (5,-0.2)--(4.5,-0.5)node[below]{ass. horiz.: $y=0$};
\draw (6,2)
node[right]{$\displaystyle{f'(x)=-(x^2-4x+2)e^{-x}}$};
\draw (6,1.5)
node[right]{$\displaystyle{f''(x)=(x^2-6x+6)e^{-x}}$};
\end{tikzpicture}\end{bmlimage}
\end{center}




\eqref{itEstFuncB70}


\begin{center}
\begin{bmlimage}\begin{tikzpicture}[yscale=0.7]
\draw [ ->] (0,0)--(2.5,0);
\draw [ ->] (0,-0.3)--(0,1.5) node[left]{$\sqrt[x]{x}$};
\draw [thick, domain=0.2:6, samples=100, <-] plot
(\x,{exp(ln(\x)/\x)});
\fill (2.718,1.444) circle (0.45mm) node[above]{máx. glob.:
$(e,\sqrt[e]{e})$};
\coordinate (A) at (0.539,0.318);
\coordinate (B) at (5.04,1.37);
\fill (A) circle (0.45mm);
\fill (B) circle (0.45mm);
\draw[<-] (A)--(1.2,-0.3) node[right]{pt. infl.:
$(x_1,f(x_1))$};
\draw[<-] (B)--(5.2,1.9) node[right]{pt. infl.:
$(x_2,f(x_2))$};
%\draw[<-] (-0.67,0.9)--(0.2,0.4)node[right]{mín. global: $(\ln \tfrac12,f(\ln
%\tfrac12))$};
%\fill (-1.365,1.033) circle (0.45mm);
%\draw[<-] (-1.4,0.9)--(-1.6,0.5)node[left]{pt. infl.};
%\fill (2.46,4.86) circle (0.45mm);
%\draw[<-] (2.55,4.7)--(3,4)node[right]{pt. infl.};
\draw[dashed] (0,1)--(6,1);
\draw (2,1) node[below right]{Ass. Horiz.: $y=1$};
%\draw (5,2.5)
%node[right]{$\displaystyle{f'(x)=\frac{e^x(2e^x-1)}{e^{2x}-e^x+3}}$};
%\draw (5,0.5)
%node[right]{$\displaystyle{f''(x)=\frac{e^x(12e^x-e^{2x}
%-3)}{(e^{2x}-e^x+3)^2}}$};
\end{tikzpicture}\end{bmlimage}
\end{center}
Os pontos de inflexão são soluções da equação $(1-\ln
x)^2-3x+2x\ln x=0$. Pode ser mostrado que esses satisfazem
$x_1\simeq 0.58$, $x_1\simeq 4.37$.

\eqref{itEstFuncB4}
\begin{center}
\begin{bmlimage}\begin{tikzpicture}[yscale=0.7]
\newcommand{\funcao}[1]{ln( 5*(#1) )/sqrt( 5*(#1) ) }
\draw [ ->] (0,0)--(8,0);
\draw [ ->] (0,-1.3)--(0,2.3) node[left]{$\frac{\ln x}{\sqrt{x}}$};
\draw [thick, domain=0.1:8, samples=100] plot (\x,{\funcao{\x}});
% \draw[dashed] (1,-2) node[left]{$\scriptscriptstyle{x=1}$}--(1,2) ;
\fill ({2.718^2/5},{\funcao{2.718^2/5}}) circle (0.40mm);
\draw ({2.718^2/5},{\funcao{2.718^2/5}}) node[above]{$(e^2,2/e)$};
\fill ({2.718^(8/3)/5},{\funcao{2.718^(8/3)/5}}) circle (0.40mm);
\draw[<-] ({2.718^(8/3)/5+0.1},{\funcao{2.718^(8/3)/5}+0.2})--
({2.718^(8/3)/5+0.3},{\funcao{2.718^(8/3)/5}+1.3})
node[above]{pt. infl.: $(e^{8/3},f(e^{8/3}))$};
\draw (6,2.8)
node[right]{$\displaystyle{f'(x)=\frac{2-\ln x}{2x^{3/2}}}$};
\draw (6,1.5)
node[right]{$\displaystyle{f''(x)=-\frac{\sqrt{x}}{2}\frac{4-\tfrac32 \ln
x}{|x|^3}}$};
\draw[<-] (5,-0.2)--(4,-0.6) node[below]{ass. horiz.: $y=0$};

\end{tikzpicture}\end{bmlimage}
\end{center}

\eqref{itEstFuncB5}
%$\frac{\ln x-2}{(\ln x)^2}$
\begin{center}
\begin{bmlimage}\begin{tikzpicture}[yscale=0.5]
\newcommand{\funcao}[1]{( ln( (#1) ) -2)/ ( (ln( (#1) ))^2 ) }
\draw [ ->] (0,0)--(12,0);
\draw [ ->] (0,-1.3)--(0,2.3) node[left]{$\frac{\ln x-2}{(\ln x)^2}$};
\draw[dashed] (1,1)node[above]{$x=1$}--(1,-5);
\draw [thick, domain=0.1:0.5, samples=100] plot (\x,{\funcao{\x}});
\draw [thick, domain=1.6:11, samples=100] plot (\x,{\funcao{\x}});
% \draw[dashed] (1,-2) node[left]{$\scriptscriptstyle{x=1}$}--(1,2) ;
% \fill ({2.718^2/5},{\funcao{2.718^2/5}}) circle (0.40mm);
% \draw ({2.718^2/5},{\funcao{2.718^2/5}}) node[above]{$(e^2,2/e)$};
% \fill ({2.718^(8/3)/5},{\funcao{2.718^(8/3)/5}}) circle (0.40mm);
% \draw[<-] ({2.718^(8/3)/5+0.1},{\funcao{2.718^(8/3)/5}+0.2})--
% ({2.718^(8/3)/5+0.3},{\funcao{2.718^(8/3)/5}+1.3})
% node[above]{pt. infl.: $(e^{8/3},f(e^{8/3}))$};
% \draw (6,2.8)
% node[right]{$\displaystyle{f'(x)=\frac{2-\ln x}{2x^{3/2}}}$};
% \draw (6,1.5)
% node[right]{$\displaystyle{f''(x)=-\frac{\sqrt{x}}{2}\frac{4-\tfrac32 \ln
% x}{|x|^3}}$};
\draw[<-] (6,0.2)--(5,0.6) node[above]{ass. horiz.: $y=0$};
\draw (7,-1.5) node[right]{máx. global em $(e^4,f(e^4))$};
\draw (7,-3) node[right]{pt. infl. em
$(e^{1+\sqrt{13}},f(e^{1+\sqrt{13}})$};
\draw (5,-4.5) node[right]{$f'(x)=\frac{4-\ln x}{x(\ln x)^3}$,
$f''(x)=\frac{(\ln x)^2-2\ln x-12}{x^2(\ln x)^4}$};
\end{tikzpicture}\end{bmlimage}
\end{center}


\eqref{itEstFuncB2}
Ass. horiz.: $y=\ln 3$. Ass. obl.: $y=2x$.
\begin{center}
\begin{bmlimage}\begin{tikzpicture}[yscale=0.7]
\draw [ ->] (-4,0)--(2.5,0);
\draw [ ->] (0,-1.3)--(0,2.3) node[left]{$\ln(e^{2x}-e^x+3)$};
\draw [thick, domain=-4:2.6, samples=100] plot (\x,{ln(exp(2*\x)-exp(\x)+3)});
\fill (-0.693,1.012) circle (0.45mm);
\draw[<-] (-0.67,0.9)--(0.2,0.4)node[right]{mín. global: $(\ln \tfrac12,f(\ln
\tfrac12))$};
\fill (-1.365,1.033) circle (0.45mm);
\draw[<-] (-1.4,0.9)--(-1.6,0.5)node[left]{pt. infl.};
\fill (2.46,4.86) circle (0.45mm);
\draw[<-] (2.55,4.7)--(3,4)node[right]{pt. infl.};
\draw[dashed] (-4,{ln(3)})node[above]{$y=\ln 3$}--(-1,{ln(3)});
\draw (5,2.5)
node[right]{$\displaystyle{f'(x)=\frac{e^x(2e^x-1)}{e^{2x}-e^x+3}}$};
\draw (5,0.5)
node[right]{$\displaystyle{f''(x)=\frac{e^x(12e^x-e^{2x}
-3)}{(e^{2x}-e^x+3)^2}}$};
\end{tikzpicture}\end{bmlimage}
\end{center}

\eqref{itEstFuncB29} Observe que $(e^{|x|}-2)^3$ é par, e não
é derivável em $x=0$.
\begin{center}
\begin{bmlimage}\begin{tikzpicture}[yscale=0.7]
\draw [ ->] (-2,0)--(2,0);
\draw [ ->] (0,-1.3)--(0,2.3) node[above]{$(e^{|x|}-2)^3$};
\draw [thick, domain=0:1.2, samples=50] plot
(\x,{(exp(\x)-2)^3});
\draw [thick, domain=0:1.2, samples=50] plot
(-\x,{(exp(\x)-2)^3});
%\fill (-0.693,1.012) circle (0.45mm);
\draw[<-] (0.1,-1.05)--(1,-1.3) node[right]{mín. global: $(0,-1)$};
%\fill (-1.365,1.033) circle (0.45mm);
\draw[<-] (0.683,-0.1)--({0.693+0.5},-0.5)
node[right]{pt. infl.: $(\ln 2,0)$};
\draw[<-] (-0.683,-0.1)--({-0.693-0.5},-0.5)
node[left]{pt. infl.: $(-\ln 2,0)$};
\fill (0.693,0) circle (0.45mm);
\fill (-0.693,0) circle (0.45mm);
%\draw[<-] (2.55,4.7)--(3,4)node[right]{pt. infl.};
%\draw[dashed] (-4,{ln(3)})node[above]{$y=\ln 3$}--(-1,{ln(3)});
%\draw (5,2.5)
%node[right]{$\displaystyle{f'(x)=\frac{e^x(2e^x-1)}{e^{2x}-e^x+3}}$};
%\draw (5,0.5)
%node[right]{$\displaystyle{f''(x)=\frac{e^x(12e^x-e^{2x}
%-3)}{(e^{2x}-e^x+3)^2}}$};
\end{tikzpicture}\end{bmlimage}
\end{center}

\eqref{itEstFuncB33}
\begin{center}
\begin{bmlimage}\begin{tikzpicture}[yscale=0.7]
\draw [ ->] (-3,0)--(3,0);
\draw [ ->] (0,-0.3)--(0,1.4) node[above]{$\frac{e^x}{e^x-x}$};
\draw [thick, domain=-3:3, samples=50] plot
(\x,{exp(\x)/(exp(\x)-\x)});
\draw[dashed] (0,1)--(3,1);
%\draw [thick, domain=0:1.2, samples=50] plot
%(-\x,{(exp(\x)-2)^3});
%\fill (-0.693,1.012) circle (0.45mm);
%\draw[<-] (0.1,-1.05)--(1,-1.3) node[right]{mín. global: $(0,-1)$};
%\fill (-1.365,1.033) circle (0.45mm);
%\draw[<-] (0.683,-0.1)--({0.693+0.5},-0.5)
%node[right]{pt. infl.: $(\ln 2,0)$};
%\draw[<-] (-0.683,-0.1)--({-0.693-0.5},-0.5)
%node[left]{pt. infl.: $(-\ln 2,0)$};
%\fill (0.693,0) circle (0.45mm);
%\fill (-0.693,0) circle (0.45mm);
%\draw[<-] (2.55,4.7)--(3,4)node[right]{pt. infl.};
%\draw[dashed] (-4,{ln(3)})node[above]{$y=\ln 3$}--(-1,{ln(3)});
%\draw (5,2.5)
%node[right]{$\displaystyle{f'(x)=\frac{e^x(2e^x-1)}{e^{2x}-e^x+3}}$};
%\draw (5,0.5)
%node[right]{$\displaystyle{f''(x)=\frac{e^x(12e^x-e^{2x}
%-3)}{(e^{2x}-e^x+3)^2}}$};
\end{tikzpicture}\end{bmlimage}
\end{center}

\eqref{itEstFuncB33a}
\begin{center}
\begin{bmlimage}\begin{tikzpicture}[yscale=1]
\draw [ ->] (-4,0)--(4,0);
\draw [ ->] (0,-0.3)--(0,3.3) node[above right]{$\displaystyle{
\arcos(\frac{1-x^2}{1+x^2})}$};
\draw [thick, domain=-3.7:3.7, samples=51] plot
(\x,{3.1415/180*acos((1-\x*\x)/(1+\x*\x))});
\draw[dashed] (-4,3.14)--(4,3.14) node[right]{$y=\pi$};
\draw (5,1.7) node{Obs: a função não é derivável em $x=0$!};
\end{tikzpicture}\end{bmlimage}
\end{center}

\eqref{itEstFuncB36}

\begin{center}
\begin{bmlimage}\begin{tikzpicture}[yscale=0.9]
%\newcommand{\funcao}[1]{(abs(#1))^(0.8)*(abs((#1)-1))^(0.2)*(-1)};
\draw [ ->] (-3,0)--(3,0);
\draw [ ->] (0,-0.3)--(0,1.4) node[above]{
$\sqrt[5]{x^4(x-1)}$};
\pgfmathsetmacro{\e}{0.002};
\coordinate (A) at (0.8,-0.606);
\fill (A) circle (0.45mm);
\draw[<-] (0.8,-0.73)--(1.3,-1) node[right]{mín. loc.:
$(\tfrac45,f(\tfrac45))$};
\draw [thick, domain=-2:-\e, samples=50] plot
%%PROBLEMA:
(\x,{-exp(0.8*ln(abs(\x))+0.2*ln(abs(\x-1)))});
\draw [thick, domain=\e:{1-\e}, samples=50] plot
%(\x,{(abs(\x))^(0.8)*(abs(\x-1))^(0.2)*(-1)});
(\x,{-exp(0.8*ln(abs(\x))+0.2*ln(abs(\x-1)))});
\draw [thick, domain={1+\e}:3, samples=50] plot
(\x,{exp(0.8*ln(abs(\x))+0.2*ln(abs(\x-1)))});
\draw[<-] (-0.1,0.1)--(-1.5,1) node[left]{máx. loc.: $(0,0)$};
\draw[thick]
({1-\e},{-exp(0.8*ln(abs(1-\e))+0.2*ln(abs(1-\e-1)))})--
({1+\e},{exp(0.8*ln(abs(1+\e))+0.2*ln(abs(1+\e-1)))});
\draw[dashed] (-2,-2.2)--(3,2.8) node[right]{Ass. obl.:
$y=x-\tfrac15$.};
\end{tikzpicture}\end{bmlimage}
\end{center}
Obs: $f'(x)=f(x)\varphi(x)$, onde
$\varphi(x)=\tfrac15(\tfrac{4}{x}+\tfrac{1}{x-1})$.
A função não é derivável nem em $x=0$, nem em $x=1$
(apesar de ser contínua nesses pontos).
$f''(x)=(\varphi(x)^2+\varphi'(x))f(x)=-\tfrac{4}{25}
\frac{f(x)}{x^2(x-1)^2}$, logo, $f$ é convexa em
$(-\infty,0)$ e $(0,1)$, côncava em $(1,\infty)$.
Essa função possui uma assíntota \emph{oblíqua}:
$y=x-\tfrac15$.
\end{Solution}
\protect \section *{Capítulo \ref {CAP:Integral}}
\begin{Solution}{9.3}
A soma associada dá, usando a fórmula sugerida,
\[
\text{área}(R_n)=\frac{e^0}{n}+\frac{e^{1/n}}{n}
+\frac{e^{2/n}}{n}+\dots+\frac{e^{(n-1)/n}}{n}
=\frac{e-1}{\frac{e^{1/n}-1}{1/n}}\,.
\]
Mas $\lim_{n\to\infty}\frac{e^{1/n}-1}{1/n}=\lim_{t\to
0^+}\frac{e^t-1}{t}=1$. Logo,
$\text{área}(R)=e-1$.
\end{Solution}
\begin{Solution}{9.5}
\eqref{itExFuncArea1} $I(x)=0$ se $x\leq \frac12$, $I(x)=(x-\frac12)$ se
$x>\frac12$
\eqref{itExFuncArea2} $I(x)=-\frac{x^2}{2}+x$
\eqref{itExFuncArea3} $I(x)=x^2-x$.
\end{Solution}
\begin{Solution}{9.6}
\eqref{itExoPrimitTriv0} $-2x+C$
\eqref{itExoPrimitTriv1} $\frac{x^2}{2}+C$
\eqref{itExoPrimitTriv2} $\frac{x^3}{3}+C$
\eqref{itExoPrimitTriv3} $\frac{x^{n+1}}{n+1}+C$
\eqref{itExoPrimitTriv35} $\tfrac{2}{3}(1+x)^{3/2}+C$
\eqref{itExoPrimitTriv5} $\sen x+C$
\eqref{itExoPrimitTriv6} $-\cos x+C$
\eqref{itExoPrimitTriv7} $\frac{1}{2}\sen (2x)+C$
\eqref{itExoPrimitTriv9} $e^x+C$
\eqref{itExoPrimitTriv95} $x+e^{-x}+C$
\eqref{itExoPrimitTriv10} $\tfrac12 e^{2x}+C$
\eqref{itExoPrimitTriv105} $-\tfrac32e^{-x^2}+C$
\eqref{itExoPrimitTriv8} $2\sqrt{x}+C$
\eqref{itExoPrimitTriv4} $\ln x+C$
\eqref{itExoPrimitTriv11} $\arctan x+C$
\eqref{itExoPrimitTriv12} Com $-1<x<1$, $\arcsen x+C$
\end{Solution}
\begin{Solution}{9.8}
Como $\tfrac{x^2}{2}-x$ é primitiva de $f(x)=x-1$, temos
$\int_0^2(x-1)\,dx=(\tfrac{x^2}{2}-x)|_0^2=0$.
Esse resultado pode ser interpretando decompondo a integral em duas partes:
$\int_0^2f(x)\,dx=\int_0^1f(x)\,dx+\int_1^2f(x)\,dx$.
Esboçando o gráfico de $f(x)$ entre $0$ e $2$,
\begin{center}
\begin{bmlimage}\begin{tikzpicture}
\fill[areagrafico] (1,0)--(2,1)--(2,0)--cycle;
\fill[areafuncaoarea] (1,0)--(0,-1)--(0,0)--cycle;
\draw (1.65,0.25) node{$+$};
\draw (0.35,-0.3) node{$-$};
\draw (1,0) node{$\shortmid$} node[above]{$1$};
\draw[dashed] (2,0)node[below]{$2$}--(2,1);
\draw[dashed] (0,0)--(0,-1);
\draw[>=latex, ->] (-0.3,0)--(2.4,0);
\draw[>=latex, ->] (0,-1.2)--(0,1.3);
\draw[thick] (0,-1)--(2,1);
\end{tikzpicture}\end{bmlimage}
\end{center}
Vemos que a primeira parte
$\int_0^1f(x)\,dx=-\tfrac12$ é a contribuição do intervalo em
que $f$ é \emph{negativa}, e é exatamente
compensada pela contribuição da parte \emph{positiva}
$\int_1^2f(x)\,dx=+\tfrac12$.
\end{Solution}
\begin{Solution}{9.9}
Não, a conta não está certa. É porqué a função $\frac{1}{x^2}$ não é
contínua (nem definida) em $0$, ora $0$ pertence ao intervalo de
integração. Logo, o Teorema Fundamental não se aplica.
No entanto, será possível dar um sentido a
$\int_{-1}^2\frac{1}{x^2}\,dx$, usando \emph{integrais impróprias}.
\end{Solution}
\begin{Solution}{9.10}
\eqref{itareaRbas1} $5$,
\eqref{itareaRbas2} $\frac{16}{3}$,
\eqref{itareaRbas3} $\frac{1}{3}$,
\eqref{itareaRbas4} $1$.
\eqref{itareaRbas5} $\tfrac{125}{6}$.
\end{Solution}
\begin{Solution}{9.11}
\mbox{}
\begin{center}
\begin{bmlimage}\begin{tikzpicture}[scale=0.6]
\fill[areagrafico]
(0,-1)--(0.368,-1)--plot[domain=0.368:7.38](\x,{ln(\x)})--(0,2)--cycle;
\draw [thick, domain=0.3:8, samples=80] plot (\x,{ln(\x)}) node[right] {$\ln
x$};
\draw [>=latex, ->] (-0.5,0)--(4,0) node[right] {$x$};
\draw [>=latex, ->] (0,-1.2)--(0,2.5);
\draw [dotted] (0,-1)--(0.368,-1);
\draw [dotted] (0,2)--(7.38,2);
\draw (0,-1) node[left]{$-1$};
\draw (0,2) node[left]{$2$};
\draw (8,0) node[right]{$A=\int_{-1}^2e^ydy=e^2-e^{-1}\,.$};
\end{tikzpicture}\end{bmlimage}
\end{center}
Observe que expressando a área com uma integral com respeito a $x$,
$$A=\int_0^{e^{-1}}(2-(-1))dx+\int_{e^{-1}}^{e^2}(2-\ln x)
dx\,.$$
Essa integral requer a primitiva de $\ln x$, o que não
sabemos (ainda) fazer.
\end{Solution}
\begin{Solution}{9.12}
Consideremos $f_\alpha$ para diferentes valores de $\alpha$:
\begin{center}
\begin{bmlimage}\begin{tikzpicture}[scale=1.3]
\newcommand{\funcao}[2]{ ( exp(-1*(#1))/((#1)^2) )*( (#1)^2 - (#2)^2)}

\foreach \a in {0.3, 0.6,1,2} {
\fill[areagrafico, opacity=0.8] (-\a,0)--
plot[domain=-\a:\a] (\x,{\funcao{\a}{\x}})--(\a,0)--cycle;
}

\foreach \a in {0.3, 0.6,1,2} {
\draw[thick, domain=-\a:\a, samples=50] plot (\x,{\funcao{\a}{\x}});
}

\draw[>=latex,->] (-2.3,0)--(2.3,0);
\draw[>=latex,->] (0,-0.1)--(0,1.3);
\end{tikzpicture}\end{bmlimage}
\end{center}
A área debaixo do gráfico de $f_\alpha$ é dada pela integral
$$
I_\alpha=\int_{-\alpha}^\alpha f_\alpha(x)\,dx=\frac{e^{-\alpha}}{\alpha^2}
\int_{-\alpha}^\alpha(\alpha^2-x^2)\,dx=(\cdots)=\tfrac43 \alpha
e^{-\alpha}\,.$$
Um simples estudo de $\alpha\mapsto I_\alpha$ mostra que o seu máximo é
atingido em $\alpha=1$.
\end{Solution}
\begin{Solution}{9.13}
Como $I_n=\frac{n}{n+1}a^{\frac{n+1}{n}}$, temos $\lim_{n\to \infty}I_n=a$.
Quando $n\to \infty$, o gráfico de $x\mapsto x^{1/n}$ em $\bR_+$ tende
ao gráfico da função constante $f(x)\equiv 1$. Ora, $\int_0^a f(x)\,dx=a$!
\end{Solution}
\begin{Solution}{9.14}
\eqref{itprimitsubst1}
$-\frac{x^4}{4}-\frac{x^3}{3}+\frac{x^2}{2}+x+C$,
\eqref{itprimitsubst2} $\frac{-1}{2x^2}-\frac{\sen (2x)}{2}+C$,
\eqref{itprimitsubst3} $-\frac{1}{7x^7}-\frac{5}{x}+C$,
%\eqref{itprimitsubst4} $-\frac{1}{2}\cos(x^2)+C$,
%\eqref{itprimitsubst5} $\frac{x}{2}+\frac{\sen x \cos x}{2}+C$,
\eqref{itprimitsubst6} $2\tan x+C$.
%\eqref{itprimitsubst7} $\tfrac12\ln(1+x^2)+C$
%\eqref{itprimitsubst8} $-\ln(\cos x)+C$
\end{Solution}
\begin{Solution}{9.15}
\eqref{itprimitsubst40} $\frac{1}{8}(x+1)^8+C$ (Obs: aqui, basta fazer a
substituição $u=x+1$. Pode também fazer sem, mas implica desenvolver um
polinômio de grau $7$!)
\eqref{itprimitsubst400} $\frac{-1}{2(2x+1)}+C$
\eqref{itprimitsubst401} $\frac{1}{8(1-4x)^2}+C$
\eqref{itprimitsubst4} $-\frac{1}{2}\cos(x^2)+C$,
\eqref{itprimitsubst4000} $\frac{1}{2}\sen^2(x)+C$, ou $-\frac{1}{2}\cos^2(x)+C$
\eqref{itprimitsubst45} $2\sen(\sqrt{x})+C$,
\eqref{itprimitsubst5} $\frac{x}{2}+\tfrac14\sen (2x)+C$,
\eqref{itprimitsubst7} $\tfrac12\ln(1+x^2)+C$,
\eqref{itprimitsubst71} $\frac{2}{3}(1+\sen x)^{\frac{3}{2}}+C$
\eqref{itprimitsubst8} $\int \tan x\,dx=\int\frac{\sen x}{\cos
x}\,dx=-\int\frac{(\cos x)'}{\cos
x}\,dx -\ln|\cos x|+C$.
\eqref{itprimitsubst9} $\tfrac32 \ln(1+x^2)+5\arctan x+C$
\eqref{itprimitsubst10} $\frac{1}{\sqrt{2}}\arctan(\frac{x+1}{\sqrt{2}})+C$
\eqref{itprimitsubst12}
Com a substituição $u:=e^x$, $du=e^x dx$,
$\int e^x\tan(e^x)dx=\int \tan u du=-\ln|\cos u|+C=-\ln|\cos(e^x)|+C$.
\eqref{itprimitsubst13} $\frac{1}{2(1+y)^2}-\frac{1}{1+y}+C$
\eqref{itprimitsubst14} $\frac{1}{3}(1+x^2)^{\frac{3}{2}}+C$
\eqref{itprimitsubst15} $\frac{-1}{2(1+x^2)}+C$
\eqref{itprimitsubst11} $-\frac{1}{3\sen^3t}+\frac{1}{\sen t}+C$ (a ideia aqui
é escrever $\frac{\cos^3t}{\sen^4t}=\frac{\cos^2t}{\sen^4t}\cos
t=\frac{1-\sen^2t}{\sen^4t}\cos t$)
\eqref{itprimitsubst16} $\frac{(\sen x)^4}{4}-\frac{(\sen x)^6}{6}$
\end{Solution}
\begin{Solution}{9.16}
\eqref{itttit1}
Com $u=1-x^2$, $du=-2x\,dx$, temos
\begin{align*}
\int \frac{2x^3dx}{\sqrt{1-x^2}}\,dx=-\int
\frac{x^2}{\sqrt{1-x^2}}(-2x)\,dx
&=-\int \frac{1-u}{\sqrt{u}}\,du\\
&=-2\sqrt{u}+\tfrac23 u^{3/2}+C\\
&=-2\sqrt{1-x^2}+\tfrac23 (1-x^2)^{3/2}+C\,.
\end{align*}
\eqref{itttit2}
Completando o quadrado, e fazendo a substituição $u=2x-1$,
\begin{align*}
\int \frac{dx}{\sqrt{x-x^2}}=\int
\frac{dx}{\sqrt{\tfrac14-(x-\tfrac12)^2}}&=
\int \frac{2 dx}{\sqrt{1-(2x-1)^2}}\\
&=\int \frac{du}{\sqrt{1-u^2}}=\arcsen u+C=\arcsen (2x-1)+C\,.
\end{align*}
\eqref{itttit3} Com $u=\ln t$, $\int \frac{\ln x}{x}\,dx=\int
u\,du=\tfrac{u^2}{2}+C=\tfrac12(\ln x)^2+C$
\eqref{itttit4} Com $u=e^x$, $\int e^{e^x}e^x\,dx=\int e^u\,du=e^u+C=e^{e^x}+C$.
\eqref{itttit5} $\int \frac{\sqrt{x}}{1+\sqrt{x}}\,dx=x-2\sqrt{x}+2\ln
(1+\sqrt{x})+C$.
\eqref{itttit6} $\int \tan^2x\,dx=\int(1+\tan^2x-1)\,dx=\tan x-x+C$.
\end{Solution}
\begin{Solution}{9.17}
\eqref{itintpartes1} $\sen x-x\cos x+C$,
\eqref{itintpartes2} $\frac{1}{5}x\sen(5x)+\frac{1}{25}\cos(5x)+C$
\eqref{itintpartes3} Integrando duas vezes por partes:
$$
\int x^2\cos x\,dx=x^2\sen x-\int (2x)\sen x\,dx=x^2\sen x-2\Bigl\{
x(-\cos x)-\int (-\cos x)\,dx\,.
\Bigr\}$$
Portanto $\int x^2\cos x\,dx=x^2\sen x-2(\sen x-x\cos x)+C$.
\eqref{itintpartes4} $(x-1)e^x+C$
\eqref{itintpartes5} $-\tfrac13 e^{-3x}(x^2-\tfrac23 x-\tfrac29)+C$
\eqref{itintpartes6}
\begin{align*}
\int x^3\cos (x^2)\,dx=\int x^2 (x\cos(x^2))\,dx&=x^2(\tfrac12
\sen(x^2))-\int(2x)(\tfrac12 \sen (x^2))\,dx\,\\
&=\tfrac12 x^2 \sen(x^2)+\tfrac12 \cos (x^2)+C\,.
\end{align*}
%\eqref{itintpartes7} $x^2(\ln x-\tfrac12)+C$
\end{Solution}
\begin{Solution}{9.18}
 \eqref{ititnpartmntriv1}
$\int \arctan x dx=x\arctan
x-\int\frac{x}{1+x^2}\,dx=x\arctan x-\tfrac12 \ln (1+x^2)+C$.
\eqref{ititnpartmntriv2} $x(\ln x)^2-2x(\ln x-1)+C$
\eqref{ititnpartmntriv3} $x\arcsen x+\sqrt{1-x^2}+C$
\eqref{ititnpartmntriv4} $\int x\arctan x\,dx=\frac12(x^2\arctan x-x+\arctan x)+C$
\end{Solution}
\begin{Solution}{9.19}
\eqref{itintpartestruc1} $-\frac{e^{-x}}{2}(\sen x+\cos x)+C$
\eqref{itintpartestruc2} $\frac{e^{-st}}{1+s^2}(\sen t- s\cos t)+C$
\eqref{itintpartestruc3} $\frac{x}{2}(\sen (\ln x)-\cos (ln x))+C$
\end{Solution}
\begin{Solution}{9.20}
Chamando $u=\sqrt{x+1}$, temos
$$
\int_0^3e^{\sqrt{x+1}}\,dx=\int_1^2
2ue^u\,du=2\bigl\{ue^u-e^u\bigr\}\big|_1^2=2e^2\,.
$$
Chamando $u=\ln x$, temos $e^u\,du=dx$, e
$$
\int x(\ln x)^2\,dx=\int
u^2e^{2u}\,du=\tfrac{u^2}{2}e^{2u}-\tfrac{u}{2}e^{2u}+\tfrac14 e^{2u}+C\,.
$$
Logo, $\int x(\ln x)^2\,dx=\tfrac12 x^2(\ln x)^2-\tfrac12 x^2\ln
x+\tfrac14x^2+C$.
\end{Solution}
\begin{Solution}{9.21}
Para ter $\frac{1}{x(x^2+1)}=\frac{A}{x}+\frac{B}{x^2+1}$, isto é
$1=A(x^2+1)+Bx$, $A$ e $B$
devem satisfazer às três condições $A=0$, $B=0$, $A=1$, que obviamente é
impossível.
\end{Solution}
\begin{Solution}{9.22}
Para ter $\frac{1}{x(x+1)^2}=\frac{A}{x}+\frac{B}{(x+1)^2}$, isto é
$1=A(x+1)^2+Bx$, $A$ e $B$ precisariam satisfazer às três condições $A=0$,
$2A+B=0$, $A=1$, que obviamente é impossível.
\end{Solution}
\begin{Solution}{9.23}
\eqref{itfracparciais1} $\tfrac{1}{\sqrt{2}}\arctan(\sqrt{2}x)+C$
\eqref{itfracparciais2} Como $\frac{x^5}{x^2+1}=x^3-x+\frac{x}{x^2+1}$, temos
$\int\frac{x^5}{x^2+1}\,dx=\tfrac{x^4}{4}-\tfrac{x^2}{2}+\tfrac12\ln (x^2+1)+C$.
\eqref{itfracparciais3} $\frac{-1}{x+2}+C$

\eqref{itfracparciais30}
A decomposição em frações parciais é da forma
$\frac{1}{x(x+1)}=\frac{A}{x}+\frac{B}{x+1}$.
Colocando no mesmo denominador, $A$ e $B$
tem que satisfazer $1=(A+B)x+A$ para todo $x$. Logo, $A=1$ e $B=-1$. Isto é,
$\frac{1}{x^2+x}=\frac{1}{x}-\frac{1}{x+1}$. Logo,
\begin{align*}
\int \frac{1}{x^2+x}\,dx&=\int \frac{1}{x}\,dx-\int\frac{1}{x+1}\,dx\\
&=\ln |x|-\ln |x+1|+C\,,\quad\quad
\end{align*}
\eqref{itfracparciais31}
O integrante é da forma $\frac{P(x)}{Q(x)}$, em que o grau
de $P$ é menor do que o de $Q$. Além disso, podemos fatorar $x^3+x=x(x^2+1)$. O
polimômio de ordem $2$ tem discriminante negativo. Logo, é irredutível,
e podemos tentar uma decomposição da forma
$$
\frac{1}{x(x^2+1)}=\frac{A}{x}+\frac{Bx+C}{x^2+1}\quad \forall x\,.
$$
Colocando no mesmo denominador, $A$ $B$ e $C$
tem que satisfazer $1=(A+B)x^2+Cx+A$ para todo $x$. Logo, $A=1$, $C=0$, e
$B=-A=-1$. Isto é,
\begin{align*}
\int \frac{1}{x^3+x}\,dx=\int \frac{1}{x}\,dx-\int\frac{x}{x^2+1}\,dx
&=\ln |x|-\int\frac{x}{x^2+1}\,dx\\
&=\ln |x|-\tfrac{1}{2}\ln (x^2+1)+C\,,\quad\quad
\end{align*}
Nesta última integral, fizemos $u=x^2+1$, $du=2x\,dx$.
\eqref{itfracparciais4} Como $\Delta=16>0$, podemos procurar fatorar e fazer uma
separação em frações parciais,
$$\int\frac{dx}{x^2+2x-3}=\int\frac{dx}{(x+3)(x-1)}=-\tfrac14\int\frac{dx}{x+3}
+\tfrac14\int\frac{dx}{x-1}=\tfrac14\ln \Bigl|\frac{x-1}{x+3}\Bigr|+C\,.
$$
\eqref{itfracparciais5} Como $\Delta=-8<0$, o denominador não se fatora.
Completando o quadrado,
$$
\int\frac{dx}{x^2+2x+3}=\int\frac{dx}{(x+1)^2+2}=\tfrac12\int\frac{dx}{(\frac{x+
1}{\sqrt{2}})^2+1}=\tfrac{1}{\sqrt{2}}\arctan\bigl(\frac{x+
1}{\sqrt{2}}\bigr)+C\,.
$$
\eqref{itfracparciais50} Como
$\frac{1}{x(x-2)^2}=\frac{1}{4x}-\frac{1}{4(x-2)}+\frac{1}{2(x-2)^2}$, temos
$$
\int\frac{dx}{x(x-2)^2}=\tfrac14\ln|x|-\tfrac14\ln|x-2|-\frac{1}{2(x-2)}+C\,.
$$
\eqref{itfracparciais51}
$\frac{1}{x^2(x+1)}=\frac{A}{x}+\frac{B}{x^2}+\frac{C}{x+1}$, com $A=-1$,
$B=1$, $C=1$. Logo,
$$
\int\frac{dx}{x^2(x+1)}=-\ln |x|-\tfrac1x+\ln|x+1|+C'\,.
$$

\eqref{itfracparciais7}
Como $t^4+t^3=t^3(t+1)$, procuramos uma separação da forma
$$
\frac{1}{t^4+t^3}=\frac{A}{t}+\frac{B}{t^2}+\frac{C}{t^3}+\frac{D}{t+1}\,\quad
\forall t.
$$
Colocando no mesmo denominador e juntando os termos vemos que $A,B,C,D$ têm que
satisfazer
$$
1=(A+D)t^3+(A+B)t^2+(B+C)t+C\quad\forall t\,.
$$
Identificando os coeficientes obtemos $C=1$, $B=-C=-1$, $A=-B=+1$, e
$D=-A=-1$. Isso implica
\begin{align*}
\int \frac{1}{t^4+t^3}dt&=\int\frac{dt}{t}-\int \frac{dt}{t^2}+\int
\frac{dt}{t^3}-\int \frac{dt}{t+1}\\
&=\ln|t|+\frac{1}{t}-\frac{1}{2t^2}-\ln|t+1|+C\,.
\end{align*}
\eqref{itfracparciais52}
\begin{align*}
\int\frac{dx}{x(x+1)^3}
&=\int
\frac{dx}{x}-\int\frac{dx}{x+1}-\int\frac{dx}{(x+1)^2}-\int\frac{dx}{(x+1)^3}\\
&=\ln|x|-\ln|x+1|+\frac{1}{x+1}+\frac{1}{2(x+1)^2}+C\,.
\end{align*}
\eqref{itfracparciais9} $\int\frac{x^2+1}{x^3+x}\,dx=\int \frac{dx}{x}=\ln|x|+C$
\eqref{itfracparciais10} Com
$u=x^4-1$, $\int\frac{x^3}{x^4-1}\,dx=\tfrac14\ln|x^4-1|+C$ (é bem mais simples do que começar uma
decomposição em frações parciais...)
\eqref{itfracparciais104} Começando com uma integração por partes,
\[
\int \frac{x\ln x}{(x^2+1)^2}\,dx=\frac{-1}{2(x^2+1)}\ln x+\frac12\int
\frac{1}{(x^2+1)x}\,dx\,,
\]
e essa última integral se calcula como no Exemplo \ref{Ex:decomppp}.
\eqref{itfracparciais6} Primeiro, observe que $x^3+1$ possui $x=-1$ como raiz.
Logo, ele pode ser fatorado como $x^3+1=(x+1)(x^2-x+1)$.
Como $x^2-x+1$ tem um discriminante negativo,
procuremos uma decomposição da forma
$$
\frac{1}{x^3+1}=\frac{A}{x+1}+\frac{Bx+C}{x^2-x+1}\,.
$$
É fácil ver que $A$, $B$ e $C$ satisfazem às três condições $A+B=0$,
$-A+B+C=0$, $A+C=1$. Logo, $A=\frac13$, $B=-\frac13$, $C=\frac23$. Escrevendo
\begin{align*}
 \int\frac{dx}{x^3+1}&=\tfrac{1}{3}\int\frac{dx}{x+1}-\tfrac13\int
\frac{x-2}{x^2-x+1}\,dx\\
&=\tfrac{1}{3}\ln|x+1|-\tfrac13\int
\frac{x-2}{x^2-x+1}\,dx\\
\end{align*}
Agora,
\begin{align*}
\int \frac{x-2}{x^2-x+1}\,dx&=\tfrac12\int \frac{2x-1}{x^2-x+1}\,dx-\tfrac{3}{2}
\int\frac{dx}{x^2-x+1}\\
&=\tfrac12 \ln|x^2-x+1|-\tfrac{3}{2}
\int\frac{dx}{x^2-x+1}\\
&=\tfrac12 \ln|x^2-x+1|-\tfrac{4}{\sqrt{3}}\arctan\bigl(\tfrac{2}{\sqrt{3}}
(x-\tfrac12) \bigr)+C\,.
\end{align*}
Juntando,
$$
\int\frac{dx}{x^3+1}=\tfrac{1}{3}\ln|x+1|-\tfrac16\ln|x^2-x+1|+\tfrac{4}{3\sqrt{
3}}\arctan\bigl(\tfrac{2}{\sqrt{3}}(x-\tfrac12) \bigr)+C\,.
$$
\end{Solution}
\begin{Solution}{9.24}
Com a dica, e a substituição $u=\sen x$,
\begin{align*}
\int \frac{dx}{\cos x}=\int\frac{\cos x}{1-\sen^2
x}dx=\int\frac{du}{1-u^2}&=-\int\frac{du}{u^2-1}\\
&=-\tfrac{1}{2}\ln\Bigl|\frac{u-1}{u+1}\Bigr|+C\\
&=\tfrac{1}{2}\ln\Bigl|\frac{1+\sen x}{1-\sen x}\Bigr|+C
\end{align*}
Observe que essa última expressão pode ser transformada da seguinte maneira:
\begin{align*}
\tfrac{1}{2}\ln\Bigl|\frac{\sen x+1}{\sen x-1}\Bigr|=
\tfrac{1}{2}\ln\Bigl|\frac{(1+\sen x)^2}{\cos^2x}\Bigr|=
\ln\Bigl|\frac{1+\sen x}{\cos x}\Bigr|=
\ln\Bigl|\frac{1}{\cos x}+\tan x\Bigr|\,.
\end{align*}
\end{Solution}
\begin{Solution}{9.25}
Como $\Delta=4^2-4\cdot 13<0$, o polinômio $x^2+4x+13$ tem discriminante
negativo. Logo, completando o quadrado:\index{completar um
quadrado}
$x^2+4x+13=(x+2)^2-4+13=(x+2)^2+9$, e
\begin{align*}
\int \frac{x}{x^2+4x+13}dx=\int
\frac{x}{(x+2)^2+9}dx=\tfrac19\int\frac{x}{(\tfrac13(x+2))^2+1}dx
\end{align*}
Com $u=\frac{1}{3}(x+2)$, $x=3u-2$, $3du=dx$,
\begin{align*}
 \tfrac19\int\frac{x}{(\tfrac13(x+2))^2+1}dx&=\tfrac13\int\frac{3u-2}{u^2+1}du\\
&=\tfrac12\int\frac{2u}{u^2+1}du-\tfrac23\int\frac{du}{u^2+1}\\
&=\tfrac12\ln (u^2+1)-\tfrac23 \arctan(u)+C\\
&=\tfrac12\ln (x^2+4x+13)-\tfrac23\arctan(\frac{1}{3}(x+2))+C
\end{align*}
\end{Solution}
\begin{Solution}{9.26}
\eqref{itPotTrig0} $-\cos x+\tfrac13\cos^3x+C$
\eqref{itPotTrig01} Com $u=\sen x$, $\int \cos^5x\,dx=\int
(1-u^2)^2\,du=\cdots=\sen x-\tfrac23\sen^3x+\tfrac15\sen^5x+C$
\eqref{itPotTrig1} Escrevemos
$\int (\cos x\sen x)^5dx=\int
\sen^5x(1-\sen^2x)^2\cos xdx$.
Com $u=\sen x$ dá
\begin{align*}
 \int \sen^5x(1-\sen^2x)^2\cos xdx&=
\int u^5(1-u^2)^2du\\
&=\int (u^5-2u^7+u^9)du\\
&=\frac{u^6}{6}-2\frac{u^8}{8}+\frac{u^{10}}{10}+C\\
&=\frac{\sen^6x}{6}-\frac{\sen^8x}{4}+\frac{\sen^{10}x}{10}+C\,.
\end{align*}
\eqref{itPotTrig10} $-\frac{\cos^{1001}x}{1001}+C$
\eqref{itPotTrig2}
Com $u=\sen t$,
$\int (\sen^2t\cos t) e^{\sen t}dt=\int u^2e^udu$.
Integrando duas vezes por partes e voltando para a variável $t$,
\begin{align*}
 \int u^2e^udu&=u^2e^u-\int (2u)e^udu\\
&=u^2e^u-2\big\{ue^u-\int e^udu\big\}\\
&=u^2e^u-2\{ue^u-e^u\}+C\\
&=e^u(u^2-2u+2)+C\\
&=e^{\sen t}(\sen^2 t-2\sen t+2)+C\,.
\end{align*}
\eqref{itPotTrig3} Com $u=\cos x$,
$\int \sen^3x \sqrt{\cos x}\,dx=-\int(1-u^2)\sqrt{u}\,du=-\int
(u^{1/2}-u^{5/2})\,du=-\tfrac23 u^{3/2}+\tfrac27 u^{7/2}+C=-\tfrac23
(\cos x)^{3/2}+\tfrac27 (\cos x)^{7/2}+C$.
\eqref{itPotTrig4} $\int
\sen^2x\cos^2x\,dx=\int(1-\cos^2x)\cos^2x\,dx=\int\cos^2x\,dx-\int\cos^4x\,dx$,
e essas duas primitivas já foram calculadas anteriormente.
\end{Solution}
\begin{Solution}{9.27}
\eqref{itInttansec3} $\int \sec^2x\,dx=\tan x+C$.
\eqref{itInttansec1} $\int\tan^2x \,dx=\int(\tan^2x+1-1)\,dx=\tan x-x+C$.
\eqref{itInttansec1a} $\int\tan^3x \,dx=\int\tan x(1+\tan^2x)\,dx-\int \tan x\,dx=\tfrac12\tan^2 x-\ln
|\cos x|+C$.
\eqref{itInttansec6} $\int \tan x\sec x\,dx=\sec x+C$.
\eqref{itInttansec2} $\int\tan^4 x\sec^4x\,dx=\int
\tan^4x(\tan^2x+1)\sec^2x\,dx=\int u^4(u^2+1)\,du=
\tfrac17u^7+\tfrac15u^5+C=\tfrac17\tan^7x+\tfrac15\tan^5x+C$.
\eqref{itInttansec4} $\int\cos^5x\tan^5x\,dx=\int
\sen^5x\,dx=\int(1-\cos^2x)^2\sen x\,dx=
-\int(1-u^2)^2\,du=-u+\tfrac23u^3-\tfrac15 u^5+C=
-\cos x+\tfrac23\cos^3x-\tfrac15 \cos^5x+C$.
\eqref{itInttansec7} $\int \sec^5x\tan^3x\,dx=\int \sec^4x(\sec^2x-1)(\tan x\sec
x)\,dx=\int w^4(w^2-1)\,dw=\tfrac17 w^7-\tfrac15 w^5+C=\tfrac17 \sec^7x-\tfrac15
\sec^5x+C$.
\eqref{itInttansec8} Por partes (lembra que
$(\sec\theta)'=\tan\theta\sec\theta$):
\begin{align*}
 \int\sec^2\theta\sec\theta\,d\theta
&=\tan \theta\sec\theta-\int\tan^2\theta\sec\theta\,d\theta\\
&=\tan \theta\sec\theta-\int(\sec^2\theta-1)\sec\theta\,d\theta\,.
\end{align*}
Logo,
$$
\int\sec^3\theta\,d\theta=
\tfrac12\tan\theta\sec\theta+\tfrac12\int\sec\theta\,d\theta\,.
$$
Já calculamos a primitiva de $\sec \theta$ no Exercício
\ref{exo:primunsurseno}:
$\int\sec\theta\,d\theta=\ln\bigl|\sec\theta+\tan \theta\bigr|+C$. Logo,
$$
\int\sec^3\theta\,d\theta=
\tfrac12\tan\theta\sec\theta+\tfrac12\ln\bigl|\sec\theta+\tan \theta\bigr|+C\,.
$$
\end{Solution}
\begin{Solution}{9.28}
De fato,
\begin{align*}
\bigl(\tfrac12\arcsen x+\tfrac12x\sqrt{1-x^2}\bigr)'&=
\tfrac12\frac{1}{\sqrt{1-x^2}}+\tfrac12
\sqrt{1-x^2}+\tfrac12 x\frac{-2x}{2\sqrt{1-x^2}} \\
&=\tfrac12\frac{1-x^2}{\sqrt{1-x^2}}+\tfrac12
\sqrt{1-x^2}\\
&=\tfrac12 \sqrt{1-x^2}+\tfrac12 \sqrt{1-x^2}=\sqrt{1-x^2}\,.
\end{align*}
\end{Solution}
\begin{Solution}{9.29}
A área é dada por $A=4\int_0^{\alpha}\beta\sqrt{1-\frac{x^2}{\alpha^2}}\,dx$.
Com $x=\alpha\sen \theta$,
$$A=4\beta\int_0^{\alpha}\sqrt{1-\frac{x^2}{\alpha^2}}\,dx=
4\alpha\beta \int_0^{\pisobredois}\cos^2\theta\,d\theta=\pi \alpha\beta\,.
$$
Quando $\alpha=\beta=R$, a elipse é um disco de raio $R$, de área $\pi
R\cdot R=\pi R^2$.
\end{Solution}
\begin{Solution}{9.30}
\eqref{itPrimSubSinus1}
Sabemos que $\int \frac{dx}{\sqrt{1-x^2}}=\arcsen x+C$, mas isso pode ser
verificado de novo fazendo a substituição $x=\sen \theta$:
$\frac{dx}{\sqrt{1-x^2}}=\int\frac{1}{\sqrt{1-\sen^2\theta}}\cos\theta\,
d\theta\int d\theta=\theta+C=\arcsen x+C$.
%\eqref{itSubstitTrig00} $\int\frac{x}{\sqrt{3-x^2}}\,dx=...$
\eqref{itSubstitTrig2} Com $x=\sqrt{10}\sen t$ dá
\begin{align*}
\int\frac{x^7}{\sqrt{10-x^2}}dx=\int\frac{\sqrt{10}^7\sen^7t}{\sqrt{10}\cos
t}\sqrt{10}\cos tdt
&=\sqrt{10}^7\int \sen^7tdt
\end{align*}
Uma segunda substituição $u=\cos t$ dá
\begin{align*}
\int \sen^7tdt&=\int (1-\cos^2t)^3\sen tdt\\
\quad&=-\int(1-u^2)^3du\\
&=-\int(1-3u^2+3u^4-u^6)du \\
&=-\Big\{u-u^3+\frac{3}{5}u^5-\frac{1}{7}u^7\Big\}+C
\end{align*}
Para voltar para $x$, observe que $u=\cos
t=\sqrt{1-\sen^2t}=\sqrt{1-(x/\sqrt{10})^2}$.
Logo,
$$
\int\frac{x^7}{\sqrt{10-x^2}}dx=\sqrt{10}^7\Bigl\{-\sqrt{1-\frac{x^2}{10}}
+\sqrt{1-\frac{x^2}{10}}^3-\frac{3}{5}\sqrt{1-\frac{x^2}{10}}^5
+\frac{1}{7}\sqrt{1-\frac{x^2}{10}}^7\Bigr\}+C
$$
\eqref{itPrimSubSinus3} Observe que $\sqrt{1-x^3}$ \emph{não é da forma
$\sqrt{a^2-b^2x^2}$!} Mas com a substituição $u=1-x^3$,
$\int \frac{x^2}{\sqrt{1-x^3}}\,dx=-\tfrac13\int
\frac{du}{\sqrt{u}}=-\tfrac23\sqrt{u}+C=-\tfrac23\sqrt{1-x^3}+C$.
\eqref{itPrimSubSinus2} Aqui uma simples substituição $u=1-x^2$ dá
$\int x\sqrt{1-x^2}\,dx=-\tfrac13(1-x^2)^{3/2}+C$. (Pode também fazer $x=\sen
\theta$, é um pouco mais longo.)
\eqref{itSubstitTrig6} Completando o quadrado,
$3-2x-x^2=4-(x+1)^2$. Chamando $x+1=2\sen \theta$,
\begin{align*}
\int\frac{x}{\sqrt{3-2x-x^2}}\,dx=\int\frac{2\sen
\theta-1}{\sqrt{4-4\sen^2\theta}}2\cos \theta\,d\theta&=
2\int \sen \theta\,d\theta-\int \,d\theta\\
&=-2\cos\theta-\theta+C\,.
\end{align*}
Voltando para $x$, temos
$$
\int\frac{x}{\sqrt{3-2x-x^2}}\,dx=-2\sqrt{1-(\tfrac{x+1}{2})^2}
-\arcsen(\tfrac{x+1}{2})+C\,.$$
\eqref{itSubstitTrig000} Com $x=3\sen \theta$ obtemos
$\int x^2{\sqrt{9-x^2}}\,dx=3^4\int \sen^2\theta\cos^2\theta\,d\theta$.
\end{Solution}
\begin{Solution}{9.31}
\eqref{itSubstitTrig3}
fazendo $x=\tfrac12\tan \theta$ dá
\begin{align*}
 \int \frac{x^3}{\sqrt{4x^2+1}}dx&=\int \frac{(\tfrac12 \tan
\theta)^3}{\sqrt{\sec^2\theta}}
\half\sec^2\theta d\theta\\
&=\tfrac{1}{16} \int\tan^3\theta \sec\theta d\theta\\
&=\tfrac{1}{16} \int (\sec^2\theta-1)\sec\theta \tan\theta d\theta
\end{align*}
Com $w=\sec \theta$, obtemos
$\int (\sec^2\theta-1)\sec\theta \tan\theta
d\theta=\frac{\sec^3\theta}{3}-{\sec\theta}+C$.
Mas $\tan \theta=2x$ implica $\sec \theta=\sqrt{\tan^2\theta+1}=\sqrt{1+4x^2}$.
Logo,
$$
\int \frac{x^3}{\sqrt{4x^2+1}}dx=\frac{(1+4x^2)^{\frac{3}{2}}}{48}
-\frac{\sqrt{1+4x^2}}{16}+C\,.
$$
Observe que pode também rearranjar um pouco a função e fazer por partes:
\begin{align*}
 \int \frac{x^3}{\sqrt{4x^2+1}}dx&=
 \tfrac14\int x^2\frac{8x}{2\sqrt{4x^2+1}}dx\\
&=\tfrac14\Bigl\{x^2\sqrt{4x^2+1}-\int (2x)\sqrt{4x^2+1}dx\Bigr\}\\
&=\tfrac14\Bigl\{x^2\sqrt{4x^2+1}-\tfrac14\frac{(4x^2+1)^{3/2}}{3/2}\Bigr\}+C\,,
\end{align*}
dá na mesma!
\eqref{itSubstitTrig31} Com $x=\tan \theta$, temos
\begin{align*}
\int x^3\sqrt{x^2+1}\,dx&=\int \tan^3\theta\sec^3\theta\,d\theta\\
&=\int (\sec^2\theta-1)\sec^2\theta(\tan\theta\sec\theta)\,d\theta\\
(\text{via }w=\sec\theta)\,
&=\tfrac15\sec^5\theta-\tfrac13\sec^3\theta+C\\
&=\tfrac15(x^2+1)^{5/2}-\tfrac13(x^2+1)^{3/2}+C\,.
\end{align*}
\eqref{itSubstitTrig32} Aqui não precisa fazer substituição trigonométrica:
$u=x^2+a^2$ dá $\int
x\sqrt{x^2+a^2}\,dx=\tfrac12\int\sqrt{u}\,du=\tfrac13u^{3/2}+C=
\tfrac13(x^2+a^2)^{3/2}+C$.
\eqref{itSubstitTrig33} Como $x^2+2x+2=(x+1)^2+1$, a substituição
$x+1=\tan\theta$ dá
$\int
\frac{dx}{\sqrt{x^2+2x+2}}=\int\frac{\sec^2\theta}{\sec\theta}\,
d\theta=\int\sec\theta\,d\theta=\ln|\sec\theta+\tan\theta|+C=\ln\bigl|x+1+\sqrt{
x^2+2x+2}\bigr|+C$.
\eqref{itSubstitTrig10} Apesar da função $\frac{1}{(x^2+1)^3}$ não possuir
raizes, façamos a substituição $x=\tan\theta$:
\begin{align*}
\int\frac{dx}{(x^2+1)^3}&=\int\frac{\sec^2\theta}{(\tan^2\theta+1)^3}\,
d\theta=\int\frac{d\theta}{\sec^4\theta}=\int\cos^4\theta\,d\theta\,.
\end{align*}
Essa última primitiva já foi calculada em \eqref{eq:intcosquatre}:
$\int \cos^4\theta\,d \theta=\tfrac14\sen
\theta\cos^3\theta+\tfrac{3\theta}{8}+\tfrac{3}{16}
\sen(2\theta)+C$. Ora, se $\tan\theta=x$, então
$\sen\theta=\frac{x}{\sqrt{1+x^2}}$ e $\cos\theta=\frac{1}{\sqrt{1+x^2}}$.
Logo,
$$
\int\frac{dx}{(x^2+1)^3}=\frac{x}{4(1+x^2)^2}+\frac38\Bigl\{\arctan
x+\frac{x}{1+x^2}\Bigr\}+C\,.
$$
\eqref{itSubstitTrig100} Com $x=2\tan \theta$,
$\int\frac{dx}{x^2\sqrt{x^2+4}}=\tfrac14\int
\frac{\cos\theta}{\sen^2\theta}\,d\theta=-\frac{1}{4\sen\theta}+C$.
Agora observe que $2\tan \theta=x$ implica $\sen\theta=\frac{x}{\sqrt{x^2+4}}$.
Logo,
$\int\frac{dx}{x^2\sqrt{x^2+4}}=-\frac{\sqrt{x^2+4}}{4x}+C$.
\end{Solution}
\begin{Solution}{9.32}
Já montamos a integral no Exemplo \ref{ex:comprparabdifiss}, e esta pode ser
calculada com os métodos dessa seção:
$L=2\int_0^1\sqrt{1+4x^2}\,dx=\frac{\sqrt{5}}{4}+\frac12\ln(\frac12+\frac{\sqrt{5}}{2})$.
\end{Solution}
\begin{Solution}{9.33}
\eqref{itPrimSubSecante1}
Seja $x=\sqrt{3}\sec \theta$. Então $dx=\sqrt{3}\sec \theta\tan \theta$, e
\begin{align*}
 \int x^3\sqrt{x^2-3}dx&=\int (\sqrt{3}\sec \theta)^3 \sqrt{3}  \tan \theta
\sqrt{3}\sec \theta\tan \theta d\theta\\
&=\sqrt{3}^5\int \{\sec^2\theta \tan^2 \theta\}\sec^2 \theta d\theta\\
(\text{ com }u=\tan \theta)&=\sqrt{3}^5\int (u^2+1)u^2du\\
&=\sqrt{3}^5(u^5/5+u^3/3)+C
\end{align*}
Mas como $\cos \theta=\sqrt{3}/x$, temos (fazer um desenho) $u=\tan
\theta=\sqrt{x^2-3}/\sqrt{3}$. Logo,
$$
\int x^3\sqrt{x^2-3}dx=\tfrac15\sqrt{x^2-3}^5+\sqrt{x^2-3}^3+C
$$
Um outro jeito de calcular essa primitiva é de começar com uma integração por
partes:
\begin{align*}
\int x^3\sqrt{x^2-3}dx=
 \tfrac12\int x^2\,
\big\{2x\sqrt{x^2-3}\big\}dx&=\tfrac12\Big\{x^2\frac{(x^2-3)^{3/2}}{3/2}-\int
2x\frac{(x^2-3)^{3/2}}{3/2}dx\Big\}\\
&=\tfrac12\Big\{x^2\frac{(x^2-3)^{3/2}}{3/2}-\tfrac23\int
2x{(x^2-3)^{3/2}}dx\Big\}\\
&=\tfrac12\Big\{x^2\frac{(x^2-3)^{3/2}}{3/2}-\tfrac23\frac{(x^2-3)^{5/2}}{5/2}
\Big\}+C\\
&=\tfrac13x^2{(x^2-3)^{3/2}}-\tfrac{2}{15}{(x^2-3)^{5/2}}+C\,\,)
\end{align*}
\eqref{itSubstitTrig8} Com $x=a\sec\theta$,
$\int
\frac{dx}{\sqrt{x^2-a^2}}\,dx=\int\sec\theta\,
d\theta=\ln|\sec\theta+\tan\theta|+C$. Como $\cos\theta=\frac{a}{x}$,
\begin{center}
\begin{bmlimage}\begin{tikzpicture}[scale=1.5]
\draw (0,0)--(2,1) node[midway, above, sloped]{$x$}--(2,0)
node[midway, right]{$\sqrt{x^2-a^2}$} --(0,0) node[midway,
below]{$a$};
\draw (0.4,0) arc (0:26.56:0.4);
\draw (0.56,0.13) node{$\theta$};
\draw (4.8,0.5) node{$\Rightarrow\,\,\tan\theta=\frac{\sqrt{x^2-a^2}}{a}$};
\end{tikzpicture}\end{bmlimage}
\end{center}
Logo,
$\int
\frac{dx}{\sqrt{x^2-a^2}}\,dx=\ln|\tfrac{x}{a}+\tfrac{\sqrt{x^2-a^2}}{a}|+C$.
\eqref{itSubstitTrig1}
Com $x=\sec \theta$, $dx=\sec \theta\tan \theta
d\theta$:
\begin{align*}
\int\frac{x^3}{\sqrt{x^2-1}}dx&=\int
\frac{\sec^3\theta}{\tan \theta}\sec \theta\tan \theta d\theta\\
&=\int \sec^2\theta \sec^2\theta d\theta\\
&=\int (\tan^2\theta+1)\sec^2\theta d\theta\\
(u\pardef \tan \theta)\quad&=\int (u^2+1)du\\
&=\frac{u^3}{3}+u+C\\
&=\frac{\tan^3\theta}{3}+\tan\theta+C\,.
\end{align*}
Mas $\sec \theta=x$ implica $\tan\theta=\sqrt{x^2-1}$.
Logo,
$$
\int\frac{x^3}{\sqrt{x^2-1}}dx=
\frac{1}{3}(x^2-1)^{\tfrac32}+\sqrt{x^2-1}+C\,.
$$

\end{Solution}
\protect \section *{Capítulo \ref {CAP:Applicacoes}}
\begin{Solution}{10.1}
 Representando a metade superior do círculo de raio $R$ centrado na origem com a função $f(x)=\sqrt{R^2-x^2}$, podemos expressar o
comprimento da circunferência como
$$
2\int_{-R}^R\sqrt{1+[(\sqrt{R^2-x^2})']^2}\,dx=2R\int_{-R}^R\frac{dx}{\sqrt{R^2-x^2}}=2R\int_{-1}^1\frac{du}{\sqrt{1-u^2}}=2\pi R\,.
$$
\end{Solution}
\begin{Solution}{10.2}
Lembrando que $\cosh'(x)=\senh x$, que $\cosh^2 x-\senh^2x=1$, e que $\cosh x$ é par,
\begin{align*}
 L=\int_{-1}^1\sqrt{1+(\senh x)^2}\,dx=2\int_{0}^1\cosh x\,dx=2\senh (1)=e-e^{-1}\,.
\end{align*}
\end{Solution}
\begin{Solution}{10.3}
O comprimento é dado por $L=\int_0^1\sqrt{1+e^{2x}}\,dx$.
Se $u=\sqrt{1+e^{2x}}$, então $dx=\frac{u}{u^2-1}du$, logo
$$L=\int_{\sqrt{2}}^{\sqrt{1+e^4}}\frac{u^2}{u^2-1}du
=\int_{\sqrt{2}}^{\sqrt{1+e^4}}1\,du+\int_{\sqrt{2}}^{\sqrt{1+e^4}}\frac{du}{
u^2-1}\,.
$$
Essa última integral pode ser calculada como no Exemplo \ref{Ex:unsurxdeuxmun}:
$\int\frac{du}{
u^2-1}=\tfrac{1}{2}\ln\Bigl|\frac{u-1}{u+1}\Bigr|+C$. Logo,
$$
L=\sqrt{1+e^4}-\sqrt{2}+\tfrac12\ln\Bigl[\frac{\sqrt{1+e^4}-1}{\sqrt{1+e^4}
+1}\cdot\frac{\sqrt{2}+1}{\sqrt{2}-1}\Bigr]\,.
$$
\end{Solution}
\begin{Solution}{10.4}
\eqref{itexsolrev1} A esfera pode ser obtida girando o semi-disco,
delimitado pelo gráfico da função
$f(x)=\sqrt{r^2-x^2}$, $x\in [-r,r]$, em torno do eixo $x$.
\eqref{itexsolrev2} O cilíndro pode ser obtido girando o gráfico da função
constante $f(x)=r$, no intervalo $[0,h]$.
\eqref{itexsolrev3} O cubo não é um sólido de revolução.
\eqref{itexsolrev4} O cone pode ser obtido girando o gráfico da função
$f(x)=\frac{r}{h}x$ (ou $f(x)=r-\frac{r}{h}x$), no intervalo $[0,h]$.
\end{Solution}
\begin{Solution}{10.5}
 $11\pi$
 
\end{Solution}
\begin{Solution}{10.6}
$\tfrac{\pi}{6}$.
\end{Solution}
\begin{Solution}{10.8}
A área é dada por
$$\int_{\pi/2}^\pi\sen (x)dx=-\cos (x)|^{\pi}_{\pi/2}=-(-1)-0=1\,.$$
Girando em torno do eixo $x$:
$V_1=\int_{\pi/2}^{\pi}\pi(\sen x)^2\,dx$.
Ou, com as cascas: $V_1=\int_0^12\pi y (\pi/2-\arcsen y)\,dy$.
Em torno da reta $x=\pi$, usando as cascas:
$V_2=\int_{\pi/2}^\pi2\pi(\pi-x)\sen x\,dx$.
Sem usar as cascas:
$V_2=\pi(\tfrac{\pi}{2})^2\cdot 1-\int_0^1\pi (\arcsen y)^2\,.dy$.
\end{Solution}
\begin{Solution}{10.9}
O cone pode ser (tem vários jeitos, mas esse é o mais simples)
obtido girando o gráfico da função $f(x)=\frac{R}{H}x$, $0\leq x\leq H$, em
torno do eixo $x$. Logo,
$$
V=\int_0^H\pi\big(\frac{R}{H}x\Big)^2dx=\pi\frac{R^2}{H^2}\int_0^Hx^2dx=
\pi\frac{R^2}{H^2}\frac{H^3}{3}=\frac{1}{3}\pi R^2H \,\,$$
Obs: pode também rodar o gráfico da função $f(x)=-\frac{H}{R}x+H$, $0\leq x\leq
R$, em torno do eixo $y$.
\end{Solution}
\begin{Solution}{10.10}
O volume é dado por $V=\int_1^e\pi(\sqrt{x}\ln x)^2dx$. Integrando duas vezes
por partes, obtem-se
\begin{align*}
\int x(\ln x)^2dx&=\frac{x^2}{2}(\ln x)^2-\int \frac{x^2}{2}2(\ln
x)\frac{1}{x}dx\\
&=\frac{x^2}{2}(\ln x)^2-\int x\ln xdx\\
&=\frac{x^2}{2}(\ln x)^2-\big\{\frac{x^2}{2}\ln x-\int\frac{x^2}{2}\frac{1}{x}dx
\big\}\\
&=\frac{x^2}{2}(\ln x)^2-\frac{x^2}{2}\ln x+\frac{x^2}{4}+C
\end{align*}
Logo, $V=\pi\frac{e^2-1}{4}$.
\end{Solution}
\begin{Solution}{10.11}
\eqref{itexorotsol1}
Cil.: $\int_0^1\pi (x^2)^2\,dx$,
Casc.:
$\int_0^12\pi y(1-\sqrt{y})\,dy$.
\eqref{itexorotsol2}
Cil.: $\int_0^1\pi(1^2-(1-x^2)^2)\,dx$
Casc.: $\int_0^12\pi(1-y)(1-\sqrt{y})\,dy$,
\eqref{itexorotsol3}
Cil.: $\int_0^1\pi((1+x^2)^2-1^2)\,dx$
Casc.: $\int_0^1 2\pi(1+y)(1-\sqrt{y})\,dy$
\eqref{itexorotsol4}
Cil.: $\int_0^1\pi(1^2-\sqrt{y}^2)\,dy$
Casc.: $\int_0^12\pi x\cdot x^2\,dx$
\eqref{itexorotsol5}
Cil. $\int_0^1\pi(1-\sqrt{y})^2\,dy$
Casc.: $\int_0^12\pi(1-x)x^2\,dx$
\eqref{itexorotsol6}
Cil.: $\int_0^1\pi(2^2-(1+\sqrt{y})^2)\,dy$
Casc. $\int_0^12\pi(1+x)x^2\,dx$
\end{Solution}
\begin{Solution}{10.12}
Com o método dos cilíndros,
$$V=\int_1^3\pi 2^2dx-\int_1^3\pi\big(2-(1-(x-2)^2)\big)^2dx\,\,.$$
OU, usando o método das cascas,
$$
V=\int_0^12\pi(2-y)2\sqrt{1-y}dy\,.
$$
OU, transladando o gráfico da função, e girando a nova região (finita,
delimitada pela nova curva $y=-1-x^2$ e o eixo $x$),
$$V=\int_{-1}^{+1}\pi 2^2dx-\int_{-1}^{+1}\pi(-1-x^2)^2dx\,.$$
\end{Solution}
\begin{Solution}{10.13}
O volume é dado pela integral
\begin{align*}
V=\int_{-1}^{+1}\pi \cosh^2xdx&=\pi\int_{-1}^{+1}\frac{e^{2x}+2+e^{-2x}}{4}dx\\
&=\frac{\pi}{4}\Big\{
\frac{e^{2x}}{2}+2x-\frac{e^{-2x}}{2}
\Big\}_{-1}^{+1}\\
&=\frac{\pi}{4}\big\{e^2+4-e^{-2}\big\}
\end{align*}
\end{Solution}
\begin{Solution}{10.14}
Em torno da reta $x=\pi$:
$$
V=\int_{\pi/2}^{\pi}2\pi(\pi-x)|\cos x|\,dx\,,\quad\text{ ou }
\quad V=\int_{-1}^0\pi(\pi-\arcos y)^2\,dy\,.
$$
Em torno da reta $y=-1$:
$$
V=\int_{\pi/2}^\pi \pi\cdot 1^2\,dx-\int_{\pi/2}^\pi\pi(\cos
x -(-1))^2\,dx\,,\quad\text{ ou }
\quad V=\int_{-1}^02\pi (y-(-1))(\pi-\arcos y)\,dy\,.
$$
\end{Solution}
\begin{Solution}{10.16}
Se trata de mostrar que
a área lateral de um cone truncado de raios $r\leq R$
e de altura $h$ é dada por
$$
A=\pi (R+r)\sqrt{h^2+(R-r)^2}\,.
$$
De fato, fazendo o corte,
\begin{center}
\begin{bmlimage}\begin{tikzpicture}
\coordinate (A) at (0,0);
\coordinate (B) at (0,1);
\coordinate (C) at (0,3);
\coordinate (D) at (1.333,1);
\coordinate (E) at (2,0);
\draw (A)--(B) node[midway, left]{$h$}--
(C)--(E)--(A) node[midway, above]{$R$};
\draw (B)--(D) node[midway, above]{$r$};
\draw (C) node[left]{$C$};
\draw (D) node[right]{$D$};
\draw (E) node[right]{$E$};
\end{tikzpicture}\end{bmlimage}
\end{center}
Chamando a distância $CD$ de $l$, e a distância $CE$ de $L$, temos
$A=\pi R L-\pi rl$. Uma conta elementar mostra que
$l=\frac{r}{R-r}\sqrt{h^2+(R-r)^2}$, e que
$L=\frac{R}{R-r}\sqrt{h^2+(R-r)^2}$.
Isso dá a fórmula desejada.
\end{Solution}
\begin{Solution}{10.17}
Como a esfera é obtida girando o gráfico de
$f(x)=\sqrt{R^2-x^2}$, a sua área é dada por
$$
A=2\pi\int_{-R}^R\sqrt{R^2-x^2}\sqrt{1+\bigl(\sqrt{R^2-x^2}'\bigr)^2}\,dx
=2\pi R\int_{-R}^R\,dx= 4\pi R^2\,.
$$
\end{Solution}
\protect \section *{Capítulo \ref {CAP:Improprias}}
\begin{Solution}{11.1}
\eqref{itImproprias0} Com $u=x-2$,
$\int_3^\infty\frac{dx}{x-2}=\lim_{L\to \infty}\int_3^L\frac{dx}{x-2}=
\lim_{L\to \infty}\int_1^{L-2}\frac{du}{u}=\lim_{L\to \infty}\ln(L-2)=\infty
$, diverge.
\eqref{itImproprias1} Diverge (é a área da região contida entre a parábola
$x^2$ e o eixo $x$!)
\eqref{itImproprias2} $\int_{1}^\infty
\tfrac{dx}{x^7}=\lim_{L\to\infty}\int_{1}^L \tfrac{dx}{x^7}=
\tfrac16\lim_{L\to\infty}\{1-\frac{1}{L^6}\}=\tfrac16$, logo converge.
\eqref{itImproprias3} Como $\int_0^L\cos x\,dx=\sen L$, e que $\sen L$ não
possui limite quando $L\to \infty$, a integral imprópria $\int_0^\infty\cos
x\,dx$ diverge.
\eqref{itImproprias4} $\int_{0}^\infty \frac{dx}{x^2+1}=\frac{\pi}{2}$, logo
converge.
\eqref{itImproprias5} Temos
$\frac{1}{x^2+x}=\frac{1}{x(x+1)}=\frac{1}{x}-\frac{1}{x+1}$, logo
$$\int_1^L\frac{dx}{x^2+x}=\{\ln x\}|_1^L-\{\ln(x+1)\}|_1^L
=\ln L-\ln(L+1)+\ln 2\,.
$$
Mas como $\lim_{L\to \infty}\{\ln L-\ln(L+1)\}=
\lim_{L\to \infty}\ln\frac{L}{L+1}=\ln 1=0$, temos
$\int_{1}^\infty \frac{dx}{x^2+x}=\ln 2<\infty$, logo converge.
\eqref{itImproprias6} converge.
\eqref{itImproprias65} Com $u=\ln x$, $\int\frac{\ln x}{x}\,dx=\int
u\,du=\frac{u^2}{2}+C$, logo $\int_{3}^\infty \frac{\ln x}{x}\,dx$ diverge.
\eqref{itImproprias7} converge (pode escrever $x^4=u^2$, onde $u=x^2$)
\end{Solution}
\begin{Solution}{11.2}
\eqref{itTRansfLapl1} $L(s)=\frac{k}{s}$.
\eqref{itTRansfLapl2} $L(s)=\frac{1}{s^2}$.
\eqref{itTRansfLapl4} Integrando duas vezes por partes, é fácil
verificar que $L(s)$ satisfaz $L(s)=\frac{1}{s}(\frac{1}{s}-\frac{1}{s}L(s))$.
Logo, $L(s)=\frac{1}{1+s^2}$.
\eqref{itTRansfLapl5} $L(s)=\frac{1}{s+\alpha}$.
\end{Solution}
\begin{Solution}{11.3}
A função tem domínio $\bR$, é ímpar e possui a assíntota horizontal
$y=0$, a
direita e esquerda.
A sua derivada vale $f'(x)=\frac{1-x^2}{(x^2+1)^2}$. Logo, $f$ decresce em
$(-\infty,-1]$, possui um mínimo local em $(-1,\tfrac{1}{2})$, cresce em
$[-1,+1]$, possui um
um máximo local em $(+1,\tfrac{1}{2})$, e decresce em $[1,+\infty)$.
A derivada segunda vale $f''(x)=\frac{2x(x^2-3)}{(x^2+1)^3}$. Logo, $f$ possui
três pontos de inflexão: em $(-\sqrt{3},-\frac{\sqrt{3}}{4})$,
$(0,0)$ e $(\sqrt{3},\frac{\sqrt{3}}{4})$, e é côncava em
$(-\infty,-\frac{\sqrt{3}}{4}]$, convexa em $[-\tfrac{\sqrt{3}}{4},0]$, côncava
em $[0,\tfrac{\sqrt{3}}{4}]$, e convexa em
$[\tfrac{\sqrt{3}}{4},+\infty)$.
\begin{center}
\begin{bmlimage}\begin{tikzpicture}
\newcommand{\funcao}[1]{( 2*(#1)/( (#1)^2 + 1 ))}
\draw[>=latex, ->] (-6,0)--(6,0);
\draw[>=latex, ->] (0,-1.5)--(0,1.5);
\draw[thick, domain=-6:6, samples=70] plot (\x,{\funcao{\x}});
\fill[areagrafico] (0,0)--plot[domain=0:5.5,
samples=50](\x,{\funcao{\x}})--(5.5,0)--cycle;
\draw[thick, domain=-6:6, samples=70] plot (\x,{\funcao{\x}});
\end{tikzpicture}\end{bmlimage}
\end{center}
Vemos que a área procurada é dada pela integral imprópria
$$
\int_0^\infty\frac{x}{x^2+1}dx=\lim_{L\to\infty}\int_0^L\frac{x}{x^2+1}dx=\lim_{
L\to\infty} \ln (L^2+1)=+\infty\,.
$$
\end{Solution}
\begin{Solution}{11.4}
$f$ tem domínio $\bR$, e é sempre positiva. Já que
$$
\lim_{x\to +\infty} \frac{e^x}{1+e^x}=\lim_{x\to+\infty}
\frac{1}{1+e^{-x}}=1\,,\quad
\lim_{x\to-\infty} \frac{e^x}{1+e^x}=0\,,
$$
$f$ tem duas assíntotas horizontais: a reta $y=0$ a esquerda, e a reta $y=1$ a
direita.
Como $f'(x)=\frac{e^x}{(1+e^x)^2}$ é sempre positiva, $f$ é crescente em todo
$x$ (não possui mínimos ou máximos locais). Como
$f''(x)=\frac{e^x(1-e^x)}{(1+e^x)^2}$,
e que essa é positiva quando $x\leq 0$, negativa quando $x\geq 0$, temos que $f$
é convexa em $(-\infty,0]$, côncava em $[0,\infty)$, e possui um ponto de
inflexão em $(0,\tfrac12)$:
\begin{center}
\begin{bmlimage}\begin{tikzpicture}
\newcommand{\funcao}[1]{(exp(#1))/(1+ exp(#1))}
\draw[>=latex, ->] (-6,0)--(6,0);
\draw[>=latex, ->] (0,-0.2)--(0,1.5);
\draw[dashed] (0,1)--(6,1);
\fill[areagrafico] (0,0.5)--plot[domain=0:5.5,
samples=50](\x,{\funcao{\x}})--(5.5,1)--(0,1)--cycle;
\draw[thick, domain=-6:6, samples=60] plot (\x,{\funcao{\x}});
\end{tikzpicture}\end{bmlimage}
\end{center}

A área procurada é dada pela integral imprópria
$$\int_0^\infty\Big\{1-\frac{e^x}{1+e^x}\Big\}dx=\int_0^\infty\frac{1}{1+e^x}
dx$$
Com $u=e^x+1$ dá $du=e^x\,dx=(u-1)\,dx$, e
$$
\int \frac{1}{1+e^x}dx=\int\frac{1}{u(u-1)}du\,.
$$
A decomposição desta última fração dá
$$
\int\frac{1}{u(u-1)}du=-\int\frac{du}{u}+\int \frac{du}{u-1}=-\ln|u|+\ln|u-1|+C
$$
Logo,
\begin{align*}
\int_0^\infty\frac{1}{1+e^x}dx=\lim_{L\to\infty}\int_0^L\frac{1}{1+e^x}dx=
&\lim_{L\to\infty}\Big\{
-\ln (e^x+1)+\ln e^x
\Big\}_0^L\\
&=\lim_{L\to\infty}\Big\{
-\ln (1+e^{-x})
\Big\}_0^L\\
&=\ln 2
\end{align*}
\end{Solution}
\begin{Solution}{11.5}
Considere por exemplo a seguinte função $f$:
\begin{center}
\begin{bmlimage}\begin{tikzpicture}
\pgfmathsetmacro{\n}{7};
\draw[>=latex, ->] (-0.2,0)--({\n+0.5},0);
\draw[>=latex, ->] (0,-0.2)--(0,1.3);
\draw (0,1) node{$-$} node[left]{$1$};
\foreach \k in {1,...,\n} {
\draw (\k,0) node{$\shortmid$} node[below]{$\k$};
\coordinate (A) at (\k,1);
\coordinate (B) at ({\k-(1/(2^(\k)))},0);
\coordinate (C) at ({\k+(1/(2^(\k)))},0);
\fill[areagrafico] (B)--(A)--(C)--cycle;
\draw[thick] (B)--(A)--(C);
}
\end{tikzpicture}\end{bmlimage}
\end{center}
Fora dos triângulos, $f$ vale zero.
O primeiro triângulo tem base de largura $1$, o segundo $\frac{1}{2}$, o
$k$-ésimo $\frac{1}{2^{k-1}}$, etc. Logo, a integral de $f$ é igual à soma
das áreas dos triângulos:
$$
\int_0^\infty f(x)\,dx=\tfrac12+\tfrac14+\tfrac18+\tfrac{1}{16}+\dots=1\,.
$$
Assim, a integral imprópria converge. Por outro lado, já que $f(k)=1$ para
todo inteiro positivo $k$, $f(x)$ não tende a zero quando $x\to \infty$.
\end{Solution}
\begin{Solution}{11.7}
\eqref{itIntImpropp1} Como $\frac{1}{\sqrt{x}^\alpha}=\frac{1}{x^{p}}$ com
$p=\alpha/2$, a integral
converge se e somente se $\alpha>2$.
\eqref{itIntImpropp2} Defina $p:=\alpha^2-3$.
Pelo Teorema \ref{Teo:ConvSerieHarmon}, sabemos que a integral converge se
$p>1$, diverge caso contrário. Logo, a integral converge se
$\alpha>2$ ou $\alpha<-2$, e ela diverge se $-2\leq \alpha\leq 2$.
\eqref{itIntImpropp3} Converge se e somente se $\alpha>1/2$ (pode fazer $u=\ln
x$).
\end{Solution}
\begin{Solution}{11.8}
O volume do sólido é dado pela integral imprópria
$$
V=\pi\int_1^\infty\Bigl(\frac{1}{x^q}\Bigr)^2\,dx=\pi\int_1^\infty\frac{dx}{
x^{2q}}\,.
$$
Pelo Teorema \ref{Teo:ConvSerieHarmon}, essa integral converge se $2q>1$ (isto
é se $q>\tfrac12$), diverge caso contrário.
\end{Solution}
\begin{Solution}{11.9}
\eqref{itIntCompar1} Como $x^2+x\geq x^2$ para
todo $x\in [1,\infty)$, temos também $\frac{1}{x^2+x}\leq \frac{1}{x^2}$ neste
intervalo, logo
$\int_1^\infty\frac{dx}{x^2+x}\leq
\int_1^\infty\frac{dx}{x^2}<\infty$, converge.
\eqref{itIntCompar2} Como $x+1\geq x$ para todo $x\geq 1$,
$\int_1^\infty
\frac{dx}{\sqrt{x}(x+1)}\leq
\int_1^\infty \frac{dx}{\sqrt{x}x}= \int_1^\infty \frac{dx}{x^{3/2}}<\infty$,
converge.
\eqref{itIntCompar3} $\int_0^\infty \frac{dx}{1+e^x}\leq \int_0^\infty
e^{-x}\,dx<\infty$, converge.
\eqref{itIntCompar4} $\int_1^\infty \frac{e^x}{e^x-1}\,dx\geq
\int_1^\infty \frac{e^x}{e^x}\,dx=\int_1^\infty dx=\infty$, diverge.
\eqref{itIntCompar5} Como
$\int_0^\infty\frac{dx}{2x^2+1}=\int_0^1\frac{dx}{2x^2+1}
 +\int_1^\infty\frac{dx}{2x^2+1}$ e
$\int_1^\infty\frac{dx}{2x^2+1}\leq
\int_1^\infty\frac{dx}{2x^2}<\infty$, temos que $\int_0^\infty\frac{dx}{2x^2+1}$
converge.
\eqref{itIntCompar6} Escrevendo
$\tfrac{1}{x^2-1}=\tfrac{1}{x^2}\tfrac{x^2}{x^2-1}$, e observando que o máximo
da função $\tfrac{x^2}{x^2-1}$ no intervalo $[3,\infty)$ é $\tfrac98$, temos
$\int_3^\infty\frac{dx}{x^2-1}\leq \tfrac98
\int_3^\infty\frac{dx}{x^2}<\infty$, logo a integral converge.
Um outro jeito de fazer é de observar que se $x\geq 3$, então $x^2-1\geq x^{3/2}$.
\eqref{itIntCompar22}
Como $\sqrt{x^2+1}\geq \sqrt{x^2}=x$ em todo o intervalo de integração,
$\int_1^\infty\frac{\sqrt{x^2+1}}{x^2}dx\geq
\int_1^\infty\frac{x}{x^2}dx=\int_1^\infty\frac{1}{x}dx$.
Como aqui é uma integral do tipo $\int_1^\infty\frac{1}{x^p}dx$ com $p=1$, ela é
divergente. Logo, pelo critério de comparação,
$\int_1^\infty\frac{\sqrt{x^2+1}}{x^2}dx$ {diverge} também.\\
% (Obs: pode também calcular a primitiva da função, o que dá mais trabalho. Por
% exemplo, por partes,
% \begin{align*}
% \int\frac{1}{x^2}\sqrt{x^2+1}dx&=\frac{-1}{x}\sqrt{x^2+1}-\int(\frac{-1}{x}
% )\frac{2x}{2\sqrt{x^2+1}}dx\\
% &=\frac{-1}{x}\sqrt{x^2+1}+\int\frac{1}{\sqrt{x^2+1}}dx
% \end{align*}
% Usando $x=\sec\theta$ nesta última integral dá
% $\int\frac{1}{\sqrt{x^2+1}}dx=\int \sec\theta
% d\theta=(\cdots)=\ln|x+\sqrt{x^2+1}|+C$. Logo,
% $$\int\frac{1}{x^2}\sqrt{x^2+1}dx=\ln|x+\sqrt{x^2+1}|-\frac{\sqrt{x^2+1}}{x}+C\,
% .$$
% Calculando aquele limite, obtem-se  também que a integral diverge.)
\eqref{itIntCompar7} $\int_1^\infty\frac{x^2-1}{x^4+1}\,dx\leq
\int_1^\infty\frac{x^2}{x^4}\,dx=\int_1^\infty\frac{1}{x^2}\,dx<\infty$,
converge.
\eqref{itIntCompar8} Como $\sen x\geq -1$,
$\int_1^\infty\frac{x^2+1+\sen x}{x}\,dx\geq
\int_1^\infty\frac{x^2}{x}\,dx
=\int_1^\infty x\,dx=\infty$, diverge.
\eqref{itIntCompar9}
Como $\ln x \geq 2$ para todo $x\geq e^2$, temos que
$\int_{e^2}^\infty e^{-(\ln x)^2}\,dx\leq
\int_{e^2}^\infty e^{-2\ln x}\,dx=\int_{e^2}^\infty\frac{dx}{x^2}$, que converge.
\end{Solution}
\begin{Solution}{11.10}
 Observe que se $0\leq x<1$, então $e^{-x^2/2t}\leq 1$,
e se $x\geq 1$, então $x^2\geq x$, logo
$e^{-x^2/2t}\leq e^{-x/2t}$. Logo,
$$\int_{0}^\infty
e^{-\frac{x^2}{2t}}\,dx= \int_{0}^1
e^{-\frac{x^2}{2t}}\,dx+ \int_{1}^\infty
e^{-\frac{x^2}{2t}}\,dx \leq \int_0^1 \,dx+\int_1^\infty
e^{-x/2t}\,dx\,.$$
Como essa última integral converge (ela pode ser calculada
explicitamente), por comparação $\int_{0}^\infty
e^{-\frac{x^2}{2t}}\,dx$ converge também. Como $x\mapsto
e^{-x^2/2t}$ é par, isso implica que $f(t)$ é bem definida.
Com a mudança $y=x/\sqrt{t}$, temos
$$
 \frac{1}{\sqrt{2\pi t}}\int_{0}^\infty
e^{-\frac{x^2}{2t}}\,dx= \frac{1}{\sqrt{2\pi}}\int_{0}^\infty
e^{-\frac{y^2}{2}}\,dy\,,
$$
que não depende de $t$. Assim, $f$ é constante.
\end{Solution}
\begin{Solution}{11.11}
\eqref{itIntimpropr0} Por definição,
$\int_{0}^{1^-}\frac{dx}{\sqrt{1-x}}=\lim_{\epsilon\to
0^+}\int_0^{1-\epsilon}\frac{dx}{\sqrt{1-x}}=
\lim_{\epsilon\to 0^+}\{-2\sqrt{1-x}\}_0^{1-\epsilon}=2$. Logo, a integral
converge.
\eqref{itIntimpropr1} $\int_{0^+}^1\frac{\ln(x)}{\sqrt{x}}dx=\lim_{\epsilon\to
0^+}\int_\epsilon^1\frac{\ln(x)}{\sqrt{x}}dx$.
Integrando por partes, definindo $f'(x)\pardef \frac{1}{\sqrt{x}}$, $g(x)\pardef
\ln
(x)$, temos $f(x)=2\sqrt{x}$, $g'(x)=\frac{1}{x}$, e
\begin{align*}
\int \frac{\ln(x)}{\sqrt{x}}dx
=2\sqrt{x}\ln (x)-2\int \frac{\sqrt{x}}{x}dx
&=2\sqrt{x}\ln (x)-2\int \frac{1}{\sqrt{x}}dx\\
&=2\sqrt{x}\ln (x)-4\sqrt{x}+C\,.
\end{align*}
(Obs: pode também começar com $u=\sqrt{x}$, e acaba calculando $4\int
\ln(u)du$.)
Logo,
\begin{align*}
\int_{0^+}^1\frac{\ln(x)}{\sqrt{x}}dx&=\lim_{\epsilon\to 0^+}
\big\{
2\sqrt{x}\ln (x)-4\sqrt{x}+C
\big\}_\epsilon^1\\
&=\lim_{\epsilon\to 0^+}
-4-2\sqrt{\epsilon}\ln (\epsilon)+4\sqrt{\epsilon}=-4\,.\\
\end{align*}
Este último passo é justificado porqué $\lim_{\epsilon\to
0^+}\sqrt{\epsilon}=0$, e porqué uma simples aplicação da Regra de
Bernoulli-l'Hôpital dá $\lim_{\epsilon\to 0^+}\sqrt{\epsilon}\ln
(\epsilon)=-\lim_{y\to +\infty}\frac{\ln (y)}{\sqrt{y}}=0$.
Como o limite existe e é finito, a integral imprópria acima { converge e o
seu valor é $-4$}.

\eqref{itIntimpropr2}
Observe que a função $\frac{1}{\sqrt{e^t-1}}$ não é definida em $t=0$, logo é
necessário dividir a integral em duas integrais impróprias:
\begin{align*}
\int_{0^+}^\infty\frac{1}{\sqrt{e^t-1}}dt&=\int_{0^+}^1\frac{1}{\sqrt{e^t-1}}
dt+\int_1^\infty\frac{1}{\sqrt{e^t-1}}dt\\
&=\lim_{\epsilon\to 0^+}\int_\epsilon^1\frac{1}{\sqrt{e^t-1}}dt+\lim_{L\to
\infty}\int_1^L\frac{1}{\sqrt{e^t-1}}dt\,.
\end{align*}
Para calcular a primitiva, seja $u=\sqrt{e^t-1}$,
$du=\frac{e^t}{2\sqrt{e^t-1}}dt$, i.é. $dt=\frac{2u}{u^2+1}du$, e
\begin{align*}
\int\frac{1}{\sqrt{e^t-1}}dt=2\int\frac{du}{u^2+1} &=2\arctan (u)+C\\
&=2\arctan\sqrt{e^t-1}+C
\end{align*}
Logo,
$$\lim_{\epsilon\to
0^+}\int_\epsilon^1\frac{1}{\sqrt{e^t-1}}dt=2\lim_{\epsilon\to
0^+}\arctan\sqrt{e^t-1}\big|_\epsilon^1=2\arctan\sqrt{e-1}$$
$$\lim_{L\to \infty}\int_1^L\frac{1}{\sqrt{e^t-1}}dt=2\lim_{L\to
\infty}\arctan\sqrt{e^t-1}\big|_1^L=\pi-2\arctan\sqrt{e-1}$$
Como esses dois limites existem,
$\int_0^\infty\frac{dt}{\sqrt{e^t-1}}$ {converge, e o seu valor é
$\pi$}.
\end{Solution}
